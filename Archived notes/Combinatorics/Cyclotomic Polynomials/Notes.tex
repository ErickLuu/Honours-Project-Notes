\documentclass[]{article}
\usepackage[margin=1in]{geometry}
\usepackage{amsmath}
\usepackage{amssymb}
%opening
\title{Assignment 1}
\author{Eric Luu}

\begin{document}
Notes adapted from AMSI summer school- Dirichlet's theorem on prime progressions. 
Notes created by Tim Trudgian, Valeriia Starichkova, and Michaela Cully-Hugill.
\maketitle
\section{Cyclotomic function definition}
Let $n \geq 1$. We define the $n$-th cyclotomic polynomial as:
\begin{equation}
	\Phi_n(x) = \prod_{k \in \mathbb{Z}_n^*}\left(x - e^{\frac{2 i \pi k}{n}}\right)
\end{equation}
where $\mathbb{Z}_n^* := \left\lbrace k : 1 \leq k \leq n, \gcd(k, n) = 1 \right\rbrace$. 

We have that $z = e^{\frac{2 i \pi k}{n}}$ are the primitive $n$-th roots of unity, as we have that the order of $z$ is $n$, meaning that $z^n = 1$ and $n$ is minimal.

\subsection{Properties of cyclotomic polynomials}
The degree of $\Phi_n(x)$ is $\varphi(n)$, where $\varphi$ is the Euler totient function. 
We have that the degree of the highest term is 1 and that the constant term is also 1, as we have that when $n \geq 3$, we have that each $k \in \mathbb{Z}_n^*$ has a multiplicative inverse $k^{-1}$ distinct from $k$, as $\varphi$ has even order. Therefore, $\prod_{k \in \mathbb{Z}^*_n} - e^{\frac{2 i \pi k}{n}} = 1$. If $n = 2$, then $\Phi_2(x) = x + 1$. 
Let $P_n(x) = x^n - 1$. We can show that $P_n(x) = \prod_{1 \leq k \leq n} \left( x - e^{2 \pi i k/n} \right)$, by the algebraic closure of $\mathbb{C}$. 

In fact, we have that $P_n(x) = \prod_{d|n} \Phi_d(x)$, by using the bijection between $1/n \mathbb{Z}_n$ and $\left\lbrace 1/d \mathbb{Z}_d^* : d | n \right\rbrace$. Therefore, $n = \sum_{d | n} \varphi(d)$. 

We additionally have that $\Phi_n(1/x) = \Psi_n(x)/(x^{\varphi(n)})$.
We have that
$\Phi_n(1/x) x^{\varphi(n)} = \prod_{k \in \mathbb{Z}_n^*}x\left(1/x - e^{\frac{2 i \pi k}{n}}\right) = \prod_{k \in \mathbb{Z}_n^*} e^{\frac{-2 i \pi k}{n}} \left(e^{\frac{2 i \pi k}{n}} - x\right) = \prod_{k \in \mathbb{Z}_n^*} \left(e^{\frac{2 i \pi k}{n}} - x\right) = \Phi_n(x)$.

\section{Monic polynomials}
We can show that if $f$ is a monic polynomial with integer coefficients and $g$ is a monic polynomial with integer coefficients, then if $f = gh$, where $h$ has complex coefficients, then $h$ has integer coefficients. Thus $\Phi_n(x)$ has integer coefficients, from $P_n(x) = \Phi_n(x) \prod_{d | n, d < n} \Phi_d(x)$. 

\subsection{Factorisation}
We have that if $p$ is prime, and $d, m, d \neq m$ are positive integers where $d | m$, $p \nmid m$, then $P_d(x)$ and $\phi_m(x)$ have no common roots$\mod p$. 

Suppose not, and let $a$ be a common root mod $p$ such that $P_d(a) \equiv \phi_m(a) \mod p$.  Then we can factor out $a$ on both sides to yield $P_d(a) \equiv (x - a)f(x) \mod p$, $\phi_m(a) \equiv (x - a) g(x) \mod p$. We can also write $X^m - 1$ as $\Phi_m(x) (x^d - 1) \prod_{d' | m, d'\neq m, d' \nmid d} \Phi_{d'}(x)$, where $\prod_{d' | m, d'\neq m, d' \nmid d} \Phi_{d'}(x)$ has integer coefficients. 
However, this means that $\Phi_m(x) \equiv (x - a)^2 s(x) \mod p$. Differentiating both sides, we have that $mx^{m-1} \equiv (x-a) s_1(x) \mod p$, so $ma^{m-1} \equiv 0 \mod p$. But as $m$ is coprime to $p$, then $a^{m-1} \equiv 0 \mod p$, so $p|a^{m-1}$, implying $p|a$. But $a^d \equiv -1 \mod p$, so contradiction. 

\section{Proof of Dirichlet's theorem for $1 \mod m$}.

Assume there are finitely many primes congruent to $1 \mod m$, of the form $p_1, p_2, ..., p_N$.
We have that $\phi_m(lmp_1 ... p_N)> 1$ for a large enough integer $l$, as $\phi$ is monic and thus the largest term dominates all others in the limit, leading to infinity. Fix $l$.

Let $a = lmp_1 ... p_N$. 
Then let $p$ be a prime factor of $\Phi_m(a)$. We have that $p | a^d + k_1 a^{d-1} + ... + k_{d-1} a + 1$, so we cannot have that $p | a$, as if that was the case, then $\Phi_m(a) \equiv 1 \mod p$. Thus $p$ and $a$ share no common prime factors and $\gcd(p,a) = 1$.

As we have that $x^m - 1 = \prod_{d | m} \Psi_d(x)$, then $\Psi_m(a)$ divides $a^m - 1$. Therefore, $p | a^m - 1$. Assume that $m$ is not the order of $a \mod p$. Then we have that there exists a $d$ such that $a^d \equiv 1 \mod p$. But that implies that $d | m$, $p \nmid m$, and $p | a^d - 1$ and $p| \Psi_m(a)$, so $a$ is a common root of both. However, this is a contradiction.

Therefore, the order of $a \mod p$ is $m$. By Fermat's little theorem, we have that $a^{p-1} \equiv 1 \mod p$, so $m | p-1$. Therefore as $m < p -1$, we have that $p \equiv 1 \mod m$, and $p$ is not in $p_1, ..., p_N$. Thus shown. 
\end{document}
