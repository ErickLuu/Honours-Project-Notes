\documentclass[]{beamer}
\usepackage{amsmath}
\usepackage{amssymb}
\makeatletter
\setbeamertemplate{footline}
{%
	\leavevmode%
	\hbox{%
		\begin{beamercolorbox}[wd=1\paperwidth,ht=2.25ex,dp=1ex,center]{author in head/foot}%
		\end{beamercolorbox}%
	}%
	\vskip0pt%
}
\makeatother

%\setbeameroption{show notes on second screen}
\setbeamertemplate{page number in head/foot}{}
\setbeamertemplate{title in head/foot}{}
\setbeamertemplate{author in head/foot}{}
%opening
\title{Cyclotomic Polynomials}
\subtitle{Proving there are infinitely many primes of the form $1 \mod n$}
\author{Eric Luu}

\begin{document}
\begin{frame}
	Notes adapted from AMSI summer school- Dirichlet's theorem on prime progressions. 
	Notes created by Tim Trudgian, Valeriia Starichkova, and Michaela Cully-Hugill.
\end{frame}

\begin{frame}
	
\frametitle{Cyclotomic Polynomial}
The $n$-th cyclotomic polynomial is:
	\begin{equation}
		\Phi_n(x) = \prod_{k \in \mathbb{Z}_n^*}\left(x - e^{\frac{2 i \pi k}{n}}\right)
	\end{equation}
	where $\mathbb{Z}_n^* := \left\lbrace k : 1 \leq k \leq n, \gcd(k, n) = 1 \right\rbrace$. 
	\begin{block}{Facts about cyclotomic polynomials}
		\begin{enumerate}
			\item Monic polynomial
			\item Integer polynomial
			\item $\Phi_n(0) = 1$
			\item 		\begin{equation}
				x^n - 1 = \prod_{d|n} \Phi_d(x).
			\end{equation}
		\end{enumerate}
	\end{block}
\end{frame}

\begin{frame}
	\frametitle{Technical lemma}
	If $p$ is prime, and $d, m, d \neq m$ are positive integers where $d | m$, $p \nmid m$, then $P_d(x)$ and $\phi_m(x)$ have no common roots$\mod p$. 
\note{
Suppose not, and let $a$ be a common root mod $p$ such that $P_d(a) \equiv \phi_m(a) \mod p$.  Then we can factor out $a$ on both sides to yield $P_d(a) \equiv (x - a)f(x) \mod p$, $\phi_m(a) \equiv (x - a) g(x) \mod p$. We can also write $X^m - 1$ as $\Phi_m(x) (x^d - 1) \prod_{d' | m, d'\neq m, d' \nmid d} \Phi_{d'}(x)$, where $\prod_{d' | m, d'\neq m, d' \nmid d} \Phi_{d'}(x)$ has integer coefficients. 
However, this means that $\Phi_m(x) \equiv (x - a)^2 s(x) \mod p$. Differentiating both sides, we have that $mx^{m-1} \equiv (x-a) s_1(x) \mod p$, so $ma^{m-1} \equiv 0 \mod p$. But as $m$ is coprime to $p$, then $a^{m-1} \equiv 0 \mod p$, so $p|a^{m-1}$, implying $p|a$. But $a^d \equiv -1 \mod p$, so contradiction. 
}
\end{frame}
\begin{frame}
	\frametitle{Proof of Dirichlet's theorem $1 \mod m$}
	\begin{enumerate}
		\item Assume not. Finitely many primes, $p_1, ..., p_N$.
		\item There exists an integer $l$ such that $\Phi_m(l m p_1 ... p_N) > 1$(monic).
		\item Let $a = l m p_1 ... p_N$. 
		\item Let $p$ be a prime factor of $\Phi_m(a)$. As $\Phi_n(0) = 1$, then $p \nmid a$. Therefore, $\gcd(a, p) = 1$.
		\item From above, $p | a^m - 1$. Assume for contradiction $m$ is not the order of $a \mod p$. Then use technical lemma to find contradiction.
		\item Order of $a \mod p$ is $m$. Then $a^{p-1} \equiv 1 \mod p$. Thus $m | p-1$. But that implies $ p \equiv 1 \mod m$, $p \nmid a$.
	\end{enumerate}
\end{frame}
\end{document}
