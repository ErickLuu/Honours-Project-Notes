\documentclass[]{beamer}
\usepackage{amsmath}
\usepackage{amssymb}
\makeatletter
\setbeamertemplate{navigation symbols}{}
\setbeamertemplate{footline}
{%
	\leavevmode%
	\hbox{%
		\begin{beamercolorbox}[wd=1\paperwidth,ht=2.25ex,dp=1ex,center]{author in head/foot}%
		\end{beamercolorbox}%
	}%
	\vskip0pt%
}
\makeatother

%\setbeameroption{show notes on second screen}
\setbeamertemplate{page number in head/foot}{}
\setbeamertemplate{title in head/foot}{}
\setbeamertemplate{author in head/foot}{}
%opening
\title{Cyclotomic Polynomials}
\author{Eric Luu}
\newcommand{\ops}{\overset{\text{ops}}{\leftrightarrow}}
\begin{document}


\begin{frame}
\frametitle{Cyclotomic Polynomials}
\begin{block}{What is a cyclotomic polynomial?}
		\begin{equation}
		\Phi_n(x) = \prod_{k \in \mathbb{Z}_n^*}\left(x - e^{\frac{2 i \pi k}{n}}\right)
	\end{equation}
	where $\mathbb{Z}_n^* := \left\lbrace k : 1 \leq k \leq n, \gcd(k, n) = 1 \right\rbrace$. 
\end{block}

\end{frame}

\begin{frame}{Calculating Cyclotomic polynomials}
	\begin{align*}
		\Phi_1(x) &= x - 1\\
		\Phi_2(x) &= \frac{x^2 - 1}{x - 1} = x + 1\\
		\Phi_3(x) &= \frac{x^3 - 1}{x-1} = x^2 + x + 1\\
		\Phi_4(x) &= x^2 + 1\\
		\Phi_5(x) &= x^4 + x^3 + x^2 + x + 1\\
		\Phi_6(x) &= x^2 - x + 1
	\end{align*}
	Properties:
	\begin{enumerate}
		\item Monic polynomial
		\item Integer polynomial
		\item Constant term 1
		\item $x^n - 1 = \prod_{d|n} \Phi_d(x)$.
		\item $\Phi_p(x) = \sum_{i = 0}^{p-1} x^i$, p prime
	\end{enumerate}
\end{frame}
\begin{frame}{Calculations}
	\begin{align*}
		x^6 - 1 &= \prod_{d | 6} \Phi_d(x)\\
		&= \Phi_1(x) \Phi_2(x) \Phi_3(x) \Phi_6(x)\\
	\end{align*}
	So:
	\begin{equation}
		\frac{\Phi_6(x)}{1-x} = \frac{1 - x^6}{(1 - x^2)(1 - x^3)}
	\end{equation}
\end{frame}
\begin{frame}[allowframebreaks]{Counting}
	\begin{theorem}[Schur, 1926]
		The number of partitions of $n$ where no part appears exactly once, $p_n$, is the same as the number of partitions of $n$ into parts congruent to $0, 2, 3, 4 \mod 6$. 
	\end{theorem}
		Proof by Richard P. Stanley.
		$P(x) \ops \lbrace p_n \rbrace_{n \geq 0}$.
		Then:
		\begin{align*}
			P(x) &= \prod_{k \geq 1}(1 + x^{2k} + x^{3k} + ...)\\
			&= \prod_{k \geq 1}\left(\frac{1}{1-x^k} - x^k\right)\\
			&= \prod_{k \geq 1} \left(\frac{1 - x^k + x^{2k}}{1-x^k}\right)\\
			&= \prod_{k \geq 1} \frac{\Phi_6(x^k)}{1-x^k}\\
			&= \prod_{k \geq 1} \frac{1 - x^{6k}}{(1 - x^{2k})(1 - x^{3k})}\\
			&= \prod \frac{1}{(1 - x^{2})}\frac{1}{(1 - x^{3})}\frac{1}{(1 - x^{4})}\frac{1}{(1 - x^{6})}...
		\end{align*}
\note{We double count each $1 - x^{6k}$ for every higher $k$ so what remains are multiples of 2 and multiples of 3, counted exactly once.}
\end{frame}

\begin{frame}{Cyclotomic sets}
	\begin{definition}
		A cyclotomic set is a subset of integers such that:
		\begin{equation}
			\frac{1}{1-x} - \sum_{j \in S} x^j = \frac{N_S(x)}{1-x}
		\end{equation}
		where $N_S(x)$ is a finite product of cyclotomic polynomials.
	\end{definition}
	Classification is wide open!
\end{frame}
\end{document}
