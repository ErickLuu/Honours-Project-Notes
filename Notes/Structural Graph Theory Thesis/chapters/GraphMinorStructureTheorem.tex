\chapter{Graph Minors}\label{chap:gmst}
This chapter discusses graph minors and some parameters which bound the size of the largest clique minor in a graph. The parameters that are discussed are treewidth, pathwidth, planar graphs, and graphs on surfaces. 
The Graph Minor Structure Theorem, proven by \textcite{robertsonGraphMinorsXVI2003} is an important characterisation of $K_t$-minor-free graphs. The Graph Minor Structure Theorem was developed to solve the Graph Minor Theorem, also by \textcite{robertsonGraphMinorsXX2004}. Many problems of the form: ``Every $K_t$-minor-free graph has some bounded parameter, with the bound being $f(t)$'' uses some version of the Graph Minor Structure Theorem in its proof. 
Each component of the graph minor structure theorem is gone through in detail. This chapter then discusses the Graph Minor Theorem, which is what the Graph Minor Structure Theorem was designed to prove.


\subsection{Basic definitions}\label{sec: Basic definitions}
A graph $G$ is a pair of sets; a vertex set $V(G)$ and an edge set $E(G)$. $E(G)$ is a set that contains two-element subsets of $V(G)$. An edge $ \{v, w\}$ \textit{joins} vertices $v$ and $w$. A graph is \textit{simple} if all edges join two distinct vertices and there is at most one edge between any two vertices. In this paper, all graphs are simple unless stated. Furthermore, all graphs $G$ are finite, so $|V(G)| < \infty$. The graph with all possible edges on $n$ vertices is the \textit{complete graph} $K_n$. Graphs are defined up to isomorphism, or up to relabelling of the vertices.
Throughout this report, the set $\lbrace 1\ldots n \rbrace$ is notated as $[n]$. 
A graph \(G\) is \(k\)-colourable if there exists a function \(f: V(G) \rightarrow [k]\) such that if $f(v) = f(w)$, then $v$ and $w$ do not share an edge. The \textit{chromatic number} \(\chi(G)\) is the smallest \(k\) such that \(G\) is \(k\)-colourable. 

Let $G$ be a graph. A \textit{subgraph} $H$ in $G$ is a graph with vertex set $V(H) \subseteq V(G)$ and edge set $E(H)$ with the property that if $vw$ is an edge in $E(H)$, then $vw$ is an edge in $E(G)$.
Let $G$ be a graph and let $S$ be a non-empty subset of the vertex set of $G$. The \textit{induced subgraph} of $S$ in $G$ is the graph $G[S]$ with vertex set $S$ and edge set containing precisely all edges in $G$ incident to two vertices in $S$. Removing a set of vertices $S \subseteq V(G)$ from $G$ forms the induced subgraph $G - S := G[V(G) - S]$. 
$H$ is a \textit{spanning subgraph} of $G$ if $H$ is a subgraph of $G$ and $V(H) = V(G)$. 
The \textit{neighbourhood} of a set of vertices $A \subseteq V(G)$ are precisely all vertices that are adjacent to a vertex in $A$ and not in $A$ and is denoted as $N_G(A)$. A \textit{clique} is a subgraph isomorphic to a complete graph. 

\section{Graph minors}
A graph \(H\) is a \textit{minor} of a graph \(G\) if a graph isomorphic to \(H\) can be obtained from \(G\) by deleting vertices, deleting edges, and \textit{contracting} edges. Let $G$ be a graph and let $uv$ be an edge in $E(G)$. To \textit{contract} \(uv\), we delete both \(u\) and \(v\) and create a new vertex \(w\) with neighbourhood \(N(w) = N_G(u) \cup N_G(v)\). The graph obtained after contracting the edge \(uv\) in $G$ is written as \(G/uv\).
A description of edge contraction is in \cref{fig:edge_contraction}.  Much of structural graph theory involves graph minors in some way. Many of the theorems that we will discuss throughout this report discuss graph minors. 
\begin{figure}[h!]
	\centering
	\includesvg[pretex=\tiny, width = 0.5\textwidth]{figures/edge_contraction.svg}
	\caption[Edge contraction]{Contraction of the edge $\{u, v\}$ to the vertex $w$. Note that edges incident to common neighbours of both $u$ and $v$ becomes a single edge in the contraction. This maintains the property that the graph after edge contraction is simple.}\label{fig:edge_contraction}
\end{figure}

The statement ``\(H\) is a minor of \(G\)'' is written as \(H \leq G\). A graph \(G\) is \textit{\(H\)-minor-free} if $H$ is not a minor of $G$. A family of graphs \(\mathcal{F}\) is \textit{minor-closed} if and only if for all $G$ in \(\mathcal{F}\) and \(H \leq G\), then $H$ in \(\mathcal{F}\).

An example of a minor-closed class is the class of planar graphs.

Let $G$ and $H$ be graphs. A \textit{model} of \(H\) in \(G\) is a function $\rho$ which assigns to \(H\) vertex-disjoint connected subgraphs of \(G\), and if $uv$ is an edge in \(E(H)\), then some edge in \(G\) joins the two subgraphs \(\rho(u)\) and \(\rho(v)\). A description of a model is in \cref{fig:model_of_P5}.
\begin{figure}[h!]
	\centering
	\includesvg[width = 0.5\textwidth]{figures/model.svg}
	\caption[A model $H$ in a graph $G$.]{An illustration of a model $H$ in a graph $G$. The coloured boxes are the connected subgraphs contracted to a single vertex on the right. Note that one vertex is deleted.}\label{fig:model_of_P5}
\end{figure}

\begin{lemma}
	A graph \(H\) is a model of a graph \(G\) if and only if $H$ is a minor of $G$.
\end{lemma}

\begin{proof}
	See \textcite{norinMath599GraphMinors2017}. Suppose \(H\) is a model of \(G\). Then over all \(x\) in \(V(H)\), contract \(\rho(x)\) in \(G\) to a single vertex. This is a well-defined operation as the image $\rho(x)$ is connected and disjoint from all $\rho(y)$ where $y$ is a distinct vertex in $H$. Then delete edges to form \(H\).

	Suppose $H$ is a minor of $G$. Use induction to show that \(H\) has a model in \(G\). Suppose \(H\) is obtained from \(G\) by contraction operations only. We can assume this by taking a subgraph of \(G\) if necessary. Let \(uv\) be the first contracted edge and let \(G' := G / uv\). Let \(w\) be the vertex obtained after contracting \(uv\). Then by induction, a model \(\rho\) of \(H\) in \(G'\) exists. Then find $x \in V(H)$ such that $w \in V(\rho(x))$. If no such $x$ exists, then $\rho$ is a model of $H$ in $G$. Otherwise, delete \(w\) from \(V(\rho(x)) \) and add $u, v$ to $V(\rho(x))$, the edge $uv$, and the edges from $u$ and $v$ to the neighbours in $w$ in $\rho(x)$. Then this is a model of \(H\) in \(G\). 
\end{proof}

An important minor-closed graph family is the set of \(K_t\)-minor-free graphs for a fixed $t \geq 0$. For a graph \(G\), \(\had(G)\) is the largest \(t\) such that \(K_t\) is a minor of \(G\). The set of $K_t$-minor-free graphs is the set of graphs $G$ where $\had(G) \leq t - 1$. The function $\had(G)$ is named after Hugo Hadwiger due to his conjecture below.
\begin{conjecture}[Hadwiger's conjecture \cite{hadwigerUeberKlassifikationStreckenkomplexe1943}]\label{conj:Hadwiger's Conjecture}
	For every graph \(G\), \(\chi(G) \leq \had(G)\).
\end{conjecture}
Much work has been done on solving Hadwiger's conjecture, with a survey by \textcite{seymourHadwigersConjecture2016} on the latest progress. However, \cref{conj:Hadwiger's Conjecture} remains unsolved.

 Any result on $K_t$-minor-free graphs implies results about minor-closed families. This is due to \cref{lem:minor-closed-Kt}. 

\begin{lemma}\label{lem:minor-closed-Kt}
    Every proper minor-closed family is $K_t$-minor-free for some $t$. 
\end{lemma}
\begin{proof}
    As $\mathcal{F}$ is proper, there exists a graph $H$ which is not in $\mathcal{F}$. But this means that $H$ is not a minor of any graph in $\mathcal{F}$. Then $K_t$ is not a minor of $\mathcal{F}$, where $t = |V(H)|$. Then $\mathcal{F}$ is $K_t$-minor-free. 
\end{proof}
Note that $H$ may not be the smallest forbidden graph in $\mathcal{F}$, but the existence of such an $H$ is sufficient. 
Then \cref{conj:bded_had_pn} implies \cref{lem:Minor-Closed_Pagenumber}. From \cref{lem:minor-closed-Kt}, every proper minor-closed graph family is $K_t$-minor-free. Therefore, every graph in a proper minor-closed graph family can be embedded in a bounded number of pages, which is \cref{lem:Minor-Closed_Pagenumber}. 


\section{Treewidth}\label{chap:treewidth}
This section discusses treewidth and pathwidth.
The \textit{treewidth} of a graph \(G\) measures how similar $G$ is to a forest.
\begin{definition}[Tree-decomposition]\label{def:tree-decomposition}
	A tree-decomposition \(\tree\) of a graph \(G\) is defined as a tree \(T\) with associated \textit{bags} \(\lbrace B_x : x \in V(T) \rbrace\) such that:
	\begin{itemize}
		\item $\bigcup_{x \in V(T)} B_x = V(G)$.
		\item For all \(v \in V(G)\), the subset of vertices \(\left\lbrace x \in V(T): v \in B_x \right\rbrace\) induces a connected subtree in \(V(T)\).
		\item For all edges \(vw \in E(G)\), there exists a bag \(B_x\) such that both \(v\) and \(w\) are in \(B_x\).
	\end{itemize}
\end{definition}
We refer to the vertices of the tree \(T\) as \textit{nodes}.
The \textit{width} of the tree decomposition \(\tree\) is defined as \(\max \lbrace |B_x| - 1 : x \in V(T) \rbrace\).
The treewidth of a graph \(G\), denoted as \(\tw(G)\), is defined to be the smallest width for all tree decompositions of the graph \(G\).

\begin{lemma}\label{lem:Helly}
	Let \(T_1, \ldots, T_k\) be subtrees of a tree \(T\) such that for every pair of trees $T_i$, $T_j \in T_1, \ldots, T_k$, $V(T_i) \cap V(T_j)$ is nonempty. Then there exists a vertex which is common to all trees.
\end{lemma}
\begin{proof}
	This proof is by induction on the number of vertices of $T$. Suppose $T$ has a single vertex. Then it is obvious that the Helly property holds. By induction, suppose the Helly property holds for all trees with at most $n$ vertices. Suppose $T$ has $n + 1$ vertices and \(T_1, \ldots, T_k\) are subtrees which satisfy the property above. Let $v$ be a leaf vertex of $T$ with neighbour $w$. If one of the subtrees $T_i = \{v\}$, then by non-empty intersection, all trees contain $v$. $v$ is the common intersection. Otherwise, consider $T - v$ and the subtrees $(T_1 - v, \ldots, T_k - v)$. If $v \in T_i \cap T_j$, then as none of the subtrees is the single vertex $\{v\}$, $w \in T_i \cap T_j$. Therefore, $T_i - v \cap T_j - v$ is non-empty. By the induction hypothesis, $T - v$ has a vertex common to all $(T_1 - v, \ldots, T_k - v)$, so \(T_1, \ldots, T_k\) has a common vertex in $T$. 
\end{proof}
This property is known as the Helly property. 
The Helly property is most commonly associated with convex subsets of a Euclidean space, but has generalisations to other spaces. 

\begin{proposition}\label{lem:clique}
	Let $G$ be a graph and $(T, (B_x)_{x \in V(T)})$ be a tree-decomposition. Then for every clique \(C\) in \(G\), there exists a node \(x \in V(T)\) such that \(C \subseteq B_x\).
\end{proposition}

\begin{proof}
	Let \(\tree\) be a tree-decomposition. Every vertex \(v\) induces a connected subtree \(T_v\) in \(T\). Then for any two vertices \(x, y\) in \(C\), \(T_x\) and \(T_y\) must intersect as the edge \(xy\) is inside a bag \(B_z\) corresponding to a node \(z\). Then by the Helly property, there exists a node \(v\) such that \(C \subseteq B_v\).
\end{proof}

\begin{corollary}\label{cor:complete_tw}
	Recall that $K_n$ is the complete graph on $n$ vertices. It holds that \(\tw(K_n) = n-1\).
\end{corollary}
\begin{proof}
	By \cref{lem:clique}, $\tw(K_n)\geq n-1$. Placing all vertices of $K_n$ in a single bag is a tree-decomposition of width $n-1$. Therefore, $\tw(K_n) = n-1$. 
\end{proof}

\begin{proposition}\label{thm:tw_minor_closure}
	Let $G, H$ be graphs. If \(H\) is a minor of \(G\), then \(\tw(H) \leq \tw(G)\).
\end{proposition}
\begin{proof}
	Let \((B_x : x \in V(T))\) be a tree-decomposition of \(G\). Remove an edge $e$ from $G$. Then \((B_x : x \in V(T))\) is a tree-decomposition of $G - e$. Remove a vertex $v$ from $G$. Then \((B_x - v : x \in V(T))\) is a tree-decomposition of $G - v$. Contract an edge $vw$ in $G$ to $u$. Define a new tree-decomposition $\tree'$ by relabelling \(v\) and \(w\) in all $B_x$ to \(u\). $\tree'$ is a valid tree-decomposition of $G / uv$. The induced subtree of \(u\) is the union of the induced subtrees of \(v\) and \(w\), which is a subtree. As $v$ and $w$ share the edge $vw$, then there exists a bag $B_x$ such that $v, w \in B_x$. Every neighbour of \(v\) or \(w\) is a neighbour of \(u\). The edges in the neighbourhood do not change. Notice that the size of each bag in each operation does not increase. Therefore, if $H \leq G$ by a series of vertex deletions, edge deletions, and edge contractions, the tree-decomposition \((B_x : x \in V(T))\) of $G$ can have the algorithm applied above to build a tree-decomposition of $H$ with width at most the tree-decomposition of $G$. Then by the minimality of the treewidth, \(\tw(H) \leq \tw(G)\). 
\end{proof}

\begin{proposition}\label{lem:treewidth_forest}
	Let $G$ be a graph. \(\tw(G) = 1\) if and only if \(G\) is a forest.
\end{proposition}

\begin{proof}
	Suppose \(G\) is a tree. Root the graph \(G\) at the vertex \(r\). Then let \(T = G\) and \(B_x:= \lbrace x, p \rbrace\) where \(p\) is the parent of \(x\) and $x \neq r$. The bag \(B_r\) will just contain \(r\). Then all edges \(vw\) will be between parent \(v\) and child \(w\), so the edge $vw$ will be in bag \(B_w\). Finally, the subgraph induced by vertex \(x\) in \(T\) will be \(B_x\) and the children of \(B_x\), which is a connected subtree.
	\par
	If \(G\) is a forest, then we perform this operation on every connected component of \(G\) and connect the roots to form a new tree. Then this tree is a tree-decomposition of $G$. This forms a tree-decomposition of width at most 1. An example is in \cref{fig:tree-treedecomp}.
	\par
	If \(G\) has a cycle \(C\), then $G$ has a $K_3$-minor. By \cref{cor:complete_tw}, $\tw(K_3) = 2$. By \cref{thm:tw_minor_closure}, $2 \leq \tw(G)$. Therefore, $G$ has treewidth at least two. 
	\begin{figure}[ht]
		\centering
		\usegdlibrary {circular,trees}
\tikz \graph [simple necklace layout] {
	tree 1[draw, circle] // [tree layout] { a -> {1, 2}; }
	-> b
	-> c
	-> tree 2[draw] // [tree layout] { d -> {3, 4 -> {5, 6} } }
	-> e
	-> f
	-> tree 1;
};
		\tikz 
\graph [tree layout, nodes={draw,circle}] {
	1 -- {1 2 -- {2 3 , 2 4},1 5 -- {5 6, 5 7 -- 5 8}};
};
		\caption[Tree-Decomposition of a tree]{A tree and its tree-decomposition. Every non-root bag consists of a vertex and its parent. The root bag contains a single vertex. Every edge is contained within a single edge.}\label{fig:tree-treedecomp}
	\end{figure}
\end{proof}

\begin{proposition}\label{ex:tw_outerplanar}
	The treewidth of an outerplanar graph is at most 2.
\end{proposition}
\begin{proof}
	Let \(G\) be an outerplanar graph, and let \(G'\) be a \textit{weak triangulation} of \(G\), meaning that every face except for the outerface has three vertices. Since \(G\) is a minor of \(G'\), \(\tw(G) \leq \tw(G')\). We look at the \textit{weak dual} of \(G'\). This is a tree \(T\), where every node \(v_f\) in \(T\) corresponds to an internal face \(f\) in \(G'\). Then let \(B_{v_f}\) be the bag of the tree-decomposition, where \(B_{v_f}\) is the set of vertices on the boundary of the face \(f\). Then the tree \(T\) with bags \(B_{v_f}\) is a valid tree-decomposition of \(G'\). Every vertex is on the boundary of some internal face, so every vertex is in some bag. Every bag has at most 3 vertices. Furthermore, every edge is on the boundary of some internal face, so every edge is in some bag. Finally, let $v$ be a vertex. Then the bags that contain $v$ must be connected in $T$ as there is a sequence of internal faces which are adjacent to $v$ and are connected in $T$. Thus, \(\tw(G) \leq 2\). Refer to \cref{fig:outerplanar_treedecomp} for an example of a tree-decomposition. The green vertices and black edges are an outerplanar graph. The red vertices and blue edges are the weak dual. The magenta circles around green vertices are examples of bags in the tree-decomposition.
	\begin{figure}[h!]
		\centering
		\includesvg[width = 0.7\textwidth]{figures/outerplanar_tree_decomposition.svg}
		\caption[Tree-decomposition of outerplanar graph.]{The red vertices and blue edges are the weak dual. The magenta circles around green vertices are examples of bags in the tree-decomposition.}\label{fig:outerplanar_treedecomp}
	\end{figure}
\end{proof}

In fact, graphs of treewidth $\leq 2$ have a very simple characterisation.

\begin{proposition}\label{prop:k4-minor}
	A graph $G$ has treewidth $\leq 2$ if and only if $G$ is $K_4$-minor free. 
\end{proposition}

To prove \cref{prop:k4-minor}, we want to show the following fact:
\begin{lemma}
	Suppose a graph $G$ is $3$-connected. Then $G$ has a $K_4$ minor. 
\end{lemma}
\begin{proof}
	Suppose $G$ is $3$-connected and $G$ is $K_4$ minor free. Then let $u, v \in V(G)$ be distinct vertices. By $3$-connectedness, there are three internally disjoint paths $P, Q, R$ from $u$ to $v$. Then without loss of generality, there exists a vertex $p$ on $P -\{u, v\}$ and $q$ on $Q -\{u, v\}$ where there exists a path $S$ on $G - \{u, v\}$. Then by finding a minimal path, there is a path $S'$ internally disjoint from $P, Q, R$ which goes from a vertex in $P - \{u, v\}$ to a vertex in $Q - \{u, v\}$. Then $P \cup Q \cup R \cup S'$ is a $K_4$ minor in $G$. 
\end{proof}

This implies that every $K_4$-minor free is not $3$-connected, therefore contains a vertex of degree $\leq 2$. 

\begin{proof}
	$\Rightarrow$ Suppose $G$ contains $K_4$ as a minor. Then $\tw(G) \geq \tw(K_4) = 3$. Therefore, $G$ has treewidth $> 2$. 

	$\Leftarrow$ Suppose $G$ is $K_4$ minor free. We will prove this using induction on the number of vertices. For the base case, suppose $G$ is $K_3$. Then $G$ has a tree-decomposition with every bag containing $3$ vertices. Now suppose $|V(G)| > 3$. Then $G$ contains a vertex $v$ of degree $\leq 2$. Take $u, w$ to be the neighbours of $v$. By induction on the number of vertices, $G / \{uv\}$ is also $K_4$ minor free and has a tree-decomposition of width two. In fact, there exists a bag $B$ that contains $u$ and $w$ as $uw$ is an edge. Then add a leaf bag $B'$ to $B$ containing $u, v, w$. This is a tree-decomposition of $G$ with treewidth $\leq 2$. Thus shown. 
\end{proof}

Define a \(k\)-tree inductively. The complete graph \(K_{k+1}\) is a \(k\)-tree. If \(G\) is a \(k\)-tree, then adding any new vertex to \(G\) that is adjacent to a $k$-clique in \(G\) results in another \(k\)-tree.
A \(k\)-tree is a maximal graph with treewidth \(k\). The following is a well-known fact about $k$-trees. \todo{should there be a proof of this statement?}
\begin{proposition}
	For all graphs $G$, \(\tw(G) \leq k\) if and only if \(G\) is a subgraph of a \(k\)-tree.
\end{proposition}

$k$-trees characterise edge-maximal graphs with bounded treewidth.


\begin{proposition}\label{thm:treewidth_clique-minor-free}
	For all graphs $G$, if \(\tw(G) \leq k\), then \(G\) is \(K_{k+2}\)-minor-free.
\end{proposition}
\begin{proof}
	We shall prove the contrapositive: If \(K_t\) is a minor of \(G\), then \(\tw(G) \geq t-1\).
	If \(K_t\) is a minor of \(G\), then from \cref{thm:tw_minor_closure} that \(\tw(K_t) \leq \tw(G)\). As \(\tw(K_t) = t-1\), then \(\tw(G) \geq t - 1\).
\end{proof}

Treewidth was introduced by \textcite{berteleChapterEliminationVariables1972} with applications to dynamic programming under the name ``dimension''. Treewidth was then rediscovered by \textcite{halinSfunctionsGraphs1976}. Neither of the papers above discuss treewidth with an explicit construction.

\textcite{robertsonGraphMinorsIII1984} introduced tree-decompositions as defined in \cref{def:tree-decomposition}. This definition is concrete and could be calculated explicitly. They showed that if $\mathcal{F}$ is a graph family with bounded treewidth, then there exists a planar graph $H$ such that $H$ is a forbidden minor of $\mathcal{F}$. This was used to prove the Graph Minor Theorem. Furthermore, \textcite{robertsonQuicklyExcludingPlanar1994} refined this theorem. They showed that if a graph $G$ has large treewidth, then $G$ contains a large $n \times n$ grid as a minor. This is the Grid Minor Theorem.

\section{Path-width}\label{sec:Pathwidth}
Similar to treewidth, the pathwidth of a graph \(G\) defines how similar $G$ is from a path.

A path-decomposition of a graph \(G\) is defined as a path $P$ of some length $n$ with associated \textit{bags} \(\lbrace B_i : 1 \leq i \leq n \rbrace\) such that:
\begin{itemize}
	\item $\bigcup_{1 \leq i \leq n} B_i = V(G)$.
	\item For all \(v \in V(G)\), the subset of vertices \(\left\lbrace x \in V(P): v \in B_x \right\rbrace\) induces a connected subpath in \(V(P)\).
	\item For all edges \(vw \in E(G)\), there exists a bag \(B_x\) such that both \(v\) and \(w\) are in \(B_x\).
\end{itemize}

Typically, $P$ is suppressed.
If a graph $G$ has a path-decomposition \({(B_i)}_i\), then $G$ has a tree-decomposition \(\left(P,{(B_i)}_i\right)\). Therefore,
\begin{equation*}
	\pw(G) \geq \tw(G).
\end{equation*}
The pathwidth of \(G\) is the largest pathwidth over all connected components.

A graph \(G\) is a \textit{caterpillar} if \(G\) is a tree and $G$ has a path \(P\) where every vertex not in $P$ is adjacent to a vertex on the path \(P\). Alternatively, a tree \(G\) is a caterpillar if removing every leaf yields a path. This path is called the \textit{central path}.
\begin{proposition}
	A graph $G$ has pathwidth at most 1 if and only if every connected component of $G$ is a caterpillar.
\end{proposition}
\begin{proof}
	$\Leftarrow$ Suppose \(G\) is a caterpillar.
	Denote \(P =\left( v_1, v_2, \ldots, v_n\right)\) as the central path. The leaves of vertex \(p_i\) are denoted as \(v_{i, 1}, v_{i, 2} \dots, v_{i, k}\). Define the bags as
	\begin{equation*}
		(v_{1, 1}, v_1), (v_{1, 2}, v_1) ,\ldots ,(v_{1, j}, v_1),  (v_1, v_2), (v_{2, 1}, v_2), (v_{2,2}, v_2,),\ldots ,(v_{n-1}, v_n), (v_{n,1}, v_n), (v_{n,2}, v_n) .
	\end{equation*}
	Each leaf appears once and each vertex on the central path is on a subpath of the path. Every edge is in one bag. Therefore, the pathwidth of \(G\) is 1. If every component of $G$ is a caterpillar, then repeat for every component.

	$\Rightarrow$ Suppose \(G\) has pathwidth 1. Then $G$ is a tree, because $G$ having pathwidth 1 implies $G$ has treewidth 1. Then for each connected component of \(G\), choose a vertex \(v\) in \(B_1\) and a vertex \(w\) in \(B_n\), the final bag, and look at a path from \(v\) to \(w\). This path goes through every bag. Every vertex not on this path is adjacent to a vertex on this path. Therefore $G$ is a caterpillar. 
	
	An example of a caterpillar is in \cref{fig:caterpillar}.
\end{proof}
\begin{figure}[h!]
	\centering
	\includesvg[pretex=\small, width = 0.8\textwidth]{figures/caterpillar}
	\caption[Caterpillar graph]{A caterpillar graph with central path \((v_1, v_2, v_3, v_4, v_5, v_6)\).}\label{fig:caterpillar}
\end{figure}

\begin{example}
	Recall that $K_n$ is the complete graph on $n$ vertices. It holds that \(\pw(K_n) = \tw(K_n) = n - 1\).
\end{example}
\begin{proof}
	The pathwidth of \(K_n\) is at least the treewidth of \(K_n\). But the pathwidth is at most \(n- 1\) (where all the vertices are in the same bag), but the treewidth of \(K_n\) is \(n - 1\). Therefore, \(\pw(K_n) = n - 1\).
\end{proof}

\begin{proposition}
	The pathwidth of a tree \(T\) is \(\min_{P \subseteq T} \left\lbrace 1 + \pw(T - V(P))\right\rbrace \) where \(P\) is a path.
\end{proposition}

\begin{proof}[Proof]
	Show \(\pw(T) \leq 1 + \pw(T - V(P))\) for all $P$. If \(P\) is a path in \(T\) with vertices \(v_1, v_2, \ldots v_i\), then consider the subtrees hanging off \(v_i\) for all \(i\). \(T - V(P)\) will have a path-decomposition of width $\pw(T - P)$. Order each connected component such that they appear in the order of their parents on the paths. Then adding \(v_i\) to the bags of subtrees connected to \(v_i\), and the bag \((v_i, v_{i+1})\) between the subtrees \(v_i\) and \(v_{i + 1}\) will yield a path-decomposition of width \(1 + \pw(T - V(P))\).

	Show there exists $P$ such that \(\pw(T) \geq 1 + \pw(T - V(P))\). Proceed by construction. Let \(B_1, \ldots B_n\) be a path-decomposition of \(T\). Let \(x\) live in bag \(B_1\) and \(y\) live in bag \(B_n\), the final bag. Then let \(P\) be the unique path from \(x\) to \(y\). Then \(P\) traverses through every bag in the path-decomposition. Then \(\pw(T) \geq 1 + \pw(T - P)\) by adding every parent to the bag of each component. 
\end{proof}


\subsection{Planar graph bounds}
This subsection uses \cref{lem:planar_graphs_4_connected_cliqesums} and \cref{thm:clique_sum_pagenumber_bound} to find a book-embedding of 4-connected planar graphs. This proof is different from previous proofs as it does not require a triangulation of a planar graph. Because of this fact, this proof is used in future sections with respect to adding vortices on faces. 
Then use a theorem of Tutte to prove a fact for all $4$-connected planar graphs. 

\begin{theorem}[Tutte\cite{tutteTheoremPlanarGraphs1956}]\label{thm:4-connected_planar_ham_cycle}
	All 4-connected planar graphs are Hamiltonian.
\end{theorem}

As a corollary to \textcite{hickingbothamStackNumberCliqueSum2023}, the pagenumber of planar graphs are bounded.

\begin{corollary}\label{thm:Planar Graph Hickingbotham Bound}
	Let \(G\) be a 2-connected planar graph. Then $G$ can be embedded on $11$ pages, with book-embedding $(<, \rho)$. $<$ restricted to the outer cycle $C$ is $C$. Furthermore, for every face cycle $C$, $<_{V(C) - \{u, v, w\}} = C - \{u, v, w\}$ for some vertices $u$, $v$, $w$. 
\end{corollary}
\begin{proof}
	From \cref{thm:clique_sum_pagenumber_bound} with tree-decomposition from \cref{lem:planar_graphs_4_connected_cliqesums}, the pagenumber is at most \(2 \cdot 4 + 3 = 11\).

	Furthermore, from the construction given in \cref{lem:planar_graphs_4_connected_cliqesums}, every $4$-connected class are glued on faces. Therefore, every face only changes by $3$ vertices, from \cref{thm:clique_sum_pagenumber_bound}. Therefore removing $3$ vertices from every face preserves the cyclic ordering of every face.
\end{proof}

We will discuss the \(K_5\)-minor free case. If \(G\) is \(K_5\)-minor free, then we can use Wagner's theorem.
\begin{theorem}[Wagner's theorem\cite{wagnerUeberEigenschaftEbenen1937}]\label{thm:WagnersTheorem}
	Let \(G\) be a \(K_5\)-minor-free graph. Then \(G\) has a tree-decomposition of adhesion $\leq 3$ where every torso is either a planar graph or the Wagner graph \(V_8\).
\end{theorem}
A description of the Wagner graph is in \cref{fig:wagner}. The edges are coloured such that the internal edges are on different pages. The spine edges (the edges that are on the outerface) are the ones which can go on any page.
\begin{figure}[h!]
	\centering
	\begin{tikzpicture}[thick,scale=2, every node/.style={scale=2}]
		\tikz \graph [nodes = {draw, circle}, clockwise, empty nodes] {
	subgraph C_n [n=8, red];
	1 --[red] 5;
	2 --[blue] 6;
	3 --[green] 7;
	4 --[yellow] 8;
};

	\end{tikzpicture}
	\caption[Wagner graph]{The Wagner graph $V_8$. Notice how the clockwise circular ordering of the vertices of the Wagner graph needs 4 pages to embed the graph. }\label{fig:wagner}
\end{figure}

\begin{theorem}
	Let \(G\) be a \(K_5\)-minor free graph. Then \(G\) has pagenumber \(\leq 19\).
\end{theorem}

\begin{proof}
	Suppose \(G\) is \(K_5\)-minor free. Then by Wagner's theorem \cite{wagnerUeberEigenschaftEbenen1937}, \(G\) has a tree-decomposition of adhesion at most 3 where every torso is either a planar graph or the Wagner graph.
	Planar graphs are \(4\)-colourable and can be embedded on four pages. The Wagner graph is \(3\)-colourable and can be embedded on four pages. Therefore, if \(G\) is \(K_5\)-minor free, then \(G\) has pagenumber at most \(4 \cdot 4 + 3 = 19\) from \cref{thm:clique_sum_pagenumber_bound}.
\end{proof}

\subsection{Surfaces and graphs on surfaces}
Graphs on surfaces are a natural extension to graphs on planes. This section is an introduction to surfaces and graphs on surfaces. Readers are expected to be familiar with point-set topology. This section is based on \textcite{moharGraphsSurfaces2001}.

An \textit{$n$-manifold} $M$ is a second-countable Hausdorff space where every point in $M$ has an open neighbourhood homeomorphic to an open ball in $\mathbb{R}^n$.  
A \textit{surface} is a $2$-manifold. Surfaces are typically denoted as $\Sigma$. Examples of surfaces are the sphere $S^2$, the torus $T^2$, the real projective plane $\mathbb{R}P^2$, and the Klein bottle $K$. 

\textit{Handles} are added to a surface \(\Sigma\) by removing two disks in \(\Sigma\) and identifying the boundaries such that one goes clockwise and the other goes counter-clockwise. \textit{Crosscaps} are added to a surface $\Sigma$ by removing a disk in \(\Sigma\) and identifying opposite points on the boundary. Every surface is homeomorphic to a sphere with $m$ handles and $n$ crosscaps. The \textit{Euler genus} of a surface \(\Sigma\) with $m$ handles and $n$ crosscaps is $2m + n$. In fact, a sphere with a mix of crosscaps and handles is homeomorphic to a sphere with all crosscaps, as a sphere with a handle and crosscap is homeomorphic to three crosscaps.

An \textit{embedding} of $G$ on a surface $\Sigma$ is a drawing of $G$ on $\Sigma$ such that no two edges cross. 
A \textit{$2$-cell embedding} of a graph $G$ on a surface $\Sigma$ is an embedding of $G$ in $\Sigma$ such that $\Sigma - G$ is homeomorphic to a finite number of disks. The \textit{Euler Genus} of a \textit{graph} \(G\) is the smallest Euler genus \(g\) surface \(\Sigma\) such that \(G\) can be $2$-cell embedded on $\Sigma$.

An extension for Euler's formula is below. Suppose $G$ is $2$-cell embedded on a surface $\Sigma$ of genus $g$. Let \(|F(G)|\) be the number of faces in a graph \(G\). Then \(|V(G)| - |E(G)| + |F(G)| = 2 - g = \chi\). When $g = 0$, then $\Sigma$ is a $2$-sphere and this is the original Euler's formula. 
The value $\chi$ is known as the \textit{Euler characteristic} of a topological space, in this case a surface. The Euler characteristic is invariant under homeomorphism. Calculating the Euler characteristic of any space is done through \textit{homological algebra}, specifically by looking at the free rank of homology groups. 

Graphs that can be embedded on the plane are called \textit{planar} graphs. Graphs that can be 2-cell embedded on the torus are called \textit{toroidal} graphs, and graphs that can be 2-cell embedded on the projective plane are called \textit{projective-planar} graphs. Graphs that can be 2-cell embedded on a surface of genus $g$ are called \textit{genus $g$} graphs. Similarly to plane graphs, graph drawings on the torus are called torus graphs, and graphs drawings on the projective plane are called projective-plane graphs. 


Graphs on surfaces have been studied extensively. A famous conjecture involving graphs on surfaces is Heawood's conjecture, from \textcite{heawoodMapcolourTheorem1890}. The conjecture states that the minimum number of colours sufficient to colour all Euler genus $g$ graphs when $g \geq 0$ is
	\begin{equation*}
		\gamma(g) := \left\lfloor 
		\frac{7 + \sqrt{1 + 24g}}{2}
		\right\rfloor.
	\end{equation*}\todo{is this right?}
\textcite{ringelMapColorTheorem1974} showed that for almost every case, $\gamma(g)$ is also necessary. The case where this does not hold is the Klein bottle case. There exists a 6-colourable Klein bottle graph, but $\gamma(g) = 7$. 

\subsection{Graph Minor Structure Theorem}
\textcite{robertsonGraphMinorsXVII1999} provides a rough characterisation of all \(K_t\)-minor-free graphs. 

Every graph that is $K_t$-minor-free can be constructed from the following ingredients. This is a coarse characterisation of $K_t$-minor-free graphs, meaning that a subset, or a single one of these ingredients constitutes a $K_t$-minor-free graph. 
\begin{itemize}
	\item Graphs of bounded Euler genus.
	\item Sets of apex vertices.
	\item Graphs of bounded treewidth.
	\item Sets of vortices on graphs.
\end{itemize}
\textcite{robertsonGraphMinorsXVII1999} showed that every \(K_t\)-minor-free graph can be built up from smaller graphs with the above ingredients.

\section{Graph Minor Theorem}\label{sec:Graph Minor Theorem}
We move on to one of the most important and deepest theorems in graph theory, the Graph Minor Theorem. This was proven in a series of 23 papers by Robertson and Seymour, from 1983 to 2004. As part of the proof, the Graph Minor Structure Theorem was developed. For the case when the infinite set is the family of trees, this is Kruskal's tree theorem \cite{kruskalWellQuasiOrderingTreeTheorem1960}. 
\begin{theorem}[\textcite{robertsonGraphMinorsXX2004} Graph Minor Theorem]
	Every infinite set of graphs contains two distinct graphs \(G\) and \(H\) such that \(H\) is a minor of \(G\).
\end{theorem}
For the case when the infinite set is the family of trees, this is Kruskal's tree theorem \cite{kruskalWellQuasiOrderingTreeTheorem1960}. In fact, in the case of trees, the following holds:
\begin{theorem}[\textcite{kruskalWellQuasiOrderingTreeTheorem1960}]
	Every infinite set of trees contains two distinct trees \(G\) and \(H\) such that \(H\) is a subdivision of \(G\).
\end{theorem}
%However, this infinite family can be extremely large. Let $T_1, \ldots T_m$ be a sequence of rooted trees from the labels $\{1, 2, 3\}$ where each $T_i$ has at most $i$ vertices. By Kruskal's theorem, when $m$ becomes large enough, there exists a $i, j$ such that $1 \leq i < j$ such that $T_i$ is a label-preserving minor of $T_j$. $TREE(3)$ is the largest $m$ such that there is no label-preserving minor. 
Let $\mathcal{F}$ be a minor-closed graph family. A graph $H$ is a \textit{minimal forbidden minor} of $\mathcal{F}$ if $H \notin \mathcal{F}$ and every proper minor $H'$ of $H$ (meaning $H' \neq H$) is in $\mathcal{F}$. A graph family $\mathcal{F}$ is characterised by a set of graphs $\mathcal{H}$ if a graph $G$ is in $\mathcal{F}$ if and only if $G$ is $H$-minor-free for every $H$ in $\mathcal{H}$. 
The Graph Minor Theorem is equivalent to the statement:
\begin{theorem}
	Every minor-closed graph family $\mathcal{F}$ is characterised by a finite set of minimal forbidden minors $\mathcal{H}$. 
\end{theorem}

\begin{proof}
	Suppose every minor-closed graph family is characterised by a finite set of minimal forbidden minors, and let $\mathcal\{G\}$ is an infinite set of graphs. Let $\mathcal{F}$ be the set of graphs that exclude every graph in $\mathcal{G}$ as a minor. As $\mathcal{F}$ is minor-closed, $\mathcal{F}$ has a finite set of minimal forbidden minors, so two graphs in $\mathcal{G}$ are a minor of the other. 

	Suppose every infinite set of graphs has two graphs $G, H$ where $H \leq G$. Now let $\mathcal{F}$ be a minor-closed graph family and let $\mathcal{H}$ be the set of minimal forbidden minors. Suppose on the contrary $\mathcal{H}$ is infinite. Then there are two graphs in $\mathcal{H}$ where one is a minor of the other, meaning that $\mathcal{H}$ is not the set of minimal forbidden minors.
\end{proof}

The family of graphs that can be embedded on a torus are the toroidal graphs. 17,523 forbidden toroidal minors have been found, with a database maintained by \textcite{myrvoldLargeSetTorus2018}. \textcite{moharExcludedMinorsKlein2024} showed that there are precisely $668$ two-conected minimal forbidden minors of the torus.

A graph $G$ is \textit{linkless} if $G$ has an embedding in $\mathbb{R}^3$ such that no two cycles are linked. If no embedding of $G$ has this property, then $G$ is \textit{inherently linked}. The family of linkless graphs is minor-closed. If $G$ is linkless, then contracting any edge maintains the linkless property. \textcite{robertsonSachsLinklessEmbedding1995} proved that linkless graphs have seven minimal forbidden minors, the Petersen family. The Petersen family is generated by a series of $Y \Delta$ and $\Delta Y$-transformations on the Petersen graph. If a graph $G$ contains a vertex $v$ with neighbour set $\{x,y,z\}$, then a \textit{$Y \Delta$-transformation} deletes $v$ and adds the edges $xy,xz, yz$ to $G$. If a graph $G$ contains a triangle $xyz$, then a \textit{$\Delta Y$-transformation} deletes the edges $xy, xz, yz$ and adds a vertex $v$ with neighbours $x, y, z$. 

A graph $G$ is \textit{knotless} if $G$ can be embedded in $\mathbb{R}^3$ so that every cycle of $G$ is the unknot. Otherwise, $G$ is \textit{inherently knotted}. The family of knotless graphs is minor-closed, since contracting any edge preserves the knotless property. An example of a minimal inherently knotted graph is $K_7$. \textcite{goldbergManyManyMore2014} found that there exists at least 263 minimal minors. These include two families generated by $Y \Delta$ and $\Delta Y$-transformations on $K_7$ and $K_{3,3,1,1}$. 

%This chapter is a brief look into some of the deepest theorems in graph theory and a discussion of some of the theorems that were used to prove the Graph Minor Theorem. This chapter discusses the Graph Minor Structure Theorem, which is an important theorem that will be used throughout this thesis to prove some theorems. 


The Wagner graph $V_8$ is in \cref{fig:wagner}. In fact, the Wagner graph can be drawn with a single crossing. \textcite{robertsonExcludingGraphOne1993} showed that for every graph $H$ that can be drawn with a single crossing, every $H$-minor-free graph has a tree-decomposition of adhesion $\leq 3$ where each torso is a planar graph or a graph with treewidth $\leq \ell(H)$. Therefore, $V_8$-minor-free graphs have a tree-decomposition of adhesion $\leq 3$ where each torso is a planar graph or has bounded treewidth. 
\begin{figure}[h!]
	\centering
	\begin{tikzpicture}[thick,scale=1.5, every node/.style={scale=2}]
		\tikz \graph [nodes = {draw, circle}, clockwise, empty nodes] {
	subgraph C_n [n=8, red];
	1 --[red] 5;
	2 --[blue] 6;
	3 --[green] 7;
	4 --[yellow] 8;
};

	\end{tikzpicture}
	\caption[Wagner graph]{The Wagner graph $V_8$}\label{fig:wagner}
\end{figure}
\newpage
\section{Minor-closed graph families}\label{sec:minor_closed_families}
This is a compilation of important information on small $K_t$ minor free graphs and some important minor-closed families. \cref{tab:kt_minor_free} compiles a list of $K_t$ minor free graphs and their characterisation. There is no known good characterisation of $K_6$ minor free graphs. 

\begin{table}[h!]
    \centering
    \caption{$K_t$-minor-free characterisations}\label{tab:kt_minor_free}
    \begin{tabular*}{\textwidth}{@{}lll@{}}
        \toprule
        $K_n$ minor free graph  & Characterisation  & Reference \\
        \midrule
        1                       & Empty graph       &           \\
        2                       & Edgeless graph    &           \\
        3                       & Forests           &           \\
        4                       & Treewidth $\leq 2$&  See {\textcite{norinMath599GraphMinors2017}}         \\
        5                       & Clique-sums of planar and Wagner graphs & {\textcite{wagnerUeberEigenschaftEbenen1937}}\\
        \bottomrule
    \end{tabular*}
\end{table}

\cref{tab:minor-closed families} compiles a list of minor-closed graph families and its forbidden minors. \# MFM stands for ``Number of forbidden minors''.
This is listed in no particular order. There are infinitely many minor-closed graph families. As an example, a family $\mathcal{F}_G$ could be the family of graphs that do not contain graph $G$ as a minor. This is minor-closed, and there are infinitely many of these families.

\begin{table}[h!]
    
    \centering
    \caption{Minor-closed families}\label{tab:minor-closed families}
    \begin{tabular*}{\textwidth}{@{}llll@{}}
    \toprule
    Family name                  & \# MFM & List/Partial list of minors                      & References \\ \midrule
    Forests                      & 1                                  & $K_3$                                            &            \\
    Planar Graphs                & 2                                  & $K_5$, $K_{3,3}$                                 & \tablefootnote{\textcite{wagnerUeberEigenschaftEbenen1937}}           \\
    Toroidal Graphs              & $\geq 17523$                       & $K_8$                                            & \tablefootnote{\textcite{myrvoldLargeSetTorus2018}}           \\
    Projective-Planar Graphs     & 35                                 &                                   & \tablefootnote{\textcite{archdeaconKuratowskiTheoremProjective1980}}           \\
    Graphs embedded on a fixed surface & & &\\
    $\tw(G) \leq 2$              & 1                                  & $K_4$                                            &            \\
    $\tw(G) \leq 3$              & 4                                  & $K_5$, $K_{2,2,2}$, $P_2 \square C_5$, Wagner graph & \tablefootnote{\textcite{arnborgForbiddenMinorsCharacterization1990}}           \\
    $\tw(G) \leq 4$              & $\geq 75$                          & $K_6$                                            & \tablefootnote{\textcite{sandersLinearAlgorithmsGraphs1993}}           \\
    $\tw(G) \leq k$              &                                    &                                                  &            \\
    Linklessly Embeddable graphs & 7                                  & $K_6$, Petersen Family                           & \tablefootnote{\textcite{robertsonLinklessEmbeddingsGraphs1993}} \\
    Knotlessly Embeddable graphs & $\geq 3$                           &                                                  & \tablefootnote{\textcite{conwayKnotsLinksSpatial1983,foisyIntrinsicallyKnottedGraphs2002,foisyNewlyRecognizedIntrinsically2003}}\\
    \bottomrule
    \end{tabular*}
    
\end{table}
