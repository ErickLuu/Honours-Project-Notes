\subsection{Surfaces and graphs on surfaces}
Graphs on surfaces are a natural extension to graphs on planes.
An \textit{$n$-manifold} $M$ is a second-countable Hausdorff space where every point in $M$ has an open neighbourhood homeomorphic to an open ball in $\mathbb{R}^n$.  
A \textit{surface} is a $2$-manifold. Surfaces are typically denoted as $\Sigma$. Examples of surfaces are the sphere $S^2$, the torus $T^2$, the real projective plane $\mathbb{R}P^2$, and the Klein-bottle $K$. 

Graphs on surfaces have been studied extensively. A famous theorem involving map colourings on surfaces is Heawood's conjecture, from \textcite{heawoodMapcolourTheorem1890}. This theorem is also called the Map Colour Theorem. Piecewise linearly partition a surface $\Sigma$ into path-connected faces homeomorphic to a disk. Then a \textit{map} of $\Sigma$ is the graph obtained by placing a vertex at each face and placing an edge when two faces touch at a line. Heawood showed that the minimum number of colours sufficient to colour all Euler genus $g$ maps when $g \geq 1$ is
	\begin{equation*}
		\gamma(g) := \left\lfloor 
		\frac{7 + \sqrt{1 + 24g}}{2}
		\right\rfloor.
	\end{equation*}
When $g = 0$, this is the Four-Colour theorem, which was unproven when \textcite{ringelMapColorTheorem1974} was written.  
However, Heawood did not show that $\gamma(g)$ is necessary, which became Heawood's conjecture. 
Ringel and Young \cite{ringelMapColorTheorem1974} showed that for almost every case, $\gamma(g)$ is also necessary, and proved Heawood's conjecture. The case where this does not hold is the Klein-bottle case. Every Klein-bottle graph is $6$-colourable, but $\gamma(g) = 7$. 