
\subsection{Basic definitions}\label{sec: Basic definitions}
A graph $G$ is a pair of sets; a vertex set $V(G)$ and an edge set $E(G)$. $E(G)$ is a set that contains two-element subsets of $V(G)$. An edge $ \{v, w\}$ \textit{joins} vertices $v$ and $w$. A graph is \textit{simple} if all edges join two distinct vertices and there is at most one edge between any two vertices. In this paper, every graph is simple unless stated. Graphs which are not simple are \textit{multigraphs}. Furthermore, every graph is finite, meaning that the vertex set is finite. The graph with all possible edges on $n$ vertices is the \textit{complete graph} $K_n$. Graphs are defined up to isomorphism, or up to relabelling of the vertices.
Throughout this report, the set $\lbrace 1\ldots n \rbrace$ is notated as $[n]$. 
A graph \(G\) is \(k\)-colourable if there exists a function \(f: V(G) \rightarrow [k]\) such that if $f(v) = f(w)$, then $v$ and $w$ do not share an edge. The \textit{chromatic number} \(\chi(G)\) is the smallest \(k\) such that \(G\) is \(k\)-colourable. 

Let $G$ be a graph. A \textit{subgraph} $H$ in $G$ is a graph with vertex set $V(H) \subseteq V(G)$ and edge set $E(H) \subseteq E(G)$. The statement ``$H$ is a minor of $G$'' is notated as $H \subseteq G$.
Let $G$ be a graph and let $S$ be a non-empty subset of the vertex set of $G$. The \textit{induced subgraph} of $S$ in $G$ is the graph $G[S]$ with vertex set $S$ and edge set containing precisely all edges in $G$ incident to two vertices in $S$. Removing a set of vertices $S \subseteq V(G)$ from $G$ forms the induced subgraph $G - S := G[V(G) - S]$. 
$H$ is a \textit{spanning subgraph} of $G$ if $H$ is a subgraph of $G$ and $V(H) = V(G)$. 
The \textit{neighbourhood} of a set of vertices $A \subseteq V(G)$ is the set $N_G(A)$ with vertices $x \in V(G) - A$ where $xy$ is an edge for some $y \in A$. A \textit{clique} is a subgraph isomorphic to a complete graph. 