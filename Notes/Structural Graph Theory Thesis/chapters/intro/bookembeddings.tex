\subsection{Book-Embeddings}
A \textit{book} with \(k\) \textit{pages} consists of \(k\) half-planes glued together on a common boundary. The boundary is the \textit{spine}, and the individual half-planes are \textit{pages}. In topology, these are referred to as \textit{fans} of half-planes.\ \textcite{persingerSubsetsNbooksE31966,atneosenOnedimensionalNleavedContinua1972} described fans in the 1960s.
A \textit{book-embedding} of a graph \(G\) is an embedding of \(G\) on a book. Place the vertices of \(G\) on the spine, and place each edge on a single page such that no two edges cross.
The \textit{pagenumber} of a graph \(G\) is the minimum number of pages required to embed \(G\) on a book. This is also referred to as \textit{book-thickness}, or \textit{stack-number}. The pagenumber of a graph $G$ is $\pn(G)$.
An embedding of $K_5$ in three pages is in \cref{fig:book-embedding}.

\begin{figure}[h!]
	\centering
	\includesvg[height = 0.5\textheight]{figures/3page_K5.svg}
	\caption[Three-page book-embedding of $K_5$]{Book-embedding of $K_5$ on three pages. Image by \textcite{eppsteinBookEmbedding2014}}\label{fig:book-embedding}
\end{figure}