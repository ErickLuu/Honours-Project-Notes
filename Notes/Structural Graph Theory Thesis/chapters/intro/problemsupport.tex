%\subsection{Support for conjecture}
We have good reason to believe \cref{conj:bded_had_pn} is true. Firstly, \textcite{yannakakisEmbeddingPlanarGraphs1989} showed that every planar graph can be embedded on 4 pages. \textcite{heathPagenumberGenusGraphs1992} then showed that every graph of orientable genus $g$ can be embedded on $O(g)$ pages. Finally, \textcite{ganleyPagenumberTrees2001} showed that graphs with bounded treewidth have bounded pagenumber. \textcite{dujmovicGraphTreewidthGeometric2007} showed that the bound given by \citeauthor{ganleyPagenumberTrees2001} is tight when $\tw(G) \geq 2$.
We discuss some relevant papers that are used to prove \cref{conj:bded_had_pn}.
We aim to solve this question using the Graph Minor Structure Theorem \cite{robertsonGraphMinorsXVI2003}, which describes the structure of graphs that do not contain a \(K_t\) minor. 
We have some useful results that can be paired with the Graph Minor Structure Theorem to prove \cref{conj:bded_had_pn}.
\begin{itemize}
	\item \textcite{heathPagenumberGenusGraphs1992} showed that every graph of bounded orientable genus have bounded pagenumber.
	\item \textcite{ganleyPagenumberTrees2001} and \textcite{dujmovicGraphTreewidthGeometric2007} showed that every graph of bounded treewidth have bounded pagenumber.
	\item \textcite{hickingbothamStackNumberCliqueSum2023} showed that if a graph \(G\) has a \textit{tree-decomposition} where every \textit{torso} has bounded pagenumber, then \(G\) has bounded pagenumber.
	\item \textcite{nakamotoBookEmbeddingProjectiveplanar2015} showed that every planar-projective graph has bounded pagenumber.
\end{itemize}
These results individually show that the constituent ingredients of the Graph Minor Structure Theorem, except surfaces with non-orientable genus at least 2, have bounded pagenumber. We summarise some relevant technology that will be used to obtain some partial results for \cref{conj:bded_had_pn}. 
The biggest hurdle is showing that adding vortices on surfaces will not blow up the pagenumber. In \cref{chap:planar}, we discuss this attempt in the case that the surface is the sphere. 