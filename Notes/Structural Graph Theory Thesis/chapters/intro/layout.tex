\section{Layout of report}
The following is how the rest of the report is laid out. 
\begin{itemize}
	\item \cref{sec:background} gives an overview of some basic graph-theoretic definitions. Then there is a discussion of graphs embedded on surfaces. Then there is a section on books and book-embeddings of graphs. Finally, graph minors are discussed, as well as \cref{conj:bded_had_pn}. 
	\item \cref{chap:gmst} discusses every component of the Graph Minor Structure Theorem and how these components combine to form the Graph Minor Structure Theorem. The components that are discussed in detail are treewidth and graphs on surfaces. Finally, the Graph Minor Theorem is discussed.
	\item \cref{chap:book-embeddings} discusses book-embeddings and book-embeddings of graphs of bounded treewidth. There is also a discussion of book-embeddings of graphs with a tree-decomposition with torsos of bounded pagenumber. The paper that is discussed in this section is by \textcite{hickingbothamStackNumberCliqueSum2023} and by \textcite{ganleyPagenumberTrees2001}. There is no original research in this section but much of the technology in both papers is used. 
	\item \cref{chap:orientable} discusses graphs embedded on orientable surface and a book-embedding of graphs on orientable surfaces. \textcite{heathPagenumberGenusGraphs1992} proved that graphs embedded on orientable surfaces of some fixed genus have bounded pagenumber. This is then extended to include graphs embedded on orientable surfaces with vortices attached. 
	\item \cref{chap:nonorientable} discusses graphs embedded on nonorientable surfaces. There is a discussion of a proof by \textcite{nakamotoBookEmbeddingProjectiveplanar2015} with embedding projective planar graphs with a bounded number of pages. This result is extended to graphs embedded on projective planes with vortices attached. There is also a discussion of embedding Klein-bottle graphs on a bounded number of pages. Finally, an open problem of embedding graphs on higher nonorientable genus on a bounded number of pages is discussed. A strengthening of this open problem is also discussed, which implies \cref{conj:bded_had_pn}. Then 
\end{itemize}

Readers are expected to have at least an undergraduate understanding in graph theory and point-set topology. 
