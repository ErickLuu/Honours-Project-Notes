\section{Layout of report}
The structure of the report is as follows.
\begin{itemize}
	\item \cref{chap:gmst} discusses every component of the Graph Minor Structure Theorem and how these components combine to form the Graph Minor Structure Theorem. The components that are discussed in detail are treewidth and graphs on surfaces. Finally, the Graph Minor Theorem is discussed.
	\item \cref{chap:book-embeddings} discusses book-embeddings of graphs of bounded treewidth. There is also a discussion of book-embeddings of graphs with a tree-decomposition with torsos of bounded pagenumber. The papers discussed in this section are by \textcite{hickingbothamStackNumberCliqueSum2023} and by \textcite{ganleyPagenumberTrees2001}. 
	\item \cref{chap:orientable} discusses graphs embedded on orientable surface and a book-embedding of graphs on orientable surfaces. \textcite{heathPagenumberGenusGraphs1992} proved that graphs embedded on orientable surfaces of some fixed genus have bounded pagenumber. This is then extended to include graphs embedded on orientable surfaces with vortices attached. 
	\item \cref{chap:nonorientable} discusses graphs embedded on non-orientable surfaces. There is a discussion of a proof by \textcite{nakamotoBookEmbeddingProjectiveplanar2015} with embedding projective-planar graphs with a bounded number of pages. This result is extended to graphs embedded on projective planes with vortices attached. There is also a discussion of embedding Klein bottle graphs on a bounded number of pages. Finally, an open problem of embedding graphs on higher non-orientable genus surfaces on a bounded number of pages is discussed. A strengthening of this open problem is also discussed, which implies \cref{conj:bded_had_pn}.
	\item \cref{chap:Future Work} discusses some corollaries. We also discuss \cref{conj:nonorientable_monochromatic_paths}, which implies \cref{conj:bded_had_pn}. 
\end{itemize}

Readers are expected to have at least an undergraduate understanding in graph theory and point-set topology. 
