% !TEX root = ./thesis.tex
\section{Problem Statement}

A \textit{book-embedding} of a graph $G$ arranges the vertices of $G$ on the ``spine'' of a book and arranges the edges of $G$ on ``pages'' of a book. The \textit{pagenumber} of a graph \(G\) is the minimum number of pages necessary in a book-embedding of \(G\). The concept of the \textit{pagenumber} of a graph was introduced by \textcite{ollmannBookThicknessVarious1973} in the context of VLSI design and integrated circuitry. 
The driving question of this report is \cref{conj:bded_had_pn}. This conjecture proves that every graph in a proper minor-closed graph family can be embedded on a constant number of pages. 
\begin{conjecture}\label{conj:bded_had_pn}
	There exists a function $f$ such that for all integers $t \geq 1$, every $K_t$ minor free graph $G$ can be embedded on $f(g)$ pages.
\end{conjecture}

We begin this report by discussing some background to the topics in our literature, which includes structural graph theory and topological graph theory. We introduce some basic concepts and definitions. We discuss planar graphs, graphs on surfaces, graph minors and the Graph Minor Structure Theorem, with a discussion on some famous theorems and conjectures connected to each topic. 

Robertson and Seymour showed that graphs with no \(K_t\) minor can be built from smaller building blocks. This is a rough overview of the building blocks. Firstly, start with a graph \(G\) embedded on a genus \(g\) surface. Then add on \(p\) \textit{vortices} to \(G\), with \textit{pathwidth} at most \(k\). Then add on \(a\) \textit{apex vertices} to \(G\). Then \(G\) is \((g, p, k, a)\)-\textit{almost embeddable}. \textcite{robertsonGraphMinorsXVI2003} proved that all graphs with no \(K_t\) minor has a \textit{tree-decomposition} where every \textit{torso} is a \((g, p, k, a)\) almost-embeddable graph, with \((g, p, k, a)\) bounded by a function of \(t\). The main purpose of \cref{chap:gmst} is to explain the Graph Minor Structure Theorem in sufficient detail so that it can be applied to \cref{conj:bded_had_pn}.

In her PhD thesis, \textcite{Blankenship-PhD03} claimed to prove \cref{conj:bded_had_pn}.\footnote{
	We did not read Blankenship's thesis in the course of writing this thesis. Only at the very end did we read over her thesis to see how she handled some cases. 
	Blankenship also uses \textcite{heathPagenumberGenusGraphs1992} to do a planar-nonplanar decomposition. Apex vertices are handled the same. However, Blankenship deals with vortices differently. She uses a ``cap edges'' solution to deal with vortices to embed a graph. This is different to our approach using monochromatic paths on vortices, and using a tree-decomposition. 
	Blankenship also uses a similar theme of having some vertices being moved to the front of a book-embedding, with extra pages needed. However, her lemma was simpler than \cref{thm:clique_sum_pagenumber_bound}. Her proof relied on \textcite{heathPagenumberGenusGraphs1992} embedding graphs on surfaces of genus $g$, which is incomplete when the surface is nonorientable. 
} However, this result has not been published and has not been independently verified. Furthermore, a key proof used by Blankenship, from \textcite{heathPagenumberGenusGraphs1992}, was lacking in detail when discussing book-embeddings of nonorientable surfaces. \textcite{nakamotoBookEmbeddingProjectiveplanar2015} agrees with ths assessment of \textcite{heathPagenumberGenusGraphs1992}. Heath and Istrail's proof is discussed in more detail in \cref{chap:orientable}. \textcite{nakamotoBookEmbeddingProjectiveplanar2015} provides a proof that every projective-planar graph can be embedded on nine pages. We present this proof in \cref{chap:nonorientable}. 

This honours project has two goals. The first goal is to investigate and learn more about structural graph theory. We will discuss some important machinery in structural graph theory, the main ones being the Graph Minor Theorem and the Graph Minor Structure Theorem. To this end, we discuss book-embeddings and graphs with bounded tree-decompositions. The two papers that is discussed in a fair amount of detail are by \textcite{hickingbothamStackNumberCliqueSum2023} and by \textcite{ganleyPagenumberTrees2001}. \textcite{ganleyPagenumberTrees2001} proves that treewidth $k$ graphs can be embedded on $k+1$ pages. \textcite{hickingbothamStackNumberCliqueSum2023} proves that graphs with a tree-decomposition where every torso has pagenumber $k$ can be embedded on $2k^2 + 4k + 1$ pages. This is discussed in \cref{chap:book-embeddings}.

The second goal is to solve \cref{conj:bded_had_pn}. We prove that when a surface $\Sigma$ is orientable or the projective plane, all graphs that are almost-embeddable on a surface of genus $g$ with $p$ vortices of depth $k$ on some faces is embeddable in $f(g, p, k)$ pages. The case when $\Sigma$ is orientable is discussed in \cref{chap:orientable} and the case when $\Sigma$ is the projective plane is discussed in \cref{chap:nonorientable}. We also provide some proof directions for the case when $\Sigma$ is the Klein Bottle in \cref{chap:nonorientable}. 

We also state a conjecture that every graph embeddable on a surface of genus $g$ can be embedded on $f(g)$ pages where every face has a bounded number of monochromatic paths. If this conjecture is true, then it proves \cref{conj:bded_had_pn}. Additionally, we show that if \cref{conj:bded_had_pn} is true, then this result can be extended to all minor-closed families. This is discussed in \cref{chap:Future Work}.