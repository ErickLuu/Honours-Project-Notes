% !TEX root = ./thesis.tex
\section{Problem statement}

A \textit{book-embedding} of a graph $G$ arranges the vertices of $G$ on the ``spine'' of a book and arranges the edges of $G$ on ``pages'' of a book. The \textit{pagenumber} of a graph \(G\) is the minimum number of pages necessary in a book-embedding of \(G\).
An embedding of $K_5$ in three pages is in \cref{fig:book-embedding}.

\begin{figure}[h!]
	\centering
	\includesvg[height = 0.4\textheight]{figures/3page_K5.svg}
	\caption[Three-page book-embedding of $K_5$]{Book-embedding of $K_5$ on three pages. Image by \textcite{eppsteinBookEmbedding2014}}\label{fig:book-embedding}
\end{figure}

The concept of the \textit{pagenumber} of a graph was introduced by \textcite{ollmannBookThicknessVarious1973} in the context of VLSI design and integrated circuitry. 
The driving question of this report is \cref{conj:bded_had_pn}. A family of graphs $\mathcal{F}$ is \textit{proper} if $\mathcal{F}$ is not the set of all graphs. 
\begin{conjecture}\label{lem:Minor-Closed_Pagenumber}
	Every proper minor-closed class can be embedded on a bounded number of pages.
\end{conjecture}

This is implied by \cref{conj:bded_had_pn}. A full proof of the implication is in \cref{lem:minor-closed-Kt}.

\begin{conjecture}\label{conj:bded_had_pn}
	There exists a function $f$ where every $K_t$-minor-free graph can be embedded on $f(t)$ pages.
\end{conjecture}

The reason $K_t$-minor-free graphs are important is because of the Graph Minor Structure Theorem by Robertson and Seymour \cite{robertsonGraphMinorsXVI2003}. Robertson and Seymour showed that graphs with no \(K_t\) minor can be built from smaller building blocks. This is a rough overview of the building blocks. Firstly, start with a graph \(G\) embedded on an Euler genus \(g\) surface. Add on \(p\) \textit{vortices} to \(G\), each with \textit{pathwidth} at most \(k\). Add on \(a\) \textit{apex vertices} to \(G\). Then \(G\) is \((g, p, k, a)\)-\textit{almost embeddable}. \textcite{robertsonGraphMinorsXVI2003} proved that every graph with no \(K_t\) minor has a \textit{tree-decomposition} where every \textit{torso} is a \((g, p, k, a)\) almost-embeddable graph, with \((g, p, k, a)\) bounded by a function of \(t\). The main purpose of \cref{chap:gmst} is to explain the Graph Minor Structure Theorem in sufficient detail so that it can be applied to \cref{conj:bded_had_pn}.

In her PhD thesis, \textcite{Blankenship-PhD03} claimed to prove \cref{conj:bded_had_pn}.\footnote{
	We did not read Blankenship's thesis in the course of writing this thesis. Only at the very end did we read over her thesis to see how she handled some cases. 
	Blankenship also uses a result by \textcite{heathPagenumberGenusGraphs1992} to do a planar-nonplanar decomposition. Apex vertices are handled the same. However, Blankenship deals with vortices differently. She uses a ``cap edges'' solution to deal with vortices to embed a graph. This is different to our approach using monochromatic paths on vortices, and using a tree-decomposition. 
	Blankenship also uses a similar theme of having some vertices being moved to the front of a book-embedding, with extra pages needed. However, her lemma was simpler than \cref{thm:clique_sum_pagenumber_bound}. 
}
However, this result has not been published and has not been independently verified. Furthermore, Blankenship used a result by \textcite{heathPagenumberGenusGraphs1992}. However, was lacking in detail when discussing book-embeddings of non-orientable surfaces. \textcite{nakamotoBookEmbeddingProjectiveplanar2015} agrees with this assessment of the claimed proof by \textcite{heathPagenumberGenusGraphs1992}. Heath and Istrail's proof is discussed in more detail in \cref{chap:orientable}. \textcite{nakamotoBookEmbeddingProjectiveplanar2015} provides a proof that every projective-planar graph can be embedded on nine pages. We present this proof in \cref{chap:nonorientable}. 

This honours project has two goals. The first goal is to investigate and learn more about structural graph theory. We will discuss some important machinery in structural graph theory, the Graph Minor Theorem and the Graph Minor Structure Theorem. To this end, we discuss book-embeddings and graphs with bounded tree-decompositions. The two papers that is discussed in a fair amount of detail are by \textcite{hickingbothamStackNumberCliqueSum2023} and by \textcite{ganleyPagenumberTrees2001}. \textcite{ganleyPagenumberTrees2001} prove that treewidth $k$ graphs can be embedded on $k+1$ pages. \textcite{hickingbothamStackNumberCliqueSum2023} prove that graphs with a tree-decomposition where every torso has pagenumber $k$ can be embedded on $2k^2 + 4k + 1$ pages. This is discussed in \cref{chap:book-embeddings}.

The second goal is to attempt to solve \cref{conj:bded_had_pn}. We aimed to prove that every planar graph can be embedded on $s$ pages with some properties that allow us to embed a vortex on a face, but we were unable to prove the main theorem. We also assumed that \cref{thm:planar_graph__be_mono_paths} implies that every planar graph has a book-embedding with a bounded number of monochromatic paths, but this was not the case. We conjecture that every planar graph has a book-embedding where every face has a bounded number of monochromatic paths. The case when $\Sigma$ is orientable is discussed in \cref{chap:orientable} and the case when $\Sigma$ is the projective plane is discussed in \cref{chap:nonorientable}. We also provide some proof directions for the case when $\Sigma$ is the Klein bottle in \cref{chap:nonorientable}. 

We also state a conjecture that every graph embeddable on a surface of Euler genus $g$ can be embedded on $f(g)$ pages where every face has a bounded number of monochromatic paths. If this conjecture is true, then it would imply \cref{conj:bded_had_pn}. This is discussed in \cref{chap:Future Work}.