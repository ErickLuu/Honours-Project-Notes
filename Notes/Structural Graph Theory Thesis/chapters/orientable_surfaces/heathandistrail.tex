% !TEX root = ./thesis.tex

\section{Graphs embedded on an orientable surface of bounded genus}\label{sec:pagenumber_bounded_genus}

\textcite{yannakakisEmbeddingPlanarGraphs1989} showed that every planar graph can be embedded on four pages.

From \cref{thm:Planar Graph Hickingbotham Bound}, every planar graph can be embedded on $11$ pages. However, the proof given by Yannakakis is tight, as there exist planar graphs that need four pages, from \textcite{yannakakisPlanarGraphsThat2020,bekosFourPagesAre2020}. 

\textcite{heathPagenumberGenusGraphs1992} proved the following theorem:
\begin{theorem}[\textcite{heathPagenumberGenusGraphs1992}]\label{thm:Genus_pagenumber_bound}
	When $g \geq 1$, every orientable genus $g$ graph can be embedded in $18g+2$ pages.
\end{theorem}
Note that this bound extends the one found by \textcite{yannakakisEmbeddingPlanarGraphs1989} to graph families of bounded orientable genus.
The best known bound is \(O(\sqrt{g})\), found by \textcite{malitzGenusGraphsHave1994}. This is best possible as complete graphs $K_n$ have pagenumber $O(n)$ but have Euler genus $O(n^2)$, as shown by \textcite{ringelMapColorTheorem1974}. Therefore, if $K_n$ has Euler genus $g$ then the pagenumber of $K_n$ is $\Theta(\sqrt{g})$.

%In fact, there exists a family of graphs such that $\tilde{\gamma}(G) \in \Omega(|V(G)|)$ for every graph $G$ in the family. Consider $n$ copies of $K_6$ with a single vertex in each copy identified to a single vertex. Then each $K_6$ will occupy a single crosscap with no room for another copy of $K_6$ to fit. Therefore, this graph is of non-orientable genus $2n$. 

This section \cref{thm:Genus_pagenumber_bound}. We first decompose a graph $G$ on an orientable surface into a planar spanning subgraph and some `nonplanar' edges. This decomposition can be used to embed a graph on a book with a bounded number of pages.

Let $G$ be a connected graph embedded on a surface $\Sigma$. A subgraph $H$ of $G$ inherits an embedding on $\Sigma$ if the embedding of $H$ is a subembedding of $G$ on $\Sigma$. Heath and Istrail partition the edge-set of \(G\) to a planar spanning connected subgraph \(G_p\) embedded on $\Sigma$, with the same embedding as $G$, and set of `nonplanar' edges $E_N$. This decomposition of $G$ has the following properties:
\begin{enumerate}
	\item Every edge in \(E_N\) are between vertices on the boundary of $F_0$, the outerface of $G_p$.
	\item Adding any edge from $E_N$ to \(G_p\) breaks conditions 1.
\end{enumerate}
$G_p$ is a maximal planar subgraph with respect to the conditions above. The set $(G_p, E_N)$ is a \textit{planar-nonplanar decomposition} of $G$. 
% To talk about graphs embedded in surfaces, we assign to each face a cyclic permutation \(\sigma_v\) which represents the sequence of vertices encountered when traversing the boundary of a face in counterclockwise order.

% $\sigma_v$ enough to represent any graph in an orientable surface, but not enough for a non-orientable surface. We have to attach on an orientation to each edge, where each edge is either orientation-preserving or orientation-reversing.

% A planar-nonplanar decomposition of \(G\) is a triple \((R, G_P, E_N)\) where \(R\) is a rotation of \(G\) representing the surface embedding on the surface \(S\), \(G\) is a spanning planar graph, and \(E_N = E - E(G_P)\).
% This satisfies a list of properties:
% \begin{enumerate}
% 	\item The subrotation induces a planar embedding of \(G_p\), so we can arrange \(G\) on the surface \(S\) such that the embedding of \(G_p\) is planar.
%	\item For each \(vw \in E_N\), \(v\) and \(w\) live on the outerface \(F_0\).
%	\item \(E(G_P)\) is maximal, so we cannot add edges from \(E_N\) to \(G_P\) without breaking properties 1 and 2.
%\end{enumerate}
We build an equivalence class of nonplanar edges in a planar-nonplanar decomposition according to homotopy class. Let $G$ be a graph embedded on $\Sigma$ and let $(G_P, E_n)$ be a planar-nonplanar decomposition. Let \(C_0\) be the facial walk of \(F_0\), the outerface of $G_P$. Each vertex on the boundary of \(F_0\) appears at least once on $C_0$. If a vertex $v$ appears more than once on $C_0$, then $v$ is an \textit{articulation point}, meaning $v$ separates biconnected components. Heath and Istrail refer to a directed subpath of \(C_0\) as a \textit{trace}. Let \(T = v_1 \rightarrow v_2 \rightarrow \cdots \rightarrow v_t\) be a trace. The inverse trace is \(T^{-1} = v_t \rightarrow v_{t-1} \rightarrow \cdots \rightarrow v_1\).

Nonplanar edges need to be classified so that they can be embedded on a book. Suppose that \(u_1v_1, u_2v_2 \in E_N\) are part of the boundary of the same face \(F\) on the embedding of \(G\). Then \(u_1v_1\) and \(u_2v_2\) are \textit{homotopic} if both conditions hold.
\begin{enumerate}
	\item \(u_1v_1\) and \(u_2v_2\) are the only edges of \(E_N\) on the boundary of \(F\).
	\item There exist traces \(T_u = u_1 \rightarrow \cdots \rightarrow u_2\) and \(T_v = v_1 \rightarrow \cdots \rightarrow v_2\) such that \(T_u\) and \(T_v\) are on the boundary of \(F\).
\end{enumerate}
Then extend homotopy to be an equivalence relation, with homotopy classes being the classes of loops that are homotopic to each other. 
Think of \(G_n\) as living on a disk on \(S\) and the loops \(u_1v_1\) and \(u_2 v_2\) living on a handle or passing through a crosscap. Taking \(G_n\) to a point, \(u_1v_1\) and \(u_2v_2\) are homotopic if and only if the two loops are homotopic in a topological sense.

We now do a planar-nonplanar decomposition of graphs embedded on surfaces.
\begin{claim}\label{claim:planar_nonplanar_decomp}
	There exists a planar-nonplanar decomposition of any graph 2-cell embedded on any surface, and the trace of any homotopy class lies on the same block, except for two vertices adjacent to special endpoints $x$ and $y$. 
\end{claim}
\begin{proof}
	Suppose \(G\) is a graph 2-cell embedded on a surface \(\Sigma\). Start at a single face and define this face as $G_p$. Add edges and vertices to the planar part incrementally. At each step, set \(G_P\) to be the current planar part and \(E_N\) to be the edges that are outside the planar part. There are two types of edges in \(E_N\): edges which have both endpoints on the boundary \(V(G_P)\), so cannot become edges of \(G_P\), and edges that have either one or no endpoints in \(V(G_P)\). We maintain the property that for any \(v \in V(G_P)\), and edge \(vw \in E_n\), then \(v\) is a vertex on the boundary of \(F_0\).
	\paragraph{Adding vertices to biconnected block:}
	Suppose $F$ is the outerface of \(G_P\). Define an oriented facial walk $F$ to be a \textit{trace}. If \(v_i \rightarrow v_j \rightarrow v_k\) is a walk with no edge of \(E_N\) incident to \(v_j\), then \(v_i v_k \in E(G_T)\) is called a \textit{safe} edge. If \(v_i \rightarrow v_j\) is on the boundary of \(G_P\), and \(v_k \notin V(G_P)\), and \(v_i,v_k,v_j\) is the boundary of a face, then \(v_k\) is a \textit{safe vertex} and can be added to \(G_P\). This face is biconnected as there are no cut-vertices, as triangles are added at every step. Add as many vertices $v_k$ as possible. 

	\paragraph{Creating a new biconnected block:}
	If no \(v_k\) exists, then find a \(w'\) which is the newest vertex in \(V(G_P)\) adjacent to a vertex \(w\) not in \(V(G_P)\). Add the vertex \(w\) and the edge \(w w'\) to \(G_P\). Then add all safe edges. This is so that every edge $xy$ in \(E_p\) has $x$ and $y$ on the boundary of $F_0$. 
	
	After repeating this operation, every edge in \(E_N\) are between two vertices on the boundary of $F_0$ and adding any edge from $E_N$ to $G_p$ breaks the first condition. If an edge \(vw\) has \(v\) in $G_p$ and $w$ not in $G_p$, then at some step \(w\) is added as a safe vertex or biconnected block. If an edge \(vw\) has neither \(v\) nor \(w\) added to \(G_P\), then the algorithm has not finished yet. This algorithm always finds some vertex $v$ which is neighbouring $G_P$ and adds $v$. Since $G$ is connected, then every vertex is added to $G_P$. Thus, $G_P$ is spanning.
	
	Now every edge in \(E_N\) cannot be added to \(G_P\) without crossing over another edge, and \(G_P\) is maximal. Therefore, every edge in \(E_N\) are between two vertices on the boundary of $F_0$ and adding any edge from $E_N$ to $G_p$ breaks the first condition.

	Suppose $v_k$ was just added to $G_P$ and $v_k, z$ is a nonplanar edge where $z$ is not in the same block as $v_k$. Then no other edge incident to $v_k$ is homotopic to $v_k, z$ otherwise $v_k$ would have been a safe vertex and added to $z$. Then the homotopy class will just have $z$. 

	Suppose a new homotopy class $H$ has multiple edges added when $v_k$ is added. Then the endpoints of $H$ must be in the same block as $v_k$, as edges in a single block is added. Let $T_1 = u_1, \ldots, u_s$ and $T_2 = v_1, \ldots, v_t$ be the trace of $H$. The only vertices that are not in the same block must be the endpoints of $H$. If $w$ extends $H$, then either $u_1 w$ and $v_1 w$ are edges or $u_s w$, $v_t w$ are edges. Then $T_1$ and $T_2$ cannot be both extended as $v_t w$ or $u_s w$ is now a cut edge. If $v_t$ is $w'$ and $w$ is its neighbour, then $w$ is cut off. Then every new edge in $H$ must be adjacent to $w$. Define $x := w$. This argument can be repeated for $u_1, v_1$, and define $y := u_1$ when $u_1$ is the vertex that is cut off. 
\end{proof}


Safe vertices and unsafe vertices are illustrated in \cref{fig:safe_vertices}.

\begin{figure}[h!]
	\centering
	\includesvg[width = 0.8\textwidth]{figures/safe_vertices.svg}
	\caption[Safe and unsafe vertices]{On the left is a safe vertex $v_k$. Notice how this is not on $G_P$ but can be added without breaking the biconnected planarity condition. On the right is an unsafe vertex $w$. This cannot be added to a block in $G_P$ as there is a bridge edge. Therefore, $w$ is added as on its own separate block.}\label{fig:safe_vertices}
\end{figure}

% \subsubsection{Level sets and cycles}
% On a planar graph \(G\), we want to separate out vertices depending on how far away they are from the outerface. Fix a single outerface \(F_0\) and define the first level set \(V_0\) as the vertices adjacent to \(F_0\). Define the \(i\)-th level set, \(V_i\) inductively. Consider the induced graph on \(V(G) - \cup_{k = 0}^{i-1} V_k\). Define the vertices adjacent to \(F_0\) in this induced graph, where we expand \(F_0\) after deleting the vertices. This partitions \(V(G)\).

% We then define \(C_0\) to be the edges adjacent to \(F_0\) in this decomposition. Then we want \(C_i\) to be the edges adjacent to \(F_0\) in this decomposition. We define the chord edges \(K_i\) to be the edges between vertices in \(V_i\) that are not edges in \(C_i\). Finally, we define the edges between faces, \(E_i\) as the edges that are between vertices on level \(V_i\) and \(V_{i + 1}\).

% \begin{claim}
% 	For all faces \(F\) in \(G\), the vertices around \(F\) are either all in one \(C_i\) or they are in \(C_i\) and \(C_{i + 1}\) for some \(i\).
% \end{claim}

% \begin{proof}
% 	Let \(i\) be the smallest value such that \(v \in V_i\) is on the boundary of \(F\). Now \(G[V(G) - \cup_{j = 1}^{i} V_i]\) will also remove \(v\). However, this removes all the edges next to \(v\), therefore all vertices that are on the boundary of \(F\) will either be in \(V_i\) or \(V_{i + 1}\).
% \end{proof}
% We refer to the faces that have vertices in only \(V_i\) as chordal and the faces that are between \(V_i\) and \(V_{i + 1}\) as non-chordal.

% Recall a weak triangulation of \(G\) is a triangulation \(G'\) where all faces except for the outerface is a triangulation.
% \begin{claim}
% 	There exists a weak triangulation $G'$ of \(G\) that preserves the level sets \(V_i\) and edge sets \(E_i\), \(C_i\), \(K_i\) for all \(i\).
% \end{claim}

% \begin{proof}
% 	If \(F\) is a chordal face of \(G\), then any triangulation maintains the property. If \(F\) is non-chordal and the boundary has edges in \(V_i\) and \(V_{i + 1}\), then add edges that are only between vertices in \(V_i\) and \(V_{i + 1}\). This will suffice to build a new triangulated graph \(G'\) where all vertices and edges are in the correct place.
% \end{proof}


Now homotopy classes need to be embedded on a book, and the orientability of the surface plays an important role. Let \(u_1 v_1, \ldots, u_k v_k\) be a homotopy class of edges such that $u_1, \ldots u_k$ are in cyclic order around $C_0$. Let trace \(T_1\) be the concatenation of traces from \(u_1\) to \(u_2\), $u_2$ to $u_3$, up to $u_k$. Finally, let trace \(T_2\) be the concatenation of traces from \(v_1\) to \(v_2\), $v_2$ to $v_3$, up to $v_k$.
Two traces $(T_1, T_2)$ are \textit{orientable} if \(T_1\) and \(T_2\) go in opposite directions on $C_0$, and \textit{non-orientable} if \(T_1\) and \(T_2\) go in the same direction on $C_0$.

\begin{lemma}\label{lem:orientable_traces}
	Let $G$ be a graph embedded in an orientable surface $\Sigma$. Let $(G_P, E_N)$ be a planar-nonplanar decomposition of $G$. Then every homotopy class of $E_N$ is orientable. 
\end{lemma}
\begin{proof}
	Let $H$ be a homotopy class of edges in $E_N$. Since $H$ is a homotopy class, there exists a closed disk $D_1$ on $\Sigma$ of which every edge in $E_N$ is on. Furthermore, the maximal traces $T_1, T_2$ of $H$ are on the boundary of $D_1$. Add a loop $L$ around $G_P$ which bounds a closed disk $D_0$ containing $G_P$. $L$ intersects every edge in $E_N$ twice. Add vertices at every intersection point, and add edges between consecutive vertices on $L$. Now add an orientation to $D_0$, suppose this orientation goes anticlockwise. Then this forces $D_1$ to go anticlockwise as well, meaning that $T_1$ is going in the opposite direction to $T_2$ on the boundary of $T_1$. Then $T_1$ and $T_2$ are orientable. As this holds for every homotopy class of $H$, $T_1$ and $T_2$ are orientable. 
\end{proof}

\begin{lemma}
	Let $G$ be a graph. If \(G\) is \(2\)-cell embedded on a surface of Euler genus \(g\), then any planar-nonplanar decomposition has at most \(3g-3\) homotopy classes. 
\end{lemma}
\begin{proof}
	Decompose \(G\) to a planar-nonplanar decomposition of \(G\), \((G_P, E_N)\). Suppose \(E_N \neq \emptyset\). Identify \(G_P\) to a single point, and identify each homotopy class to a single edge. Draw a circle $S$ that bounds a disk around \(G_P\), and place new vertices where $S$ intersects an edge in $E_N$. Add edges between vertices on $S$. Delete \(G_P\), and call the new graph \(H\) embedded on $\Sigma$. Let \(n = |V(H)|\), \(m = |E(H)|\). Let \(h\) be the number of homotopy classes, and \(f\) is the number of faces on this embedding of $H$. Since $H$ is connected, we can apply Euler's formula. So, \(n - m + f = 2 - g\). Since \(H\) is cubic as every vertex has two edges on $S$ and one on the homotopy class, then \(3n = 2m\) by the handshaking lemma. Since there is only one nonplanar edge for each homotopy class, and every homotopy class is incident to two edges, \(n = 2h\). The interior face of \(H\) has \(v\) incident edges, and the remaining \(f-1\) faces have at least 3 incident edges each. If a face has four edges, then the two nonplanar edges are homotopy equivalent. Then every face has at least six edges. Every edge is incident to two faces. Place a 'dot' on the left and right side of each edge. $f-1$ faces get at least $6$ dots and $1$ face gets $n$ dots. Each edge has two dots. Therefore, \(  6(f-1) + n \leq 2m \). Use the previous facts that $n = 2h$, $3n = 2m$, and $n - m + f = 2 - g$. After an algebraic manipulation, \(h \leq 3g - 3 \). 
	\begin{align*}
		3n  & \geq 6(f - 1) + n         && \text{from } 2n \leq 3m \\
		2n  & \geq 6f - 6               \\
		4h  & \geq 6(2 - g + m - n) - 6 && \text{from } n = 2h \text{ and } f = 2 - g + m - n\\
		-2h & \geq 6 - 6g               \\
		h   & \leq 3g - 3
	\end{align*}
\end{proof}

Let \(G\) be a graph embedded on some surface $\Sigma$. Let $F$ be a face on $\Sigma$ and let $S$ be a facial walk on $F$. If a vertex $v$ appears multiple times in $S$, then $v$ is a cut vertex. A sub-ordering of $S$ is a sub-walk of $S$ where if a vertex $v$ appears multiple times in $S$, then $v$ appears exactly once restricted to where $v$ previously appeared. A vertex ordering \((<)\) \textit{preserves} \(F\) if $<$ restricted to the boundary of $F$ is equal to some sub-ordering of $S$.

\begin{lemma}\label{lem:planar_nonplanar_orientable}
	Suppose a graph \(G\) has a planar-nonplanar decomposition \((G_P, E_N)\) on an Euler genus $g \geq 1$ orientable surface \(\Sigma\). Then \(G\) can be embedded on \(18g + 2\) pages.
\end{lemma}
\begin{proof}
	Use \cref{thm:Planar Graph Hickingbotham Bound} to embed $G_P$ on four pages with the property that the outer-cycle is preserved. Let $H$ be a homotopy class in $E_P$.
	Take a block-cut tree $T$ of $G_P$. Edges in $H$ that go between the same block go on the same page. These edges do not cross because of \cref{lem:orientable_traces}. Now use the property from \cref{claim:planar_nonplanar_decomp} that every edge in $H$ that is not on the same block is adjacent to either $x$ or $y$. Colour all the edges adjacent to $x$ a new colour and all adjacent to $y$ the same colour. In total, there are three colours. 
	Therefore, \(11 + 3(6g - 3) = 18g + 2\) pages suffice if \(G\) has a planar-nonplanar decomposition.
\end{proof}