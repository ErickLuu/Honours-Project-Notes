% !TEX root = ./thesis.tex
\begin{abstract}
	This thesis discusses graph minors and graphs embedded on topological spaces. A graph $H$ is a minor of a graph $G$ if $H$ can be obtained from $G$ through vertex deletion, edge deletion and edge contraction. The class of graphs that we focus on are $K_t$-minor-free graphs. The spaces that we focus on are books, which are half-planes joined on the boundary. An embedding of a graph on a book places the vertices on the common boundary of the half-planes. Edges are placed on a single page so that no two edges cross. The smallest number of half-planes necessary to embed a graph on a book is the pagenumber. A long-standing conjecture in structural graph theory is that every proper minor-closed class can be embedded on a book with a bounded number of pages. We identify issues with a claimed proof by Heath and Istrail for book-embeddings of non-orientable surfaces.

	We use Robertson and Seymour's Graph Minor Structure Theorem to obtain results on $K_t$-minor-free graphs. This theorem states that every $K_t$-minor-free graph can be built up from four ingredients: graphs embedded on surfaces, vortices, apex sets, and clique-sums. The major contribution of this thesis is a partial result for when the Graph Minor Structure Theorem is restricted to only using graphs embedded on orientable surfaces or the projective plane. Then the graph can be embedded on a bounded number of pages.
	We prove that $K_6$-minor-free graphs (or any minor-closed family of graphs which exclude a Klein-bottle graph) have bounded pagenumber. 
	
	%However, the problem of bounding the pagenumber of graphs embedded on a non-orientable surface of higher genus is still open. It is open when the surface is the Klein bottle. We also discuss some approaches to solving this open problem. 
\end{abstract}

% Graphs and minors
% Graphs embedded on topological spaces

% Books
% Graph embedded on Books
% Pagenumber
% Conjecture on proper minor-closed classes.
% Survey of results about book-embeddings
% Robertson and Seymour Graph Minor Structure Theorem
% Ingredients of GMST
% Orientable case
% Conjecture for Klein bottle, open for Klein bottles.


%\epigraph{Man, in his quest for knowledge and progress, is determined and cannot be deterred.}{John F. Kennedy}
%\newpage

\section*{Acknowledgements}
I am grateful to David Wood for guiding and supervising me throughout this project. He has taught me everything that I know about graph theory and I would not be where I am now without his guidance. I would also like to thank fellow graph theory students Jofre Costa, Nikolai Karol, Robert Hickingbotham and Marc Distel for their support and discussions. Additionally, I would also like to thank PhD student Corbin Reid for his invaluable insight to a topological problem. I would also like to thank members of the Monash discrete maths community, including Ian Wanless, Nick Wormald, Daniel Horsley and Graham Farr for teaching me combinatorial and discrete mathematics and inspiring me to take this path. Most of all, I thank my family, friends, and Diesel for supporting me through my honours year and allowing me to do my studies. I would like to mention fellow honours student Brice Arrigo for copy editing my thesis and providing useful suggestions.

\section*{Declaration}

I declare that this document is my own work and not the work of others or generative AI. All new research is contained within \cref{chap:orientable} and \cref{chap:nonorientable}, and this is the joint work of David Wood and me. Ten pages are dedicated to pictures, bringing the main body of writing to around 40 pages. 