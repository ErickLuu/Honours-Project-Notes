\chapter{Concluding remarks}\label{chap:Future Work}

This thesis begins by discussing the Graph Minor Structure Theorem in \cref{chap:gmst} and some consequences of the Graph Minor Structure Theorem, including the Graph Minor Theorem. We discuss book-embeddings in \cref{chap:book-embeddings}, including some results by \textcite{ganleyPagenumberTrees2001} and \textcite{hickingbothamStackNumberCliqueSum2023} on book-embeddings and tree-decompositions. After, there is a discussion of some potential ways to embed a vortex on a graph on a surface. The simplest graph we embed a vortex on is a plane graph. We discuss some potential methods to embed vortices on planar graphs in \cref{chap:planar}. We then discuss orientable graphs in \cref{chap:orientable}, and present a proof by \textcite{heathPagenumberGenusGraphs1992} that every graph embedded on an orientable surface can be embedded on a bounded number of pages. Finally, we discuss non-orientable surfaces in \cref{chap:nonorientable}. We present a proof by \textcite{nakamotoBookEmbeddingProjectiveplanar2015} that projective-planar graphs are embedded on a bounded number of pages. We conjecture an extension of this theorem. This is \cref{conj:nonorientable}, which states that any graph embedded on any surface $\Sigma$ can be embedded on $f(\Sigma)$ pages. 

We then discuss a strengthening of \cref{conj:nonorientable} so that the book-embedding has special properties. The two properties that we isolate are monochromatic paths and bounded pairwise crossings. \cref{conj:nonorientable_monochromatic_paths} states that any graph embedded on any surface $\Sigma$ can be embedded on $f(\Sigma)$ pages with $g(\Sigma)$ monochromatic paths. \cref{conj:pairwise_nested} states that any graph embedded on any surface $\Sigma$ can be embedded on $f(\Sigma)$ pages with $g(\Sigma)$ pairwise crossings. 

We state a series of conjectures which, if proven, prove \cref{conj:bded_had_pn}. \cref{chap:planar} shows that either \cref{conj:nonorientable_monochromatic_paths} or \cref{conj:pairwise_nested} implies \cref{conj:bded_had_pn}.

There are clear steps to proving \cref{conj:bded_had_pn}. The first step to take is to prove \cref{conj:klein_bottle}, which states that Klein bottle graphs can be embedded on a bounded number of pages. This is the first step to proving \cref{conj:nonorientable}. Then if \cref{conj:nonorientable} holds, it will need to be strengthened to \cref{conj:pairwise_nested} or \cref{conj:nonorientable_monochromatic_paths}. 