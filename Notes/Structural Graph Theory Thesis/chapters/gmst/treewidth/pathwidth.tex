\section{Path-width}\label{sec:Pathwidth}
Similar to treewidth, the pathwidth of a graph \(G\) defines how similar $G$ is from a path. A path-decomposition is the same as a tree-decomposition restricted to a path. 

Typically, the path is suppressed in notation, so the bags are placed in order from 1 to $n$, and is denoted as $(B_i)_{i \in [n]}$, or simply $(B_i)_i$. Similarly for treewidth, the pathwidth of \(G\) is the largest pathwidth over all connected components.
If a graph $G$ has a path-decomposition \({(B_i)}_i\), then $G$ has a tree-decomposition \(\left(P,{(B_i)}_i\right)\). Therefore,
\begin{equation*}
	\pw(G) \geq \tw(G).
\end{equation*}

A graph \(G\) is a \textit{caterpillar} if \(G\) is a tree and $G$ has a path \(P\) where every vertex not in $P$ is adjacent to a vertex on the path \(P\). Alternatively, a tree \(G\) is a caterpillar if removing every leaf yields a path. This path is called the \textit{central path}.
\begin{proposition}
	A graph $G$ has pathwidth at most 1 if and only if every connected component of $G$ is a caterpillar.
\end{proposition}
\begin{proof}
	For the forwards direction, suppose \(G\) has pathwidth 1. Then $G$ is a forest, because $G$ having pathwidth 1 implies $G$ has treewidth 1. Then choose a vertex \(v\) in \(B_1\) and a vertex \(w\) in \(B_n\), the final bag, and look at a path from \(v\) to \(w\). This path goes through every bag. Every vertex not on this path is adjacent to a vertex on this path. Therefore, $G$ is a caterpillar. 
	An example of a caterpillar is in \cref{fig:caterpillar}.

	For the backwards direction, suppose \(G\) is a caterpillar.
	Denote \(P =\left( v_1, v_2, \ldots, v_n\right)\) as the central path. The leaves of vertex \(p_i\) are denoted as \(v_{i, 1}, v_{i, 2} \dots, v_{i, k}\). Define the bags as
	\begin{equation*}
		(v_{1, 1}, v_1), (v_{1, 2}, v_1) ,\ldots ,(v_{1, j}, v_1),  (v_1, v_2), (v_{2, 1}, v_2), (v_{2,2}, v_2,),\ldots ,(v_{n-1}, v_n), (v_{n,1}, v_n), (v_{n,2}, v_n) .
	\end{equation*}
	Each leaf appears once and each vertex on the central path is on a subpath of the path. Every edge is in one bag. Therefore, the pathwidth of \(G\) is 1. If every component of $G$ is a caterpillar, then repeat for every component.
\end{proof}
\begin{figure}[h!]
	\centering
	\includesvg[width = 0.8\textwidth]{figures/caterpillar}
	\caption[Caterpillar graph]{A caterpillar graph with central path \((v_1, v_2, v_3, v_4, v_5, v_6)\).}\label{fig:caterpillar}
\end{figure}

\begin{example}
	Recall that $K_n$ is the complete graph on $n$ vertices. It holds that \(\pw(K_n) = \tw(K_n) = n - 1\).
\end{example}
\begin{proof}
	The pathwidth of \(K_n\) is at least the treewidth of \(K_n\). But the pathwidth is at most \(n- 1\) (where all the vertices are in the same bag), but the treewidth of \(K_n\) is \(n - 1\). Therefore, \(\pw(K_n) = n - 1\).
\end{proof}

\begin{proposition}
	The pathwidth of a tree \(T\) equals \(\min_{P \subseteq T} \left\lbrace 1 + \pw(T - V(P))\right\rbrace \) where \(P\) is a path.
\end{proposition}

\begin{proof}[Proof]
	We prove using induction. The pathwidth of a path $P$ is $1$.

	We show \(\pw(T) \leq 1 + \pw(T - V(P))\) for all $P$. If \(P\) is a path in \(T\) with vertices \(v_1, v_2, \ldots v_i\), then consider the subtrees hanging off \(v_i\) for all \(i\). So \(T - V(P)\) will have a path-decomposition of width $\pw(T - P)$. Order each connected component such that they appear in the order of their parents on the paths. Then adding \(v_i\) to the bags of subtrees connected to \(v_i\), and the bag \((v_i, v_{i+1})\) between the subtrees \(v_i\) and \(v_{i + 1}\) will yield a path-decomposition of width \(1 + \pw(T - V(P))\) inductively.

	We show there exists $P$ such that \(\pw(T) \geq 1 + \pw(T - V(P))\). Proceed by induction. Let \(B_1, \ldots B_n\) be a path-decomposition of \(T\). Let \(x\) live in bag \(B_1\) and \(y\) live in bag \(B_n\), the final bag. Then let \(P\) be the unique path from \(x\) to \(y\). Then \(P\) traverses through every bag in the path-decomposition. Therefore, \(\pw(T) \geq 1 + \pw(T - P)\) by adding every parent to the bag of each component. 
\end{proof}

The set of graphs $\{G : \pw(G) \leq k\}$ is a minor-closed class for the same reason that the set of graphs $\{G : \tw(G) \leq k\}$ is a minor-closed class.
Ternary graphs of depth $d$ have pathwidth $d-1$. Graphs with path-width at least $w - 1$ contain every $w$-vertex forest as a minor. \textcite{seymourShorterProofPathwidth2023} provides a short proof of the above statement. 
\begin{theorem}
	$H$-minor-free graphs have bounded pathwidth if and only if $H$ is a forest. 
\end{theorem}

\begin{proof}
	Suppose $H$ is a forest. Then \textcite{seymourShorterProofPathwidth2023} proved that every graph that does not contain $H$ as a minor has path-width at most $|V(H)| - 2$. Note that the original proof was by \textcite{robertsonGraphMinorsExcluding1983}. 
	Suppose $H$ is not a forest. Then $H$ is not a minor of any tree, so $H$ is not a minor of any ternary tree. But ternary trees have unbounded pathwidth, so $H$-minor-free graphs have unbounded pathwidth.
\end{proof}
