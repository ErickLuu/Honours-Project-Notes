\section{Graph Minor Theorem}\label{sec:Graph Minor Theorem}
We move on to one of the most important and deepest theorems in graph theory, the Graph Minor Theorem. This was proven in a series of 23 papers by Robertson and Seymour, from 1983 to 2004. As part of the proof, the Graph Minor Structure Theorem was developed. For the case when the infinite set is the family of trees, this is Kruskal's tree theorem \cite{kruskalWellQuasiOrderingTreeTheorem1960}. 
\begin{theorem}[\textcite{robertsonGraphMinorsXX2004} Graph Minor Theorem]
	Every infinite set of graphs contains two distinct graphs \(G\) and \(H\) such that \(H\) is a minor of \(G\).
\end{theorem}
For the case when the infinite set is the family of trees, this is Kruskal's tree theorem \cite{kruskalWellQuasiOrderingTreeTheorem1960}. In fact, in the case of trees, the following holds:
\begin{theorem}[\textcite{kruskalWellQuasiOrderingTreeTheorem1960}]
	Every infinite set of trees contains two distinct trees \(G\) and \(H\) such that \(H\) is a subdivision of \(G\).
\end{theorem}
%However, this infinite family can be extremely large. Let $T_1, \ldots T_m$ be a sequence of rooted trees from the labels $\{1, 2, 3\}$ where each $T_i$ has at most $i$ vertices. By Kruskal's theorem, when $m$ becomes large enough, there exists a $i, j$ such that $1 \leq i < j$ such that $T_i$ is a label-preserving minor of $T_j$. $TREE(3)$ is the largest $m$ such that there is no label-preserving minor. 
Let $\mathcal{F}$ be a minor-closed graph family. A graph $H$ is a \textit{minimal forbidden minor} of $\mathcal{F}$ if $H \notin \mathcal{F}$ and every proper minor $H'$ of $H$ (meaning $H' \neq H$) is in $\mathcal{F}$. A graph family $\mathcal{F}$ is \textit{characterised} by a set of graphs $\mathcal{H}$ if a graph $G$ is in $\mathcal{F}$ if and only if $G$ is $H$-minor-free for every $H$ in $\mathcal{H}$. 
The Graph Minor Theorem is equivalent to the statement:
\begin{theorem}
	Every minor-closed graph family $\mathcal{F}$ is characterised by a finite set of minimal forbidden minors $\mathcal{H}$. 
\end{theorem}

\begin{proof}
	Suppose every minor-closed graph family is characterised by a finite set of minimal forbidden minors, and let $\mathcal\{G\}$ is an infinite set of graphs. Let $\mathcal{F}$ be the set of graphs that exclude every graph in $\mathcal{G}$ as a minor. As $\mathcal{F}$ is minor-closed, $\mathcal{F}$ has a finite set of minimal forbidden minors, so two graphs in $\mathcal{G}$ are a minor of the other. 

	Suppose every infinite set of graphs has two graphs $G, H$ where $H \leq G$. Now let $\mathcal{F}$ be a minor-closed graph family and let $\mathcal{H}$ be the set of minimal forbidden minors. Suppose on the contrary $\mathcal{H}$ is infinite. Then there are two graphs in $\mathcal{H}$ where one is a minor of the other, meaning that $\mathcal{H}$ is not the set of minimal forbidden minors.
\end{proof}

The family of graphs that can be embedded on a torus are the toroidal graphs. 17,523 forbidden toroidal minors have been found, with a database maintained by \textcite{myrvoldLargeSetTorus2018}. \textcite{moharExcludedMinorsKlein2024} showed that there are precisely $668$ two-conected minimal forbidden minors of the torus.

A graph $G$ is \textit{linkless} if $G$ has an embedding in $\mathbb{R}^3$ such that no two cycles are linked. If no embedding of $G$ has this property, then $G$ is \textit{inherently linked}. The family of linkless graphs is minor-closed. If $G$ is linkless, then contracting any edge maintains the linkless property. \textcite{robertsonSachsLinklessEmbedding1995} proved that linkless graphs have seven minimal forbidden minors, the Petersen family. The Petersen family is generated by a series of $Y \Delta$ and $\Delta Y$-transformations on the Petersen graph. If a graph $G$ contains a vertex $v$ with neighbour set $\{x,y,z\}$, then a \textit{$Y \Delta$-transformation} deletes $v$ and adds the edges $xy,xz, yz$ to $G$. If a graph $G$ contains a triangle $xyz$, then a \textit{$\Delta Y$-transformation} deletes the edges $xy, xz, yz$ and adds a vertex $v$ with neighbours $x, y, z$. 

A graph $G$ is \textit{knotless} if $G$ can be embedded in $\mathbb{R}^3$ so that every cycle of $G$ is the unknot. Otherwise, $G$ is \textit{inherently knotted}. The family of knotless graphs is minor-closed, since contracting any edge preserves the knotless property. An example of a minimal inherently knotted graph is $K_7$. \textcite{goldbergManyManyMore2014} found that there exists at least 263 minimal minors. These include two families generated by $Y \Delta$ and $\Delta Y$-transformations on $K_7$ and $K_{3,3,1,1}$. 

%This chapter is a brief look into some of the deepest theorems in graph theory and a discussion of some of the theorems that were used to prove the Graph Minor Theorem. This chapter discusses the Graph Minor Structure Theorem, which is an important theorem that will be used throughout this thesis to prove some theorems. 


The Wagner graph $V_8$ is in \cref{fig:wagner}. In fact, the Wagner graph can be drawn with a single crossing, in \cref{fig:wagner_single_crossing}. \textcite{robertsonExcludingGraphOne1993} showed that for every graph $H$ that can be drawn with a single crossing, every $H$-minor-free graph has a tree-decomposition of adhesion $\leq 3$ where each torso is a planar graph or a graph with treewidth $\leq \ell(H)$. Therefore, $V_8$-minor-free graphs have a tree-decomposition of adhesion $\leq 3$ where each torso is a planar graph or has bounded treewidth. 
\begin{figure}[h!]
	\centering
	\begin{tikzpicture}[thick,scale=1.5, every node/.style={scale=2}]
		\tikz \graph [nodes = {draw, circle}, clockwise, empty nodes] {
	subgraph C_n [n=8];
	1 --[red] 5;
	2 -- 6;
	3 -- 7;
	4 -- 8;
};

	\end{tikzpicture}
	\caption[Wagner graph]{The Wagner graph $V_8$}\label{fig:wagner}
\end{figure}