\section{Graphs on surfaces}

The terminology in this section is based on \citetitle{moharGraphsSurfaces2001} by \textcite{moharGraphsSurfaces2001}. An \textit{$n$-manifold} $M$ is a second-countable Hausdorff space where every point in $M$ has an open neighbourhood homeomorphic to an open ball in $\mathbb{R}^n$. A surface is a compact $2$-manifold. 

\textit{Handles} are added to a surface \(\Sigma\) by removing two disks in \(\Sigma\) and identifying the boundaries such that one goes clockwise, and the other goes counter-clockwise. \textit{Crosscaps} are added to a surface $\Sigma$ by removing a disk in \(\Sigma\) and identifying opposite points on the boundary. Every surface is homeomorphic to a sphere with $m$ handles and $n$ crosscaps. This is known as the classification of surfaces. The first rigorous proof of the classification of surfaces was by \textcite{brahanaSystemsCircuitsTwoDimensional1921}, but was proven with less rigour by various authors. See \cite{gallierClassificationTheoremCompact2013} by \citeauthor{gallierClassificationTheoremCompact2013} for a discussion about the history of the classification of surfaces. The \textit{Euler genus} of a surface \(\Sigma\) with $m$ handles and $n$ crosscaps is $2m + n$. 

Furthermore, a sphere with one handle and one crosscap is homeomorphic to a sphere with three crosscaps. Therefore, any sphere with a mix of handles and crosscaps is homeomorphic to one with all crosscaps. Euler genus is an invariant under homeomorphism. 

These are the Euler genus of some surfaces.
\begin{enumerate}
	\item The Euler genus of the sphere is \(0\).
	\item The Euler genus of the torus is \(2\).
	\item The Euler genus of the projective plane is \(1\). 
	\item The Euler genus of the Klein bottle is \(2\). 
\end{enumerate}

The orientability of a surface is an important tool to distinguish surfaces. A surface \(\Sigma\) is \textit{orientable} if \(\Sigma\) can be obtained from \(S^2\) by only adding handles. An example of an orientable surface is the torus. A surface \(\Sigma\) is \textit{non-orientable} if \(\Sigma\) can only be obtained from \(S^2\) by adding at least one crosscap. An example of a non-orientable surface is the projective plane or the Klein bottle. Compact orientable surfaces can be embedded in $\mathbb{R}^3$, but non-orientable surfaces cannot. The \textit{genus} of an orientable surface is the number of handles. The \textit{genus} of a non-orientable surface is the number of crosscaps, and is equal to the Euler genus. 

\subsection{Graphs on surfaces}

An \textit{embedding} of $G$ on a surface $\Sigma$ is a drawing of $G$ on $\Sigma$ such that no two edges cross. 
A \textit{$2$-cell embedding} of a graph $G$ on a surface $\Sigma$ is an embedding of $G$ in $\Sigma$ such that every connected component of $\Sigma - G$ is homeomorphic to an open disk. This is also referred to as a \textit{map}.

We now discuss some terminology of graphs on surfaces. Let $G$ be a graph and $\Sigma$ be a surface. Embedding $G$ in $\Sigma$ is referred to as the \textit{embedding} of $G$ in $\Sigma$, whereas embedding $G$ in a book is referred to as the \textit{layout} of $G$. The orientable genus of a graph \(G\), denoted \(\gamma(G)\), is the minimum genus of an orientable surface $\Sigma$ such that $G$ has an embedding on $\Sigma$. The non-orientable genus of a graph \(G\), denoted \(\tilde{\gamma}(G)\), is the minimum genus of a non-orientable surface $\Sigma$ such that $G$ has an embedding on $\Sigma$. 

The \textit{Euler genus} of a \textit{graph} \(G\), $\eg(G)$, is the smallest integer \(g\) such that \(G\) can be embedded on $\Sigma$ of Euler genus $g$. By the classification of surfaces, $\eg(G) = \min(2 \gamma(G), \tilde{\gamma}(G))$. 

\textcite{moharOrientableGenusGraphs1998} showed that \(\tilde{\gamma}(G) \leq 2 \gamma(G) + 1\) for every graph, meaning that if the orientable genus is bounded, then the non-orientable genus is bounded.\ \textcite{auslanderImbeddingGraphsManifolds1963} showed that there exists graphs which are embeddable on the projective plane that have arbitrarily large orientable genus. 

An extension for Euler's formula is below. 

\begin{theorem}
	Let $G$ be a graph $2$-cell embedded on a surface $\Sigma$ of Euler genus $g$. Let \(|F(G)|\) be the number of faces in a graph \(G\). Then \[|V(G)| - |E(G)| + |F(G)| = 2 - g.\] 
\end{theorem}

When $g = 0$, then $\Sigma$ is a $2$-sphere and this is the original Euler's formula. 
The value $\chi := 2-g$ is known as the \textit{Euler characteristic} of a topological space, in this case a surface. The Euler characteristic is invariant under homeomorphism. Calculating the Euler characteristic of any space is done through \textit{homological algebra}, specifically by looking at the free rank of homology groups. 

Graphs that can be embedded on the plane are called \textit{planar} graphs. Graphs that can be 2-cell embedded on the torus are called \textit{toroidal} graphs, and graphs that can be 2-cell embedded on the projective plane are called \textit{projective-planar} graphs. Similarly to plane graphs, \textit{torus graphs} are graph drawings on the torus, and \textit{projective-plane graphs} are graphs drawings on the projective plane. 

The family of graphs embeddable on a fixed surface $\Sigma$ is a minor-closed family. If $G$ is embedded on $\Sigma$, then $G - v$ for any vertex $v$ and $G - e$ for any edge $e$ is also embeddable on $\Sigma$. Furthermore, contracting any edge $e$ in $G$ maintains the property that no two edges cross. Edge contraction is a topological action on a graph and can be viewed as an ambient isotopy of $G$ on $\Sigma$. 
If $G$ is 2-cell embedded on a surface $\Sigma$ and every face in $G$ has three distinct vertices on its boundary, then $G$ is a \textit{triangulation} of $\Sigma$. Given graphs $G$ and $H$ with Euler genus $g_1, g_2$, a new graph with Euler genus $g_1 + g_2$ can be constructed.
\begin{theorem}[\textcite{millerAdditivityTheoremGenus1987}]\label{thm:additivity_genus}
	For all graphs $G$ and $H$, if $|V(G) \cap V(H)| \leq 1$, then $\eg(G \cup H) = \eg(G) + \eg(H)$. 
\end{theorem}

Next is an extension of \cref{thm:K5_Free_Planar} for graphs embedded on surfaces. 

\begin{theorem}\label{thm:bounded_genus_kt_free}
	If \(G\) is an Euler genus \(g\) graph, then \(G\) is \(K_t\)-minor free, where \(t = \left\lfloor \sqrt{6g} \right\rfloor + 5\). 
\end{theorem}
\begin{proof}
	This proof mimics the above proof for planarity.
	Suppose \(G\) has \(n\) vertices, \(m\) edges and Euler genus $g$. Then \(n - m + f = 2-g\), from Euler's formula on surfaces. Since three vertices bound each face and each edge touches exactly two faces, then \(f \leq 2m/3\). Therefore, \(m \leq 3(n + g - 2)\). If \(K_t\) is embeddable on an Euler genus \(g\) graph, then \(\binom{t}{2} \leq 3 (t + g - 2)\). \(t \leq \sqrt{6g} + 4\). So if $G$ has Euler genus \(g\), then $G$ is \(K_t\)-minor free, where \(t > \sqrt{6g} + 4\), so \(t = \left\lfloor \sqrt{6g} \right\rfloor + 5\). 
\end{proof}
When $\Sigma$ is a sphere, then $g = 0$. So $t = 5$. So planar graphs are $K_5$-minor free. \cref{thm:bounded_genus_kt_free} is an extension of \cref{thm:K5_Free_Planar} to higher genus surfaces. 
When $\Sigma$ is a torus, then $g = 2$. Then $t = 8$, so every toroidal graph is $K_8$-minor free. $K_7$ is a toroidal graph but $K_8$ is not. An example of an embedding of $K_7$ on a torus is in \cref{fig:k7_on_torus}. However, even though the torus and the Klein bottle have the same Euler genus, $K_7$ does not embed on the Klein bottle. 

\begin{figure}[h!]
	\centering
	\includesvg[height = 0.3\textheight]{figures/k7 on torus.svg}
	\caption[Toroidal graph]{An example of a toroidal graph $K_7$ embedded on a torus.}\label{fig:k7_on_torus}
\end{figure}

A famous theorem involving map colourings on surfaces is Heawood's conjecture, by \textcite{heawoodMapcolourTheorem1890}. This theorem is also called the Map Colour Theorem. Piecewise linearly partition a surface $\Sigma$ into path-connected faces homeomorphic to a disk. Then a \textit{map} of $\Sigma$ is the graph obtained by placing a vertex at each face and placing an edge when two faces touch at a line. Let $\chi(g)$ be the minimum number of colours sufficient to colour all Euler genus $g$ maps. Heawood showed that when $g \geq 1$, 
	\begin{equation*}
		\chi(g) \leq \left\lfloor 
		\frac{7 + \sqrt{1 + 24g}}{2}
		\right\rfloor.
	\end{equation*}
When $g = 0$, this is the Four-Colour theorem, which was unproven \citetitle{ringelMapColorTheorem1974} by \textcite{ringelMapColorTheorem1974} was written.  
Ringel and Young \cite{ringelMapColorTheorem1974} showed that for almost every case,
\begin{equation*}
	\chi(g) \geq \left\lfloor 
	\frac{7 + \sqrt{1 + 24g}}{2}
	\right\rfloor.
\end{equation*}
The case where this does not hold is the Klein-bottle case. Every Klein-bottle graph is $6$-colourable, but when $g = 2$, $\left\lfloor 
\frac{7 + \sqrt{1 + 24g}}{2}
\right\rfloor = 7$. This is related to the fact that $K_7$ does not embed on the Klein bottle. 

Let $I = [0, 1]$.
A \textit{loop} is a continuous function $\gamma : I \rightarrow X$ where $\gamma(0) = \gamma(1) = x_0$. The point $x_0$ is the \textit{base point}. A \textit{homotopy} between two loops $\alpha, \beta$ is a continuous map $h : I \times I \rightarrow (x)$ where $h(0, t) = h(1, t) = x$ for all $t$, $h(\cdot, 0) = \alpha$, $h(\cdot, 1) = \beta$. A \textit{null-homotopic} loop is a loop homotopy to the constant map at $x_0$, and a \textit{nontrivial} loop is one that is not null-homotopic. On a sphere or plane, all loops are null-homotopic. Homotopic and null-homotopic loops come up in our discussion of graphs on surfaces as they can be used to classify edges embedded on a surface when the graph is a single point $x_0$. 
