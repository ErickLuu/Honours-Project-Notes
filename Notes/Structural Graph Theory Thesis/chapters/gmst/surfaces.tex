\section{Surfaces and graphs on surfaces}

The terminology in this section is based on \textcite{moharGraphsSurfaces2001} Graphs on Surfaces. An \textit{$n$-manifold} $M$ is a second-countable Hausdorff space where every point in $M$ has an open neighbourhood homeomorphic to an open ball in $\mathbb{R}^n$. A surface is a compact $2$-manifold. 

\textit{Handles} are added to a surface \(\Sigma\) by removing two disks in \(\Sigma\) and identifying the boundaries such that one goes clockwise, and the other goes counter-clockwise. \textit{Crosscaps} are added to a surface $\Sigma$ by removing a disk in \(\Sigma\) and identifying opposite points on the boundary. Every surface is homeomorphic to a sphere with $m$ handles and $n$ crosscaps. This is known as the classification of surfaces. The \textit{Euler genus} of a surface \(\Sigma\) with $m$ handles and $n$ crosscaps is $2m + n$.

Furthermore, a sphere with one handle and one crosscap is homeomorphic to a sphere with three crosscaps. Therefore, any sphere with a mix of handles and crosscaps is homeomorphic to one with all crosscaps. Euler genus is an invariant under homeomorphism. 

These are the Euler genus of some surfaces.
\begin{enumerate}
	\item The Euler genus of the sphere is \(0\).
	\item The Euler genus of the torus is \(2\).
	\item The Euler genus of the projective plane is \(1\). 
	\item The Euler genus of the Klein bottle is \(2\). 
\end{enumerate}

Note that ``genus'' and ``Euler genus'' are two distinct concepts. In many works, ``genus'' refers to the orientable genus. 

The orientability of a surface is an important tool to distinguish surfaces. A surface \(\Sigma\) is \textit{orientable} if \(\Sigma\) can be obtained from \(S^2\) by only adding handles. An example of an orientable surface is the torus. A surface \(\Sigma\) is \textit{non-orientable} if \(\Sigma\) can only be obtained from \(S^2\) by adding at least one crosscap. An example of a non-orientable surface is the projective plane or the Klein bottle. Compact orientable surfaces can be embedded on $\mathbb{R}^3$, but non-orientable surfaces cannot.

An \textit{embedding} of $G$ on a surface $\Sigma$ is a drawing of $G$ on $\Sigma$ such that no two edges cross. 
A \textit{$2$-cell embedding} of a graph $G$ on a surface $\Sigma$ is an embedding of $G$ in $\Sigma$ such that every connected component of $\Sigma - G$ is homeomorphic to an open disk. This is also referred to as a \textit{map}.

We now discuss some terminology of graphs on surfaces. Let $G$ be a graph and $\Sigma$ be a surface. Embedding $G$ in $\Sigma$ is referred to as the \textit{embedding} of $G$ in $\Sigma$, whereas embedding $G$ in a book is referred to as the \textit{layout} of $G$. The orientable genus of a graph \(G\), denoted \(\gamma(G)\), is the minimum genus of an orientable surface $\Sigma$ such that $G$ has an embedding on $\Sigma$. The non-orientable genus of a graph \(G\), denoted \(\tilde{\gamma}(G)\), is the minimum genus of a non-orientable surface $\Sigma$ such that $G$ has an embedding on $\Sigma$. 
The \textit{Euler Genus} of a \textit{graph} \(G\) is the smallest Euler genus \(g\) surface \(\Sigma\) such that \(G\) can be $2$-cell embedded on $\Sigma$.

\textcite{moharOrientableGenusGraphs1998} showed that \(\tilde{\gamma}(G) \leq 2 \gamma(G) + 1\) for all graphs, meaning that if the orientable genus is bounded, then the non-orientable genus is bounded.\ \textcite{auslanderImbeddingGraphsManifolds1963} showed that there exists graphs which are embeddable on the projective plane that have arbitrarily large orientable genus. 

An extension for Euler's formula is below. Suppose $G$ is $2$-cell embedded on a surface $\Sigma$ of genus $g$. Let \(|F(G)|\) be the number of faces in a graph \(G\). Then \(|V(G)| - |E(G)| + |F(G)| = 2 - g = \chi\). When $g = 0$, then $\Sigma$ is a $2$-sphere and this is the original Euler's formula. 
The value $\chi$ is known as the \textit{Euler characteristic} of a topological space, in this case a surface. The Euler characteristic is invariant under homeomorphism. Calculating the Euler characteristic of any space is done through \textit{homological algebra}, specifically by looking at the free rank of homology groups. 

Graphs that can be embedded on the plane are called \textit{planar} graphs. Graphs that can be 2-cell embedded on the torus are called \textit{toroidal} graphs, and graphs that can be 2-cell embedded on the projective plane are called \textit{projective-planar} graphs. Graphs that can be 2-cell embedded on a surface of genus $g$ are called \textit{genus $g$} graphs. Similarly to plane graphs, \textit{torus graphs} are graph drawings on the torus, and \textit{projective-plane graphs} are graphs drawings on the projective plane. 

The family of graphs embeddable on a fixed surface $\Sigma$ is a minor-closed family. If $G$ is embedded on $\Sigma$, then $G - v$ for any vertex $v$ and $G - e$ for any edge $e$ is also embeddable on $\Sigma$. Furthermore, contracting any edge $e$ in $G$ maintains the property that no two edges cross. Edge contraction is a topological action on a graph and can be viewed as an ambient isotopy of $G$ on $\Sigma$. 
If $G$ is 2-cell embedded on a surface $\Sigma$ and every face in $G$ has three distinct vertices on its boundary, then $G$ is a \textit{triangulation} of $\Sigma$. Given graphs $G$ and $H$ with genus $g_1, g_2$,a new graph with genus $g_1 + g_2$ can be constructed.
\begin{theorem}[\textcite{millerAdditivityTheoremGenus1987}]\label{thm:additivity_genus}
	Let graphs $G$ and $H$ have genus $g_1$, $g_2$. Then the graph obtained from identifying a vertex in $G$ to a vertex in $H$ has genus $g_1 + g_2$. 
\end{theorem}

Next is an extension of \cref{thm:K5_Free_Planar} for graphs embedded on surfaces. 

\begin{theorem}\label{thm:bounded_genus_kt_free}
	If \(G\) is an Euler genus \(g\) graph, then \(G\) is \(K_t\)-minor free, where \(t > \sqrt{6g} + 4\). 
\end{theorem}
\begin{proof}
	This proof mimics the above proof for planarity, but on surfaces of higher genus. 
	Suppose \(G\) has \(n\) vertices and \(m\) edges and of Euler genus $g$. Then \(n - m + f = \chi = 2-g\), from Euler's theorem on surfaces. As at least three vertices bound each face and each edge touches exactly two faces, then \(f \leq 2m/3\). Therefore, \(m \leq 3(n + g - 2)\). If \(K_t\) is embeddable on a genus \(g\) graph, then \(\binom{t}{2} \leq 3 (t + g - 2)\). Thus \(t \leq \sqrt{6g} + 4\). So if $G$ has genus \(g\), then $G$ is \(K_t\)-minor free, where \(t > \sqrt{6g} + 4\). 
\end{proof}

In the case when the surface is a torus, $K_7$ is a toroidal graph but $K_8$ is not. An example of an embedding of $K_7$ on a torus is in \cref{fig:k7_on_torus}.

\begin{figure}[h!]
	\centering
	\includesvg[height = 0.3\textheight]{figures/k7 on torus.svg}
	\caption[Toroidal graph]{An example of a toroidal graph $K_7$ embedded on a torus.}\label{fig:k7_on_torus}
\end{figure}

\begin{proposition}
	$K_8$ is not embeddable on the torus.
\end{proposition}
\begin{proof}
	A torus has genus 2. By Euler's equation, if a graph $G$ is embedded on a torus, then $|V(G)| - |E(G)| + |F(G)| = 2 - 2 = 0$, where $|F(G)|$ counts the number of faces on the surface. Every face bounds at least three vertices and every edge touches two faces. Therefore, $|F(G)| \leq 2|E(G)|/3$. Suppose $K_8$ is embeddable on the torus. Then $|V(G)| = 8$ and $|E(G)| = 28$. Therefore, $|F(G)| = 20$. But $|F(G)| \leq 2 (28)/3 \leq 19$. Therefore, $K_8$ is not embeddable on the torus.
\end{proof}

A famous theorem involving map colourings on surfaces is Heawood's conjecture, from \textcite{heawoodMapcolourTheorem1890}. This theorem is also called the Map Colour Theorem. Piecewise linearly partition a surface $\Sigma$ into path-connected faces homeomorphic to a disk. Then a \textit{map} of $\Sigma$ is the graph obtained by placing a vertex at each face and placing an edge when two faces touch at a line. Heawood showed that the minimum number of colours sufficient to colour all Euler genus $g$ maps when $g \geq 1$ is
	\begin{equation*}
		\gamma(g) := \left\lfloor 
		\frac{7 + \sqrt{1 + 24g}}{2}
		\right\rfloor.
	\end{equation*}
When $g = 0$, this is the Four-Colour theorem, which was unproven when \textcite{ringelMapColorTheorem1974} was written.  
However, Heawood did not show that $\gamma(g)$ is necessary, which became Heawood's conjecture. 
Ringel and Young \cite{ringelMapColorTheorem1974} showed that for almost every case, $\gamma(g)$ is also necessary, and proved Heawood's conjecture. The case where this does not hold is the Klein-bottle case. Every Klein-bottle graph is $6$-colourable, but $\gamma(g) = 7$. 

Let $I = [0, 1]$.
A \textit{loop} is a continuous function $\gamma : I \rightarrow X$ where $\gamma(0) = \gamma(1) = x_0$. The point $x_0$ is the \textit{base point}. A \textit{homotopy} between two loops $\alpha, \beta$ is a continuous map $h : I \times I \rightarrow (x)$ where $h(0, t) = h(1, t) = x$ for all $t$, $h(\cdot, 0) = \alpha$, $h(\cdot, 1) = \beta$. A \textit{null-homotopic} loop is a loop homotopy to the constant map at $x_0$, and a \textit{nontrivial} loop is one that is not null-homotopic. On a sphere or plane, all loops are null-homotopic. Homotopic and null-homotopic loops come up in our discussion of graphs on surfaces as they can be used to classify edges embedded on a surface when the graph is a single point $x_0$. 
