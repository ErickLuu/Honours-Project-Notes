\section{Surfaces and graphs on surfaces}

This section goes into more detail about graphs on surfaces. 

\subsection{Surfaces}

The terminology in this section is based on \textcite{moharGraphsSurfaces2001} Graphs on Surfaces. Recall a surface is a compact $2$-manifold. 

\textit{Handles} are added to a surface \(\Sigma\) by removing two disks in \(\Sigma\) and identifying the boundaries such that one goes clockwise, and the other goes counter-clockwise. \textit{Crosscaps} are added to a surface $\Sigma$ by removing a disk in \(\Sigma\) and identifying opposite points on the boundary. Every surface is homeomorphic to a sphere with $m$ handles and $n$ crosscaps. This is known as the classification of surfaces. The \textit{Euler genus} of a surface \(\Sigma\) with $m$ handles and $n$ crosscaps is $2m + n$.

Furthermore, a sphere with one handle and one crosscap is homeomorphic to a sphere with three crosscaps. Therefore, any sphere with a mix of handles and crosscaps is homeomorphic to one with all crosscaps. Euler genus is an invariant under homeomorphism. 

These are the Euler genus of some surfaces.
\begin{enumerate}
	\item The Euler genus of the sphere is \(0\).
	\item The Euler genus of the torus is \(2\).
	\item The Euler genus of the projective plane is \(1\). 
	\item The Euler genus of the Klein bottles is \(2\). 
\end{enumerate}

Note that ``genus'' and ``Euler genus'' are two distinct concepts. In many works, ``genus'' refers to the orientable genus. 

The orientability of a surface is an important tool to distinguish surfaces. A surface \(\Sigma\) is \textit{orientable} if \(\Sigma\) can be obtained from \(S^2\) by only adding handles. An example of an orientable surface is the torus. A surface \(\Sigma\) is \textit{non-orientable} if \(\Sigma\) can only be obtained from \(S^2\) by adding at least one crosscap. An example of a non-orientable surface is the projective plane or the Klein bottle. Compact orientable surfaces can be embedded on $\mathbb{R}^3$, but nonorientable surfaces cannot. This implies that a sphere with $n$ handles can be viewed as a subset of $\mathbb{R}^3$. 

An \textit{embedding} of $G$ on a surface $\Sigma$ is a drawing of $G$ on $\Sigma$ such that no two edges cross. 
A \textit{$2$-cell embedding} of a graph $G$ on a surface $\Sigma$ is an embedding of $G$ in $\Sigma$ such that every connected component of $\Sigma - G$ is homeomorphic to an open disk. This is also referred to as a \textit{map}. The \textit{Euler Genus} of a \textit{graph} \(G\) is the smallest Euler genus \(g\) surface \(\Sigma\) such that \(G\) can be $2$-cell embedded on $\Sigma$.

Graphs that can be embedded on the plane are called \textit{planar} graphs. Graphs that can be 2-cell embedded on the torus are called \textit{toroidal} graphs, and graphs that can be 2-cell embedded on the projective plane are called \textit{projective-planar} graphs. Graphs that can be 2-cell embedded on a surface of genus $g$ are called \textit{genus $g$} graphs. Similarly to plane graphs, \textit{torus graphs} are graph drawings on the torus, and \textit{projective-plane graphs} are graphs drawings on the projective plane. 

The \textit{Euler Genus} of a \textit{graph} \(G\) is the smallest Euler genus \(g\) surface \(\Sigma\) such that \(G\) can be embedded on \(\Sigma\) without crossings. If $G$ is restricted to be embedded on only orientable surfaces, then in the literature this is referred to as the ``genus'' of the graph. This thesis does not use this definition of genus, and in fact the orientability of the surface plays an important role throughout this paper.

An extension for Euler's formula is below. Suppose $G$ is $2$-cell embedded on a surface $\Sigma$ of genus $g$. Let \(|F(G)|\) be the number of faces in a graph \(G\). Then \(|V(G)| - |E(G)| + |F(G)| = 2 - g = \chi\). When $g = 0$, then $\Sigma$ is a $2$-sphere and this is the original Euler's formula. 
The value $\chi$ is known as the \textit{Euler characteristic} of a topological space, in this case a surface. The Euler characteristic is invariant under homeomorphism. Calculating the Euler characteristic of any space is done through \textit{homological algebra}, specifically by looking at the free rank of homology groups. 

\subsection{Graphs on surfaces}
The family of graphs embeddable on a fixed surface $\Sigma$ is a minor-closed family. If $G$ is embedded on $\Sigma$, then $G - v$ for any vertex $v$ and $G - e$ for any edge $e$ is also embeddable on $\Sigma$. Furthermore, contracting any edge $e$ in $G$ maintains the property that no two edges cross. Edge contraction is a topological action on a graph and can be viewed as an ambient isotopy of $G$ on $\Sigma$. 
If $G$ is 2-cell embedded on a surface $\Sigma$ and every face in $G$ has three distinct vertices on its boundary, then $G$ is a \textit{triangulation} of $\Sigma$. Given graphs $G$ and $H$ with genus $g_1, g_2$,a new graph with genus $g_1 + g_2$ can be constructed.
\begin{theorem}[\textcite{millerAdditivityTheoremGenus1987}]\label{thm:additivity_genus}
	Let graphs $G$ and $H$ have genus $g_1$, $g_2$. Then the graph obtained from identifying a vertex in $G$ to a vertex in $H$ has genus $g_1 + g_2$. 
\end{theorem}

Next is an extension of \cref{thm:K5_Free_Planar} for graphs embedded on surfaces. 

\begin{theorem}\label{thm:bounded_genus_kt_free}
	If \(G\) is an Euler genus \(g\) graph, then \(G\) is \(K_t\)-minor free, where \(t > \sqrt{6g} + 4\). 
\end{theorem}
\begin{proof}
	This proof mimics the above proof for planarity, but on surfaces of higher genus. 
	Suppose \(G\) has \(n\) vertices and \(m\) edges and of Euler genus $g$. Then \(n - m + f = \chi = 2-g\), from Euler's theorem on surfaces. As at least three vertices bound each face and each edge touches exactly two faces, then \(f \leq 2m/3\). Therefore, \(m \leq 3(n + g - 2)\). If \(K_t\) is embeddable on a genus \(g\) graph, then \(\binom{t}{2} \leq 3 (t + g - 2)\). Thus \(t \leq \sqrt{6g} + 4\). So if $G$ has genus \(g\), then $G$ is \(K_t\)-minor-free, where \(t > \sqrt{6g} + 4\). 
\end{proof}