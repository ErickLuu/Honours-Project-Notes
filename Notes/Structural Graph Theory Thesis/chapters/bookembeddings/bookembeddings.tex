\section{Book-Embeddings and Pagenumber}\label{sec:Book Embedding}
A \textit{book} with \(k\) \textit{pages} consists of \(k\) half-planes glued together on a common boundary. The boundary is the \textit{spine}, and the individual half-planes are \textit{pages}. In topology, these are referred to as \textit{fans} of half-planes.\ \textcite{persingerSubsets$n$books$E^3$1966, atneosenOnedimensional$n$leavedContinua1972} described fans in the 1960s.
A \textit{book-embedding} of a graph \(G\) is an embedding of \(G\) on a book. Place the vertices of \(G\) on the spine, and place each edge on a single page such that no two edges cross.
The \textit{pagenumber} of a graph \(G\) is the minimum number of pages required to embed \(G\) on a book. This is also referred to as \textit{book-thickness}, or \textit{stack-number}. The pagenumber of a graph $G$ is $\pn(G)$.

Book-embeddings have an equivalent combinatorial definition. A \textit{book embedding} of a graph \(G\) is an arrangement of the vertices of \(G\) in a total ordering \(v_1 < v_2 < \cdots < v_n\). The edges in \(E(G)\) are coloured so that if there are vertices with ordering \(v_a < v_b < v_c < v_d\) and edges \(v_a v_c\) and \(v_b v_d\) in $E(G)$, then $v_a v_c$ and $v_b v_d$ are assigned different colours.
The ordering of \(V(G)\) in the book embedding is denoted as \((<)\) or as \((\leq)\). For a book-embedding \((v_1, v_2, \ldots, v_{|G|})\), the edges \( \left\{ v_1 v_2, v_2 v_3, \ldots, v_{|G| - 1}v_{|G|}, v_{|G|}v_{1} \right\} \) are called \textit{spine edges}.

A circular ordering of a set $S$ is an ordering where the elements of $S$ are in a circle rather than on a straight line. A circular ordering of a book-embedding is equivalent to a linear ordering of a book-embedding. We can go from a circular ordering to a linear ordering by choosing a vertex to be at the start of the sequence. By ignoring the starting vertex, we can go from a linear ordering to a circular ordering. Another book-embedding of $K_5$, this time with a circular ordering, is in \cref{fig:circular_book-embedding}.

\begin{figure}[h!]
	\centering
	\includesvg[width = 0.3\textwidth]{figures/3page_K5_circular.svg}
	\caption[Three-page circular book-embedding of $K_5$]{Another book-embedding of $K_5$ on three pages, this time with a circular ordering.}\label{fig:circular_book-embedding}
\end{figure}

Book-embeddings were introduced by \textcite{kainenRecentResultsTopological1974, ollmannBookThicknessVarious1973} in the 1970s. A paper by \textcite{bernhartBookThicknessGraph1979} calculated the book-thickness of complete and bipartite graphs.
\begin{proposition}\label{lem:Pagenumber_1}
	A graph \(G\) can be embedded on a single page if and only if \(G\) is an outerplanar graph.
\end{proposition}
\begin{proof}
	Suppose $G$ is outerplanar, and embedded in $\mathbb{R}^2$. The outer-cycle formed by the vertices on the outerplanar graph partitions $\mathbb{R}^2$ into two sections, an inner-face and an outerface. Every vertex can be placed on the boundary of the inner-face and every edge can be placed on the inner-face without crossing. Then traversing around the boundary of the inner-face yields a circular ordering, where no two edges cross. Then this is a one-page book-embedding of $G$. 

	If $G$ has a one-page book-embedding, then embedding the circular book-embedding onto $\mathbb{R}^2$ through the inclusion map shows that $G$ is outerplanar. 
\end{proof}
\begin{proposition}\label{lem:Pagenumber_2}
	A graph \(G\) can be embedded on two pages if and only if \(G\) is a subgraph of a planar graph with a Hamiltonian cycle.
\end{proposition}

\begin{proof}
	Suppose $G$ is a subgraph of a planar graph $G'$ with a Hamiltonian cycle $C$. By the Jordan curve theorem, $\mathbb{R}^2 - C$ has two connected components $F_1$ and $F_2$. Choose a vertex $x_0$ and order the vertices with respect to the Hamiltonian cycle $C$ where $x_0$ is the first vertex. Give edges on $C$ colour $1$ or $2$. For all chordal edges of $C$ that lie on $F_1$, give these edges colour $1$. For all chordal edges of $C$ that lie on $F_2$, give these edges colour $2$. This is a book-embedding of $G'$ on two pages. 

	Suppose $G$ is embedded in a book with two pages. All spine edges can be added onto a single page with no crossings, creating a Hamiltonian cycle. Two pages are homeomorphic to $\mathbb{R}^2$ by flipping one page over, so $G$ is a subgraph of a planar graph with a Hamiltonian cycle. 
\end{proof}
\subsection{Properties of pagenumber}\label{ssec:Related_Properties}
\cref{lem:Edge_Bound} is contained within \textcite{bernhartBookThicknessGraph1979}.
\begin{proposition}\label{lem:Edge_Bound}
	If a graph \(G\) with $n$ vertices can be embedded on $k$ pages, then \(G\) has at most \(n + k(n-3)\) edges.
\end{proposition}
\begin{proof}
	Given a vertex ordering \(v_1 \leq v_2 \leq \cdots \leq v_n\), the spine edges can appear on any page. Furthermore, at most \(n-3\) non-spine edges appear on each page, as the edges on any page form a planar subgraph. The maximum number of edges in an outerplanar graph is \(2n - 3\) from \cref{thm:outerplanar_bound}, but spine edges are counted multiple times. Excluding spine edges, there are $n-3$ edges per page. Adding back on $n$ spine edges, the number of edges $m$ is at most $n + k (n - 3)$.
\end{proof}
\begin{proposition}\label{thm:Pagenumber_Complete_Graph}
	The complete graph $K_n$ has pagenumber $\lceil \frac{n}{2} \rceil$ when $n \geq 4$.
\end{proposition}
\begin{proof}
	Firstly, $\pn(K_n) \leq \lceil \frac{n}{2} \rceil$. Arrange the vertices of $K_n$ in any circular ordering $v_1 < v_2 < \cdots < v_n$. Then colour edges $v_1 v_2, v_2 v_{n}, v_{n} v_{3}, v_{3} v_{n-1}, \ldots$ in a zigzag pattern. Refer to \cref{fig:k8 coloured with colours} for a description of a zigzagging pattern. Rotate this pattern $\lceil n/2 \rceil$ times. 

	Secondly, $\pn(K_n) \geq \lceil \frac{n}{2} \rceil$. Use \cref{lem:Edge_Bound}. \(K_n\) has \(n\) vertices and \(\binom{n}{2}\) edges. Then \(\pn(K_n) \geq \frac{\binom{n}{2} - n}{n - 3} = \frac{n}{2}\) when \(n \geq 4\). As \(\pn(K_n)\) is an integer, take the ceiling of \(\frac{n}{2}\).
\end{proof}
\begin{figure}[ht]
	\centering
	\usetikzlibrary{graphs,graphs.standard}

\tikz
	\graph[nodes={circle, draw}] { 
		subgraph K_n [n=8,clockwise,radius=2cm];
		
		{[induced path, edges= red] 1,2,8,3,7,4,6,5},
		{[induced path, edges= blue] 8,1,7,2,6,3,5,4},
		{[induced path, edges= green] 7,8,6,1,5,2,4,3},
		{[induced path, edges= yellow] 6,7,5,8,4,1,3,2},
 };
	\caption[Embedding $K_8$ on four pages]{Circular embedding of \(K_8\) with four pages, the minimum possible.}\label{fig:k8 coloured with colours}
\end{figure}
The proof of \cref{thm:Pagenumber_Complete_Graph} is from \textcite{bernhartBookThicknessGraph1979}
Therefore, for any graph \(G\) on \(n\) vertices, \(n \geq 4\), \(\pn(G) \leq \lceil n/2 \rceil\).

\cref{thm:Colouring_Bound} bounds the chromatic number and is from \textcite{bernhartBookThicknessGraph1979}. This proof uses degeneracy. Recall $\delta(G)$ is the minimum degree of $G$. A graph $G$ is \textit{$k$-degenerate} if every subgraph $H \subseteq G$ has a vertex of degree at most $k$. 
A simple lemma connecting $k$-degenerate graphs and chromatic number helps prove this theorem. 
\begin{lemma}
	Every $k$-degenerate graph is $k + 1$-colourable.
\end{lemma}
\begin{proof}
	Let $G$ be a $k$-degenerate graph. If $G$ has a single vertex, then $G$ can be coloured in $k + 1$ colours. Now suppose this holds for all $k$-degenerate graphs with $|V(G)|- 1$ vertices, where $|V(G)| \geq 2$. Let $v$ be a vertex in $G$ with degree $k$. Then $G - v$ can be coloured inductively with $k + 1$ colours as $G - v$ is also $k$-degenerate. Now look at the neighbours of $v$ and colour $v$ with a colour that is not used by one of its neighbours. This can be done as $v$ has $n$ neighbours but $n + 1$ colours can be used. 
\end{proof}

This lemma is applied to \cref{thm:Colouring_Bound}. 

\begin{proposition}\label{thm:Colouring_Bound}
	For all graphs \(G\), \(\chi(G) \leq 2 \pn(G) + 2\).
\end{proposition}
\begin{proof}
	Let \(\pn(G) = k\) and suppose \(G\) has \(n\) vertices and \(m\) edges. Then the average degree of \(G\), \(d(G) = 2m/n\) by the handshaking lemma. So \(2\frac{m}{n} \leq 2 \frac{n + k(n-3)}{n} = 2 + 2k \frac{n-3}{n} < 2k + 2\). But this implies that \(\delta(G) < 2k + 1\), and for every subgraph $G' \subseteq G$, \(\partial(G') \leq 2k + 1\). However, this implies \(G\) is \((2k + 1)\)-degenerate, thus \(\chi(G) \leq 2k + 2\).
\end{proof}

Let $G$ be a graph. A \textit{subdivision} of an edge $uv \in E(G)$ deletes $uv$ and adds a new vertex $w$ with edges $uw$ and $wv$. A graph subdivision of $G$ is to do this for all edges in $G$. A $k$-subdivision of $G$ is to subdivide each edge $k$ times in $G$, so that the edge $e$ is replaced with a path $P$ of length $k$.\ \textcite{atneosenEmbeddabilityCompacta$n$books} proved that all graphs can be subdivided a finite number of times such the subdivision has pagenumber 3.\ \textcite{dujmovicStacksQueuesTracks2005} showed that it is sufficient to subdivide $O(\log\pn(G))$ times. 

Subdivisions show that the class of graphs with pagenumber $\leq p$ is not a minor-closed class, meaning that it does not hold that every minor $H$ of a graph $G$ has the property that $\pn(H) \leq \pn(G)$. 
Subdivide $K_n$ $n$ times. From \textcite{atneosenEmbeddabilityCompacta$n$books}, the subdivision of $K_n$ can be embedded on three pages. But $K_n$ is a minor of its subdivision, and from \cref{thm:Pagenumber_Complete_Graph}, $\pn(K_n) = \lceil \frac{n}{2} \rceil$. 

\begin{theorem}
	There exists a family of 2-degenerate graphs with unbounded pagenumber.
\end{theorem}
\begin{proof}
	Let $G_n$ be the complete graph $K_n$ with every edge subdivided once. Then $G_n$ is bipartite, so is $2$-colourable, and $2$-degenerate. However, from \textcite{eppsteinSeparatingThicknessGeometric2002}, for every $t$ there exists an $n$ such that $G_n$ cannot be embedded in $t-1$ pages. This proof uses \textit{Ramsey theory}.
\end{proof}

An \textit{expander graph} is a sparse, highly connected graph. Expander graphs share many properties with random graphs, but are constructed explicitly. One type of expander graph is a \textit{bipartite \varepsilon-expander}, where $\varepsilon \in (0, 1]$. A graph $G$ is a bipartite \varepsilon-expander if there exists a bipartition $ \{A, B\}$ of $V(G)$ such that $|A| = |B|$ and for all subsets $S \subset A$ where $|S| \leq \frac{|A|}{2}$, $|N(S)| \geq (1 + \varepsilon) |S|$. 
\textcite{dujmovicLayoutsExpanderGraphs2016} showed that there exists a bipartite \varepsilon-expander graph that can be embedded in 3 pages for all $\varepsilon$. 


Book-embeddings of graphs has applications in processor design by \textcite{chungEmbeddingGraphsBooks1987}, bioinformatics by \textcite{haslingerRNAStructuresPseudoknots1999}, and in graph drawings by \textcite{woodBoundedDegreeBook2002}. 
The project of finding upper and lower bounds of the pagenumber of planar graphs was started by \textcite{bernhartBookThicknessGraph1979} when they conjectured that planar graphs had unbounded pagenumber. However, \textcite{bussPagenumberPlanarGraphs1984} showed that all graphs could be embedded in nine pages, and \textcite{heathEmbeddingPlanarGraphs1984} brought down the number of needed pages to seven.\ \textcite{yannakakisEmbeddingPlanarGraphs1989} devised an algorithm to embed a graph in four pages. Yannakakis, in this paper, claimed that there exists planar graphs which cannot be embedded in three pages. However, his proof was incomplete, and the full proof was left unpublished. In 2020, \textcite{yannakakisPlanarGraphsThat2020} published his full proof. At around the same time, \textcite{bekosFourPagesAre2020} also proved that there exists a planar graph requiring four pages.

\textcite{malitzGraphsEdgesHave1994} proved that any graph with $e$ edges has pagenumber $O(\sqrt{e})$. Additionally, he proved that for every random $d$-regular graph $G$ with $n$ vertices, $\pn(G) \in \Omega(\sqrt{d} n^{1/2 - 1/d})$. For random 3-regular graphs $G$ with $n$ vertices, $\pn(G) \in \Omega(n^{1/6})$. These constructions of $\Omega(n^d)$ pagenumber graphs are not explicit.\ \textcite{eppsteinThreeDimensionalGraphProducts2024} showed that $\pn(P_n \boxtimes P_n \boxtimes P_n) \in \Theta(n^{1/3})$. This is an explicit construction of a graph of degree at most 26 with pagenumber in $\Theta(n^{1/3})$. 