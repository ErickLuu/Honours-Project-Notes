\subsection{Decomposing planar graphs}
This subsection discusses a tree-decomposition of planar graphs into $4$-connected components. Then this tree-decomposition can be applied to \cref{thm:clique_sum_pagenumber_bound}. Firstly, Menger's theorem is discussed as this is important for future proofs. Then a decomposition of planar graphs into $4$-connected torsos is proven.

\subsubsection{Menger's theorem}

Menger's theorem \cite{mengerZurAllgemeinenKurventheorie1927} is an essential theorem and is used throughout this section. 
Let \(G\) be a graph and \(A, B \subseteq V(G)\). An \(AB\)-path is a path in \(G\) that starts in \(A\) and ends in \(B\) with no internal vertices in \(A \cup B\). An \(AB\)-connector is a set of disjoint \(AB\)-paths. An \(AB\)-separator is a set \(S \subseteq V(G)\) so that \(G - S\) contains no \(AB\)-path. Then:
\begin{theorem}[Menger's theorem]\label{thm:Menger}
	Let $G$ be a graph and let $A, B \subseteq V(G)$. Then the size of the smallest \(AB\)-separator of \(G\) is equal to the size of the largest \(AB\)-connector.
\end{theorem}
Now take two distinct vertices \(x, y\). Let \(A = N_G(x) \cup \{x\} \) and \(B = N_G(y) \cup \{y\} \). Then \cref{thm:Menger} implies that:
\begin{theorem}[Menger's theorem, vertex-connectivity version]\label{thm:Menger_Vertex}
	A graph \(G\) is \(k\)-connected if and only if for any two distinct vertices, there are at least \(k\) internally disjoint paths between the two vertices.
\end{theorem}

\subsubsection{Planar Graph Decomposition}
The aim of this section is to prove \cref{lem:planar_graphs_4_connected_cliqesums}. This is done by an argument by induction to decompose a graph into smaller elements. 
\begin{proposition}\label{lem:planar_graphs_4_connected_cliqesums}
	Every planar graph has a tree-decomposition where each torso is a \(4\)-connected planar graph or $K_{\leq 4}$, with adhesion at most \(3\). Furthermore, adhesion sets between $4$-connected torsos are separating triangles.
\end{proposition}
This proof is built from the observation that a planar triangulation is $4$-connected if and only if it has no separating triangle. This observation is then extended through the proof to deal with all planar graphs rather than triangulations. Finally, induction is used to reduce the graph to smaller components. We first prove some properties about planar graphs, namely that a triangle can be added without breaking the property that the graph is planar.

Let $G$ be a plane graph. A \textit{separating triangle} on $G$ is a triangle that separates $\mathbb{R}^2$ into two components $U, V$ and there exist vertices in both the interior of $U$ and $V$. All nonseparating triangles bound faces. \cref{fig:separating_triangle} illustrates an example of a separating triangle. 
\begin{figure}[h!]
	\centering
	\includesvg[width = 0.8\textwidth]{figures/separating triangle.svg}
	\caption[Separating faces]{Two plane graphs. The one on the left has a blue separating triangle and the one on the right has a blue nonseparating triangle.}\label{fig:separating_triangle}
\end{figure}

\cref{lem:planar_graphs_4_connected_cliqesums} uses the fact that planar graphs are a minor-closed class, so contracting edges preserves planarity. \cref{lem:edge_contraction_2connected} is necessary to prove \cref{lem:planar_graphs_4_connected_cliqesums}. \cref{lem:rooted_triangle} is contained within \textcite{woodThomassensChoosabilityArgument2010}.

A \textit{rooted triangle} on a set $u, v, w$ on a graph $G$ is a model of $K_3$ in $G$ where $u, v, w$ are in separate components. 

\begin{lemma}\label{lem:edge_contraction_2connected}
	Let $v$ be a vertex in a 2-connected graph $G$. Then $v$ has a neighbour $w$ where $G/vw$ is 2-connected.
\end{lemma}
\begin{proof}
	Suppose on the contrary that for all neighbours $w$ of $v$, $G/vw$ is not $2$-connected, meaning that $vw$ is a cut-edge for all $w$. Now iterate over every neighbour of $v$. Let $H$ be the smallest component in $G - \{v,w\}$ over all neighbours $w$ of $G$, and let $w$ be the neighbour which achieves this minimum. As $G$ is $2$-connected, $v$ has a neighbour $x$ in $H$. Then $G - \{v, x\}$ is disconnected as $\{v,x\}$ is a cut-set. Furthermore, a component of $G - \{v,x\}$ is a proper subgraph of $H$. If this was not the case, then $G - \{v,w\}$ breaks $H$ into two components, but $H$ is connected. However, $H$ is minimal over all neighbours. Therefore, this subgraph cannot exist. By contradiction, $v$ has some neighbour $w$ where $G/vw$ is $2$-connected. 
\end{proof}

\begin{lemma}\label{lem:rooted_triangle}
	Every graph $G$ with vertices $u, v, w$ has a rooted $K_3$ on $u, v, w$ if and only if there exists no cut-set $S$ where $|S| \leq 1$ and $H - S$ places the set $\{u,v,w\} \setminus S$ each in separate components. 
\end{lemma}
\begin{proof}
	For the forwards direction, let $U,V,W$ be the three subgraphs that contain $u, v, w$ respectively, and contracting $U, V, W$ to a single vertex creates a $K_3$ subgraph. Let $x \in V(G)$ be a vertex. Then without loss of generality, $x$ is not in $U$ or $V$. Therefore, $G - x$ has $U$ and $V$ in the same component. No cut-set $S$ which places $\{u,v,w\} \setminus S$ in separate components exists. 

	For the backwards direction, we use induction. Suppose $|V(G)| = 3$, but no $K_3$ minor exists. Without loss of generality, $V(G)$ is a subgraph of a path $uvw$. Then setting $S = v$, $u$ and $w$ are disconnected. No such $S$ exists when $G = K_3$. 

	If $G$ is connected but not $2$-connected, take $x$ to be a cut-vertex. Since $G - x$ does not place $u,v,w$ in separate components, then without loss of generality, $u,v$ are in the same component. Let $y$ be a vertex that is a neighbour of $x$ in a component that does not contain $u,v$. Let $G'$ be the graph obtained by contracting $xy$ to a vertex $z$. If $y = w$, then $z = w$. In $G'$, $u,v$ are also in the same component of $G' - z$. Furthermore, any other cut-vertex in $G'$ is also a cut-vertex in $G$. $|V(G')| < |V(G)|$, so induction can be applied on $G'$. 

	If $G$ is 2-connected, then choose a vertex $x \in V(G)$ that is not $u,v,w$. Then there exists an edge $xy$ where $G/xy$ is $2$-connected. Therefore, no vertex can be a cut vertex in $G/xy$. Applying induction on $G/xy$, $G$ has a $K_3$ minor rooted at $u,v,w$.
\end{proof}


\cref{lem:rooted_triangle} proves \cref{thm:cutset_added_edges}. 

\begin{lemma}\label{thm:cutset_added_edges}
	Suppose $G$ is a $3$-connected planar graph with cut-set $u,v,w$. Then $G + uv, uw, vw$ is planar. 
\end{lemma}

\begin{proof}
	Let $G_1$ and $G_2$ be the two subgraphs that are disconnected in $G - \{u,v,w\}$. Then if $G_i + \{u,v,w\}$ has a rooted $K_3$ minor on $u,v,w$ for $i = 1,2$, then contract this minor to $K_3$ and delete all other vertices in $G_1$. Repeat for $G_2$. Then $G_i + uv + uw + vw$ is a planar graph, $i = 1,2$. Because $uvw$ is not a separating triangle in $G_i$, then $uvw$ bounds a face in $G_1$ and $G_2$. Draw $G_1$ and $G_2$ on the sphere and stereographically project $G_2$ so that $uvw$ bounds the outerface. Then draw $G_2$ on $uvw$ in $G_1$ for a graph drawing of $G + uv, uw, vw$. 
	
	Suppose $G_2 + \{u,v,w\}$ has no rooted $K_3$ minor. By \cref{lem:rooted_triangle}, $G_2$ has a cut-set $S$, $|S| \leq 1$, where $G_2 - S$ places $u,v,w - S$ in different components. Call the components of $G_2 - S$ that contain $u,v,w$, $G_u, G_v, G_w$ respectively. If $S$ is empty, then $G_2$ has components $G_u, G_v, G_w$, and assume without loss of generality that $G_u$ contains a vertex distinct from $u$. But this implies that $u$ is a cut vertex from $G_u$ to $G$, so $G$ is not $3$-connected. Now suppose that $G_u,G_v, G_w$ only contain $u$, $v$, $w$ respectively. Then $\{u,v,w\}$ is not a cutset. 
	
	Suppose $S = x$ distinct from $u,v,w$. Without loss of generality, suppose $G_u$ contains a vertex distinct from $u$. Then $x, u$ disconnects $G_u$ from $G$, so there exists a cutset of size 2. Now suppose $G_u, G_v, G_w$ only contain $u,v,w$ respectively. Then add the edges $uv, uw, vw$ without any crossings around $x$. Suppose without loss of generality that $S = v$. Then suppose without loss of generalisation $G_u$ contains another vertex distinct from $u$. Then $G - u - v$ separates $G_u$ from $G$, so $\{u,v\}$ is a cutset. Finally, suppose $G_u, G_w$ only contain $u$ and $w$ respectively. Then $v$ is a separator, so $G$ is not $3$-connected, or $G - u,v,w$ separates nothing. Then every single case contradicts the fact that $u,v,w$ is a minimal cut-set. Thus shown.
\end{proof}
An example of the cut-set vertices and the particular cases that arise is in \cref{fig:cutsets}.

\begin{figure}[h!]
	\centering
	\includesvg[width = 0.8\textwidth]{figures/planar_separators.svg}
	\caption[Decomposition of graph]{An illustration of the cases that are handled when dealing with cut-sets.}\label{fig:cutsets}
\end{figure}

Finally, this is the proof of \cref{lem:planar_graphs_4_connected_cliqesums}.
\begin{proof}
	Do induction on the number of vertices of $G$. 
	
	Suppose $|V(G)| \leq 3$. Then $\mathcal{T}$ has nodes $x$, $y$ and edge $xy$, and $B_x = B_y = V(G)$. Then $G \langle B_x \rangle = G \langle B_y \rangle = K_3$. An example of this is in \cref{fig:treedecomp_k4}. 

	\begin{figure}[h!]
		\centering
		\includesvg[width = 0.8\textwidth]{figures/decomposition4.svg}
		\caption[Tree-decomposition of $K_4$]{This is a tree-decomposition of $K_3$ and $K_4$ of adhesion $\leq 3$. Every torso is either $K_3$ or $K_4$.}\label{fig:treedecomp_k4}
	\end{figure}
	
	Suppose $|V(G)| = 4$, say $V(G) = \{a,b,c,d\}$.  Then $\mathcal{T}$ has nodes $u, i,j,k,l$ and edges $ui, uj, uk, ul$. Let $B_u = \{a,b,c,d\}$, $B_i = \{b,c,d\}$, $B_j = \{a,c,d\}$, $B_k = \{a,b,d\}$, $B_l = \{a,b,c\}$. Then $G\langle B_u \rangle = K_4$, $G\langle B_i \rangle =G\langle B_j\rangle = G\langle B_k \rangle = G\langle B_l \rangle = K_3$. Then this also satisfies the proposition. 

	Now suppose $G$ is $4$-connected. Then place the entirety of $G$ in a single bag.

	Suppose $G$ is disconnected. Then run this algorithm on every component and add arbitrary edges between each subtree to make a tree-decomposition.

	Suppose $G$ is connected but not $2$-connected, so there exists a cut vertex $v$. Now let $G_1, G_2$ be the two disconnected subgraphs of $G - v$, where both $G_1$ and $G_2$ are nonempty subgraphs. $G_1 + v$ and $G_2 + v$ are planar graphs as they are planar subgraphs. Then $|V(G_i + v)| < |V(G)|$, so use induction on $G_1 + v$ and $G_2 + v$. This yields a tree-decomposition $\tree_1, \tree_2$ with the properties above. Add an edge between a bag $B_1$ in $\tree_1$ containing $v$ and a bag $B_2$ in $\tree_2$ containing $v$. Then the adhesion of these two bags is $1$, so it satisfies the property above. 

	Suppose $G$ is $2$-connected but not $3$-connected, so there exists a cut-set $\{u,v\}$. Let $G_1, G_2$ be the two subgraphs that are separated by $G - \{u, v\}$. Firstly, $G_1' := G_1 + \{u,v\} + uv$ is a planar graph. As $G$ is $2$-connected, there exists a cycle of $u$ and $v$, so there either exists an edge $uv$ or there exists a path $P$ in $G_2$ from $u$ to $v$. Contracting this path $P$ to an edge yields another planar graph $G_1'$ where $uv$ is an edge. Repeating this, $G_2' := G_2 + \{u,v\} + uv$ is also a planar graph. Then use induction on $G_1'$ and $G_2'$ with tree-decomposition $\tree_1, \tree_2$ respectively. As $uv$ is an edge in both subgraphs, there exists a bag $B_1, B_2$ in $\tree_1, \tree_2$ that contains both $u$ and $v$. Add an edge between both bags. Finally, the edge $uv$ is in every torso $G\langle B_x \rangle$ where $u, v \in B_x$. Without loss of generality, suppose $B_x$ was originally in $\tree_1$. If $u$ and $v$ are in a bag $B_x$, then there exists a unique path of nodes from $B_x$ to $B_1$ containing $u$ and $v$. This is because of the property that the bags that contain $u$ are connected, and between any two vertices on a tree there exists a unique path. Then $B_x$ is adjacent to a bag that contains $u$ and $v$, so $G \langle B_x \rangle$ contains $uv$. 

	To show that $G$ has a graph drawing with $uv$ as an edge, take an embedding of $G_1'$ on the sphere and project this embedding to the plane so that $uv$ is adjacent to the outerface. Do the same for $G_2'$. Then glue these two embeddings along $uv$ both on the outerface. 

	Suppose that $G$ is $3$-connected but not $4$-connected. Then $G$ has a cut-set $u,v,w$. Let $G' = G + \{uv, uw, vw\}$. $G'$ is planar, from \cref{thm:cutset_added_edges}. 
	
	If $uvw$ does not bound a face, then $uvw$ is a separating triangle. Use the Jordan curve theorem to split $\mathbb{R}^2 - uvw$ into two regions, an interior and an exterior. Let $G_1$ be the interior of $uvw$ and the triangle $uvw$ and let $G_2$ be the exterior of $uvw$ and the triangle $uvw$. 

	If $uvw$ bounds a face, take $G_1$ and $G_2$ to be the nonempty separating components of $G - \{u,v,w\}$. Then $G_1' = G_1 + uv + uw + vw$ is a planar graph where $uvw$ bounds a face, and $G_2' = G_2 + uv + uw + vw$ is a planar graph where $uvw$ bounds a face. Embed $G_2'$ on a sphere. Then there exists a stereographic projection of $G_2'$ where $uvw$ is the outerface, and place this embedding in $G_1'$. 

	Use induction on $G_1$ and $G_2$ with tree-decompositions $\tree_1$ and $\tree_2$ respectively. As $uvw$ is a clique, there exists a bag $B_1$ and a bag $B_2$ that contains $uvw$. Add an edge between bags $uvw$. To show that every torso $G\langle B_x \rangle$ that contains $u$ and $v$ contains $uv$, repeat the same argument for the $2$-connected case. $B_x$ has a neighbouring bag $B_y$ that contains $u$ and $v$, where $B_y$ is on the unique path from $B_x$ to $B_1$. Then $G\langle B_x \rangle$ contains $u, v$ as $B_y$ contains $u$ and $v$. Repeat this argument for $w$. Then every torso is $4$-connected as no edges are deleted in the torso operation and every torso is planar as adding $uv, uw, vw$ preserves planarity. 

	Since all torso edges are edges which preserve planarity in $G$, then $G$ with all torso edges is also planar. Therefore, every torso of $G$ in this tree-decomposition is planar.
	As this handles every case, then by induction, $G \langle B_x \rangle$ is a 4-connected planar graph or $K_{\leq 4}$. 
\end{proof}