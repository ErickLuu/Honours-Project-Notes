\subsection{Bounding the number of pages of a planar graph}
This subsection uses \cref{lem:planar_graphs_4_connected_cliqesums} and \cref{thm:clique_sum_pagenumber_bound} to find a book-embedding of planar graphs. This book-embedding is different from the one provided by \textcite{yannakakisEmbeddingPlanarGraphs1989} as it does not require a planar triangulation. 


\cref{thm:4-connected_planar_ham_cycle} is used to prove \cref{thm:Planar Graph Hickingbotham Bound}. A \textit{facial walk} is a sequence of edges \(e_1, \ldots, e_n\) that bound a face such that \(e_i\) is incident to \(e_{i + 1}\) modulo \(n\) for every \(i\). The length of the facial walk is \(n\).


\begin{corollary}\label{thm:Planar Graph Hickingbotham Bound}
	Let \(G\) be a 2-connected planar graph. Then $G$ can be embedded on $11$ pages, with book-embedding $(<, \rho)$. The book-embedding $<$ restricted to the vertices outer cycle $C$ traverses every vertex in order of the traversal of the boundary of the outerface except for vertices in a cut-set of size 3.
\end{corollary}
\begin{proof}
	From \cref{thm:4-connected_planar_ham_cycle}, every $4$-connected planar graph is Hamiltonian. Furthermore, $K_4$ is a Hamiltonian planar graph.
	Recall that Hamiltonian planar graphs can be embedded on two pages, from \cref{lem:Pagenumber_2}. 
	Then apply \cref{thm:clique_sum_pagenumber_bound} with tree-decomposition from \cref{lem:planar_graphs_4_connected_cliqesums} to $G$. Then $G$ can be embedded on \(2 \cdot 4 + 3 = 11\) pages. Furthermore, the embedding restricted a face boundary preserves the ordering of the facial walk.

	From the construction given in \cref{lem:planar_graphs_4_connected_cliqesums}, every $4$-connected component is glued to another $4$-connected component by a separating face. Now a vertex must be moved to the front, so the vertex is not preserved. 
\end{proof}

