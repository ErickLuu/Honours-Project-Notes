% !TEX root = ./thesis.tex

\section{Bounded treewidth and pagenumber}\label{sec:Bounded_Treewidth}

This section discusses bounding the number of pages necessary to embed a graph of bounded treewidth. This section starts with a very simple proof of colouring graphs of bounded treewidth. Next is a discussion of bounding the number of pages of a graph with bounded treewidth.

Let $G$ be a graph. Recall a clique is a subgraph that is a complete graph. The clique-number $\omega(G)$ is the largest clique in $G$. A graph $G$ is \textit{perfect} if $\chi(G) = \omega(G)$. Let $T$ be a tree and $\mathcal{T}$ be a set of subtrees in $T$. The \textit{intersection graph} $(T, \mathcal{T})$ has vertex set $\mathcal{T}$ and edge set $T_u T_v$ if and only if $T_u$ and $T_v$ have a vertex in common. Intersection graphs are perfect graphs. 

This is a simple proof to bound the chromatic number of graphs of bounded treewidth. Many themes in this proof appear in \cref{thm:bded_treewidth_bded_pagenumber} and \cref{thm:clique_sum_pagenumber_bound}. 
\begin{theorem}
	Every graph with treewidth at most $k$ is $k + 1$-colourable.
\end{theorem}

\begin{proof}
	Let $G$ be a graph of treewidth at most $k$ and let $(T, {(B_x)}_x)$ be a tree-decomposition of $G$ of width $k$. For all $v \in G$, let $T_v$ be the induced subtree of bags containing $v$ in $T$. Let $G^*$ be the intersection graph of $\{T_v : v \in G\}$. There is a natural correspondence between $V(G)$ and $V(G^*)$, by sending a vertex to its intersection tree and vice versa. Then $G^*$ is perfect as $G^*$ is an intersection graph. Furthermore, $G$ is a subgraph of $G^*$. If $uv$ is an edge in $G$, then $u$ and $v$ are in the same bag $B_x$. But then $T_u$ and $T_v$ intersect at $x$. Then $T_u$ and $T_v$ share an edge. The largest clique in $G^*$ is of size $k + 1$. If there is a clique $C$ in $G$, then each tree in $C$ is pairwise adjacent, so the pairwise intersection of any two trees in $C$ is nonempty. Then by the Helly property, there exists a bag $B_x$ where every tree in $C$ has a vertex $x$. Then the corresponding vertices of $C$ in $G$ are all in $B_x$. But $|B_x| \leq k + 1$, so the largest clique in $G^*$ is of size $k + 1$. Since $G^*$ is perfect, then $G^*$ is $k + 1$-colourable. As $G$ is a subgraph of $G^*$, $G$ is $k + 1$-colourable. 
\end{proof}

\begin{theorem}[\textcite{ganleyPagenumberTrees2001}]\label{thm:bded_treewidth_bded_pagenumber}
	Every graph \(G\) with treewidth at most $k$ can be embedded on $k + 1$ pages.
\end{theorem}
\textcite{ganleyPagenumberTrees2001} considered the case where \(G\) is a \(k\)-tree. This proof does not use $k$-trees and a tree-decomposition of \(G\) is used instead. 

\begin{proof}
	Let  $(T, (B_x)_x)$ be a tree-decomposition of \(G\). Perform a depth-first search on \(T\), starting at an arbitrary root node \(r\). Let \(h(v)\) of a vertex \(v\) in \(V(G)\) be the first time \(v \in B_x\), $x \in T$ appears in the depth-first search. The function $h(v)$ is a 'home' function and determines when $v$ appears in the book-embedding. Within each bag $B_x$, if two vertices $v,w$ first appear in $B_x$, then $v$ and $w$ are ordered arbitrarily in the book-embedding. Now consider the subtree \(T_v\) induced by the bags \(B_x\) containing \(v\). Let \(H\) be the intersection graph of the subtrees. Let \(V(H) = \lbrace T_v : v \in G \rbrace\) and \(T_u T_v \in E(H)\) if there exists a common bag containing both $u$ and $v$. Then \(H\) is perfect as all intersection graphs of subtrees are perfect. Therefore, \(\chi(H) = \omega(H)\). Then \(H\) is \(k + 1\)-colourable. Let $z: V(H) \rightarrow [k + 1]$ be a proper colouring of $H$. 

	Assign the edges of \(G\) a colour. Colour each edge \(uv \in E(G)\) as follows:
	\begin{equation}
		c(uv) =
		\begin{cases}
			z(T_u) & \text{ if } h(u) \leq h(v), \\
			z(T_v) & \text{ if } h(v) \leq h(u)
		\end{cases}
	\end{equation}
	Then $(h, c)$ is a proper book-embedding of \(G\). Suppose that edges \(uv\), \(xy\) cross, so \(h(u) \leq h(x) \leq h(v) \leq h(y)\). However, this implies that $u,x,v$ are in some bag $B$. Then \(uv\) is an edge in \(B\), and we do a depth-first search to establish the ordering of $h(u)$. So \(u, x, v\) are in the same bags. However, this implies that the trees \(T_u\) and \(T_x\) intersect, meaning that \(c(uv) \neq c(xy)\). Therefore, all crossing edges are assigned different colours. Finally, the number of pages used is \(\chi(H) = k + 1\), so \(\pn(G) \leq k + 1\). 
\end{proof}
For a simpler example, consider the case when $T$ is a path. The proof given below is identical in spirit to \cref{thm:bded_treewidth_bded_pagenumber}, but omits many of the details specific to trees. 
Let $G$ be a graph. A linear book-embedding $<$ \textit{maintains} the order of a path $P = (x_1, x_2, \ldots, x_n)$ if $x_i, x_j \in P$ and $i < j$, then $x_i < x_j$ in the book-embedding. A circular book-embedding $<$ maintains the order of a cycle $C$ if we can start $<$ and $C$ at the same vertex and the new linear ordering maintains the order of the induced path. 

\begin{theorem}\label{thm:pw_book_embedding}
	Every graph \(G\) with a path-decomposition $\{B_1, B_2, \ldots , B_n\}$ of width $k$ and a path $P = x_1, x_2, \ldots, x_n$ where $x_i \in B_i$ can be embedded on $k + 1$ pages. Furthermore, this embedding restricted to $P$ maintains the order of $P$.
\end{theorem}

\begin{proof}
	For every vertex $v$ not in $P$, let $h(v)$ be the smallest $i$ where $v \in B_i$. Let $C_i$ be the set containing all vertices where $h(v) = i$. The vertex ordering $<$ is $P$, but we place every vertex in $C_i$ after $x_i$. Colour each vertex so that in each bag, no two vertices are assigned the same colour. This is possible to do in $k+1$ colours as interval intersection graphs are perfect. Then colour edges where the edge $uv$ is given the colour of $u$ when $u < v$. Then this is a proper book-embedding. If two edges $uv$, $xy$ cross and $u < x < v < y$, then $u,v$ is in some bag $B$. But as $x$ is a subpath, and is between $u$ and $v$, then $x$ is also in the same bag as $uv$. So $u,x,v$ are in the same bag. But then $uv$ and $xy$ are given different colours. 
\end{proof}

\textcite{dujmovicGraphTreewidthGeometric2007} showed that when $k \leq 2$, every graph with treewidth $k$ can be embedded on $k$ pages. However, when $k \geq 3$, there exist graphs with treewidth $k$ that cannot be embedded on $k$ pages. Therefore, when $k \geq 3$, the bound by \textcite{ganleyPagenumberTrees2001} is tight. 

Let $\mathcal{F}_H$ be the family of graphs that exclude a planar graph $H$. The seminal result from \cite{robertsonQuicklyExcludingPlanar1994} by \citeauthor{robertsonQuicklyExcludingPlanar1994} states that there exists a function $t = t(H)$ where every graph in $\mathcal{F}_H$ has treewidth at most $t$. From \cref{thm:bded_treewidth_bded_pagenumber}, every graph in  $\mathcal{F}_H$ can be embedded on $t + 1$ pages.