\section{Tree-decompositions and pagenumber}\label{sec:BoundedPagenumber}
This section discusses tree-decompositions and book-embeddings of graphs where every component has a tree-decomposition. \textcite{hickingbothamStackNumberCliqueSum2023} proved that every graph with a tree-decomposition where every torso is embeddable on $s$ pages can be embedded on $2s^2 + 4s + 1$ pages. This proof can be seen as an extension of \cref{thm:bded_treewidth_bded_pagenumber}, where each torso has bounded pagenumber, rather than a bounded number of vertices. This proof is then applied to planar graphs, using a lemma that every planar graph has an embedding and tree-decomposition where every torso is either $K_{\leq 4}$ or a $4$-connected planar graph. 

A rooted tree $(T, r)$ is a tree with $r$ as a root vertex. $ ((T,r), (B_x)_x)$ is a tree-decomposition with $r$ as the root node. 
\begin{theorem}[By \textcite{hickingbothamStackNumberCliqueSum2023}]\label{thm:clique_sum_pagenumber_bound}
	Let \(G\) be a graph with a rooted tree-decomposition \(((T, r), (B_x)_x)\). Suppose every torso \(G \langle B_x \rangle\) can be embedded on \(s\) pages and is \(c\)-colourable. Further, suppose the adhesion of the tree-decomposition is at most \(\ell\).
	Then $G$ can be embedded on \( cs + \ell \) pages. Let this book-embedding be $(<, \psi)$. Let $G\langle B_x \rangle$ be embedded on $(<_x, \psi_x)$, with parent $B_y$. 
	
	Restricting $(<, \psi)$ to $B_x$ yields the same book-embedding as $<_x$, except the adhesion set $B_x \cap B_y$ where $|B_x \cap B_y|\ell$ is moved to the start of the book-embedding, and each vertex in $B_x \cap B_y$ are given at most $\ell$ new colours. 
	
	Furthermore, if the adhesion set $B_x \cap B_y$ is consecutive in $<_x$, then $<_x = <$ up to a circular ordering.
\end{theorem}

To prove\cref{thm:clique_sum_pagenumber_bound}, the hypothesis in \cref{thm:clique_sum_pagenumber_bound} is strengthened so that induction can be applied, (\cref{lem:Hickingbotham_Lemma}) Then given a tree-decomposition, a leaf is removed. Then induction is done on the rest of the tree. Finally, the leaf is added back in so that the stronger induction hypothesis can be shown to hold. 

Let \(C\) be a clique in \(G\) and let \(\sigma_C = (u_1, \ldots , u_k)\) be a vertex ordering of \(V(C)\), and let \(C \leq \ell \). Let $J$ be a clique in $G$. A vertex $v$ is a \textit{rainbow vertex} in $J$ in a book-embedding $(<, \psi)$ if the set of edges $\{u_i v | u_i < v, u_i \in J\}$ each have distinct colours. The structure of the book-embedding will look like this: \((\underbrace{u_1, u_2, \ldots, u_k}_{\text{Vertices in } C}, \underbrace{v_1, v_2, \ldots, v_l}_{\text{Vertices not in }C})\).

\cref{lem:Hickingbotham_Lemma} is a strengthening of \cref{thm:clique_sum_pagenumber_bound} in the case that the tree-decomposition of $G$ is a single bag. The reason for the strengthening is to apply the induction hypothesis.
\begin{lemma}\label{lem:Hickingbotham_Lemma}
	Let \(G\) be an $s$-page embeddable, $c$-colourable graph, with book-embedding \((\leq_a, \psi_a)\). Let $C$ be a clique in $G$ with an ordering \(\sigma_C\), and $|C| \leq \ell$. There exists a \(cs + \ell\)-page layout \((\leq, \psi)\) of \(G\) where:
	\begin{enumerate}
		\item The vertex ordering has the structure \((\underbrace{u_1, u_2, \ldots, u_k}_{\text{Vertices in } C}, \underbrace{v_1, v_2, \ldots, v_l}_{\text{Vertices not in }C})\).
		\item For every \(u \in V(C)\), the edges \(\lbrace uv \in E(G) : u \leq v \rbrace\) are a monochromatic star.
		\item For every clique \(J\), the last vertex of \(J\) is a rainbow-vertex.
	\end{enumerate}

	If $C$ is consecutive in $\leq_a$, then $\leq_a = \leq$ up to a circular ordering.
\end{lemma}
\begin{proof}
	Let \(\rho: V(G) \rightarrow [c]\) be a proper colouring of \(V(G)\).

	Let \(u_1, \ldots, u_k\) be the vertices of \(C\) ordered by \(\sigma_C\). Note that \(k \leq \ell\). Then the new ordering starts with \(u_1 \leq u_2 \leq \ldots, \leq u_k\), and all vertices not in \(K\) are placed after, according to \(\leq_a\).

	If $C$ is consecutive in $\leq_a$, then $C$ is at the start of the ordering. 

	The edge-colouring \(\psi\) is defined as follows. For every edge \(u_i v\) where \(u_i \in V(C)\) and \(u_i \leq v\), \(\psi(u_i v) = i\). If neither \(u\) nor \(v\) are in \(V(C)\), and \(u \leq v\), then let \(\psi(uv) = (\rho(u), \psi_a(uv))\). Then this edge-colouring requires \(|\rho| |\psi_a| + k \leq cs + \ell\) pages.

	Now we show \((\leq, \psi)\) is a proper book-embedding. Suppose there exists edges \(uv\) and \(xy\) where \(\psi(uv) = \psi(xy)\). Suppose that \(u\) is the smallest vertex in the ordering \(\leq\). If \(u \in V(C)\), then the edge \(uv\) is assigned the page consisting of only edges adjacent to $u$. So \(x = u\), but this is a star. Therefore, the edges do not cross. Therefore, \(u, v, x, y\) are not in \(V(C)\). But \((\leq, \psi)\) restricted to the subgraph $G - C$ is \((\leq_a, \psi_a)\), which is a proper book-embedding. Therefore, \((\leq, \psi)\) is a proper book-embedding.  

	Properties 1 and 2 are immediate from the construction of \((\leq, \psi)\). For property 3, consider a clique \(J\) in \(G\). Then the last vertex of \(J\) is rainbow. Suppose the last vertex of \(J\) is \(w\), and let \(u, v\) be earlier vertices. Since \(u\) and \(v\) necessarily are assigned different colours in the colouring, then \(\psi(uw) = (\rho(u), \psi_a(uw))\) and \(\psi(vw) = (\rho(v), \psi_a(vw))\). Therefore, the two edges are on different pages. Thus, \(w\) is a rainbow vertex.
\end{proof}

\begin{theorem}[\textcite{hickingbothamStackNumberCliqueSum2023}]
	Suppose a graph \(G\) has a rooted tree-decomposition \(((T,r), (B_x)_x)\) with torsos \(G \langle B_x \rangle\) and adhesion at most \(\ell\). Suppose that for every torso $G\langle B_x \rangle$, \(\pn(G\langle B_x \rangle) \leq s\) and \(\chi(G\langle B_x \rangle) \leq c\). Then \(G\) can be embedded on \(cs + \ell\) pages.

	Let this book-embedding be $(<, \psi)$. Let $G\langle B_x \rangle$ be embedded on $(<_x, \psi_x)$, with parent $B_y$. 
	
	Restricting $(<, \psi)$ to $B_x$ yields the same book-embedding as $<_x$, except the adhesion set $B_x \cap B_y$ where $|B_x \cap B_y|\ell$ is moved to the start of the book-embedding, and each vertex in $B_x \cap B_y$ are given at most $\ell$ new colours. 
	
	Furthermore, if the adhesion set $B_x \cap B_y$ is consecutive in $<_x$, then $<_x = <$ up to a circular ordering.
\end{theorem}
Order the nodes \(v_0, \ldots, v_k\) in $T$ with respect to a breath-first search. Let $B_i = B_{v_i}$. 
A breadth-first search maintains the property that for each \(i\), \(T[v_0, \ldots, v_{i}]\) is a tree and \(v_i\) is a leaf in \(T[v_0, \ldots, v_{i}]\).
\begin{proof}
	We prove the stronger statement that there exists a book-embedding with the property that the last vertex of any clique \(J\) is a rainbow vertex. For short, this property is the \textit{rainbow-clique} property. 

	Suppose $G$ has a tree-decomposition consisting of a single torso with the properties above. Then \(G\langle B_0 \rangle\) is a single graph with \(\pn(G) \leq s\). Choose \(C\) to be the first vertex \(v\) in \( < \). Then by \cref{lem:Hickingbotham_Lemma}, $G$ can be embedded on \(cs + 1\) pages and every last vertex in a clique is a rainbow vertex.

	Suppose $G$ has a tree-decomposition $(T, (B_x)_x)$ with the properties above. Take a breadth-first search of $T$, with vertex ordering $v_0, \ldots, v_n$. For the induction hypothesis, suppose that the induced subgraph $G' := G[B_0 \cup B_1 \cup \ldots \cup B_{n-1}]$ maintains the rainbow-clique property with pagenumber of at most \(cs + \ell\).  
	Let \(C\) be the adhesion clique between \(G \langle B_n \rangle\) and $G'$. Then let \((\leq_n, \psi_n)\) be the \(cs + \ell\)-pagenumber book-embedding of \(G \langle B_n \rangle\) that starts with \(V(C)\). Let \((\leq_{n-1}, \psi_{n-1})\) be the book-embedding of \(G'\). By the induction hypothesis, \((\leq_{n-1}, \psi_{n-1})\) is a \(cs + \ell\)-page book-embedding with the rainbow-clique property.

	We construct a new book-embedding \((\leq, \psi)\).
	Let \(w \in V(C)\) be the last vertex of \(C\) with respect to \(\leq_{n-1}\). Then insert \(V(G \langle B_n \rangle) - C\) between \(w\) and its successor in $G'$ with the order of \(\leq_{n-1}\) to make $\leq$. For the page assignment \(\psi\), if \(uv \in E(G')\), then \(\psi(uv) = \psi_{n-1}(uv)\). For edges in $G \langle B_n \rangle$, permute the edge assignments of \(\psi_n\) such that for all \(u \in V(C)\), \(\psi(uv) = \psi_n(uw)\) for $v \in C$ and $u \leq_n v$. This is possible as \(w\) is a rainbow vertex and the edges \(\{uv : v \in C, u \leq_n v\}\) are assigned to a unique page in \(\psi_n\). Finally, let \(\psi(uv) = \psi_n(uv)\) for all remaining edges in $G \langle B_n \rangle$. 

	We claim that \((\leq , \psi)\) is a book-embedding that satisfies the induction hypothesis. Suppose that \(\psi(uv) = \psi(xy)\). If \(uv, xy \in E(G')\), then by the induction hypothesis, they do not cross. Similarly, if \(uv, xy \in E(G \langle B_n \rangle)\), then they do not cross as well. If \(uv\) is in \(E(G')\) and \(xy \in E(G \langle B_n \rangle)\), then they will go over each other or be sequential and therefore will not cross.
	Finally, if \(u, v, x, y \in C\), then by the induction hypothesis on \(G'\), they do not cross either. The final case is if \(uv \in E(G\langle B_{n} \rangle)\) and \(u \in V(C)\), \(v \notin V(C)\), \(xy \in E(G')\). If \(uv\) and \(xy\) cross, then \(xy\) and \(uw\) will cross. But this will contradict the book-embedding of \(G'\) as $u, w, x, y$ are in $G'$.

	Let \(J\) be a clique in \(G\). Then $J$ is either only in $G'$, only in $G\langle B_n \rangle$, or shares vertices with $C$. This is because $V(G') \cap V(G\langle B_n \rangle) = C$, so $C$ is a separator between $G'$ and $G \langle B_n \rangle$. If $J$ is only in $G'$ or only in $G\langle B_n \rangle$, then the last vertex of $J$ is a rainbow vertex by hypothesis. If $J$ shares vertices with $C$, then the last vertex of $J$ with $\leq$ is in $G\langle B_n \rangle$ by construction. However, this vertex is a rainbow vertex, as the last vertex in $J$ with respect to $\leq$ is the last vertex in $G\langle B_n \rangle$ with respect to $\leq_n$. 
\end{proof}

\cref{corr:pagenumber_clique_bound} is the contrapositive of \cref{thm:Pagenumber_Complete_Graph}.

\begin{corollary}\label{corr:pagenumber_clique_bound}
	Suppose a graph $G$ can be embedded on $s$ pages where $s \geq 2$. Then \(G\) does not have $K_{2s + 1}$ as a subgraph.
\end{corollary}
Therefore, \(\ell \leq 2s + 1\).

Let $G$ be a graph with a book-embedding on $s$ pages. Then $G$ is $2s + 2$-colourable, from \cref{thm:Colouring_Bound}.
\begin{corollary}[\textcite{hickingbothamStackNumberCliqueSum2023}]\label{corr:bded_pn_tree_decomp}
	Let \(G\) be a graph with a tree-decomposition \((T, (B_x)_x)\) where each torso \(G \langle B_x \rangle\) can be embedded on $s$ pages. Then from \cref{thm:clique_sum_pagenumber_bound}, with $\ell \leq 2s + 1$ and $\chi(G) \leq 2 s + 2$, \(G\) can be embedded on \(2s^2 + 4s + 1\) pages.
\end{corollary}

We can apply \cref{thm:clique_sum_pagenumber_bound} to prove that every $K_5$-minor-free graph has bounded pagenumber. \cref{thm:WagnersTheorem} provides a useful characterisation of $K_5$-minor-free graphs. 
\begin{theorem}[Wagner's theorem\cite{wagnerUeberEigenschaftEbenen1937}]\label{thm:WagnersTheorem}
	Let \(G\) be a \(K_5\)-minor-free graph. Then \(G\) has a tree-decomposition of adhesion $\leq 3$ where every torso is either a planar graph or \(V_8\).
\end{theorem}

A drawing of $V_8$ with a single crossing is in \cref{fig:wagner_single_crossing}. 

\begin{figure}[h!]
	\centering
	\begin{minipage}{0.5\textwidth}
		\centering
		\includesvg[width = 0.8\textwidth]{figures/wagner-graph.svg}
		\caption{$V_8$ drawn with a single crossing}\label{fig:wagner_single_crossing}
	\end{minipage}%
	\begin{minipage}{0.5\textwidth}
		\centering
		\includesvg[width = 0.8\textwidth]{figures/wagner-graph_bookembedding.svg}
		\caption[V8 book embedding]{A 3-page book-embedding of $V_8$}\label{fig:wagner_bookembedding}
	\end{minipage}
\end{figure}

\begin{theorem}
	Let \(G\) be a \(K_5\)-minor-free graph. Then \(G\) has pagenumber \(\leq 19\).
\end{theorem}

\begin{proof}
	Suppose \(G\) is \(K_5\)-minor-free. Then by Wagner's theorem \cite{wagnerUeberEigenschaftEbenen1937}, \(G\) has a tree-decomposition of adhesion at most 3 where every torso is either a planar graph or the Wagner graph.
	\textcite{yannakakisEmbeddingPlanarGraphs1989} showed that planar graphs are \(4\)-colourable and can be embedded on four pages. The Wagner graph is \(3\)-colourable and can be embedded on three pages, as shown in \cref{fig:wagner_bookembedding}. Therefore, if \(G\) is \(K_5\)-minor-free, then \(G\) has pagenumber at most \(4 \cdot 4 + 3 = 19\) from \cref{thm:clique_sum_pagenumber_bound}.
\end{proof}
