\documentclass{article}
\usepackage[margin=1in]{geometry}
\usepackage{amsmath}
\usepackage{amssymb}
\usepackage{amsthm}
\usepackage{url}
\usepackage{todonotes}

% Environments

\newtheorem{theorem}{Theorem}
\newtheorem{proposition}[theorem]{Proposition}
\newtheorem{corollary}[theorem]{Corollary}
\newtheorem{lemma}[theorem]{Lemma}
\newtheorem{definition}[theorem]{Definition}
\newtheorem{conjecture}[theorem]{Conjecture}
\newtheorem{remark}[theorem]{Remark}


\theoremstyle{definition}
\newtheorem{example}[theorem]{Example}

\numberwithin{theorem}{section}
\numberwithin{equation}{section}

\DeclareMathOperator{\Int}{int}
\DeclareMathOperator{\Star}{st}
\DeclareMathOperator{\Lk}{lk}

%opening
\title{Low Dimensional Topology}
\author{Eric Luu}

\begin{document}

\section{Knots}
A \textit{knot} is a subset $K$ of a topological space $X$ that is homeomorphic to a sphere $S^p$. As an example, $K = S^0$ in $X = S^1$. A \textit{link} is a subset $K$ homeomorphic to a disjoint union of spheres, as $S^{p_1} \sqcup  S^{p_2} \sqcup \ldots \sqcup S^{p_n}$. A theme throughout will be seeing when two knots are the same. We say that $K$ and $K'$ in $X$ are equivalent if there exists a homeomorphism $h : X \rightarrow X$ such that $h(K) = K'$. 
Such a homeomorphism is also written as $h : (X, K) \rightarrow (X, K')$. For links, we assign an ordering and require the homeomorphism to preserve the ordering. 
An alternative definition is that a knot is an embedding $K: S^p \rightarrow X$, but we have that $K = K(S^p)$ as its image. The corresponding version of equivalence between knots $K, K'$ is a map $h: X \rightarrow X$ such that $h \circ K = K'$. 

There exists an ambient isotopy $H : X \times I \rightarrow X$ where $H(x, t) = h_t(x)$ such that $h_1(k) = k'$. Isotopy is a homotopy with the following condition that $h_0(x) = Id$ and $h_t$ is a homeomorphism for all $t$. 

A tame knot can be embedded using finitely many closed connected line segments. Wild knots are defined as knots which are not tame. 

\section{Categories}
\subsection{Top}
Top is the category of manifolds. An $n$-manifold $M$ is a topological space which is:
\begin{itemize}
    \item Hausdorff
    \item Second-Countable
    \item Locally Euclidean.
\end{itemize}
Locally Euclidean means that for all points $x$ in $M$, there exists an open neighbourhood $U$ of $x$ and a homeomorphism $\varphi:U \rightarrow U' \subseteq \mathbb{R}^n$. We say that $\varphi$ is a chart. Morphisms are homeomorphisms between manifolds. 

\subsection{Smooth}
A smooth manifold $M$ is a manifold where there is an atlas (series of neighbourhoods and homeomorphisms which cover $M$) whose transition maps are smooth. We say that the charts are smoothly compatible The morphisms are diffeomorphisms. 
$f: M \rightarrow N$ is a diffeomorphism if it is a bijective and smooth with smooth inverse. 

\subsubsection{Manifolds with boundary}
We say $H^n$ is the upper half plane, defined as $\left\{(x_1, \ldots, x_n) \in \mathbb{R}^n | x_n \geq 0\right\}$. An $n$-manifold with boundary is a topological space which is:

\begin{itemize}
    \item Hausdorff
    \item Second-Countable
    \item Locally Euclidean with boundary.
\end{itemize}
Locally Euclidean with boundary means that for all points $x$ in $M$, there exists an open neighbourhood $U$ of $x$ and a homeomorphism $\varphi:U \rightarrow U' \subseteq H^n$. 
The boundary $\partial M$ are points in $M$ that has neighbourhoods homeomorphic to $H^n$ but not to $\mathbb{R}^n$. 

\subsubsection{Submanifold}
$S \subseteq M$ is a smooth $k$-submanifold if each point $p$ in $S$ lies in the domain of a chart $\varphi : U \rightarrow \Int(D^n)$ and $U \cap S = \varphi^{-1} (\{x_1, x_2, \ldots,  x_k, 0, \ldots, 0\})$. 

\begin{example}
    $W \subseteq \mathbb{R}^n$ open, $f : W \rightarrow \mathbb{R}^m$ smooth. Suppose $y \in f(w)$ satisfies: $Df$ has rank $m$ at every point at $x = f^{-1}(y)$. By implicit function theorem, $f^{-1}(y)$ is a submanifold of $W$ with dimension $n - m$. Then this means that we can vary for any smooth manifold. 
\end{example}

\begin{definition}[Tubular neighbourhood]
    Let $M$ be a smooth $n$-manifold, let $S \subseteq M$ a smooth $k$-submanifold. A tubular neighbourhood is an embedding of $f : N \rightarrow M$ where $f(N)$ is an open neighbourhood in $M$ containing $S$ and for all $x \in S$, there exists a neighbourhood $U$ of $x$ in $M$ such that $U \cap f(N) \cong  (U \cap S) \times D^{n - k}$. 
\end{definition}

\subsection{Piecewise Linear}

Say we have vertices $\{v_0, \ldots , v_n\} \subseteq \mathbb{R}^n$ such that they are in general position. The $n$-simplex is the convex hull of these vertices. 

\begin{definition}
    [Simplicial complex]
    A simplicial complex $K$ satisfies the following properties: Every face of $K$ is in $K$, and if $K_1, K_2$ is in $K$, then $K_1 \cap K_2$ is in $K$. 
\end{definition}
We say $|K|$ is the geometric realisation of $K$.

\begin{definition}
    [Subcomplex]
    A subcomplex $K'$ of $K$ is a simplicial subcollection in $K$.
\end{definition}

\begin{definition}
    [m-skeleton]
    An $m$-skeleton of a simplex $K$ are all the simplices with dimension $m' \leq m$. 
\end{definition}

\begin{definition}[Star of simplex]
    The star of a particular simplex $\sigma$ in $K$, denoted $\Star(\sigma; K)$ is the simplices of $K$ which share a vertex with $\sigma$. 
\end{definition}
\begin{definition}
    [Link of simplex]
    The link of a simplex $\sigma$ in $K$ is the intersection of all simplices in $\Star(\sigma; K)$ disjoint from $\sigma$.
\end{definition} 

\begin{definition}[Star of subcomplex]
    The star of a particular subcomplex $K'$ in $K$, denoted $\Star(K'; K)$ is the simplices of $K$ which share a vertex with $K'$. 
\end{definition}
\begin{definition}
    [Link of subcomplex]
    The link of a subcomplex $K'$ in $K$ is the intersection of all simplices in $\Star(K'; K)$ disjoint from $K'$.
\end{definition} 

\begin{definition}[Star of point]
    The star of a particular point $p$ in $K$, denoted $\Star(p; K)$ is the simplices of $K$ which intersect with $p$. 
\end{definition}
\begin{definition}
    [Link of point]
    The link of a point $p$ in $K$ is the intersection of all simplices in $\Star(p; K)$ disjoint from $p$.
\end{definition} 


\subsubsection{PL homeomorphism}
$|K|$, $|L|$ are PL-homeomorphic if there exists a subdivision $K'$ of $K$, $L'$ of $L$ and there is a face-preserving bijection between $K'$ and $L'$. 

\subsubsection{PL Manifold}
$M$ is an $n$- manifold if there exists a triangulation of $M$ to $(K, t)$ where $K$ is a simplicial complex and there exists a homeomorphism $t : M \rightarrow K$, and for all $x \in M$, the link of $t(x)$ in $K$ is PL-homeomorphic to a PL $(n-1)$-sphere $|\partial \sigma^n|$ or a PL $(n-1)$-ball $|\sigma^{(n-1)}|$. 

\section{Jordan Curve Theorem}

\begin{definition}
    A simple closed curve (scc) is an embedding of $S^1$ in an $n$-manifold. 
\end{definition}

\begin{theorem}[Jordan Curve Theorem]
    Let $J$ be a scc on $\mathbb{R}^2$. Then $\mathbb{R}^2 - J$ has two components and $J$ is the boundary of each. 
\end{theorem}

We can replace $\mathbb{R}^2$ with $S^2$. 

\begin{theorem}[Schoenflies Theorem]
    Let $J$ be a scc on $\mathbb{R}^2$. Then $\mathbb{R}^2 - J$ has two components and $J$ is the boundary of each and the closure of one component is homeomorphic to the unit disk $D^2$.
\end{theorem}

\begin{corollary}
    Let $J$ be a scc on $S^2$. Then the closure of each component of $S^2 - J$ is the unit disk $D^2$. 
\end{corollary}

\begin{proof}
    Use Schoenflies theorem and the one-point compactification of the disk that contains infinity. 
\end{proof}

\begin{theorem}[Alexander's Lemma]
    Every homeomorphism of a boundary of the $n$-ball extends to a homeomorphism of the balls. 
\end{theorem}

\begin{proof}
    \begin{equation}
        h(tx) = t h(x)
    \end{equation}
\end{proof}

\begin{corollary}
    Any two knots are equivalent. 
\end{corollary}

\begin{corollary}[Shoenflies Annulus Theorem]
    The closure of the region between 2 scc $J_1$ and $J_2$ in $S^2$ is homeomorphic to an annulus. 
\end{corollary}

\begin{proof}
    Draw two arcs between $J_1$ and $J_2$. Then cut along $J_1$ and $J_2$ for two regions bounded by curves. Then this can be regulated to an annulus. 
\end{proof}

\begin{corollary}
    Any two links $S^1 \cup S^1$ in $S^2$ are equivalent. 
\end{corollary}

\begin{theorem}
    Any two knots are ambient isotopic.
\end{theorem}

\begin{theorem}[Jordan Line Theorem]
    Any curve $L$ homeomorphic to $\mathbb{R}$ in $\mathbb{R}^2$ has two components where $L$ is the boundary of each. 
\end{theorem}

\begin{theorem}
    Any curve which forms a bigon are ambient isotopic. 
\end{theorem}

\section{The Torus}

\begin{definition}
    A fundamental region in a covering map $p: \widetilde{X} \rightarrow X$ is a set $Y$ in a neighbourhood $U$ with such that $p$ restricted to $U$ is a homeomorphism.
\end{definition}

\begin{theorem}[Transversality isotopy theorem]
    Suppose $J$ and $K$ are curves on the surface. Then there is a curve $J^*$ ambient isotopic to $J$ such that $J^*$ intersects $K$ transversally finitely many times. 
\end{theorem}

\begin{theorem}[Meridians are ambient isotopic]
    Any two meridian knots are ambient isotopic.
\end{theorem}

\subsection{Classification of torus knots}

We will classify homeomorphisms of the torus.

\begin{theorem}
    The mapping class group of the torus is $\mathbb{SL}_2(\mathbb{Z})$.
\end{theorem}

\begin{theorem}
    $(a, b)$ is a scc iff $\gcd(a, b) = 1$. 
\end{theorem}

\begin{theorem}
    For any non null-homotopic knot $K$, there exists a homeomorphism twist $h$ such that $h(T) = \mu$ meridian.
\end{theorem}

\begin{proof}
    Proof by application of Dehn twists around $(0, 1)$ and $(1, 0)$. 
\end{proof}

\begin{theorem}
    There are two knot types up to homeomorphism.
\end{theorem}
\begin{proof}
    The two knots are $(0,0)$ and $(1, 0)$. 
\end{proof}

\begin{theorem}
    Two knots are ambient isotopic iff the homotopy classes are the same. 
\end{theorem}

\section{Mapping Class Group}

Homeomorphisms from a space $X \rightarrow X$ form a group $Homeo(X)$.
Two homeomorphisms $h, g$ are isotopic if there exists $F: X \times I \rightarrow X$ such that for each $t \in I$, $F_t$ is a homeomorphism in $X$ and $F_0(x) = h(x)$, $F_1(x) = g(x)$. 

The space $Homeo(X)/N$ where $N$ are isotopic maps is the mapping class group. 

\begin{theorem}
    If $f: D^n \rightarrow D^n$ is a homeomorphism fixing the boundary, then $f$ is isotopic to the identity map on $D^n$. 
\end{theorem}

The mapping class group corresponding to orientation-preserving homeomorphisms is $MCG^+$. The mapping class groups of disks fixing surfaces is $1$. 


\subsection{Dehn twists}

A Dehn twist around a curve $\alpha$ takes a twist of the tubular annulus around $\alpha$ rightwards. 

\begin{theorem}
    Dehn twists generate the mapping class group.
\end{theorem}

\begin{theorem}[Lickorish-Twist theorem]
    If $S$ is a closed, oriented surface, then $MCG^+(S)$ is generated by Dehn twists. If $f : S \rightarrow S$ is a homeomorphism, then $f$ is isotopic to  $T$, where $T$ is a series of Dehn twists. 
\end{theorem}

This is a proof by induction.

\begin{theorem}
    All genus $0$ surfaces are generated by Dehn twists.
\end{theorem}
The proof is by induction on the number of boundary components.

\begin{proof}
    For the base case $S = S^2$, or $S = D^2$. The number of boundary components is trivial. For the case of $k$ holes, consider taking an arc between any two holes. Then a cut along this arc is a surface of genus $g - 1$ holes. Any homeomorphism which preserves this arc is of a surface with $k-1$ holes and by the induction hypothesis is generated by a Dehn twist. Furthermore, the Dehn twist around a hole generates any local homeomorphisms near the boundary, therefore all genus 0 surfaces holds. 
\end{proof}
\end{document}
