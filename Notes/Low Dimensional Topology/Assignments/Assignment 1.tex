\documentclass{article}
\usepackage[margin=1in]{geometry}
\usepackage{amsmath}
\usepackage{amssymb}
\usepackage{amsthm}
\usepackage{url}
\usepackage{todonotes}
\usepackage{svg}
\usepackage{cleveref}

% Environments

\newtheorem{theorem}{Theorem}
\newtheorem{proposition}[theorem]{Proposition}
\newtheorem{corollary}[theorem]{Corollary}
\newtheorem{lemma}[theorem]{Lemma}
\newtheorem{definition}[theorem]{Definition}
\newtheorem{conjecture}[theorem]{Conjecture}
\newtheorem{remark}[theorem]{Remark}


\theoremstyle{definition}
\newtheorem{example}[theorem]{Example}

\numberwithin{theorem}{section}
\numberwithin{equation}{section}

\DeclareMathOperator{\Int}{int}
\DeclareMathOperator{\Star}{st}
\DeclareMathOperator{\Lk}{lk}
\newcommand{\atlas}{\mathcal{A}}

%opening
\title{Low Dimensional Topology}
\author{Eric Luu}

\begin{document}

\section{Sheet 1}

\subsection{Question 1}
Look at \cref{fig:W1Q1} for a list of all link types. To show that none of these links are homeomorphic to each other, we delete the green $S^0$ and have a look at the intervals that remain and the location of the vertices. In the case of $S^0 \sqcup S^0 \sqcup S^0 $ in $S^1$, we delete the blue vertices as well to have some intervals of link types in $S^1$. 
\begin{figure}\label{fig:W1Q1}
    \includesvg[width = 0.8 \textwidth]{Figures/W1Q1 Link Types.svg}
    \caption{Link types of $S^0 \sqcup S^0$ in $S^1$ and $S^0 \sqcup S^0 \sqcup S^0 $ in $S^1$}
\end{figure}

We have that there are 2 link types of $S^0 \sqcup S^0$ in $S^1$ and 11 link types of $S^0 \sqcup S^0 \sqcup S^0$ in $S^1$.  

\subsection{Question 2}
\subsubsection{a}
I'm not doing this in inkscape. One time is enough. We have that there are 2 link types of $S^0 \sqcup S^1$ in $S^2$ as we have 1 equator and 2 dots which we can place on the same hemisphere or different hemispheres. 

\subsubsection{b}
There are 5 different link types of $S^0 \sqcup S^0 \sqcup S^1$ in $S^2$. Label the two $S^0$ as red and blue, like above. Use $S^1$ to split $S^2$ into two hemispheres. We have that there is:
\begin{itemize}
    \item All 4 dots are on one hemisphere
    \item 1 red dot is on 1 hemisphere, 3 dots is on the other
    \item 1 blue dot is on 1 hemisphere, 3 dots is on the other
    \item Two reds on 1 hemisphere, 2 blues on other
    \item 1 blue and 1 red on each hemisphere.
\end{itemize}

\subsection{5}
We have that $S^n$ is an $n$-manifold with smooth structure. Take the open sets $U = S^n - (1, 0, \ldots, 0)$ and $V = S^n - (-1, 0, \ldots, 0)$ with maps $\varphi : U \rightarrow \mathbb{R^n}$ where we take $(x_0, x_1, \ldots, x_n) \mapsto (\frac{x_1}{1- x_0}, \frac{x_2}{1- x_0}, \ldots, \frac{x_n}{1- x_0} )$, and $\psi : V \rightarrow \mathbb{R^n}$ where $(x_0, x_1, \ldots, x_n) \mapsto (\frac{x_1}{1+x_0}, \frac{x_2}{1+ x_0}, \ldots, \frac{x_n}{1+x_0} )$. 

The inverse of $\varphi$, going from $\mathbb{R}^n \rightarrow S^n\setminus \{(1, 0, \ldots, 0)\}$ is given by this operation. Take $(x_1, \ldots, x_n)$. Let $s = \sum_{i = 1}^n x_i^2$. Then $(x_1, \ldots, x_n) \mapsto (\frac{s - 1}{s + 1}, \frac{2x_1}{1 + s}, \ldots, \frac{2x_n}{1 + s})$. 

The inverse of $\psi$ is: Take $(x_1, \ldots, x_n)$. Let $s = \sum_{i = 1}^n x_i^2$. Then $(x_1, \ldots, x_n) \mapsto (\frac{1 - s}{1 + s}, \frac{2x_1}{1 + s}, \ldots, \frac{2x_n}{1 + s})$. 

Then we have that $\psi \circ \varphi^{-1}$ and $\varphi^{-1} \circ \psi$ sends $x$ to $ \frac{x}{|x|}$ which is a smooth function in all coordinates. Thus we have that the sphere is a smooth map. 
\subsection{9}
We have that every compact $1$-manifold without boundary is homeomorphic to a finite disjoint union of $S^1$, as the only $1$-manifolds that exist are either homeomorphic to $S^1$ or $\mathbb{R}$, and $\mathbb{R}$ is not compact. We have that the 2-simplex $[v_1, v_1]$ is homeomorphic to $S^1$ directly, so $S^1$ is a $PL$-manifold. As this property holds for every connected component of a compact $1$-manifold, every compact $1$-manifold is also a $PL$-manifold. 
\end{document}
