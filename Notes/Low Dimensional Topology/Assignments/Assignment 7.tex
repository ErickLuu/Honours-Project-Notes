\documentclass{article}
\usepackage[margin=1in]{geometry}
\usepackage{amsmath}
\usepackage{amssymb}
\usepackage{amsthm}
\usepackage{url}
\usepackage{todonotes}
\usepackage{svg}
\usepackage{cleveref}

% Environments

\newtheorem{theorem}{Theorem}
\newtheorem{proposition}[theorem]{Proposition}
\newtheorem{corollary}[theorem]{Corollary}
\newtheorem{lemma}[theorem]{Lemma}
\newtheorem{definition}[theorem]{Definition}
\newtheorem{conjecture}[theorem]{Conjecture}
\newtheorem{remark}[theorem]{Remark}


\theoremstyle{definition}
\newtheorem{example}[theorem]{Example}

\numberwithin{theorem}{section}
\numberwithin{equation}{section}

\DeclareMathOperator{\Int}{int}
\DeclareMathOperator{\Star}{st}
\DeclareMathOperator{\Lk}{lk}
\newcommand{\atlas}{\mathcal{A}}
\newcommand{\cover}{\widetilde{X}}
\DeclareMathOperator{\Homeo}{Homeo}
\newcommand*{\inter}{\hat{i}}
\newcommand*{\twequiv}{\sim_c}


%opening
\title{Assignment 7}
\author{Eric Luu}

\begin{document}
\section{6}
\subsection{10}

Cut the sphere into sections with radius $\frac{2\pi}{p}$. Start at the point $(0, 1, 0)$ and label the sections going counterclockwise $1 ... p$. Then section $1$ on the top is glued to $q + 1$ on the bottom, in this construction. Now cut out the cylinder $x^2 + y^2 \leq \frac{1}{4}$. This cylinder is a torus $V_1$ but with a twist of $\frac{p}{q}$ around the centre.

The shape left by removing this cylinder from the sphere is also a torus $V_2$. $V_2$ can be viewed as rotating each section around and gluing together to form a big wheel. However, there is still a longitudinal loop that is noncontractible in this big wheel by traversing through every shape. Since this space was connected before the quotient map, it is still connected after. 

Now let the section $x^2 + y^2 = \frac{1}{4}, z^2 = 3/4$ be the meridian of $V_1$. Then this curve runs along the edge of $V_2$ as well. Now the number of twists that it does is $p$ longitudinal curves and $q$ meridinal curves, as the top and the bottom are twisted around by $q/p$ full turns. Therefore, the meridinal curve is a $p, q$ curve on $V_2$. Since by definition, $p$ and $q$ are coprime, then $p$ and $q$ are the only unique natural numbers that fulfil this definition. 

We assume that $p > 0$ always. If this is not the case, then the manifold is $S^2 \times S^1$. 
\section{7}

\subsection{2}
The Heegaard diagram below is $S^3$ on a genus $g$ handlebody. The Heegaard diagram maps a meridian from handlebody $H_1$ to each longitude in $H_2$. If we glue a neighbourhood of each meridian to the longitude, then $H_1$ cut along every meridian yields $B^2$, and gluing neighbourhoods of meridians to $H_2$ in the usual way fills out each hole, yielding $B^2$, the surface is $B^2 \cup_f B^2$ for some orientable homeomorphism $f$. But as every gluing of $B^2$ to itself is homeomorphic to $S^3$, this is $S^3$ for all surfaces. 
\begin{figure}[h]
    \includesvg[width = 0.8 \textwidth]{Figures/W7Q2heegarddiagram.svg}
\end{figure}

\subsection{4}
The contrapositive of question $3$ is that if the fundamental group of $M$ has at $g$ generators, then $M$ has genus at least $g$. 

The fundamental group of $T^3$ has no presentation with two generators, as the fundamental group of $T^3$ is $\mathbb{Z} \times \mathbb{Z} \times \mathbb{Z}$. Since $(1, 0, 0)$, $(0, 1, 0)$, and $(0,0, 1)$ are linearly independent in this group, then every representation must have at least three generators. By question 3, the genus must be at least 3. 

\subsection{5}
Let $M_g = H_1 \cup_{Id} H_2$ where $H_1$, $H_2$ is the handlebody of genus $g$. $\pi_1(M_g) = \pi_1(H_1) * \pi_1(H_2)/N$. But $\pi_1(H_1) = \langle x_1, x_2, ..., x_g \rangle$ and $\pi_1(H_2) = \langle y_1, y_2, ..., y_g \rangle$. But the only relators are $x_i = y_i$. Therefore, $\pi_1(M_g) = F_g$, the free group on $g$ generators. However, $F_g$ has no presentation with $g-1$ generators. From section 3, $M_g$ has no Heegaard splitting of smaller genus. 

\end{document}
