\documentclass{article}
\usepackage[margin=1in]{geometry}
\usepackage{amsmath}
\usepackage{amssymb}
\usepackage{amsthm}
\usepackage{url}
\usepackage{todonotes}
\usepackage{svg}
\usepackage{cleveref}

% Environments

\newtheorem{theorem}{Theorem}
\newtheorem{proposition}[theorem]{Proposition}
\newtheorem{corollary}[theorem]{Corollary}
\newtheorem{lemma}[theorem]{Lemma}
\newtheorem{definition}[theorem]{Definition}
\newtheorem{conjecture}[theorem]{Conjecture}
\newtheorem{remark}[theorem]{Remark}


\theoremstyle{definition}
\newtheorem{example}[theorem]{Example}

\numberwithin{theorem}{section}
\numberwithin{equation}{section}

\DeclareMathOperator{\Int}{int}
\DeclareMathOperator{\Star}{st}
\DeclareMathOperator{\Lk}{lk}
\DeclareMathOperator{\Homeo}{Homeo}
\newcommand{\atlas}{\mathcal{A}}
\newcommand{\cover}{\widetilde{X}}

%opening
\title{Low Dimensional Topology}
\author{Eric Luu}

\begin{document}

\section{Sheet 1}


\subsection{Question 3}

\subsubsection{a}
Using the 1 point compatification, we have that the solid torus is produced of $S^3 - U$. Embed $U$ in $S^3$ on the $z$-axis, which passes through $\infty$. Then we have that $S^3 - U$ is simply $\mathbb{R}^3$ without the $z$-axis, which is homeomorphic to the solid torus. 
\subsubsection{b}
From above, we have that $\pi_1(S^3 - U) = \mathbb{Z}$ as this is the fundamental group of the solid torus. It can also be seen by the only nontrivial loop (up to homotopy) being ones which wrap around the hole left behind by $U$. 
\subsubsection{c}
Label one of the loops $U$ and the other loop $L$. We have that on the left the homotopy equivalence class of $L$ in $S^1 - U$ is nontrivial but on the right the homotopy class of $L$ in $S^1 - U$ is trivial. Therefore, we have that the left and the right cannot be equivalent as homeomorphisms sends homotopy equivalence classes to each other in the natural way. 

\subsection{6}
We have that $M(n)$ is a smooth manifold. Now consider the function $\det : M(n) \rightarrow \mathbb{R}$, which is a smooth function between two manifolds as it is a polynomial. Now look at $\det^{-1} (1)$, which is precisely the matrices of determinant $1$. Now we want to show that the Jacobian on this set is not the zero vector, so that the preimage is a submanifold. 

Then as we have that $(D \det A)_{ij} = (-1)^{i + j} \det(A^*_{i, j})$ where $A^*_{i, j}$ is the cofactor matrix of $A$ deleting $i$ and $j$. Then this means that this is zero iff the determinant of the cofactor matrix is zero, but this is never the case as the determinant of $A$ itself is nonzero, meaning $A$ is invertible and so are all of its cofactors. Therefore, $D \det(A)$ is 0 iff $\det A = 0$, meaning that the preimage is a submanifold. 

Finally, we have that the dimension is $n^2-1$ and the codimension is 1. 

\section{Sheet 2}

\subsection{2}
Show that $A \cup B$ is homeomorphic to $S^1$. Then use Schoenflies to show that $A \cup B$ has an interior and an exterior. Then use a homeomorphism taking $A \cup B$ to the disc, with one point at (1, 0) and another point at (-1, 0). The ambient isotopy should be easy to find at this point. 

\subsection{3}
Suppose Jordan Line Theorem is true. Let $L$ be a simple closed curve in $S^2$. Then take $N$ to be on $L$ and stereograph project to $L'$ in $\mathbb{R}^2$. Then $L'$ divides up plane to two halves by Jordan Line Theorem. Then we go backwards and project both the two components and $L$ back to $S^2$. 

Suppose Jordan Curve Theorem is true. Take line $L$ in $\mathbb{R}^2$. Then stereograph project $L$ to $S^2$. Will divide into two sections. Then stereograph project $L'$ and 2 components to $\mathbb{R}^2$.  

\subsection{4}
Consider thickening the plane in the $z$ axis by dragging it to $[-1,1]$ and letting the two open sets be $z > -1/2$ and $z < 1/2$ with intersection homeomorphic to the plane. 
\subsection{5}
Take any arc $L$ in $\mathbb{R}^3$ and a plane $P$. There is an isotopy of $P$ to the $xy$-plane. Then there exists an ambient isotopy from the arc to the $x$-axis. 

\subsection{6}

Take any arc and have an ambient isotopy from the plane to the $xy$-plane. Then there is an ambient isotopy to $S^1$. 


\section{3}

\subsection{1}
\subsubsection*{a}
Take a point $(e^{2\pi i \theta}, e^{2 \pi i \varphi})$ on $T^2 \cong S^1 \times S^1$ where $S^1 \subset \mathbb{C}$. The preimage of this set in $\mathbb{C}$ is $(m + \theta) + i (n + \varphi)$ where $m, n \in \mathbb{Z}$. Consider the open neighbourhood $U = (e^{2 \pi i (\theta - 1/100)}, e^{2 \pi i (\theta + 1/100)}) \times (e^{2 \pi i (\varphi - 1/100)}, e^{2 \pi i (\varphi + 1/100)})$. Then $p^{-1}(U)$ are the points $x + i y$ where $x \in (\theta + m -1/100, \theta + m + 1/100)$, $y \in (\varphi + n - 1/100, \varphi + n + 1/1000)$, and $m, n \in \mathbb{Z}$. Then the open sheets are of the form $U_{m,n} = \{x + i y : m + \theta - 1/100 \leq x \leq m + \theta + 1/100, n + \varphi - 1/100 \leq y \leq n + \varphi + 1/100\}$ for some $m, n \in \mathbb{Z}$. These sheets are disconnected for each $m, n$. Finally, $p|_{U_{m,n}}$ is the map from $(\theta + m -1/100, \theta + m + 1/100) \times (\varphi + n - 1/100, \varphi + n + 1/100)$ to $(e^{2 \pi i (\theta - 1/100)}, e^{2 \pi i (\theta + 1/100)}) \times (e^{2 \pi i (\varphi - 1/100)}, e^{2 \pi i (\varphi + 1/100)})$. This is a bijection and a homeomorphism, as $n \mapsto e^{2 i \pi n}$ is a homeomorphism on the range $[\theta - 1/100, \theta + 1/100]$ for any $\theta$. 

\subsubsection*{b}
$\exp$ is a continuous map as it is entire. Let $R e^{i \theta}$ be a point in $\mathbb{C} - 0$. The preimage of this point are the points $(\ln(R), \theta + 2 \pi z)$ where $z \in Z$.

\subsection{2}

\begin{theorem}
    Let $X$ be a topological space and let $p : \cover \rightarrow X$ be a covering space. Let $F: I \times I \rightarrow X$ be a homotopy of paths where $F(\cdot, 0) = \gamma_0$ and $F(\cdot, 1) = \gamma_1$ rel endpoints. Suppose $F(0,t) = x_0$ and $F(1, t)= x_1$. $\tilde{x}_0 \in p^{-1}(x_0)$ there exists a unique lifted homotopy $f_t: I \rightarrow \cover$ of paths starting at $x_0$.
\end{theorem}

\begin{proof}
    For each $F(s, t) \in X$, there exists open sheets $U_{F(s, t)}$ that contains $F(s, t)$ and is evenly covered. Because $F(s, t)$ is continuous, then $F^{-1}(U_{F(s, t)})$ are open in $I \times I$. By the compactness of $I \times I$, there exists a finite number of intervals $0 = s_0 < s_1 < \cdots < s_m = 1$ and $0 = t_0 < t_1 \ldots < t_n = 1$ such that intervals $[s_i, s_{i + 1}] \times [t_j, t_{j + 1}]$ is contained inside some $U_{F(s,t)}$.

    From the path lifting theorem, there exists a unique path $\widetilde{\gamma_0} : I \rightarrow \tilde{X}$ where $\widetilde{\gamma_0}(0) = \tilde{x}_0$. Now induct over $t_i$. Let $\gamma_i = F(\cdot, t_i)$. Suppose for induction, $\widetilde{\gamma_i}$ is the unique path of $\gamma_i$. Let $\widetilde{\gamma_i}(0) = \tilde{x}_0$. Then there exists a homotopy from $\widetilde{\gamma_i}$ to $\widetilde{\gamma_{i + 1}}$. Consider $[0, s_1] \times [t_i, t_{i + 1}]$. Then there exists a neighbourhood $U_{s, t}$ which contains $ \tilde{x}_0$ where $[0, s_1] \times [t_i, t_{i + 1}]$. Now define the path \todo{finish later!}
\end{proof}

\section{4}
\subsection{2}
Let $K$ and $K'$ be two null-homotopic knots on $T^2$. Then there exists an ambient isotopy of $T^2$ taking $K$ to $K'$. 

Let $h : \mathbb{C} \rightarrow T^2$ be its cover.
Since $K$ and $K'$ are null-homotopic, they lift to loops $\widetilde{K}, \widetilde{K'}$  in $\mathbb{C}$ from Hatcher 1.31. Then there exists an ambient isotopy $\gamma$ in $\mathbb{C} \cong \mathbb{R}^2$ from $\widetilde{K}$ to $\widetilde{K'}$ by Corr 2.6. Therefore, $h \circ \gamma \circ h^{-1}$ is an ambient isotopy from $K$ to $K'$. 

Note that $T^2$ can be replaced by any surface $S$ which have a universal cover of $\mathbb{C}$ or $\mathbb{R}$. 

\subsection{3}
Show that the homeomorphisms isotopic to the identity is a normal subgroup and that two maps are isotopic iff they are in the same coset mod identity homeomorphisms.
Firstly, we will show that $N$ is a subgroup of $\Homeo(S)$. Firstly, if $g \in N$, then $g^{-1} \in N$. Let $H(x, t) : S \times I \rightarrow S$ be an isotopy where $h_0 = Id$ and $h_1 = g$. Then $G(x, t) = h_1^{-1}(H(x, 1-t))$ is a continuous map by composition. Furthermore, $G(x, 0) = h_1^{-1}(H(x, 1)) = Id$ and $G(x, 1) = h_1^{-1}(H(x, 0)) = g^{-1}$. Therefore, $g^{-1}$ is isotopic to $N$. If $f$ and $g$ are isotopic to $N$ with isotopies $F$ and $G$ with $F_1 = f$, $G_1 = g$, $F_0 = G_0 = Id$, then $f \circ g$ is isotopic to $Id$ with isotopy:
\begin{equation*}
    H(x, t) =
    \begin{cases}
        F(x, 2t) & 0 \leq t \leq 1/2\\
        G(F(x, 1), 2t-1) & 1/2 \leq t \leq 1
    \end{cases}
\end{equation*}
This is continuous as this is a composition of continuous functions and at time $t = 1/2$, the two maps agree. At time $t = 0$, $H_0 = Id$ and at $t = 1$, $H_t = g \circ f$. Therefore, if $f, g \in N$, then $g \circ f \in N$. 

Now we will show that $N$ is normal in $\Homeo(S)$. 
Let $f \in \Homeo(S)$, $g \in N$. Then $f g f^{-1} \in N$. Let $G: S \times I \rightarrow S$ be an isotopy with $G_0 = Id$, $G_1 = g$. Then $H(x, t) = f \circ G(\cdot, t) \circ f^{-1} (x) $ is an isotopy. Firstly, $H(x, t)$ is continuous as it is a composition of continuous functions for fixed $x$ and fixed $t$. Then $H(x, 0) = f \circ g_0 \circ f^{-1} = f \circ Id \circ f^{-1} = f \circ f^{-1} = Id$. Then $H(x, 1) = f \circ g_1 \circ f^{-1} = f \circ g \circ f^{-1}$. Therefore, $f \circ g \circ f^{-1} \in N$. Therefore, $N$ is normal.

Finally, suppose $f$ is isotopic to $g$. Then $f^{-1} g \in N$. Take an isotopy $F : S \times I \rightarrow S$ where $F_0 = f$ and $F_1 = g$. Then take $H(x, t) = f^{-1}(F(x, t))$. This is continuous, as this is the composition of two continuous functions. Then $H_0 = f^{-1} \circ f = Id$ and $H_1 = f^{-1} \circ g$. Therefore, $f$ and $g$ are in the same coset. 

Suppose $f^{-1} g \in N$. Then there exists a isotopy $F : S \times I \rightarrow I$ where $F_0 = Id$ and $F_1 = f^{-1} g$. Then take $H = f \circ F$. This is continuous as it is the composition of two continuous functions. Furthermore, $F_0 = f \circ Id = f$ and $F_1 = f \circ f^{-1} \circ g = g$. Therefore, $f$ and $g$ are isotopic.  

\subsection{4}
Suppose $\begin{bmatrix}
    a & b\\
    c & d\\
\end{bmatrix} \in GL(2, \mathbb{Z})$. Therefore, $ad - bc = \pm 1$.

\subsection{5}
Call all homeomorphisms isotopic to $id$ null-isotopies. 

Between any two points $p_0, p_1 \in S^2$, there is a homeomorphism isotopic to the identity such that $f(p_0) = p_1$. This is done by rotating the sphere around. 

Take any $f$ which is orientation preserving. Then pick a point $p_0$. Then there exists a null-isotopy $\gamma : S^2 \rightarrow S^2$ such that $\gamma(f(p_0)) = p_0$, through a rotation of the sphere. Then $ \gamma \circ f$ is an isotopy that takes $p_0$ to $p_0$ (This step could be simplified with a Brower-type proof). However, $S^2 \cong D^2/{\partial D^2}$. Consider the quotient map $q : D^2 \rightarrow S^2$, where $p_0$ is the point on $\partial D^2$. Let $g : D^2 \rightarrow D^2$ such that $g \circ q = \gamma \circ f$. This map exists as $\gamma \circ f$ keeps $p_0$ fixed and changes $S^2 - \{p_0\}$, so $g = \gamma \circ f$ on the open disk under the map. Then $g|_{\partial D^2} = Id$. Then by Alexander's isotopy lemma, $g$ is an isotopy. Therefore, $\gamma \circ f$ is also an isotopy, therefore $f$ is an isotopy. Thus shown. 

Orientation-reversing homeomorphims cannot be transformed to orienting preserving homeomorphisms on $S^2$. 

\end{document}
