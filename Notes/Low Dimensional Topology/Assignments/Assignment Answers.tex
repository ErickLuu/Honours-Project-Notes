\documentclass{article}
\usepackage[margin=1in]{geometry}
\usepackage{amsmath}
\usepackage{amssymb}
\usepackage{amsthm}
\usepackage{url}
\usepackage{todonotes}
\usepackage{svg}
\usepackage{cleveref}

% Environments

\newtheorem{theorem}{Theorem}
\newtheorem{proposition}[theorem]{Proposition}
\newtheorem{corollary}[theorem]{Corollary}
\newtheorem{lemma}[theorem]{Lemma}
\newtheorem{definition}[theorem]{Definition}
\newtheorem{conjecture}[theorem]{Conjecture}
\newtheorem{remark}[theorem]{Remark}


\theoremstyle{definition}
\newtheorem{example}[theorem]{Example}

\numberwithin{theorem}{section}
\numberwithin{equation}{section}

\DeclareMathOperator{\Int}{int}
\DeclareMathOperator{\Star}{st}
\DeclareMathOperator{\Lk}{lk}
\newcommand{\atlas}{\mathcal{A}}
\newcommand{\cover}{\widetilde{X}}

%opening
\title{Low Dimensional Topology}
\author{Eric Luu}

\begin{document}

\section{Sheet 1}


\subsection{Question 3}

\subsubsection{a}
Using the 1 point compatification, we have that the solid torus is produced of $S^3 - U$. Embed $U$ in $S^3$ on the $z$-axis, which passes through $\infty$. Then we have that $S^3 - U$ is simply $\mathbb{R}^3$ without the $z$-axis, which is homeomorphic to the solid torus. 
\subsubsection{b}
From above, we have that $\pi_1(S^3 - U) = \mathbb{Z}$ as this is the fundamental group of the solid torus. It can also be seen by the only nontrivial loop (up to homotopy) being ones which wrap around the hole left behind by $U$. 
\subsubsection{c}
Label one of the loops $U$ and the other loop $L$. We have that on the left the homotopy equivalence class of $L$ in $S^1 - U$ is nontrivial but on the right the homotopy class of $L$ in $S^1 - U$ is trivial. Therefore, we have that the left and the right cannot be equivalent as homeomorphisms sends homotopy equivalence classes to each other in the natural way. 

\subsection{6}
We have that $M(n)$ is a smooth manifold. Now consider the function $\det : M(n) \rightarrow \mathbb{R}$, which is a smooth function between two manifolds as it is a polynomial. Now look at $\det^{-1} (1)$, which is precisely the matrices of determinant $1$. Now we want to show that the Jacobian on this set is not the zero vector, so that the preimage is a submanifold. 

Then as we have that $(D \det A)_{ij} = (-1)^{i + j} \det(A^*_{i, j})$ where $A^*_{i, j}$ is the cofactor matrix of $A$ deleting $i$ and $j$. Then this means that this is zero iff the determinant of the cofactor matrix is zero, but this is never the case as the determinant of $A$ itself is nonzero, meaning $A$ is invertible and so are all of its cofactors. Therefore, $D \det(A)$ is 0 iff $\det A = 0$, meaning that the preimage is a submanifold. 

Finally, we have that the dimension is $n^2-1$ and the codimension is 1. 

\section{Sheet 2}

\subsection{2}
Show that $A \cup B$ is homeomorphic to $S^1$. Then use Schoenflies to show that $A \cup B$ has an interior and an exterior. Then use a homeomorphism taking $A \cup B$ to the disc, with one point at (1, 0) and another point at (-1, 0). The ambient isotopy should be easy to find at this point. 

\subsection{3}
Suppose Jordan Line Theorem is true. Let $L$ be a simple closed curve in $S^2$. Then take $N$ to be on $L$ and stereograph project to $L'$ in $\mathbb{R}^2$. Then $L'$ divides up plane to two halves by Jordan Line Theorem. Then we go backwards and project both the two components and $L$ back to $S^2$. 

Suppose Jordan Curve Theorem is true. Take line $L$ in $\mathbb{R}^2$. Then stereograph project $L$ to $S^2$. Will divide into two sections. Then stereograph project $L'$ and 2 components to $\mathbb{R}^2$.  

\subsection{4}
Consider thickening the plane in the $z$ axis by dragging it to $[-1,1]$ and letting the two open sets be $z > -1/2$ and $z < 1/2$ with intersection homeomorphic to the plane. 
\subsection{5}
Take any arc $L$ in $\mathbb{R}^3$ and a plane $P$. There is an isotopy of $P$ to the $xy$-plane. Then there exists an ambient isotopy from the arc to the $x$-axis. 

\subsection{6}

Take any arc and have an ambient isotopy from the plane to the $xy$-plane. Then there is an ambient isotopy to $S^1$. 


\section{3}

\subsection{2}

\begin{theorem}
    Let $X$ be a topological space and let $p : \cover \rightarrow X$ be a covering space. Let $F: I \times I \rightarrow X$ be a homotopy of paths where $F(\cdot, 0) = \gamma_0$ and $F(\cdot, 1) = \gamma_1$ rel endpoints. Suppose $F(0,t) = x_0$ and $F(1, t)= x_1$. $\tilde{x}_0 \in p^{-1}(x_0)$ there exists a unique lifted homotopy $f_t: I \rightarrow \cover$ of paths starting at $x_0$.
\end{theorem}

\begin{proof}
    For each $F(s, t) \in X$, there exists open sheets $U_{F(s, t)}$ that contains $F(s, t)$ and is evenly covered. Because $F(s, t)$ is continuous, then $F^{-1}(U_{F(s, t)})$ are open in $I \times I$. By the compactness of $I \times I$, there exists a finite number of intervals $0 = s_0 < s_1 < \cdots < s_m = 1$ and $0 = t_0 < t_1 \ldots < t_n = 1$ such that intervals $[s_i, s_{i + 1}] \times [t_j, t_{j + 1}]$ is contained inside some $U_{F(s,t)}$.

    From the path lifting theorem, there exists a unique path $\widetilde{\gamma_0} : I \rightarrow \tilde{X}$ where $\widetilde{\gamma_0}(0) = \tilde{x}_0$. Now induct over $t_i$. Let $\gamma_i = F(\cdot, t_i)$. Suppose for induction, $\widetilde{\gamma_i}$ is the unique path of $\gamma_i$. Let $\widetilde{\gamma_i}(0) = \tilde{x}_0$. Then there exists a homotopy from $\widetilde{\gamma_i}$ to $\widetilde{\gamma_{i + 1}}$. Consider $[0, s_1] \times [t_i, t_{i + 1}]$. Then there exists a neighbourhood $U_{s, t}$ which contains $ \tilde{x}_0$ where $[0, s_1] \times [t_i, t_{i + 1}]$. Now define the path \todo{finish later!}
\end{proof}

\subsection{3}
We will show that $\rho_1 \times \rho_2: \cover_1 \times \cover_2 \rightarrow X_1 \times X_2$ is a covering space. 
Take a point $(x_1, x_2) \in X_1 \times X_2$. Then let $U$ be an evenly covered neighbourhood in $X_1$ such that $x_1 \in U$ and $p_1^{-1}(U) = \{U_\alpha \}_{\alpha \in A}$ be its sheets, where $p_1|_{U_\alpha}: \cover_1 \rightarrow U$ is a homeomorphism. Similarly, define an evenly covered neighbourhood $V \subseteq X_2$ such that $x_2 \in V$ and $p_2^{-1}(V) = \{V_\beta \}_{\beta \in B}$ be its sheets. 

We claim $U \times V$ is an evenly covered neighbourhood of $(x_1, x_2) \in X_1 \times X_2$. Firstly, $U \times V$ is connected by definition of the product of connected sets. Then the preimage, $(p_1 \times p_2)^{-1}(U \times V) = p_1^{-1}(U) \times p_2^{-1}(V)$ are the sets $\{U_\alpha \}_{\alpha \in A} \times \{V_\beta \}_{\beta \in B} = \{U_\alpha \times V_\beta : \alpha \times \beta \in A \times B\}$. These sets are open from the product topology. $U_i \times V_j$ is disconnected from $U_k \times V_\ell$. If $U_i \neq U_k$, then $U_i$ is disconnected from $U_k$. Therefore, the products are also disconnected. Repeat for $V_j, V_\ell$. Therefore, $\{U_\alpha \}_{\alpha \in A} \times \{V_\beta \}_{\beta \in B} $ is open and disconnected.

Finally, $\rho_1 \times \rho_2 |_{U_\alpha \times V_\beta}: \cover_1 \times \cover_2 \rightarrow U \times V$ is a homeomorphism. It suffices to show this fact. If $f_1 : X_1 \rightarrow Y_1$ is a homeomorphism, $f_2 : X_2 \rightarrow Y_2$ is a homeomorphism, then $f_1 \times f_2 : X_1 \times X_2 \rightarrow Y_1 \times Y_2$ is a homeomorphism. This map is a bijection as $f_1, f_2$ is a bijection. This map is continuous by the product topology. Similarly,$(f_1 \times f_2 )^{-1} = f_1^{-1} \times f_2^{-1}$ is also continuous by the product topology. Therefore, $f_1 \times f_2 : X_1 \times X_2 \rightarrow Y_1 \times Y_2$ is a homeomorphism. Thus $\rho_1 \times \rho_2 |_{U_\alpha \times V_\beta}: \cover_1 \times \cover_2 \rightarrow U \times V$ is a homeomorphism. 
\end{document}
