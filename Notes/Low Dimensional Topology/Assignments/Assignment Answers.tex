\documentclass{article}
\usepackage[margin=1in]{geometry}
\usepackage{amsmath}
\usepackage{amssymb}
\usepackage{amsthm}
\usepackage{url}
\usepackage{todonotes}
\usepackage{svg}
\usepackage{cleveref}

% Environments

\newtheorem{theorem}{Theorem}
\newtheorem{proposition}[theorem]{Proposition}
\newtheorem{corollary}[theorem]{Corollary}
\newtheorem{lemma}[theorem]{Lemma}
\newtheorem{definition}[theorem]{Definition}
\newtheorem{conjecture}[theorem]{Conjecture}
\newtheorem{remark}[theorem]{Remark}


\theoremstyle{definition}
\newtheorem{example}[theorem]{Example}

\numberwithin{theorem}{section}
\numberwithin{equation}{section}

\DeclareMathOperator{\Int}{int}
\DeclareMathOperator{\Star}{st}
\DeclareMathOperator{\Lk}{lk}
\newcommand{\atlas}{\mathcal{A}}

%opening
\title{Low Dimensional Topology}
\author{Eric Luu}

\begin{document}

\section{Sheet 1}


\subsection{Question 3}

\subsubsection{a}
Using the 1 point compatification, we have that the solid torus is produced of $S^3 - U$. Embed $U$ in $S^3$ on the $z$-axis, which passes through $\infty$. Then we have that $S^3 - U$ is simply $\mathbb{R}^3$ without the $z$-axis, which is homeomorphic to the solid torus. 
\subsubsection{b}
From above, we have that $\pi_1(S^3 - U) = \mathbb{Z}$ as this is the fundamental group of the solid torus. It can also be seen by the only nontrivial loop (up to homotopy) being ones which wrap around the hole left behind by $U$. 
\subsubsection{c}
Label one of the loops $U$ and the other loop $L$. We have that on the left the homotopy equivalence class of $L$ in $S^1 - U$ is nontrivial but on the right the homotopy class of $L$ in $S^1 - U$ is trivial. Therefore, we have that the left and the right cannot be equivalent as homeomorphisms sends homotopy equivalence classes to each other in the natural way. 
\subsection{Question 4}
\subsubsection{a}
Let $\atlas = \left\{(U_\alpha, \varphi_\alpha) | \alpha \in J\right\}$. Let $(U_\alpha, \varphi_\alpha), (U_\beta, \varphi_\beta)$ be two charts in $\overline{\atlas}$. We want to show that they are smoothly compatible. If either one of them is in $\atlas$, then we have that they are smoothly compatible by definition. Therefore, the interesting case is when neither are in $\atlas$. 

Consider the intersection $U_\alpha \cap U_\beta$. As $\atlas$ is a cover of $M$, then we can write $U_\alpha \cap U_\beta = \cup_{\gamma \in J} (U_\alpha \cap U_\beta) \cap U_\gamma$, and we have that $(\varphi_\beta \circ \varphi_\alpha^{-1}) = (\varphi_\beta \circ \varphi_\gamma^{-1}) \circ (\varphi_\gamma \circ \varphi_\alpha^{-1})$ is the transition map on $(U_\alpha \cap U_\beta) \cap U_\gamma$. This is a smooth map as this is a composition of smooth maps through $\varphi_\gamma(U_\gamma)$. Therefore, we have an open cover of $U_\alpha \cap U_\beta$ which is smooth on the transition maps in $\atlas$. Therefore, we have that the transition map is smooth.

To show that this transition map is maximal, take any maximal atlas $A$ containing $\atlas$. Then we have that this is the collection of maps which contain $\atlas$ and are pairwise smoothly compatible, so it also contains $\overline{\atlas}$ above. If $\alpha$ is in $A$, then it is also smoothly compatible with $\atlas$, so it is in $\overline{\atlas}$. Therefore, $A \subseteq \overline{\atlas}$ and $\overline{\atlas} \subseteq A$. Therefore, $A = \overline{\atlas}$, thus $\overline{\atlas}$ is maximal. 
\subsubsection{b}

Suppose $\mathcal{A}$ and $\mathcal{B}$ are two atlases which determine the same maximal smooth atlas. So $\overline{\mathcal{A}} = \overline{\mathcal{B}}$. Then this means that $\mathcal{A} \subseteq \overline{\mathcal{B}}$, so $\mathcal{A}$ is smoothly compatible with $\mathcal{B}$. Thus the union $\mathcal{A} \cup \mathcal{B}$ is also a smooth atlas. 

Suppose $\mathcal{A}$ and $\mathcal{B}$ have the property that the union is a smooth atlas. Therefore, we have that $\mathcal{A} \subseteq \overline{\mathcal{B}}$ by definition of $\overline{\mathcal{B}}$. But this means that $\overline{\mathcal{A}} \subseteq \overline{\mathcal{B}}$ as it contains $\mathcal{A}$ and is maximal. Changing letters, we have that $\overline{\mathcal{B}} \subseteq \overline{\mathcal{A}}$ by symmetry. Therefore, $\overline{\mathcal{A}} = \overline{\mathcal{B}}$.

\subsection{6}
We have that $M(n)$ is a smooth manifold. Now consider the function $\det : M(n) \rightarrow \mathbb{R}$, which is a smooth function between two manifolds as it is a polynomial. Now look at $\det^{-1} (1)$, which is precisely the matrices of determinant $1$. Now we want to show that the Jacobian on this set is not the zero vector, so that the preimage is a submanifold. 

Then as we have that $(D \det A)_{ij} = (-1)^{i + j} \det(A^*_{i, j})$ where $A^*_{i, j}$ is the cofactor matrix of $A$ deleting $i$ and $j$. Then this means that this is zero iff the determinant of the cofactor matrix is zero, but this is never the case as the determinant of $A$ itself is nonzero, meaning $A$ is invertible and so are all of its cofactors. Therefore, $D \det(A)$ is 0 iff $\det A = 0$, meaning that the preimage is a submanifold. 

Finally, we have that the dimension is $n^2-1$ and the codimension is 1. 

\end{document}
