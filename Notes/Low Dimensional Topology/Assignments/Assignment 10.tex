\documentclass{article}
\usepackage[margin=1in]{geometry}
\usepackage{amsmath}
\usepackage{amssymb}
\usepackage{amsthm}
\usepackage{url}
\usepackage{todonotes}
\usepackage{svg}
\usepackage{cleveref}

% Environments

\newtheorem{theorem}{Theorem}
\newtheorem{proposition}[theorem]{Proposition}
\newtheorem{corollary}[theorem]{Corollary}
\newtheorem{lemma}[theorem]{Lemma}
\newtheorem{definition}[theorem]{Definition}
\newtheorem{conjecture}[theorem]{Conjecture}
\newtheorem{remark}[theorem]{Remark}


\theoremstyle{definition}
\newtheorem{example}[theorem]{Example}

\numberwithin{theorem}{section}
\numberwithin{equation}{section}

\DeclareMathOperator{\Int}{int}
\DeclareMathOperator{\Star}{st}
\DeclareMathOperator{\Lk}{lk}
\newcommand{\atlas}{\mathcal{A}}
\newcommand{\cover}{\widetilde{X}}
\DeclareMathOperator{\Homeo}{Homeo}
\newcommand*{\inter}{\hat{i}}
\newcommand*{\twequiv}{\sim_c}


%opening
\title{Assignment 10}
\author{Eric Luu}

\begin{document}
\section{9}
\subsection{1}
The standard longitude is unique up to ambient isotopy. Let $K$ be a knot and let $\lambda$ be a standard longitude of $N(K)$. Then $\lambda$ goes around $\partial N(K)$ and is homologically trivial in $S^3 - N(K)$. Now let $\mu$ be the meridian of $N(K)$, which crosses $\lambda$ once. $\mu$ is the generator of the homology of $S^3 - N(K)$, from applying Mayer-Vietoris (Last assignment I think I wrote it the wrong way around). Every curve $C \in \partial N(K)$ is ambient isotopic to a curve of the form $m \mu + n \lambda$. Therefore, the homology group $H_1(S^3 - N(K))$ sends $m \mu + n \lambda$ to $m$. But if $C$ is homologically trivial, then $m = 0$, therefore $n = \pm 1$. Then $C$ is ambient isotopic to the standard longitude. 

\subsection{2}
The surface on the left has two Seifert circles, four crossings and two components. Then the genus of the Seifert surface is $1 - \frac{2 + 2 - 4}{2} = 1$. The surface on the right has four Seifert circles, four crossings and two components. Then the genus of the Seifert surface is $1 - \frac{4 + 2 - 4}{2} = 0$. Then the genus of the link is 0 as this is the smallest genus surface possible. 

\subsection{4}
Let $S$ be the Seifert surface. 
Deformation retract the Seifert disks to a point and deformation retract the crossings to edges. There is one crossing per edge and there is one Seifert circle per Seifert disk. Then the Euler characteristic $\chi$ is invariant under homotopy, so we can calculate the Euler characteristic of this surface. Then $\chi(S) = s - c$, which is the number of vertices minus the number of edges. Secondly, $\chi(S) = 2 - 2g - n$, where $g$ is the genus of $S$ with disks added, and $n$ are the number of disks removed from the surface. However, $n$ is also equal to the number of boundary components as every disk added to $S$ removes one boundary component. Therefore, $s - c = 2 - 2g - n$, so by algebraic manipulation, $g = 1 - (s + n - c)/2$. 

\subsection{5}
Draw $p$ concentric circles, and from the centre draw $q$ straight lines. Then add points of crossing and label them $x_{i,j}$ where $i$ is the $i$-th circle from the inside out, and $j$ is the $j$-th straight line clockwise. Draw a line from $x_{p, j}$ to $x_{1, j+1}$. Then draw strands from $x_{i, j}$ to $x_{i + 1, j + 1}$, going under the other strand. Then this is a $(p, q)$ torus knot. An illustration is in Rolfsen. Every strand is going clockwise around the circle. Each concentric circle is a Seifert circle, with the Seifert disks stacked on top of each other. There are $p$ Seifert disks. The only crossings are the two lines on $x_{p, j} - x_{1, j+1}$ and $x_{i, j} - x_{i + 1, j}$. There are $q(p-1)$ crossings. There is one boundary component. From question 3, the genus $g$ is
\[1 - \frac{p + 1 - q(p-1)}{2} = (p-1)(q-1)/2.\]
\end{document}
