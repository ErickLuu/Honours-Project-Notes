\documentclass{article}
\usepackage[margin=1in]{geometry}
\usepackage{amsmath}
\usepackage{amssymb}
\usepackage{amsthm}
\usepackage{url}
\usepackage{todonotes}
\usepackage{svg}
\usepackage{cleveref}

% Environments

\newtheorem{theorem}{Theorem}
\newtheorem{proposition}[theorem]{Proposition}
\newtheorem{corollary}[theorem]{Corollary}
\newtheorem{lemma}[theorem]{Lemma}
\newtheorem{definition}[theorem]{Definition}
\newtheorem{conjecture}[theorem]{Conjecture}
\newtheorem{remark}[theorem]{Remark}


\theoremstyle{definition}
\newtheorem{example}[theorem]{Example}

\numberwithin{theorem}{section}
\numberwithin{equation}{section}

\DeclareMathOperator{\Int}{int}
\DeclareMathOperator{\Star}{st}
\DeclareMathOperator{\Lk}{lk}
\newcommand{\atlas}{\mathcal{A}}
\newcommand{\cover}{\widetilde{X}}
\DeclareMathOperator{\Homeo}{Homeo}
\newcommand*{\inter}{\hat{i}}
\newcommand*{\twequiv}{\sim_c}


%opening
\title{Assignment 9}
\author{Eric Luu}

\begin{document}

\section{8}
\subsection{1}

Let $A$ be $N(K)$ and let $B$ be a tubular neighbourhood of $S^3 - N(K)$. Then $A \cup B = S^3$, $A$ deformation retracts to $S^1$, $B$ deformation retracts to $S^3 - K$. Furthermore, $A \cap B$ deformation retracts to $T^2$. Therefore, their homology groups are identical by homotopy equivalence. Then we apply Mayer-Vietoris on $S^3$, $A$ and $B$ to have the exact sequence:

\begin{equation*}
    H_2(S^3) \rightarrow H_1(A \cap B) \xrightarrow{\alpha} H_1(A) \oplus H_1(B) \rightarrow H_1(S^3)
\end{equation*}

From homology, $H_2(S^3) = H_1(S^3) = 0$. $H_1(A \cap B) = H_1(T^2) = \mathbb{Z} \oplus \mathbb{Z}$. $H_1(A) = H_1(S^3 - K)$ and $H_1(B) = H_1(S^1) = \mathbb{Z}$. Plugging this all in:

\begin{equation*}
    0 \rightarrow \mathbb{Z} \oplus \mathbb{Z} \xrightarrow{\alpha} H_1(S^3 - K) \oplus \mathbb{Z} \rightarrow 0
\end{equation*}
is exact. Then this implies that $\alpha$ is an isomorphism. Therefore by the fundamental theorem of abelian groups, $H_1(S^3 - K) \cong \mathbb{Z}$. 
\subsection{4}
The fundamental group of the figure-eight knot has presentation 

\begin{equation}
    \left\langle
        x_1, x_2, x_3, x_4|
        x_1 x_3 = x_3 x_2,
        x_4 x_2 = x_3 x_4,
        x_3 x_1 = x_1 x_4
    \right\rangle
\end{equation}
We want to show that $x_2 x_4 = x_1 x_2$ is a consequence of the other three. 

First, from relation 3, $x_4 = x_1^{-1} x_3 x_1$. Then from relation $1$, $x_1^{-1} = x_3 x_2^{-1} x_3^{-1}$. Therefore, $x_4 = x_3 x_2^{-1} x_1$. Then from relation $2$, $x_3 = x_4 x_2 x_4^{-1}$. Then $x_4 = x_4 x_2 x_4^{-1} x_2^{-1} x_1$. Then $x_2^{-1} x_1^{-1} = x_4^{-1} x_2^{-1}$. Then $x_2 x_4 = x_1 x_2$.
\subsection{7}
Firstly, $A$ is homeomorphic to $B^3$ with $n$ disjoint cylinders $C_1, \ldots , C_n$ drilled out. Now $B^3$ with a single cylinder drilled out deformation retracts to $S^1$. Use Seifert Van Kampen on $B$. Take $S_i$ to be $B$ excluding a neighbourhood around every cylinder except for the $i$-th cylinder. Then the pairwise overlap of $S_i \cap S_j$ is a $3$-ball, so every overlap yields no relators. Furthermore, $\pi_1(S_i) = \mathbb{Z}$ with generator the loop that goes around the cylinder, which is $x_i$ in $A$. Then $B = \bigcup_{i = 1}^n S_i$. Using Seifert-Van Kampen, $\pi_1(A)$ is the free group generated by $\pi_1(S_i)$ which is the free group on $n$ generators. Each generator of $\pi_1(S_i)$ is $x_i$. Then $\pi_1(A)$ is the free group generated by $x_1, ..., x_n$. 

\subsection{8}
Firstly, $\langle x, y | xyx = yxy \rangle$ implies that $(xyx)(yxy) = (xyx)(xyx)$. Then letting $\rho(a) = xyx$, $\rho(b) = xy$, then $a^2 = b^3$. To show that this map $\rho : \langle a, b | a^2 = b^3 \rangle \rightarrow \langle x, y | xyx = yxy \rangle$ is an isomorphism, we show $\rho$ is a homomorphism and has an inverse. First, $\rho$ is a homomorphism because it acts on every letter individually. Then letting $\rho^{-1}(x) = b^{-1} a, y = a^{-1} b^2$, then $\rho^{-1}(\rho(a)) = \rho^{-1}(xyx) = (b^{-1} a)(a^{-1} b^2)(b^{-1} a) =a$. Then $\rho^{-1}(\rho(b)) = \rho^{-1}(xy) = b^{-1} a a^{-1} b^2 = b$. Next, $\rho(\rho^{-1}(x)) = \rho(b^{-1} a) = (y^{-1} x^{-1}) (xyx) = x$ and $ \rho(\rho^{-1}(y)) = \rho(a^{-1} b^2 ) = (x^{-1} y^{-1} x^{-1})(xy)(xy) = y$. Then $\rho$ has an inverse, so $\rho$ is an isomorphism. 
\end{document}
