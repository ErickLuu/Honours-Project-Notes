\documentclass{article}
\usepackage[margin=1in]{geometry}
\usepackage{amsmath}
\usepackage{amssymb}
\usepackage{amsthm}
\usepackage{url}
\usepackage{todonotes}
\usepackage{svg}
\usepackage{cleveref}

% Environments

\newtheorem{theorem}{Theorem}
\newtheorem{proposition}[theorem]{Proposition}
\newtheorem{corollary}[theorem]{Corollary}
\newtheorem{lemma}[theorem]{Lemma}
\newtheorem{definition}[theorem]{Definition}
\newtheorem{conjecture}[theorem]{Conjecture}
\newtheorem{remark}[theorem]{Remark}


\theoremstyle{definition}
\newtheorem{example}[theorem]{Example}

\numberwithin{theorem}{section}
\numberwithin{equation}{section}

\DeclareMathOperator{\Int}{int}
\DeclareMathOperator{\Star}{st}
\DeclareMathOperator{\Lk}{lk}
\newcommand{\atlas}{\mathcal{A}}
\newcommand{\cover}{\widetilde{X}}


%opening
\title{Assignment 3}
\author{Eric Luu}

\begin{document}
\section{Sheet 3}
\subsection{3}
We will show that $\rho_1 \times \rho_2: \cover_1 \times \cover_2 \rightarrow X_1 \times X_2$ is a covering space. 
Take a point $(x_1, x_2) \in X_1 \times X_2$. Then let $U$ be an evenly covered neighbourhood in $X_1$ such that $x_1 \in U$ and $p_1^{-1}(U) = \{U_\alpha \}_{\alpha \in A}$ be its sheets, where $p_1|_{U_\alpha}: \cover_1 \rightarrow U$ is a homeomorphism. Similarly, define an evenly covered neighbourhood $V \subseteq X_2$ such that $x_2 \in V$ and $p_2^{-1}(V) = \{V_\beta \}_{\beta \in B}$ be its sheets. 

We claim $U \times V$ is an evenly covered neighbourhood of $(x_1, x_2) \in X_1 \times X_2$. Firstly, $U \times V$ is connected by definition of the product of connected sets. Then the preimage, $(p_1 \times p_2)^{-1}(U \times V) = p_1^{-1}(U) \times p_2^{-1}(V)$ are the sets $\{U_\alpha \}_{\alpha \in A} \times \{V_\beta \}_{\beta \in B} = \{U_\alpha \times V_\beta : \alpha \times \beta \in A \times B\}$. These sets are open from the product topology. $U_i \times V_j$ is disconnected from $U_k \times V_\ell$. If $U_i \neq U_k$, then $U_i$ is disconnected from $U_k$. Therefore, $U_i \times V_j$ is disconnected from $U_k \times V_\ell$. Repeat for $V_j, V_\ell$. Therefore, $\{U_\alpha \}_{\alpha \in A} \times \{V_\beta \}_{\beta \in B} $ is open and disconnected.
Finally, $\rho_1 \times \rho_2 |_{U_\alpha \times V_\beta}: \cover_1 \times \cover_2 \rightarrow U \times V$ is a homeomorphism. 

If $f_1 : X_1 \rightarrow Y_1$ is a homeomorphism, $f_2 : X_2 \rightarrow Y_2$ is a homeomorphism, then $f_1 \times f_2 : X_1 \times X_2 \rightarrow Y_1 \times Y_2$ is a homeomorphism. This map is a bijection as $f_1, f_2$ is a bijection. This map is continuous by the product topology. Similarly,$(f_1 \times f_2 )^{-1} = f_1^{-1} \times f_2^{-1}$ is also continuous by the product topology. Therefore, $f_1 \times f_2 : X_1 \times X_2 \rightarrow Y_1 \times Y_2$ is a homeomorphism. Thus $\rho_1 \times \rho_2 |_{U_\alpha \times V_\beta}: \cover_1 \times \cover_2 \rightarrow U \times V$ is a homeomorphism. 

\subsection{Exercise 4}
For convenience, denote $\mathbb{C} - \{0\} = \mathbb{C}^*$. 
Let $q(r e^{i \theta}) = (e^{i \ln(r)}, e^{i \theta})$ be a covering map from $\mathbb{C}^* \rightarrow T^2$. Recall from algebraic topology (Hatcher, Prop 1.31) that the induced map $q_* : \pi_1(\mathbb{C}^*) \rightarrow \pi_1(T^2)$ is injective and that the image subgroup $q_*(\pi_1(\mathbb{C}^*))$ are precisely the loops of $X$ that lift to loops. 

As $\pi_1(\mathbb{C}^*) = \mathbb{Z}$ and $\pi_1(T^2) = \mathbb{Z} \times \mathbb{Z}$, it is enough to show that $1 \in \pi_1(\mathbb{C}^*)$ is sent to $\langle 0, 1 \rangle$ in $\pi_1(T^2)$. This is from the injective property of $q_*$. 

Take the loop $\gamma : I \rightarrow \mathbb{C}^*: \gamma(t) = e^{2 \pi i t}$, where $[\gamma]$ is the loop corresponding to $1 \in \mathbb{Z}$. Then $q \circ \gamma$ is the loop $(0, e^{2 \pi i t})$ in $T$, which is the meridian loop. The meridian loop is in the homotopy equivalence of $\langle 0, 1 \rangle \in \mathbb{Z} \times \mathbb{Z}$. Therefore, $q_\ast$ sends $1 \in \pi_1(\mathbb{C}^*)$ to $\langle 0, 1 \rangle$ in $\pi_1(\mathbb{C}^*)$. Therefore loops of the form $ \langle a, b \rangle \in T^2$ will lift to a loop in $\mathbb{C}^*$ if and only if $a = 0$.

\subsection{Exercise 6}

Any two knots in $T^2$ in the homotopy class of $\langle 0, \pm 1 \rangle$ are ambient isotopic.

Recall that Lemma 3.15 goes as follows:
\begin{lemma}
    Suppose $K$ is a knot ambient isotopic to the standard meridian $\mu \simeq \langle 0, 1 \rangle$ in $T^2$ and $K$ intersects $\mu$ transversally a finite number of times. Then $K$ is ambient isotopic to $\mu$. 
\end{lemma}

Recall that Theorem 3.16 goes as follows.

\begin{theorem}
    Suppose $J$ and $K$ are curves on a surface. Then within any $\varepsilon$-neighbourhood of $J$ lies a simple closed curve, ambient isotopic to $J$, and that $J'$ intersects $K$ transversally finitely many times at every point in their intersection.
\end{theorem}

We shall show that any knot $K$ in $\langle 0, \pm 1 \rangle$ is ambient isotopic to $\mu$, the standard meridian. If this is the case, then as ambient isotopy is an equivalence relation, then any $J, K$ in the homotopy class of $\langle 0, \pm 1 \rangle$ is ambient isotopic to each other.

Let $K$ be a knot in $\langle 0, \pm 1 \rangle$ and let $\mu$ be the standard meridian. Apply theorem 3.16 to an arbitrarily small neighbourhood around $K$. There exists a simple closed curve $K'$ ambient isotopic to $K$ such that $K'$ only transversally intersects $\mu$ and the number of intersection points is finite. Apply theorem 3.15 to $K'$. There exists an ambient isotopy from $K'$ to $\mu$. Therefore, $K$ is ambient isotopic to $\mu$ by the equivalence relation of ambient isotopy.


\subsection{Ambient isotopy is an equivalence relation}
Let $X$ be the topological space we are working over. Let $J, K, L$ be three knots in $X$. If $J$ and $K$ are ambient isotopic, then $J \cong K$.
\paragraph{Ambient isotopy is reflexive}
Show that $K \cong K$. Let $H : X \times I \rightarrow X$ where $H(x, t) = x$. Then $H(\cdot, 0) = Id$ and $H(K, 1) = K$ as $H(x, 1) = x$. Therefore, $K \cong K$. 

\paragraph{Ambient isotopy is symmetric}
Show that if $K \cong L$, then $L \cong K$. Let $H : X \times I \rightarrow X$ be an ambient isotopy from $K$ to $L$. Write $H( \cdot, t) = h_t : X \rightarrow X$. From the definition of ambient isotopy, $h_0 = Id$ and $h_1(K) = L$. Now let $G : X \times I \rightarrow X$ be defined as $G(x, t) = h_1^{-1}(H(x, 1-t))$. Then $G(\cdot, 0) = h_1^{-1}\circ h_1 = Id$. This is a continuous map as it is the composition of continuous maps. Furthermore, as $h_1$ is a homeomorphism and $H(\cdot, 1-t)$ is also a homeomorphism, $G(\cdot, t)$ is a homeomorphism. Finally, $G(L, 1) = h_1^{-1}(H(L, 0)) = h_1^{-1}(L) = K$. Therefore, $G$ is an ambient isotopy from $K$ to $L$. 

\paragraph{Ambient isotopy is transitive}

If $J \cong K$, $K \cong L$ then $J \cong L$. Suppose $F, G : X \times I \rightarrow X$ where $F$ is an ambient isotopy from $J$ to $K$ and $G$ is an ambient isotopy from $K$ to $L$. Then define:

\begin{equation*}
    H(x, t) =
    \begin{cases}
        F(x, 2t) & 0 \leq t \leq 1/2\\
        G(F(x, 1), 2t-1) & 1/2 \leq t \leq 1
    \end{cases}
\end{equation*}
On fixed $t$, $H$ is a homeomorphism as $H$ is a composition of homeomorphisms. $H$ is continuous as at time $t = 1/2$, $H(x, t) = G(F(x, 1) , 0) = F(x, 1)$. By the gluing lemma, the map is piecewise continuous. Finally, $H(x, 0) = x$ and $H(J, 1) = G(F(J, 1), 1) = G(K, 1) = L$. Therefore, $H$ is an ambient isotopy from $J$ to $L$. 
\end{document}
