\documentclass{article}
\usepackage[margin=1in]{geometry}
\usepackage{amsmath}
\usepackage{amssymb}
\usepackage{amsthm}
\usepackage{url}
\usepackage{todonotes}
\usepackage{svg}
\usepackage{cleveref}

% Environments

\newtheorem{theorem}{Theorem}
\newtheorem{proposition}[theorem]{Proposition}
\newtheorem{corollary}[theorem]{Corollary}
\newtheorem{lemma}[theorem]{Lemma}
\newtheorem{definition}[theorem]{Definition}
\newtheorem{conjecture}[theorem]{Conjecture}
\newtheorem{remark}[theorem]{Remark}


\theoremstyle{definition}
\newtheorem{example}[theorem]{Example}

\numberwithin{theorem}{section}
\numberwithin{equation}{section}

\DeclareMathOperator{\Int}{int}
\DeclareMathOperator{\Star}{st}
\DeclareMathOperator{\Lk}{lk}
\newcommand{\atlas}{\mathcal{A}}
\newcommand{\cover}{\widetilde{X}}
\DeclareMathOperator{\Homeo}{Homeo}
\newcommand*{\inter}{\hat{i}}
\newcommand*{\twequiv}{\sim_c}


%opening
\title{Assignment 6}
\author{Eric Luu}

\begin{document}

\section{6}

\subsection{1}
Show that $M_1 \cup_f M_2$ is a 3-manifold and orientable. 

We will first show $M_1 \cup_f M_2$ is locally Euclidean. Any point in the interior of $M_1$ or $M_2$ is in an open neighbourhood homeomorphic to an open ball, by $M$ being a manifold. Every point on $\partial M_1$ has a neighbourhood $N$ homeomorphic to an open set in $H^3$, the $3$-half plane with the subset topology of $\mathbb{R}^3$. As $f$ is a homeomorphism on $\partial M_1 \rightarrow \partial M_2$, $f|_{\partial N}$ has a corresponding open and connected neighbourhood $K$ in $\partial M_2$. Take any neighbourhood $N'$ homeomorphic to an open neighbourhood in $H^3$ in $M_2$ where $K = N' \cap \partial M_2$ (which exists as $K$ is continuous). Then $N \cup_f N'$ is an open neighbourhood around $p$. Furthermore, as $N$ and $N'$ are homeomorphic to open subsets of $H^3$ with neighbourhoods homeomorphic on $\partial H^3$, then gluing these along the homeomorphisms is an open disk in $\mathbb{R}^3$. Therefore, $M_1 \cup_f M_2$ is locally Euclidean.

Secondly, $M_1 \cup_f M_2$ is Hausdorff. By the quotient map, the union of the interior of $M_1$ and $M_2$ is Hausdorff, and every point in $M_1$ with a point on the boundary of $M_1$ have disjoint neighbourhoods as $M_1$ is Hausdorff. Furthermore, any two points on $\partial M_1$ and $\partial M_2$ are Hausdorff, as we take the intersection of Hausdorff open sets in $M_1$ and $M_2$. We can do this as $\partial M_1$ and $\partial M_2$ are homeomorphic. Therefore, $M_1 \cup_f M_2$ is Hausdorff. 

The union of bases for $M_1 \sqcup M_2$ is a basis for $M_1 \cup_f M_2$. As this is second-countable, so is $M_1 \cup_f M_2$. Therefore, $M_1 \cup_f M_2$ is a $3$-manifold.

To show orientability, take an orientation of $M_1 \sqcup M_2$. Since we glue outward normals to inward normals, we glue counterclockwise faces to clockwise faces in the simplicial approximation of $M_1$ and $M_2$. Therefore, the orientation given by $M_1$ is compatible with the orientation of $M_2$, therefore $M_1 \cup_f M_2$ is orientable. 

\subsection{3}

\paragraph*{$2 \Rightarrow 1$} Suppose $\gamma$ bounds a disk $D^2 \subseteq V$. Now take the deformation retraction $f : D^2 \times I \rightarrow D^2$ by the function $f(x, t) = xt$. At time $t = 1$, this is the identity map, but at time $t = 0$, this sends the disk to a single point. This is also a homotopy of $\partial D^2$ to a point. Therefore, $\gamma = \partial D^2$ is homotopic to a point on $D^2$, therefore it is homotopic to a point on $V$.

\paragraph*{$3 \Rightarrow 2$} Take the image $h({1} \times D^2)$. The boundary of this image is $\gamma$ and as this is a framing then the image is homeomorphic to a disk. Therefore, $\gamma$ bounds a disk in $V$.

\paragraph{$2 \Rightarrow 3$} Cut $V$ along $\gamma$ and the disk it bounds. Then this is a 2-ball, where $\gamma$ is on the boundary of the $2$-ball twice. By the annulus theorem, there exists a homeomorphism of the closed cylinder $D^2 \times I$ that takes $\partial D^2 \times {0}$ to one copy of $\gamma$ and $\partial D^2 \times {1}$ to another copy of $\gamma$. Gluing these two disks togehter in $D^2 \times I$ in the orientable way yields a framing $h : S^1 \times D^2$ where $h(\{1\} \times \partial D^2) = \gamma$. 

\paragraph{$1 \Rightarrow 2$} Let $f : S^1 \rightarrow V$ be a null-homotopic map, so there exists an $F: S^1 \times I \rightarrow V$ be a homotopy from $f$ to $V$. Now $F(x, 1) = x_1$ for all $x \in S^1$. Futhermore, the space $S^1 \times I$ where $S^1 \times \{1\}$ is homeomorphic to $D^2$. Therefore, $F : D^2 \rightarrow V$ is a map where $\gamma = \partial D^2$ is the map, therefore $\gamma$ bounds a disk in $V$.

\subsection{5}

There is an ambient isotopy $h : \partial V \times I \rightarrow \partial V$ such that $h(\alpha, 1) = \beta$, as all meridians on the torus are ambient isotopic. But this induces an isotopy from $h_0 = Id$ to $h_1$ which is a homeomorphism from $\alpha$ to $\beta$. Therefore, $h_t$ can be extended to be a homeomorphism from a disk in $\alpha$ to a disk in $h_t(\alpha)$. Using Alexander's isotopy lemma from the identity to $h_1$, $h : \partial V \cup D^2 \times I \rightarrow \partial V$ is an ambient isotopy. Then if we cut $V$ along $D$, then this is a closed ball $B$, so $h$ has an extension from the boundary of the ball to the whole ball by Alexander's isotopy. Now because $\alpha$ becomes two circles on $B$, then we use the annulus theorem to split $\partial B$ into two disks and an annulus. Since $h: \partial B_n \rightarrow \partial B_n$ takes both copies of $\alpha$ to both copies of $\beta$ in the cut, then it also takes disks to disks and therefore this is an ambient isotopy. Therefore, we can extend $h : B_n \times I \rightarrow B_n$ for an ambient isotopy of $B_n$ from $Id$ to a map which takes both copies of $\alpha$ to $\beta$. Gluing along the boundaries again, we get a map $V \times I \rightarrow V$ that takes $\alpha $ to $\beta$. 

\subsection{6}
\subsubsection{a}
One pair of relatively prime integers that gives a longitude of $V$ is $\langle 1, 1\rangle$. 
\subsubsection{b}
\begin{lemma}
    A curve $\langle a, b \rangle$ on the torus $T^2$ intersects a particular meridian exactly once if and only if $|a| = 1$. 
\end{lemma}

\begin{proof}
    Consider the universal cover, $\mathbb{R}^2 \rightarrow T^2$. The curve $\langle a, b \rangle$ lifts to a curve which starts at $(0,0)$ and goes to $(a, b)$. Now draw lines when $x = \frac{1}{2}$. This is the preimage of the meridian in $\mathbb{R}^2$. TTherefore, $\langle a, b \rangle$ intersects the meridian exactly once if and only if $|a| = 1$. Thus shown. 
\end{proof}

\begin{theorem}
    $\zeta$ is a longitude if and only if the intersection number with the class of meridians up to isotopy is 1. 
\end{theorem}

\begin{proof}
    Since meridians are ambient isotopic, then moving meridians around, $i(\zeta, [m]) = 1$ for all $m$. 
\end{proof}

Therefore, $\langle a, b \rangle$ is a longitude is a meridian if and only if $|a| = 1$, $b \in \mathbb{Z}$. 

\end{document}
