\documentclass{article}
\usepackage[margin=1in]{geometry}
\usepackage{amsmath}
\usepackage{amssymb}
\usepackage{amsthm}
\usepackage{url}
\usepackage{todonotes}
\usepackage{svg}
\usepackage{cleveref}

% Environments

\newtheorem{theorem}{Theorem}
\newtheorem{proposition}[theorem]{Proposition}
\newtheorem{corollary}[theorem]{Corollary}
\newtheorem{lemma}[theorem]{Lemma}
\newtheorem{definition}[theorem]{Definition}
\newtheorem{conjecture}[theorem]{Conjecture}
\newtheorem{remark}[theorem]{Remark}


\theoremstyle{definition}
\newtheorem{example}[theorem]{Example}

\numberwithin{theorem}{section}
\numberwithin{equation}{section}

\DeclareMathOperator{\Int}{int}
\DeclareMathOperator{\Star}{st}
\DeclareMathOperator{\Lk}{lk}
\newcommand{\atlas}{\mathcal{A}}
\newcommand{\cover}{\widetilde{X}}
\DeclareMathOperator{\Homeo}{Homeo}
\newcommand*{\inter}{\hat{i}}
\newcommand*{\twequiv}{\sim_c}


%opening
\title{Assignment 4}
\author{Eric Luu}

\begin{document}
\section{5}
\subsection{1}
Let $S$ be homeomorphic to the surface $K = (S^2)_g \# T_1 \# ... \# T_g$, where all boundaries are on $S^2$. Then let $a_0, ... a_{g-1}$ be sccs where $a_i$ is on the intersection of $T_{i}$ and $T_{i + 1}$ and $a_0$ is on the intersection of $(S^2)_g$ and $T_1$. Then let $\alpha$ be a nonseparating scc on $K$. There exists a homeomorphism $j : K \rightarrow K$ such that $j(\alpha)$ is the meridian loop only on $T_g$. Now let $\gamma$ be $a_h$. Then $K - \gamma \cup j(\alpha)$ will have components of genus $h$ and another component of genus $g - h - 1$. Therefore, $K - j^{-1}(\gamma) \cup \alpha$ will also have components of genus $h$ and $g - h - 1$. Thus shown. 
\subsection{2}
\subsubsection{a}
Let $\alpha, \alpha'$ and $\beta, \beta'$ be isotopic with isotopies $h_\alpha$ and $h_\beta$. Firstly, $h_\beta$ maps $\beta$ to $\beta'$ along a strict homeomorphism. Because this map is an ambient isotopy, all transverse intersections between $\beta$ and $\alpha$ must either be moved along $\alpha$, be removed as part of some bigon, or added as some part of a bigon. 

However, every bigon on a surface bounds a disk so traversing the boundary of the bigon must hold that the crossing must go left and then right. Therefore, the algebraic sum of the bigon is $0$. 
Therefore, $\inter(\alpha, \beta') =  \inter(\alpha, \beta)$. Now repeat this argument for $\inter(\alpha, \beta')$ and $\inter(\alpha', \beta')$ but traverse along $\beta'$ instead, and use subsection $b$. 

\subsubsection{b}
Proof by picture. At every point in $\alpha \cap \beta$, consider traversing along $\alpha$ and then traversing along $\beta$. If at this point $i = \pm 1$ along $\alpha$ then along $\beta$ $i = \mp 1$. Red is $\alpha$, black is $\beta$. 

\begin{figure}[ht]
    \centering
    \includesvg{Figures/w5Q2svg.svg}
\end{figure}

Therefore, $\inter(\alpha, \beta) = - \inter(\beta, \alpha)$. 
As a consequence, setting $\beta = \alpha$, $\inter(\alpha, \alpha) = - \inter(\alpha, \alpha)$. Therefore, $\inter(\alpha, \alpha) = 0$. 
\subsection{3}
\subsubsection{a}
Outside of a neighbourhood of $a$, $T_a^k b$ and $b$ do not intersect. On this neighbourhood, when $b$ crosses $a$, the Dehn twist will cross over $b$ $|k| i(a, b)$ times. This is because the twist passes by every point in $a \cap b$ $|k|$ times. As this is done for every point in $a \cap b$, then the number of crossings is $|k| i(a, b)^2$. 

\subsubsection{b}
Let $\alpha$ be an essential simple closed curve. Suppose for contradiction $T^k_\alpha$ is the identity in $MCG(S)$, and $k$ is the smallest nonzero that has this property. Then for all curves $\beta$, $T_a^k(\beta) \sim \beta$. But this means that $i(T^k_a(\beta), \beta) = 0$ for all $\beta$. But as $a$ is essential, let $\beta$ be a curve such that $i(a, \beta) = 1$. Therefore, $0 = |k| i(a, b)^2$, so $0 = |k|$. Thus contradiction. 
\subsection{Exercise 8}

\paragraph*{i}
Firstly, $T_a T_b a = b$ is equivalent to $T_{T_a T_b(a)} = T_b$ from exercise 4. But this is equivalent to $(T_a T_b) T_a (T_a T_b)^{-1} = T_b$ from the definition of conjugation in exercise $5$. But this is equivalent to $T_a T_b T_a = T_b T_a T_b$. Therefore, this satisfies the braid relation. 

\paragraph*{ii}
We want to show that for all curves $c$, $i(a, c) = i(a, T_a(c))$. Doing a Dehn twist of $c$ around $a$ only shifts the intersection point to another place on $a$. Furthermore, this operation introduces no bigons as any new bigon in the Dehn twist is already a bigon in $c$. Therefore, $i(a, c) = i(a, T_a(c))$. Therefore, $i(a, T_b a) = i(a, T_a T_b a) = i(a, b)$. 
Therefore, $i (a, b) = i(a, T_b(a)) = i(a, b)^2$ from exercise 3. Therefore, $i(a, b) = 1$.
\paragraph*{iii}
From part ii, $T_a T_b a = b$ if and only if $i(a, b) = 1$. From part i, $T_a T_b a = b$ if and only if $T_a T_b T_a = T_b T_a T_b$. Therefore, $i(a, b) = 1$ if and only if $T_a T_b T_a = T_b T_a T_b$.

\end{document}
