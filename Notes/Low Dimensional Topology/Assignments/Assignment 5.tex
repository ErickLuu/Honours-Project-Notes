\documentclass{article}
\usepackage[margin=1in]{geometry}
\usepackage{amsmath}
\usepackage{amssymb}
\usepackage{amsthm}
\usepackage{url}
\usepackage{todonotes}
\usepackage{svg}
\usepackage{cleveref}

% Environments

\newtheorem{theorem}{Theorem}
\newtheorem{proposition}[theorem]{Proposition}
\newtheorem{corollary}[theorem]{Corollary}
\newtheorem{lemma}[theorem]{Lemma}
\newtheorem{definition}[theorem]{Definition}
\newtheorem{conjecture}[theorem]{Conjecture}
\newtheorem{remark}[theorem]{Remark}


\theoremstyle{definition}
\newtheorem{example}[theorem]{Example}

\numberwithin{theorem}{section}
\numberwithin{equation}{section}

\DeclareMathOperator{\Int}{int}
\DeclareMathOperator{\Star}{st}
\DeclareMathOperator{\Lk}{lk}
\newcommand{\atlas}{\mathcal{A}}
\newcommand{\cover}{\widetilde{X}}
\DeclareMathOperator{\Homeo}{Homeo}



%opening
\title{Assignment 4}
\author{Eric Luu}

\begin{document}
\section{4}
\subsection{2}
Let $K$ and $K'$ be two null-homotopic knots on $T^2$. Then there exists an ambient isotopy of $T^2$ taking $K$ to $K'$.

Let $h : \mathbb{C} \rightarrow T^2$ be the standard covering map.  Without loss of generalisation, suppose $K'$ lifts to a simple closed curve in $\{x + i y : x \in [0, 1], y \in [0, 1]\} \subset \mathbb{C}$. 
Since $K$ and $K'$ are null-homotopic, they lift to loops $\widetilde{K}, \widetilde{K'}$  in $\mathbb{C}$ from Hatcher 1.31. There exists an ambient isotopy from $\widetilde{K}$ to $\widetilde{K'}$ by 2.10. Furthermore, there is a homeomorphism from $\widetilde{K'}$ to $K'$ as it lies in a fundamental region.  

Then there exists an ambient isotopy $\gamma$ in $\mathbb{C} \cong \mathbb{R}^2$ from $\widetilde{K}$ to $\widetilde{K'}$ by Corr 2.6. Therefore, $h \circ \gamma \circ h^{-1}$ is an ambient isotopy from $K$ to $K'$. This map is well-defined as $h|_{\gamma \circ h^{-1}(K')}$ is a homeomorphism. By the equivalency of ambient isotopy, any two null-homotopic knots are ambient isotopic. 

Note that $T^2$ can be replaced by any surface $S$ which have a universal cover of $\mathbb{C}$ or $\mathbb{R}$. 

\subsection{3}
Show that the homeomorphisms isotopic to the identity is a normal subgroup and that two maps are isotopic iff they are in the same coset mod identity homeomorphisms.
Firstly, we will show that $N$ is a subgroup of $\Homeo(S)$. Firstly, if $g \in N$, then $g^{-1} \in N$. Let $H(x, t) : S \times I \rightarrow S$ be an isotopy where $h_0 = Id$ and $h_1 = g$. Then $G(x, t) = h_1^{-1}(H(x, 1-t))$ is a continuous map by composition. Furthermore, $G(x, 0) = h_1^{-1}(H(x, 1)) = Id$ and $G(x, 1) = h_1^{-1}(H(x, 0)) = g^{-1}$. Therefore, $g^{-1}$ is isotopic to $N$. If $f$ and $g$ are isotopic to $N$ with isotopies $F$ and $G$ with $F_1 = f$, $G_1 = g$, $F_0 = G_0 = Id$, then $f \circ g$ is isotopic to $Id$ with isotopy:
\begin{equation*}
    H(x, t) =
    \begin{cases}
        F(x, 2t) & 0 \leq t \leq 1/2\\
        G(F(x, 1), 2t-1) & 1/2 \leq t \leq 1
    \end{cases}
\end{equation*}
This is continuous as this is a composition of continuous functions and at time $t = 1/2$, the two maps agree. At time $t = 0$, $H_0 = Id$ and at $t = 1$, $H_t = g \circ f$. Therefore, if $f, g \in N$, then $g \circ f \in N$. 

Now we will show that $N$ is normal in $\Homeo(S)$. 
Let $f \in \Homeo(S)$, $g \in N$. Then $f g f^{-1} \in N$. Let $G: S \times I \rightarrow S$ be an isotopy with $G_0 = Id$, $G_1 = g$. Then $H(x, t) = f \circ G(\cdot, t) \circ f^{-1} (x) $ is an isotopy. Firstly, $H(x, t)$ is continuous as it is a composition of continuous functions for fixed $x$ and fixed $t$. Then $H(x, 0) = f \circ g_0 \circ f^{-1} = f \circ Id \circ f^{-1} = f \circ f^{-1} = Id$. Then $H(x, 1) = f \circ g_1 \circ f^{-1} = f \circ g \circ f^{-1}$. Therefore, $f \circ g \circ f^{-1} \in N$. Therefore, $N$ is normal.

Finally, suppose $f$ is isotopic to $g$. Take an isotopy $F : S \times I \rightarrow S$ where $F_0 = f$ and $F_1 = g$. Then take $H(x, t) = f^{-1}(F(x, t))$. This is continuous, as this is the composition of two continuous functions. Then $H_0 = f^{-1} \circ f = Id$ and $H_1 = f^{-1} \circ g$. Then $f^{-1} g \in N$. Therefore, $f$ and $g$ are in the same coset. 

Suppose $f^{-1} g \in N$. Then there exists a isotopy $F : S \times I \rightarrow I$ where $F_0 = Id$ and $F_1 = f^{-1} g$. Then take $H = f \circ F$. This is continuous as it is the composition of two continuous functions. Furthermore, $F_0 = f \circ Id = f$ and $F_1 = f \circ f^{-1} \circ g = g$. Therefore, $f$ and $g$ are isotopic.  

\subsection{4}
We want to show that $GL(2, \mathbb{Z})$ is generated by $
\begin{bmatrix}
    1 & 1\\
    0 & 1\\
\end{bmatrix},
\begin{bmatrix}
    1 & 0\\
    1 & 1\\
\end{bmatrix},
\begin{bmatrix}
    0 & 1\\
    1 & 0\\
\end{bmatrix}$. We want to show that any matrix in $GL(2, \mathbb{Z})$ can be reduced to the identity by applying a series of the matrices above, and then taking the inverse of the operations to build back the matrix. 
Suppose $\begin{bmatrix}
    a & b\\
    c & d\\
\end{bmatrix} \in GL(2, \mathbb{Z})$. Therefore, $ad - bc = \pm 1$.

If $A = \begin{bmatrix}
    a & b\\
    c & d\\
\end{bmatrix}$
has determinant $-1$, then 
$\begin{bmatrix}
    a & b\\
    c & d\\
\end{bmatrix} 
\begin{bmatrix}
    0 & 1\\
    1 & 0\\
\end{bmatrix}
=
\begin{bmatrix}
    b & a\\
    d & c\\
\end{bmatrix}
$ has determinant 1. Therefore, consider the case when $ad - bc = 1$.

If $|a| > |c| > 0$, then do the operation:

\begin{equation*}
    \begin{bmatrix}
        1 & \pm 1\\
        0 & 1
    \end{bmatrix}
    \begin{bmatrix}
        a & b\\
        c & d\\
    \end{bmatrix}
    = 
    \begin{bmatrix}
        a \pm c &b\\
        c & d
    \end{bmatrix}
\end{equation*}
so that $|a \pm c| < |a|$.
If $|a| \leq |c| > 0$, then do the operation:
\begin{equation*}
    \begin{bmatrix}
        1 & 0\\
        \pm 1 & 1
    \end{bmatrix}
    \begin{bmatrix}
        a & b\\
        c & d\\
    \end{bmatrix}
    = 
    \begin{bmatrix}
        a &b\\
        c \pm a & d
    \end{bmatrix}
\end{equation*}
such that $|c \pm a| < |c|$. Repeat this operation until the matrix becomes:
\begin{equation*}
    \begin{bmatrix}
        \gcd(a, c) & x\\
        0 & y\\
    \end{bmatrix}
\end{equation*}
for some $x, y \in \mathbb{Z}$. However, $ad - bc = 1$. As $ax - yc = k \gcd(a, c)$ over $\mathbb{Z}$ for some $k \in \mathbb{Z}$, then $\gcd(a, c) = 1$. Therefore, 

\begin{equation*}
    A' = \begin{bmatrix}
        1 & x\\
        0 & y\\
    \end{bmatrix}
\end{equation*}
However, $\det(A') = 1$ as $A'$ is the composition of matrices with determinant 1. Therefore, $y = 1$. Therefore, $A'$ is of the form

\begin{equation*}
    \begin{bmatrix}
        1 & x\\
        0 & 1\\
    \end{bmatrix}
\end{equation*}
for some integer $x \in \mathbb{Z}$. But $
\left(\begin{bmatrix}
    1 & 1\\
    0 & 1\\
\end{bmatrix}\right)^x = \begin{bmatrix}
    1 & x\\
    0 & 1\\
\end{bmatrix}$. Therefore, $A$ can be composed from the above three matrices. 

\subsection{5}
Call all homeomorphisms isotopic to $id$ null-isotopies. 

Between any two points $p_0, p_1 \in S^2$, there is a null-isotopy such that $f(p_0) = p_1$. This is done by rotating the sphere around. 

Take any $f$ which is orientation preserving. Then pick a point $p_0$. Then there exists a null-isotopy $\gamma : S^2 \rightarrow S^2$ such that $\gamma(f(p_0)) = p_0$, through a rotation of the sphere. Then $ \gamma \circ f$ is an isotopy that takes $p_0$ to $p_0$. However, $S^2 \cong D^2/{\partial D^2}$. Consider the quotient map $q : D^2 \rightarrow S^2$, where $p_0$ is the point on $\partial D^2$. Let $g : D^2 \rightarrow D^2$ such that $g \circ q = \gamma \circ f$. This map exists as $\gamma \circ f$ keeps $p_0$ fixed and changes $S^2 - \{p_0\}$, so $g = \gamma \circ f$ on the open disk under the map. Then $g|_{\partial D^2} = Id$. Then by Alexander's isotopy lemma, $g$ is an isotopy. Therefore, $\gamma \circ f$ is also an isotopy, therefore $f$ is an isotopy. Thus shown. 

Orientation-reversing homeomorphims cannot be transformed to orientation preserving homeomorphisms on $S^2$. Consider the Jacobian of the map $(\theta, \phi) \mapsto (-\theta, \phi)$, the map which reverses on one coordinate. This is orientation-reversing. The Jacobian of this map is -1, but the Jacobian of all orientation-preserving maps is positive. All orientation-reversing homeomorphisms is isotopic to the map above, by a similar argument in section 5. Isotopies preserve Jacobians. Therefore, orientation-reversing homeomorphisms cannot be turned into orientation-preserving homeomorphisms. 

\end{document}
