\documentclass[]{article}
\usepackage[margin=1in]{geometry}
\usepackage{amsmath}
\usepackage{amssymb}
\newcommand{\d}{\mathop{}\!{d}}
%opening
\title{Assignment 1}
\author{Eric Luu}

\begin{document}

\maketitle
\section{Question 5'}
We have that:
\begin{equation}
	e^x = \sum_{n \geq 0} \frac{1}{n!}x^n.
\end{equation}
Now consider:
\begin{equation}
	\frac{1}{4} \left(e^x + e^{-x} + e^{ix} + e^{-ix} \right) = \sum_{n \geq 0} \frac{1}{n!} \cdot \frac{1}{4} \left(x^n + (-1)^n x^n + (i)^n x^n + (-i)^n x^n\right).
\end{equation}
We have that if: 
\begin{itemize}
	\item $n \equiv 0 \mod 4$, then the coefficient of $x^n$ is $\frac{1}{n!}\frac{1}{4} \left(1 + 1 + 1 + 1\right) = \frac{1}{n!}$. 
	\item $n \equiv 1 \mod 4$, then the coefficient of $x^n$ is $\frac{1}{n!}\frac{1}{4} \left(1 + -1 + i -i \right) = 0$. 
	\item $n \equiv 2 \mod 4$, then the coefficient of $x^n$ is $\frac{1}{n!}\frac{1}{4} \left(1 + 1 + -1 -1 \right) = 0$. 
	\item $n \equiv 3 \mod 4$, then the coefficient of $x^n$ is $\frac{1}{n!}\frac{1}{4} \left(1 + -1 -i + i \right) = 0$. 
\end{itemize}
Therefore, the only terms that remain are $n \equiv 0 \mod 4$. Therefore, we have that:

\begin{equation}
	\frac{1}{4} \left(e^x + e^{-x} + e^{ix} + e^{-ix} \right) = \sum_{n \geq 0} \frac{1}{(4n)!} x^{4n}.
\end{equation}
Now consider taking the integral of both sides. We have that
\end{document}
