\documentclass[]{article}
\usepackage[margin=1in]{geometry}
\usepackage{amsmath}
\usepackage{amssymb}

\newcommand{\ops}{\overset{\text{ops}}{\leftrightarrow}}
%opening
\title{Assignment 1}
\author{Eric Luu}

\begin{document}

\maketitle
\section{Question 5'}
We have that:
\begin{equation}
	e^x = \sum_{n \geq 0} \frac{1}{n!}x^n.
\end{equation}
Now consider:
\begin{equation}
	\frac{1}{4} \left(e^x + e^{-x} + e^{ix} + e^{-ix} \right) = \sum_{n \geq 0} \frac{1}{n!} \cdot \frac{1}{4} \left(x^n + (-1)^n x^n + (i)^n x^n + (-i)^n x^n\right).
\end{equation}
We have that if: 
\begin{itemize}
	\item $n \equiv 0 \mod 4$, then the coefficient of $x^n$ is $\frac{1}{n!}\frac{1}{4} \left(1 + 1 + 1 + 1\right) = \frac{1}{n!}$. 
	\item $n \equiv 1 \mod 4$, then the coefficient of $x^n$ is $\frac{1}{n!}\frac{1}{4} \left(1 + -1 + i -i \right) = 0$. 
	\item $n \equiv 2 \mod 4$, then the coefficient of $x^n$ is $\frac{1}{n!}\frac{1}{4} \left(1 + 1 + -1 -1 \right) = 0$. 
	\item $n \equiv 3 \mod 4$, then the coefficient of $x^n$ is $\frac{1}{n!}\frac{1}{4} \left(1 + -1 -i + i \right) = 0$. 
\end{itemize}
Therefore, the only terms that remain are $n \equiv 0 \mod 4$. Therefore, we have that:

\begin{equation}
	\frac{1}{4} \left(e^x + e^{-x} + e^{ix} + e^{-ix} \right) = \sum_{n \geq 0} \frac{1}{(4n)!} x^{4n}.
\end{equation}
Now consider taking the integral of both sides.
\begin{align*}
	\int \frac{1}{4} \left(e^x + e^{-x} + e^{ix} + e^{-ix} \right) \,dx &= \int \sum_{n \geq 0} \frac{1}{(4n)!} x^{4n} \,dx\\
	\frac{1}{4} \left(e^x - e^{-x} -i e^{ix} + i e^{-ix} \right) + C &= \sum_{n \geq 0} \frac{1}{(4n+1)!} x^{4n + 1}\\
\end{align*}
We want that when we have that $x = 0$, then both sides equal 0.
\begin{align*}
	\frac{1}{4} \left(e^{(0)} - e^{-(0)} -i e^{i(0)} + i e^{-i(0)} \right) + C &= 0\\
	\frac{1}{4} \left(1 - 1 - i + i \right) + C &= 0\\
	C &= 0
\end{align*}
Therefore, we have that the generating function $f \ops \left\lbrace 0, \frac{1}{1!}, 0, 0, 0, \frac{1}{5!}, 0, 0, 0, \frac{1}{9!}, ...\right\rbrace$ is
\begin{equation}
	f(x) = \frac{1}{4} \left(e^x - e^{-x} -i e^{ix} + i e^{-ix} \right) 
\end{equation}
Therefore, we use the rules of starting the sequence at 0 by multiplying by $\frac{1}{x}$ and using the map $x \mapsto \sqrt[4]{x}$ to remove the zeroes. So we can let
\begin{equation}
	g(x) = \frac{1}{x} f(x)  \circ (\sqrt[4]{x}) \ops \left\lbrace \frac{1}{(4n + 1)!} \right\rbrace_{n \geq 0}. 
\end{equation}
where:
\begin{equation}
	g(x) = \frac{1}{4\sqrt[4]{x}} \left(e^{\sqrt[4]{x}} - e^{- \sqrt[4]{x}} -i e^{i \sqrt[4]{x}} + i e^{-i \sqrt[4]{x}} \right) 
\end{equation}
Now we use the polynomial rule on $g(x)$. Let $P(n) = (2 n + 1)^2 = 4n^2 + 4n + 1$. As $	g(x) \ops \left\lbrace \frac{1}{(4n + 1)!} \right\rbrace_{n \geq 0}$, then $P(xD) g(x) \ops \left\lbrace \frac{P(n)}{(4n + 1)!} \right\rbrace_{n \geq 0}$. Therefore, we have that:

\begin{equation}
	(4 (xD)^2 + 4(xD) + 1) g(x) \ops \left\lbrace \frac{(2n + 1)^2}{(4n + 1)!} \right\rbrace_{n \geq 0}.
\end{equation}
Let us calculate $(4 (xD)^2 + 4(xD) + 1) g(x)$. We have that $(xD)^2 g(x) = (xD)(xD) g(x) = (xD) (x Dg(x)) = x Dg(x) + x^2 D^2 g(x)$. 
Therefore, we have that
\begin{equation}
	(4 (xD)^2 + 4(xD) + 1) g(x) = 4x^2 D^2 g(x) + 8 x (D g(x)) + g(x).
\end{equation}

Now evaluating this at $x = 1$, we have that
\begin{equation}
	4 (D^2 g(x))|_{x = 1} + 8 (D g(x))|_{x = 1} + g(1) = \sum_{n \geq 0}\frac{(2n + 1)^2}{(4n + 1)!}
\end{equation}
Using a computer to evaluate the functions at 1, we have that
\begin{align*}
	g(1) &= \frac{-1}{4e} + \frac{e}{4} + \frac{\sin(1)}{2}\\
	8D(g(x))|_{x = 1} &= \frac{1}{e} - \sin(1) + \cos(1)\\
	4 D^2(g(x))|_{x = 1} &= -\frac{11}{16e} + \frac{e}{16} + \frac{\sin(1)}{2} - \frac{5 \cos(1)}{8}
\end{align*}
Then we have that:

\begin{equation}
	4 (D^2 g(x))|_{x = 1} + 8 (D g(x))|_{x = 1} + g(1) = \frac{1}{16}e^{-1} + \frac{5e}{16} + \frac{3 \cos(1)}{8}\approx 1.075
\end{equation}
\end{document}
