\documentclass[]{article}
\usepackage[margin=1in]{geometry}
\usepackage{amsmath}
\usepackage{amssymb}

\usepackage{hyperref}
\usepackage{cleveref}

\newcommand{\ops}{\overset{\text{ops}}{\leftrightarrow}}
\newcommand{\egf}{\overset{\text{egf}}{\leftrightarrow}}
%opening
\title{Assignment 2}
\author{Eric Luu}

\begin{document}

\maketitle
\section{Question 1}
\subsection{Part a}
We choose $n$ numbers from the set $[2n]$, label them $1$ and label the rest $0$. Therefore, we have that $b_n = \binom{2n}{n}$.
\subsection{Part b}
We have that:
\begin{align*}
	\frac{b_{n+1}}{b_n} &= \frac{\binom{2n + 2}{n+1}}{\binom{2n}{n}}\\
	&= 
	\frac{\frac{(2n + 2)!}{((n+1)!)^2}}{\frac{(2n!)}{(n!)^2}}\\
	&=
	\frac{(2n + 2)(2n + 1)}{(n+1)^2}\\
	&=\frac{2(2n + 1)}{n+1}\\
	&= 4 - \frac{2}{n+ 1}
\end{align*}
Therefore, $(n+1)b_{n+1} = 4(n+1) b_n - 2 b_n$. 
\subsection*{Part c}
Let \begin{equation}
	B(x) \ops \lbrace b_n \rbrace_{n \geq 0}. 
\end{equation}
We have that:
\begin{equation}
	D (B(x)) = \sum_{n \geq 1} (n) x^{n-1} = \sum_{n \geq 0} (n + 1) b_{n+1} x^n.
\end{equation}
We also have that \begin{equation}
	x D B(x) + B(x) = D (x B(x)) = \sum_{n \geq 0} (n + 1) b_n x^{n}. 
\end{equation}
Therefore, we have that:
\begin{align*}
	D(B(x)) &= 4 xD B(x) + 4 B(x) - 2 B(x)\\
	(1-4x) D(B(x))&= 2 B(x)\\
	\frac{B'(x)}{B(x)} &= \frac{2}{1-4x}\\
\end{align*}
\subsection{Part e}
Taking the integral on both sides, we have that:
\begin{align*}
	\log(B(x)) &= -1/2 \log(1-4x) + C\\
	B(x) &= \frac{A}{\sqrt{1-4x}}
\end{align*}
As we have that $B(0) = 1$, then $A = 1$. Therefore,
\begin{equation}
	B(x) = \frac{1}{\sqrt{1-4x}}
\end{equation}

\subsection{Part e}
We have that $(1-4x)^{-1/2} =\sum_{n \geq 0} \binom{-\frac{1}{2}}{n}(-1)^n (4)^n x^n$. Therefore, we have that $[x^n] B(x) = \binom{-\frac{1}{2}}{n}(-1)^n (4)^n$. But we have that:
\begin{align*}
	\binom{-\frac{1}{2}}{n}(-1)^n (4)^n &= \frac{(-\frac{1}{2})(-\frac{1}{2} - 1) ... (-\frac{1}{2} - n + 1)}{n!} (-1)^n (4)^n\\
	&= \frac{(\frac{1}{2})(\frac{1}{2} + 1) ... (\frac{1}{2} + n - 1)}{n!}(4)^n\\
	&= \frac{(1)(3)(5) ... (2n-1)}{n!} 2^n\\
	&= \frac{(2n)!}{2^n (n!)^2}2^n\\
	&= \frac{(2n)!}{(n!)^2}\\
	&= \binom{2n}{n}
\end{align*}
\newpage
\section{Question 2}
\subsection*{a}
Let $w$ be a trivial word of length $2n$, and break apart $w$ into words $w_1, w_2, ... , w_k$ where $w_1, ..., w_k$ are trivial words themselves. Then consider the sub-words $w_a$, $w_b$, $w_c$ where $w_a$ is the word generated by the homomorphism $\varphi_a: G \rightarrow \mathbb{F}_1$ which sends $a$ to $a$ and $b, c$ to $1$. Define $w_b$, $w_c$ analogously. Then consider $w_{i, \ell} = \varphi_\ell(w_i)$. We have that each of these will be a balanced parentheses, with $\ell = "("$ and $\ell^{-1} = ")"$, or interchanging them. Therefore, there is an injection between $t_n$ and six balanced parentheses, so:
\begin{equation}
	t_n \leq [x^n]C(6x)
\end{equation}
where $C(x)$ is the generating function for the number of balanced parentheses of length $2n$. 
To show that there are balanced parentheses which do not correspond to a trivial word, consider $aba^{-1}b^{-1}$. Thus shown inequality for $n \geq 2$. 
\subsection*{b}
We have that all prime words $w$ can be formed in this fashion:
\begin{enumerate}
	\item Start with any $\lambda \in \Lambda$.
	\item Create a word of length $2n - 2$ that avoids $\lambda^{-1}$.
	\item Add the word $\lambda^{-1}$.
\end{enumerate}
Therefore, $s_n = 6 a_{n-1}$, so
\begin{equation}
	S(x) = 6x A(x) + 1 = \frac{3}{5}(1 - \sqrt{1 - 20x}) + 1.
\end{equation}
We can construct any trivial word $w$ of length $2n$ by:
\begin{enumerate}
	\item starting with a prime trivial word $w'$ of length $2k$ where $1 \leq k \leq n$
	\item Continuing with any trivial word $w''$ of length $2(n - k)$.
\end{enumerate}
Therefore, we have that:

\begin{align*}
	t_n &= \sum_{k = 1}^n s_k t_{n-k}\\
		&= \sum_{k = 0}^{n} s_k t_{n-k} - t_ns_0
\end{align*}
However, we have that $s_0 = 1$. From rule 3, we have that:

\begin{equation}
	\sum_{n \geq 1} t_n x^n= \sum_{n \geq 1} \left( \sum_{k = 0}^{n} s_k t_{n-k} - t_n\right)  x^n
\end{equation}

Rewriting this, we have that:
\begin{align*}
	T(x) - 1 &= S(x)T(x) - T(x)\\ 
	1 &= 2T(x) - S(x) T(x)\\
	T(x) &= \frac{1}{2 - S(x)}
\end{align*}
Therefore,
\begin{equation}
	T(x) = \frac{1}{1- \frac{3}{5}\left(1 - \sqrt{1 - 20x}\right)}
\end{equation}
and simplifying, we have that:
\begin{equation}
	T(x) = \frac{5}{3\sqrt{1 - 20x} + 2}
\end{equation}
\newpage
\section{Question 3}
\subsection{Part a}
Consider the path $P$ that goes through the root vertex $r$. $P$ goes through two subtrees of height $h-1$, $T^1$ and $T^2$, and all other subtrees are unaffected. Consider $T_h - P$. In the subtrees $T^1$ and $T^2$ in $T_h - P$, we have that there are $2(k-1)$ subtrees of height $h$, where $h$ ranges between $0$ and $h-2$. Therefore, we have that $T_h - P$ will have $(k-2)p_{h-1} + 2(k-1)\sum_{i = 0}^{h-2} p_i$ paths.
Therefore, we have that:
\begin{align}
	p_h &= (k-2)p_{h-1} + 2(k-1)\sum_{i = 0}^{h-2} p_i + 1\\
	p_0 &= 1
\end{align}
where the $+1$ is the path containing the root. 
\subsection{Part b}
Consider $P(x) \ops \lbrace p_n \rbrace_{n \geq 0}$.
We have that:

\begin{equation}
	\sum_{n \geq 0} p_n x^n = (k-2) \sum_{n \geq 0} p_{n-1} x^n + 2(k-1)\sum_{i = 0}^{h-2} p_i x^n + \frac{1}{1-x}\\
\end{equation}
We use rule 5 to get the sum $ \sum_{i = 0}^{h-2} p_i x^n$. We have that $P(x) \ops \lbrace p_n \rbrace_{n \geq 0}$, therefore, $\frac{P(x)}{1-x} \ops \left\lbrace \sum_{j \geq 0}^n a_j  \right\rbrace_{n \geq 0}$. Therefore, we have that:
\begin{equation}
	\frac{P(x) x^2}{1-x} = \sum_{n \geq 0} \left(\sum_{j = 0}^{n - 2} p_n\right) x^n.
\end{equation}

Therefore, we have that:

\begin{equation}
	P(x) = (k-2) x P(x)  + 2(k-1) \frac{x^2 P(x)}{1-x}+ \frac{1}{1-x}
\end{equation}

Let us solve this recurrence. We have that

\begin{align*}
	(1-x)P(x) &= (k-2) x(1-x) P(x)  + 2(k-1) x^2 P(x)+ 1 \\
	1 &=(1-x)P(x) - (k-2)x(1-x) P(x) - 2(k-1) x^2 P(x)\\
	P(x) &= \frac{1}{1-x - (k-2)x (1-x) - 2 (k-1) x^2}\\
	P(x) &= \frac{1}{(x + 1)(1 - kx)}
\end{align*}
Using the rule of partial fractions, we have that:
\begin{equation}
	P(x) = \frac{A}{(x + 1)} + \frac{B}{1-kx}
\end{equation}
where $A = \frac{1}{k + 1}$ and $B = \frac{k}{k+1}$.
To calculate the coefficients, we have that:
\begin{align*}
	\frac{1}{1 + x} &= \sum_{n \geq 0} (-1)^n x^n\\
	\frac{1}{1-kx} &= \sum_{n \geq 0} k^n x^n
\end{align*}
Therefore, 
\begin{equation}
	[x^n]P(x) = \frac{k^{n + 1} + (-1)^n}{k + 1}
\end{equation}
When $k = 2$, these are the Jacobsthal numbers. 
\newpage
\section{Question 4}
\subsection{Part a}
Consider any partition of $[n]$, and consider the partition where $n$ belongs. We have that if $n$ is in a single partition, we remove $\lbrace n \rbrace$ to have a partition of $[n-1]$. If $n$ is paired with another number, we have $n-1$ choices of what $n$ can be paired with, so we have $(n-1) p_{n-2}$ other pairings. Therefore,
\begin{equation}
	p_n = p_{n-1} + (n-1) p_{n-2}.
\end{equation}
or that:
\begin{equation}\label{eqn:recurrence}
	p_{n + 2} = p_{n+1} + (n+1) p_{n}.
\end{equation}
where $p_0 = p_1 = 1$. 

\subsection{Part b}
Consider the number of ways to partition $[n]$ into sets of size exactly one. There is exactly one way to partition these sets, and the exponential generating function is $e^x$.
Now consider the number of perfect matchings of $K_{2n}$. We have that there are $\frac{(2n)!}{2^n n!}$ by building a bijection to $S_{2n}$ and the number of permutations with cycle structure $(2,2, ..., 2)$. 

Therefore, the generating function for partitioning $[n]$ into sets of size exactly two is
\begin{align*}
	G(x) &= \sum_{n \geq 0} \left(\frac{(2n)!}{2^n (n!)(2n)!}\right) x^{2n}\\
	G(x) &= \sum_{n \geq 0} \left(\frac{1}{2^n (n!)}\right) x^{2n}
\end{align*}
However, consider $e^x$. It has the power series
\begin{equation}
	e^x = \sum_{n \geq 0} \left(\frac{1} {n!}\right) x^{n}
\end{equation}
Mapping $x \mapsto x/2$ into the equation yields:
\begin{equation}
	e^{\frac{x}{2}} = \sum_{n \geq 0} \left(\frac{1} {2^n n!}\right) x^{n}
\end{equation}
Finally, mapping $x \mapsto x^2$ yields
\begin{equation}
	e^{\frac{x^2}{2}} = \sum_{n \geq 0} \left(\frac{1} {2^n n!}\right) x^{2n}
\end{equation}
So $G(x) = e^{\frac{x^2}{2}}$. 
\subsubsection{Calculating partitions of $[n]$ into sizes of one and two}
Let $\lbrace a_n \rbrace_{n \geq 0}$ be the set of ways to partition $[n]$ into subsequences of sizes exactly one, and $\lbrace b_n \rbrace_{n \geq 0}$ be the set of ways to partition $[n]$ into subsequences of sizes exactly two, where $e^x \egf \lbrace a_n \rbrace_{n \geq 0}$ and $G(x) \egf \lbrace b_n \rbrace_{n \geq 0}$. 
Consider any partition of $[n]$ of sizes one and two, where there are $k$ numbers in a size of set $1$ and $\ell$ numbers in a size of set two. The number of ways to partition $S$ is exactly $a_k$ and the number of ways to partition $Q$ is $b_{\ell}$. There are $\binom{n}{k}$ ways to choose the elements in the first set. Therefore, we have that $p_n = \sum_{r = 0}^{n} \binom{n}{r} a_r b_{n-r}$. Therefore 
\begin{equation}
	P(x) = e^x G(x) = e^{x + \frac{x^2}{2}}.
\end{equation}

\subsection*{Part c}
To show that it satisfies the recurrence above, we use rules 2' and 3' and \cref{eqn:recurrence}, to yield:
\begin{equation}
	P''(x) = P'(x) + xP'(x) + P(x)
\end{equation}
We have that:
\begin{align*}
	P(x) &= e^{x + \frac{x^2}{2}}\\
	P'(x) &= (1 + x)e^{x + \frac{x^2}{2}}\\
	P''(x) &= e^{x + \frac{x^2}{2}} + (1 + x)^2 e^{x + \frac{x^2}{2}}
\end{align*}
We have that:
\begin{align*}
	P'(x) + x P'(x) + P(x) &= e^{x + \frac{x^2}{2}} \left(1 + x + x + x^2 + 1\right)\\
	&= e^{x + \frac{x^2}{2}} \left(1 + (1 + x)^2 \right)\\
	&= P''(x)
\end{align*}
Therefore, they satisfy the recurrence \cref{eqn:recurrence}. Therefore, this equation does count the number of involutions correctly.

To finish this question, let us calculate the coefficients of $P(x)$. 
\begin{equation}
	[x^n/n!] P(x) = \sum_{r = 0}^{n} \binom{n}{r} \frac{1}{2^r}\delta_r^{\mod 2}
\end{equation}
where $\delta_r^{\mod 2} = 0$ if $r$ is odd and $1$ if $r$ is even. 
Therefore, 
\begin{equation}
	[x^n/n!] P(x) = \sum_{r = 0}^{\lfloor \frac{n}{2} \rfloor} \frac{n!}{2^r r!(n - 2r)!}
\end{equation}
These count the number of involutions.
\newpage
\section{Question 5'}
We have that:
\begin{equation}
	e^x = \sum_{n \geq 0} \frac{1}{n!}x^n.
\end{equation}
Now consider:
\begin{equation}
	\frac{1}{4} \left(e^x + e^{-x} + e^{ix} + e^{-ix} \right) = \sum_{n \geq 0} \frac{1}{n!} \cdot \frac{1}{4} \left(x^n + (-1)^n x^n + (i)^n x^n + (-i)^n x^n\right).
\end{equation}
\begin{itemize}
	\item $n \equiv 0 \mod 4$, then the coefficient of $x^n$ is $\frac{1}{n!}\frac{1}{4} \left(1 + 1 + 1 + 1\right) = \frac{1}{n!}$. 
	\item $n \equiv 1 \mod 4$, then the coefficient of $x^n$ is $\frac{1}{n!}\frac{1}{4} \left(1 + -1 + i -i \right) = 0$. 
	\item $n \equiv 2 \mod 4$, then the coefficient of $x^n$ is $\frac{1}{n!}\frac{1}{4} \left(1 + 1 + -1 -1 \right) = 0$. 
	\item $n \equiv 3 \mod 4$, then the coefficient of $x^n$ is $\frac{1}{n!}\frac{1}{4} \left(1 + -1 -i + i \right) = 0$. 
\end{itemize}
Therefore, the only terms that remain are $n \equiv 0 \mod 4$. Therefore, we have that:

\begin{equation}
	\frac{1}{4} \left(e^x + e^{-x} + e^{ix} + e^{-ix} \right) = \sum_{n \geq 0} \frac{1}{(4n)!} x^{4n}.
\end{equation}
Now consider taking the integral of both sides.
\begin{align*}
	\int \frac{1}{4} \left(e^x + e^{-x} + e^{ix} + e^{-ix} \right) \,dx &= \int \sum_{n \geq 0} \frac{1}{(4n)!} x^{4n} \,dx\\
	\frac{1}{4} \left(e^x - e^{-x} -i e^{ix} + i e^{-ix} \right) + C &= \sum_{n \geq 0} \frac{1}{(4n+1)!} x^{4n + 1}\\
\end{align*}
We want that when we have that $x = 0$, then both sides equal 0.
\begin{align*}
	\frac{1}{4} \left(e^{(0)} - e^{-(0)} -i e^{i(0)} + i e^{-i(0)} \right) + C &= 0\\
	\frac{1}{4} \left(1 - 1 - i + i \right) + C &= 0\\
	C &= 0
\end{align*}
Therefore, we have that the generating function $f \ops \left\lbrace 0, \frac{1}{1!}, 0, 0, 0, \frac{1}{5!}, 0, 0, 0, \frac{1}{9!}, ...\right\rbrace$ is
\begin{equation}
	f(x) = \frac{1}{4} \left(e^x - e^{-x} -i e^{ix} + i e^{-ix} \right) 
\end{equation}
Therefore, we use the rules of starting the sequence at 0 by multiplying by $\frac{1}{x}$ and using the map $x \mapsto \sqrt[4]{x}$ to remove the zeroes. So we can let
\begin{equation}
	g(x) = \frac{1}{x} f(x)  \circ (\sqrt[4]{x}) \ops \left\lbrace \frac{1}{(4n + 1)!} \right\rbrace_{n \geq 0}. 
\end{equation}
where:
\begin{equation}
	g(x) = \frac{1}{4\sqrt[4]{x}} \left(e^{\sqrt[4]{x}} - e^{- \sqrt[4]{x}} -i e^{i \sqrt[4]{x}} + i e^{-i \sqrt[4]{x}} \right) 
\end{equation}
Now we use the polynomial rule on $g(x)$. Let $P(n) = (2 n + 1)^2 = 4n^2 + 4n + 1$. As $	g(x) \ops \left\lbrace \frac{1}{(4n + 1)!} \right\rbrace_{n \geq 0}$, then $P(xD) g(x) \ops \left\lbrace \frac{P(n)}{(4n + 1)!} \right\rbrace_{n \geq 0}$. Therefore, we have that:

\begin{equation}
	(4 (xD)^2 + 4(xD) + 1) g(x) \ops \left\lbrace \frac{(2n + 1)^2}{(4n + 1)!} \right\rbrace_{n \geq 0}.
\end{equation}
Let us calculate $(4 (xD)^2 + 4(xD) + 1) g(x)$. We have that $(xD)^2 g(x) = (xD)(xD) g(x) = (xD) (x Dg(x)) = x Dg(x) + x^2 D^2 g(x)$. 
Therefore, we have that
\begin{equation}
	(4 (xD)^2 + 4(xD) + 1) g(x) = 4x^2 D^2 g(x) + 8 x (D g(x)) + g(x).
\end{equation}

Now evaluating this at $x = 1$, we have that
\begin{equation}
	4 (D^2 g(x))|_{x = 1} + 8 (D g(x))|_{x = 1} + g(1) = \sum_{n \geq 0}\frac{(2n + 1)^2}{(4n + 1)!}
\end{equation}
Using a computer to evaluate the functions at 1, we have that
\begin{align*}
	g(1) &= \frac{-1}{4e} + \frac{e}{4} + \frac{\sin(1)}{2}\\
	8D(g(x))|_{x = 1} &= \frac{1}{e} - \sin(1) + \cos(1)\\
	4 D^2(g(x))|_{x = 1} &= -\frac{11}{16e} + \frac{e}{16} + \frac{\sin(1)}{2} - \frac{5 \cos(1)}{8}
\end{align*}
Then we have that:

\begin{equation}
	4 (D^2 g(x))|_{x = 1} + 8 (D g(x))|_{x = 1} + g(1) = \frac{1}{16}e^{-1} + \frac{5e}{16} + \frac{3 \cos(1)}{8}\approx 1.075.
\end{equation}
\end{document}
