\documentclass[]{article}
\usepackage[margin=1in]{geometry}
\usepackage{amsmath}
\usepackage{amssymb}

\newcommand{\ops}{\overset{\text{ops}}{\leftrightarrow}}
\newcommand{\egf}{\overset{\text{egf}}{\leftrightarrow}}
%opening
\title{Assignment 1}
\author{Eric Luu}

\begin{document}

\maketitle
\section{Question 1}
\subsection{Part a}
We choose $n$ numbers from the set $[2n]$, label them $1$ and label the rest $0$. Therefore, we have that $b_n = \binom{2n}{n}$.
\subsection{Part b}
We have that:
\begin{align*}
	\frac{b_{n+1}}{b_n} &= \frac{\binom{2n + 2}{n+1}}{\binom{2n}{n}}\\
	&= 
	\frac{\frac{(2n + 2)!}{((n+1)!)^2}}{\frac{(2n!)}{(n!)^2}}\\
	&=
	\frac{(2n + 2)(2n + 1)}{(n+1)^2}\\
	&=\frac{2(2n + 1)}{n+1}\\
	&= 4 - \frac{2}{n+ 1}
\end{align*}
Therefore, $(n+1)b_{n+1} = 4(n+1) b_n - 2 b_n$. 
\subsection*{Part c}
Let \begin{equation}
	B(x) \ops \lbrace b_n \rbrace_{n \geq 0}. 
\end{equation}
We have that:
\begin{equation}
	D (B(x)) = \sum_{n \geq 1} (n) x^{n-1} = \sum_{n \geq 0} (n + 1) b_{n+1} x^n.
\end{equation}
We also have that \begin{equation}
	x D B(x) + B(x) = D (x B(x)) = \sum_{n \geq 0} (n + 1) b_n x^{n}. 
\end{equation}
Therefore, we have that:
\begin{align*}
	D(B(x)) &= 4 xD B(x) + 4 B(x) - 2 B(x)\\
	(1-4x) D(B(x))&= 2 B(x)\\
	\frac{B'(x)}{B(x)} &= \frac{2}{1-4x}\\
\end{align*}
\subsection{Part e}
Taking the integral on both sides, we have that:
\begin{align*}
	\log(B(x)) &= -1/2 \log(1-4x) + C\\
	B(x) &= \frac{A}{\sqrt{1-4x}}
\end{align*}
As we have that $B(0) = 1$, then $A = 1$. Therefore,
\begin{equation}
	B(x) = \frac{1}{\sqrt{1-4x}}
\end{equation}

\subsection{Part e}
We have that $(1-4x)^{-1/2} =\sum_{n \geq 0} \binom{-\frac{1}{2}}{n}(-1)^n (4)^n x^n$. Therefore, we have that $[x^n] B(x) = \binom{-\frac{1}{2}}{n}(-1)^n (4)^n$. But we have that:
\begin{align*}
	\binom{-\frac{1}{2}}{n}(-1)^n (4)^n &= \frac{(-\frac{1}{2})(-\frac{1}{2} - 1) ... (-\frac{1}{2} - n + 1)}{n!} (-1)^n (4)^n\\
	&= \frac{(\frac{1}{2})(\frac{1}{2} + 1) ... (\frac{1}{2} + n - 1)}{n!}(4)^n\\
	&= \frac{(1)(3)(5) ... (n-1)}{n!} 2^n\\
	&= \frac{(2n)!}{2^n (n!)^2}2^n\\
	&= \frac{(2n)!}{(n!)^2}\\
	&= \binom{2n}{n}
\end{align*}
\section{Question 4}
\subsection{Part a}
Consider any partition of $[n]$, and consider the partition where $n$ belongs. We have that if $n$ is in a single partition, we remove $\lbrace n \rbrace$ to have a partition of $[n-1]$. If $n$ is paired with another number, we have $n-1$ choices of what $n$ can be paired with, so we have $(n-1) p_{n-2}$ other pairings. Therefore,
\begin{equation}
	p_n = p_{n-1} + (n-1) p_{n-2}.
\end{equation}

\subsection{Part b}
\section{Question 5'}
We have that:
\begin{equation}
	e^x = \sum_{n \geq 0} \frac{1}{n!}x^n.
\end{equation}
Now consider:
\begin{equation}
	\frac{1}{4} \left(e^x + e^{-x} + e^{ix} + e^{-ix} \right) = \sum_{n \geq 0} \frac{1}{n!} \cdot \frac{1}{4} \left(x^n + (-1)^n x^n + (i)^n x^n + (-i)^n x^n\right).
\end{equation}
We have that if: 
\begin{itemize}
	\item $n \equiv 0 \mod 4$, then the coefficient of $x^n$ is $\frac{1}{n!}\frac{1}{4} \left(1 + 1 + 1 + 1\right) = \frac{1}{n!}$. 
	\item $n \equiv 1 \mod 4$, then the coefficient of $x^n$ is $\frac{1}{n!}\frac{1}{4} \left(1 + -1 + i -i \right) = 0$. 
	\item $n \equiv 2 \mod 4$, then the coefficient of $x^n$ is $\frac{1}{n!}\frac{1}{4} \left(1 + 1 + -1 -1 \right) = 0$. 
	\item $n \equiv 3 \mod 4$, then the coefficient of $x^n$ is $\frac{1}{n!}\frac{1}{4} \left(1 + -1 -i + i \right) = 0$. 
\end{itemize}
Therefore, the only terms that remain are $n \equiv 0 \mod 4$. Therefore, we have that:

\begin{equation}
	\frac{1}{4} \left(e^x + e^{-x} + e^{ix} + e^{-ix} \right) = \sum_{n \geq 0} \frac{1}{(4n)!} x^{4n}.
\end{equation}
Now consider taking the integral of both sides.
\begin{align*}
	\int \frac{1}{4} \left(e^x + e^{-x} + e^{ix} + e^{-ix} \right) \,dx &= \int \sum_{n \geq 0} \frac{1}{(4n)!} x^{4n} \,dx\\
	\frac{1}{4} \left(e^x - e^{-x} -i e^{ix} + i e^{-ix} \right) + C &= \sum_{n \geq 0} \frac{1}{(4n+1)!} x^{4n + 1}\\
\end{align*}
We want that when we have that $x = 0$, then both sides equal 0.
\begin{align*}
	\frac{1}{4} \left(e^{(0)} - e^{-(0)} -i e^{i(0)} + i e^{-i(0)} \right) + C &= 0\\
	\frac{1}{4} \left(1 - 1 - i + i \right) + C &= 0\\
	C &= 0
\end{align*}
Therefore, we have that the generating function $f \ops \left\lbrace 0, \frac{1}{1!}, 0, 0, 0, \frac{1}{5!}, 0, 0, 0, \frac{1}{9!}, ...\right\rbrace$ is
\begin{equation}
	f(x) = \frac{1}{4} \left(e^x - e^{-x} -i e^{ix} + i e^{-ix} \right) 
\end{equation}
Therefore, we use the rules of starting the sequence at 0 by multiplying by $\frac{1}{x}$ and using the map $x \mapsto \sqrt[4]{x}$ to remove the zeroes. So we can let
\begin{equation}
	g(x) = \frac{1}{x} f(x)  \circ (\sqrt[4]{x}) \ops \left\lbrace \frac{1}{(4n + 1)!} \right\rbrace_{n \geq 0}. 
\end{equation}
where:
\begin{equation}
	g(x) = \frac{1}{4\sqrt[4]{x}} \left(e^{\sqrt[4]{x}} - e^{- \sqrt[4]{x}} -i e^{i \sqrt[4]{x}} + i e^{-i \sqrt[4]{x}} \right) 
\end{equation}
Now we use the polynomial rule on $g(x)$. Let $P(n) = (2 n + 1)^2 = 4n^2 + 4n + 1$. As $	g(x) \ops \left\lbrace \frac{1}{(4n + 1)!} \right\rbrace_{n \geq 0}$, then $P(xD) g(x) \ops \left\lbrace \frac{P(n)}{(4n + 1)!} \right\rbrace_{n \geq 0}$. Therefore, we have that:

\begin{equation}
	(4 (xD)^2 + 4(xD) + 1) g(x) \ops \left\lbrace \frac{(2n + 1)^2}{(4n + 1)!} \right\rbrace_{n \geq 0}.
\end{equation}
Let us calculate $(4 (xD)^2 + 4(xD) + 1) g(x)$. We have that $(xD)^2 g(x) = (xD)(xD) g(x) = (xD) (x Dg(x)) = x Dg(x) + x^2 D^2 g(x)$. 
Therefore, we have that
\begin{equation}
	(4 (xD)^2 + 4(xD) + 1) g(x) = 4x^2 D^2 g(x) + 8 x (D g(x)) + g(x).
\end{equation}

Now evaluating this at $x = 1$, we have that
\begin{equation}
	4 (D^2 g(x))|_{x = 1} + 8 (D g(x))|_{x = 1} + g(1) = \sum_{n \geq 0}\frac{(2n + 1)^2}{(4n + 1)!}
\end{equation}
Using a computer to evaluate the functions at 1, we have that
\begin{align*}
	g(1) &= \frac{-1}{4e} + \frac{e}{4} + \frac{\sin(1)}{2}\\
	8D(g(x))|_{x = 1} &= \frac{1}{e} - \sin(1) + \cos(1)\\
	4 D^2(g(x))|_{x = 1} &= -\frac{11}{16e} + \frac{e}{16} + \frac{\sin(1)}{2} - \frac{5 \cos(1)}{8}
\end{align*}
Then we have that:

\begin{equation}
	4 (D^2 g(x))|_{x = 1} + 8 (D g(x))|_{x = 1} + g(1) = \frac{1}{16}e^{-1} + \frac{5e}{16} + \frac{3 \cos(1)}{8}\approx 1.075.
\end{equation}
\end{document}
