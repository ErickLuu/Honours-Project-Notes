\documentclass[]{article}
\usepackage[margin=1in]{geometry}

\usepackage{amsmath}
\usepackage{amssymb}
\usepackage{amsthm}
\usepackage{url}

% Environments

\newtheorem{theorem}{Theorem}
\newtheorem{proposition}[theorem]{Proposition}
\newtheorem{corollary}[theorem]{Corollary}
\newtheorem{lemma}[theorem]{Lemma}
\newtheorem{definition}[theorem]{Definition}
\newtheorem{conjecture}[theorem]{Conjecture}

\theoremstyle{definition}
\newtheorem{example}[theorem]{Example}

\numberwithin{theorem}{section}
\numberwithin{equation}{section}

\newcommand{\ops}{\overset{\text{ops}}{\leftrightarrow}}

%opening
\title{Combinatorics Notes}
\author{Eric Luu}

\begin{document}

\maketitle
\section{Generating functions}
We have that formal power series form a ring of the form:
\begin{equation}
	a_0 + a_1x + a_2 x^2 + ...
\end{equation}
where $\lbrace a_n \rbrace_0^\infty$ is the sequence of coefficients. Two sequences are equal if they have the same coefficients. Convergence does not matter. A polynomial is one where after a point all terms are zero.

We say that $f \ops \lbrace a_n \rbrace_{n \geq 0} $ is the \textit{ordinary power series} generating function of the sequence $\lbrace a_n \rbrace_{n \geq 0}$. We have that $[x^k] \sum_{n \geq 0} a_n x^n = a_k$, where $[x^k]$ extracts out the $k$-th coefficient.

So $[x^n]1/(1-x) = 1$, $[x^n] e^x = 1/n!$, $[x^n](x^a f) = [x^{n-a}] f$. More generally, $[cx^n] f = 1/c [x^n] f$.
\subsection{Operations on power series}
We have that addition and subtraction is defined in the usual way. We also have that:
$f = \sum_{n \geq 0} a_n x^n, g = \sum_{n \geq 0} b_n x^n$, then $fg = \sum_{n \geq 0} \sum_{k=0}^{n} a_k b^{n-k} a_k$.  
\subsubsection{Inverses}
We have that $(1-x)(\sum_{n \geq 0}x^n) = 1$, thus
\begin{equation}
	\frac{1}{1-x} = \sum_{n \geq 0} x^n.
\end{equation}
\begin{theorem}
	A power series has an inverse iff its constant term is nonzero. 
\end{theorem}
\subsubsection{Composition of formal series}
We have that $f(g(x))$ is well-defined iff $g(0) = 0$ or $f$ is a polynomial. 

\subsubsection{Derivatives}
If $f = \sum_{n \geq 0} a_n x^n$, then $Df = \sum_{n \geq 1} na_nx^{n-1}$. 
Additionally, $xD f = \sum_{n \geq 1} na_nx^n$.
\subsection{Recurrences}
We have that $F_0 = 0$, $F_1 = 1$, $F_{n+2} = F_{n+1} + F_n$. 
Let $f \overset{\ops}{\leftrightarrow} \lbrace F_n \rbrace_{n \geq 0} $ be its generating function. Then $\sum_{n \geq 0}F_{n+2} x^n = \sum_{n \geq 0} F_{n+1} x^n + \sum_{n \geq 0}F_n x^n$.
But that implies that:
\begin{align*}
	1/x^2 \sum_{n \geq 0}F_{n+2} x^{n+2} &= 1/x\sum_{n \geq 0} F_{n+1} x^{n+1} + \sum_{n \geq 0}F_n x^n\\
	1/x^2(f(x) - F_0 - F_1 x) &= 1/x(f(x) - F_0) + f(x)\\
	1/x^2(f(x) - x) &= 1/x f(x) + f(x)\\
	f(x) &= \frac{x}{1-x-x^2}.
\end{align*}
Let $r_\pm = (1 \pm \sqrt{5})/2$. Then:
\begin{align*}
	\frac{x}{1-x-x^2} &= \frac{x}{(1-xr_+)(1-xr_-)}\\
	&= \frac{1}{r_+ - r_-}\left(\frac{1}{1 - xr_+}- \frac{1}{1-x r_-}\right)\\
	&= \frac{1}{\sqrt{5}} \left(\sum_{n \geq 0}r_+^n x^n -\sum_{n \geq 0}r_-^n x^n  \right)
\end{align*}
\subsection{Linear recurrences}
A sequence $\lbrace a_n \rbrace_{n \geq 0}$ satisfies a linear recurrence:
\begin{equation}
	a_n = c_1 a_{n-1} + c_2 a_{n-2} + c_3a_{n-3} + ... + c_ka_{n-k}
\end{equation}
iff $A(x)$ is a rational function, where $A \ops \lbrace a_n \rbrace_{n \geq 0}$.

\subsection{Multivariate generating functions}
$\binom{n}{k}$ has two parameters, so consider $B \ops {\binom{n}{k}}_{n, k\geq 0}$, $B(x) = \sum_{n,k \geq 0}\binom{n}{k} x^n y^k$ and $\binom{0}{0} = 1$. 
We have that \begin{align*}
	\sum_{\substack{
		n,k \geq 0\\
		(n,k) \neq (0,0)
}}
\binom{n}{k} x^n y^k &= \sum \binom{n}{k}x^n y^k + \sum \binom{n-1}{k} x^n y^k\\
B(x,y) - 1 &= xyB(x,y) + xB(x,y)\\
\end{align*}
Therefore,
\begin{equation}
	B(x,y) = \frac{1}{1-xy-x}.
\end{equation}
So we have that 
\begin{equation}
	[x^n]B(x,y) = [x^n]\frac{1}{1-(1+y) x} = (1 + y)^n.
\end{equation}
This makes sense as $(1 + y)^n$ is a one variable generating function for the binomial coefficients.

We also have that
\begin{equation}
	\frac{1}{1-xy-x} = \frac{1}{1-x} \frac{1}{1 - \frac{x}{1-x} y} = \frac{1}{1-x} \sum_{k \geq 0}\left(\frac{x}{1-x}\right)^k y^k
\end{equation}
So \begin{equation}
	[y^k] B(x,y) = \frac{x^k}{(1-x)^{k+1}}
\end{equation}
And:
\begin{align*}
	[x^n] \frac{x^k}{(1-x)^{k+1}} &= [x^{n-k}]\frac{1}{(1-x)^{k+1}}\\
	&= [x^{n-k}]\frac{1}{(n-k)!}\left( \left(\dfrac{d}{dx}\right)^{n-k} \frac{1}{(1-x)^{k+1}}  \right)\\
	&= \frac{n!}{(n-k) k!} = \binom{n}{k}
\end{align*}
\end{document}
