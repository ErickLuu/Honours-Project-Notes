\documentclass[]{article}
\usepackage[margin=1in]{geometry}
\usepackage{amsmath}
\usepackage{amssymb}

\usepackage{hyperref}
\usepackage{cleveref}

\newcommand{\ops}{\overset{\text{ops}}{\leftrightarrow}}
\newcommand{\egf}{\overset{\text{egf}}{\leftrightarrow}}
%opening
\title{Assignment 3}
\author{Eric Luu}

\begin{document}

\maketitle
\section{Question 1}
\subsection{Part a}
We use the Taylor expansion of $\log(1 - x) = -x - \frac{x^2}{2} + O(x^3)$ with $\log(1 - n^{-1}) = -n^{-1} - \frac{n^{-2}}{2}+ O(n^{-3})$. We then have that:

\begin{align*}
	f(n) &= (n^2 + 7)\log(1 - n^{-1}) - n\\
	&= (n^2 + 7)\left(-n^{-1} - \frac{n^{-2}}{2} + O(n^{-3})\right) - n\\
	&= -n -\frac{1}{2} + O(n^{-1})- 7n^{-1} - \frac{7}{2}n^{-2} + O(n^{-3}) - n\\
	&= -2n -\frac{1}{2} + O(n^{-1})
\end{align*}
Therefore, we have that $f(n) = -2n -\frac{1}{2} + o(1)$, therefore $f(n) \sim -2n -\frac{1}{2}$. 
\subsection{Part b}
We use the same result above. We have that:

\begin{align*}
	f(n) &= (n^2 + 7)\log(1 - n^{-1}) + n - \frac{1}{2}\\
	&= (n^2 + 7)\left(-n^{-1} - \frac{n^{-2}}{2} + O(n^{-3})\right) + n - \frac{1}{2}\\
	&= -n -\frac{1}{2} + O(n^{-1})- 7n^{-1} - \frac{7}{2}n^{-2} + O(n^{-3}) + n - \frac{1}{2}\\
	&= -1 + O(n^{-1})
\end{align*}
Then we have that $f(n) \sim -1$. 

\section{Question 2}
\subsection{Part a}
We have that $f(n) = n^{n^2}(1 + \frac{1}{n})^{n^2}$. We have that $(1 + \frac{1}{n})^{n^2} = \exp\left(n^2 \log(1 + \frac{1}{n})\right)$ and we use the fact that $\log(1 + x) = x - \frac{1}{2}x^2 + O(x^3)$ when $x = o(1)$ to write $\log(1 + \frac{1}{n}) = \frac{1}{n} - \frac{1}{2}x^{-2} + O(n^{-3})$. Then we have that:
\begin{align*}
	\exp\left(n^2 \log(1 + \frac{1}{n})\right) &= \exp\left(n^2 \left(\frac{1}{n} - \frac{1}{2}x^{-2} + O(n^{-3}) \right)\right)\\
	&=  \exp\left( n  - \frac{1}{2}+ O(n^{-1})\right)\\
	&= \exp\left( n  - \frac{1}{2}+ o(1)\right)
\end{align*}
Therefore, we can write $f(n)$ as:
\begin{equation}
	f(n) = n^{n^2}e^{n - \frac{1}{2}}e^{o(1)}
\end{equation}
so $f(n) \sim n^{n^2}e^{n - \frac{1}{2}}$. 
Therefore, we have that $g(n) = n^{A(n)}e^{B(n)}$ where $A(n) = n^2$ and $B(n) = n - \frac{1}{2}$. 

\subsection{Part b}
We have that $f(n) - g(n) = (n + 1)^{n^2} - n^{n^2}e^{n - \frac{1}{2}} = n^{n^2}(1 + \frac{1}{n})^{n^2} - n^{n^2}e^{n - \frac{1}{2}}$. Factoring out the $n^{n^2}$ term, we have that $f(n) - g(n) = n^{n^2}\left((1 + \frac{1}{n})^{n^2} - e^{n - \frac{1}{2}} \right)$.

We can write $(1 + \frac{1}{n})^{n^2} = e^{\log(1 + \frac{1}{n})^{n^2})}$. Write $\log(1 + \frac{1}{n}) = \frac{1}{n} - \frac{1}{2 n^2} + \frac{1}{3 n^3} + O(\frac{1}{n^4})$. Then we can say:
\begin{equation}
	(1 + \frac{1}{n})^{n^2} = \exp(n - \frac{1}{2} + \frac{1}{3n} + O(\frac{1}{n^2}))
\end{equation}
So we have that:
\begin{equation}
	f(n) - g(n) = n^{n^2}e^{n - \frac{1}{2}}(e^{\frac{1}{3n} - O(\frac{1}{n^2})} - 1)
\end{equation}
We then use the Taylor series approximation as $e^{\frac{1}{3n} - O(\frac{1}{n^2})} = 1 + \frac{1}{3n} + O(\frac{1}{n^2})$.
Therefore, we get that:
\begin{equation}
	f(n) - g(n) = n^{n^2}e^{n - \frac{1}{2}}(1 + \frac{1}{3n} + O(\frac{1}{n^2}) - 1)
\end{equation}
Then we simplify to get that:
\begin{equation}
	f(n) - g(n) = \frac{1}{3} n^{n^2 - 1}e^{n - \frac{1}{2}} + O(n^{n^2 - 2}e^{n - \frac{1}{2}}) = \frac{1}{3} n^{n^2 - 1}e^{n - \frac{1}{2}}( 1 + O(\frac{1}{n})) \sim \frac{1}{3} n^{n^2 - 1}e^{n - \frac{1}{2}}. 
\end{equation}

\section{Question 3}
We can write the LHS as $\exp(f(n) \log(1 + o(1)))$. However, we have that $\log(1 + o(1)) = o(1)$, therefore we want that $\exp(f(n) o(1)) \sim 1$. Therefore, we want $f(n) o(1) = o(1)$, meaning that if we have $(1 + g(n))^{f(n)} \sim 1$, then $f(n)g(n) \sim 0$.  

\section{4}
\subsection{a}
For fixed $n$ and $k$ large, we claim that $\frac{n! n^k}{(n + k)! 2^k} = \frac{1}{2^k}  + o(1)$. Let $a_k = \frac{n! n^k}{(n + k)! 2^k}$ and $b_k = \frac{1}{2^k}$. We claim that $c_k \frac{a_k}{b_k} = \frac{n! n^k}{(n + k)!}$ is bounded. We have that $c_0 = 1$ and that $c_{k + 1}/c_k = \frac{n}{n + k + 1}$ which is a decreasing function for fixed $n$ and taking $k$ to infinity. In fact, it decreases to 0. Therefore, we claim that $c_k = 1 + o(1)$, so $a_k = b_k g(k)$ with an above bound of $g(k) = \prod_{i = 1}^k \frac{n}{n + i + 1}$ for all $k$.

However, how big is $g(k)$ asymptotically? We can rewrite the fraction as $\frac{n/i}{n/i + 1/i + 1}$ where the bottom goes to 1 asymptotically, so we can rewrite it as $\frac{n}{i}(1 + 1/i + n/i + O(n^2/i^2))$. Simplifying, we get that this is $n/i + O(n^2/i^2)$. Therefore, the product of $g(k) =  \prod_{i = 1}^k n/i + O(n^2/i^2)$ which goes to 0 very quickly. 

 Thus, $ \sum_{k = 0}^{J} \frac{n! n^k}{(n + k)! 2^k}  \sim 1 + \frac{2^{J} - 1}{2^{J + 1}}$. 
 
 \subsection{b}
 We have that the largest value of deviation is for the first few terms of $n$ and asymtotically we approach $\frac{1}{2^k}$ when $k$ is much larger than $n$. Therefore, we have that:
 $ \sum_{k = 0}^{\infty} \frac{n! n^k}{(n + k)! 2^k}  = 2 + o(1)$. 
\section{5}
\subsection{a}
We can choose $(n - 1)(n - 2) ... (n-k + 1)$ ways to choose $k - 1$ vertices on the tail, which is denoted as $[n - 1]_{k - 1}$. Then we have that the number of trees that do not contain vertices in the  $(n - k)^{n - k - 2}$. We can finally choose which vertex to attach the tail to, where we have $(n - k)$ tails. Therefore, we have that the number of tails is:
\begin{equation}
	[n - 1]_{k - 1} (n - k)^{n - k - 1}
\end{equation}

\subsection{b}
We wish to show $[n - 1]_{k - 1}(n - k)^{n - k - 1} = (n - k)^{n - 2}( 1 + o(1))$, so we want to show that $[n - 1]_{k - 1} \sim (n - k)^{k - 1}$.

We have that 
\begin{equation}
	n - i = (n - k) ( 1 + \frac{k - i}{n - k})
\end{equation}
and as $k$ is of $\sqrt n$, we have that $\frac{k - i}{n - k} = \frac{k/n - i/n}{1 - k/n} = (k/n - i/n)( 1 + k/n + O(k^2/n^2)) = 2k/n - i/n + O(k^2/n^2)$. Therefore, the error is bounded by $k/n + O(k^2/n^2)$. Therefore, we have that $[n - 1]_{k - 1} = \exp(O(k^2/n)) \prod_{i = 1}^{k - 1}(n - k)$. But as $k = o(\sqrt{n})$, then $k^2 = o(n)$ so $\exp(O(k^2/n))  = \exp(o(1)) = 1$. Therefore, $[n - 1]_{k - 1}(n - k)^{n - k - 1}  \sim (n - k)^{n - 2}$. Thus shown. 

\subsection{c}
We can write $\frac{(n - k)^{n - 2}}{n^{n - 2}} = (1 - \frac{k}{n})^{n - 2}$. Then we can write $(1 - \frac{k}{n})^{n - 2} = \exp(\log((1 - \frac{k}{n})^{n - 2})) = \exp((n - 2) \log( 1 - \frac{k}{n}))$. We can write $\log( 1 - \frac{k}{n}) = - \frac{k}{n} + O(\frac{k^2}{n^2})$, so $(n -2)\log( 1 - \frac{k}{n}) = - k - \frac{2k}{n} + O(\frac{k^2}{n})$. Therefore, we have that $\exp((n - 2) \log( 1 - \frac{k}{n})) = e^{-k + o(1)}$, therefore we have $e^{-k}$. 

\subsection{d}

We have that we have $[n - 1]_{2k - 2}$ ways to choose the tails of $1$ and $2$. Then we have that there are $(n- 2k)^{n - 2k - 2}$ ways to choose the tree to attach the tails to. Then we have $(n - 2k)^2$ vertices that we can attach $1 $ and $2$ to. Then we have that the number of graphs with $1$ and $2$ as ends of $k$-tails is:
\begin{equation}
	[n - 1]_{2k - 2} (n - 2k)^{n - 2k - 2}(n - 2k)^2 = [n - 1]_{2k - 2} (n - 2k)^{n - 2k}
\end{equation}
Asymptotically, we have that $[n - 1]_{2k - 2} \sim (n - 2k)^{2k - 2}$. Therefore, we have that the number of trees with $1$ and $2$ on the ends of $k$-tails is:
\begin{equation}
	(n - 2k)^{n - 2}
\end{equation}

Therefore, we have that the proportion of trees with $k$ tails is:
\begin{equation}
	\frac{(n - 2k)^{n - 2}}{n^{n - 2}} = e^{-2k}
\end{equation}

We have the proportion is $e^{-2k}$. 

\subsection{e}
We will use PIE. Let $A$ be the set of all labelled trees on $[n]$ vertices. $|A| = n^{n - 2}$. Let $P_i$ be the property that vertex $i$ is at the end of a $k$-tail, and let $S$ be the set of all properties. Let $Y \subseteq S$ where $|Y| = \ell$ for some fixed $\ell = o(\sqrt{n}) $. Then the proportion of these trees is $e^{-\ell k}$. The number of each $Y$ with $\ell$ properties is $\binom{n}{\ell} = \frac{[n]_{n - \ell}}{\ell!} \sim \frac{(n - \ell)^{n - \ell}}{\ell!}$. Then we can say that from the Poisson Paradigm that:
\begin{equation}
	\frac{1}{N}\sum_{|Y| = \ell}N_\geq(Y) = \frac{(n - \ell)^{(n - \ell)}}{\ell!} e^{-\ell k}.
\end{equation}
We then write $(n - \ell)^{n - \ell} = e^{(n - \ell) \log(n - \ell)}$ to get that 

\begin{equation}
	\frac{1}{N}\sum_{|Y| = \ell}N_\geq(Y) = \frac{(e^{ n/\ell \log(n - \ell) -\log(n - \ell) - k})^\ell}{\ell!}.
\end{equation}
Using the Poisson paradigm, we have that $\lambda = e^{ n/\ell \log(n - \ell) -\log(n - \ell) - k}$. 
Then from the Poisson paradigm, we have that:

\begin{equation}
	\frac{1}{N}N_= (\emptyset) = e^{e^{ n/\ell \log(n - \ell) -\log(n - \ell) - k}} = \exp \left(e^{-k} (n - \ell)^{n/ \ell - 1}\right)
\end{equation}
Then substituting in a function of $\ell = o(\sqrt{n})$, perhaps $\log n$, then we have that the proportion is
\begin{equation}
	\frac{1}{N}N_= (\emptyset)= \exp \left(e^{-k} (n - \log n)^{n/ \log n - 1}\right)
\end{equation}

and the number of trees is:
\begin{equation}
	N_= (\emptyset)= \exp \left(e^{-k} (n - \log n)^{n/ \log n - 1}\right) n^{n - 2}
\end{equation}
\end{document}
