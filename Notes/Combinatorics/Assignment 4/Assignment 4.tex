\documentclass[]{article}
\usepackage[margin=1in]{geometry}
\usepackage{amsmath}
\usepackage{amssymb}

\usepackage{hyperref}
\usepackage{cleveref}

\newcommand{\ops}{\overset{\text{ops}}{\leftrightarrow}}
\newcommand{\egf}{\overset{\text{egf}}{\leftrightarrow}}
%opening
\title{Assignment 4}
\author{Eric Luu}

\begin{document}

\maketitle
\section{1}
\subsection{Finding maximum of asymptotic}
Firstly, let $a_k = \frac{(n + k)!}{n!(k!)^2}$. Then let us calculate 
\begin{equation}
	R_k = \frac{a_{k + 1}}{a_k} = \frac{n!(k!)^2}{(n + k)!}\frac{(n + k + 1)^2}{n! (k + 1)!} = \frac{n + k + 1}{(k + 1)^2}. 
\end{equation}
Then we have that $R_k = 1$ when $n + k + 1 = (k + 1)^2$. Solving for $k$, we have that $(k + 1)^2 - (k + 1) - n = 0$, therefore $k + 1 = \frac{1 + \sqrt{1 + 4n}}{2}$ by solving the quadratic. Therefore, $k \sim \sqrt{n}$, and around this the function falls off. 
\subsection{Calculating near maximum}
Let us calculate this equation around $\sqrt{n}$. 

We want to calculate:
\begin{equation}
	\sum_{k = \lfloor \sqrt{n} - M \rfloor }^{ \lceil \sqrt{n} + M \rceil}  \frac{(n + k)!}{n!(k!)^2}.
\end{equation}
Now let $y = k - \sqrt{n}$. We will re-index the equation to match what we have.

We use Stirling's approximation.
\begin{align*}
	(n + k)! &= \left(\frac{n + k}{e}\right)^{n + k}\sqrt{2\pi (n + k)}\\
	n! &= \left(\frac{n}{e}\right)^n \sqrt{2 \pi n}\\
	k! &= \left(\frac{k}{e}\right)^k \sqrt{2 \pi k}\\
\end{align*}
Substituting this into the equation, we have that:
\begin{align*}
	\sum_{k = \lfloor \sqrt{n} - M \rfloor }^{ \lceil \sqrt{n} + M \rceil}  \frac{(n + k)!}{n!(k!)^2}
	&=
	\sum_{k = \lfloor \sqrt{n} - M \rfloor }^{ \lceil \sqrt{n} + M \rceil} \frac{\left(\frac{n + k}{e}\right)^{n + k}\sqrt{2\pi (n + k)}}{\left(\frac{n}{e}\right)^n \sqrt{2 \pi n} \left(\frac{k}{e}\right)^{2k} (2 \pi k)}\\
	&=
	\sum_{k = \lfloor \sqrt{n} - M \rfloor }^{ \lceil \sqrt{n} + M \rceil} \frac{1}{e^k 2\pi} \frac{\left(n + k\right)^{n + k}\sqrt{(n + k)}}{\left(n\right)^n \sqrt{n} \left(k\right)^{2k}k }\\
\end{align*}

Now suppose $M = 

\section{2}
We have that $w \log (w + 1) = n$. 
From bounds, we have that when $w = 0$, the left hand side is $0$. When $w = n$, we have that the left hand side is $n \log(n + 1) > n$. Therefore as the function is increasing, $0 < w < n$. 

Rearrange so that we have that $w = \frac{n}{\log(w + 1)}$. Now suppose $w = \alpha n$ for some $\alpha$ in $O(1)$. Then we have that:
\begin{align*}
	w &= \frac{n}{\log(\alpha n + 1)}\\
	&= \frac{n}{\log(\alpha n) (1 + \frac{1}{\alpha n})}\\
	&=\frac{n}{\log(\alpha) + \log(n) - \log(1 + \frac{1}{\alpha n})}
\end{align*}
We use the fact that $\log(1 + x) = O(x)$ when $x = o(1)$, so we have that:

\begin{align*}
	w &=\frac{n}{\log(\alpha) + \log(n) - \log(1 + \frac{1}{\alpha n})}\\
	&= \frac{n}{
		\log(\alpha) + \log(n) - O(\frac{1}{n})
		}\\
	&= \frac{\frac{n}{\log n}}{
	1 + \log(\alpha)/\log(n) - O(\frac{1}{n \log n})
	}
\end{align*}
Recall that $\frac{1}{1 + x} = 1 + O(x)$ when $x = o(1)$. Then we have that:
\begin{align*}
	w &= \frac{n}{\log n} ( 1 + O(\frac{1}{\log n}) + O(\frac{1}{n \log n}))\\
	&= \frac{n}{\log n} + O(\frac{n}{\log^2 n})
\end{align*}

Now we bootstrap again.
\begin{align*}
	w &= \frac{n}{
	\log(
	\frac{n}{\log n}( 1 + O(\frac{1}{\log n}))
	+ 1
	)
	}
	\\
	&=
	\frac{n}{
	\log(\frac{n}{\log n}
	(1 + \frac{\log n}{n} + O(\frac{1}{\log n}))
	)
	}\\
	&=
	\frac{n}{
	\log(n) - \log \log n + \log(1 + \frac{\log n}{n} + O(\frac{1}{\log n}))
	}
\end{align*}
We have that $\log(1 + \frac{\log n}{n} + O(\frac{1}{\log n})) = O(\frac{\log n}{n}) + O(\frac{1}{\log n})$.
So we have that:
\begin{align*}
		w &=
	\frac{n}{
		\log(n) - \log \log n + O(\frac{\log n}{n}) + O(\frac{1}{\log n})
	}\\
	&=
	\frac{n/\log n}{
	1 - \frac{\log \log n}{\log n} + O(\frac{1}{n}) + O(\frac{1}{\log^2 n})
	}\\
	&=
	\frac{n}{\log n}( 1 + O(\frac{\log \log n}{\log n}) + O(\frac{1}{n}) + O(\frac{1}{\log^2 n}))
\end{align*}
\section{3}
\subsection{a}
We have the equation:
\begin{equation}
	k = n + \sqrt{n + k} \log k.
\end{equation}
Now notice that both sides of the equation are strictly increasing with fixed $n$ and $k > 1$. Now if $k = n$, we have that the LHS is $n$ and the RHS is $n + \sqrt{2n} \log n$, which means $RHS > LHS$. Now when $k = 2n$, we have the LHS is $2n$ and the RHS is $n + \sqrt{3n} \log n$ which for large $n$ is less than $2n$. Therefore, $LHS > RHS$. By continuity and the fact that both sides are nondecreasing, we have that $k$ has a solution and it is between $n$ and $2n$. 
\subsection{b}
Let $k = \alpha n$ where $\alpha = O(1)$ and $1 < \alpha < 2$. 
Then we have that 
$k = n + \sqrt{n + \alpha n} \log (\alpha n) = n + \sqrt{n} \sqrt{\alpha + 1} (\log(\alpha) + \log n)$. Then we distribute and use the fact that $\alpha = O(1)$ to get that
\begin{align*}
	k &= n + \sqrt{n} O(1) + \log n \sqrt{n} O(1)\\
	 &= n + O(\sqrt{n}) + O(\log n \sqrt{n})\\
	 &= n + O(\log n \sqrt n).
\end{align*}
However, we can write the right hand side as $k = n(1 + O(\frac{\log n}{\sqrt{n}}))$ which is $k = n(1 + o(1))$. So $k \sim n$. 
\subsection{c}
Let $z = k - n$. From part $a$, we have that $0 < k < n$ and $z =  O(\log n \sqrt n)$. Then we rewrite the above equation in terms of $z$ to yield:
\begin{equation}
	z = \sqrt{z + 2n} \log (z + n).
\end{equation}
Now we plug in the approximation that $z = O(\log n \sqrt{n})$. We have that:

\begin{align*}
	z &= \sqrt{(O(\log n \sqrt{n}) + 2n)}\log(n + O(\log n \sqrt n))\\
	&= \sqrt{2n} \left(\sqrt{1 + O(\frac{\log n}{\sqrt{n}})}\right) \left(\log( n(1 + O(\frac{\log n}{\sqrt{n}} )) \right)\\
	&= \sqrt{2n}\left(\sqrt{1 + O(\frac{\log n}{\sqrt{n}})}\right) \log(n) + \sqrt{2n}\left(\sqrt{1 + O(\frac{\log n}{\sqrt{n}})}\right) \log((1 + O(\frac{\log n}{\sqrt{n}}))
	\end{align*}
Now we have that from doing a Taylor expansion that $\sqrt{1 + x} = 1 + O(x)$ when $x = o(1)$. We also have that $\log(1 + x) = O(x)$ when $x = o(1)$. Plugging this both into the equation, we have that:
\begin{align*}
	z &= \sqrt{2n}(1 + O(\frac{\log n}{\sqrt{n}})) \log(n) + \sqrt{2n}(1 + O(\frac{\log n}{\sqrt{n}}))O(\frac{\log n}{\sqrt{n}})\\
	&= \sqrt{2n} \log n + O(\log n) + O(\log n) + O(\frac{\log^2 n}{\sqrt{n}})\\
	&= \sqrt{2n} \log n + O(\log n)
\end{align*}
as we have that $\log^2(n) = o(\sqrt{n})$. Finally, we have that:
\begin{equation}
	z = \sqrt{2n} \log n ( 1 + O(\frac{1}{\sqrt{n}})) = \sqrt{2n} \log n (1 + o(1)).
\end{equation}
So $z \sim \sqrt{2n} \log n$.


\end{document}