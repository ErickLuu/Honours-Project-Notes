\documentclass[]{article}
\usepackage[margin=1in]{geometry}
\usepackage{amsmath}
\usepackage{amssymb}

\usepackage{hyperref}
\usepackage{cleveref}

\newcommand{\ops}{\overset{\text{ops}}{\leftrightarrow}}
\newcommand{\egf}{\overset{\text{egf}}{\leftrightarrow}}
%opening
\AddToHook{cmd/section/before}{\clearpage}
\title{Assignment 4}
\author{Eric Luu}

\begin{document}

\maketitle
\section{1}
\subsection{Finding maximum of asymptotic}
Firstly, let $a_k = \frac{(n + k)!}{n!(k!)^2}$. Then let us calculate 
\begin{equation}
	R_k = \frac{a_{k + 1}}{a_k} = \frac{n!(k!)^2}{(n + k)!}\frac{(n + k + 1)^2}{n! (k + 1)!} = \frac{n + k + 1}{(k + 1)^2}. 
\end{equation}
Then we have that $R_k = 1$ when $n + k + 1 = (k + 1)^2$. Solving for $k$, we have that $(k + 1)^2 - (k + 1) - n = 0$, therefore $k + 1 = \frac{1 + \sqrt{1 + 4n}}{2}$ by solving the quadratic.

This means that $k \sim \sqrt{n}$ multiplicatively. 

\subsection{Calculating near maximum}
Let us calculate this equation around $\sqrt{n}$. 

We want to calculate:
\begin{equation}
	\sum_{k = \lfloor \sqrt{n} - M \rfloor }^{ \lceil \sqrt{n} + M \rceil}  \frac{(n + k)!}{n!(k!)^2}.
\end{equation}

We use Stirling's approximation.
\begin{align*}
	(n + k)! &= \left(\frac{n + k}{e}\right)^{n + k}\sqrt{2\pi (n + k)}\\
	n! &= \left(\frac{n}{e}\right)^n \sqrt{2 \pi n}\\
	k! &= \left(\frac{k}{e}\right)^k \sqrt{2 \pi k}\\
\end{align*}
Substituting this into the equation, we have that:
\begin{align*}
	\sum_{k = \lfloor \sqrt{n} - M \rfloor }^{ \lceil \sqrt{n} + M \rceil}  \frac{(n + k)!}{n!(k!)^2}
	&=
	\sum_{k = \lfloor \sqrt{n} - M \rfloor }^{ \lceil \sqrt{n} + M \rceil} \frac{\left(\frac{n + k}{e}\right)^{n + k}\sqrt{2\pi (n + k)}}{\left(\frac{n}{e}\right)^n \sqrt{2 \pi n} \left(\frac{k}{e}\right)^{2k} (2 \pi k)}\\
	&=
	\sum_{k = \lfloor \sqrt{n} - M \rfloor }^{ \lceil \sqrt{n} + M \rceil} \frac{e^{k}}{2\pi} \frac{\left(n + k\right)^{n + k}\sqrt{(n + k)}}{\left(n\right)^n \sqrt{n} \left(k\right)^{2k}k }\\
\end{align*}

We have that as $k = \sqrt{n} + M = o(n^{2/3})$, we have that:
\begin{equation}
	\left(n + k\right)^{n + k} = n^{n + k} e^{k + \frac{k^2}{2n} + O(k^3/n^2)}
\end{equation}
which as $k = O(n^{1/3})$, the $O(k^3/n^2) = o(1)$. Therefore, we have that:

\begin{align*}
	\sum_{k = \lfloor \sqrt{n} - M \rfloor }^{ \lceil \sqrt{n} + M \rceil} \frac{e^{k}}{2\pi} \frac{\left(n + k\right)^{n + k}\sqrt{(n + k)}}{\left(n\right)^n \sqrt{n} \left(k\right)^{2k}k } &= 
	\sum_{k = \lfloor \sqrt{n} - M \rfloor }^{ \lceil \sqrt{n} + M \rceil} \frac{e^{k}}{2\pi} \frac{n^{n + k} e^{k + \frac{k^2}{2n}}\sqrt{(n + k)}}{\left(n\right)^n \sqrt{n} \left(k\right)^{2k}k }\\
	&=
	\sum_{k = \lfloor \sqrt{n} - M \rfloor }^{ \lceil \sqrt{n} + M \rceil}
	\frac{e^{k}}{2\pi}
	\frac{n^{k} e^{k + \frac{k^2}{2n}}\sqrt{(n + k)}}
	{\sqrt{n} \left(k\right)^{2k}k }\\
\end{align*}


We can note that $\sqrt{\frac{n + k}{n}}= \sqrt{1 + \frac{k}{n}} = 1 + O(k^2/n^2) = 1 + o(1)$. Therefore, this is negligible around $k = \sqrt{1/n}$. Then we have that:

\begin{equation*}
	\sum_{k = \lfloor \sqrt{n} - M \rfloor }^{ \lceil \sqrt{n} + M \rceil}
	\frac{e^{k}}{2\pi}
	\frac{n^{k} e^{k + \frac{k^2}{2n}}}
	{ \left(k\right)^{2k}k }\\
\end{equation*}
Now we let $y = k - \sqrt{n}$, or that $k = \sqrt{n} + y$. Then we have that:
\begin{equation*}
	=
	\sum_{y = \lfloor -M \rfloor }^{y =  \lceil + M \rceil}
	\frac{e^{\sqrt{n} + y}}{2\pi}
	\frac{n^{\sqrt{n} + y} e^{\sqrt{n} + y + \frac{(\sqrt{n} + y)^2}{2n}}}
	{ \left(\sqrt{n} + y\right)^{2(\sqrt{n} + y)}(\sqrt{n} + y) }\\
\end{equation*}
Now we have that if $y = o(n^{1/3})$, then we have that:
\begin{align*}
	\left(\sqrt{n} + y\right)^{(\sqrt{n} + y)} &= \sqrt{n}^{(\sqrt{n} + y)} \exp(\sqrt{n} + y) \log(1 + y/\sqrt{n})
\end{align*}
We have that:
\begin{equation}
	\log(1 + y/\sqrt{n}) = y/\sqrt{n} + y^2/(2n) + O(y^3/n^{3/2}),
\end{equation}
therefore, we have that:
\begin{align*}
	(\sqrt{n} + y) \log(1 + y/\sqrt{n}) &= (\sqrt{n} + y) ( y/\sqrt{n} + y^2/(2n) + O(y^3/n^{3/2}))\\
	&= y + + y^2/(2 \sqrt{n}) + y^2/\sqrt{n} + y^3/(2n) + O(y^3/n)\\
	&= y + 3y^2/(2\sqrt{n}) + o(1)
\end{align*}

So we have that $\left(\sqrt{n} + y\right)^{(\sqrt{n} + y)} = \sqrt{n}^{(\sqrt{n} + y)} \exp(y + 3y^2/(2 \sqrt{n}))$, so we have that $\left(\sqrt{n} + y\right)^{2(\sqrt{n} + y)} = n^{\sqrt{n} + y} \exp(2y + 3y^2/\sqrt{n})$. 
 So we have that:
\begin{align*}
	&\sum_{y = \lfloor -M \rfloor }^{y =  \lceil + M \rceil}
	\frac{e^{\sqrt{n} + y}}{2\pi}
	\frac{n^{\sqrt{n} + y} e^{\sqrt{n} + y + \frac{(\sqrt{n} + y)^2}{2n}}}
	{ n^{\sqrt{n} + y}\exp(2y + 3y^2/\sqrt{2n}) (\sqrt{n} + y) }\\
	&=
	\sum_{y = \lfloor -M \rfloor }^{y =  \lceil + M \rceil}
	\frac{e^{\sqrt{n} + y}}{2\pi}
	\frac{ e^{\sqrt{n} - y + \frac{(\sqrt{n} + y)^2}{2n} - \frac{3y^2}{\sqrt{n}}}}
	{(\sqrt{n} + y) }\\
	&=
	\sum_{y = \lfloor -M \rfloor }^{y =  \lceil + M \rceil}
	\frac{ e^{2\sqrt{n} + \frac{(\sqrt{n} + y)^2}{2n} - \frac{3y^2}{\sqrt{n}}}}
	{2 \pi(\sqrt{n} + y) }\\
\end{align*}
Expanding out, we get:

\begin{equation}
	\sum_{y = \lfloor -M \rfloor }^{y =  \lceil + M \rceil}
	\frac{\exp\left(2 \sqrt{n} + 1/2 + \frac{y}{\sqrt{n}} - 3y^2/\sqrt{n} + y^2/(2n)\right)}
	{2 \pi(\sqrt{n} + y) }\\
\end{equation}

Completing the square and simplifying, we get:
\begin{equation}
	\sum_{y = \lfloor -M \rfloor }^{y =  \lceil + M \rceil}
	\frac{\exp\left(
		2 \sqrt{n} + 1/2 + o(1) - (y + O(1))^2 (3/\sqrt{n} - 1/(2n))
		\right)}
	{2 \pi(\sqrt{n} + y) }\\
\end{equation}
We have that $1/(2 \pi (\sqrt{n} + y)) \sim 1/(2 \pi \sqrt{n})$ uniformly with $y = n^{3/10}$. 

\begin{align*}
	&=
	\frac{e^{2 \sqrt{n} + 1/2}}{2 \pi \sqrt{n}}
	\sum_{y = \lfloor -M \rfloor }^{y =  \lceil + M \rceil}
	e^{- (y + O(1))^2 (3/\sqrt{n} - 1/(2n))}
\end{align*}
If we have that $\alpha = (3/\sqrt{n} - 1/2n)$ then we have that $\alpha$ goes to 0, $M \sqrt{\alpha} = n^{3/10} n^{-1/4} = \infty$, so we have convergence. $\alpha^{-1} = \frac{2n}{6 \sqrt{n} - 1} \sim \sqrt{\frac{n}{\sqrt{3n}}}$.  Therefore, we have that:
\begin{equation}
	\sum_{y = \lfloor -M \rfloor }^{y =  \lceil + M \rceil}
	e^{-(y + O(1))^2 (3/\sqrt{n} - 1/(2n))}
	=
	\sqrt{\frac{2n \pi}{6 \sqrt{n} - 1}}
\end{equation}

and the final summation becomes:
\begin{equation}
	\frac{e^{2\sqrt{n} + 1/2}}{2 \pi \sqrt{n}} 
	\sqrt{\frac{2n \pi}{6 \sqrt{n}}}
	= 
	\frac{e^{2 \sqrt{n} + 1/2} }{n^{-1/4} \sqrt{12 \pi} }
\end{equation}


\subsection{Showing tails are negligible}
We break up the tail into the sections of $\sqrt{n} + n^{3/10}$ to $n$ and from $n$ to infinity.

When $M  = n^{3/10}$, consider $k = \sqrt{n} + M$. Then we have that
$R_k = (n + \sqrt{n}+ n^{3/10})/(\sqrt{n} + n^{3/10})^2$. Expanding this out, we get:
$R_k = (n + \sqrt{n} + n^{3/10})/(n + 2 n^{8/10} + n^{9/100})$, which approaches 1. Therefore, we have that the bound in this section $k = [\sqrt{n} + M, n]$ is bounded above by $n \times (n + k)!/n! (k!)^2$ which is of $o(S)$, where s is the sum. By symmetry, we have the same result around $k = \sqrt{n} - M$ as well. 
When $k = n$, then the term becomes $(2n)!/ (n!)^3$, and $R_k \leq 2/n$. Therefore,
we have that there is a geometric sequence with exponent $2/n$ and constant $(2n)!/(n!)^3$, meaning that we get that the tail is at most $n/(n - 2) (2n)!/(n!)^3 = o(S)$. 

\section{2}
Let $k = w + y$ for some $y = O(1)$. Then we have that:
\begin{equation}
	e^{k} k^{n - k} = e^{w + y} (w + y)^{n - w - y}
\end{equation}
Now we have that:
\begin{equation}
	(w + y)^{-w} = w^{-w} \left(1 + \frac{y}{w}\right)^{-w}
\end{equation}

But we have that $w \rightarrow \infty$ as $n \rightarrow \infty$, to be proven later. So we have that: 

\begin{equation}
	\left(1 + \frac{y}{w}\right)^{-w} = \left(\left(1 + \frac{y}{w}\right)^{w}\right)^{-1} \sim \left(e^{y}\right)^{-1} = e^{-y}
\end{equation}

Therefore, we have that:
\begin{equation}
	(w + y)^{-w} = w^{-w} e^{-y}.
\end{equation}

Now consider
\begin{equation}
	(w + y)^{n - y}.
\end{equation}
We first take out $w$ to get:
\begin{equation}
	(w + y)^{n - y} = w^{n - y} \left(1 + \frac{y}{w}\right)^{n - y}
\end{equation}
We have that $n = w \log (w + 1)$. We then substitute this into the equation above to get:
\begin{equation}
	\left(1 + \frac{y}{w}\right)^{n - y} = \left(1 + \frac{y}{w}\right)^{w \log(w + 1) - y}.
\end{equation}
Now we have that this is equal to:
\begin{equation}
	\exp((w \log(w + 1) - y) \log\left(1 + \frac{y}{w}\right)).
\end{equation}
But we have that $\log\left(1 + \frac{y}{w}\right) = \frac{y}{w} + O(y^2/w^2)$. Substituting this into the equation above, and using the fact that $y = O(1)$, we get that:
\begin{align*}
	&(w \log(w + 1) - y)(\frac{y}{w} + O(y^2/w^2))\\ 
	&= y \log(w + 1) - O(1/w) + O(y \log w/w) + O(y^3/w^2)\\ 
	&=  y \log(w + 1) + O(\log w / w).
\end{align*}
Therefore, we have that:
$\left(1 + \frac{y}{w}\right)^{n - y} = \exp(y \log(w + 1) + o(1)) = (w + 1)^y$. 
Finally, we have that $(w + 1)^y = w^y (1 + 1/w)^y \sim w^y$ as $(1 + 1/w)^y$ is bounded above and goes to 1 very quickly. 
Therefore, we have that:

\begin{align*}
	e^k k^{n - k} &= e^{w + y}(w + y)^{n - w - y}\\
	&=e^w e^y (w + y)^{-w} (w + y)^{n - y}\\
	&= e^w e^y w^{-w} \left(1 + \frac{y}{w}\right)^{- w}  w^{n - y}   \left(1 + \frac{y}{w}\right)^{n - y}\\
	&\sim e^w e^y w^{-w} e^{-y}  w^{n - y} w^y\\
	&= e^w w^{n - w}
\end{align*}
which is what we wanted. 
\section{3}
\subsection{a}
We have the equation:
\begin{equation}
	k = n + \sqrt{n + k} \log k.
\end{equation}
Now notice that both sides of the equation are strictly increasing with fixed $n$ and $k > 1$. Now if $k = n$, we have that the LHS is $n$ and the RHS is $n + \sqrt{2n} \log n$, which means $RHS > LHS$. Now when $k = 2n$, we have the LHS is $2n$ and the RHS is $n + \sqrt{3n} \log n$ which for large $n$ is less than $2n$. Therefore, $LHS > RHS$. By continuity and the fact that both sides are nondecreasing, we have that $k$ has a solution and it is between $n$ and $2n$. 
\subsection{b}
Let $k = \alpha n$ where $\alpha = O(1)$ and $1 < \alpha < 2$. 
Then we have that 
$k = n + \sqrt{n + \alpha n} \log (\alpha n) = n + \sqrt{n} \sqrt{\alpha + 1} (\log(\alpha) + \log n)$. Then we distribute and use the fact that $\alpha = O(1)$ to get that
\begin{align*}
	k &= n + \sqrt{n} O(1) + \log n \sqrt{n} O(1)\\
	 &= n + O(\sqrt{n}) + O(\log n \sqrt{n})\\
	 &= n + O(\log n \sqrt n).
\end{align*}
However, we can write the right hand side as $k = n(1 + O(\frac{\log n}{\sqrt{n}}))$ which is $k = n(1 + o(1))$. So $k \sim n$. 
\subsection{c}
Let $z = k - n$. From part $a$, we have that $0 < k < n$ and $z =  O(\log n \sqrt n)$. Then we rewrite the above equation in terms of $z$ to yield:
\begin{equation}
	z = \sqrt{z + 2n} \log (z + n).
\end{equation}
Now we plug in the approximation that $z = O(\log n \sqrt{n})$. We have that:

\begin{align*}
	z &= \sqrt{(O(\log n \sqrt{n}) + 2n)}\log(n + O(\log n \sqrt n))\\
	&= \sqrt{2n} \left(\sqrt{1 + O(\frac{\log n}{\sqrt{n}})}\right) \left(\log( n(1 + O(\frac{\log n}{\sqrt{n}} )) \right)\\
	&= \sqrt{2n}\left(\sqrt{1 + O(\frac{\log n}{\sqrt{n}})}\right) \log(n) + \sqrt{2n}\left(\sqrt{1 + O(\frac{\log n}{\sqrt{n}})}\right) \log((1 + O(\frac{\log n}{\sqrt{n}}))
	\end{align*}
Now we have that from doing a Taylor expansion that $\sqrt{1 + x} = 1 + O(x)$ when $x = o(1)$. We also have that $\log(1 + x) = O(x)$ when $x = o(1)$. Plugging this both into the equation, we have that:
\begin{align*}
	z &= \sqrt{2n}(1 + O(\frac{\log n}{\sqrt{n}})) \log(n) + \sqrt{2n}(1 + O(\frac{\log n}{\sqrt{n}}))O(\frac{\log n}{\sqrt{n}})\\
	&= \sqrt{2n} \log n + O(\log n) + O(\log n) + O(\frac{\log^2 n}{\sqrt{n}})\\
	&= \sqrt{2n} \log n + O(\log n)
\end{align*}
as we have that $\log^2(n) = o(\sqrt{n})$. Finally, we have that:
\begin{equation}
	z = \sqrt{2n} \log n ( 1 + O(\frac{1}{\sqrt{n}})) = \sqrt{2n} \log n (1 + o(1)).
\end{equation}
So $z \sim \sqrt{2n} \log n$.

\subsection{b}
Let
\begin{equation}
	k = \frac{n^2}{n + k \log k}
\end{equation}
Let $k = 1$. Then we have the LHS is $1$ and the RHS is $\frac{n^2}{n + 0} = n$. Let $k = n$. Then we have the LHS is $n$ and the RHS is $\frac{n^2}{n + n \log n} = \frac{n}{1 + \log n}$.

Now rearrange the equation. We have that $n^2 = nk + k^2 \log k$. But this implies that $n^2 - nk - k^2 \log k = 0$. Therefore, we have that:
\begin{equation}
	n = \frac{k \pm \sqrt{k^2 + 4 k^2 \log k}}{2}.
\end{equation}
The equation that makes the most sense is if $\pm = +$ otherwise $n$ is negative. 
Simplifying, we have that $n = \frac{k + k \sqrt{(1 + 4\log k)}}{2} = \frac{k (1 + \sqrt{1 + 4\log k})}{2}$. Now we rearrange to get that:
\begin{equation}
	k = \frac{2n}{\sqrt{1 + 4\log k}}
\end{equation}
Now suppose $k = \alpha n$ for $\alpha = O(1)$, $0 < \alpha < 1$. 
Then we have that:
\begin{align*}
	k = \frac{2n}{\sqrt{1 + 4(\log \alpha) + 4\log n}}\\
	k = \frac{2n}{\sqrt{4\log n} \sqrt{1 + (\log \alpha/4\log n) + 1/4\log n}}
\end{align*}
But we have that $\frac{1}{\sqrt{1 + x}} = 1 + O(x)$ when $x = o(1)$. Therefore, we have that
\begin{align*}
	k &= \frac{n}{\sqrt{\log n}} \left( 1 + O(1/ \log n) \right)\\
	k &=\frac{n}{\sqrt{\log n}} (1 + o(1))\\
	k &\sim \frac{n}{\sqrt{\log n}}.
\end{align*}

\section{4}
\subsection{a}
We have that $b_0 = 0$, $b_1 = a_1$. We have that $a_0 = 0$ as well. For the recurrence, we have that $b_n = \sum_{i = 1}^{n} a_i b_{n - i} = \sum_{i = 0}^{n} a_i b_i + a_n$. We have that $[x^n]a(x) b(x) = \sum_{i = 0}^{n} a_i b_{n - i}$.  Therefore, we have that
\begin{equation}
	b(x) =  a(x)(b(x) + 1) = a(x) b(x) + a(x). 
\end{equation}
Shuffling around, we have that:
\begin{equation}
	b(x) = \frac{a(x)}{1 - a(x)}.
\end{equation}
\subsection{b}
Let $S$ be the radius of convergence of $a$. We have that $A > 1/S$ as $|a_n|^{1/n} \leq A$ i.o. Therefore, we have that $1/A < S$. As $a(x)$ is strictly increasing on $0$ to $S$, we have that $a(x) < a(1/A)$ when $x < 1/A$ on the boundary. We have that the only singularity of $b$ occurs when $a(x) = 1$. Note that we have that $a(x e^{i\theta}) \leq a(x)$ for all $x> 0$, so we have that $b$ converges on a radius smaller than $1/A$, as $a(1/A - \varepsilon) < 1$ by the argument above. Therefore, we have that $b(x)$ has a radius of convergence which includes $|z| = 1/A$. Therefore, we have that $b_n < \left(A + \varepsilon \right)^n$ eventually. 

\subsection{c}
Firstly, we have that we have that $a(x e^{i\theta}) \leq a(x)$ for all $x > 0$. We also have that $a(x)$ is increasing between $0$ and $1/A$. Therefore, if $a(C) = 1$ and $C$ is the minimum positive real satisfying this equation, then for all $z$ where $|z| < C$, we have that $b$ converges. Therefore, radius of convergence of $b$ is $C$. Therefore, $b_n > \left(1/C - \varepsilon \right)^n$ i.o. 

\section{5}

\subsection{a}
We will prove this using deck and hands.
We have that the deck that we are dealing with are $n$-cycles which are even. In other words, we want that every connected component has an even number of vertices. We know that the deck enumerator for connected $2$-regular graphs is:
\begin{equation}
	D(x) = \frac{1}{2}\left(\log\left(\frac{1}{1 - x}\right) - x - \frac{x^2}{2}\right)
\end{equation}
Therefore, the deck enumerator for connected $2$-regular graphs of even cycle length is:
\begin{align}
	E(x) = \frac{1}{2} \left(D(x) + D(-x)\right) &= \frac{1}{4} \left[ \left(\log\left(\frac{1}{1 - x}\right) - x - \frac{x^2}{2}\right) + \left(\log\left(\frac{1}{1 + x}\right) + x - \frac{x^2}{2}\right) \right]\\
	&=\frac{1}{4} \left(\log\left(\frac{1}{(1 - x)(1 + x)}\right) - x^2\right) \\
	&=\frac{1}{4} \left(\log\left(\frac{1}{1 - x^2}\right)- x^2\right) 
\end{align}
Then the egf for the number of $2$-regular bipartite graphs is:
\begin{equation}
	e^{E(x)} = \exp\left(\log\left(\frac{1}{(1 - x^2)^{1/4}}\right)- \frac{x^2}{4}\right)  = \frac{e^{-x^2/4}}{(1 - x^2)^{1/4}}
\end{equation}

\subsection{b}
We have that $x$ is analytic in $\left\{ z: |z| < 1 + \varepsilon \right\} \setminus \left\{ -1, 1 \right\}$. Fix a $\delta > 0$. Let $C_1$ be the curve $|z| = 1 + \delta$ moving counterclockwise  and let $C_2$ be the curve $|z - 1| = \delta$ going clockwise. Let $C_3$ be the curve $|z + 1| = \delta$ moving clockwise. Let $C$ be the curve, starting at $1 + 0i$ moving counterclockwise according to $C_1$ to $-1 + 0i$, moving clockwise around $C_3$ back to $-1 + 0 i$, moving counterclockwise around $C_1$ to $1 + 0i$, and then going around $C_2$ moving clockwise. We will use a variant of Darboux's theorem to find the asymptotics.


\subsubsection{Darboux's method}

We want to write $f(z) =\frac{e^{-z^2/4}}{(1 - z^2)^{1/4}}$ as a summation $\sum_{i = 0}^{J'} c_j (1 - z)^{\alpha + j \beta} + \sum_{i = 0}^{J'} c_j (1 + z)^{\alpha + j \beta}$.

We can do this by evaluating $f$ on a series of curves $C_1 := \left\{z : |z| = 1 + \delta\right\}$ counterclockwise, $C_2 := \left\{z : |1 - z| = \delta\right\}$ clockwise, and $C_3 := \left\{z : |1 + z| = \delta\right\}$ clockwise. From the proof of Darboux's method, $C_1$ is negligible, and $C_2$ and $C_3$ are the most important curves. Now we have to evaluate $f$ around these curves. 


We can do this by writing $f(z) = g(z) + h(z)$ where $g(z)$ is $\frac{e^{-z^2/4}}{(1 - z^2)^{1/4}}$ evaluated around $z = 1$ and $h(z) = \frac{e^{-z^2/4}}{(1 - z^2)^{1/4}}$ evaluated around $z = -1$.

\subsubsection{Evaluating near $z = 1$}
 We have that when $z$ is near $1$, then we can write
\begin{equation}
	e^{-z^2/4} = e^{-1/4} + \frac{(1-z)}{2} e^{-1/4}  - \frac{(1 - z)^2}{8} e^{-1/4} + O((1 - z)^3)
\end{equation}
Now this means that if we have that $z$ is near $1$, the denominator becomes $(1 - z)^{1/4} (1 + z)^{1/4}$. But $(1 + z)^{1/4}$ can be evaluated, and is $2^{1/4}$. Therefore, we have that:

\begin{equation}
	g(z) = \frac{e^{-z^2/4}}{(1 - z)^{1/4}} = \frac{1}{e^{1/4} 2^{1/4}} \left(\frac{1}{(1 - z)^{1/4}} + \frac{(1 - z)^{3/4}}{2} - \frac{(1 - z)^{7/4}}{8} +  O((1 - z)^{11/4})\right)
\end{equation}
We have that $\alpha = -1/4$, $\beta = 1$ and $J' = 2$. We have that $c_0 = \frac{1}{e^{1/4} 2^{1/4}}$, $c_1 = \frac{1}{2 e^{1/4} 2^{1/4}}$, and $c_2 = -\frac{1}{8 e^{1/4} 2^{1/4}}$.
Therefore, by Darboux' method, we have that 
\begin{equation}
	[z^n]g(z) = \frac{1}{e^{1/4} 2^{1/4}} \left( \binom{-1/4}{n}(-1)^{-n} +  \frac{1}{2}\binom{3/4}{n}(-1)^{-n} - \frac{1}{8} \binom{7/4}{n} (-1)^{-n} + O(n^{-15/4} \right).
\end{equation}

With the approximation that $\binom{q}{n}(-1)^n = \frac{n^{-q - 1}}{\Gamma(-q)}$, we have that $\binom{-1/4}{n}(-1)^{-n} \sim \frac{n^{-3/4}}{\Gamma(1/4)}$.

We have that:
\begin{equation}
	\binom{3/4}{n}(-1)^{-n} \sim \frac{n^{-7/4}}{\Gamma(-3/4)}
\end{equation}

And
\begin{equation}
	\binom{7/4}{n}(-1)^{-n} \sim \frac{n^{-11/4}}{\Gamma(-7/4)}
\end{equation}
We have that $\Gamma(-3/4) = 1/4 \Gamma(1/4)$ and that $\Gamma(-7/4) = -3/4 \Gamma(-3/4) = -3/16 \Gamma(-1/4)$. 

Therefore, we have that:

\begin{equation}
	[z^n]g(z) = \frac{1}{e^{1/4} 2^{1/4}\Gamma(1/4)} \left( n^{-3/4} +  2 n^{-7/4} + \frac{2}{3} n^{-11/4} + O(n^{-15/4} \right).
\end{equation}

\subsubsection{Evaluating near $z = -1$}
Now we use Darboux's theorem for near $z_0 = -1$. We have that $h(z(-1)) = g(z)$. 

Now we want to write $e^{-z^2/4}$ around $z_0 = -1$. We have that:

\begin{equation}
	e^{-z^2/4} = e^{-1/4} \left(1 + \frac{1 + z}{2} - \frac{(1 + z)^2}{8} + O((z + 1)^3)\right)
\end{equation}
Therefore, we have that the denominator $\frac{1}{(1 - z)^{1/4} (1 + z)^{1/4}}$ becomes $\frac{1}{2^{1/4}(1 + z)^{1/4}}$. Therefore, we have that:

\begin{equation}
	h(z) = \frac{1}{e^{1/4} 2^{1/4}} \left(\frac{1}{(1 + z)^{1/4}} + \frac{(1 + z)^{3/4}}{2} - \frac{(1 + z)^{7/4}}{8} +  O((1 + z)^{11/4})\right)
\end{equation}

We have that $\alpha = -1/4$, $\beta = 1$ and $J = 2$. 
Now we use Darboux's theorem near $z_0 = -1$. Then 
\begin{equation}
	[z^n] h(z) = \frac{1}{e^{1/4} 2^{1/4}} \left(\binom{-1/4}{n} 1^{-n}+ \frac{1}{2} \binom{3/4}{n} 1^{-n} - 1/8 \binom{7/4}{n} (1)^{-n }+  O(n^{-15/4})\right)
\end{equation}
So 
\begin{equation}
	[z^n] h(z) = \frac{1}{e^{1/4} 2^{1/4}} \left(\binom{-1/4}{n} + \frac{1}{2} \binom{3/4}{n} - 1/8 \binom{7/4}{n}+  O(n^{-15/4})\right)
\end{equation}
Now we have that $\binom{-1/4}{n} = \frac{\Gamma(3/4)}{\Gamma(n + 1) \Gamma(-1/4 - n)}$, $\binom{3/4}{n} = \frac{\Gamma(7/4)}{\Gamma(n + 1) \Gamma(3/4 - n)}$, and $\binom{7/4}{n} = \frac{\Gamma(11/4)}{\Gamma(n + 1) \Gamma(7/4 - n)}$. We have that $\Gamma(7/4) = 3/4 \Gamma(3/4)$ and $\Gamma(11/4) = 21/16 \Gamma(3/4)$. 

In the same vein, we have that $\Gamma(3/4 - n) = (-1/4 - n)\Gamma(-1/4 - n)$ and $\Gamma(7/4 - n) = (3/4 - n)(-1/4 - n) \Gamma(-1/4 - n)$. 

Therefore, we have that
\begin{align*}
	[z^n] h(z) &= \frac{\Gamma(3/4)}{e^{1/4} 2^{1/4} n!} \left(\frac{1}{\Gamma(-1/4 - n)}+ \frac{3}{8 \Gamma(3/4 - n)} - \frac{21}{128 \Gamma(7/4 - n)}+  O(n^{-15/4})\right)\\
	&= 
	\frac{\Gamma(3/4)}{e^{1/4} 2^{1/4} n! \Gamma(-1/4 - n)} \left(1+ \frac{3}{8 (-1/4 - n)} - \frac{21}{128 (3/4 - n)(-1/4 - n)}+  O(n^{-15/4} n^{-n})\right)\\
\end{align*}

Now we put these two sums together to get that:
\begin{align*}
	[z^n] f(z) &= [z^n] \left(g(z) + h(z)\right)\\
	&=
	\frac{1}{\Gamma(1/4) e^{1/4} 2^{1/4} }\left( n^{-3/4} +  2 n^{-7/4} + 2/3 n^{-11/4} + O(n^{-15/4}) \right)\\ 
	&+
	\frac{\Gamma(3/4)}{e^{1/4} 2^{1/4} n! \Gamma(-1/4 - n)} \left(1+ \frac{3}{8 (-1/4 - n)} - \frac{21}{128 (3/4 - n)(-1/4 - n)}+  O(n^{-15/4} n^{-n})\right)
\end{align*}


Therefore, we have that the number of labelled $2$-regular bipartite graphs is:

\begin{align}
	n! [z^n] f(z)&=
\frac{n!}{\Gamma(1/4) e^{1/4} 2^{1/4}}\left( n^{-3/4} +  2 n^{-7/4} + 2/3 n^{-11/4} + O(n^{-15/4}) \right)\\ 
&+
\frac{\Gamma(3/4)}{e^{1/4} 2^{1/4} \Gamma(-1/4 - n)}  \left(1+ \frac{3}{8 (-1/4 - n)} - \frac{21}{128 (3/4 - n)(-1/4 - n)}+  O(n^{-15/4} n^{-n})\right)
\end{align}

Finally, we can say $n! \sim (n/e)^n\sqrt{2 \pi n}$ and approximate. 
\begin{align}
	n! [z^n] f(z)&=
	\frac{n^n \sqrt{2 \pi n}}{\Gamma(1/4) e^{1/4 + n} 2^{1/4}}\left( n^{-3/4} +  2 n^{-7/4} + 2/3 n^{-11/4} + O(n^{-15/4}) \right)\\ 
	&+
	\frac{\Gamma(3/4)}{e^{1/4} 2^{1/4} \Gamma(-1/4 - n)}  \left(1+ \frac{3}{8 (-1/4 - n)} - \frac{21}{128 (3/4 - n)(-1/4 - n)}+  O(n^{-15/4} n^{-n})\right)
\end{align}

However, we have that when $n$ is odd, the left hand side cancels out the right hand side.

When $n$ is even, then we have that the left hand side and the right hand side should be the same, so we have that

$f(z) \sim 2 \frac{\Gamma(3/4)}{e^{1/4} 2^{1/4} \Gamma(-1/4 - n)}$ when $n$ is even, or
$f(z) \sim 2\frac{n^n \sqrt{2 \pi n}}{\Gamma(1/4) e^{1/4 + n} 2^{1/4} n^{3/4}}$. 
\section{6}
\subsection{a}
We shall think about this in the form of decks. Let a card be a connected digraph where no two edges share a vertex. We have that $d_1 = 1$ as there is only one standard card of size 1, which is the singleton vertex with label $1$. We have $d_2 = 2$ as there are two cards, the vertex $1$ pointing to the vertex $2$, and the vertex $2$ pointing to the vertex $1$. Then the deck enumerator is $x + \frac{1}{2} (2) x^2 = x + x^2$, so the egf for $\left\{ c_n\right\}_{n \geq 0}$ is $f(x) := e^{x + x^2}$.

\subsection{b}
We have that $f$ is entire on $\mathbb{C}$. Therefore, the radius of convergence is infinite. Thus, we can evaluate $f$ everywhere. We have that from the saddlepoint bound for egfs that
\begin{equation}
	[x^n] f(n) \leq x_0^{-n} f(x_0)
\end{equation}
so substituting this in, we have that:
\begin{equation}
	c_n/n! \leq r^{-n} e^{r + r^2}.
\end{equation}
To find when $r$ is minimised, we take the derivative. So we get that:
\begin{align*}
	 r^{-n} e^{r + r^2} &= e^{-n \log r + r + r^2}\\
	 0 &= \dfrac{d}{dr}(e^{-n \log r + r + r^2}) \\
	 0 &= \left(2r + 1 - \frac{n}{r}\right) e^{-n \log r + r + r^2}\\
	 0 &= 2r + 1 - \frac{n}{r}\\
	 0 &= 2r^2 + r - n\\
\end{align*}
From the quadratic equation, we have that $r = \frac{-1 \pm \sqrt{1 + 8n}}{4}$, so taking the smaller of the two in absolute magnitude, we have that:
\begin{equation}
	r = \frac{\sqrt{1 + 8n} - 1}{4}
\end{equation}

\subsection{c}
We rewrite $r$ as:
\begin{equation}
	r = \frac{\sqrt{8n + 1}}{4} - \frac{1}{4}
\end{equation}
and then take a factor of $8n$ out of the square root. So we have that:
\begin{equation}
	r = \frac{\sqrt{8n}\sqrt{1+ 1/8n}}{4} - \frac{1}{4}
\end{equation}
Now we have that $\sqrt{1 + x} = 1 + x/2 + O(x^2)$ so we have that:
\begin{align*}
	r &= \frac{\sqrt{8n}\left(1 + \frac{1}{16n} + O(\frac{1}{n^2})\right)}{4} - \frac{1}{4}\\
	&= \sqrt{\frac{n}{2}} + \frac{1}{16 \sqrt{2 n}} - \frac{1}{4} + O(\frac{1}{n^{3/2}})
\end{align*}
which has additive error $o(\frac{1}{n})$. 

\subsection{d}
We have that $|f(r e^{i \theta})| = |\exp\left(r e^{i \theta} + r^2 e^{2 i \theta}\right)|$. Now we have that $e^{i \theta} = \cos(\theta) + i \sin (\theta)$, but $|e^{a + ib}| = e^a$. Therefore, the $i \sin \theta$ terms disappear in the magnitude, so we have that 
\begin{equation}
	|f(r e^{i \theta})| = e^{r \cos (\theta) + r^2 \cos (2 \theta)}. 
\end{equation}

Now we have that $\cos \theta = 1 - \theta^2/2 + O(\theta^4)$. 

We want that:
\begin{equation}
	\exp(r (1 - \theta^2/2 + O(\theta^4)) + r^2 (1 - 2 \theta^2 + O(\theta^4)) - r - r^2) = o(1/n)
\end{equation}
We have that the above equation evaluates to:
\begin{equation}
	\exp(
	-\theta^2/8 - \theta^2/(256n) + \theta^2/(32 \sqrt{2n}) + \sqrt{n} \theta^2/(2 \sqrt{2}) - n \theta^2 + O(\theta^4 n)
	)
\end{equation}
meaning that if $t = n^{-1/4 - \epsilon}$ for any $\epsilon > 0$, then we have that the $O(\theta^4 n)$ goes to 0. But we want that $\exp^{- n \theta^2} = o(-n)$ so we want that $\theta^2 n = n^\varepsilon$. Therefore, we want that $\theta = n^{-1/2 + \varepsilon}$ for some $\varepsilon > 0$. Therefore, if we set $\theta = n^{-2/5}$, then we have that $\exp(-n^{1/5}) = o(1/n)$.

\subsection{e}

Let $g(\theta) = \frac{f(r e^{i \theta})}{r^n e^{i n \theta}}$. Then we have that:
\begin{equation}
	g(\theta) = \frac{ e^{r e^{i \theta} + r^2 e^{2 i \theta}}}{r^n e^{i n \theta}}
\end{equation}

We have that $\log(g (\theta))$ is:

\begin{equation}
	\log(g(\theta)) = r e^{i \theta} + r^2 (e^{i \theta})^2 - n \log r - i n \theta.
\end{equation}

We have that $e^{i \theta} = 1 + i \theta - \theta^2/2  + O(\theta^3)$ and as $\theta = n^{-2/5}$ we have that $\theta^3 = o(n^{-1})$.

We expand out $\log r$ as $\log(\sqrt{\frac{n}{2}} \left(1 - \frac{1}{2 \sqrt{2n}} + \frac{1}{16n} + O(1/n^2)\right))$. So

\begin{equation}
	\log(r) = \frac{1}{2}\log(n/2) + \log\left(1 - \frac{1}{2 \sqrt{2n}} + \frac{1}{16n} + O(1/n^2)\right).
\end{equation} 
Now we have that 
\begin{align*}
	&\log\left(1 - \frac{1}{2 \sqrt{2n}} + \frac{1}{16n} + O(1/n^2)\right) \\
	&= - \frac{1}{2 \sqrt{2n}} + \frac{1}{16n} + (- \frac{1}{2 \sqrt{2n}} + \frac{1}{16n})^2/2 + O(1/n^2)\\
	&= -\frac{1}{2 \sqrt{2n}} + \frac{1}{32 \sqrt{2}n^{3/2}} + O(1/n^2)
\end{align*}
Therefore, we have that $n \log r = \frac{n}{2}\log \left(\frac{n}{2}\right) - \frac{\sqrt{n}}{2 \sqrt{2}} + \frac{1}{32 \sqrt{2} n^{1/2}} + O(1/n)$. 

Finally, we have that
\begin{align*}
	 &r e^{i \theta} + r^2 (e^{i \theta})^2 \\
	 &= \sqrt{\frac{n}{2}} \left(1 - \frac{1}{2 \sqrt{2n}} + \frac{1}{16n} + O(1/n^2)\right)( 1 + i \theta - \theta^2/2 + O(\theta^3)) \\
	 &+ \frac{n}{2} \left(1 - \frac{1}{2 \sqrt{2n}} + \frac{1}{16n} + O(1/n^2)\right)^2( 1 + i \theta - \theta^2/2 + O(\theta^3))^2
\end{align*}

So we have that 
\begin{align}
	\log g(\theta) &=r e^{i \theta} + r^2 (e^{i \theta})^2 - n \log r - i n \theta \\
	&=
	-1/8 + \sqrt{n/2} + n/2 - n \theta^2 -n/2 \log(n/2) + o(1) 
\end{align}
So we have that $g(\theta) = \exp(-1/8 + \sqrt{n/2} + n/2 - n\theta^2)(n/2)^{-n/2}$. 

Finally, we evaulate the integral on the domain $\theta = [n^{-2/5}, n^{2/5}]$. We have that:
\begin{equation}
	[z^n] f(z) = \frac{1}{2 \pi} \int_{-n^{-2/5}}^{n^{-2/5}} \exp(-1/8 + \sqrt{n/2} + n/2 - n\theta^2) (n/2)^{-n/2}\, d\theta
\end{equation}
We can take out all terms except for $-n \theta^2$ and evaluate. 
\begin{align*}
	[z^n] f(z) &= \frac{\exp(-1/8 + \sqrt{n/2} + n/2) }{2 \pi(n/2)^{n/2}} \int_{-n^{-2/5}}^{n^{-2/5}} \exp(- n\theta^2)\, d\theta\\
	&=
	 \frac{\exp(-1/8 + \sqrt{n/2} + n/2)}{2 \pi(n/2)^{n/2}} \sqrt{\pi/n}\, d\theta \\
\end{align*}

Therefore, we have that $c_n = [z^n] f(z) n!$. We have that $n! = (n/e)^n \sqrt{2 \pi n}$
so we have that:
\begin{equation}
	c_n = \frac{n^{n/2} 2^{n/2 - 1/2} e^{-1/8 + \sqrt{n/2}} }{ e^{n/2}}
\end{equation}

which matches Hayman's method. 

\subsubsection{Using Hayman's method}
As an addendum, we will also use Hayman's method to calculate $c_n/n!$. 


We have that $a(n) = r f'(x)/f(r) = r(2r + 1)$. Then we get that 
\begin{equation}
	r_n = \frac{-1 + \sqrt{1 + 8n}}{4} = \sqrt{\frac{n}{2}} \left(1 - \frac{1}{2 \sqrt{2n}} + \frac{1}{16n} + O(1/n^2)\right)
\end{equation}
which agrees with above. 

Now we have that:
\begin{align*}
	f(r_n) = e^{r_n + r_n^2} &= \exp\left(\sqrt{\frac{n}{2}} - 1/4 + o(1) + (n/2)\left(\frac{n}{2} \left(1 - \frac{1}{2 \sqrt{2n}} + \frac{1}{16n}\right) ^2 \right)\right)\\
	&=
	\exp \left(
	\sqrt{\frac{n}{2}} - 1/4 + n/2 - 1/4\sqrt{\frac{n}{2}} + 1/32 - 1/4\sqrt{\frac{n}{2}} + 1/16 + 1/32 + o(1)
	\right)\\
	&=
	\exp \left(
	\frac{n}{2} + \frac{1}{2} \sqrt{\frac{n}{2}} - 1/8 + o(1)
	\right)
\end{align*}
We have that $b(r_n) = r_n + 4 r_n^2 \sim 2n$ as $r_n \sim \sqrt{n/2}$. Finally, we have that:
\begin{equation}
	r_n^n = \left(\sqrt{\frac{n}{2}}\right)^n  \exp( n  \log(r_n/ \sqrt{n/2}))
\end{equation}
We have that 
\begin{align*}
	\log(r_n/\sqrt{n/2}) &= \log(1 - 1/(2 \sqrt{2n}) + 1/16n + o(1/n))\\
	& -\frac{1}{2 \sqrt{2n}} + \frac{1}{16n} - 1/2 \frac{1}{8n} + o(1/n)\\
	&= - \frac{1}{2 \sqrt{2n}} + o(1/n)
\end{align*}
So $r_n^n \sim \left(\sqrt{\frac{n}{2}}\right)^n e^{n \left(- \frac{1}{2 \sqrt{2n}} + o(1/n)\right)}$, so $r_n^n = n^{n/2} 2^{-n/2} e^{-\frac{\sqrt{n}}{2\sqrt{2}}}$. 

From Hayman's method, we have that:
\begin{equation}
	c_n/n! \sim \frac{f(r_n)}{r_n^n \sqrt{2 \pi b(r_n)}}
\end{equation}
so we input in what we have to get:
\begin{align}
	c_n/n! &\sim \frac{e^{n/2 + 1/2 \sqrt{n/2} - 1/8}}{n^{n/2} 2^{-1/2} e^{-1/2 \sqrt{n/2}} \sqrt{2 \pi (2n)}}\\
	&\sim 
	\frac{e^{n/2 + \sqrt{n/2} - 1/8} 2^{n/2}}{n^{n/2} \sqrt{4 \pi n}}
\end{align}
Now we have that $n! \sim (n/e)^n \sqrt{2 \pi n}$ so we have that:
\begin{align}
	c_n &\sim n^n e^{-n} \sqrt{2 \pi n}\frac{e^{n/2 + \sqrt{n/2} - 1/8} 2^{n/2}}{n^{n/2} \sqrt{4 \pi n}}\\
	&=
	n^{n/2} e^{\sqrt{n/2} - n/2 - 1/8} 2^{n/2 - 1/2}
\end{align}


\subsection{f}
Recall that:
\begin{equation}
	|f(r e^{i \theta})| = \exp(
	-\theta^2/8 - \theta^2/(256n) + \theta^2/(32 \sqrt{2n}) + \sqrt{n} \theta^2/(2 \sqrt{2}) - n \theta^2 + O(\theta^4 n)
	).
\end{equation}
Then when $\theta > n^{-2/5}$, we have that 
\begin{align*}
	&\frac{1}{2 \pi n^n} \int^\pi_{n^{-2/5}} \exp(
	-\theta^2/8 - \theta^2/(256n) + \theta^2/(32 \sqrt{2n}) + \sqrt{n} \theta^2/(2 \sqrt{2}) - n \theta^2 + O(\theta^4 n)
	) \, d\theta \\
	&= O(1/n^n \exp(n^{1/5}))\\
	&= o(n^{n/2} e^{-n/2} 2^{n/2})
\end{align*}
as wanted. 
\end{document}