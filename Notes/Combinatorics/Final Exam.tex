\documentclass[]{article}
\usepackage[margin=1in]{geometry}

\usepackage{amsmath}
\usepackage{amssymb}
\usepackage{amsthm}
\usepackage{url}

% Environments

\newtheorem{theorem}{Theorem}
\newtheorem{proposition}[theorem]{Proposition}
\newtheorem{corollary}[theorem]{Corollary}
\newtheorem{lemma}[theorem]{Lemma}
\newtheorem{definition}[theorem]{Definition}
\newtheorem{conjecture}[theorem]{Conjecture}

\theoremstyle{definition}
\newtheorem{example}[theorem]{Example}

\numberwithin{theorem}{section}
\numberwithin{equation}{section}

\newcommand{\ops}{\overset{\text{ops}}{\leftrightarrow}}
\newcommand{\egf}{\overset{\text{egf}}{\leftrightarrow}}

%opening
\title{Exam}
\author{Eric Luu}

\begin{document}

\maketitle
\section{Question 1}

\section{Question 2}

\section{Question 3}
We can use Cauchy-Frobenius with the group $\mathbb{Z}_n$ acting on the set of circular sequences.

\section{Question 4}

\section{Question 5}

\section{Question 6}

\section{Question 7}

\section{Question 8}

\section{Question 9}
\end{document}
