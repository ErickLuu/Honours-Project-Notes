\documentclass[]{article}
\usepackage[margin=1in]{geometry}

\usepackage{amsmath}
\usepackage{amssymb}
\usepackage{amsthm}
\usepackage{url}

% Environments

\newtheorem{theorem}{Theorem}
\newtheorem{proposition}[theorem]{Proposition}
\newtheorem{corollary}[theorem]{Corollary}
\newtheorem{lemma}[theorem]{Lemma}
\newtheorem{definition}[theorem]{Definition}
\newtheorem{conjecture}[theorem]{Conjecture}

\theoremstyle{definition}
\newtheorem{example}[theorem]{Example}

\numberwithin{theorem}{section}
\numberwithin{equation}{section}

\newcommand{\ops}{\overset{\text{ops}}{\leftrightarrow}}
\newcommand{\egf}{\overset{\text{egf}}{\leftrightarrow}}

%opening
\title{Exam}
\author{Eric Luu}

\begin{document}

\maketitle
\section{Question 1}
We have that there are $6$ current Australian coins, being the 5c, 10c, 20c, 50c, \$1, \$2. 
Let $d_n$ be the number of sequences where the first and last coins are different denominations.

For simplicity, suppose there are $m$ coins.

We have that:
\begin{equation}
	d_n = m \times (m - 1)^{n-1} - d_{n-1}
\end{equation}
where $m \times (m - 1)^{n-1}$ are the total number of ways to count the number of sequences where no coin appears twice in a row and $d_{n-1}$ counts the number of ways where the first and last coins are the same coin.

We have that $d_1 = m$ and $d_2 = m (m - 1)$. 

We claim that this satisfies the recurrence:
\begin{equation}
	d_n = \begin{cases}
		m &, \text{ if } n = 1\\
		(m -1)^n + (-1)^n (m - 1) &, \text{ if } n \geq 2
	\end{cases}
\end{equation}
When $n = 2$, we claim that this satisfies the equation above. Suppose it holds for $n = k$. Then for $k + 1$, we have that:
\begin{align*}
	d_{k + 1} &= m \times (m - 1)^{k} - ((m -1)^k + (-1)^k (m - 1))\\
	&= (m - 1)^{k + 1} + (m-1)^k - (m - 1)^k - (-1)^k (m - 1)\\
	&= (m - 1)^{k + 1} + (-1)^{k + 1}(m - 1).
\end{align*}
Thus this satisfies the recurrence outlined above.

We can get to the recurrence by treating a sequence as a labelled graph $C_n$ with labels $1, 2, ...n$ around the cycle. Then each sequence is a way to properly colour $C_n$. Therefore, the number of sequences $d_n$ comes from the chromatic polynomial of $C_n$ and we can use the contraction-deletion formula to have the result above. Wikipedia has a list of useful chromatic polynomials and $C_n$ is one of them.

\subsection{Counting the cost}
We have that each coin appears the same number of times as another coin. This is because of symmetry. If we have the list of all sequences and we permute the coins around, then we must have the same list back, otherwise we are missing some sequences in the list. As we claim to have the list of all sequences, this is impossible. Let $c_n$ be the cost of a sequence. Then we have that the cost is equal to the average coin denomination times the number of sequences times the number of coins in each sequence. Therefore,
\begin{equation}
	c_n = \frac{(0.05 + 0.10 + 0.20 + 0.50 + 1 + 2) }{6}(n) \left[5^n + 5(-1)^n \right] =  \frac{77}{120} n \left[5^n + 5(-1)^n \right]
\end{equation}
when $n \geq 2$.
So:
\begin{equation}
	c_n = 
	\begin{cases}
		3.85 &, \text{ if } n = 1\\
		\frac{77}{120} n \left[5^n + 5(-1)^n \right] &, \text{ if } n \geq 2
	\end{cases}
\end{equation}

\section{Question 2}

\section{Question 3}
\subsection{a}
We have that the total number of sequences is $\binom{2n-1}{n}= \frac{(2n -1 )!}{n! (n - 1)!}$. Now for the number of circular sequences, we identify $(2n - 1)$ sequences with each other. These sequences are distinct as if two sequences were the same when shifted, it would mean that the number of $2$s and $0$s would divide the length of the sequence, which is impossible. So we have that:
\begin{equation}
	|C_n| = \frac{(2n - 2)!}{n!(n - 1)!} = \mathcal{C}_{n - 1}
\end{equation}
where $\mathcal{C}_{n}$ is the $n$-th Catalan number, and $|C_0| = 1$. 

\subsection{b}
\subsection{Correspondence between tree and sequence of $0$s and $2$s}
Let $T$ be a tree with root $r$ with $2n$ vertices embedded in the plane. Then for each tree, there is a specific preorder string where we "eat" the tree by starting at $r$, traversing to its neighbour, and deleting $r$.

Then at every step, we look at the vertex $v$ we are at, record down its degree, then "eat" $v$ and move down to the left vertex. After the left subtree of $v$ is eaten, we then move onto the right subtree.

We are doing a pre-order traversal where we record either $0$ if the vertex is a leaf or $2$ if the vertex is an internal vertex. We then ignore the root vertex in this sequence. This gives us a string with $n$ zeroes and $n-1$ 2s. 

\subsubsection{Correspondence between circular sequence and trees}
Let $s$ be a sequence of $n - 1$ 2s and $n$ zeros. Let $[s]$ be the set of circular rotations of $s$. We have that there is a unique $s' \in [s]$ such that for every prefix of $s'$ of length at most $2n - 2$, we have that the number of $2$s is $\geq$ the number of $0$s. The reason for this is to consider $s$ as a mountain range where we start at $0$ and we go up 1 step when we see a 2 and down 1 step when we see a 0. We then look at all the points where we hit the maximum depth and we go to the one on the farthest left. Then we have that at this point, we do not go down by too much. The reason this works is because we always end up at -1, so we can always find a maximum depth. At this depth, we want the one which will be above 0 until we hit -1, which must be unique. This sequence $s'$ must start with a $2$ and end with a $0$. 

When we have this unique point in the circular sequence, we then do the same operation. We start with a root vertex $r$, which is a leaf vertex. Then for any "run" of 2s, we go to the left until we hit a 0. Then we draw a leaf vertex and then go to the right of the farthest left vertex, and so on and so forth until we finish a tree. This defines a unique tree.
\subsubsection{Proving these correspondences are the same}
We have that if we have a tree $T$, then the sequence we get out from the first operation $s$ eats the tree and is also a Dyck path. This means that the number of $2$s is at least the number of zeroes in every prefix. Therefore, it is the unique starting point for the cycle. Then with this cycle, doing the operation above recreates the tree by design.

If we have a sequence, then doing the operation above is just a cyclic shift of the sequence. Therefore, pulling out the tree that comes out and decomposing it will yield a cyclic shift of the sequence. 

\subsection{c}
Therefore, there is a correspondence between $B_{2n}$ and $C_n$. Therefore, $|B_{2n}| = |C_n| = \mathcal{C}_{n - 1}$. Therefore, $\mathcal{C}_n = |B_{2n + 2}|$. 

n.b. A similar question was on a FIT3155 assignment. 
\section{Question 4}

\section{Question 5}
Calculate the total number of diagrams and use a many to one relation.

Use inclusion-exclusion to count the number of ways that a station can have at least 1 member inside all.

\section{Question 6}

\subsection{a}
We have that the total number of games is $\binom{n}{3}$. We have that the number of schedules is $T_n := \binom{n}{3}^n$, where we allow conflics. Therefore, the proportion of schedules where players 1 and 2 play together in the first game is the same as the proportion of games where $1$ and $2$ play together out of all games. 

We have that the games where $1$ and $2$ play together are of the form $\{ 1, 2, k\}$ where $k \in [3, ..., n]$. There are $(n-2)$ of these games.
Therefore, the proportion is:
\begin{equation}
	\frac{(n-2)}{\binom{n}{3}} = \frac{(n-2)}{\frac{n(n-1)(n-2)}{6}} = \frac{6}{n(n-1)}.
\end{equation}

\subsection{b}
The number of games where player 1 and player 2 never play a game together is:
\begin{equation}
	\binom{n}{3} - (n-2)
\end{equation}
using the same logic above. Therefore, the number of schedules where player 1 and 2 never play a game together is:
\begin{equation}
	\left(\binom{n}{3} - (n-2)\right)^n.
\end{equation}
Then the proportion of games where $1$ and $2$ never play together is:
\begin{equation}
	\frac{\left(\binom{n}{3} - (n-2)\right)^n}{\binom{n}{3}^n} = \left(1 - \frac{(n-2)}{\binom{n}{3}}\right)^n = \left(1 - \frac{6}{n(n-1)}\right)^n.
\end{equation}
Now we have that
\begin{equation}
	\left(1 - \frac{6}{n(n-1)}\right)^n = \exp\left( n \log \left(1 - \frac{6}{n(n-1)}\right) \right) = \exp \left(n O(1/n^2)\right) = \exp( O(1/n)) = e^{o(1)} = (1 + o(1))
\end{equation}
therefore, we have that the number of schedules where 1 and 2 never play together is asymptotic to the number of schedules. 

\subsection{c}
We have that the number of games where no player in $[k] := \lbrace 1, 2, ..., k \rbrace$ plays together is:
\begin{equation}
	\underbrace{\binom{n - k}{2} k}_{\text{number of ways to choose two players not in $[k]$ and one from k}} + \underbrace{\binom{n-k}{3}}_{\text{number of ways to choose no players from $[k]$ }}
\end{equation}
Therefore, the proportion of games where no player in $[k]$ plays together is:
\begin{equation}
	\left(\binom{n - k}{2} k + \binom{n-k}{3}\right)^n \binom{n}{3}^{-n} = \left(\frac{\binom{n-k}{2} k}{\binom{n-k}{3}} + 1\right)^n \left(\frac{\binom{n-k}{3}}{\binom{n}{3}}\right)^n
\end{equation}

We have that:
\begin{align*}
	&= \left(\frac{\binom{n-k}{3}}{\binom{n}{3}}\right)^n\\
	&= \left( \frac{\frac{(n-k)(n-k - 1)(n - k - 2)}{6}}{\frac{n(n-1)(n-2)}{6}}  \right)^n\\
	&= \left(\frac{n-k}{n} \frac{n-k - 1}{n - 1} \frac{n - k - 2}{n - 2}\right)^n\\
	&= \left(1 - k/n\right)^n \left(1 - k/(n - 1)\right)^n \left(1 - k/(n-2)\right)^n
\end{align*}
Taking the log (will take exponential later), we have that this is equal to:
\begin{equation}
	n \left( \log(1 - k/n) +  \log(1 - k/(n-1)) + \log(1 - k/(n-2)) \right)
\end{equation}
and this is asymptotic to 
$n \left(3 \log(1 - k/n)\right)$ as we have that for large $n$, $1/n$ is asymptotic to $1/(n + o(1))$.

Now we have that:
\begin{align*}
	&\left(\frac{\binom{n-k}{2} k}{\binom{n-k}{3}} + 1\right)^n\\
	&= \left(\frac{(n-k)(n - k - 1)k}{2} \frac{6}{(n - k)(n - k - 1)(n - k - 2)} + 1\right)^n\\
	&= \left(\frac{3k}{n - k - 2} + 1\right)^n
\end{align*}
Taking the logs again, we have that this is
\begin{equation}
	n \log(1 + \frac{3k}{n - k - 2})
\end{equation}
Therefore, the log of the proportion is:
\begin{equation}
	n \left( 3 \log(1 - k/n) + \log\left(1 + \frac{3k}{n - k - 2}\right)\right)
\end{equation}

\subsubsection{i}
Now suppose $k \sim n^{1/4}$.
Take the log expansion of
$\log(1 - k/n) = -k/n + O(k^2/n^2)$ and $\log\left(1 + \frac{3k}{n - k - 2}\right) = \frac{3k}{n - k - 2} + O(k^2/(n- k)^2)$. 

Then we have that:
\begin{align*}
	n \left( 3 \log(1 - k/n) + \log\left(1 + \frac{3k}{n - k - 2}\right)\right) &= n \left(- 3k/n + 3k/(n - k - 2) + O(k^2/(n - k)^2)\right) \\
	&= -3k + 3kn/(n - k - 2) + O(k^2/(n - k)).
\end{align*}

We have that $3kn/(n - k - 2) = 3k/(1 - k/n - 2/n) = 3k(1 + O(k/n)) = 3k + O(k^2/n) =  3k + o(1)$. 
Now taking the exponential and the fact that $O(k^2/(n - k)) = O(1/\sqrt{n} = o(1))$, we have the proportion is:

\begin{equation}
	\frac{e^{3k}}{e^{3k}} = 1. 
\end{equation}

\subsubsection{ii}
Suppose $k \sim \sqrt{n}$. 
Take the log expansion 
$\log(1 - k/n)  = -k/n - \frac{k^2}{2n^2} + O(k^3/n^3)$, and $\log\left(1 + \frac{3k}{n - k - 2}\right) = \frac{3k}{n - k - 2} - \frac{9k^2}{2(n - k - 2)^2} + O(k^3/(n - k)^3)$. 

Then:
\begin{align*}
	&n \left( 3 \log(1 - k/n) + \log\left(1 + \frac{3k}{n - k - 2}\right)\right) \\
	&= n \left(-3k/n - 3k^2/2n^2 + \frac{3k}{n - k - 2} - \frac{9k^2}{2(n - k - 2)^2} + O(k^3/(n - k)^3)\right)\\
	&= -3k - 3k^2/2n + \frac{3kn}{n - k - 2} - \frac{9k^2 n}{(n - k - 2)^2} + O(k^3/(n - k)^2)
\end{align*}
Now we have that $\frac{3kn}{n - k - 2}  \sim 3k$ from the argument above, and $\frac{9k^2 n}{(n - k - 2)^2} = \frac{1}{n} \frac{9 k^2}{(1 - k/n - 2/n)^2} = \frac{9k^2}{n}(1 + O(k/n)) = \frac{9k^2}{n} + o(1)$. Therefore, taking the exponent of both sides, we have that this is:
\begin{equation}
	\frac{e^3k }{e^{3k + 3k^2/(2n) + 9k^2/n}} = e^{-\frac{21}{2} k^2/n}
\end{equation}
and as $k \sim \sqrt{n}$, so $k^2/n \sim 1$, then this becomes $e^{-21/2}$. 

\subsection{d}
Fix distinct players $x, y, z$ such that $x$ has a conflict with $y$ and $x$ has a conflict with $z$. One scenario is that there are players (not necessarily distinct) and games $G_i = (x, y, a)$, $G_j = (x, y, b)$, $G_k = (x, z, c)$, $G_\ell = (x, z, d)$. 
The number of ways to choose the placement of $G_i, G_j, G_k, G_\ell$ in the schedule is $\binom{n}{4}$, the number of ways to choose $a, b, c, d$ is $(n - 2)^4$, the number of ways to choose the other games is $\binom{n}{3}^{n - 4}$, and the number of ways to choose distinct $x, y, z$ is $n(n - 1)(n-2)$. Putting this together, the proportion of schedules is this:
\begin{align*}
	&\frac{\binom{n}{4} (n-2)^4 \binom{n}{3}^{n - 4} n (n-1)(n-2)}{\binom{n}{3}^n}\\
	&= \frac{n(n-1)(n-2)(n-3) (n-2)^4 n (n-1)(n-2)}{24 \binom{n}{3}^4}\\
	&= \frac{n(n-1)(n-2)(n-3) (n-2)^4 n (n-1)(n-2) 6^4 }{24 n^4 (n-1)^4 (n-2)^4}\\
	&= \frac{(n - 2)^2 (n - 3) 6^3}{4 n^2 (n-1)^2}\\
	&= O(1/n).
\end{align*}

\subsection{e}
Fix distinct $x, y$ and suppose $G_1 = (x, y, a)$ and $G_2 = (x, y, b)$, where $a $ and $b$ are not necessarily distinct. We have that there are $\binom{n}{2}$ ways to choose $x,y$ and there are $(n-2)^2$ ways to choose $a$ and $b$. So we have that the fraction is:
\begin{equation}
	\frac{\binom{n}{2} (n - 2)^2}{\binom{n}{3}^2} = \frac{36 n (n-1) (n-2)^2 }{2 n^2 (n-1)^2(n-2)^2} = \frac{18}{n (n-1)} \sim \frac{18}{n^2}
\end{equation}

\subsection{f}
Suppose we have games $G_1, ..., G_j$ such that $|G_1 \cup G_2 \cup ... \cup G_j| = k$. We assume that $j >> k$ in this scenario. Then we choose $k$ candidates and assign them to $3j$ different rounds, ordering each player $1, 2, ..., 3j$. The number of ways to do this is roughly $(3j)^k$, which does not take into account the fact that all candidates must be in at least 1 round and cannot be twice in the same round. However, the number of ways to choose $k$ candidates is $[n]_k  = O(n^k)$ for fixed $k$. Therefore, we have that the number of ways to choose $G_1, ..., G_j$ with cardinality $k$ (when $j$ is large enough) is $O(n^j)$. 
\par
If $G_1, G_2, G_3$ are games that have cardinality $4$, then by the pigeonhole principle $G_1$ conflicts with $G_2$ and $G_3$. Therefore, we have that there are $O(n^4)$ different games. But we can (with symmetry) pick any two other positions for $G_2$ and $G_3$ so there are $\binom{n - 1 }{2} = O(n^2)$ positions that the other 2 games $G_1$ is conflicted with. Then we have that there are:
\begin{equation}
	\frac{O(n^4) O(n^2)}{\binom{n}{3}^3} = O(1/n^3)
\end{equation}
...

\subsection{g}

We have that 

\section{Question 7}
Find an asymptotic formula for:
\begin{equation}
	\sum_{k = 0}^n \binom{3n - k}{k} 2^k.
\end{equation}
Let $a_k := \binom{3n - k}{k} 2^k = \frac{(3n - k)!}{k! (3n - 2k)!} 2^k$.
We have that:
\begin{align*}
	a_{k + 1} &= \frac{(3n - k - 1)!}{(k + 1)! (3n - 2k - 2)!} 2^{k + 1}\\
	\frac{a_{k + 1}}{a_k} = \frac{(3n - 2k)(3n - 2k - 1) 2}{(k + 1)(3n - k)}
\end{align*}
Setting $\frac{a_{k + 1}}{a_k} = 1$ to find the peak, we have that:
\begin{equation}
	(3n - 2k)(3n - 2k - 1) 2 = (k + 1)(3n - k)
\end{equation}

Plugging this equation into WolframAlpha and solving for $k$, we have that:
\begin{equation}
	k = \frac{1}{18}\left(- \sqrt{81 n^2 + 54 n + 25} + 27 n + 5\right)
\end{equation}
and expanding out $\sqrt{81 n^2 + 54 n + 25} = 9 n(1 + O(1))$, we have that:
$k \sim n$. 

Around $n$, we have that $\binom{3n - k}{k} 2^k$ approximates a symmetric function. So we need to evaluate the summation
\begin{equation}
	\frac{1}{2}\sum_{y = n - M}^{n + M} \binom{3n - y}{y} 2^y
\end{equation}
and show that the tail $\sum_{k = 0}^{n - M} \binom{3n - k}{k} 2^k$ is negligible.

\subsection{Evaluating main sum}
Now let $z = y - n$. We have that the summation becomes:
\begin{equation}
	\sum_{z = -M}^M \binom{2n - z}{n + z} 2^n 2^z = \sum_{z = -M}^M\frac{(2n - z)!}{(n + z)! (n - 2z)!} 2^{n + z}. 
\end{equation}

Now we have that by Stirling:
\begin{align*}
	(2n - z)! &\sim (2n - z)^{2n-z}/e^{2n - z} \sqrt{2 \pi (2n-z)}\\
	(n + z)! &\sim (n + z)^{n + z}/e^{n + z} \sqrt{2 \pi (n + z)}\\
	(n - 2z)! &\sim (n - 2z)^{n - 2z}/e^{n - 2z} \sqrt{2 \pi (n -2z)}
\end{align*}
so we have that the total sum becomes:
\begin{align*}
	&\sum_{z = -M}^M \frac{(2n - z)^{2n - z}}{(n + z)^{n + z} (n - 2z)^{n - 2z}} \frac{e^{n + z} e^{n - 2z}}{e^{2n - z}} \frac{\sqrt{2\pi (2n - z)}}{\sqrt{2\pi (n + z)} \sqrt{2\pi (n - 2z)}} 2^{n} 2^z \\
	&= \frac{1}{\sqrt{2\pi}}\sum_{z = -M}^M \frac{(2n - z)^{2n - z}}{(n + z)^{n + z} (n - 2z)^{n - 2z}} \sqrt{\frac{2n - z}{(n + z)(n - 2z)}} 2^{n} 2^z
\end{align*}

Now we have that when $M = o(n^{2/3})$,
\begin{align*}
	(2n - z)^{2n-z}  &=  (2n)^{2n-z} \exp(-z + z^2/4n + O(z^3/n^2)) = (2n)^{2n-z} \exp(-z + z^2/4n)\\
	(n + z)^{n + z} &= n^{(n + z)} \exp(z + 1/2 z^2/n + O(z^3/n^2)) = n^{(n + z)} \exp(z + 1/2 z^2/n) \\
	(n - 2z)^{n - 2z} &= n^{n -2z} \exp\left( (n - 2z) \log(1 - 2z/n)\right)\\
	&= n^{n -2z} \exp\left( (n - 2z) (-2z/n - 2z^2/n^2 + O(z^3/n^3))\right)\\
	&= n^{n -2z} \exp\left( -2z + 2z^2/n + O(z^3/n^2)\right) = n^{n -2z} \exp( -2z + 2z^2/n)
\end{align*}
Therefore, we have that the sum becomes:
\begin{align*}
	&\frac{1}{\sqrt{2\pi}}\sum_{z = -M}^M \frac{(2n)^{2n-z} \exp(-z + z^2/4n) }{n^{(n + z)} \exp(z + 1/2 z^2/n) n^{n -2z} \exp( -2z + 2z^2/n) } \sqrt{\frac{2n - z}{(n + z)(n - 2z)}} 2^{n} 2^z\\
	&= \frac{1}{\sqrt{2\pi}}\sum_{z = -M}^M 2^{2n-z} \exp(- 9z^2/4n) \sqrt{\frac{2n - z}{(n + z)(n - 2z)}} 2^{n} 2^z\\
	&= \frac{2^{3n}}{\sqrt{2\pi}} \sum_{z = -M}^M  \exp(- 9z^2/4n) \sqrt{\frac{2n - z}{(n + z)(n - 2z)}}
\end{align*}
We have that as $z = o(n^2/3)$, then
\begin{align*}
	\sqrt{\frac{2n - z}{(n + z)(n - 2z)}} &= \sqrt{1/(n + z) + 1/(n - 2z)}\\
	 &= \frac{1}{\sqrt{n}} \sqrt{1/(1 + z/n) + 1/(1 - 2z/n)}\\
	 &= \frac{1}{\sqrt{n}} \sqrt{1 -z/n + 1 + 2z/n + O(z^2/n^2)}\\
	 &= \frac{1}{\sqrt{n}} \sqrt{2 + o(1)}\\
	 &= \frac{\sqrt{2}}{\sqrt{n}} 
\end{align*}
Therefore, we have that the summation is:
\begin{equation}
	\frac{2^{3n}}{\sqrt{\pi n}} \sum_{z = -M}^M  \exp(- 9z^2/4n) 
\end{equation}
and using the fact that $ M = n^{2/3}$, we can use the fact that $\alpha(n) = 9/(4n)$ and we have that $M \alpha \rightarrow 0$, $M \sqrt{\alpha} \rightarrow \infty$ to yield:
\begin{equation}
	\frac{2^{3n}}{\sqrt{\pi n}} \sum_{z = -M}^M  \exp(- 9z^2/4n)  = \frac{2^{3n}}{\sqrt{\pi n}} \sqrt{\frac{4 \pi n}{9}} = \frac{2}{3} 2^{3n}
\end{equation}
Therefore, we have that the summation:
\begin{equation}
	\sum_{y = n - M}^n \binom{3n - y}{y} 2^y = \frac{1}{2} \sum_{y = n - M}^{n + M}\binom{3n - y}{y} 2^y = \frac{2^{3n}}{3}
\end{equation}
\subsection{Tail is negligible}
To show that the tail is negligible, we want to show
\begin{equation}
	\sum_{k = 0}^{n^{2/3}}\binom{3n - k}{k} 2^k = o(S)
\end{equation}
where $S$ is the sum above. 
We have that at $n^{2/3}$, the term above is at most $\left(\frac{e(3n - n^{2/3})}{n^{2/3}}\right)^{n^{2/3}} 2^{n^{2/3}}$, as we have that $\binom{n}{k} \leq \left(\frac{n \cdot e}{k}\right)^k$ (from Wikipedia). Now we have that this is at most $(3e \cdot n^{1/3})^{n^{2/3}} 2^{n^{2/3}} = (6e)^{n^{2/3}} n^{n^{2/3}/3}$, so the full summation is bounded above by:
\begin{equation}
	(6e)^{n^{2/3}} n^{n^{2/3}/3} n^{2/3}.
\end{equation} 
But notice that $(2^n)^3$ bounds the product of each of these terms as when taking logs of both equations, we have a $3n$ term out but the largest term out is a $ n^{2/3} \log n$ which is asymptotically much smaller. Therefore, we have that the term above is of $o(S)$. Therefore, the tail is negligible. 
\section{Question 8}
\subsection{a}
We shall use cards, decks and hands. Each card will be a standard deck with a number in $[n]$ in the centre. Now we have that the number of decks is $n$ for each $n \geq 1$, therefore the deck enumerator is $D(x) = \sum_{n \geq 1} n \frac{x^n}{n!} = \sum_{n \geq 1} \frac{x^n}{(n - 1)!} = x e^x$. 
Therefore, the egf for the number of labelled galaxies is $f(x) = e^{D(x)} = e^{x e^x}$. 

\subsection{b}
Now we have that for all $r > 0$, $r \in \mathbb{R}$, we have that:
\begin{equation}
	[z^n] f(z) \leq r^{-n} f(r).
\end{equation}
To find the minimum, take $D \log$ of $r^{-n} f(r)$. We have that this is $D(-n \log r + r e^r) = 0$, or 
\begin{equation}
	-n/r + e^r + r e^r = 0.
\end{equation}
Therefore, we have that:
\begin{equation}
	-n + r e^r + r^2 e^r = 0
\end{equation}
or that:
\begin{equation}
	r (r + 1) e^r = n.
\end{equation}
Therefore, the saddlepoint upper bound of $[x^n] f(x)$ is at $r(n)$, so the upper bound of $[x^n] f(x)$ is
\begin{equation}
	r(n)^{-n} e^{r(n) e^{r(n)}}
\end{equation}
 where $r(n)$ is defined in the question. 

\subsection{c}
We will evaluate the integral $[z^n] f(z) = \frac{1}{2 \pi i }\oint_C f(z)/z^{n + 1} dz =  \frac{1}{2\pi}\int_{-\pi}^{\pi} g(\theta) \, dx$ where $ g(\theta) := \frac{f(r e^{i \theta})}{r^n e^{i n \theta}}$.

Taking the log of $g$, we have that:
\begin{equation}
	\log(g(\theta)) = (r e^{i \theta + r e^{i \theta}}) - n \log r - i n \theta. 
\end{equation}
We have that $e^{i \theta} =  1 + i \theta - \frac{\theta^2}{2} + O(\theta^3)$. So we have that:
\begin{equation}
	\log(g(\theta)) = (r e^{i \theta + r ( 1 + i \theta - \frac{\theta^2}{2} + O(\theta^3))}) - n \log r - i n \theta. 
\end{equation}

Expanding out, we get that:
\begin{equation}
	 (r e^{i \theta + r ( 1 + i \theta - \frac{\theta^2}{2} + O(\theta^3))}) = r e^r e^{(r + 1) i \theta- \frac{\theta^2 r}{2} + O(r\theta^3)}.
\end{equation}
Expanding out, we have that:
\begin{equation}
	e^{(r + 1) i \theta- \frac{\theta^2 r}{2} + O(r\theta^3)} = (1 + (r + 1) i \theta - \frac{\theta^2 r}{2} - \frac{(r + 1)^2 \theta^2}{2} + O(r^2 \theta^3))
\end{equation}
Therefore, we have:
\begin{equation}
	(r e^{i \theta + r ( 1 + i \theta - \frac{\theta^2}{2} + O(\theta^3))}) = r e^r + r(r + 1) e^r i \theta - r e^r \frac{r + (r + 1)^2 }{2} \theta^2 + O(r^2 \theta^3).
\end{equation}
We have that $r(r + 1) e^r = n$, so we have that:
\begin{equation}
	(r e^{i \theta + r ( 1 + i \theta - \frac{\theta^2}{2} + O(\theta^3))}) = r e^r + n i \theta - \frac{r^2 e^r + (r + 1) n}{2} \theta^2 + O(r^2 \theta^3).
\end{equation}
Therefore, we have that
\begin{equation}
	\log(g(\theta)) = r e^r - \frac{r^2 e^r + (r + 1) n }{2} \theta^2 - n \log r + O(r^2 \theta^3).
\end{equation}
Therefore, we have that from integrating $e^{\log g(\theta)}$ between $-n^{-2/5}$ and $n^{-2/5}$, 
\begin{align*}
	\frac{1}{2 \pi}\int_{-n^{-2/5}}^{n^{-2/5}} e^{\log g \theta} \, d\theta &= \frac{e^{r e^r}}{r^n} \int_{-n^{-2/5}}^{n^{-2/5}} e^{- \frac{r^2 e^r + (r + 1) n }{2} \theta^2 + o(1)} \, d\theta\\
	&=  \frac{e^{r e^r}}{ 2 \pi r^n} \int_{-n^{-2/5}}^{n^{-2/5}} e^{- \frac{r^2 e^r + (r + 1) n }{2} \theta^2 + o(1)} \, d\theta\\
	&= \frac{e^{r e^r}}{2 \pi r^n}\sqrt{\frac{2\pi}{r^2 e^r + (r + 1) n}}\\
	&= \frac{e^{r e^r}}{r^n \sqrt{r e^r(r + (r + 1)^2)} \sqrt{2\pi}}
\end{align*}
Inserting $r(r + 1) e^r = n$, we have that
\begin{equation}
	[z^n] f(z) = \frac{e^{r e^r}}{r^{r (r + 1) e^r} \sqrt{r e^r(r + (r + 1)^2)} \sqrt{2 \pi}}
\end{equation}

\subsection{d}
From $r (r + 1) e^r = n$ and taking logs, we have that $r = \log(n) - 2\log(r)$ approximately. If we say that $r = O(\log n)$, then as $\log \log (n)$ and the summations is negligible, then we have that $r \sim \log(n)$.

\subsection{e}
We will use the bootstrapping method to find an additive bound. 
We have that $r = \log(n) - 2 \log(O(\log n))$, so $r = \log n - O(\log \log n)$. Then we have that $r = \log(n) - 2 \log(\log (n) - O(\log \log n))$. Then we have that:
\begin{equation}
	r = \log(n) - 2 \log \log n - 2\log \left(1 -  O\left(\frac{\log \log n}{\log n}\right)\right).
\end{equation}
We have that $\log \left(1 -  O\left(\frac{\log \log n}{\log n}\right)\right) = O\left(\frac{\log \log n}{\log n}\right)$, so we have that:
\begin{equation}
	r = \log(n) - 2 \log \log n - O\left(\frac{\log \log n}{\log n}\right).
\end{equation}
As we have that $ O\left(\frac{\log \log n}{\log n}\right) = o(1)$, then we have that $g(n) = \log(n) - 2 \log \log n$. 
\section{Question 9}
\subsection{a}

We will use Darboux's method to solve this equation. We have an algebraic singularity at $z = 1/2$, so we will use this fact to calculate Darboux's method.
Let 
\begin{equation}
	b(z) = \frac{e^z \sqrt{1 - 2z}}{1 - z^3}.
\end{equation}
Then let 
\begin{equation}
	g(z) = b(z/2) = \frac{e^{z/2} \sqrt{1-z}}{1 - z^3/8}.
\end{equation}
We have that $[z^n] b(z) = 2^n [z^n] g(z)$. 

Now we have for $g(z)$ an algebraic singularity at $z = 1$. Then we can use Darboux's theorem. We have that $g(z) = \frac{e^{z/2}}{1 - z^3/8} \sqrt{1 - z}$, and we can evaluate $f(z) := \frac{e^{z/2}}{1 - z^3/8} $ around the point where $z = 1$. 
We have that $f(z) = f(1) - f'(1) (1 - z) + O((z - 1)^2)$. So we have $f(z) = \frac{8}{7}\sqrt{e} -  \frac{52}{49}\sqrt{e}(1-z) + O((1 - z)^2)$, so $g(z) = \frac{8}{7}\sqrt{e}(1 - z)^{1/2} -  \frac{52}{49}\sqrt{e}(1-z)^{3/2} + O((1 - z)^{5/2})$. 
We have that $J = 1$, $J' = 1$, $\alpha = 1/2$, $\beta = 1$. We also have $z_0 = 1$.
Then by Darboux's theorem, 
\begin{equation}
	[z^n] g(z) = \frac{8}{7}\sqrt{e} \binom{1/2}{n} (-1)^{-n} -  \frac{52}{49}\sqrt{e} \binom{3/2}{n}(-1)^n + O(n^{-7/2})
\end{equation}
Then we have that $\binom{1/2}{n} (-1)^{-n} = \frac{n^{-3/2}}{\Gamma(-1/2)}$ and  $\binom{3/2}{n} (-1)^{-n} = \frac{n^{-5/2}}{\Gamma(-3/2)}$. 
We have that $\Gamma(1/2) = -1/2 \Gamma(-1/2) = \sqrt{\pi}$, so $\Gamma(-1/2) = - 2 \sqrt{\pi}$. We additionally have that $\Gamma[-1/2] = -3/2 \Gamma[-3/2] = -2 \sqrt{\pi}$, so $\Gamma[-3/2] = 4/3 \sqrt{\pi}$. Therefore, we have that:
\begin{align*}
	[z^n] g(z) &= \frac{8}{7}\sqrt{e} \frac{n^{-3/2}}{\Gamma(-1/2)} -  \frac{52}{49}\sqrt{e} \frac{n^{-5/2}}{\Gamma(-3/2)} + O(n^{-7/2})\\
	&= -\frac{4 \sqrt{e}}{7 \sqrt{\pi}} n^{-3/2} - \frac{39 \sqrt{e}}{49\sqrt{\pi}} n^{-5/2} + O(n^{-7/2})
\end{align*}
So we have that:

\begin{equation}
	[z^n] b(z) = 2^n \left( -\frac{4 \sqrt{e}}{7 \sqrt{\pi}} n^{-3/2} - \frac{39 \sqrt{e}}{49\sqrt{\pi}} n^{-5/2} + O(n^{-7/2})\right)
\end{equation}
For an asymptotic formula, we have that:
\begin{equation}
	[z^n] b(z) \sim -\frac{4 \sqrt{e}}{7 \sqrt{\pi}} n^{-3/2} 2^n
\end{equation}
\subsection{b}
For an approximation with a relative error $o(1/n)$, we have that:
\begin{equation}
	b_n = 2^n ( a n^{-3/2} + b n^{-5/2} + O(n^{-7/2})),
\end{equation}
where $a$ and $b$ are the coefficients above. 
Now multiply and divide by $2^n(a n^{-3/2} + b n^{-5/2})$. We have that this is:
\begin{equation}
	2^n(a n^{-3/2} + b n^{-5/2}) \frac{a n^{-3/2} + b n^{-5/2}}{a n^{-3/2} + b n^{-5/2} + O(n^{-7/2})}.
\end{equation}
Dividing both by $a n^{-3/2}$, we get:
\begin{align*}
	2^n(a n^{-3/2} + b n^{-5/2}) \frac{1 + b/a n^{-1}}{1 + b/a n^{-1} + O(n^{-2})}.
\end{align*}
We then expand out to have a relative error $(1 + b/a n^{-1}) (1 -b/a n^{-1} + O(n^{-2})) = 1 + O(n^{-2})$. As $O(n^{-1/2})  = o(1/n)$, then we have what we want. 

\begin{equation}
	[z^n] b(z) = 2^n \left( -\frac{4 \sqrt{e}}{7 \sqrt{\pi}} n^{-3/2} - \frac{39 \sqrt{e}}{49\sqrt{\pi}} n^{-5/2} + O(n^{-7/2})\right)
\end{equation}
\end{document}
