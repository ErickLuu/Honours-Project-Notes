\documentclass[]{article}
\usepackage[margin=1in]{geometry}

\usepackage{amsmath}
\usepackage{amssymb}
\usepackage{amsthm}
\usepackage{url}

% Environments

\newtheorem{theorem}{Theorem}
\newtheorem{proposition}[theorem]{Proposition}
\newtheorem{corollary}[theorem]{Corollary}
\newtheorem{lemma}[theorem]{Lemma}
\newtheorem{definition}[theorem]{Definition}
\newtheorem{conjecture}[theorem]{Conjecture}

\theoremstyle{definition}
\newtheorem{example}[theorem]{Example}

\numberwithin{theorem}{section}
\numberwithin{equation}{section}

\newcommand{\ops}{\overset{\text{ops}}{\leftrightarrow}}

%opening
\title{Combinatorics Notes}
\author{Eric Luu}

\begin{document}

\maketitle
\section{Asymptotics}

We like counting things! But what about big things? How can we estimate what happens when numbers go big?

We say $a_n \sim b_n (n \rightarrow \infty)$ if $\lim_{n \rightarrow \infty} \frac{a_n}{b_n} = 1$. 
Example: $n! \sim \left(\frac{n}{e}\right)^n \sqrt{2 \pi n}$. 

We say $a_n = O(b_n)$ if there is $c$ such that $|a_n| \leq c |b_n|$ when $n$ is sufficiently large. 

We say $a_n = o(b_n)$ if $\frac{a_n}{b_n} \rightarrow 0$ as $n \rightarrow \infty$. 

Note that $=$ is not the same ``equals'' that we're used to.

Simple case:
Let $c_n = n - o(n)$, which says ``there exists a function $g(n)$ where $g(n) = o(n)$ such that $c_n = n - g(n)$''.

\begin{lemma}[Reciprocals lemma]
	If $f(n) \rightarrow 0$ then $\frac{1}{1 + f(n)} = 1 + O(f(n))$. 
\end{lemma}

This brings us to the most important rule:
\begin{theorem}[Asymptotic equivalence]
	$f(n) \sim g(n)$ iff $f(n) = g(n)(1 + o(1))$.
\end{theorem}

\subsection{Rules for $O$}
\begin{enumerate}
	\item $O(f(n)) + O(f(n)) = O(f(n))$ and $o(f(n)) + o(f(n)) = o(f(n))$
	\item $a O(f(n)) = O(f(n))$ and $a o(f(n)) = o(f(n))$ when $a$ is constant. 
	\item Functions distribute, so $g(n) O(f(n)) = O(g(n) f(n))$ and $g(n) o(f(n)) = o(g(n) f(n))$. Furthermore, $O(g(n) f(n)) = g(n) O(f(n))$ and  $o(g(n) f(n)) = g(n) o(f(n))$. 
	\item $o(f(n)) = O(f(n))$, $o(O(f(n))) = o(f(n))$, $O(o(f(n))) = o(f(n))$. 
	\item If $f(n) \sim g(n)$ and $g(n) \sim h(n)$ then $f(n) \sim h(n)$. 
	\item If $f= O(g)$ then for all functions $h$, $h(f) = h(O(g))$. 
	\item If $f= o(g)$ then for all functions $h$, $h(f) = h(o(g))$. 
\end{enumerate}

Very important! $f_1 \sim g_1$ and $f_2 \sim g_2$ does not imply $f_1 + f_2 \sim g_1 + g_2$. 
\subsection{Taylor's theorem}
Suppose $f^{(k)}$ exists on $[0, x]$ where $x > 0$. Then:
\begin{equation}
	f(x) = \sum_{i = 0}^{k - 1} \frac{1}{i!}f^{(i)}(0) x^i + \frac{1}{k!} f^{(k)}(\eta) x^k
\end{equation}

where $0 \leq \eta \leq x$. Then if $f^(k)$ is bounded around $[0, x_1]$ for some $x_1 > 0$ then:
\begin{equation}
	f(x) = \sum_{i = 0}^{k - 1} \frac{1}{i!}f^{(i)}(0) x^i + O(x^k) \quad \text{for} \, x \in [0, x_1]
\end{equation}

\subsubsection{Special cases:}

\begin{equation}
	\frac{1}{1-x} = 1 + x + x^2 + O(x^3) \quad \text{on } [-1/2, 1/2]
\end{equation}
For $f(n) = o(1)$, $f(n) \in [-1/2, 1/2]$ for large $n$, we have that:
\begin{equation}
	\frac{1}{1-f(n)} = 1 + f(n) + f(n)^2 + O(f(n)^3) \quad \text{on } [-1/2, 1/2]
\end{equation}
or we can go to higher precision. 

Similarly:
\begin{equation}
	\log( 1 + f(n)) = f(n) - \frac{1}{2} f(n)^2 + O(f(n)^3)
\end{equation}
So $\log(1 + o(1)) = o(1)$

We also have that $e^{o(1)} = 1 + o(1)$. 

Therefore, $1 + o(1) = e^{o(1)} = 1 + o(1)$. 

\end{document}
