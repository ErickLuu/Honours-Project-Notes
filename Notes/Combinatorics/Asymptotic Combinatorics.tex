\documentclass[]{article}
\usepackage[margin=1in]{geometry}

\usepackage{amsmath}
\usepackage{amssymb}
\usepackage{amsthm}
\usepackage{url}

% Environments

\newtheorem{theorem}{Theorem}
\newtheorem{proposition}[theorem]{Proposition}
\newtheorem{corollary}[theorem]{Corollary}
\newtheorem{lemma}[theorem]{Lemma}
\newtheorem{definition}[theorem]{Definition}
\newtheorem{conjecture}[theorem]{Conjecture}

\theoremstyle{definition}
\newtheorem{example}[theorem]{Example}

\numberwithin{theorem}{section}
\numberwithin{equation}{section}

\newcommand{\ops}{\overset{\text{ops}}{\leftrightarrow}}

%opening
\title{Combinatorics Notes}
\author{Eric Luu}

\begin{document}

\maketitle
\section{Asymptotics}

We like counting things! But what about big things? How can we estimate what happens when numbers go big?

We say $a_n \sim b_n (n \rightarrow \infty)$ if $\lim_{n \rightarrow \infty} \frac{a_n}{b_n} = 1$. 
Example: $n! \sim \left(\frac{n}{e}\right)^n \sqrt{2 \pi n}$. 

We say $a_n = O(b_n)$ if there is $c$ such that $|a_n| \leq c |b_n|$ when $n$ is sufficiently large. 

We say $a_n = o(b_n)$ if $\frac{a_n}{b_n} \rightarrow 0$ as $n \rightarrow \infty$. 

Note that $=$ is not the same ``equals'' that we're used to.

Simple case:
Let $c_n = n - o(n)$, which says ``there exists a function $g(n)$ where $g(n) = o(n)$ such that $c_n = n - g(n)$''.

\begin{lemma}[Reciprocals lemma]
	If $f(n) \rightarrow 0$ then $\frac{1}{1 + f(n)} = 1 + O(f(n))$. 
\end{lemma}

This brings us to the most important rule:
\begin{theorem}[Asymptotic equivalence]
	$f(n) \sim g(n)$ iff $f(n) = g(n)(1 + o(1))$.
\end{theorem}

\subsection{Rules for $O$}
\begin{enumerate}
	\item $O(f(n)) + O(f(n)) = O(f(n))$ and $o(f(n)) + o(f(n)) = o(f(n))$
	\item $a O(f(n)) = O(f(n))$ and $a o(f(n)) = o(f(n))$ when $a$ is constant. 
	\item Functions distribute, so $g(n) O(f(n)) = O(g(n) f(n))$ and $g(n) o(f(n)) = o(g(n) f(n))$. Furthermore, $O(g(n) f(n)) = g(n) O(f(n))$ and  $o(g(n) f(n)) = g(n) o(f(n))$. 
	\item $o(f(n)) = O(f(n))$, $o(O(f(n))) = o(f(n))$, $O(o(f(n))) = o(f(n))$. 
	\item If $f(n) \sim g(n)$ and $g(n) \sim h(n)$ then $f(n) \sim h(n)$. 
	\item If $f= O(g)$ then for all functions $h$, $h(f) = h(O(g))$. 
	\item If $f= o(g)$ then for all functions $h$, $h(f) = h(o(g))$. 
\end{enumerate}

Very important! $f_1 \sim g_1$ and $f_2 \sim g_2$ does not imply $f_1 + f_2 \sim g_1 + g_2$. 
\subsection{Taylor's theorem}
Suppose $f^{(k)}$ exists on $[0, x]$ where $x > 0$. Then:
\begin{equation}
	f(x) = \sum_{i = 0}^{k - 1} \frac{1}{i!}f^{(i)}(0) x^i + \frac{1}{k!} f^{(k)}(\eta) x^k
\end{equation}

where $0 \leq \eta \leq x$. Then if $f^(k)$ is bounded around $[0, x_1]$ for some $x_1 > 0$ then:
\begin{equation}
	f(x) = \sum_{i = 0}^{k - 1} \frac{1}{i!}f^{(i)}(0) x^i + O(x^k) \quad \text{for} \, x \in [0, x_1]
\end{equation}

\subsubsection{Special cases:}

\begin{equation}
	\frac{1}{1-x} = 1 + x + x^2 + O(x^3) \quad \text{on } [-1/2, 1/2]
\end{equation}
For $f(n) = o(1)$, $f(n) \in [-1/2, 1/2]$ for large $n$, we have that:
\begin{equation}
	\frac{1}{1-f(n)} = 1 + f(n) + f(n)^2 + O(f(n)^3) \quad \text{on } [-1/2, 1/2]
\end{equation}
or we can go to higher precision. 

Similarly:
\begin{equation}
	\log( 1 + f(n)) = f(n) - \frac{1}{2} f(n)^2 + O(f(n)^3)
\end{equation}
So $\log(1 + o(1)) = o(1)$

We also have that $e^{o(1)} = 1 + o(1)$. 

Therefore, $1 + o(1) = e^{o(1)} = 1 + o(1)$. 

Take care with unbounded sums! we want to have that the summations aer bounded above to go to 1. 

\subsection{Uniformity}
We say $f_i(n) = O(g(n))$ uniformly for $i \in S_n$ if there is a $c$ and $n_0$ such that $|f_i(n)| \leq c |g(n)|$ for all $i \in S_n$ when $n \geq n_0$. 
Alternatively, we could replace $c$ with a function $h(n)$ where $h(n) \rightarrow 0$. 

\section{Example: Extracting binomial coefficients}
Let us estimate $\binom{n}{\lfloor \frac{n}{2} \rfloor}$. Suppose $a = \lfloor \frac{n}{2} \rfloor = \frac{a}{2} + O(1)$. 
We can compare $\binom{n}{a}$ with $\binom{n}{a + b}$. We have that:

\begin{equation}
	\binom{n}{a + i} = \binom{n}{a + i - 1} \frac{n - (a + i - 1)}{a + i}
\end{equation}
Let $R_i = \binom{n}{a + i}/ \binom{n}{a + i - 1} = \frac{n - (a + i - 1)}{a + i}$.

Then for $b > 0$, we have that:
\begin{equation}
	\binom{n}{a + b}/ \binom{n}{a } = \prod_{i = 1}^b R_i
\end{equation}
From above, we have that:
\begin{align*}
	R_i = \frac{n - a - i + 1}{a + i}\\
	&= \frac{n - n/2 + O(1) - i}{n/2 + O(1) + i}\\
	&= \frac{n/2 + O(1) - i}{n/2 + O(1) + i}\\
	&= \frac{1 + O(1/n) - 2i/n}{1 + O(1/n) + 2i/n}
\end{align*}
Now let $b = o(n^{2/3})$, and we have that $1 \leq 1 \leq b$. Then $2i/n + O(1/n) \rightarrow 0$ as $n \rightarrow \infty$. Using Taylor, we have that $\log(1 + x) = x + O(x^2)$, so taking the log of $R_i$ we get:
\begin{align*}
	\log R_i &= \log(1 - 2i/n + O(1/n))- \log(1 + O(1/n) + 2i/n)\\
	&= -2i/n + O(1/n) + O(i^2/n^2) - 2i/n + O(1/n) + O(i^2/n^2)\\
	&= -\frac{4i}{n} + O(1/n + i^2/n^2)\\
	&= -\frac{4i}{n} + O(1/n + b^2/n^2)\\
\end{align*}

Therefore, taking the sum of logs yields:
\begin{align*}
\sum_{i = 1}^b \log R_i &= 	\sum_{i = 1}^{b} - 4i/n + O(1/n + b^2/n^2)\\
&= \left(..\sum_{i = 1}^{b} - 4i/n\right) + O(b/n + b^3/n^2)\\
&= \frac{-2b^2 + O(b)}{n} + O(b/n + b^3/n^2)\\
&= \frac{-2b^2 + O(b)}{n} + o(1)
\end{align*}
Then $\binom{n}{a + b}/ \binom{n}{a} = \exp(-2b^2/n + o(1)) \sim e^{-2b^2/n}$. 
This gives us a uniform bound over $b$. 

\section{Asymptotics of summations}

Suppose we select a subset $T$ of $[n]$ with uniform probability. How many elements will we get?

$\mathbb{P}(|T| = k) = \frac{1}{2^n} \binom{n}{k}$ So expected number is $\mathbb{E}(|T|) = \frac{1}{2^n} \sum_{k = 0}^{n} k \binom{n}{k}$.

Let $S_n = \sum_{k = 0}^{n} k \binom{n}{k}$. Then $\sum_{k = 0} \binom{n}{k} x^k = (1 + x)^n$. Differentiating wrt with $x$, we get $\sum_{k = 1}^{n} \binom{n}{k} k x^{k - 1} = n (1 + x)^{n - 1}$. Set $x = 1$, we get $S_n = n 2^{n - 1}$, so the expected size of $|T| = n/2$. 

\subsection{Exercise}
Find
$\sum_{k = 0}^{\infty} \frac{k^2}{2^k}$.
\subsection{Number of derangements}
We have that $D_n = n! \sum_{i = 0}^{n} (-1)^i/i!$. We know $\sum_{i = 0}^{n} (-1)^i/i! = e^{-1}$ as $n \rightarrow \infty$, so $D_n = n!/e(1 + o(1))$. so $D_n \sim n!/e$. 


\section{Bonferroni inequalities}
Recall $N_=(T)$ is the number of elements with exactly $T$ properties. Then we have that for any integer $m \geq 0$:
\begin{equation}
	N_=(T) \leq \sum_{\substack{Y \supseteq T \\ |Y| \leq |T| + 2m}} (-1)^{|Y - T|}N_{\geq }(Y)
\end{equation}
and 
\begin{equation}
	N_=(T) \geq \sum_{\substack{Y \supseteq T \\ |Y| \leq |T| + 2m + 1}} (-1)^{|Y - T|}N_{\geq }(Y)
\end{equation}
Essentialy, the Main Bonneferoni Implication is:
\begin{equation}
	\left| N_=(T) - \sum_{\substack{Y \supseteq T \\ |Y| \leq |T| + k}} (-1)^{|Y - T|}N_{\geq }(Y) \right|  \leq  \sum_{\substack{Y \supseteq T \\ |Y| \leq |T| + k + 1}}N_{\geq }(Y)
\end{equation}

\subsection{Proof}
We look at the contribution that each individual element has on the whole. 
The proof relies on the fact that:
$R_a = \sum_{k = 0}^{2m} (-1)^k \binom{q}{k} \geq L_a$. 

We have that when $L_a = 1$, then $R_a = 1$. When $L_a = 0$, we want $R_a \geq 0$. 

What happens if $2m \leq q/2$? Since it is increasing then this is $\geq 1$.

If $q/2 \geq 2m \geq q$, then we pair up the $2m + k$ terms with a $k$ term to get something that is bigger than 0.

Thus the proof falls into place, and a similar proof can be used for $2m + 1$. 

\subsection{Permutations with no 3-cycles}
Let $P_{(i,j,k)}$ be the property that a permutation $\sigma$ contains $C_i$. Want no 3-cycles. set $S$ be set of all properties, $T = \emptyset$. By MBI, the estimate is:
\begin{equation}
	\sum_{\substack{Y \subseteq S}}^{Y \leq k} (-1)^{|Y|} N_\geq(Y)
\end{equation}

If we have $j$ 3-cycles which are all disjoint, then $N_\geq(Y) = (n - 3j)!$ and the number of 3-cycles with $F(Y)$ disjoint is $(n - o(n))^3/3 \sim n^3/ 3$. Therefore, step $1$ has $(n - o(n))^3/3^j \sim n^{3j}/3^j$. Thus, 

$\sum_{Y, |Y| = j} N_\geq (Y) \sim n^{3j}/3^j (n - 3j)!/j! \sim n!/3^j j!$

Then by Poisson principle, we have that $N_=(\emptyset)/n! \sim e^{- 1/3}$. 

\section{Summations}
\subsection{Poisson Paradigm}

Recall that $[x]_n = x( x - 1) (x - 2) ... (x - n + 1)$. 
\begin{theorem}[Poisson Paradigm]
	In the setting of PIE, suppose $N = N_\geq(\emptyset) = |A|$ and $M = |S|$. Then suppose for some $\lambda = \lambda(n)$, where $\lambda = O(1)$, and we have for each fixed $k \geq 0$ either:
	\begin{equation}
		\sum_{\substack{Y \supseteq T\\ |Y| = |T| + k}} N_{\geq}(Y) = N \frac{\lambda^r}{[M]_r} \left(\frac{\lambda^k}{k!} + o(1)\right)
	\end{equation}
	uniformly for all $T \subset S$ with $|T| = r$ or:
	\begin{equation}
		\sum_{T = r} \sum_{\substack{Y \supseteq T\\ |Y| = |T| + k}} N_{\geq}(Y) = N \frac{\lambda^r}{r!} \left(\frac{\lambda^k}{k!} + o(1)\right)
	\end{equation}
	
	Then:
\begin{equation}
	\sum_{\substack{Y \supseteq T\\ |Y| = |T| + k}} N_{\geq}(Y) = N \frac{\lambda^r}{[M]_r} \left(\frac{\lambda^k}{k!} + o(1)\right)
\end{equation}
\end{theorem}
\end{document}
