% !TEX root = ./thesis.tex
\begin{abstract}
	The main topics of this thesis are graph minors and graphs embedded on topological spaces. A graph $H$ is a minor of a graph $G$ if $H$ can be obtained from $G$ through vertex deletion, edge deletion and edge contraction. The class of graphs that is focused on are $K_t$ minor free graphs. The space that this thesis focuses on are books, which are half-planes joined on the boundary. An embedding of a graph on a book places the vertices of the graph on the common boundary. Edges of the graph are placed on a single page so that no two edges cross. The smallest number of half-planes necessary to embed a graph on a book is the pagenumber of the graph. It has been a long-standing conjecture that every proper minor-closed class can be embedded on a book with a bounded number of pages. This thesis surveys some results about book-embeddings. 

	This thesis applies the Graph Minor Structure Theorem, by Robertson and Seymour. This theorem states that every $K_t$ minor free graph can be built up from four ingredients. The four ingredients that the Graph Minor Structure Theorem uses are graphs embedded on surfaces, vortices on graphs, apex sets, and clique-sums. If the Graph Minor Structure Theorem is restricted to only using graphs embedded on orientable surfaces or the projective plane, then the conjecture is true. 
	
	However, the problem of bounding the pagenumber of graphs embedded on a nonorientable surface of higher genus is still open. It is open when the surface is the Klein Bottle. This thesis discusses some approaches to solving this open problem. 
\end{abstract}

% Graphs and minors
% Graphs embedded on topological spaces

% Books
% Graph embedded on Books
% Pagenumber
% Conjecture on proper minor-closed classes.
% Survey of results about book-embeddings
% Robertson and Seymour Graph Minor Structure Theorem
% Ingredients of GMST
% Orientable case
% Conjecture for Klein Bottle, open for Klein Bottles.


\epigraph{Man, in his quest for knowledge and progress, is determined and cannot be deterred.}{John F. Kennedy}
\newpage

\section{Acknowledgements}
I am grateful to David Wood for guiding and supervising me throughout this project. He has taught me everything that I know about graph theory and I would not be where I am now without his guidance. I would also like to thank fellow graph theorists Jofre Costa, Nickolai Karol, Robert Hickingbotham and Marc Distel for their support and discussions. I would also like to thank members of the Monash discrete maths community, including Ian Wanless, Nick Wormald, Daniel Horsley and Graham Farr for teaching me combinatorial and discrete mathematics and inspiring me to take this path. Additionally, I would also like to thank topologists Josh Howie, Andy Hammerlindl, Jessica Purcell and Dionne Ibarra for teaching me topology, which has been invaluable in my research. Most of all, I thank my family, friends and Diesel for supporting me through my honours year and allowing me to do my studies.
\todo{be very careful! and say what you really want to say, mention family + friends etc}

\section{Declaration}

I declare that this document is my own work and not the work of others or generative AI. All new research is contained within \cref{chap:orientable} and \cref{chap:nonorientable}. and this is the joint work of David Wood and myself.