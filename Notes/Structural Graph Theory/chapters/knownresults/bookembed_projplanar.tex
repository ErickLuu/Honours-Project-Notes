% !TEX root = ./thesis.tex

\section{Pagenumber of projective-planar graphs}
In the non-orientable case, Heath and Istrail claim to prove the below conjecture.
\begin{conjecture}\label{lem:planar_nonplanar_nonorientable}
	Suppose a graph \(G\) has a planar-nonplanar decomposition \((G_P, E_N)\) on a non-orientable surface \(\Sigma\) of genus $g$. Then \(G\) can be embedded on \(9g - 1\) pages.
\end{conjecture}
However, the proof given in their paper only holds in the case when the graph $G_P$ is a cycle, or with a bounded number of chords in the cycle. \textcite{nakamotoBookEmbeddingProjectiveplanar2015} also notices this error and proves the following theorem.

\begin{theorem}\label{thm:proj_planar_graphs_9pages}
	Every graph embedded on the projective plane has a book-embedding with $9$ pages.
\end{theorem}

This proof relies on a \textit{triangulated} projective plane graphs. This means that every face has three distinct vertices on its boundary. 

\begin{theorem}
    Every projective-planar graph is a subgraph of a triangulated projective-planar graph.
\end{theorem}

\begin{proof}
    Let $G$ be a projective-planar graph. If a face is bounded by a cycle $C$ of length $k \geq 4$, then add a vertex $v$ in the centre of the face and have every vertex in $C$ be adjacent to $v$. Suppose a face $F$ has reappearing vertices on its boundary walk, $v_1, \ldots, v_k$. Then as $G$ is simple, the reappearing vertices must be non-consecutive, and $k \geq 4$. Add a maximal outerplanar graph with $k$ vertices on $F$ with boundary $w_1, \ldots, w_k$ and add edges $v_i w_i$ and $v_i w_{i + 1}$ for $i = 1, \ldots, k$. Doing this operation on every face of $G$ is a triangulation of the projective plane, with $G$ a subgraph of this graph.  
\end{proof}

A cycle $C$ in $G$ embedded on $\Sigma$ is \textit{contractible} if $C$ in $\Sigma$ as a loop is null-homotopic. Otherwise $C$ is non-contractible. 
A \textit{link} of a vertex $v$ is the boundary of the union of the faces that touch $v$. This coincides with the definition of a link in a simplicial complex.

To prove this theorem, we prove an auxiliary lemma. 

\begin{lemma}\label{lem:proj_planar triangulation}
    Let $G$ be a triangulated projective-planar graph. Then there exists a planar spanning subgraph $G_P$ with outer cycle $B$ that is contractible, nonplanar edges $E_N$, and a non-contractible cycle $C$. Furthermore, there exists two vertices $x, y$ such that $\{x, y \} = V(C) \cap V(B)$, $xy \in E_N$ and $C - xy$ has no edges in $E_N$. 
\end{lemma}

\begin{proof}
    Let $C$ be the shortest non-contractible cycle of $G$. This exists as every face of $G$ is a disk, so if there are no non-contractible cycles in $G$, then $G$ is planar. Therefore, $G$ is not projective-planar. Let $xy$ be an edge in $C$. Let $P = C - xy$ be a path starting at $x$ and ending at $y$. 
    Number the vertices of $P = v_1, \ldots, v_m$. 

    Locally define the left hand side and right hand side of $P$. Take $r_i$ to be the vertex on the right hand of $P$ such that $v_i v_{i + 1} r_i$ bounds a face in $G$. Let $R_i$ be the right hand path from $r_i$ to $r_{i + 1}$ on the link of $v_i$, disjoint from $P$. Then let the walk $R'$ start at $v_1 r_1$, be the concatenation of all $R_i$ to $m-1$, and then go $r_{m-1} v_m$. This is a walk from $x$ to $y$ disjoint from $P$. Then take $R$ to be a path from $x$ to $y$, $R \subseteq R'$. Then repeat for $L$, the left hand side path from $x$ to $Y$. Now these three paths are disjoint. Suppose $R$ and $L$ have a common inner vertex, call it $x$. Then this means that for two vertices $x_i, x_j$, $x$ is a right hand neighbour of $x_i$ and a left hand neighbour of $x_j$. Now the cycle $(x v_i) (v_i v_{i + 1}), \ldots , (v_j x)$ non-contractible, as the cycle must pass through $C$ and thus must lie on a crosscap. However, $i, j$ is not $1$ or $m$ because we go to $r_2$ or $r_{m-1}$. Therefore, $C'$ is shorter than $C$, breaking the assumption that $C$ is the smallest non-contractible cycle.

    Now $P, R, L$ are three-internally disjoint paths from $x$ to $y$ where $P, R$ and $P, L$ are null-homotopic. As $P$ is a contractible path, the union of the two disks are also a disk, therefore $R, L$ bounds a disk $D$. If every vertex is contained in $D$, we are done. Otherwise, find a vertex $v$ not in $D$. Since triangulations are 3-connected, there are three disjoint paths $P_1, P_2, P_3$ from $v$ to $D$, ending at vertices $v_1, v_2, v_3$ respectively. Then two vertices will be distinct, suppose they are $v_1, v_2$. Then there is a path $R'$ from $v_1$ to $v_2$ on the boundary of $D$ such that $R' \cup P_1 \cup P_2$ is a contractible cycle. Then add this disk to $D$ to enlarge $D$. As this procedure can be repeated for every vertex, there exists a planar graph $G_p = D$ of $G$. 
\end{proof}

Now we will prove \cref{thm:proj_planar_graphs_9pages}.
\begin{proof}
    Let $H$ be a projective-planar graph. Let $G$ be a triangulation of $H$.
    Now apply \cref{lem:proj_planar triangulation}. Let $G_P$ be the spanning planar subgraph in $G$, let $B$ be its planar boundary, let $E_N$ be the nonplanar edges and let $C$ be the non-contractible cycle, with edges $x, y$ on $B$. $G_P$ is edge-maximal planar. Let $B_1, B_2$ be the two paths from $x$ to $y$ on the boundary of $B$. Let $D_1$ be the planar subgraph bounded by $P \cup B_1$, similarly for $D_2$. 

    From \textcite{yannakakisEmbeddingPlanarGraphs1989}, there exists a $4$-page embedding $(<_1, \sigma_1)$ of $D_1$ which preserves the cycle $P B_1$. Similarly, there exists $4$-page embedding $(<_2, \sigma_2)$ of $D_2$ which preserves the cycle $P B_2$. Combine these two embeddings along $P$ (interlace $<_1, <_2$ along $P$) to form a book-embedding $(<, \sigma)$ in $8$ pages of $G$. 

    Now for edges in $E_N$. Let $W = B_1[x, y) xy B_2(y, x]$ be a walk which starts and ends in $x$. Let $R$ be the subgraph of $G$ bounded by $W$. Note that all edges of $R$ must go from $B_1$ to $B_2$ because they all must pairwise cross. Since $G$ is a triangulation, $R$ has edges from one half to the other half. Then $R'$ is maximal outerplanar. $R'$ has a 1-page embedding with the same $(<, \sigma)$. Add all edges to a new page. For the inner pages from $B_1$ to $x$, or from $B_2$ to $y$, add to an old page. 
\end{proof}

This is a very short proof embedding a projective-planar graph in 9 pages. This upper bound was improved by \textcite{ozekiBookEmbeddingGraphs2019}. They showed that all projective-planar graphs can be embedded in 6 pages. Their proof used \textit{Tutte paths}, which are paths in planar graphs with certain properties. They tie Tutte paths with the observation in \cref{thm:4-connected_planar_ham_cycle} to find book-embeddings of every graph.  