
\section{Graph Minor Structure Theorem}\label{sec:Kt_Minor_Free}
What is the structure of \(K_t\)-minor free graphs? We shall show that we can roughly characterise all \(K_t\)-minor free graphs as graphs that are products of a series of operations. This classification comes from \cite{robertsonGraphMinorsXVI2003}.
\subsection{Clique-minor-free minor-closed families}\label{ssec:Kt_Minor_Closed_families}
In this section, we lay out important properties of graphs related to \(K_t\)-minor free graphs. This section has four different graph parameters.
\begin{itemize}
	\item Euler genus
	\item Apex vertices
	\item Treewidth
	\item Vortices
\end{itemize}
It was shown by \textcite{robertsonGraphMinorsXX2004} that every \(K_t\)-minor free graph can be built up from smaller graphs with the above parameters bounded.
\subsection{Euler Genus g graphs}
We begin our discussion of graphs on surfaces by discussing one aspect of planar graphs that we did not discuss before. 
\subsubsection{Planar graphs}\label{sssec:K_5-free_Planar}
\begin{theorem}\label{thm:K5_Free_Planar}
	If \(G\) is a planar graph, then \(G\) is \(K_5\)-minor-free.
\end{theorem}
\begin{proof}
	If \(G\) is planar with \(n\) vertices and \(m\) edges, then we have that \(m \leq 3n -6\). However, we have that \(K_5\) has \(5\) vertices and \(10\) edges, but we have that \( 10 > 3 \times 5 - 6\), so \(K_5\) is not planar. As the family of planar graphs is minor-closed, then if \(G\) is planar, then \(K_5\) is minor-free.
\end{proof}

\subsubsection{A short discussion of topology}\label{sssec:topology}
This section comes from Mohar and Thomassen's book on Graphs on Surfaces \cite{moharGraphsSurfaces2001}. A surface \(\Sigma\) is a topological space which, at every point, has a neighbourhood homeomorphic to a disk. There are three important surfaces to know- the sphere \(S^2\), the torus \(T\), and the real projective plane \(P\).
\par
We \textit{add} a \textit{handle} to a surface \(\Sigma\) by removing two disks in \(\Sigma\) and identifying the boundaries such that one goes clockwise and the other goes counterclockwise. We add a \textit{crosscap} by removing a disk in \(\Sigma\) and identifying opposite points on the boundary. We add a \textit{twisted handle} to a surface \(\Sigma\) by removing two disks in \(\Sigma\) and identifying the boundaries such that both go clockwise.
\par
\begin{definition}[Euler Genus]
	The \textit{Euler genus} of a surface \(\Sigma\) obtained from a sphere by adding \(h\) handles, \(c\) crosscaps and \(t\) twisted handles is \(2h + 2t + c\).
\end{definition}

\begin{example}
	Here are the Euler genus of some important surfaces.
	\begin{enumerate}
		\item The Euler genus of the sphere is \(0\).
		\item The Euler genus of the torus is \(2\).
		\item The Euler genus of the projective plane is \(1\). 
		\item The Euler genus of Klein bottles is \(2\). 
	\end{enumerate}
\end{example}

Note that ``genus'' and ``Euler genus'' are two distinct concepts in topology. In this paper, when we discuss genus, we will always discuss \underline{Euler genus}.

We say a surface \(\Sigma\) is \textit{orientable} if \(\Sigma\) can be obtained from \(S^2\) by only adding handles. An example of an orientable surface is the torus.

We say a surface \(\Sigma\) is \textit{non-orientable} if \(\Sigma\) can be obtained from \(S^2\) by adding at least one crosscap or twisted handle. An example of a non-orientable surface is the projective plane or the Klein bottle. 

\subsubsection{Genus-g graphs}\label{sssec:Graph_genus}

We define the \textit{Euler Genus} of a \textit{graph} \(G\) to be the smallest Euler genus \(g\) surface \(\Sigma\) such that \(G\) can be embedded on \(\Sigma\) without crossings (note that \(\Sigma\) can be orientable or nonorientable).

Let \(|F(G)|\) be the number of faces in a graph \(G\). Then we have that \(|V(G)| - |E(G)| + |F(G)| = \chi = 2 - g\). 

\begin{theorem}[Bounded genus]\label{thm:bounded_genus_kt_free}
	If \(G\) is a genus \(g\) graph, then \(G\) is \(K_t\)-minor free, where \(t > \sqrt{6g} + 4\). 
\end{theorem}
This proof mimics the above proof for planarity, but with higher dimensions. 
We can show that if \(G\) has genus \(g\), then if \(G\) has \(n\) vertices and \(m\) edges, then \(n - m + f = \chi = 2-g\), then as each face has at most 3 vertices and each edge is incident to two faces, we have that \(f \leq 2m/3\). Therefore, \(m \leq 3(n + g - 2)\), and if \(K_t\) is embeddable on a genus \(g\) graph, then \(\binom{t}{2} \leq 3 (t + g - 2)\). Thus \(t \leq \sqrt{6g} + 4\). So if a graph has genus \(g\), then it is \(K_t\)-minor-free, where \(t > \sqrt{6g} + 4\). 

\subsection{Apex vertices}\label{sssec:Apex_Vertices}
An apex vertex \(a\) is added to a graph \(G\) such that it has arbitrary edges. As such, \(a\) can \textit{dominate} all other vertices in \(G\), meaning that \(a\) can be adjacent to all vertices in \(G\). 
\begin{theorem}
	If \(G\) is \(K_t\)-minor free, \(G\) with the apex vertex \(a\) is \(K_{t+1}\)- minor free. 
\end{theorem}
\begin{proof}
	We shall prove the contrapositive. Suppose \(G + a\) has a \(K_{t + 1}\) minor. Then \(K_{t + 1}\) has a model in \(G + a\) and denote the model function as \(\rho\). Now let \(v\) be the vertex in \(K_{t + 1}\) such that \(\rho(v)\) contains \(a\). Then if we delete \(v\) from \(K_{t + 1}\) and delete all the vertices in \(\rho(v)\) from \(G\), then we have that \(K_t\) is a minor of \(G'\), where \(G'\) is \(G - \rho(v)\). But \(G'\) is a minor of \(G\), as \(G'\) does not contain \(a\). But this means that \(G\) has a \(K_t\) minor. 
\end{proof}
\subsection{Clique-sums}\label{sssec:Clique_Sums}
The \textit{\(k\)-clique-sum} of two graphs \(G\) and \(H\), denoted as \(G \# H\), is the graph obtained by performing a series of operation on the cliques of \(G\) and \(H\). We find cliques in \(G\) and \(H\), \(C_G\) and \(C_H\) respectively, such that \(C_G\) and \(C_H\) have size \(k\). Then we identify the vertices in \(C_G\) and \(C_H\) to glue \(G\) and \(H\) together. Call this new clique \(C\). We can delete some edges between vertices in \(C\). An illustration can be found in \cref{fig:clique-sum}. 

\begin{figure}[h]
	\centering
	\includesvg[width=0.5 \textwidth]{figures/Clique-sum}
	\caption{Figure of clique-sum. Public domain image from David Eppstein \cite{eppsteinCliquesum2023}.}
	\label{fig:clique-sum}
\end{figure}


\begin{lemma}
	If \(G = G_1 \# G_2\),then \(\had(G) = \max(\had(G_1), \had(G_2))\) and \(\tw(G) = \max(\tw(G_1), \tw(G_2))\).
\end{lemma}

\begin{example}\label{ex:clique_sum_genus}
	If \(G\) is the clique-sum of Euler genus \(g\) graphs, then \(G\) is \(K_{\sqrt{6g} + 5}\)-minor-free, but \(G\) possibly has unbounded genus.
\end{example}

\subsubsection{Torsos and adhesion}\label{sssec:Torsos and Adhesion}
Given a graph \(G\) and a tree-decomposition \(\tree\), the \textit{torso} of a bag \(B_x\) of \(T\) is the graph \(G\langle B_x \rangle\), obtained from \(G[B_x]\) where \(vw\) is a vertex in \(G\langle B_x \rangle\) iff \(v,w \in B_x \cap B_y\), where \(y\) is any neighbour of \(x\) in \(T\). So the set \(B_x \cap B_y\) for all neighbours \(y\) of \(x\) in \(T\) is a clique in \(G\langle B_x \rangle\).
We refer to the \textit{adhesion set} as the set \(B_x \cap B_y\). 
The \textit{adhesion} of a tree is defined as \(\max(|B_x \cap B_y|)\) where \(xy\) is an edge in \(T\).

\subsubsection{Relationship between clique-sums and treewidth}
Given \(G\) and a tree-decomposition \(\tree\), we can say that \(G\) is a clique-sum of the torsos \(G\langle B_x \rangle\) where the size of the cliques that we paste over is at most the adhesion. Note that the tree-decomposition that we use is not one which has bounded treewidth as we have been using tree-decompositions before.

Throughout the report, we will discuss decomposing graphs in the language of tree-decompositions, rather than clique-sums. This is because we can discuss the structure of the tree-decomposition which is more difficult to do with the language of clique-sums.

\subsection{Vortices}\label{sssec:vortices}
Let \(G\) be embedded on a surface \(\Sigma\), and let \(F\) be a face on \(G\). Let \(D\) be a disc in \(\Sigma\) such that \(D\) only intersects \(G\) only on vertices on the boundary of \(F\). We denote these discs as \(G\)-clean. 

Then let \(\Lambda = (x_1, x_2, \ldots, x_b)\) be a tuple of vertices on the boundary of \(F\) such that they intersect \(D\). Then we define a new graph \(H\) such that \(V(G) \cap V(H) = \Lambda\), and there is a \textit{path-decomposition} of \(H\) of bags \(B_1, B_2, \ldots B_b\) such that \(x_i \in B_i\) for all \(i\). \(H\) is denoted as a \textit{\(D\)-vortex} of \(G\). The width of a \(D\)-vortex is the width of the path above, or \(\max_i(|B_i| - 1)\). 

The reason why vortices are important are graphs like in \cref{fig:tenniscourt}. This graph is one where vortices are needed as an essential ingredient and in fact there is a vortex of pathwidth 6, with a subgraph of genus 0.

\begin{figure}[h]
	\centering
	\includesvg{figures/tenniscourt}
	\caption{An example of a family of graphs which is \(K_8\) minor free but has unbounded genus and treewidth, and apex vertices. }
	\label{fig:tenniscourt}
\end{figure}
\subsection{Robertson-Seymour Graph Minor Structure Theorem}\label{ssec:Robertson_Seymour_Graph_Structure}
The full statement of Robertson and Seymour's Graph Minor Structure Theorem \cite{robertsonGraphMinorsXVI2003}is such:
Given \(g, p, a \geq 0\), \(k \geq 1\), a graph \(G\) is \((g, p, k, a)\)- almost-embeddable if there exists an \(A \subseteq V(G)\) with \(|A| \leq a\), and there exists subgraphs \(G_0, G_1, \ldots,  G_{p'}\) of \(G\) such that:
\begin{itemize}
	\item \(G - A = G_0 \cup G_1 \cup G_2 \ldots G_{p'}\)
	\item \(p' \leq p\)
	\item There is an embedding of \(G_0\) onto a surface \(\Sigma\) of genus \(\leq g\)
	\item There exists pairwise disjoint \(G_0\)-clean discs \(D_1, D_2, \ldots, D_{p'}\) in \(\Sigma\)
	\item \(G_i\) is a \(D_i\)-vortex of width at most \(k\).
\end{itemize}

\begin{theorem}[Robertson-Seymour Graph Minor Structure Theorem for \(K_t\)-minor-free graphs]
	For all \(t\), there exists \(g, p, a \geq 0\), \(k, \ell \geq 1\), such that every \(K_t\)-minor-free graph has a tree-decomposition of adhesion \(\leq \ell\) and each torso is \((g, p, k, a)\)-almost-embeddable. We refer to the family of graphs which satisfy these constants as \(\mathcal{G}(g, p, k, a)\). 
\end{theorem}
In fact, there exists a function \(t(g, p, k, a)\) such that if a graph has a tree-decomposition of adhesion \(\leq \ell\) and each torso is \((g, p, k, a)\)-almost embeddable, then \(G\) has no \(K_t\) minor. Joret and Wood\cite{joretCompleteGraphMinors2013} found that
\begin{theorem}[Bounds on Graph Minor Structure Theorem\cite{joretCompleteGraphMinors2013}]\label{thm:graph_structure_bound_theorem}
	For all graphs \(G\),
	\(\had(G) \leq a + 48(k + 1)\sqrt{g + p} + \sqrt{6g} + 5\). Moreover, there exists an integer \(n \geq a + 1 4 k\sqrt{p + g}\) such that \(K_n\) is a minor of some \(\mathcal{G}(g, p, k, a)\) graph.
\end{theorem}

\section{Graph Minor Theorem}\label{sec:Graph Minor Theorem}
We move on to the most important and deepest theorem in structural graph theory, the Graph Minor Theorem. This proof was proven in a series of 20 papers by Robertson and Seymour, from 1983 to 2004. As part of the proof, the Graph Minor Structure Theorem was developed, as we have outlined above. 
\begin{theorem}[Graph Minor Theorem \cite{robertsonGraphMinorsXX2004}]
	Every infinite family of graphs contains two distinct graphs \(G\) and \(H\) such that \(H\) is a minor of \(G\).
\end{theorem}
Equivalently, this theorem states that:
\begin{theorem}
	Every minor-closed family of graphs can be characterised by a finite set of forbidden minors.
\end{theorem}
Importantly, graphs of bounded genus can be characterised as a set of forbidden minors.
For planar graphs, the two forbidden minors are \(K_5\) and \(K_{3,3}\). This was proven by Wagner \cite{wagnerUeberEigenschaftEbenen1937}. 
For graphs that can be embedded on a torus, there are at least 17,523 graphs which are forbidden minors, with a database maintained by Myrvold and Woodcock\cite{myrvoldLargeSetTorus2018}. 
