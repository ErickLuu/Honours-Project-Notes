\section{Graph Minor Structure Theorem}\label{sec:Kt_Minor_Free}
In this section, we discuss a coarse characterisation of \(K_t\)-minor free graphs.\ \textcite{robertsonGraphMinorsXVII1999} proved a rough characterisation of all \(K_t\)-minor free graphs. 

Each graph that is $K_t$-minor-free can be constructed from the following ingredients.
\begin{itemize}
	\item Graphs with bounded Euler genus
	\item Sets of apex vertices
	\item Graphs of bounded treewidth
	\item Sets of vortices on graphs.
\end{itemize}
It was shown by \textcite{robertsonGraphMinorsXVII1999} that every \(K_t\)-minor free graph can be built up from smaller graphs with the above ingredients.
\subsection{Euler Genus g graphs}
We begin our discussion of graphs on surfaces by discussing one aspect of planar graphs that we did not discuss before. 
\begin{theorem}\label{thm:K5_Free_Planar}
	If \(G\) is a planar graph, then \(G\) is \(K_5\)-minor-free.
\end{theorem}
\begin{proof}
	If \(G\) is planar with \(n\) vertices and \(m\) edges where $n \geq 3$, then \(m \leq 3n -6\).
	However \(K_5\) has \(5\) vertices and \(10\) edges, but  \( 10 > 3 \times 5 - 6\), so \(K_5\) is not planar. As the family of planar graphs is minor-closed, if \(G\) is planar, then $G$ is \(K_5\)-minor free.
\end{proof}

This section comes from Mohar and Thomassen's\cite{moharGraphsSurfaces2001} book on graphs on surfaces. A surface \(\Sigma\) is a topological space which, at every point, has a neighbourhood homeomorphic to a disk. There are three important surfaces to know- the sphere \(S^2\), the torus \(T\), and the real projective plane \(P\).
\par
We \textit{add} a \textit{handle} to a surface \(\Sigma\) by removing two disks in \(\Sigma\) and identifying the boundaries such that one goes clockwise and the other goes counterclockwise. We add a \textit{crosscap} by removing a disk in \(\Sigma\) and identifying opposite points on the boundary. 
\par
\begin{definition}[Euler Genus]
	The \textit{Euler genus} of a surface \(\Sigma\), obtained from a sphere by adding \(h\) handles and \(c\) crosscaps, is \(2h + c\).
\end{definition}

\begin{example}
	This is the Euler genus of some surfaces.
	\begin{enumerate}
		\item The Euler genus of the sphere is \(0\).
		\item The Euler genus of the torus is \(2\).
		\item The Euler genus of the projective plane is \(1\). 
		\item The Euler genus of the Klein bottles is \(2\). 
	\end{enumerate}
\end{example}

Note that ``genus'' and ``Euler genus'' are two distinct concepts in topology. In this paper, when we discuss genus, we will always discuss \underline{Euler genus}.

A surface \(\Sigma\) is \textit{orientable} if \(\Sigma\) can be obtained from \(S^2\) by only adding handles. An example of an orientable surface is the torus.

A surface \(\Sigma\) is \textit{non-orientable} if \(\Sigma\) can be obtained from \(S^2\) by adding at least one crosscap or twisted handle. An example of a non-orientable surface is the projective plane or the Klein bottle. 

The \textit{Euler Genus} of a \textit{graph} \(G\) is the smallest Euler genus \(g\) surface \(\Sigma\) such that \(G\) can be embedded on \(\Sigma\) without crossings (note that \(\Sigma\) can be orientable or nonorientable). 

\begin{theorem}[Euler's formula on surfaces]\label{thm:Euler_surfaces}
	Let \(|F(G)|\) be the number of faces in a graph \(G\). Then \(|V(G)| - |E(G)| + |F(G)| = \chi = 2 - g\). 
\end{theorem}

Given graphs $G$ and $H$ with genus $g_1, g_2$, it is useful to construct a new graph with genus $g_1 + g_2$. 
\begin{theorem}[\textcite{millerAdditivityTheoremGenus1987}]\label{thm:additivity_genus}
	Let graphs $G$ and $H$ have genus $g_1$, $g_2$. Then the graph obtained from identifying a vertex in $G$ to a vertex in $H$ has genus $g_1 + g_2$. 
\end{theorem}

\begin{theorem}[Bounded genus]\label{thm:bounded_genus_kt_free}
	If \(G\) is a genus \(g\) graph, then \(G\) is \(K_t\)-minor free, where \(t > \sqrt{6g} + 4\). 
\end{theorem}
\begin{proof}
	This proof mimics the above proof for planarity, but on surfaces of higher genus. 
	Suppose \(G\) has \(n\) vertices and \(m\) edges and of genus $g$. Then \(n - m + f = \chi = 2-g\), from \cref{thm:Euler_surfaces}. As at least three vertices bound each face and each edge touches exactly two faces, then \(f \leq 2m/3\). Therefore, \(m \leq 3(n + g - 2)\). If \(K_t\) is embeddable on a genus \(g\) graph, then \(\binom{t}{2} \leq 3 (t + g - 2)\). Thus \(t \leq \sqrt{6g} + 4\). So if a graph has genus \(g\), then it is \(K_t\)-minor-free, where \(t > \sqrt{6g} + 4\). 
\end{proof}

\subsection{Apex sets}\label{sssec:Apex_Vertices}
Let $G$ be a graph. A set of vertices $A \subseteq V(G)$ is an apex set if $G - A$ has some bounded parameter. Common parameters are planarity or bounded genus. 
\begin{theorem}
	Let $G$ be a graph. If \(G\) is \(K_{t + 1}\)-minor free, then for any vertex $a \in V(G)$, $G - a$ is \(K_t\)- minor free. 
\end{theorem}
\begin{proof}
	We shall prove the contrapositive. Suppose \(G\) has a \(K_{t + 1}\) minor. Then \(K_{t + 1}\) has a model in \(G\) and denote the model function as \(\rho\). Now let \(v\) be the vertex in \(K_{t + 1}\) such that \(\rho(v)\) contains \(a\). Then delete \(v\) from \(K_{t + 1}\) to form $K_t$. \(K_t\) is a minor of \(G - \rho(v)\). But \(G - \rho(v)\) is a minor of \(G - a\), as \(G - \rho(v)\) does not contain \(a\). So \(G - a\) has a \(K_t\) minor. 
\end{proof}
\subsection{Clique-sums}\label{sssec:Clique_Sums}
The \textit{\(k\)-clique-sum} of two graphs \(G\) and \(H\) is a new graph formed from both $G$ and $H$ by identifying two cliques together. The clique-sum of $G$ and $H$ is \(G \oplus_k H\), and is defined as follows. Find cliques in \(G\) and \(H\), \(C_G\) and \(C_H\) respectively, such that both \(C_G\) and \(C_H\) have size \(k\). Identify the vertices in \(C_G\) and \(C_H\) to glue \(G\) and \(H\) together, and possibly delete edges in $C_G$. An illustration can be found in \cref{fig:clique-sum}. 

\begin{figure}[h]
	\centering
	\includesvg[width=0.5 \textwidth]{figures/Clique-sum}
	\caption{Figure of clique-sum. Public domain image from David Eppstein \cite{eppsteinCliquesum2023}.}
	\label{fig:clique-sum}
\end{figure}


\begin{lemma}
	Let $t$ be an integer $\geq 1$. Let $G_1, G_2$ be two graphs with treewidth $t$. Then for all $k \leq t + 1$, $G_1 \oplus_k G_2$ has treewidth $t$. 
\end{lemma}
\begin{proof}
	Suppose $C = V(G_1) \cap V(G_2)$ be the clique that is glued over, where $|C| = k$. Let $(B_x: x \in T_1)$ be a tree-decomposition of $G_1$ of minimum width. Let $(B_x : x \in T_2)$ be a tree-decomposition of $G_2$ of minimum width. Then $C$ must appear in some bag $A_x$ and $B_y$ by \cref{lem:clique}. Let $T = T_1 \sqcup T_2$. Add a new node $u$ to $T$ and let $B_u = C$. Then add edges $ux, uy$ to $E(T)$ to form a new tree. Every vertex not in $C$ has a subtree in $T$. If $v \in C$, then the induced subgraph in $T$ is the graph $T_1(v) \cup T_2(v) \cup u$ which is a subtree. Finally, every edge in $G_1 \cup G_2$ remains in $T$. Therefore, $T$ is a tree-decomposition of $G_1 \oplus_k G_2$. The size of each bag in $T$ is still at most $t + 1$, so the treewidth of $G_1 \oplus_k G_2 \leq t$.
\end{proof}

\begin{lemma}
	Let $t$ be an integer $\geq 1$. Suppose $G$ and $H$ are graphs which are $K_t$-minor-free. Then $G \oplus_k H$ is $K_{t}$-minor free. 
\end{lemma}
\begin{proof}
	Let $C = V(G) \cap V(H)$ be the clique that is being pasted over. As $G$ and $H$ are $K_t$-minor free, then $|C| \leq k - 1$. Suppose $G \oplus_k H$ is not $K_t$-minor free. Then there exists a model $\rho: V(K_t) \rightarrow G\oplus_k H$ of $K_t$ in $G \oplus_k H$. $\rho$ cannot have its image only in $G$ or only in $H$. Suppose $\rho(x) \cap V(G) \neq \emptyset$ for all $x \in K_t$. Then contract the subgraphs which also contain vertices in $H$ to only have vertices in $G$. Then this is a $K_t$ model in $G$, which is a contradiction. However, there cannot also be vertices $x, y \in V(K_t)$ where $\rho(x)$ is in $G - C$ and $\rho(y)$ is in $H - C$ as these subgraphs are disconnected. Therefore, if $\rho(x)$ is in $G - C$, then all subgraphs have a vertex in $G$. But this means that $\rho$ cannot exist, as the two possibilities are mutually exclusive. Therefore, $\rho$ cannot exist. 
\end{proof}

\begin{corollary}\label{corr:clique_sum_genus}
	If \(G\) is the clique-sum of Euler genus \(g\) graphs, then \(G\) is \(K_{\lceil \sqrt{6g} + 5 \rceil}\)-minor-free.
\end{corollary}
The reverse does not hold. 
\begin{lemma}
	There exists graphs $G$ such that \(G\) has arbitrarily large genus, but $G$ is \(K_{6}\)-minor-free.
\end{lemma}

\begin{proof}
	Consider $n$ copies of $K_5$ and identify one vertex in every $K_5$ to a single vertex $v$ to form $G$. Then $G$ is $K_6$-minor free. However, from \cref{thm:additivity_genus}, $G$ has genus $n$. Thus, $G$ has unbounded genus. 
\end{proof}

\subsubsection{Torsos and adhesion}\label{sssec:Torsos and Adhesion}
Given a graph \(G\) and a tree-decomposition \(\tree\), the \textit{torso} of a bag \(B_x\) of \(T\) is the graph \(G\langle B_x \rangle\), with vertex set $B_x$ and edge set defined as follows: \(vw\) is an edge in \(G\langle B_x \rangle\) if and only if \(v,w \in B_x \cap B_y\), where \(y\) is any neighbour of \(x\) in \(T\). The set \(B_x \cap B_y\) for all neighbours \(y\) of \(x\) in \(T\) is a clique in \(G\langle B_x \rangle\).
The \textit{adhesion set} is the set \(B_x \cap B_y\). 
The \textit{adhesion} of a tree is defined as \(\max(|B_x \cap B_y|)\) where \(xy\) is an edge in \(T\).

Given \(G\) and a tree-decomposition \(\tree\), \(G\) is a clique-sum of the torsos \(G\langle B_x \rangle\) where the size of the cliques that we paste over is at most the adhesion of $\tree$. This holds for any arbitrary tree-decomposition.
We will discuss decomposing graphs in the language of tree-decompositions, rather than clique-sums. This is because we can discuss the structure of the tree-decomposition.

\subsection{Vortices}\label{sssec:vortices}
Let \(G\) be embedded on a surface \(\Sigma\), and let \(F\) be a face on \(G\). A disc $D$ is \textit{$G$-clean} if $D$ is a subset of some $F$ and $G \cap D$ is a tuple of vertices \(\Lambda = (x_1, x_2, \ldots, x_b)\). The ordering of $\Lambda$ is around the boundary of $D$. 
\par
Let $G$ be a graph embedded on $\Sigma$. Let $D$ be a $G$-clean disc with $G \cap D = \Lambda = (x_1, x_2, \ldots, x_b)$. A \textit{$D$-vortex} is a graph $H$ such that $V(G) \cap V(H) = \Lambda$ and there is a \textit{path-decomposition} of \(H\) of bags \(B_1, B_2, \ldots B_b\) such that \(x_i \in B_i\) for all \(i\).
\par
The reason why vortices are important are because of graphs like \cref{fig:tenniscourt}. This graph $G_n$ is one where vortices are needed as an essential ingredient. $G_n$ is $K_8$-minor free. However, $G_n$ has around $\frac{n}{3}$ $K_{3,3}$ copies, so has genus around $\frac{2n}{3}$. As there is an $n \times n$ grid minor, $G$ has treewidth at least $n$. As $G$ can be arbitrarily large, the number of apex vertices to remove to bound the treewidth and genus is arbitrarily large. However, there is a decomposition of $G_n$ into two graphs $G_0$ and $G_1$ where $G_0$ can be embedded on a surface and $G_1$ is a vortex on $G_0$ with path-width 6. 

\begin{figure}[h]
	\centering
	\includesvg{figures/tenniscourt}
	\caption{An example of an $n \times n$ \textit{tennis-court} graph $G_n$ which are \(K_8\) minor free.}
	\label{fig:tenniscourt}
\end{figure}
\subsection{Robertson-Seymour Graph Minor Structure Theorem}\label{ssec:Robertson_Seymour_Graph_Structure}
Given integers \(g, p, a \geq 0\), \(k \geq 1\), a graph \(G\) is \((g, p, k, a)\)- almost-embeddable if there exists an \(A \subseteq V(G)\) with \(|A| \leq a\), and there exists subgraphs \(G_0, G_1, \ldots,  G_{p'}\) of \(G\) such that:
\begin{itemize}
	\item \(G - A = G_0 \cup G_1 \cup G_2 \cup \ldots \cup G_{p'}\),
	\item \(p' \leq p\),
	\item there is an embedding of \(G_0\) onto a surface \(\Sigma\) of genus \(\leq g\),
	\item there exists pairwise disjoint \(G_0\)-clean discs \(D_1, D_2, \ldots, D_{p'}\) in \(\Sigma\),
	\item \(G_i\) is a \(D_i\)-vortex of width at most \(k\).
\end{itemize}

\begin{theorem}[Graph Minor Structure Theorem \cite{robertsonGraphMinorsXVI2003}]\label{thm:gmst}
	For all \(t\), there exists \(g, p, a \geq 0\) and \(k, \ell \geq 1\) such that every \(K_t\)-minor-free graph has a tree-decomposition of adhesion \(\leq \ell\) and each torso is \((g, p, k, a)\)-almost-embeddable. The  family of graphs which have tree-decomposition of adhesion $\leq \ell$ with torsos $(g, p, k, a)$-almost-embeddable is \(\mathcal{G}(g, p, k, a)\). Adhesion $\ell$ is implied. 
\end{theorem}
There exists a function \(t(g, p, k, a)\) such that if a graph has a tree-decomposition of adhesion \(\leq \ell\) and each torso is \((g, p, k, a)\)-almost embeddable, then \(G\) has no \(K_t\) minor.\ \textcite{joretCompleteGraphMinors2013} found that
\begin{theorem}[\textcite{joretCompleteGraphMinors2013}:]\label{thm:graph_structure_bound_theorem}
	For all graphs \(G \in \mathcal{G}(g, p, k, a)\),
	\(\had(G) \leq a + 48(k + 1)\sqrt{g + p} + \sqrt{6g} + 5\). Moreover, there exists an integer \(n \geq a + 1 4 k\sqrt{p + g}\) such that \(K_n\) is a minor of some graph in \(\mathcal{G}(g, p, k, a)\).
\end{theorem}

\section{Graph Minor Theorem}\label{sec:Graph Minor Theorem}
We move on to one of the most important and deepest theorems in graph theory, the Graph Minor Theorem. This was proven in a series of 23 papers by Robertson and Seymour, from 1983 to 2004. As part of the proof, the Graph Minor Structure Theorem was developed. 
\begin{theorem}[Graph Minor Theorem \cite{robertsonGraphMinorsXX2004}]
	Every infinite family of graphs contains two distinct graphs \(G\) and \(H\) such that \(H\) is a minor of \(G\).
\end{theorem}
Let $\mathcal{F}$ be a minor-closed graph family. A graph $H$ is a \textit{minimal forbidden minor} of $\mathcal{F}$ if every graph $G$ in $\mathcal{F}$ is $H$-minor-free and every proper minor $H'$ of $H$ (meaning $H'$ is not $H$) is in $\mathcal{F}$. 
The Graph Minor Theorem is equivalent to the statement:
\begin{theorem}
	Every minor-closed graph family $\mathcal{F}$ is characterised by a finite set of minimal forbidden minors $\mathcal{H}$. A graph $G$ is in $\mathcal{F}$ if and only if $G$ is $\mathcal{H}$-minor free.
\end{theorem}
Importantly, graphs of bounded genus can be characterised as a set of forbidden minors. The family of graphs that can be embedded on a fixed surface $\sigma$ is minor-closed. 
For planar graphs, the two minimal forbidden minors are \(K_5\) and \(K_{3,3}\), from \textcite{wagnerUeberEigenschaftEbenen1937}. 
The family of graphs that can be embedded on a torus are the toroidal graphs. There are at least 17,523 graphs which are forbidden minors, with a database maintained by \textcite{myrvoldLargeSetTorus2018}. $K_7$ is a toroidal graph but $K_8$ is not. An example of an embedding of $K_7$ on a torus is in \cref{fig:k7_on_torus}

\begin{figure}[h!]\label{fig:k7_on_torus}
	\centering
	\includesvg[width = 0.8\textwidth]{figures/k7 on torus.svg}
	\caption{Figure of $K_7$ embedded on a torus}
\end{figure}

\begin{lemma}
	$K_8$ is not embeddable on the torus.
\end{lemma}
\begin{proof}
	A torus has genus 2. By Euler's equation, if a graph $G$ is embedded on a torus, then $|V(G)| - |E(G)| + |F(G)| = 2 - 2 = 0$, where $|F(G)|$ counts the number of faces on the surface. Every face bounds at least three vertices and every edge touches two faces. Therefore, $|F(G)| \leq 2|E(G)|/3$. Suppose $K_8$ is embeddable on the torus. Then $|V(G)| = 8$ and $|E(G)| = 28$. Therefore, $|F(G)| = 20$. But $|F(G)| \leq 2 (28)/3 \leq 19$. Therefore, $K_8$ is not embeddable on the torus by contradiction.
\end{proof}

A graph $G$ is \textit{linkless} if $G$ has an embedding in $\mathbb{R}^3$ such that no two cycles are linked. If no embedding of $G$ has this property, then $G$ is \textit{inherently linked}. The family of linkless graphs is minor-closed. If $G$ is linkless, then contracting any edge maintains the linkless property. There are only seven minimal forbidden minors of linkless graphs, including $K_6$ and the Petersen graph, from \textcite{robertsonSachsLinklessEmbedding1995}. 

A graph $G$ is \textit{knotless} if $G$ can be embedded in $\mathbb{R}^3$ such that every cycle of $G$ is a simple knot. $G$ is \textit{inherently knotted} if this is not the case.\ \textcite{conwayKnotsLinksSpatial1983} showed that $K_7$ is inherently knotted. The family of knotless graphs is also minor-closed, since contracting any edge preserves the knotless property. $K_7$ is an example of a minimal forbidden minor. It was found by \textcite{conwayKnotsLinksSpatial1983} and \textcite{foisyIntrinsicallyKnottedGraphs2002,foisyNewlyRecognizedIntrinsically2003} that there exists at least three minimal minors, but it is unknown how many there really are.