% !TEX root = ./thesis.tex
\chapter{Introduction}\label{sec:introduction}
Structural graph theory is a fundamental topic in graph theory. Many results from structural graph theory decompose graphs, or families of graphs, into smaller graphs with bounded parameters. One of the most important theorems in structural graph theory is Robertson and Seymour's Graph Minor Theorem \cite{robertsonGraphMinorsXX2004} which states that proper minor-closed families of graphs are characterised by a finite set of forbidden minors.
The concept of the \textit{pagenumber} of a graph was introduced by Ollmann \cite{ollmannBookThicknessVarious1973} in the context of VLSI design and integrated circuitry. A \textit{book-embedding} of a graph is a way to arrange the vertices on the ``spine'' of a book and arrange the edges on ``pages'' of a book, or half-planes. The \textit{pagenumber} of a graph \(G\) is the smallest number of pages necessary in a book-embedding of \(G\).
The driving question of this report is the following:
\begin{conjecture}\label{conj:bded_had_pn}
	Given a graph \(G\) with no \(K_t\) minor, is the pagenumber of \(G\) bounded by a function of \(t\), so \(\pn(G) \leq f(t)\) for some \(f\)?
\end{conjecture}
Answering this question will yield a link between the pagenumber of a graph and the global structure of the graph. This report currently lays out the literature related to this question. We use a result in one of the papers of the Graph Minor Theorem, which is the Graph Minor Structure Theorem by \textcite{robertsonGraphMinorsXVI2003}.
In a PhD thesis, Blankenship claimed to have a proof of \cref{conj:bded_had_pn}.\cite{Blankenship-PhD03} However, this result has not been published in any journal and has not been independently verified. We aim to fill this gap in our knowledge.
We have good reason to believe \cref{conj:bded_had_pn} is true. Firstly, it was shown by \textcite{yannakakisEmbeddingPlanarGraphs1989} that all planar graphs need at most 4 pages. It was then shown by \textcite{malitzGenusGraphsHave1994} that all graphs of Euler genus $g$ have pagenumber $O(\sqrt{g})$, and it was also shown by \textcite{malitzGraphsEdgesHave1994} that graphs with $e$ edges have pagenumber pagenumber $O(\sqrt{e})$. We also have that graphs with bounded treewidth have bounded pagenumber, from \textcite{ganleyPagenumberTrees2001} and \textcite{dujmovicGraphTreewidthGeometric2007}.
\section{Plan for solving the problem}
We aim to solve the question using the Graph Minor Structure Theorem \cite{robertsonGraphMinorsXVI2003}, which describes the structure of graphs which do not contain a \(K_t\) minor.
Robertson and Seymour showed that we can build graphs with no \(K_t\) minor from smaller building blocks. We first start with a graph \(G\) embedded on a genus \(g\) surface. Then we add on \(p\) \textit{vortices} to \(G\), with \textit{pathwidth} at most \(k\). Then we add on \(a\) \textit{apex vertices} to \(G\). We say that \(G\) is \((g, p, k, a)\)-\textit{almost embeddable}. Robertson and Seymour \cite{robertsonGraphMinorsXVI2003} proved that all graphs with no \(K_t\) minor has a \textit{tree-decomposition} where every \textit{torso} is a \((g, p, k, a)\) almost-embeddable graphs, with \((g, p, k, a)\) depending on \(t\).
We have some useful results that can be paired with the Graph Minor Structure Theorem to prove this result.
\begin{itemize}
	\item From \textcite{heathPagenumberGenusGraphs1992}, all graphs of bounded genus have bounded pagenumber.
	\item From \textcite{ganleyPagenumberTrees2001}, and \textcite{dujmovicGraphTreewidthGeometric2007}, all graphs of bounded treewidth have bounded pagenumber.
	\item From \textcite{hickingbothamStackNumberCliqueSum2023}, if a graph \(G\) has a \textit{tree-decomposition} where every \textit{torso} has bounded pagenumber, then \(G\) has bounded pagenumber.
\end{itemize}
These results individually show that the constituent ingredients of the GMST have bounded pagenumber. What this report aims to do is to combine these proofs together to show that the pagenumber is bounded for all $K_t$-minor free graphs.
The most problematic combinations are showing that adding vortices on surfaces does not blow up the pagenumber. 
\subsection{Layout of report}
\begin{itemize}
	\item \cref{chap:Definitions} formally describes important definitions and concepts that will be used throughout the rest of the report.
	\item \cref{chap:Known results} discusses some known results from graph theory, including the Graph Minor Structure Theorem. We discuss some proofs related to bounded pagenumber that can be used to prove \cref{conj:bded_had_pn}.

	\item \cref{chap:Proving_The_Theorem} discusses some new avenues towards proving \cref{conj:bded_had_pn}.
\end{itemize}
Readers are expected to have an undergraduate background in graph theory.
