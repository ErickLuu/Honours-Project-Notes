% !TEX root = ./thesis.tex
\chapter{Introduction}\label{sec:introduction}
Structural graph theory is a fundamental topic in graph theory. Many results from structural graph theory decompose graphs, or families of graphs, into smaller graphs with bounded parameters. One of the most important theorems in structural graph theory is \textcite{robertsonGraphMinorsXX2004} Graph Minor Theorem which states that every proper minor-closed graph family is characterised by a finite set of forbidden minors. Furthermore, structural graph theory is a field of interest, with connections to topological graph theory, extremal graph theory and algorithmic complexity. 

Topological graph theory is also a fundamental topic in graph theory. The main questions topological graph theory aims to solve are graphs embedded on topological spaces. Topological graph theory is used in data science, optimisation, and computer science. The topological spaces that this report is concerned on are surfaces and quotient spaces of surfaces. 

% !TEX root = ./thesis.tex
\section{Problem statement}

A \textit{book-embedding} of a graph $G$ arranges the vertices of $G$ on the ``spine'' of a book and arranges the edges of $G$ on ``pages'' of a book. The \textit{pagenumber} of a graph \(G\) is the minimum number of pages necessary in a book-embedding of \(G\).
An embedding of $K_5$ in three pages is in \cref{fig:book-embedding}.

\begin{figure}[h!]
	\centering
	\includesvg[height = 0.4\textheight]{figures/3page_K5.svg}
	\caption[Three-page book-embedding of $K_5$]{Book-embedding of $K_5$ on three pages. Image by \textcite{eppsteinBookEmbedding2014}}\label{fig:book-embedding}
\end{figure}

The concept of the \textit{pagenumber} of a graph was introduced by \textcite{ollmannBookThicknessVarious1973} in the context of VLSI design and integrated circuitry. 
The driving question of this report is \cref{conj:bded_had_pn}. A family of graphs $\mathcal{F}$ is \textit{proper} if $\mathcal{F}$ is not the set of all graphs. 
\begin{conjecture}\label{lem:Minor-Closed_Pagenumber}
	Every proper minor-closed class can be embedded on a bounded number of pages.
\end{conjecture}

This is implied by \cref{conj:bded_had_pn}.

\begin{conjecture}\label{conj:bded_had_pn}
	There exists a function $f$ where every $K_t$-minor-free graph can be embedded on $f(t)$ pages.
\end{conjecture}

We begin this report by discussing basic graph theory definitions and book-embeddings.

An important result we use is the Graph Minor Structure Theorem by Robertson and Seymour \cite{robertsonGraphMinorsXVI2003}. Robertson and Seymour showed that graphs with no \(K_t\) minor can be built from smaller building blocks. This is a rough overview of the building blocks. Firstly, start with a graph \(G\) embedded on a genus \(g\) surface. Add on \(p\) \textit{vortices} to \(G\), each with \textit{pathwidth} at most \(k\). Add on \(a\) \textit{apex vertices} to \(G\). Then \(G\) is \((g, p, k, a)\)-\textit{almost embeddable}. \textcite{robertsonGraphMinorsXVI2003} proved that every graph with no \(K_t\) minor has a \textit{tree-decomposition} where every \textit{torso} is a \((g, p, k, a)\) almost-embeddable graph, with \((g, p, k, a)\) bounded by a function of \(t\). The main purpose of \cref{chap:gmst} is to explain the Graph Minor Structure Theorem in sufficient detail so that it can be applied to \cref{conj:bded_had_pn}.

In her PhD thesis, \textcite{Blankenship-PhD03} claimed to prove \cref{conj:bded_had_pn}.\footnote{
	We did not read Blankenship's thesis in the course of writing this thesis. Only at the very end did we read over her thesis to see how she handled some cases. 
	Blankenship also uses \textcite{heathPagenumberGenusGraphs1992} to do a planar-nonplanar decomposition. Apex vertices are handled the same. However, Blankenship deals with vortices differently. She uses a ``cap edges'' solution to deal with vortices to embed a graph. This is different to our approach using monochromatic paths on vortices, and using a tree-decomposition. 
	Blankenship also uses a similar theme of having some vertices being moved to the front of a book-embedding, with extra pages needed. However, her lemma was simpler than \cref{thm:clique_sum_pagenumber_bound}. Her proof relied on \textcite{heathPagenumberGenusGraphs1992} embedding graphs on surfaces of genus $g$, which is incomplete when the surface is non-orientable  
} However, this result has not been published and has not been independently verified. Furthermore, Blankenship used a result by \textcite{heathPagenumberGenusGraphs1992}. However, was lacking in detail when discussing book-embeddings of non-orientable surfaces. \textcite{nakamotoBookEmbeddingProjectiveplanar2015} agrees with this assessment of the claimed proof by \textcite{heathPagenumberGenusGraphs1992}. Heath and Istrail's proof is discussed in more detail in \cref{chap:orientable}. \textcite{nakamotoBookEmbeddingProjectiveplanar2015} provides a proof that every projective-planar graph can be embedded on nine pages. We present this proof in \cref{chap:nonorientable}. 

This honours project has two goals. The first goal is to investigate and learn more about structural graph theory. We will discuss some important machinery in structural graph theory, the Graph Minor Theorem and the Graph Minor Structure Theorem. To this end, we discuss book-embeddings and graphs with bounded tree-decompositions. The two papers that is discussed in a fair amount of detail are by \textcite{hickingbothamStackNumberCliqueSum2023} and by \textcite{ganleyPagenumberTrees2001}. \textcite{ganleyPagenumberTrees2001} prove that treewidth $k$ graphs can be embedded on $k+1$ pages. \textcite{hickingbothamStackNumberCliqueSum2023} prove that graphs with a tree-decomposition where every torso has pagenumber $k$ can be embedded on $2k^2 + 4k + 1$ pages. This is discussed in \cref{chap:book-embeddings}.

The second goal is to attempt to solve \cref{conj:bded_had_pn}. We prove that when a surface $\Sigma$ is orientable or the projective plane, every graph which is almost-embeddable on a surface of genus $g$ with $p$ vortices of depth $k$ on some faces is embeddable in $f(g, p, k)$ pages. The case when $\Sigma$ is orientable is discussed in \cref{chap:orientable} and the case when $\Sigma$ is the projective plane is discussed in \cref{chap:nonorientable}. We also provide some proof directions for the case when $\Sigma$ is the Klein bottle in \cref{chap:nonorientable}. 

We also state a conjecture that every graph embeddable on a surface of genus $g$ can be embedded on $f(g)$ pages where every face has a bounded number of monochromatic paths. If this conjecture is true, then it would imply \cref{conj:bded_had_pn}. This is discussed in \cref{chap:Future Work}.

%\subsection{Support for conjecture}
We have good reason to believe \cref{conj:bded_had_pn} is true. Firstly, \textcite{yannakakisEmbeddingPlanarGraphs1989} showed that every planar graph can be embedded on 4 pages. \textcite{heathPagenumberGenusGraphs1992} then showed that every graph of orientable genus $g$ can be embedded on $O(g)$ pages. Finally, \textcite{ganleyPagenumberTrees2001} showed that graphs with bounded treewidth have bounded pagenumber. \textcite{dujmovicGraphTreewidthGeometric2007} showed that the bound given by \citeauthor{ganleyPagenumberTrees2001} is tight.
We discuss some relevant papers that are used to prove \cref{conj:bded_had_pn}.
We aim to solve this question using the Graph Minor Structure Theorem \cite{robertsonGraphMinorsXVI2003}, which describes the structure of graphs that do not contain a \(K_t\) minor. 
We have some useful results that can be paired with the Graph Minor Structure Theorem to prove \cref{conj:bded_had_pn}.
\begin{itemize}
	\item \textcite{heathPagenumberGenusGraphs1992} showed that every graph of bounded orientable genus have bounded pagenumber.
	\item \textcite{ganleyPagenumberTrees2001} and \textcite{dujmovicGraphTreewidthGeometric2007} showed that every graph of bounded treewidth have bounded pagenumber.
	\item \textcite{hickingbothamStackNumberCliqueSum2023} showed that if a graph \(G\) has a \textit{tree-decomposition} where every \textit{torso} has bounded pagenumber, then \(G\) has bounded pagenumber.
	\item \textcite{nakamotoBookEmbeddingProjectiveplanar2015} showed that all planar-projective graphs have bounded pagenumber.
\end{itemize}
These results individually show that the constituent ingredients of the Graph Minor Structure Theorem, except surfaces with non-orientable genus at least 2, have bounded pagenumber. We summarise some relevant technology that will be used to obtain some partial results for \cref{conj:bded_had_pn}. 
The biggest hurdle is showing that adding vortices on surfaces will not blow up the pagenumber. To address this issue, we introduce a new concept when considering faces on surfaces with a fixed book-embedding, monochromatic paths. 
%\section{Layout of report}
The following list is how the rest of the report is laid out. 
\begin{itemize}
	\item \cref{sec:background} gives an overview of some basic graph-theoretic definitions. We discuss some basic terminology of graphs. We also discuss graphs embedded on the plane and on surfaces. Additionally, we also discuss the important topics of book-embeddings and graph minors and how they fit to prove \cref{conj:bded_had_pn}. 
	\item \cref{chap:gmst} discusses every component of the Graph Minor Structure Theorem and how these components combine to form the Graph Minor Structure Theorem. The components that are discussed in detail are treewidth and graphs on surfaces. We also discuss the Graph Minor Theorem. 
	\item \cref{chap:book-embeddings} discusses book-embeddings and book-embeddings of graphs of bounded treewidth. There is also a discussion of book-embeddings of graphs with a tree-decomposition with torsos of bounded pagenumber. The paper that is discussed in this section is by \textcite{hickingbothamStackNumberCliqueSum2023} and by \textcite{ganleyPagenumberTrees2001}. There is no original research in this section but much of the technology in both papers is used. 
	\item \cref{chap:orientable} discusses graphs embedded on orientable surface and a book-embedding of graphs on orientable surfaces. We first discuss a proof by \textcite{heathPagenumberGenusGraphs1992} on embedding graphs on orientable surfaces. We extend the results in \textcite{heathPagenumberGenusGraphs1992} to include graphs embedded on orientable surfaces with vortices attached. 
	\item \cref{chap:nonorientable} discusses graphs embedded on nonorientable surfaces. There is a discussion of a proof by \textcite{nakamotoBookEmbeddingProjectiveplanar2015} with embedding projective planar graphs with a bounded number of pages. We extend this result to graphs embedded on projective planes with vortices attached. There is also a discussion of the Klein Bottle case and some discussion on its difficulty. Finally, we discuss a conjecture involving non-orientable surfaces that can be used to embed any $K_t$-minor free graph on a bounded number of pages. 
	\item \cref{chap:conclusion} discusses some consequences of \cref{conj:bded_had_pn} if it was proven, and discusses Blankenship's PhD in more detail. There are many similarities between the concepts that she used and the ones used in this thesis, although we did not read her thesis in our proof. This chapter finishes with a conclusion to the thesis. 
\end{itemize}

Readers are expected to have at least an undergraduate understanding in graph theory and point-set topology. 

\newpage
\section{Background}\label{sec:background}
This section is a brief overview of some concepts in topological graph theory and structural graph theory. The concepts discussed are planar graphs, graphs embedded on surfaces, graphs on books, graph minors and the Graph Minor Structure Theorem. The Four Colour Theorem and the Map Colour Theorem are also discussed. These were conjectures for a very long time before being solved in the 1970s. The attempts to prove these theorems helped motivate the study of graphs on surfaces.

\subsection{Basic definitions}\label{sec: Basic definitions}
A graph $G$ is a pair of sets; a vertex set $V(G)$ and an edge set $E(G)$ that contains two-element subsets of $V(G)$. An edge $ \{v, w\}$ \textit{joins} vertices $v$ and $w$. A graph is \textit{simple} if all edges join two distinct vertices and there is at most one edge between any two vertices. In this paper, every graph is simple unless stated. Graphs which are not simple are \textit{multigraphs}. Furthermore, every graph is finite, meaning that the vertex set is finite. The graph with all possible edges on $n$ vertices is the \textit{complete graph} $K_n$. Graphs are defined up to isomorphism, or up to relabelling of the vertices.
Throughout this report, the set $\lbrace 1\ldots n \rbrace$ is notated as $[n]$. 
A graph \(G\) is \(k\)-colourable if there exists a function \(f: V(G) \rightarrow [k]\) such that if $f(v) = f(w)$, then $v$ and $w$ do not share an edge. The \textit{chromatic number} \(\chi(G)\) is the smallest \(k\) such that \(G\) is \(k\)-colourable. 

Let $G$ be a graph. A \textit{subgraph} $H$ in $G$ is a graph with vertex set $V(H) \subseteq V(G)$ and edge set $E(H) \subseteq E(G)$. The statement ``$H$ is a subset of $G$'' is notated as $H \subseteq G$.
Let $G$ be a graph and let $S$ be a non-empty subset of the vertex set of $G$. The \textit{induced subgraph} of $S$ in $G$ is the graph $G[S]$ with vertex set $S$ and edge set containing precisely all edges in $G$ incident to two vertices in $S$. Removing a set of vertices $S \subseteq V(G)$ from $G$ forms the induced subgraph $G - S := G[V(G) - S]$. 
$H$ is a \textit{spanning subgraph} of $G$ if $H$ is a subgraph of $G$ and $V(H) = V(G)$. 
The \textit{neighbourhood} of a set of vertices $A \subseteq V(G)$ is the set $N_G(A)$ containing vertices $x \in V(G) - A$ where $xy$ is an edge for some $y \in A$. A \textit{clique} is a subgraph isomorphic to a complete graph. 

\subsection{Bounding the number of pages of a planar graph}
This subsection uses \cref{lem:planar_graphs_4_connected_cliqesums} and \cref{thm:clique_sum_pagenumber_bound} to find a book-embedding of planar graphs. This book-embedding is different from the one provided by \textcite{yannakakisEmbeddingPlanarGraphs1989} as it does not require a planar triangulation. Because of this fact, this proof is used in future sections with respect to adding vortices on faces. 

Firstly, \textcite{tutteTheoremPlanarGraphs1956} proves an important theorem regarding $4$-connected planar graphs.

\begin{theorem}\label{thm:4-connected_planar_ham_cycle}
	Every 4-connected planar graph is Hamiltonian.
\end{theorem}

\cref{thm:4-connected_planar_ham_cycle} is used to prove \cref{thm:Planar Graph Hickingbotham Bound}.

\begin{corollary}\label{thm:Planar Graph Hickingbotham Bound}
	Let \(G\) be a 2-connected planar graph. Then $G$ can be embedded on $11$ pages, with book-embedding $(<, \rho)$. $<$ restricted to the vertices outer cycle $C$ traverses every vertex in order of the traversal of the boundary of the outerface. Furthermore, for every face cycle $C$, $<_{V(C) - \{u, v, w\}} = C - \{u, v, w\}$ for some vertices $u$, $v$, $w$. 
\end{corollary}
\begin{proof}
	From \cref{thm:4-connected_planar_ham_cycle}, every $4$-connected planar graph is Hamiltonian. Furthermore, $K_4$ is a Hamiltonian planar graph.
	Recall that Hamiltonian planar graphs can be embedded on two pages, from \cref{lem:Pagenumber_2}. 
	Then apply \cref{thm:clique_sum_pagenumber_bound} with tree-decomposition from \cref{lem:planar_graphs_4_connected_cliqesums} to $G$. Then $G$ can be embedded on \(2 \cdot 4 + 3 = 11\) pages. Furthermore, the embedding restricted a face boundary preserves the ordering of the facial walk.

	From the construction given in \cref{lem:planar_graphs_4_connected_cliqesums}, every $4$-connected component is glued to another $4$-connected component by a separating face. Therefore, every face only changes by $3$ vertices, from \cref{thm:clique_sum_pagenumber_bound}. Therefore, removing $3$ vertices from every face preserves the cyclic ordering of every face.
\end{proof}



\section{Surfaces and graphs on surfaces}

The terminology in this section is based on \textcite{moharGraphsSurfaces2001} Graphs on Surfaces. An \textit{$n$-manifold} $M$ is a second-countable Hausdorff space where every point in $M$ has an open neighbourhood homeomorphic to an open ball in $\mathbb{R}^n$. A surface is a compact $2$-manifold. 

\textit{Handles} are added to a surface \(\Sigma\) by removing two disks in \(\Sigma\) and identifying the boundaries such that one goes clockwise, and the other goes counter-clockwise. \textit{Crosscaps} are added to a surface $\Sigma$ by removing a disk in \(\Sigma\) and identifying opposite points on the boundary. Every surface is homeomorphic to a sphere with $m$ handles and $n$ crosscaps. This is known as the classification of surfaces. The \textit{Euler genus} of a surface \(\Sigma\) with $m$ handles and $n$ crosscaps is $2m + n$.

Furthermore, a sphere with one handle and one crosscap is homeomorphic to a sphere with three crosscaps. Therefore, any sphere with a mix of handles and crosscaps is homeomorphic to one with all crosscaps. Euler genus is an invariant under homeomorphism. 

These are the Euler genus of some surfaces.
\begin{enumerate}
	\item The Euler genus of the sphere is \(0\).
	\item The Euler genus of the torus is \(2\).
	\item The Euler genus of the projective plane is \(1\). 
	\item The Euler genus of the Klein bottle is \(2\). 
\end{enumerate}

Note that ``genus'' and ``Euler genus'' are two distinct concepts. In many works, ``genus'' refers to the orientable genus. 

The orientability of a surface is an important tool to distinguish surfaces. A surface \(\Sigma\) is \textit{orientable} if \(\Sigma\) can be obtained from \(S^2\) by only adding handles. An example of an orientable surface is the torus. A surface \(\Sigma\) is \textit{non-orientable} if \(\Sigma\) can only be obtained from \(S^2\) by adding at least one crosscap. An example of a non-orientable surface is the projective plane or the Klein bottle. Compact orientable surfaces can be embedded on $\mathbb{R}^3$, but non-orientable surfaces cannot.

An \textit{embedding} of $G$ on a surface $\Sigma$ is a drawing of $G$ on $\Sigma$ such that no two edges cross. 
A \textit{$2$-cell embedding} of a graph $G$ on a surface $\Sigma$ is an embedding of $G$ in $\Sigma$ such that every connected component of $\Sigma - G$ is homeomorphic to an open disk. This is also referred to as a \textit{map}.

We now discuss some terminology of graphs on surfaces. Let $G$ be a graph and $\Sigma$ be a surface. Embedding $G$ in $\Sigma$ is referred to as the \textit{embedding} of $G$ in $\Sigma$, whereas embedding $G$ in a book is referred to as the \textit{layout} of $G$. The orientable genus of a graph \(G\), denoted \(\gamma(G)\), is the minimum genus of an orientable surface $\Sigma$ such that $G$ has an embedding on $\Sigma$. The non-orientable genus of a graph \(G\), denoted \(\tilde{\gamma}(G)\), is the minimum genus of a non-orientable surface $\Sigma$ such that $G$ has an embedding on $\Sigma$. 
The \textit{Euler Genus} of a \textit{graph} \(G\) is the smallest Euler genus \(g\) surface \(\Sigma\) such that \(G\) can be $2$-cell embedded on $\Sigma$.

\textcite{moharOrientableGenusGraphs1998} showed that \(\tilde{\gamma}(G) \leq 2 \gamma(G) + 1\) for every graph, meaning that if the orientable genus is bounded, then the non-orientable genus is bounded.\ \textcite{auslanderImbeddingGraphsManifolds1963} showed that there exists graphs which are embeddable on the projective plane that have arbitrarily large orientable genus. 

An extension for Euler's formula is below. Suppose $G$ is $2$-cell embedded on a surface $\Sigma$ of genus $g$. Let \(|F(G)|\) be the number of faces in a graph \(G\). Then \(|V(G)| - |E(G)| + |F(G)| = 2 - g = \chi\). When $g = 0$, then $\Sigma$ is a $2$-sphere and this is the original Euler's formula. 
The value $\chi$ is known as the \textit{Euler characteristic} of a topological space, in this case a surface. The Euler characteristic is invariant under homeomorphism. Calculating the Euler characteristic of any space is done through \textit{homological algebra}, specifically by looking at the free rank of homology groups. 

Graphs that can be embedded on the plane are called \textit{planar} graphs. Graphs that can be 2-cell embedded on the torus are called \textit{toroidal} graphs, and graphs that can be 2-cell embedded on the projective plane are called \textit{projective-planar} graphs. Graphs that can be 2-cell embedded on a surface of genus $g$ are called \textit{genus $g$} graphs. Similarly to plane graphs, \textit{torus graphs} are graph drawings on the torus, and \textit{projective-plane graphs} are graphs drawings on the projective plane. 

The family of graphs embeddable on a fixed surface $\Sigma$ is a minor-closed family. If $G$ is embedded on $\Sigma$, then $G - v$ for any vertex $v$ and $G - e$ for any edge $e$ is also embeddable on $\Sigma$. Furthermore, contracting any edge $e$ in $G$ maintains the property that no two edges cross. Edge contraction is a topological action on a graph and can be viewed as an ambient isotopy of $G$ on $\Sigma$. 
If $G$ is 2-cell embedded on a surface $\Sigma$ and every face in $G$ has three distinct vertices on its boundary, then $G$ is a \textit{triangulation} of $\Sigma$. Given graphs $G$ and $H$ with genus $g_1, g_2$,a new graph with genus $g_1 + g_2$ can be constructed.
\begin{theorem}[\textcite{millerAdditivityTheoremGenus1987}]\label{thm:additivity_genus}
	Let graphs $G$ and $H$ have genus $g_1$, $g_2$. Then the graph obtained from identifying a vertex in $G$ to a vertex in $H$ has genus $g_1 + g_2$. 
\end{theorem}

Next is an extension of \cref{thm:K5_Free_Planar} for graphs embedded on surfaces. 

\begin{theorem}\label{thm:bounded_genus_kt_free}
	If \(G\) is an Euler genus \(g\) graph, then \(G\) is \(K_t\)-minor free, where \(t > \sqrt{6g} + 4\). 
\end{theorem}
\begin{proof}
	This proof mimics the above proof for planarity, but on surfaces of higher genus. 
	Suppose \(G\) has \(n\) vertices and \(m\) edges and of Euler genus $g$. Then \(n - m + f = \chi = 2-g\), from Euler's theorem on surfaces. As at least three vertices bound each face and each edge touches exactly two faces, then \(f \leq 2m/3\). Therefore, \(m \leq 3(n + g - 2)\). If \(K_t\) is embeddable on a genus \(g\) graph, then \(\binom{t}{2} \leq 3 (t + g - 2)\). Thus \(t \leq \sqrt{6g} + 4\). So if $G$ has genus \(g\), then $G$ is \(K_t\)-minor free, where \(t > \sqrt{6g} + 4\). 
\end{proof}

In the case when the surface is a torus, $K_7$ is a toroidal graph but $K_8$ is not. An example of an embedding of $K_7$ on a torus is in \cref{fig:k7_on_torus}.

\begin{figure}[h!]
	\centering
	\includesvg[height = 0.3\textheight]{figures/k7 on torus.svg}
	\caption[Toroidal graph]{An example of a toroidal graph $K_7$ embedded on a torus.}\label{fig:k7_on_torus}
\end{figure}

\begin{proposition}
	$K_8$ is not embeddable on the torus.
\end{proposition}
\begin{proof}
	A torus has genus 2. By Euler's equation, if a graph $G$ is embedded on a torus, then $|V(G)| - |E(G)| + |F(G)| = 2 - 2 = 0$, where $|F(G)|$ counts the number of faces on the surface. Every face bounds at least three vertices and every edge touches two faces. Therefore, $|F(G)| \leq 2|E(G)|/3$. Suppose $K_8$ is embeddable on the torus. Then $|V(G)| = 8$ and $|E(G)| = 28$. Therefore, $|F(G)| = 20$. But $|F(G)| \leq 2 (28)/3 \leq 19$. Therefore, $K_8$ is not embeddable on the torus.
\end{proof}

A famous theorem involving map colourings on surfaces is Heawood's conjecture, from \textcite{heawoodMapcolourTheorem1890}. This theorem is also called the Map Colour Theorem. Piecewise linearly partition a surface $\Sigma$ into path-connected faces homeomorphic to a disk. Then a \textit{map} of $\Sigma$ is the graph obtained by placing a vertex at each face and placing an edge when two faces touch at a line. Heawood showed that the minimum number of colours sufficient to colour all Euler genus $g$ maps when $g \geq 1$ is
	\begin{equation*}
		\gamma(g) := \left\lfloor 
		\frac{7 + \sqrt{1 + 24g}}{2}
		\right\rfloor.
	\end{equation*}
When $g = 0$, this is the Four-Colour theorem, which was unproven when \textcite{ringelMapColorTheorem1974} was written.  
However, Heawood did not show that $\gamma(g)$ is necessary, which became Heawood's conjecture. 
Ringel and Young \cite{ringelMapColorTheorem1974} showed that for almost every case, $\gamma(g)$ is also necessary, and proved Heawood's conjecture. The case where this does not hold is the Klein-bottle case. Every Klein-bottle graph is $6$-colourable, but $\gamma(g) = 7$. 

Let $I = [0, 1]$.
A \textit{loop} is a continuous function $\gamma : I \rightarrow X$ where $\gamma(0) = \gamma(1) = x_0$. The point $x_0$ is the \textit{base point}. A \textit{homotopy} between two loops $\alpha, \beta$ is a continuous map $h : I \times I \rightarrow (x)$ where $h(0, t) = h(1, t) = x$ for all $t$, $h(\cdot, 0) = \alpha$, $h(\cdot, 1) = \beta$. A \textit{null-homotopic} loop is a loop homotopy to the constant map at $x_0$, and a \textit{nontrivial} loop is one that is not null-homotopic. On a sphere or plane, all loops are null-homotopic. Homotopic and null-homotopic loops come up in our discussion of graphs on surfaces as they can be used to classify edges embedded on a surface when the graph is a single point $x_0$. 


\subsection{Book-Embeddings}
A \textit{book-embedding} of a graph \(G\) is an embedding of \(G\) on a book. The \textit{pagenumber} of a graph \(G\) is the minimum number of pages of a book required to embed \(G\).
An embedding of $K_5$ in three pages is in \cref{fig:book-embedding}.

\begin{figure}[h!]
	\centering
	\includesvg[height = 0.4\textheight]{figures/3page_K5.svg}
	\caption[Three-page book-embedding of $K_5$]{Book-embedding of $K_5$ on three pages. Image by \textcite{eppsteinBookEmbedding2014}}\label{fig:book-embedding}
\end{figure}

\section{Graph minors}\label{sec:Graph Minors}
A graph \(H\) is a \textit{minor} of a graph \(G\) if a graph isomorphic to \(H\) can be obtained from \(G\) by deleting vertices, deleting edges, and \textit{contracting} edges. Let $G$ be a graph and let $uv$ be an edge in $E(G)$. To \textit{contract} \(uv\), we delete both \(u\) and \(v\) and create a new vertex \(w\) with neighbourhood \(N(w) = N_G(u) \cup N_G(v)\). The graph obtained after contracting the edge \(uv\) in $G$ is written as \(G\setminus uv\).
The statement ``\(H\) is a minor of \(G\)'' is written as \(H \leq G\). A graph \(G\) is \textit{\(H\)-minor-free} if $H$ is not a minor of $G$. A family of graphs \(\mathcal{F}\) is \textit{minor-closed} if when $G$ is in \(\mathcal{F}\) and \(H \leq G\), then $H$ is in \(\mathcal{F}\).
An example of a minor-closed class is the class of planar graphs.
An important class of graph families are the \(K_t\)-minor free graphs. For a graph \(G\), we define \(\had(G)\) to be the largest \(t\) such that \(K_t\) is a minor of \(G\). This is named after Hugo Hadwiger and his most famous conjecture.

\begin{conjecture}[Hadwiger's conjecture]\label{conj:Hadwiger's Conjecture}
	For all graphs \(G\), \(\chi(G) \leq \had(G)\)\cite{hadwigerUeberKlassifikationStreckenkomplexe1943}.
\end{conjecture}
Much work has been done on solving Hadwiger's conjecture, with a document by \textcite{seymourHadwigerConjecture2016} on the latest progress. However, it remains unsolved. 

A \textit{model} of a graph \(H\) in a graph \(G\) is a function $\rho$ which assigns to \(H\) vertex-disjoint connected subgraphs of \(G\). If $uv$ is an edge in \(E(H)\), then some edge in \(G\) joins the two subgraphs \(\rho(u)\) and \(\rho(v)\). A description of a model is in \cref{fig:model_of_P5}.
\begin{figure}[h!]\label{fig:model_of_P5}
	\centering
	\includesvg[width = 0.8\textwidth]{figures/model.svg}
	\caption{Illustration of a model $H$ in a graph $G$. The coloured boxes are the connected subgraphs contracted to a single vertex on the right.}
\end{figure}

\begin{theorem}
	\(H\) is a model of \(G\) if and only if $H$ is a minor of $G$.
\end{theorem}

\begin{proof}
	From \textcite{norinMath599GraphMinors2017}. Suppose \(H\) is a model of \(G\). Then for all \(x\) in \(V(H)\), contract \(\rho(x)\) in \(G\) to a single vertex. This is a well-defined operation as the image $\rho(x)$ is connected and disjoint from all $\rho(y)$ where $y$ is a distinct vertex in $H$. Then delete edges to form \(H\).

	Suppose \(H \leq G\). Use induction to show that \(H\) has a model in \(G\). Suppose \(H\) is obtained from \(G\) by contraction operations only. We can assume this by taking a subgraph of \(G\) if necessary. Let \(uv\) be the first contracted edge and let \(G' := G \setminus uv\). Let \(w\) be the vertex obtained after contracting \(uv\). Then by induction, there is a model \(\rho\) of \(H\) in \(G'\). Then find $x \in V(H)$ such that $w \in V(\rho(x))$. If there is no such $x$, then it is obvious that $\rho$ is a model of $H$ in $G$. Otherwise, 
	delete \(w\) from \(V(\rho(x)) \) and add $u, v$ to $V(\rho(x))$, the edge $uv$, and the edges from $u$ and $v$ to the neighbours in $w$ in $\rho(x)$. Then this is a model of \(H\) in \(G\). 
\end{proof}

 Much of structural graph theory involves graph minors in some way. Many of the theorems that we will discuss throughout this report discuss graph minors. 

\section{Graph Minor Structure Theorem}\label{sec:Kt_Minor_Free}
\textcite{robertsonGraphMinorsXVII1999} proved a rough characterisation of all \(K_t\)-minor free graphs. 

Every graph that is $K_t$-minor-free can be constructed from the following ingredients. This is a coarse characterisation of $K_t$-minor free graphs, meaning that a subset, or a single one of these ingredients constitutes a $K_t$-minor free graph. 
\begin{itemize}
	\item Graphs of bounded Euler genus.
	\item Sets of apex vertices.
	\item Graphs of bounded treewidth.
	\item Sets of vortices on graphs.
\end{itemize}
\textcite{robertsonGraphMinorsXVII1999} showed that every \(K_t\)-minor free graph can be built up from smaller graphs with the above ingredients.

\subsection{Graphs of bounded Euler genus}

Graphs embeddable on a surface of Euler genus $g$ are $K_t$ minor-free, where \(t > \sqrt{6g} + 4\). This comes from \cref{thm:bounded_genus_kt_free}. 

In the case when the surface is a torus, $K_7$ is a toroidal graph but $K_8$ is not. An example of an embedding of $K_7$ on a torus is in \cref{fig:k7_on_torus}.

\begin{figure}[h!]
	\centering
	\includesvg[width = 0.8\textwidth]{figures/k7 on torus.svg}
	\caption[Toroidal graph]{An example of a toroidal graph $K_7$ embedded on a torus.}\label{fig:k7_on_torus}
\end{figure}

\begin{proposition}
	$K_8$ is not embeddable on the torus.
\end{proposition}
\begin{proof}
	A torus has genus 2. By Euler's equation, if a graph $G$ is embedded on a torus, then $|V(G)| - |E(G)| + |F(G)| = 2 - 2 = 0$, where $|F(G)|$ counts the number of faces on the surface. Every face bounds at least three vertices and every edge touches two faces. Therefore, $|F(G)| \leq 2|E(G)|/3$. Suppose $K_8$ is embeddable on the torus. Then $|V(G)| = 8$ and $|E(G)| = 28$. Therefore, $|F(G)| = 20$. But $|F(G)| \leq 2 (28)/3 \leq 19$. Therefore, $K_8$ is not embeddable on the torus.
\end{proof}


\subsection{Apex sets}\label{sssec:Apex_Vertices}
Let $G$ be a graph. A set of vertices $A \subseteq V(G)$ is an apex set if $G - A$ has some bounded parameter. Common parameters are planarity or bounded genus. 
\begin{proposition}
	Let $G$ be a graph. If \(G-a\) is \(K_{t}\)-minor free, then $G$ is $K_{t+1}$-minor free. 
\end{proposition}
\begin{proof}
	We shall prove the contrapositive. Suppose \(G\) has a \(K_{t + 1}\) minor. Then \(K_{t + 1}\) has a model $\rho$ in \(G\). Now let \(v\) be the vertex in \(K_{t + 1}\) such that \(\rho(v)\) contains \(a\). Then delete \(v\) from \(K_{t + 1}\) to form $K_t$. \(K_t\) is a minor of \(G - \rho(v)\). But \(G - \rho(v)\) is a minor of \(G - a\), as \(G - \rho(v)\) does not contain \(a\). So \(G - a\) has a \(K_t\) minor. 
\end{proof}
\subsection{Treewidth and clique-sums}\label{sssec:Clique_Sums}
The \textit{\(k\)-clique-sum} of two graphs \(G\) and \(H\) is a new graph formed from both $G$ and $H$ by identifying two cliques together. The clique-sum of $G$ and $H$ is \(G \oplus_k H\), and is defined as follows. Find cliques in \(G\) and \(H\), \(C_G\) and \(C_H\) respectively, such that both \(C_G\) and \(C_H\) have size \(k\). Identify the vertices in \(C_G\) and \(C_H\) to glue \(G\) and \(H\) together, and possibly delete edges in $C_G$. An illustration can be found in \cref{fig:clique-sum}. 

\begin{figure}[h]
	\centering
	\includesvg[width=0.7 \textwidth]{figures/Clique-sum}
	\caption[Clique-sum]{Figure of clique-sum. Public domain image from David Eppstein \cite{eppsteinCliquesum2023}.}
	\label{fig:clique-sum}
\end{figure}


\begin{proposition}
	Let $t$ be an integer $\geq 1$. Let $G_1, G_2$ be two graphs with treewidth $t$. Then for all $k \leq t + 1$, $G_1 \oplus_k G_2$ has treewidth $t$. 
\end{proposition}
\begin{proof}
	Suppose $C = V(G_1) \cap V(G_2)$ be the clique that is glued over, where $|C| = k$. Let $(B_x: x \in T_1)$ be a tree-decomposition of $G_1$ of minimum width. Let $(B_x : x \in T_2)$ be a tree-decomposition of $G_2$ of minimum width. Then $C$ must appear in some bag $A_x$ and $B_y$ by \cref{lem:clique}. Let $T = T_1 \sqcup T_2$. Add a new node $u$ to $T$ and let $B_u = C$. Then add edges $ux, uy$ to $E(T)$ to form a new tree. Every vertex not in $C$ has a subtree in $T$. If $v \in C$, then the induced subgraph in $T$ is the graph $T_1(v) \cup T_2(v) \cup u$. $T_1(v) \cup T_2(v) \cup u$ is a subtree of $T$. Finally, every edge in $G_1 \cup G_2$ remains in $T$. Therefore, $T$ is a tree-decomposition of $G_1 \oplus_k G_2$. The size of each bag in $T$ is still at most $t + 1$, so the treewidth of $G_1 \oplus_k G_2 \leq t$.
\end{proof}

\begin{proposition}
	Let $t$ be an integer $\geq 1$. Suppose $G$ and $H$ are $K_t$-minor-free graphs. Then $G \oplus_k H$ is $K_{t}$-minor free, $k < t$.  
\end{proposition}
\begin{proof}
	Let $C = V(G) \cap V(H)$ be the clique that is being pasted over. As $G$ and $H$ are $K_t$-minor free, then $|C| \leq k - 1$. Suppose $G \oplus_k H$ is not $K_t$-minor free. Then there exists a model $\rho: V(K_t) \rightarrow G\oplus_k H$ of $K_t$ in $G \oplus_k H$. $\rho$ cannot have its image only in $G$ or only in $H$, as that would mean that $G$ or $H$ has a $K_t$ minor. Therefore, every connected subgraph of $\rho$ must use a vertex in $C$. But $C$ has only $k$ vertices, and $k < t$. Since $\rho$ has disjoint subgraphs, every vertex in $C$ belongs in at most one subgraph in $\rho$. Therefore, $\rho$ cannot have $t$ subgraphs, which is a contradiction. Therefore, $G \oplus_k H$ is also $K_t$ minor free. 
\end{proof}

\begin{corollary}\label{corr:clique_sum_genus}
	If \(G\) is the clique-sum of Euler genus \(g\) graphs, then \(G\) is \(K_{\lceil \sqrt{6g} + 5 \rceil}\)-minor-free.
\end{corollary}
The reverse does not hold. 
\begin{proposition}
	There exists graphs $G$ where \(G\) has arbitrarily large genus, but $G$ is \(K_{6}\)-minor-free.
\end{proposition}

\begin{proof}
	Consider $n$ copies of $K_5$ and identify one vertex in every $K_5$ to a single vertex $v$ to form $G$. Then $G$ is $K_6$-minor free. However, from \cref{thm:additivity_genus}, $G$ has genus $n$. Thus, $G$ has unbounded genus. 
\end{proof}

\subsubsection{Torsos and adhesion}\label{sssec:Torsos and Adhesion}
Given a graph \(G\) and a tree-decomposition \(\tree\), the \textit{torso} of a bag \(B_x\) of \(T\) is the graph \(G\langle B_x \rangle\), with vertex set $B_x$ and edge set where \(vw\) is an edge in \(G\langle B_x \rangle\) if and only if $vw \in E(G)$ or \(v,w \in B_x \cap B_y\), where \(y\) is any neighbour of \(x\) in \(T\). The edge $uv$ where $uv \in B_x \cap B_y$ are called \textit{torso edges}. The set \(B_x \cap B_y\) for all neighbours \(y\) of \(x\) in \(T\) is a clique in \(G\langle B_x \rangle\).
The \textit{adhesion set} is the set \(B_x \cap B_y\). 
The \textit{adhesion} of a tree is defined as \(\max(|B_x \cap B_y|)\) where \(xy\) is an edge in \(T\).

Given \(G\) and a tree-decomposition \(\tree\), \(G\) is a clique-sum of the torsos \(G\langle B_x \rangle\) where the size of the cliques that we paste over is at most the adhesion of $\tree$. This holds for any arbitrary tree-decomposition.
We will discuss decomposing graphs in the language of tree-decompositions, rather than clique-sums. This is because we can discuss the structure of the tree-decomposition.

\subsection{Vortices}\label{sssec:vortices}
Let \(G\) be embedded on a surface \(\Sigma\), and let \(F\) be a face on \(G\). A disc $D$ is \textit{$G$-clean} if $D$ is an open subset of $F$ and $G \cap D$ is a tuple of vertices \(\Lambda = (x_1, x_2, \ldots, x_b)\). No vertex appears more than once in $\Lambda$. The ordering of $\Lambda$ is around the boundary of $D$. 
\par
Let $G$ be a graph embedded on $\Sigma$. Let $D$ be a $G$-clean disc with $G \cap D = \Lambda = (x_1, x_2, \ldots, x_k)$. A \textit{$D$-vortex} is a graph $H$ such that $V(G) \cap V(H) = \Lambda$ and there is a \textit{path-decomposition} of \(H\) of bags \(B_1, B_2, \ldots B_k\) such that \(x_i \in B_i\) for all \(i\). The \textit{depth} of the vortex $H$ is the path-width of $H$. 
\par
The following figure, \cref{fig:tenniscourt} demonstrates the necessity of vortices. $G_n$ is $K_8$-minor free. However, $G_n$ has around $\frac{n}{3}$ $K_{3,3}$ copies, so has genus around $\frac{2n}{3}$. As there is an $n \times n$ grid minor, $G$ has treewidth at least $n$. As $G$ can be arbitrarily large, the number of apex vertices to remove to bound the treewidth and genus is arbitrarily large. However, there is a decomposition of $G_n$ into two graphs $G_0$ and $G_1$ where $G_0$ can be embedded on a surface and $G_1$ is a vortex on $G_0$ with depth 6. $G_0$ is the $n \times n - 1$ grid and $G_1$ is the $n \times 2$ grid in the back plus the apex vertices. 

\begin{figure}[h]
	\centering
	\includesvg[width = 0.8\textwidth]{figures/tenniscourt}
	\caption[Tennis-Court graph]{An example of an $n \times n$ \textit{tennis-court} graph $G_n$ which is \(K_8\) minor free.}
	\label{fig:tenniscourt}
\end{figure}
\subsection{Robertson-Seymour Graph Minor Structure Theorem}\label{ssec:Robertson_Seymour_Graph_Structure}
Given integers \(g, p, a \geq 0\), \(k \geq 1\), a graph \(G\) is \((g, p, k, a)\)- almost-embeddable if there exists an \(A \subseteq V(G)\) with \(|A| \leq a\), and there exists subgraphs \(G_0, G_1, \ldots,  G_{p'}\) of \(G\) such that:
\begin{itemize}
	\item \(G - A = G_0 \cup G_1 \cup G_2 \cup \ldots \cup G_{p'}\),
	\item \(p' \leq p\),
	\item there is an embedding of \(G_0\) onto a surface \(\Sigma\) of genus \(\leq g\),
	\item there exists pairwise disjoint \(G_0\)-clean discs \(D_1, D_2, \ldots, D_{p'}\) in \(\Sigma\),
	\item \(G_i\) is a \(D_i\)-vortex of depth at most \(k\).
\end{itemize}

If we restrict $G_0$ to live only on orientable surfaces, then the graph $G$ is ${(g, p, k, a)}^+$-almost embeddable. If the apex set $A$ is empty, then the graph $G$ is $(g, p, k)$-almost embeddable. 

\begin{theorem}[Graph Minor Structure Theorem \cite{robertsonGraphMinorsXVI2003}]\label{thm:gmst}\todo{what is $\ell$ with respect to the other constants?}
	For all \(t\), there exists \(g, p, a \geq 0\) and \(k, \ell \geq 1\) such that every \(K_t\)-minor-free graph has a tree-decomposition of adhesion \(\leq \ell\) and each torso is \((g, p, k, a)\)-almost-embeddable. The  family of graphs with tree-decomposition of adhesion $\leq \ell$ with torsos $(g, p, k, a)$-almost-embeddable is \(\mathcal{G}(g, p, k, a)\). 
\end{theorem}
There exists a function \(t(g, p, k, a)\) such that if a graph has a tree-decomposition of adhesion \(\leq \ell\) and each torso is \((g, p, k, a)\)-almost embeddable, then \(G\) has no \(K_t\) minor.

\textcite{kawarabayashiQuicklyExcludingNonplanar2021} found upper bounds for $g, p, k, a$. 
\begin{theorem}[\textcite{kawarabayashiQuicklyExcludingNonplanar2021}]
	Let $t \geq 1$ be a positive integer. Let $G$ be a $K_t$-minor free graph. Then let $\alpha = t^{18 \cdot 10^{7} t^{26} + c}$ for a constant $c$, which is defined in the paper. Then setting $g = t(t+1)$, $p = 2t^2$, $k = \alpha$, $\ell = 4\alpha$, and $a = 3\alpha$, $G \in \mathcal{G}(g,p,k,a)$. 
\end{theorem}

\textcite{joretCompleteGraphMinors2013} studies the question of the maximum order of a complete graph minor of a graph in $\mathcal{G}(g, p, k, a)$. 
\begin{theorem}[\textcite{joretCompleteGraphMinors2013}]\label{thm:graph_structure_bound_theorem}
	For all graphs \(G \in \mathcal{G}(g, p, k, a)\),
	\(\had(G) \leq a + 48(k + 1)\sqrt{g + p} + \sqrt{6g} + 5\). Moreover, for some constant $c$, for all $g, a \geq 0$, $p \geq 1, k \geq 2$, there exists a graph $G$ in \(\mathcal{G}(g, p, k, a)\) such that in \(n \geq a + c k\sqrt{p + g}\) such that \(K_n\) is a minor of $G$.
\end{theorem}


