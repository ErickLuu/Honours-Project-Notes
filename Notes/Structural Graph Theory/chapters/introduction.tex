% !TEX root = ./thesis.tex
\chapter{Introduction}\label{sec:introduction}
Structural graph theory is a fundamental topic in graph theory. Many results from structural graph theory decompose graphs, or families of graphs, into smaller graphs with bounded parameters. One of the most important theorems in structural graph theory is Robertson and Seymour's Graph Minor Theorem \cite{robertsonGraphMinorsXX2004} which states that every proper minor-closed graph family is characterised by a finite set of forbidden minors. Furthermore, structural graph theory is a field of interest, with connections to topological graph theory, extremal graph theory and algorithmic complexity. 

Topological graph theory is also a fundamental topic in graph theory. The main questions topological graph theory aims to solve are graphs embedded on topological spaces. There are many real-world uses of topological graph theory in data science, optimisation and computer science. The topological spaces that this report is concerned on are surfaces and quotient spaces of surfaces. 

% !TEX root = ./thesis.tex
\section{Problem Statement}

A \textit{book-embedding} of a graph $G$ arranges the vertices of $G$ on the ``spine'' of a book and arranges the edges of $G$ on ``pages'' of a book. The \textit{pagenumber} of a graph \(G\) is the minimum number of pages necessary in a book-embedding of \(G\). The concept of the \textit{pagenumber} of a graph was introduced by Ollmann \cite{ollmannBookThicknessVarious1973} in the context of VLSI design and integrated circuitry. 
The driving question of this report is the following:
\begin{conjecture}\label{conj:bded_had_pn}
	There exists a function $f$ such that for all integers $t \geq 1$, every $K_t$ minor free graph $G$ can be embedded on $f(g)$ pages.
\end{conjecture}

In her PhD thesis, \textcite{Blankenship-PhD03} claimed to prove \cref{conj:bded_had_pn}. However, this result has not been published and has not been independently verified. Furthermore, a key proof used by Blankenship, from \textcite{heathPagenumberGenusGraphs1992}, was found to be missing crucial parts. \textcite{nakamotoBookEmbeddingProjectiveplanar2015} found that a crucial case was missed by Heath and Istrail.

We begin this report by discussing some background to the topics in our literature, which includes structural graph theory and topological graph theory. We introduce some basic concepts and definitions. We discuss planar graphs, graphs on surfaces, graph minors and the Graph Minor Structure Theorem, with a discussion on some famous theorems and conjectures connected to each topic. 

Robertson and Seymour showed that graphs with no \(K_t\) minor can be built from smaller building blocks. This is a rough overview of the building blocks. We first start with a graph \(G\) embedded on a genus \(g\) surface. Then we add on \(p\) \textit{vortices} to \(G\), with \textit{pathwidth} at most \(k\). Then we add on \(a\) \textit{apex vertices} to \(G\). We say that \(G\) is \((g, p, k, a)\)-\textit{almost embeddable}. Robertson and Seymour \cite{robertsonGraphMinorsXVI2003} proved that all graphs with no \(K_t\) minor has a \textit{tree-decomposition} where every \textit{torso} is a \((g, p, k, a)\) almost-embeddable graph, with \((g, p, k, a)\) bounded by a function of \(t\).

This honours project has two goals. The first goal is to investigate and learn more about structural graph theory. We will discuss some important machinery in structural graph theory, the main ones being the Graph Minor Theorem and the Graph Minor Structure Theorem. The second goal is to address an open problem within this field. To this end, an entire chapter building on the techniques discussed in previous chapters discusses this open problem. 

We prove that graphs that are almost-embeddable on a surface of genus $g$ with $p$ vortices of depth $k$ on some faces is embeddable in $f(g, p, k)$ pages, when the surface is orientable or the projective plane. Proving graphs embedded on nonorientable surfaces with higher genus remains a conjecture. 

%\subsection{Support for conjecture}
We have good reason to believe \cref{conj:bded_had_pn} is true. Firstly, \textcite{yannakakisEmbeddingPlanarGraphs1989} showed that every planar graph can be embedded on 4 pages. \textcite{heathPagenumberGenusGraphs1992} then showed that every graph of orientable genus $g$ can be embedded on $O(g)$ pages. Finally, \textcite{ganleyPagenumberTrees2001} showed that graphs with bounded treewidth have bounded pagenumber. \textcite{dujmovicGraphTreewidthGeometric2007} showed that the bound given by \citeauthor{ganleyPagenumberTrees2001} is tight.
We discuss some relevant papers that are used to prove \cref{conj:bded_had_pn}.
We aim to solve this question using the Graph Minor Structure Theorem \cite{robertsonGraphMinorsXVI2003}, which describes the structure of graphs that do not contain a \(K_t\) minor. 
We have some useful results that can be paired with the Graph Minor Structure Theorem to prove \cref{conj:bded_had_pn}.
\begin{itemize}
	\item From \textcite{heathPagenumberGenusGraphs1992}, every graph of bounded orientable genus have bounded pagenumber.
	\item From \textcite{ganleyPagenumberTrees2001}, and \textcite{dujmovicGraphTreewidthGeometric2007}, every graph of bounded treewidth have bounded pagenumber.
	\item From \textcite{hickingbothamStackNumberCliqueSum2023}, if a graph \(G\) has a \textit{tree-decomposition} where every \textit{torso} has bounded pagenumber, then \(G\) has bounded pagenumber.
	\item From \textcite{nakamotoBookEmbeddingProjectiveplanar2015}, all planar-projective graphs have bounded pagenumber.
\end{itemize}
These results individually show that the constituent ingredients of the Graph Minor Structure Theorem, except non-orientable surfaces of genus at least 2, have bounded pagenumber. We summarise some relevant technology that will be used to obtain some partial results for \cref{conj:bded_had_pn}. 
The biggest hurdle is showing that adding vortices on surfaces will not blow up the pagenumber. To address this issue, we introduce a new concept when considering faces on surfaces with a fixed book-embedding, monochromatic paths. 

\section{Background}
This section is a brief overview of some concepts in topological graph theory and structural graph theory. We will discuss these concepts in further detail throughout the report. We also introduce some of the important theorems related to these concepts to discuss the initial motivation to studying these concepts in the first place. 


\subsection{Basic definitions}\label{sec: Basic definitions}
A graph $G$ is a pair of sets; a vertex set $V(G)$ and an edge set $E(G)$. $E(G)$ is a set that contains two-element subsets of $V(G)$. An edge $ \{v, w\}$ \textit{joins} vertices $v$ and $w$. A graph is \textit{simple} if all edges join two distinct vertices and there is at most one edge between any two vertices. In this paper, all graphs are simple unless stated. Furthermore, all graphs $G$ are finite, so $|V(G)| < \infty$. The graph with all possible edges on $n$ vertices is the \textit{complete graph} $K_n$. Graphs are defined up to isomorphism, or up to relabelling of the vertices.
Throughout this report, the set $\lbrace 1\ldots n \rbrace$ is notated as $[n]$. 
A graph \(G\) is \(k\)-colourable if there exists a function \(f: V(G) \rightarrow [k]\) such that if $f(v) = f(w)$, then $v$ and $w$ do not share an edge. The \textit{chromatic number} \(\chi(G)\) is the smallest \(k\) such that \(G\) is \(k\)-colourable. 

Let $G$ be a graph. A \textit{subgraph} $H$ in $G$ is a graph with vertex set $V(H) \subseteq V(G)$ and edge set $E(H)$ with the property that if $vw$ is an edge in $E(H)$, then $vw$ is an edge in $E(G)$.
Let $G$ be a graph and let $S$ be a non-empty subset of the vertex set of $G$. The \textit{induced subgraph} of $S$ in $G$ is the graph $G[S]$ with vertex set $S$ and edge set containing precisely all edges in $G$ incident to two vertices in $S$. Removing a set of vertices $S \subseteq V(G)$ from $G$ forms the induced subgraph $G - S := G[V(G) - S]$. 
$H$ is a \textit{spanning subgraph} of $G$ if $H$ is a subgraph of $G$ and $V(H) = V(G)$. 
The \textit{neighbourhood} of a set of vertices $A \subseteq V(G)$ are precisely all vertices that are adjacent to a vertex in $A$ and not in $A$ and is denoted as $N_G(A)$. A \textit{clique} is a subgraph isomorphic to a complete graph. 

\subsection{Planar graph bounds}
This subsection uses \cref{lem:planar_graphs_4_connected_cliqesums} and \cref{thm:clique_sum_pagenumber_bound} to find a book-embedding of 4-connected planar graphs. This proof is different from previous proofs as it does not require a triangulation of a planar graph. Because of this fact, this proof is used in future sections with respect to adding vortices on faces. 
Then use a theorem of Tutte to prove a fact for all $4$-connected planar graphs. 

\begin{theorem}[Tutte\cite{tutteTheoremPlanarGraphs1956}]\label{thm:4-connected_planar_ham_cycle}
	All 4-connected planar graphs are Hamiltonian.
\end{theorem}

As a corollary to \textcite{hickingbothamStackNumberCliqueSum2023}, the pagenumber of planar graphs are bounded.

\begin{corollary}\label{thm:Planar Graph Hickingbotham Bound}
	Let \(G\) be a 2-connected planar graph. Then $G$ can be embedded on $11$ pages, with book-embedding $(<, \rho)$. $<$ restricted to the outer cycle $C$ is $C$. Furthermore, for every face cycle $C$, $<_{V(C) - \{u, v, w\}} = C - \{u, v, w\}$ for some vertices $u$, $v$, $w$. 
\end{corollary}
\begin{proof}
	From \cref{thm:clique_sum_pagenumber_bound} with tree-decomposition from \cref{lem:planar_graphs_4_connected_cliqesums}, the pagenumber is at most \(2 \cdot 4 + 3 = 11\).

	Furthermore, from the construction given in \cref{lem:planar_graphs_4_connected_cliqesums}, every $4$-connected class are glued on faces. Therefore, every face only changes by $3$ vertices, from \cref{thm:clique_sum_pagenumber_bound}. Therefore removing $3$ vertices from every face preserves the cyclic ordering of every face.
\end{proof}

We will discuss the \(K_5\)-minor free case. If \(G\) is \(K_5\)-minor free, then we can use Wagner's theorem.
\begin{theorem}[Wagner's theorem\cite{wagnerUeberEigenschaftEbenen1937}]\label{thm:WagnersTheorem}
	Let \(G\) be a \(K_5\)-minor-free graph. Then \(G\) has a tree-decomposition of adhesion $\leq 3$ where every torso is either a planar graph or the Wagner graph \(V_8\).
\end{theorem}
A description of the Wagner graph is in \cref{fig:wagner}. The edges are coloured such that the internal edges are on different pages. The spine edges (the edges that are on the outerface) are the ones which can go on any page.
\begin{figure}[h!]
	\centering
	\begin{tikzpicture}[thick,scale=2, every node/.style={scale=2}]
		\tikz \graph [nodes = {draw, circle}, clockwise, empty nodes] {
	subgraph C_n [n=8, red];
	1 --[red] 5;
	2 --[blue] 6;
	3 --[green] 7;
	4 --[yellow] 8;
};

	\end{tikzpicture}
	\caption[Wagner graph]{The Wagner graph $V_8$. Notice how the clockwise circular ordering of the vertices of the Wagner graph needs 4 pages to embed the graph. }\label{fig:wagner}
\end{figure}

\begin{theorem}
	Let \(G\) be a \(K_5\)-minor free graph. Then \(G\) has pagenumber \(\leq 19\).
\end{theorem}

\begin{proof}
	Suppose \(G\) is \(K_5\)-minor free. Then by Wagner's theorem \cite{wagnerUeberEigenschaftEbenen1937}, \(G\) has a tree-decomposition of adhesion at most 3 where every torso is either a planar graph or the Wagner graph.
	Planar graphs are \(4\)-colourable and can be embedded on four pages. The Wagner graph is \(3\)-colourable and can be embedded on four pages. Therefore, if \(G\) is \(K_5\)-minor free, then \(G\) has pagenumber at most \(4 \cdot 4 + 3 = 19\) from \cref{thm:clique_sum_pagenumber_bound}.
\end{proof}

\subsection{Surfaces and graphs on surfaces}
Graphs on surfaces are a natural extension to graphs on planes. This section is an introduction to surfaces and graphs on surfaces. Readers are expected to be familiar with point-set topology. This section is based on \textcite{moharGraphsSurfaces2001}.

An \textit{$n$-manifold} $M$ is a second-countable Hausdorff space where every point in $M$ has an open neighbourhood homeomorphic to an open ball in $\mathbb{R}^n$.  
A \textit{surface} is a $2$-manifold. Surfaces are typically denoted as $\Sigma$. Examples of surfaces are the sphere $S^2$, the torus $T^2$, the real projective plane $\mathbb{R}P^2$, and the Klein bottle $K$. 

\textit{Handles} are added to a surface \(\Sigma\) by removing two disks in \(\Sigma\) and identifying the boundaries such that one goes clockwise and the other goes counter-clockwise. \textit{Crosscaps} are added to a surface $\Sigma$ by removing a disk in \(\Sigma\) and identifying opposite points on the boundary. Every surface is homeomorphic to a sphere with $m$ handles and $n$ crosscaps. The \textit{Euler genus} of a surface \(\Sigma\) with $m$ handles and $n$ crosscaps is $2m + n$. In fact, a sphere with a mix of crosscaps and handles is homeomorphic to a sphere with all crosscaps, as a sphere with a handle and crosscap is homeomorphic to three crosscaps.

An \textit{embedding} of $G$ on a surface $\Sigma$ is a drawing of $G$ on $\Sigma$ such that no two edges cross. 
A \textit{$2$-cell embedding} of a graph $G$ on a surface $\Sigma$ is an embedding of $G$ in $\Sigma$ such that $\Sigma - G$ is homeomorphic to a finite number of disks. The \textit{Euler Genus} of a \textit{graph} \(G\) is the smallest Euler genus \(g\) surface \(\Sigma\) such that \(G\) can be $2$-cell embedded on $\Sigma$.

An extension for Euler's formula is below. Suppose $G$ is $2$-cell embedded on a surface $\Sigma$ of genus $g$. Let \(|F(G)|\) be the number of faces in a graph \(G\). Then \(|V(G)| - |E(G)| + |F(G)| = 2 - g = \chi\). When $g = 0$, then $\Sigma$ is a $2$-sphere and this is the original Euler's formula. 
The value $\chi$ is known as the \textit{Euler characteristic} of a topological space, in this case a surface. The Euler characteristic is invariant under homeomorphism. Calculating the Euler characteristic of any space is done through \textit{homological algebra}, specifically by looking at the free rank of homology groups. 

Graphs that can be embedded on the plane are called \textit{planar} graphs. Graphs that can be 2-cell embedded on the torus are called \textit{toroidal} graphs, and graphs that can be 2-cell embedded on the projective plane are called \textit{projective-planar} graphs. Graphs that can be 2-cell embedded on a surface of genus $g$ are called \textit{genus $g$} graphs. Similarly to plane graphs, graph drawings on the torus are called torus graphs, and graphs drawings on the projective plane are called projective-plane graphs. 


Graphs on surfaces have been studied extensively. A famous conjecture involving graphs on surfaces is Heawood's conjecture, from \textcite{heawoodMapcolourTheorem1890}. The conjecture states that the minimum number of colours sufficient to colour all Euler genus $g$ graphs when $g \geq 0$ is
	\begin{equation*}
		\gamma(g) := \left\lfloor 
		\frac{7 + \sqrt{1 + 24g}}{2}
		\right\rfloor.
	\end{equation*}\todo{is this right?}
\textcite{ringelMapColorTheorem1974} showed that for almost every case, $\gamma(g)$ is also necessary. The case where this does not hold is the Klein bottle case. There exists a 6-colourable Klein bottle graph, but $\gamma(g) = 7$. 

\section{Book-Embeddings and Pagenumber}\label{sec:Book Embedding}
A \textit{book} with \(k\) \textit{pages} consists of \(k\) half-planes glued together on a common boundary. We refer to the boundary as the \textit{spine}, and the individual half-planes as \textit{pages}. In topology, these are referred to as \textit{fans} of half-planes.\ \textcite{persingerSubsetsNbooksE31966,atneosenOnedimensionalNleavedContinua1972} described fans in the 1960s.
A \textit{book-embedding} of a graph \(G\) is an embedding of \(G\) on a book. We place the vertices of \(G\) on the spine, and we place each edge on a single page such that no two edges cross.
The \textit{pagenumber} of a graph \(G\) is the minimum number of pages required to embed \(G\) on a book. This is also referred to as \textit{book-thickness}, or \textit{stack-number}. An embedding of $K_5$ in three pages is in \cref{fig:book-embedding}.
\begin{figure}[h!]\label{fig:book-embedding}
	\centering
	\includesvg[height = 0.5\textheight]{figures/3page_K5.svg}
	\caption{Book-embedding of $K_5$ on three pages. Image by \textcite{eppsteinBookEmbedding2014}}
\end{figure}
\par
There is an equivalent combinatorial definition. A \textit{book embedding} of a graph \(G\) is an arrangement of the vertices of \(G\) in a total ordering \(v_1 < v_2 < \cdots < v_n\). We then \textit{colour} the edges \(E(G)\) such that if there are vertices with ordering \(v_a < v_b < v_c < v_d\) and edges \(v_a v_c\) and \(v_b v_d\) in $E(G)$, then $v_a v_c$ and $v_b v_d$ are assigned different colours.
We refer to the total ordering of \(V(G)\) in the book embedding as \((<)\) or as \((\leq)\). For a book-embedding \((v_1, v_2, \ldots, v_{|G|})\), we refer to the edges \( \left\{ v_1 v_2, v_2 v_3, \ldots, v_{|G| - 1}v_{|G|}, v_{|G|}v_{1} \right\} \) as \textit{spine edges}.
We may use a \textit{circular ordering} of the vertices rather than a linear ordering. This means that we order the vertices in a circle rather than on a straight line. The book-embedding of a circular ordering is exactly the same as for a linear ordering, and we can convert between a circular and linear ordering by choosing a vertex to be at the start of the sequence.
Book-embeddings were introduced by \textcite{kainenRecentResultsTopological1974, ollmannBookThicknessVarious1973} in the 1970s. A paper by \textcite{bernhartBookThicknessGraph1979} calculated the book-thickness of complete and bipartite graphs.

An \textit{expander graph} is a sparse, highly connected graph. Expander graphs share many properties with random graphs, but are constructed explicitly. One type of expander graph is a \textit{bipartite \varepsilon-expander}, where $\varepsilon \in (0, 1]$. We say a graph $G$ is a bipartite \varepsilon-expander if there exists a bipartition $ \{A, B\}$ of $V(G)$ such that $|A| = |B|$ and for all subsets $S \subset A$ where $|S| \leq \frac{|A|}{2}$, $|N(S)| \geq (1 + \varepsilon) |S|$. 
\textcite{dujmovicLayoutsExpanderGraphs2016} showed that all bipartite \varepsilon-expander graphs can be embedded in 3 pages. 


Book-embeddings of graphs were has applications in VLSI and processor designs, bioinformatics by \textcite{haslingerRNAStructuresPseudoknots1999}, and in graph drawings by \textcite{woodBoundedDegreeBook2002}. 
The project of finding upper and lower bounds of the pagenumber of planar graphs was started by \textcite{bernhartBookThicknessGraph1979} when they conjectured that planar graphs had unbounded pagenumber. However, \textcite{bussPagenumberPlanarGraphs1984} showed that all graphs could be embedded in nine pages, and \textcite{heathEmbeddingPlanarGraphs1984} brought down the number of needed pages to seven.\ \textcite{yannakakisEmbeddingPlanarGraphs1989} devised an algorithm to embed a graph in four pages. Yannakakis, in this paper, claimed that there exists planar graphs which cannot be embedded in three pages. However, his proof was incomplete and the full proof was left unpublished. In 2020, Yannanakis published his full proof \cite{yannakakisPlanarGraphsThat2020}. At around the same time, \textcite{kaufmannFourPagesAre2020} published the same lower bound.

\textcite{malitzGraphsEdgesHave1994} proved that any graph with $e$ edges has pagenumber $O(\sqrt{e})$. Additionally, he proved that random $d$-regular graphs $G$ with $n$ vertices have the property that $\pn(G) \in \Omega(\sqrt{d} n^{1/2 - 1/d})$. For random 3-regular graphs $G$ with $n$ vertices, $\pn(G) \in \Omega(n^{1/6})$. These constructions of $\Omega(n^d)$ pagenumber graphs are not explicit.\ \textcite{eppsteinThreeDimensionalGraphProducts2024} showed that $\pn(P_n \boxtimes P_n \boxtimes P_n) \in \Theta(n^{1/3})$. This is an explicit construction of a graph which has pagenumber in $\Theta(n^{d})$. 

\section{Graph minors}
A graph \(H\) is a \textit{minor} of a graph \(G\) if a graph isomorphic to \(H\) can be obtained from \(G\) by deleting vertices, deleting edges, and \textit{contracting} edges. Let $G$ be a graph and let $uv$ be an edge in $E(G)$. To \textit{contract} \(uv\), we delete both \(u\) and \(v\) and create a new vertex \(w\) with neighbourhood \(N(w) = N_G(u) \cup N_G(v)\). The graph obtained after contracting the edge \(uv\) in $G$ is written as \(G/uv\).
A description of edge contraction is in \cref{fig:edge_contraction}.  Much of structural graph theory involves graph minors in some way. Many of the theorems that we will discuss throughout this report discuss graph minors. 
\begin{figure}[h!]
	\centering
	\includesvg[pretex=\tiny, width = 0.5\textwidth]{figures/edge_contraction.svg}
	\caption[Edge contraction]{Contraction of the edge $\{u, v\}$ to the vertex $w$. Note that edges incident to common neighbours of both $u$ and $v$ becomes a single edge in the contraction. This maintains the property that the graph after edge contraction is simple.}\label{fig:edge_contraction}
\end{figure}

The statement ``\(H\) is a minor of \(G\)'' is written as \(H \leq G\). A graph \(G\) is \textit{\(H\)-minor-free} if $H$ is not a minor of $G$. A family of graphs \(\mathcal{F}\) is \textit{minor-closed} if and only if for all $G$ in \(\mathcal{F}\) and \(H \leq G\), then $H$ in \(\mathcal{F}\).

An example of a minor-closed class is the class of planar graphs.

Let $G$ and $H$ be graphs. A \textit{model} of \(H\) in \(G\) is a function $\rho$ which assigns to \(H\) vertex-disjoint connected subgraphs of \(G\), and if $uv$ is an edge in \(E(H)\), then some edge in \(G\) joins the two subgraphs \(\rho(u)\) and \(\rho(v)\). A description of a model is in \cref{fig:model_of_P5}.
\begin{figure}[h!]
	\centering
	\includesvg[width = 0.5\textwidth]{figures/model.svg}
	\caption[A model $H$ in a graph $G$.]{An illustration of a model $H$ in a graph $G$. The coloured boxes are the connected subgraphs contracted to a single vertex on the right. Note that one vertex is deleted.}\label{fig:model_of_P5}
\end{figure}

\begin{lemma}
	A graph \(H\) is a model of a graph \(G\) if and only if $H$ is a minor of $G$.
\end{lemma}

\begin{proof}
	See \textcite{norinMath599GraphMinors2017}. Suppose \(H\) is a model of \(G\). Then over all \(x\) in \(V(H)\), contract \(\rho(x)\) in \(G\) to a single vertex. This is a well-defined operation as the image $\rho(x)$ is connected and disjoint from all $\rho(y)$ where $y$ is a distinct vertex in $H$. Then delete edges to form \(H\).

	Suppose $H$ is a minor of $G$. Use induction to show that \(H\) has a model in \(G\). Suppose \(H\) is obtained from \(G\) by contraction operations only. We can assume this by taking a subgraph of \(G\) if necessary. Let \(uv\) be the first contracted edge and let \(G' := G / uv\). Let \(w\) be the vertex obtained after contracting \(uv\). Then by induction, a model \(\rho\) of \(H\) in \(G'\) exists. Then find $x \in V(H)$ such that $w \in V(\rho(x))$. If no such $x$ exists, then $\rho$ is a model of $H$ in $G$. Otherwise, delete \(w\) from \(V(\rho(x)) \) and add $u, v$ to $V(\rho(x))$, the edge $uv$, and the edges from $u$ and $v$ to the neighbours in $w$ in $\rho(x)$. Then this is a model of \(H\) in \(G\). 
\end{proof}

An important minor-closed graph family is the set of \(K_t\)-minor-free graphs for a fixed $t \geq 0$. For a graph \(G\), \(\had(G)\) is the largest \(t\) such that \(K_t\) is a minor of \(G\). The set of $K_t$-minor-free graphs is the set of graphs $G$ where $\had(G) \leq t - 1$. The function $\had(G)$ is named after Hugo Hadwiger due to his conjecture below.
\begin{conjecture}[Hadwiger's conjecture \cite{hadwigerUeberKlassifikationStreckenkomplexe1943}]\label{conj:Hadwiger's Conjecture}
	For every graph \(G\), \(\chi(G) \leq \had(G)\).
\end{conjecture}
Much work has been done on solving Hadwiger's conjecture, with a survey by \textcite{seymourHadwigersConjecture2016} on the latest progress. However, \cref{conj:Hadwiger's Conjecture} remains unsolved.

 Any result on $K_t$-minor-free graphs implies results about minor-closed families. This is due to \cref{lem:minor-closed-Kt}. 

\begin{lemma}\label{lem:minor-closed-Kt}
    Every proper minor-closed family is $K_t$-minor-free for some $t$. 
\end{lemma}
\begin{proof}
    As $\mathcal{F}$ is proper, there exists a graph $H$ which is not in $\mathcal{F}$. But this means that $H$ is not a minor of any graph in $\mathcal{F}$. Then $K_t$ is not a minor of $\mathcal{F}$, where $t = |V(H)|$. Then $\mathcal{F}$ is $K_t$-minor-free. 
\end{proof}
Note that $H$ may not be the smallest forbidden graph in $\mathcal{F}$, but the existence of such an $H$ is sufficient. 
Then \cref{conj:bded_had_pn} implies \cref{lem:Minor-Closed_Pagenumber}. From \cref{lem:minor-closed-Kt}, every proper minor-closed graph family is $K_t$-minor-free. Therefore, every graph in a proper minor-closed graph family can be embedded in a bounded number of pages, which is \cref{lem:Minor-Closed_Pagenumber}. 


\subsection{Graph Minor Structure Theorem}
\textcite{robertsonGraphMinorsXVII1999} provides a rough characterisation of all \(K_t\)-minor-free graphs. 

Every graph that is $K_t$-minor-free can be constructed from the following ingredients. This is a coarse characterisation of $K_t$-minor-free graphs, meaning that a subset, or a single one of these ingredients constitutes a $K_t$-minor-free graph. 
\begin{itemize}
	\item Graphs of bounded Euler genus.
	\item Sets of apex vertices.
	\item Graphs of bounded treewidth.
	\item Sets of vortices on graphs.
\end{itemize}
\textcite{robertsonGraphMinorsXVII1999} showed that every \(K_t\)-minor-free graph can be built up from smaller graphs with the above ingredients.

The following list is how the rest of the report is laid out. 
\begin{itemize}
	\item \cref{chap:Definitions} contains definitions and concepts that will be used throughout the rest of the report. Some of these concepts are part of any undergraduate graph theory unit. Some other concepts, like book-embeddings and treewidth, are unlikely to appear in an undergraduate graph theory unit.
	\item \cref{chap:Known results} discusses some known results from graph theory, including the Graph Minor Structure Theorem. We discuss some proofs related to bounded pagenumber that can be used to prove \cref{conj:bded_had_pn}. The results we discuss are the Graph Minor Structure Theorem itself, a result from \textcite{heathPagenumberGenusGraphs1992}, a result from \textcite{ganleyPagenumberTrees2001} and a result from \textcite{hickingbothamStackNumberCliqueSum2023}. \cref{chap:Definitions} and \cref{chap:Known results} form the literature review section of the report.

	\item \cref{chap:Proving_The_Theorem} is an attempt at a proof to \cref{conj:bded_had_pn}. The main bulk of the argument is showing that the construction given by the Graph Minor Structure Theorem can be used to bound the pagenumber of the graph. Concepts and constructions in the literature introduced in the previous sections is used to show this result. 
\end{itemize}

Readers are expected to have at least an undergraduate understanding in graph theory and point-set topology. 


