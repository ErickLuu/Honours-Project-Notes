% !TEX root = ./thesis.tex
\chapter{Introduction}\label{sec:introduction}
Structural graph theory is a fundamental topic in graph theory. Many results from structural graph theory decompose graphs, or families of graphs, into smaller graphs with bounded parameters. One of the most important theorems in structural graph theory is Robertson and Seymour's Graph Minor Theorem \cite{robertsonGraphMinorsXX2004} which states that every proper minor-closed graph family is characterised by a finite set of forbidden minors.
A \textit{book-embedding} of a graph $G$ arranges the vertices of $G$ on the ``spine'' of a book and arranges the edges of $G$ on ``pages'' of a book. The \textit{pagenumber} of a graph \(G\) is the minimum number of pages necessary in a book-embedding of \(G\). The concept of the \textit{pagenumber} of a graph was introduced by Ollmann \cite{ollmannBookThicknessVarious1973} in the context of VLSI design and integrated circuitry. 
The driving question of this report is the following:
\begin{conjecture}\label{conj:bded_had_pn}
	Suppose a graph $G$ is $K_t$-minor-free. Then the pagenumber of \(G\) is bounded by a function of \(t\).
\end{conjecture}
In her PhD thesis, \textcite{Blankenship-PhD03} claimed to prove \cref{conj:bded_had_pn}. However, this result has not been published and has not been independently verified. We aim to fill this gap in knowledge. 
We begin this report by discussing the relevant literature. In their seminal work, \textcite{robertsonGraphMinorsXVI2003} proved the Graph Minor Theorem. In their papers, they introduced many important concepts and theorems that are still used to this day. The result we use is the Graph Minor Structure Theorem, which coarsely describes $K_t$-minor free graphs.

Robertson and Seymour showed that graphs with no \(K_t\) minor can be built from smaller building blocks. This is a rough overview of the building blocks. We first start with a graph \(G\) embedded on a genus \(g\) surface. Then we add on \(p\) \textit{vortices} to \(G\), with \textit{pathwidth} at most \(k\). Then we add on \(a\) \textit{apex vertices} to \(G\). We say that \(G\) is \((g, p, k, a)\)-\textit{almost embeddable}. Robertson and Seymour \cite{robertsonGraphMinorsXVI2003} proved that all graphs with no \(K_t\) minor has a \textit{tree-decomposition} where every \textit{torso} is a \((g, p, k, a)\) almost-embeddable graph, with \((g, p, k, a)\) bounded by a function of \(t\).

We have good reason to believe \cref{conj:bded_had_pn} is true. Firstly, \textcite{yannakakisEmbeddingPlanarGraphs1989} showed that all planar graphs can be embedded on 4 pages.\ \textcite{malitzGenusGraphsHave1994} then showed that all graphs of Euler genus $g$ can be embedded on $O(\sqrt{g})$ pages. Finally, graphs with bounded treewidth have bounded pagenumber, from \textcite{ganleyPagenumberTrees2001} and \textcite{dujmovicGraphTreewidthGeometric2007}.
We discuss some relevant papers that are used to prove \cref{conj:bded_had_pn}.
We aim to solve this question using the Graph Minor Structure Theorem \cite{robertsonGraphMinorsXVI2003}, which describes the structure of graphs that do not contain a \(K_t\) minor. 
We have some useful results that can be paired with the Graph Minor Structure Theorem to prove \cref{conj:bded_had_pn}.
\begin{itemize}
	\item From \textcite{heathPagenumberGenusGraphs1992}, all graphs of bounded genus have bounded pagenumber.
	\item From \textcite{ganleyPagenumberTrees2001}, and \textcite{dujmovicGraphTreewidthGeometric2007}, all graphs of bounded treewidth have bounded pagenumber.
	\item From \textcite{hickingbothamStackNumberCliqueSum2023}, if a graph \(G\) has a \textit{tree-decomposition} where every \textit{torso} has bounded pagenumber, then \(G\) has bounded pagenumber.
\end{itemize}
These results individually show that the constituent ingredients of the Graph Minor Structure Theorem have bounded pagenumber. Once stating the requisite technology, we may proceed with the main aim of the thesis of proving \cref{conj:bded_had_pn}. 
The biggest hurdle is showing that adding vortices on surfaces will not blow up the pagenumber. To address this issue, we introduce a new concept when considering faces on surfaces with a fixed book-embedding. 

This honours project has two goals. The first goal is to investigate and learn more about structural graph theory. We will discuss some important machinery in structural graph theory, the main ones being the Graph Minor Theorem and the Graph Minor Structure Theorem. The second goal is to address an open problem within this field. To this end, an entire chapter building on the techniques discussed in previous chapters discusses this open proble. 

The following list is how the rest of the report is laid out. 
\begin{itemize}
	\item \cref{chap:Definitions} contains definitions and concepts that will be used throughout the rest of the report. Some of these concepts are part of any undergraduate graph theory unit. Some other concepts, like book-embeddings and treewidth, are unlikely to appear in an undergraduate graph theory unit.
	\item \cref{chap:Known results} discusses some known results from graph theory, including the Graph Minor Structure Theorem. We discuss some proofs related to bounded pagenumber that can be used to prove \cref{conj:bded_had_pn}. The results we discuss are the Graph Minor Structure Theorem itself, a result from \textcite{heathPagenumberGenusGraphs1992}, a result from \textcite{ganleyPagenumberTrees2001} and a result from \textcite{hickingbothamStackNumberCliqueSum2023}. \cref{chap:Definitions} and \cref{chap:Known results} form the literature review section of the report.

	\item \cref{chap:Proving_The_Theorem} is an attempt at a proof to \cref{conj:bded_had_pn}. The main bulk of the argument is showing that the construction given by the Graph Minor Structure Theorem can be used to bound the pagenumber of the graph. Concepts and constructions in the literature introduced in the previous sections is used to show this result. 
\end{itemize}

Readers are expected to have at least an undergraduate understanding in graph theory and point-set topology. 
