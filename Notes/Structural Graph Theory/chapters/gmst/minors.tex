\section{Graph minors}
Recall a graph \(H\) is a \textit{minor} of a graph \(G\) if a graph isomorphic to \(H\) can be obtained from \(G\) by deleting vertices, deleting edges, and \textit{contracting} edges.
A \textit{model} of a graph \(H\) in a graph \(G\) is a function $\rho$ which assigns to \(H\) vertex-disjoint connected subgraphs of \(G\). If $uv$ is an edge in \(E(H)\), then some edge in \(G\) joins the two subgraphs \(\rho(u)\) and \(\rho(v)\). A description of a model is in \cref{fig:model_of_P5}.
\begin{figure}[h!]\label{fig:model_of_P5}
	\centering
	\includesvg[width = 0.8\textwidth]{figures/model.svg}
	\caption[Illustration of a model $H$ in a graph $G$.]{The coloured boxes are the connected subgraphs contracted to a single vertex on the right. Note that one vertex is deleted.}
\end{figure}

\begin{lemma}
	\(H\) is a model of \(G\) if and only if $H$ is a minor of $G$.
\end{lemma}

\begin{proof}
	From \textcite{norinMath599GraphMinors2017}. Suppose \(H\) is a model of \(G\). Then for all \(x\) in \(V(H)\), contract \(\rho(x)\) in \(G\) to a single vertex. This is a well-defined operation as the image $\rho(x)$ is connected and disjoint from all $\rho(y)$ where $y$ is a distinct vertex in $H$. Then delete edges to form \(H\).

	Suppose \(H \leq G\). Use induction to show that \(H\) has a model in \(G\). Suppose \(H\) is obtained from \(G\) by contraction operations only. We can assume this by taking a subgraph of \(G\) if necessary. Let \(uv\) be the first contracted edge and let \(G' := G \setminus uv\). Let \(w\) be the vertex obtained after contracting \(uv\). Then by induction, there is a model \(\rho\) of \(H\) in \(G'\). Then find $x \in V(H)$ such that $w \in V(\rho(x))$. If there is no such $x$, then it is obvious that $\rho$ is a model of $H$ in $G$. Otherwise, 
	delete \(w\) from \(V(\rho(x)) \) and add $u, v$ to $V(\rho(x))$, the edge $uv$, and the edges from $u$ and $v$ to the neighbours in $w$ in $\rho(x)$. Then this is a model of \(H\) in \(G\). 
\end{proof}