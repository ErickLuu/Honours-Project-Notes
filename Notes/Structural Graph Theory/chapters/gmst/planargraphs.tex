\section{Planar graphs}\label{sec:Planar graphs}
Recall a graph \(G\) is \textit{planar} if \(G\) can be drawn in the Euclidean plane \( \mathbb{R}^2 \) or the $2$-sphere $S^2$ such that no two edges cross.  A drawing of $G$ in $\mathbb{R}^2$ is referred to as an \textit{embedding} of $G$ in $\mathbb{R}^2$. A drawing of $G$ is also referred to as a \textit{plane} graph. If \(G\) is embedded in \(\mathbb{R}^2 \), then remove the drawing of $G$ from $\mathbb{R}^2$. The connected components in $\mathbb{R}^2 - G$ are called \textit{faces}. The \textit{outerface} is the unbounded face in $\mathbb{R}^2$, meaning that the outerface goes to infinity. If the graph is embedded on $S^2$, then no outerface exists. Any arbitrary face can become the outerface through stereographic projection from $\mathbb{R}^2$ to $S^2$ and back to $\mathbb{R}^2$. The \textit{internal faces} of $G$ are all the faces which are not the outerface. A set of vertices $S$ in $V(G)$ \textit{lie} on a face $F$, or \textit{bound} $F$ if $S$ is in the closure of $F$. A set of edges $T$ in $V(G)$ also bound a face $F$ if $T$ is in the closure of $F$. If an edge $e$ bounds a face $F$, then $e$ \textit{touches} $F$. \(G\) is \textit{outerplanar} if \(G\) is planar and there exists an embedding such that all vertices in \(G\) lie on the outerface.

Let \(F(G)\) be the set of faces of \(G\) embedded on \(\mathbb{R}^2\). 
A \textit{dual graph} $G^*$ of a connected planar graph $G$ is the graph obtained by looking at the faces. $G^*$ has vertex set $F(G)$. $F_i$ and $F_j$ are adjacent in $G^*$ if and only if there exists an edge $e$ that lies on the common boundary of $F_i$ and $F_j$. For every edge on the common boundary of $F_i$ and $F_j$, add one edge. Note that faces may have multiple edges on their boundary, so $G^*$ may not be simple. There exists a natural dual between $E(G^*)$ and $E(G)$ where an edge $e \in E(G)$ corresponds with the edge $e^* \in E(G^*)$, where $e^*$ crosses over $e$. 

\begin{theorem}[Euler's formula]\label{thm:Euler_planar}
	For every connected plane graph $G$, 
	\begin{equation}
		|V(G)| - |E(G)| + |F(G)| = 2.
	\end{equation}
\end{theorem}

A simple proof of this fact is from \textcite{staudtGeometrieLage1847}.

\begin{proof}
	Choose any spanning tree $T$ in $G$. $T$ is connected and without cycles. Now consider the edges $E(G - T)$ and consider the corresponding edge set in $G^*$. $T$ is not a simple closed curve, therefore $\mathbb{R}^2 - T$ has a single connected component. Therefore, the corresponding edge set in $G^*$ is connected. Furthermore, this edge set is acyclic, as if a cycle exists then the cycle of faces bounds some vertex, contradicting $T$ being spanning. Therefore, $G^*$ has a corresponding spanning tree, partitioning the edges of $G$. Therefore, $|E(G)| = (|V(G)| - 1) + (|F(G)| - 1)$, so $|V(G)| - |E(G)| + |F(G)| = 2$. 
\end{proof}

The most immediate application is bounding the number of edges in any planar graph.
\begin{proposition}\label{thm:planar_graph_edge_bound}
	Let $G$ be a planar graph with $n$ vertices and $m$ edges. Then $m \leq 3n - 6$.
\end{proposition}
\begin{proof}
	Suppose $G$ is embedded on $\mathbb{R}^2$. Then $G$ has $f$ faces, where every edge touches exactly two faces. Every face touches at least three edges. Therefore, $3f \leq 2m$. From Euler's formula, $n - m + f = 2$, so $m \leq 3n - 6$. 
\end{proof}

Another application of Euler's formula is to bound the number of edges in an outerplanar graph.
\begin{proposition}\label{thm:outerplanar_bound}
	If \(G\) is an outerplanar graph with \(n\) vertices and \(m\) edges, then \(m \leq 2n - 3\).
\end{proposition}

\begin{proof}
	Suppose \(G\) is \textit{maximal outerplanar}, meaning adding any edge will break the outerplanar property. Let there be \(f\) internal faces in $G$. Then the outerface has \(n\) edges on the boundary as every vertex in $G$ is on the boundary of the outerface. Each internal face will have exactly \(3\) edges on the boundary. However, each edge is touching exactly two distinct faces. By counting the number of edges on every face,
	\begin{equation*}
		3 f + n = 2m.
	\end{equation*}
	Combining with \cref{thm:Euler_planar},
	\begin{equation*}
		n - m + (f + 1) = 2
	\end{equation*}
	we have, after some rearrangement,
	\begin{equation*}
		2n = 3 + m.
	\end{equation*}
	Therefore, \(m = 2n - 3\). Since every outerplanar graph is a spanning subgraph of a maximal outerplanar graph, \(m \leq 2n - 3\).
\end{proof}
\cref{thm:K5_Free_Planar} shows that the class of planar graphs excludes $K_5$. 
\begin{lemma}\label{thm:K5_Free_Planar}
	If \(G\) is a planar graph, then \(G\) is \(K_5\)-minor-free.
\end{lemma}
\begin{proof}
	If \(G\) is planar with \(n\) vertices and \(m\) edges where $n \geq 3$, then \(m \leq 3n -6\).
	However, \(K_5\) has \(5\) vertices and \(10\) edges, but  \( 10 > 3 \cdot 5 - 6\), so \(K_5\) is not planar. As the family of planar graphs is minor-closed, if \(G\) is planar, then $G$ is \(K_5\)-minor free.
\end{proof}

This forms a part of Wagner's theorem, which states that a graph $G$ is planar if and only if $G$ is $K_5$-minor free and $K_{3,3}$-minor free. 