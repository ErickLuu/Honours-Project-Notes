\section{Path-width}\label{sec:Pathwidth}
Similar to treewidth, the pathwidth of a graph \(G\) defines how similar $G$ is from a path.

Define the path-decomposition of a graph \(G\) to be a sequence of bags \(B_i\) such that the subsequence of bags containing a vertex \(v\) induces a nontrivial subpath and each edge \(vw\) is in a bag \(B_i\). Then define the width of a path-decomposition as \(\max_i \lbrace |B_i| \rbrace -1\), same as treewidth.

If a graph has a path-decomposition \({(B_i)}_i\), then it has a tree-decomposition \(\left({(B_i)}_i, P\right)\). Therefore,
\begin{equation*}
	\pw(G) \geq \tw(G).
\end{equation*}
The pathwidth of \(G\) is the largest pathwidth over all connected components.

A graph \(G\) is a \textit{caterpillar} if \(G\) is a tree and $G$ has a path \(P\) where every vertex not in $P$ is adjacent to a vertex on the path \(P\). Alternatively, a tree \(G\) is a caterpillar if removing every leaf yields a path. We refer to the path where every leaf is connected to as the \textit{central path}.
\begin{proposition}
	Graphs have pathwidth at most 1 if and only if every connected component is a caterpillar.
\end{proposition}
\begin{proof}
	Suppose \(G\) is a caterpillar.
	Denote \(P =\left( p_1, p_2, \dots, p_n\right)\) as the central path. The leaves of vertex \(p_i\) are denoted as \(v_{i, 1}, v_{i, 2} \dots, v_{i, k}\). Define the bags as \((v_{1, 1}, v_1)\), \((v_{1, 2}, v_1)\) \dots \((v_{1, j}, v_1)\),  \((v_1, v_2)\), \((v_{2, 1}, v_2)\), \((v_{2,2}, v_2,)\) \dots. We can see that each leaf appears once and each vertex on the central path is on a subpath of the path. Therefore, the pathwidth of \(G\) is 1. We can repeat this for every connected component.
	\par
	Suppose \(G\) has pathwidth 1. Then for each connected component of \(G\), we choose a vertex \(v\) in \(B_1\) and a vertex \(w\) in \(B_n\), the final bag, and look at a path from \(v\) to \(w\). This path must go through every bag, thus the non-path vertices must have as their neighbour the path vertex in the bag and thus the graph is a caterpillar. An example of this is in \cref{fig:caterpillar}.
\end{proof}
\begin{figure}[ht]
	\centering
	\includesvg[pretex=\small, width = 0.8\textwidth]{figures/caterpillar}
	\caption[Caterpillar graph]{A caterpillar graph with central path \((v_1, v_2, v_3, v_4, v_5, v_6)\).}\label{fig:caterpillar}
\end{figure}

\begin{example}
	Recall that $K_n$ is the complete graph on $n$ vertices. It holds that \(\pw(K_n) = \tw(K_n) = n - 1\).
\end{example}
\begin{proof}
	The pathwidth of \(K_n\) is at least the treewidth of \(K_n\). But the pathwidth is at most \(n- 1\) (where all the vertices are in the same bag), but the treewidth of \(K_n\) is \(n - 1\). Therefore, \(\pw(K_n) = n - 1\).
\end{proof}

\begin{proposition}
	The pathwidth of a tree \(T\) equals \(\min_{P \subseteq T} \left\lbrace 1 + \pw(T - V(P))\right\rbrace \) where \(P\) is a path.
\end{proposition}

\begin{proof}[Proof]
	Start by showing the upper bound, \(\pw(T) \leq 1 + \pw(T - V(P))\). If \(P\) is a path in \(T\) with vertices \(v_1, v_2, \ldots v_i\), then consider the subtrees hanging off \(v_i\) for all \(i\). \(T - V(P)\) will have a minimal width path-decomposition. We can order each connected component such that they appear in the order of their parents on the paths. Then adding \(v_i\) to the bags of subtrees connected to \(v_i\), and the bag \((v_i, v_{i+1})\) between the subtrees \(v_i\) and \(v_{i + 1}\) will yield a path-decomposition of width \(1 + \pw(T - V(P))\).
	\par
	To show there exists a path \(P\) such that \(\pw(T) \geq 1 + \pw(T - V(P))\), we proceed by induction. Let \(B_1, \ldots B_n\) be a path-decomposition of \(T\). Let \(x\) live in bag \(B_1\) and \(y\) live in bag \(B_n\), the final bag. Then let \(P\) be the unique path from \(x\) to \(y\). Then \(P\) traverses through every bag in the path-decomposition. Then \(\tw(T) \geq 1 + \tw(T - P)\) by induction.
\end{proof}
