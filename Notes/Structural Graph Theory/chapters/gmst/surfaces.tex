\section{Surfaces and graphs on surfaces}

This section goes into more detail about graphs on surfaces. We start by recalling some important definitions from topology to discuss surfaces. 

\subsection{Important definitions from topology}

Let $I = [0, 1]$.
A \textit{loop} is a continuous function $\gamma : I \rightarrow X$ where $\gamma(0) = \gamma(1) = x_0$. The point $x_0$ is the \textit{base point}. A \textit{homotopy} between two loops $\alpha, \beta$ is a continuous map $h : I \times I \rightarrow (x)$ where $h(0, t) = h(1, t) = x$ for all $t$, $h(\cdot, 0) = \alpha$, $h(\cdot, 1) = \beta$. A \textit{null-homotopic} loop is a loop homotopy to the constant map at $x_0$. Homotopic and null-homotopic loops come up in our discussion of graphs on surfaces as they can be used to classify edges embedded on a surface when the graph is a single point $x_0$. 

\subsection{Surfaces}

This section comes from Mohar and Thomassen's\cite{moharGraphsSurfaces2001} book on graphs on surfaces. Recall a surface is a compact $2$-manifold. Further recall that every surface is homeomorphic to a sphere with $h$ handles and $c$ crosscaps for some natural number $h$ and $c$. The \textit{Euler genus} of a surface \(\Sigma\), obtained from a sphere by adding \(h\) handles and \(c\) crosscaps, is \(2h + c\).

Furthermore, a sphere with one handle and one crosscap is homeomorphic to a sphere with three crosscaps. Therefore, any sphere with a mix of handles and crosscaps is homeomorphic to one with all crosscaps. Note that Euler genus is an invariant under homeomorphism. 

\begin{example}
	This is the Euler genus of some surfaces.
	\begin{enumerate}
		\item The Euler genus of the sphere is \(0\).
		\item The Euler genus of the torus is \(2\).
		\item The Euler genus of the projective plane is \(1\). 
		\item The Euler genus of the Klein bottles is \(2\). 
	\end{enumerate}
\end{example}

Note that ``genus'' and ``Euler genus'' are two distinct concepts in topology. In this paper, when we discuss genus, we will always discuss \underline{Euler genus}.

A surface \(\Sigma\) is \textit{orientable} if \(\Sigma\) can be obtained from \(S^2\) by only adding handles. An example of an orientable surface is the torus.

A surface \(\Sigma\) is \textit{non-orientable} if \(\Sigma\) can be obtained from \(S^2\) by adding at least one crosscap. An example of a non-orientable surface is the projective plane or the Klein bottle. 

The \textit{Euler Genus} of a \textit{graph} \(G\) is the smallest Euler genus \(g\) surface \(\Sigma\) such that \(G\) can be embedded on \(\Sigma\) without crossings (note that \(\Sigma\) can be orientable or nonorientable). 


\subsection{Graphs on surfaces}

If $G$ is embedded on a surface $\Sigma$ and every face in $G$ has three distinct vertices on its boundary, then $G$ is a \textit{triangulation} of $\Sigma$. 

Given graphs $G$ and $H$ with genus $g_1, g_2$, it is useful to construct a new graph with genus $g_1 + g_2$. 
\begin{theorem}[\textcite{millerAdditivityTheoremGenus1987}]\label{thm:additivity_genus}
	Let graphs $G$ and $H$ have genus $g_1$, $g_2$. Then the graph obtained from identifying a vertex in $G$ to a vertex in $H$ has genus $g_1 + g_2$. 
\end{theorem}

\begin{lemma}\label{thm:K5_Free_Planar}
	If \(G\) is a planar graph, then \(G\) is \(K_5\)-minor-free.
\end{lemma}
\begin{proof}
	If \(G\) is planar with \(n\) vertices and \(m\) edges where $n \geq 3$, then \(m \leq 3n -6\).
	However \(K_5\) has \(5\) vertices and \(10\) edges, but  \( 10 > 3 \cdot 5 - 6\), so \(K_5\) is not planar. As the family of planar graphs is minor-closed, if \(G\) is planar, then $G$ is \(K_5\)-minor free.
\end{proof}

This forms a part of Wagner's theorem, which states that a graph $G$ is planar if and only if $G$ is $K_5$-minor free and $K_{3,3}$-minor free. 

\begin{theorem}[Bounded genus]\label{thm:bounded_genus_kt_free}
	If \(G\) is a genus \(g\) graph, then \(G\) is \(K_t\)-minor free, where \(t > \sqrt{6g} + 4\). 
\end{theorem}
\begin{proof}
	This proof mimics the above proof for planarity, but on surfaces of higher genus. 
	Suppose \(G\) has \(n\) vertices and \(m\) edges and of genus $g$. Then \(n - m + f = \chi = 2-g\), from \cref{thm:Euler_surfaces}. As at least three vertices bound each face and each edge touches exactly two faces, then \(f \leq 2m/3\). Therefore, \(m \leq 3(n + g - 2)\). If \(K_t\) is embeddable on a genus \(g\) graph, then \(\binom{t}{2} \leq 3 (t + g - 2)\). Thus \(t \leq \sqrt{6g} + 4\). So if a graph has genus \(g\), then it is \(K_t\)-minor-free, where \(t > \sqrt{6g} + 4\). 
\end{proof}