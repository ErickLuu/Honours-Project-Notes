\section{Surfaces and graphs on surfaces}

This section goes into more detail about graphs on surfaces. This section starts with a discussion on surfaces in more detail, then discusses graphs embedded on a surface.

\subsection{Surfaces}

This section comes from \textcite{moharGraphsSurfaces2001}. Recall a surface is a compact $2$-manifold. Recall also that every surface is homeomorphic to a sphere with $h$ handles and $c$ crosscaps, for some integer $h$ and $c$. This is the classification of surfaces. The \textit{Euler genus} of a surface \(\Sigma\), obtained from a sphere by adding \(h\) handles and \(c\) crosscaps, is \(2h + c\).

Furthermore, a sphere with one handle and one crosscap is homeomorphic to a sphere with three crosscaps. Therefore, any sphere with a mix of handles and crosscaps is homeomorphic to one with all crosscaps. Note that Euler genus is an invariant under homeomorphism. 

These are the Euler genus of some surfaces.
\begin{enumerate}
	\item The Euler genus of the sphere is \(0\).
	\item The Euler genus of the torus is \(2\).
	\item The Euler genus of the projective plane is \(1\). 
	\item The Euler genus of the Klein bottles is \(2\). 
\end{enumerate}

Note that ``genus'' and ``Euler genus'' are two distinct concepts. ``Genus'' refers to the orientable genus, but distinguishing between orientable and nonorientable genus is not important in this thesis. Genus is a synonym \underline{Euler genus} unless stated otherwise.

A surface \(\Sigma\) is \textit{orientable} if \(\Sigma\) can be obtained from \(S^2\) by only adding handles. An example of an orientable surface is the torus.

A surface \(\Sigma\) is \textit{non-orientable} if \(\Sigma\) can only be obtained from \(S^2\) by adding at least one crosscap. An example of a non-orientable surface is the projective plane or the Klein bottle. 

The \textit{Euler Genus} of a \textit{graph} \(G\) is the smallest Euler genus \(g\) surface \(\Sigma\) such that \(G\) can be embedded on \(\Sigma\) without crossings. If $G$ is restricted to be embedded on only orientable surfaces, then in the literature this is referred to as the ``genus'' of the graph. This thesis does not use this definition of genus, and in fact the orientability of the surface plays an important role throughout this paper.


\subsection{Graphs on surfaces}
The family of graphs embeddable on a fixed surface $\Sigma$ is a minor-closed family. If $G$ is embedded on $\Sigma$, then $G - v$ for any vertex $v$ and $G - e$ for any edge $e$ is also embeddable on $\Sigma$. Furthermore, contracting any edge $e$ in $G$ maintains the property that no two edges cross. Edge contraction is a topological action on a graph and can be viewed as an ambient isotopy of $G$ on $\Sigma$. 
If $G$ is 2-cell embedded on a surface $\Sigma$ and every face in $G$ has three distinct vertices on its boundary, then $G$ is a \textit{triangulation} of $\Sigma$. 

Given graphs $G$ and $H$ with genus $g_1, g_2$,a new graph with genus $g_1 + g_2$ can be constructed.
\begin{theorem}[\textcite{millerAdditivityTheoremGenus1987}]\label{thm:additivity_genus}
	Let graphs $G$ and $H$ have genus $g_1$, $g_2$. Then the graph obtained from identifying a vertex in $G$ to a vertex in $H$ has genus $g_1 + g_2$. 
\end{theorem}

Next is an extension of \cref{thm:K5_Free_Planar} for graphs embedded on surfaces. 

\begin{theorem}\label{thm:bounded_genus_kt_free}
	If \(G\) is an Euler genus \(g\) graph, then \(G\) is \(K_t\)-minor free, where \(t > \sqrt{6g} + 4\). 
\end{theorem}
\begin{proof}
	This proof mimics the above proof for planarity, but on surfaces of higher genus. 
	Suppose \(G\) has \(n\) vertices and \(m\) edges and of Euler genus $g$. Then \(n - m + f = \chi = 2-g\), from Euler's theorem on surfaces. As at least three vertices bound each face and each edge touches exactly two faces, then \(f \leq 2m/3\). Therefore, \(m \leq 3(n + g - 2)\). If \(K_t\) is embeddable on a genus \(g\) graph, then \(\binom{t}{2} \leq 3 (t + g - 2)\). Thus \(t \leq \sqrt{6g} + 4\). So if $G$ has genus \(g\), then $G$ is \(K_t\)-minor-free, where \(t > \sqrt{6g} + 4\). 
\end{proof}