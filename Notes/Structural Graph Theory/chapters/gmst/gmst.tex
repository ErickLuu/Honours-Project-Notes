\section{Graph Minor Structure Theorem}\label{sec:Kt_Minor_Free}
\textcite{robertsonGraphMinorsXVII1999} proved a rough characterisation of all \(K_t\)-minor free graphs. 

Every graph that is $K_t$-minor-free can be constructed from the following ingredients. This is a coarse characterisation of $K_t$-minor free graphs, meaning that a subset, or a single one of these ingredients constitutes a $K_t$-minor free graph. 
\begin{itemize}
	\item Graphs of bounded Euler genus.
	\item Sets of apex vertices.
	\item Graphs of bounded treewidth.
	\item Sets of vortices on graphs.
\end{itemize}
\textcite{robertsonGraphMinorsXVII1999} showed that every \(K_t\)-minor free graph can be built up from smaller graphs with the above ingredients.

\subsection{Graphs of bounded Euler genus}

Graphs embeddable on a surface of Euler genus $g$ are $K_t$ minor-free, where \(t > \sqrt{6g} + 4\). This comes from \cref{thm:bounded_genus_kt_free}. 

In the case when the surface is a torus, $K_7$ is a toroidal graph but $K_8$ is not. An example of an embedding of $K_7$ on a torus is in \cref{fig:k7_on_torus}.

\begin{figure}[h!]
	\centering
	\includesvg[width = 0.8\textwidth]{figures/k7 on torus.svg}
	\caption[Toroidal graph]{An example of a toroidal graph $K_7$ embedded on a torus.}\label{fig:k7_on_torus}
\end{figure}

\begin{proposition}
	$K_8$ is not embeddable on the torus.
\end{proposition}
\begin{proof}
	A torus has genus 2. By Euler's equation, if a graph $G$ is embedded on a torus, then $|V(G)| - |E(G)| + |F(G)| = 2 - 2 = 0$, where $|F(G)|$ counts the number of faces on the surface. Every face bounds at least three vertices and every edge touches two faces. Therefore, $|F(G)| \leq 2|E(G)|/3$. Suppose $K_8$ is embeddable on the torus. Then $|V(G)| = 8$ and $|E(G)| = 28$. Therefore, $|F(G)| = 20$. But $|F(G)| \leq 2 (28)/3 \leq 19$. Therefore, $K_8$ is not embeddable on the torus.
\end{proof}


\subsection{Apex sets}\label{sssec:Apex_Vertices}
Let $G$ be a graph. A set of vertices $A \subseteq V(G)$ is an apex set if $G - A$ has some bounded parameter. Common parameters are planarity or bounded genus. 
\begin{proposition}
	Let $G$ be a graph. If \(G-a\) is \(K_{t}\)-minor free, then $G$ is $K_{t+1}$-minor free. 
\end{proposition}
\begin{proof}
	We shall prove the contrapositive. Suppose \(G\) has a \(K_{t + 1}\) minor. Then \(K_{t + 1}\) has a model $\rho$ in \(G\). Now let \(v\) be the vertex in \(K_{t + 1}\) such that \(\rho(v)\) contains \(a\). Then delete \(v\) from \(K_{t + 1}\) to form $K_t$. \(K_t\) is a minor of \(G - \rho(v)\). But \(G - \rho(v)\) is a minor of \(G - a\), as \(G - \rho(v)\) does not contain \(a\). So \(G - a\) has a \(K_t\) minor. 
\end{proof}
\subsection{Treewidth and clique-sums}\label{sssec:Clique_Sums}
The \textit{\(k\)-clique-sum} of two graphs \(G\) and \(H\) is a new graph formed from both $G$ and $H$ by identifying two cliques together. The clique-sum of $G$ and $H$ is \(G \oplus_k H\), and is defined as follows. Find cliques in \(G\) and \(H\), \(C_G\) and \(C_H\) respectively, such that both \(C_G\) and \(C_H\) have size \(k\). Identify the vertices in \(C_G\) and \(C_H\) to glue \(G\) and \(H\) together, and possibly delete edges in $C_G$. An illustration can be found in \cref{fig:clique-sum}. 

\begin{figure}[h]
	\centering
	\includesvg[width=0.7 \textwidth]{figures/Clique-sum}
	\caption[Clique-sum]{Figure of clique-sum. Public domain image from David Eppstein \cite{eppsteinCliquesum2023}.}
	\label{fig:clique-sum}
\end{figure}


\begin{proposition}
	Let $t$ be an integer $\geq 1$. Let $G_1, G_2$ be two graphs with treewidth $t$. Then for all $k \leq t + 1$, $G_1 \oplus_k G_2$ has treewidth $t$. 
\end{proposition}
\begin{proof}
	Suppose $C = V(G_1) \cap V(G_2)$ be the clique that is glued over, where $|C| = k$. Let $(B_x: x \in T_1)$ be a tree-decomposition of $G_1$ of minimum width. Let $(B_x : x \in T_2)$ be a tree-decomposition of $G_2$ of minimum width. Then $C$ appears in some bag $A_x$ and $B_y$ by \cref{lem:clique}. Let $T = T_1 \sqcup T_2$. Add a new node $u$ to $T$ and let $B_u = C$. Then add edges $ux, uy$ to $E(T)$ to form a new tree. Every vertex not in $C$ has a subtree in $T$. If $v \in C$, then the induced subgraph in $T$ is the graph $T_1(v) \cup T_2(v) \cup u$. $T_1(v) \cup T_2(v) \cup u$ is a subtree of $T$. Finally, every edge in $G_1 \cup G_2$ remains in $T$. Therefore, $T$ is a tree-decomposition of $G_1 \oplus_k G_2$. The size of each bag in $T$ is still at most $t + 1$, so the treewidth of $G_1 \oplus_k G_2 \leq t$.
\end{proof}

\begin{proposition}
	Let $t$ be an integer $\geq 1$. Suppose $G$ and $H$ are $K_t$-minor-free graphs. Then $G \oplus_k H$ is $K_{t}$-minor free, $k < t$.  
\end{proposition}
\begin{proof}
	Let $C = V(G) \cap V(H)$ be the clique that is being pasted over. As $G$ and $H$ are $K_t$-minor free, then $|C| \leq k - 1$. Suppose $G \oplus_k H$ is not $K_t$-minor free. Then there exists a model $\rho: V(K_t) \rightarrow G\oplus_k H$ of $K_t$ in $G \oplus_k H$. $\rho$ cannot have its image only in $G$ or only in $H$, as that implies that $G$ or $H$ has a $K_t$ minor. Therefore, every connected subgraph of $\rho$ uses a vertex in $C$. But $C$ has only $k$ vertices, and $k < t$. Since $\rho$ has disjoint subgraphs, every vertex in $C$ belongs in at most one subgraph in $\rho$. Therefore, $\rho$ cannot have $t$ subgraphs, which is a contradiction. Therefore, $G \oplus_k H$ is also $K_t$ minor free. 
\end{proof}

\begin{corollary}\label{corr:clique_sum_genus}
	If \(G\) is the clique-sum of Euler genus \(g\) graphs, then \(G\) is \(K_{\lceil \sqrt{6g} + 5 \rceil}\)-minor-free.
\end{corollary}
The reverse does not hold. 
\begin{proposition}
	There exists graphs $G$ where \(G\) has arbitrarily large genus, but $G$ is \(K_{6}\)-minor-free.
\end{proposition}

\begin{proof}
	Consider $n$ copies of $K_5$ and identify one vertex in every $K_5$ to a single vertex $v$ to form $G$. Then $G$ is $K_6$-minor free. However, from \cref{thm:additivity_genus}, $G$ has genus $n$. Thus, $G$ has unbounded genus. 
\end{proof}

\subsubsection{Torsos and adhesion}\label{sssec:Torsos and Adhesion}
Given a graph \(G\) and a tree-decomposition \(\tree\), the \textit{torso} of a bag \(B_x\) of \(T\) is the graph \(G\langle B_x \rangle\), with vertex set $B_x$ and edge set where \(vw\) is an edge in \(G\langle B_x \rangle\) if and only if $vw \in E(G)$ or \(v,w \in B_x \cap B_y\), where \(y\) is any neighbour of \(x\) in \(T\). The edge $uv$ where $uv \in B_x \cap B_y$ are called \textit{torso edges}. The set \(B_x \cap B_y\) for all neighbours \(y\) of \(x\) in \(T\) is a clique in \(G\langle B_x \rangle\).
The \textit{adhesion set} is the set \(B_x \cap B_y\). 
The \textit{adhesion} of a tree is defined as \(\max(|B_x \cap B_y|)\) where \(xy\) is an edge in \(T\).

Given \(G\) and a tree-decomposition \(\tree\), \(G\) is a clique-sum of the torsos \(G\langle B_x \rangle\) where the size of the cliques that we paste over is at most the adhesion of $\tree$. This holds for any arbitrary tree-decomposition.
We prefer to use tree-decompositions and torsos over clique-sums. This is because the structure of tree-decompositions can be discussed more easily with respect to a set tree.

\subsection{Vortices}\label{sssec:vortices}
Let \(G\) be embedded on a surface \(\Sigma\), and let \(F\) be a face on \(G\). A disc $D$ is \textit{$G$-clean} if $D$ is an open subset of $F$ and $G \cap D$ is a tuple of vertices \(\Lambda = (x_1, x_2, \ldots, x_b)\). No vertex appears more than once in $\Lambda$. The ordering of $\Lambda$ is around the boundary of $D$. 
\par
Let $G$ be a graph embedded on $\Sigma$. Let $D$ be a $G$-clean disc with $G \cap D = \Lambda = (x_1, x_2, \ldots, x_k)$. A \textit{$D$-vortex} is a graph $H$ where $V(G) \cap V(H) = \Lambda$. Furthermore, there exists a \textit{path-decomposition} of \(H\) of bags \(B_1, B_2, \ldots B_k\) and \(x_i \in B_i\) for all \(i\). The \textit{depth} of the vortex $H$ is the path-width of $H$. 
\par
The following figure, \cref{fig:tenniscourt} demonstrates the necessity of vortices. $G_n$ is $K_8$-minor free. However, $G_n$ has around $\frac{n}{3}$ $K_{3,3}$ copies, so has genus around $\frac{2n}{3}$. As $G$ has an $n \times n$ grid minor, $G$ has treewidth at least $n$. As $G$ can be arbitrarily large, the number of apex vertices to remove to bound the treewidth and genus is arbitrarily large. However, let $G_0$ is the $n \times n - 1$ grid and $G_1$ is the $n \times 2$ grid in the back plus the apex vertices. Then $G_0$ is planar and $G_1$ is a vortex on $G_0$, on the outerface. 

\begin{figure}[h]
	\centering
	\includesvg[width = 0.8\textwidth]{figures/tenniscourt}
	\caption[Tennis-Court graph]{An example of an $n \times n$ \textit{tennis-court} graph $G_n$ which is \(K_8\) minor free.}
	\label{fig:tenniscourt}
\end{figure}
\subsection{Robertson-Seymour Graph Minor Structure Theorem}\label{ssec:Robertson_Seymour_Graph_Structure}
Given integers \(g, p, a \geq 0\), \(k \geq 1\), a graph \(G\) is \((g, p, k, a)\)- almost-embeddable if there exists an \(A \subseteq V(G)\) with \(|A| \leq a\), and there exists subgraphs \(G_0, G_1, \ldots,  G_{p'}\) of \(G\) such that:
\begin{itemize}
	\item \(G - A = G_0 \cup G_1 \cup G_2 \cup \ldots \cup G_{p'}\).
	\item The number of vortices $p'$ is at most $p$.
	\item There exists an embedding of \(G_0\) onto a surface \(\Sigma\) of genus \(\leq g\).
	\item There exists pairwise disjoint \(G_0\)-clean discs \(D_1, D_2, \ldots, D_{p'}\) in \(\Sigma\).
	\item \(G_i\) is a \(D_i\)-vortex of depth at most \(k\).
\end{itemize}

If we restrict $G_0$ to live only on orientable surfaces, then the graph $G$ is ${(g, p, k, a)}^+$-almost embeddable. If the apex set $A$ is empty, then the graph $G$ is $(g, p, k)$-almost embeddable. 

\begin{theorem}[Graph Minor Structure Theorem \cite{robertsonGraphMinorsXVI2003}]\label{thm:gmst}\todo{what is $\ell$ with respect to the other constants?}
	For all \(t\), there exists \(g, p, a \geq 0\) and \(k, \ell \geq 1\) such that every \(K_t\)-minor-free graph has a tree-decomposition of adhesion \(\leq \ell\) and each torso is \((g, p, k, a)\)-almost-embeddable. The  family of graphs with tree-decomposition of adhesion $\leq \ell$ with torsos $(g, p, k, a)$-almost-embeddable is \(\mathcal{G}(g, p, k, a)\). 
\end{theorem}
There exists a function \(t(g, p, k, a)\) such that if a graph has a tree-decomposition of adhesion \(\leq \ell\) and each torso is \((g, p, k, a)\)-almost embeddable, then \(G\) has no \(K_t\) minor.

\textcite{kawarabayashiQuicklyExcludingNonplanar2021} found upper bounds for $g, p, k, a$. 
\begin{theorem}[\textcite{kawarabayashiQuicklyExcludingNonplanar2021}]
	Let $t \geq 1$ be a positive integer. Let $G$ be a $K_t$-minor free graph. Then let $\alpha = t^{18 \cdot 10^{7} t^{26} + c}$ for a constant $c$, which is defined in the paper. Then setting $g = t(t+1)$, $p = 2t^2$, $k = \alpha$, $\ell = 4\alpha$, and $a = 3\alpha$, $G \in \mathcal{G}(g,p,k,a)$. 
\end{theorem}

\textcite{joretCompleteGraphMinors2013} studies the question of the maximum order of a complete graph minor of a graph in $\mathcal{G}(g, p, k, a)$. 
\begin{theorem}[\textcite{joretCompleteGraphMinors2013}]\label{thm:graph_structure_bound_theorem}
	For all graphs \(G \in \mathcal{G}(g, p, k, a)\),
	\(\had(G) \leq a + 48(k + 1)\sqrt{g + p} + \sqrt{6g} + 5\). Moreover, for some constant $c$, for all $g, a \geq 0$, $p \geq 1, k \geq 2$, there exists a graph $G$ in \(\mathcal{G}(g, p, k, a)\) such that in \(n \geq a + c k\sqrt{p + g}\) such that \(K_n\) is a minor of $G$.
\end{theorem}