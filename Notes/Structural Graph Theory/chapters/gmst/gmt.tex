\section{Graph Minor Theorem}\label{sec:Graph Minor Theorem}
We move on to one of the most important and deepest theorems in graph theory, the Graph Minor Theorem. This was proven in a series of 23 papers by Robertson and Seymour, from 1983 to 2004. As part of the proof, the Graph Minor Structure Theorem was developed. 
\begin{theorem}[\textcite{robertsonGraphMinorsXX2004} Graph Minor Theorem ]
	Every infinite set of graphs contains two distinct graphs \(G\) and \(H\) such that \(H\) is a minor of \(G\).
\end{theorem}
For the case when the infinite set is the family of trees, this is Kruskal's tree theorem \cite{kruskalWellQuasiOrderingTreeTheorem1960}. 

%However, this infinite family can be extremely large. Let $T_1, \ldots T_m$ be a sequence of rooted trees from the labels $\{1, 2, 3\}$ where each $T_i$ has at most $i$ vertices. By Kruskal's theorem, when $m$ becomes large enough, there exists a $i, j$ such that $1 \leq i < j$ such that $T_i$ is a label-preserving minor of $T_j$. $TREE(3)$ is the largest $m$ such that there is no label-preserving minor. 

Let $\mathcal{F}$ be a minor-closed graph family. A graph $H$ is a \textit{minimal forbidden minor} of $\mathcal{F}$ if every graph $G$ in $\mathcal{F}$ is $H$-minor-free and every proper minor $H'$ of $H$ (meaning $H'$ is not $H$) is in $\mathcal{F}$. 

The Graph Minor Theorem is equivalent to the statement:
\begin{theorem}
	Every minor-closed graph family $\mathcal{F}$ is characterised by a finite set of minimal forbidden minors $\mathcal{H}$. A graph $G$ is in $\mathcal{F}$ if and only if $G$ is $H$-minor free for every $H$ in $\mathcal{H}$. 
\end{theorem}
Importantly, graphs of bounded genus can be characterised as a set of forbidden minors. The family of graphs that can be embedded on a fixed surface $\sigma$ is minor-closed. 
For planar graphs, the two minimal forbidden minors are \(K_5\) and \(K_{3,3}\), from \textcite{wagnerUeberEigenschaftEbenen1937}. 
The family of graphs that can be embedded on a torus are the toroidal graphs. 17,523 minimal forbidden toroidal minors have been found, with a database maintained by \textcite{myrvoldLargeSetTorus2018}. A complete enumeration of minimal forbidden minors has not been found yet. 

A graph $G$ is \textit{linkless} if $G$ has an embedding in $\mathbb{R}^3$ such that no two cycles are linked. If no embedding of $G$ has this property, then $G$ is \textit{inherently linked}. The family of linkless graphs is minor-closed. If $G$ is linkless, then contracting any edge maintains the linkless property. Linkless graphs only have seven minimal forbidden minors, including $K_6$ and the Petersen graph, which was found by \textcite{robertsonSachsLinklessEmbedding1995}. 

A graph $G$ is \textit{knotless} if $G$ can be embedded in $\mathbb{R}^3$ such that every cycle of $G$ is a simple knot. $G$ is \textit{inherently knotted} if this is not the case.\ \textcite{conwayKnotsLinksSpatial1983} showed that $K_7$ is inherently knotted. The family of knotless graphs is also minor-closed, since contracting any edge preserves the knotless property. $K_7$ is an example of a minimal forbidden minor. \textcite{conwayKnotsLinksSpatial1983,foisyNewlyRecognizedIntrinsically2003,foisyIntrinsicallyKnottedGraphs2002} found that there exists at least three minimal minors, but a complete enumeration of minimal forbidden minors have not been found. 

%This chapter is a brief look into some of the deepest theorems in graph theory and a discussion of some of the theorems that were used to prove the Graph Minor Theorem. This chapter discusses the Graph Minor Structure Theorem, which is an important theorem that will be used throughout this thesis to prove some theorems. 