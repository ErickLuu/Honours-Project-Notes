\chapter{Graph Minors}\label{chap:gmst}
This chapter discusses graph minors and some parameters that restrict certain graph minors. 
The Graph Minor Structure Theorem, proven by \textcite{robertsonGraphMinorsXVI2003} is an important characterisation of $K_t$-minor free graphs. The Graph Minor Structure Theorem was developed to solve the Graph Minor Theorem, also by \textcite{robertsonGraphMinorsXX2004}. Many problems of the form: ``Every $K_t$-minor free graph has some bounded parameter, where the bound is of the form $f(t)$'' uses some form of the Graph Minor Structure Theorem in its proof. 


This chapter starts by discussing graph minors. Then each component of the graph minor structure theorem is gone through in detail. Finally, the Graph Minor Theorem, which is what the Graph Minor Structure Theorem was designed to prove, is also discussed.

The Graph Minor Structure Theorem states that every $K_t$-minor-free graph can be decomposed to a set of four ingredients, 
\begin{itemize}
	\item Graphs on surfaces of fixed genus
	\item Apex sets
	\item Clique-sums of graphs
	\item Vortices.
\end{itemize}
We go through every single ingredient in this section and discuss how each component is individually $K_t$-minor free. 

\section{Graph minors}\label{sec:Graph Minors}
A graph \(H\) is a \textit{minor} of a graph \(G\) if a graph isomorphic to \(H\) can be obtained from \(G\) by deleting vertices, deleting edges, and \textit{contracting} edges. Let $G$ be a graph and let $uv$ be an edge in $E(G)$. To \textit{contract} \(uv\), we delete both \(u\) and \(v\) and create a new vertex \(w\) with neighbourhood \(N(w) = N_G(u) \cup N_G(v)\). The graph obtained after contracting the edge \(uv\) in $G$ is written as \(G\setminus uv\).
The statement ``\(H\) is a minor of \(G\)'' is written as \(H \leq G\). A graph \(G\) is \textit{\(H\)-minor-free} if $H$ is not a minor of $G$. A family of graphs \(\mathcal{F}\) is \textit{minor-closed} if when $G$ is in \(\mathcal{F}\) and \(H \leq G\), then $H$ is in \(\mathcal{F}\).
An example of a minor-closed class is the class of planar graphs.
An important class of graph families are the \(K_t\)-minor free graphs. For a graph \(G\), we define \(\had(G)\) to be the largest \(t\) such that \(K_t\) is a minor of \(G\). This is named after Hugo Hadwiger and his most famous conjecture.

\begin{conjecture}[Hadwiger's conjecture]\label{conj:Hadwiger's Conjecture}
	For all graphs \(G\), \(\chi(G) \leq \had(G)\)\cite{hadwigerUeberKlassifikationStreckenkomplexe1943}.
\end{conjecture}
Much work has been done on solving Hadwiger's conjecture, with a document by \textcite{seymourHadwigerConjecture2016} on the latest progress. However, it remains unsolved. 

A \textit{model} of a graph \(H\) in a graph \(G\) is a function $\rho$ which assigns to \(H\) vertex-disjoint connected subgraphs of \(G\). If $uv$ is an edge in \(E(H)\), then some edge in \(G\) joins the two subgraphs \(\rho(u)\) and \(\rho(v)\). A description of a model is in \cref{fig:model_of_P5}.
\begin{figure}[h!]\label{fig:model_of_P5}
	\centering
	\includesvg[width = 0.8\textwidth]{figures/model.svg}
	\caption{Illustration of a model $H$ in a graph $G$. The coloured boxes are the connected subgraphs contracted to a single vertex on the right.}
\end{figure}

\begin{theorem}
	\(H\) is a model of \(G\) if and only if $H$ is a minor of $G$.
\end{theorem}

\begin{proof}
	From \textcite{norinMath599GraphMinors2017}. Suppose \(H\) is a model of \(G\). Then for all \(x\) in \(V(H)\), contract \(\rho(x)\) in \(G\) to a single vertex. This is a well-defined operation as the image $\rho(x)$ is connected and disjoint from all $\rho(y)$ where $y$ is a distinct vertex in $H$. Then delete edges to form \(H\).

	Suppose \(H \leq G\). Use induction to show that \(H\) has a model in \(G\). Suppose \(H\) is obtained from \(G\) by contraction operations only. We can assume this by taking a subgraph of \(G\) if necessary. Let \(uv\) be the first contracted edge and let \(G' := G \setminus uv\). Let \(w\) be the vertex obtained after contracting \(uv\). Then by induction, there is a model \(\rho\) of \(H\) in \(G'\). Then find $x \in V(H)$ such that $w \in V(\rho(x))$. If there is no such $x$, then it is obvious that $\rho$ is a model of $H$ in $G$. Otherwise, 
	delete \(w\) from \(V(\rho(x)) \) and add $u, v$ to $V(\rho(x))$, the edge $uv$, and the edges from $u$ and $v$ to the neighbours in $w$ in $\rho(x)$. Then this is a model of \(H\) in \(G\). 
\end{proof}

 Much of structural graph theory involves graph minors in some way. Many of the theorems that we will discuss throughout this report discuss graph minors. 

\section{Treewidth}\label{chap:treewidth}
This section discusses treewidth and pathwidth.
The \textit{treewidth} of a graph \(G\) measures how similar $G$ is to a forest.
\begin{definition}[Tree-decomposition]\label{def:tree-decomposition}
	A tree-decomposition \(\tree\) of a graph \(G\) is defined as a tree \(T\) with associated \textit{bags} \(\lbrace B_x : x \in V(T) \rbrace\) such that:
	\begin{itemize}
		\item $\bigcup_{x \in V(T)} B_x = V(G)$,
		\item For all \(v \in V(G)\), the subset of vertices \(\left\lbrace x  \in V(T): v \in B_x \right\rbrace\) induces a connected subtree in \(V(T)\),
		\item For all edges \(vw \in E(G)\), there exists a bag \(B_x\) such that both \(v\) and \(w\) are in \(B_x\).
	\end{itemize}
\end{definition}
The vertices of the tree \(T\) are \textit{nodes}, to distinguish them from the vertices of $G$. 
The tree-decomposition is denoted as \((T, (B_x)_x)\), where $T$ is the tree, $x \in V(T)$, and $(B_x)$ is the bag associated with node $x$.
The \textit{width} of the tree decomposition \(\tree\) is defined as \(\max \lbrace |B_x| - 1 : x~\in~V(T) \rbrace\).
The treewidth of a graph \(G\), denoted as \(\tw(G)\), is defined to be the smallest width over all tree-decompositions of the graph \(G\). 

Treewidth was introduced by \textcite{berteleChapterEliminationVariables1972} with applications to dynamic programming under the name ``dimension''. Treewidth was then rediscovered by \textcite{halinSfunctionsGraphs1976}, under the name of ``S-function''. Neither of the papers above introduced the notion of a tree-decomposition.

\textcite{robertsonGraphMinorsIII1984} introduced treewidth as defined in \cref{def:tree-decomposition}. They introduced treewidth in the context of a tree-decomposition. This definition is concrete and could be calculated explicitly for any graph. They showed that if $\mathcal{F}$ is a graph family with bounded treewidth, then there exists a planar graph $H$ such that $H$ is a minimal forbidden minor of $\mathcal{F}$. This was used to prove the Graph Minor Theorem. Furthermore, \textcite{robertsonQuicklyExcludingPlanar1994} refined this theorem. They showed that if a graph $G$ has treewidth $k$ if and only if $G$ contains a grid minor of size $f(k) \times f(k)$. This is the Grid Minor Theorem. \textcite{chekuriPolynomialBoundsGridMinor2016} showed that $f \in \Omega(k^\delta)$, which is a polynomial bound.

\cref{lem:Helly} discusses the \textit{Helly property} of trees, which will be used to prove some statements on tree-decomposition. 

\begin{lemma}\label{lem:Helly}
	Let \(T_1, \ldots, T_k\) be subtrees of a tree \(T\) such that for all $i, j \in \{1 \ldots k\}$, $V(T_i) \cap V(T_j)$ is nonempty. Then there exists a vertex in $T_1 \cap T_2 \cap \ldots \cap T_k$.
\end{lemma}
\begin{proof}
	This proof is by induction on the number of vertices of $T$. Suppose $T$ has a single vertex. Then the Helly property holds. By induction, suppose the Helly property holds for all trees with at most $n$ vertices. Suppose $T$ has $n + 1$ vertices and \(T_1, \ldots, T_k\) are subtrees which satisfy the property above. Let $v$ be a leaf vertex of $T$ with neighbour $w$. If one of the subtrees $T_i = \{v\}$, then by non-empty intersection, all trees contain $v$. Otherwise, consider $T - v$ and the subtrees $(T_1 - v, \ldots, T_k - v)$. If $v \in T_i \cap T_j$, then as none of the subtrees is the single vertex $\{v\}$, $w \in T_i \cap T_j$. Therefore, $T_i - v \cap T_j - v$ is non-empty. By the induction hypothesis, $T - v$ has a vertex common to all $(T_1 - v, \ldots, T_k - v)$, so \(T_1, \ldots, T_k\) has a common vertex in $T$. 
\end{proof}
The Helly property is most commonly associated with convex subsets of a Euclidean space, but other objects have the Helly property. 

\begin{proposition}\label{lem:clique}
	Let $G$ be a graph and $(T, (B_x)_{x})$ be a tree-decomposition. Then for every clique \(C\) in \(G\), there exists a node \(x \in V(T)\) such that \(C \subseteq B_x\).
\end{proposition}

\begin{proof}
	Let \(\tree\) be a tree-decomposition. Every vertex \(v\) induces a connected subtree \(T_v\) in \(T\). For any two vertices \(x, y\) in \(C\), the edge \(xy\) is inside a bag \(B_z\) corresponding to a node \(z\). Therefore, \(T_x\) and \(T_y\) intersect. Then by the Helly property, there exists a node \(v\) such that \(C \subseteq B_v\).
\end{proof}

Recall that $K_n$ is the complete graph on $n$ vertices.

\begin{corollary}\label{cor:complete_tw}
	The treewidth of $K_n$ equals $n-1$. 
\end{corollary}
\begin{proof}
	By \cref{lem:clique}, $\tw(K_n)\geq n-1$. Placing all vertices of $K_n$ in a single bag is a tree-decomposition of width $n-1$. Therefore, $\tw(K_n) = n-1$. 
\end{proof}

\begin{proposition}\label{thm:tw_minor_closure}
	Let $G, H$ be graphs. If \(H\) is a minor of \(G\), then \(\tw(H) \leq \tw(G)\).
\end{proposition}
\begin{proof}
	Let \( \{T, {(B_x)}_{x} \} \) be a tree-decomposition of \(G\). Remove an edge $e$ from $G$. Then \( \{T, {(B_x)}_{x} \} \) is a tree-decomposition of $G - e$. Remove a vertex $v$ from $G$. Then \( \{ T, {(B_x - v)}_{x \in V(T)} \} \) is a tree-decomposition of $G - v$. Contract an edge $vw$ in $G$ to $u$. Define a new tree-decomposition $\tree'$ by relabelling \(v\) and \(w\) in all $B_x$ to \(u\). Then $\tree'$ is a valid tree-decomposition of $G / uv$. The induced subtree of \(u\) is the union of the induced subtrees of \(v\) and \(w\), which share a node. As $v$ and $w$ share the edge $vw$, then there exists a bag $B_x$ such that $v, w \in B_x$. Therefore $T_u$ is connected. Every neighbour of \(v\) or \(w\) is a neighbour of \(u\). The edges in the neighbourhood do not change. Notice that the size of each bag in each operation does not increase. Therefore, if $H \leq G$ by a series of vertex deletions, edge deletions, and edge contractions, the tree-decomposition \( \{T, {(B_x)}_{x} \} \) of $G$ can have the algorithm applied above to build a tree-decomposition of $H$ with width at most the tree-decomposition of $G$. Then by the minimality of the treewidth, \(\tw(H) \leq \tw(G)\). 
\end{proof}

This implies that the set of graphs $\left\{G : \tw(G) \leq k\right\}$ is a minor-closed class. 

\begin{proposition}\label{lem:treewidth_forest}
	Let $G$ be a graph. The treewidth of $G$ equals $1$ if and only if \(G\) is a forest.
\end{proposition}

\begin{proof}
	For the forwards direction, suppose $G$ is not a forest. Then \(G\) has a cycle \(C\), so $G$ has a $K_3$-minor. By \cref{cor:complete_tw}, $\tw(K_3) = 2$. By \cref{thm:tw_minor_closure}, $2 \leq \tw(G)$. Therefore, $G$ has treewidth at least 2. 

	For the backwards direction, suppose \(G\) is a tree. Root the graph \(G\) at the vertex \(r\). Then let \(T = G\). For all non-root vertices $x$, let \(B_x:= \lbrace x, p \rbrace\) where \(p\) is the parent of \(x\). The bag \(B_r\) will just contain \(r\). Then all edges \(vw\) will be between parent \(v\) and child \(w\), so the edge $vw$ will be in bag \(B_w\). Finally, the subgraph induced by vertex \(x\) in \(T\) will be \(B_x\) and the children of \(B_x\), which is a connected subtree.

	If \(G\) is a forest, then perform this operation on every connected component of \(G\) and connect the root vertices to form a new tree. Then this tree is a tree-decomposition of $G$. This forms a tree-decomposition of width 1. An example is in \cref{fig:tree-treedecomp}.
	\begin{figure}[ht]
		\centering
		\usetikzlibrary {graphs,graphdrawing}
 \usegdlibrary {trees}
\tikz [subgraph text bottom=text centered,
subgraph nodes={font=\itshape}]
\graph [tree layout] {
	1 -> { 2 -> {3, 4}, 5 -> {6, 7 -> 8} };
	left [draw] // { b, c, d };
	right [draw] // { e, f, g, h};
};
		\usetikzlibrary {graphs,graphdrawing}
\usegdlibrary {trees}
\tikz 
\graph [tree layout] {
	1 -> { 2 -> {3, 4}, 5 -> {6, 7 -> 8} };
};
		\caption[Tree-decomposition of a tree]{A tree and its tree-decomposition. Every non-root bag consists of a vertex and its parent. The root bag contains a single vertex. Every edge is contained within a single edge.}\label{fig:tree-treedecomp}
	\end{figure}
\end{proof}

\begin{proposition}\label{ex:tw_outerplanar}
	The treewidth of an outerplanar graph $G$ is at most 2.
\end{proposition}
\begin{proof}
	Let \(G'\) be a \textit{weak triangulation} of \(G\), meaning that every face except for the outerface has three vertices. Since \(G\) is a minor of \(G'\), \(\tw(G) \leq \tw(G')\). Then look at the \textit{weak dual} of \(G'\), a tree \(T\) where every vertex \(v_F\) in \(T\) corresponds to an internal face \(F\) in \(G'\). Let \(B_{v_f}\) be the bag of the tree-decomposition, where \(B_{v_f}\) is the set of vertices on the boundary of the face \(f\). Then the tree \(T\) with bags \(B_{v_f}\) is a valid tree-decomposition of \(G'\). Every vertex is on the boundary of some internal face, so every vertex is in some bag. Every bag has at most 3 vertices. Furthermore, every edge is on the boundary of some internal face, so every edge is in some bag. Finally, let $v$ be a vertex. Then the bags that contain $v$ are connected in $T$. This is because the sequence of internal faces which are adjacent to $v$ are connected in $T$. Thus, \(\tw(G) \leq 2\). Refer to \cref{fig:outerplanar_treedecomp} for an example of a tree-decomposition. The green vertices and black edges are an outerplanar graph. The red vertices and blue edges are the weak dual. The magenta circles around green vertices are examples of bags in the tree-decomposition.
	\begin{figure}[h!]
		\centering
		\includesvg[width = 0.5\textwidth]{figures/outerplanar_tree_decomposition.svg}
		\caption[Tree-decomposition of outerplanar graph.]{ The green vertices and black edges is the outerplanar graph. The red vertices and blue edges are the weak dual. The magenta circles around green vertices are examples of bags in the tree-decomposition.}\label{fig:outerplanar_treedecomp}
	\end{figure}
\end{proof}

In fact, graphs of treewidth at most 2 have a simple characterisation.

\begin{proposition}\label{prop:k4-minor}
	A graph $G$ has treewidth at most 2 if and only if $G$ is $K_4$-minor-free. 
\end{proposition}

To prove \cref{prop:k4-minor}, the following lemma is used:
\begin{lemma}
	Every $3$-connected graph have a $K_4$ minor. 
\end{lemma}
\begin{proof}
	Suppose $G$ is $3$-connected. Let $u, v \in V(G)$ be distinct vertices. By $3$-connectedness, there are three internally disjoint paths $P, Q, R$ from $u$ to $v$. Without loss of generality, there exists a vertex $p$ on $P -\{u, v\}$ and $q$ on $Q -\{u, v\}$ where there exists a path $S$ on $G - \{u, v\}$. By finding a minimal path, there exists a path $S'$ internally disjoint from $P, Q, R$ which goes from a vertex in $P - \{u, v\}$ to a vertex in $Q - \{u, v\}$. Then $P \cup Q \cup R \cup S'$ is a $K_4$ minor in $G$. 
\end{proof}

This implies that every $K_4$-minor-free graph is not $3$-connected, therefore every $K_4$-minor-free graph contains a vertex of degree at most 2. Next, we prove \cref{prop:k4-minor}.

\begin{proof}
	For the forwards direction, suppose $G$ contains $K_4$ as a minor. Then from \cref{thm:tw_minor_closure}, $\tw(G) \geq \tw(K_4) = 3$. Therefore, $G$ has treewidth greater than 2. 

	For the backwards direction, we use induction on the number of vertices. Suppose $G$ is $K_4$-minor-free. For the base case, suppose $G$ is $K_3$. Then $G$ has a tree-decomposition where every bag contains at most $3$ vertices. Now suppose $|V(G)| > 3$. Then $G$ contains a vertex $v$ of degree at most 2. Take $u, w$ to be the neighbours of $v$. Now contract $v$ into $u$. By induction on the number of vertices, $G / \{uv\}$ is also $K_4$-minor-free and has a tree-decomposition of width 2. As $vw$ is an edge, there exists a bag $B$ that contains $u$ and $w$. Then add a leaf bag $B'$ to $B$ containing $u, v, w$. This is a tree-decomposition of $G$ with treewidth at most 2.
\end{proof}

%Define a \(k\)-tree inductively. The complete graph \(K_{k+1}\) is a \(k\)-tree. If \(G\) is a \(k\)-tree, then adding any new vertex to \(G\) that is adjacent to a $k$-clique in \(G\) results in another \(k\)-tree.
%A \(k\)-tree is a maximal graph with treewidth \(k\). The following is a well-known fact about $k$-trees. \todo{should there be a proof of this statement?}
%\begin{proposition}
%	For every graph $G$, \(\tw(G) \leq k\) if and only if \(G\) is a subgraph of a \(k\)-tree.
%\end{proposition}

%$k$-trees characterise edge-maximal graphs with bounded treewidth.


\begin{proposition}\label{thm:treewidth_clique-minor-free}
	Every graph with treewidth at most $t$ is $K_{t+2}$-minor-free. 
\end{proposition}
\begin{proof}
	We will prove the contrapositive. Let $G$ be a graph with a $K_t$ minor. Then \(\tw(G) \geq t-1\).
	If \(K_t\) is a minor of \(G\), then from \cref{thm:tw_minor_closure}, \(\tw(K_t) \leq \tw(G)\). As \(\tw(K_t) = t-1\), then \(\tw(G) \geq t - 1\). Therefore, if $\tw(G) \leq t$, then $K_{t+1}$ is the largest complete minor in $G$. Therefore, $G$ is $K_{t + 2}$-minor-free. 
\end{proof}

A graph $G$ is \textit{$k$-degenerate} if every subgraph $H \subseteq G$ has a vertex of degree at most $k$. 

\begin{proposition}
	Every graph with treewidth at most $k$ is $k$-degenerate.
\end{proposition}
\begin{proof}
	Let \( (T, (B_x)_x) \) be a tree-decomposition of $G$. Suppose for neighbouring bags $B_x, B_y$, $B_x \nsubseteq B_y$ and $B_y \nsubseteq B_x$. If $B_x \subseteq  B_y$, then contracting the edge $xy \in T$ is a tree-decomposition of $G$. Now take a leaf bag $B_1$ and let $B_2$ be its neighbour. Now let $v$ be a vertex in $B_1$ but not in $B_2$. So $v$ has degree at most $k$ as there are $k+1$ vertices in $B_1$. As subgraphs of $G$ have treewidth at most $k$, then $G$ is $k$-degenerate.
\end{proof}

\begin{proposition}\label{prop:treewidth_edge_bound}
	For every graph $G$ where $|V(G)| \geq k$, if $\tw(G) \leq k$ then $|E(G)|\leq k |V(G)| - \binom{k + 1}{2}$. 
\end{proposition}

\begin{proof}
	Proof by induction. Suppose $|V(G)| = k$. Then $|E(G)| \leq \binom{k}{2} = k |V(G)| - \binom{k + 1}{2}$. 

	Fix $n > k$. Assume this property holds for every graph with $n-1$ vertices and treewidth at most $k$. Let $G$ be a graph where $\tw(G) \leq k$ and $G \leq n$. Now $G$ is $k$-degenerate, so $G$ has a vertex $v$ of degree at most $k$. So $G - v$ has treewidth at most $k$. Then by the inductive hypothesis, $|E(G - v)| \leq k (n - 1) - \binom{k + 1}{2}$. Since $v$ has at most $k$ neighbours, so $|E(G)| \leq k(n -1) + k - \binom{k + 1}{2} = k |V(G)| - \binom{k + 1}{2}$. 
\end{proof}

\section{Path-width}\label{sec:Pathwidth}
Similar to treewidth, the pathwidth of a graph \(G\) defines how similar $G$ is from a path. A path-decomposition is the same as a tree-decomposition restricted to a path. 

Typically, the path is suppressed in notation, so the bags are placed in order from 1 to $n$. Like for treewidth, The pathwidth of \(G\) is the largest pathwidth over all connected components.
If a graph $G$ has a path-decomposition \({(B_i)}_i\), then $G$ has a tree-decomposition \(\left(P,{(B_i)}_i\right)\). Therefore,
\begin{equation*}
	\pw(G) \geq \tw(G).
\end{equation*}

A graph \(G\) is a \textit{caterpillar} if \(G\) is a tree and $G$ has a path \(P\) where every vertex not in $P$ is adjacent to a vertex on the path \(P\). Alternatively, a tree \(G\) is a caterpillar if removing every leaf yields a path. This path is called the \textit{central path}.
\begin{proposition}
	A graph $G$ has pathwidth at most 1 if and only if every connected component of $G$ is a caterpillar.
\end{proposition}
\begin{proof}
	For the forwards direction, suppose \(G\) has pathwidth 1. Then $G$ is a forest, because $G$ having pathwidth 1 implies $G$ has treewidth 1. Then choose a vertex \(v\) in \(B_1\) and a vertex \(w\) in \(B_n\), the final bag, and look at a path from \(v\) to \(w\). This path goes through every bag. Every vertex not on this path is adjacent to a vertex on this path. Therefore, $G$ is a caterpillar. 
	An example of a caterpillar is in \cref{fig:caterpillar}.

	For the backwards direction, suppose \(G\) is a caterpillar.
	Denote \(P =\left( v_1, v_2, \ldots, v_n\right)\) as the central path. The leaves of vertex \(p_i\) are denoted as \(v_{i, 1}, v_{i, 2} \dots, v_{i, k}\). Define the bags as
	\begin{equation*}
		(v_{1, 1}, v_1), (v_{1, 2}, v_1) ,\ldots ,(v_{1, j}, v_1),  (v_1, v_2), (v_{2, 1}, v_2), (v_{2,2}, v_2,),\ldots ,(v_{n-1}, v_n), (v_{n,1}, v_n), (v_{n,2}, v_n) .
	\end{equation*}
	Each leaf appears once and each vertex on the central path is on a subpath of the path. Every edge is in one bag. Therefore, the pathwidth of \(G\) is 1. If every component of $G$ is a caterpillar, then repeat for every component.
\end{proof}
\begin{figure}[h!]
	\centering
	\includesvg[pretex=\small, width = 0.8\textwidth]{figures/caterpillar}
	\caption[Caterpillar graph]{A caterpillar graph with central path \((v_1, v_2, v_3, v_4, v_5, v_6)\).}\label{fig:caterpillar}
\end{figure}

\begin{example}
	Recall that $K_n$ is the complete graph on $n$ vertices. It holds that \(\pw(K_n) = \tw(K_n) = n - 1\).
\end{example}
\begin{proof}
	The pathwidth of \(K_n\) is at least the treewidth of \(K_n\). But the pathwidth is at most \(n- 1\) (where all the vertices are in the same bag), but the treewidth of \(K_n\) is \(n - 1\). Therefore, \(\pw(K_n) = n - 1\).
\end{proof}

\begin{proposition}
	The pathwidth of a tree \(T\) equals \(\min_{P \subseteq T} \left\lbrace 1 + \pw(T - V(P))\right\rbrace \) where \(P\) is a path.
\end{proposition}

\begin{proof}[Proof]
	We prove using induction. The pathwidth of a path $P$ is $1$.

	We show \(\pw(T) \leq 1 + \pw(T - V(P))\) for all $P$. If \(P\) is a path in \(T\) with vertices \(v_1, v_2, \ldots v_i\), then consider the subtrees hanging off \(v_i\) for all \(i\). \(T - V(P)\) will have a path-decomposition of width $\pw(T - P)$. Order each connected component such that they appear in the order of their parents on the paths. Then adding \(v_i\) to the bags of subtrees connected to \(v_i\), and the bag \((v_i, v_{i+1})\) between the subtrees \(v_i\) and \(v_{i + 1}\) will yield a path-decomposition of width \(1 + \pw(T - V(P))\) inductively.

	We show there exists $P$ such that \(\pw(T) \geq 1 + \pw(T - V(P))\). Proceed by induction. Let \(B_1, \ldots B_n\) be a path-decomposition of \(T\). Let \(x\) live in bag \(B_1\) and \(y\) live in bag \(B_n\), the final bag. Then let \(P\) be the unique path from \(x\) to \(y\). Then \(P\) traverses through every bag in the path-decomposition. Therefore, \(\pw(T) \geq 1 + \pw(T - P)\) by adding every parent to the bag of each component. 
\end{proof}

The set of graphs $\{G : \pw(G) \leq k\}$ is a minor-closed class for the same reason that the set of graphs $\{G : \tw(G) \leq k\}$ is a minor-closed class.
Ternary graphs of depth $d$ have pathwidth $d-1$. Graphs with path-width at least $w - 1$ contain every $w$-vertex forest as a minor. \textcite{seymourShorterProofPathwidth2023} provides a short proof of the above statement. 
\begin{theorem}
	$H$-minor free graphs have bounded pathwidth if and only if $H$ is a forest. 
\end{theorem}

\begin{theorem}
	Suppose $H$ is a forest. Then \textcite{seymourShorterProofPathwidth2023} proved that every graph that does not contain $H$ as a minor has path-width at most $|V(H)| - 2$. Note that the original proof was by \textcite{robertsonGraphMinorsExcluding1983}. 
	Suppose $H$ is not a forest. Then $H$ is not a minor of any tree, so $H$ is not a minor of any ternary tree. But ternary trees have unbounded pathwidth, so $H$-minor free graphs have unbounded pathwidth.
\end{theorem}


\subsection{Bounding the number of pages of a planar graph}
This subsection uses \cref{lem:planar_graphs_4_connected_cliqesums} and \cref{thm:clique_sum_pagenumber_bound} to find a book-embedding of planar graphs. This book-embedding is different from the one provided by \textcite{yannakakisEmbeddingPlanarGraphs1989} as it does not require a planar triangulation. Because of this fact, this proof is used in future sections with respect to adding vortices on faces. 

Firstly, \textcite{tutteTheoremPlanarGraphs1956} proves an important theorem regarding $4$-connected planar graphs.

\begin{theorem}\label{thm:4-connected_planar_ham_cycle}
	Every 4-connected planar graph is Hamiltonian.
\end{theorem}

\cref{thm:4-connected_planar_ham_cycle} is used to prove \cref{thm:Planar Graph Hickingbotham Bound}.

\begin{corollary}\label{thm:Planar Graph Hickingbotham Bound}
	Let \(G\) be a 2-connected planar graph. Then $G$ can be embedded on $11$ pages, with book-embedding $(<, \rho)$. $<$ restricted to the vertices outer cycle $C$ traverses every vertex in order of the traversal of the boundary of the outerface. Furthermore, for every face cycle $C$, $<_{V(C) - \{u, v, w\}} = C - \{u, v, w\}$ for some vertices $u$, $v$, $w$. 
\end{corollary}
\begin{proof}
	From \cref{thm:4-connected_planar_ham_cycle}, every $4$-connected planar graph is Hamiltonian. Furthermore, $K_4$ is a Hamiltonian planar graph.
	Recall that Hamiltonian planar graphs can be embedded on two pages, from \cref{lem:Pagenumber_2}. 
	Then apply \cref{thm:clique_sum_pagenumber_bound} with tree-decomposition from \cref{lem:planar_graphs_4_connected_cliqesums} to $G$. Then $G$ can be embedded on \(2 \cdot 4 + 3 = 11\) pages. Furthermore, the embedding restricted a face boundary preserves the ordering of the facial walk.

	From the construction given in \cref{lem:planar_graphs_4_connected_cliqesums}, every $4$-connected component is glued to another $4$-connected component by a separating face. Therefore, every face only changes by $3$ vertices, from \cref{thm:clique_sum_pagenumber_bound}. Therefore, removing $3$ vertices from every face preserves the cyclic ordering of every face.
\end{proof}



\section{Surfaces and graphs on surfaces}

The terminology in this section is based on \textcite{moharGraphsSurfaces2001} Graphs on Surfaces. An \textit{$n$-manifold} $M$ is a second-countable Hausdorff space where every point in $M$ has an open neighbourhood homeomorphic to an open ball in $\mathbb{R}^n$. A surface is a compact $2$-manifold. 

\textit{Handles} are added to a surface \(\Sigma\) by removing two disks in \(\Sigma\) and identifying the boundaries such that one goes clockwise, and the other goes counter-clockwise. \textit{Crosscaps} are added to a surface $\Sigma$ by removing a disk in \(\Sigma\) and identifying opposite points on the boundary. Every surface is homeomorphic to a sphere with $m$ handles and $n$ crosscaps. This is known as the classification of surfaces. The \textit{Euler genus} of a surface \(\Sigma\) with $m$ handles and $n$ crosscaps is $2m + n$.

Furthermore, a sphere with one handle and one crosscap is homeomorphic to a sphere with three crosscaps. Therefore, any sphere with a mix of handles and crosscaps is homeomorphic to one with all crosscaps. Euler genus is an invariant under homeomorphism. 

These are the Euler genus of some surfaces.
\begin{enumerate}
	\item The Euler genus of the sphere is \(0\).
	\item The Euler genus of the torus is \(2\).
	\item The Euler genus of the projective plane is \(1\). 
	\item The Euler genus of the Klein bottle is \(2\). 
\end{enumerate}

Note that ``genus'' and ``Euler genus'' are two distinct concepts. In many works, ``genus'' refers to the orientable genus. 

The orientability of a surface is an important tool to distinguish surfaces. A surface \(\Sigma\) is \textit{orientable} if \(\Sigma\) can be obtained from \(S^2\) by only adding handles. An example of an orientable surface is the torus. A surface \(\Sigma\) is \textit{non-orientable} if \(\Sigma\) can only be obtained from \(S^2\) by adding at least one crosscap. An example of a non-orientable surface is the projective plane or the Klein bottle. Compact orientable surfaces can be embedded on $\mathbb{R}^3$, but non-orientable surfaces cannot.

An \textit{embedding} of $G$ on a surface $\Sigma$ is a drawing of $G$ on $\Sigma$ such that no two edges cross. 
A \textit{$2$-cell embedding} of a graph $G$ on a surface $\Sigma$ is an embedding of $G$ in $\Sigma$ such that every connected component of $\Sigma - G$ is homeomorphic to an open disk. This is also referred to as a \textit{map}.

We now discuss some terminology of graphs on surfaces. Let $G$ be a graph and $\Sigma$ be a surface. Embedding $G$ in $\Sigma$ is referred to as the \textit{embedding} of $G$ in $\Sigma$, whereas embedding $G$ in a book is referred to as the \textit{layout} of $G$. The orientable genus of a graph \(G\), denoted \(\gamma(G)\), is the minimum genus of an orientable surface $\Sigma$ such that $G$ has an embedding on $\Sigma$. The non-orientable genus of a graph \(G\), denoted \(\tilde{\gamma}(G)\), is the minimum genus of a non-orientable surface $\Sigma$ such that $G$ has an embedding on $\Sigma$. 
The \textit{Euler Genus} of a \textit{graph} \(G\) is the smallest Euler genus \(g\) surface \(\Sigma\) such that \(G\) can be $2$-cell embedded on $\Sigma$.

\textcite{moharOrientableGenusGraphs1998} showed that \(\tilde{\gamma}(G) \leq 2 \gamma(G) + 1\) for every graph, meaning that if the orientable genus is bounded, then the non-orientable genus is bounded.\ \textcite{auslanderImbeddingGraphsManifolds1963} showed that there exists graphs which are embeddable on the projective plane that have arbitrarily large orientable genus. 

An extension for Euler's formula is below. Suppose $G$ is $2$-cell embedded on a surface $\Sigma$ of genus $g$. Let \(|F(G)|\) be the number of faces in a graph \(G\). Then \(|V(G)| - |E(G)| + |F(G)| = 2 - g = \chi\). When $g = 0$, then $\Sigma$ is a $2$-sphere and this is the original Euler's formula. 
The value $\chi$ is known as the \textit{Euler characteristic} of a topological space, in this case a surface. The Euler characteristic is invariant under homeomorphism. Calculating the Euler characteristic of any space is done through \textit{homological algebra}, specifically by looking at the free rank of homology groups. 

Graphs that can be embedded on the plane are called \textit{planar} graphs. Graphs that can be 2-cell embedded on the torus are called \textit{toroidal} graphs, and graphs that can be 2-cell embedded on the projective plane are called \textit{projective-planar} graphs. Graphs that can be 2-cell embedded on a surface of genus $g$ are called \textit{genus $g$} graphs. Similarly to plane graphs, \textit{torus graphs} are graph drawings on the torus, and \textit{projective-plane graphs} are graphs drawings on the projective plane. 

The family of graphs embeddable on a fixed surface $\Sigma$ is a minor-closed family. If $G$ is embedded on $\Sigma$, then $G - v$ for any vertex $v$ and $G - e$ for any edge $e$ is also embeddable on $\Sigma$. Furthermore, contracting any edge $e$ in $G$ maintains the property that no two edges cross. Edge contraction is a topological action on a graph and can be viewed as an ambient isotopy of $G$ on $\Sigma$. 
If $G$ is 2-cell embedded on a surface $\Sigma$ and every face in $G$ has three distinct vertices on its boundary, then $G$ is a \textit{triangulation} of $\Sigma$. Given graphs $G$ and $H$ with genus $g_1, g_2$,a new graph with genus $g_1 + g_2$ can be constructed.
\begin{theorem}[\textcite{millerAdditivityTheoremGenus1987}]\label{thm:additivity_genus}
	Let graphs $G$ and $H$ have genus $g_1$, $g_2$. Then the graph obtained from identifying a vertex in $G$ to a vertex in $H$ has genus $g_1 + g_2$. 
\end{theorem}

Next is an extension of \cref{thm:K5_Free_Planar} for graphs embedded on surfaces. 

\begin{theorem}\label{thm:bounded_genus_kt_free}
	If \(G\) is an Euler genus \(g\) graph, then \(G\) is \(K_t\)-minor free, where \(t > \sqrt{6g} + 4\). 
\end{theorem}
\begin{proof}
	This proof mimics the above proof for planarity, but on surfaces of higher genus. 
	Suppose \(G\) has \(n\) vertices and \(m\) edges and of Euler genus $g$. Then \(n - m + f = \chi = 2-g\), from Euler's theorem on surfaces. As at least three vertices bound each face and each edge touches exactly two faces, then \(f \leq 2m/3\). Therefore, \(m \leq 3(n + g - 2)\). If \(K_t\) is embeddable on a genus \(g\) graph, then \(\binom{t}{2} \leq 3 (t + g - 2)\). Thus \(t \leq \sqrt{6g} + 4\). So if $G$ has genus \(g\), then $G$ is \(K_t\)-minor free, where \(t > \sqrt{6g} + 4\). 
\end{proof}

In the case when the surface is a torus, $K_7$ is a toroidal graph but $K_8$ is not. An example of an embedding of $K_7$ on a torus is in \cref{fig:k7_on_torus}.

\begin{figure}[h!]
	\centering
	\includesvg[height = 0.3\textheight]{figures/k7 on torus.svg}
	\caption[Toroidal graph]{An example of a toroidal graph $K_7$ embedded on a torus.}\label{fig:k7_on_torus}
\end{figure}

\begin{proposition}
	$K_8$ is not embeddable on the torus.
\end{proposition}
\begin{proof}
	A torus has genus 2. By Euler's equation, if a graph $G$ is embedded on a torus, then $|V(G)| - |E(G)| + |F(G)| = 2 - 2 = 0$, where $|F(G)|$ counts the number of faces on the surface. Every face bounds at least three vertices and every edge touches two faces. Therefore, $|F(G)| \leq 2|E(G)|/3$. Suppose $K_8$ is embeddable on the torus. Then $|V(G)| = 8$ and $|E(G)| = 28$. Therefore, $|F(G)| = 20$. But $|F(G)| \leq 2 (28)/3 \leq 19$. Therefore, $K_8$ is not embeddable on the torus.
\end{proof}

A famous theorem involving map colourings on surfaces is Heawood's conjecture, from \textcite{heawoodMapcolourTheorem1890}. This theorem is also called the Map Colour Theorem. Piecewise linearly partition a surface $\Sigma$ into path-connected faces homeomorphic to a disk. Then a \textit{map} of $\Sigma$ is the graph obtained by placing a vertex at each face and placing an edge when two faces touch at a line. Heawood showed that the minimum number of colours sufficient to colour all Euler genus $g$ maps when $g \geq 1$ is
	\begin{equation*}
		\gamma(g) := \left\lfloor 
		\frac{7 + \sqrt{1 + 24g}}{2}
		\right\rfloor.
	\end{equation*}
When $g = 0$, this is the Four-Colour theorem, which was unproven when \textcite{ringelMapColorTheorem1974} was written.  
However, Heawood did not show that $\gamma(g)$ is necessary, which became Heawood's conjecture. 
Ringel and Young \cite{ringelMapColorTheorem1974} showed that for almost every case, $\gamma(g)$ is also necessary, and proved Heawood's conjecture. The case where this does not hold is the Klein-bottle case. Every Klein-bottle graph is $6$-colourable, but $\gamma(g) = 7$. 

Let $I = [0, 1]$.
A \textit{loop} is a continuous function $\gamma : I \rightarrow X$ where $\gamma(0) = \gamma(1) = x_0$. The point $x_0$ is the \textit{base point}. A \textit{homotopy} between two loops $\alpha, \beta$ is a continuous map $h : I \times I \rightarrow (x)$ where $h(0, t) = h(1, t) = x$ for all $t$, $h(\cdot, 0) = \alpha$, $h(\cdot, 1) = \beta$. A \textit{null-homotopic} loop is a loop homotopy to the constant map at $x_0$, and a \textit{nontrivial} loop is one that is not null-homotopic. On a sphere or plane, all loops are null-homotopic. Homotopic and null-homotopic loops come up in our discussion of graphs on surfaces as they can be used to classify edges embedded on a surface when the graph is a single point $x_0$. 


\section{Graph Minor Structure Theorem}\label{sec:Kt_Minor_Free}
\textcite{robertsonGraphMinorsXVII1999} proved a rough characterisation of all \(K_t\)-minor free graphs. 

Every graph that is $K_t$-minor-free can be constructed from the following ingredients. This is a coarse characterisation of $K_t$-minor free graphs, meaning that a subset, or a single one of these ingredients constitutes a $K_t$-minor free graph. 
\begin{itemize}
	\item Graphs of bounded Euler genus.
	\item Sets of apex vertices.
	\item Graphs of bounded treewidth.
	\item Sets of vortices on graphs.
\end{itemize}
\textcite{robertsonGraphMinorsXVII1999} showed that every \(K_t\)-minor free graph can be built up from smaller graphs with the above ingredients.

\subsection{Graphs of bounded Euler genus}

Graphs embeddable on a surface of Euler genus $g$ are $K_t$ minor-free, where \(t > \sqrt{6g} + 4\). This comes from \cref{thm:bounded_genus_kt_free}. 

In the case when the surface is a torus, $K_7$ is a toroidal graph but $K_8$ is not. An example of an embedding of $K_7$ on a torus is in \cref{fig:k7_on_torus}.

\begin{figure}[h!]
	\centering
	\includesvg[width = 0.8\textwidth]{figures/k7 on torus.svg}
	\caption[Toroidal graph]{An example of a toroidal graph $K_7$ embedded on a torus.}\label{fig:k7_on_torus}
\end{figure}

\begin{proposition}
	$K_8$ is not embeddable on the torus.
\end{proposition}
\begin{proof}
	A torus has genus 2. By Euler's equation, if a graph $G$ is embedded on a torus, then $|V(G)| - |E(G)| + |F(G)| = 2 - 2 = 0$, where $|F(G)|$ counts the number of faces on the surface. Every face bounds at least three vertices and every edge touches two faces. Therefore, $|F(G)| \leq 2|E(G)|/3$. Suppose $K_8$ is embeddable on the torus. Then $|V(G)| = 8$ and $|E(G)| = 28$. Therefore, $|F(G)| = 20$. But $|F(G)| \leq 2 (28)/3 \leq 19$. Therefore, $K_8$ is not embeddable on the torus.
\end{proof}


\subsection{Apex sets}\label{sssec:Apex_Vertices}
Let $G$ be a graph. A set of vertices $A \subseteq V(G)$ is an apex set if $G - A$ has some bounded parameter. Common parameters are planarity or bounded genus. 
\begin{proposition}
	Let $G$ be a graph. If \(G-a\) is \(K_{t}\)-minor free, then $G$ is $K_{t+1}$-minor free. 
\end{proposition}
\begin{proof}
	We shall prove the contrapositive. Suppose \(G\) has a \(K_{t + 1}\) minor. Then \(K_{t + 1}\) has a model $\rho$ in \(G\). Now let \(v\) be the vertex in \(K_{t + 1}\) such that \(\rho(v)\) contains \(a\). Then delete \(v\) from \(K_{t + 1}\) to form $K_t$. \(K_t\) is a minor of \(G - \rho(v)\). But \(G - \rho(v)\) is a minor of \(G - a\), as \(G - \rho(v)\) does not contain \(a\). So \(G - a\) has a \(K_t\) minor. 
\end{proof}
\subsection{Treewidth and clique-sums}\label{sssec:Clique_Sums}
The \textit{\(k\)-clique-sum} of two graphs \(G\) and \(H\) is a new graph formed from both $G$ and $H$ by identifying two cliques together. The clique-sum of $G$ and $H$ is \(G \oplus_k H\), and is defined as follows. Find cliques in \(G\) and \(H\), \(C_G\) and \(C_H\) respectively, such that both \(C_G\) and \(C_H\) have size \(k\). Identify the vertices in \(C_G\) and \(C_H\) to glue \(G\) and \(H\) together, and possibly delete edges in $C_G$. An illustration can be found in \cref{fig:clique-sum}. 

\begin{figure}[h]
	\centering
	\includesvg[width=0.7 \textwidth]{figures/Clique-sum}
	\caption[Clique-sum]{Figure of clique-sum. Public domain image from David Eppstein \cite{eppsteinCliquesum2023}.}
	\label{fig:clique-sum}
\end{figure}


\begin{proposition}
	Let $t$ be an integer $\geq 1$. Let $G_1, G_2$ be two graphs with treewidth $t$. Then for all $k \leq t + 1$, $G_1 \oplus_k G_2$ has treewidth $t$. 
\end{proposition}
\begin{proof}
	Suppose $C = V(G_1) \cap V(G_2)$ be the clique that is glued over, where $|C| = k$. Let $(B_x: x \in T_1)$ be a tree-decomposition of $G_1$ of minimum width. Let $(B_x : x \in T_2)$ be a tree-decomposition of $G_2$ of minimum width. Then $C$ must appear in some bag $A_x$ and $B_y$ by \cref{lem:clique}. Let $T = T_1 \sqcup T_2$. Add a new node $u$ to $T$ and let $B_u = C$. Then add edges $ux, uy$ to $E(T)$ to form a new tree. Every vertex not in $C$ has a subtree in $T$. If $v \in C$, then the induced subgraph in $T$ is the graph $T_1(v) \cup T_2(v) \cup u$. $T_1(v) \cup T_2(v) \cup u$ is a subtree of $T$. Finally, every edge in $G_1 \cup G_2$ remains in $T$. Therefore, $T$ is a tree-decomposition of $G_1 \oplus_k G_2$. The size of each bag in $T$ is still at most $t + 1$, so the treewidth of $G_1 \oplus_k G_2 \leq t$.
\end{proof}

\begin{proposition}
	Let $t$ be an integer $\geq 1$. Suppose $G$ and $H$ are $K_t$-minor-free graphs. Then $G \oplus_k H$ is $K_{t}$-minor free, $k < t$.  
\end{proposition}
\begin{proof}
	Let $C = V(G) \cap V(H)$ be the clique that is being pasted over. As $G$ and $H$ are $K_t$-minor free, then $|C| \leq k - 1$. Suppose $G \oplus_k H$ is not $K_t$-minor free. Then there exists a model $\rho: V(K_t) \rightarrow G\oplus_k H$ of $K_t$ in $G \oplus_k H$. $\rho$ cannot have its image only in $G$ or only in $H$, as that would mean that $G$ or $H$ has a $K_t$ minor. Therefore, every connected subgraph of $\rho$ must use a vertex in $C$. But $C$ has only $k$ vertices, and $k < t$. Since $\rho$ has disjoint subgraphs, every vertex in $C$ belongs in at most one subgraph in $\rho$. Therefore, $\rho$ cannot have $t$ subgraphs, which is a contradiction. Therefore, $G \oplus_k H$ is also $K_t$ minor free. 
\end{proof}

\begin{corollary}\label{corr:clique_sum_genus}
	If \(G\) is the clique-sum of Euler genus \(g\) graphs, then \(G\) is \(K_{\lceil \sqrt{6g} + 5 \rceil}\)-minor-free.
\end{corollary}
The reverse does not hold. 
\begin{proposition}
	There exists graphs $G$ where \(G\) has arbitrarily large genus, but $G$ is \(K_{6}\)-minor-free.
\end{proposition}

\begin{proof}
	Consider $n$ copies of $K_5$ and identify one vertex in every $K_5$ to a single vertex $v$ to form $G$. Then $G$ is $K_6$-minor free. However, from \cref{thm:additivity_genus}, $G$ has genus $n$. Thus, $G$ has unbounded genus. 
\end{proof}

\subsubsection{Torsos and adhesion}\label{sssec:Torsos and Adhesion}
Given a graph \(G\) and a tree-decomposition \(\tree\), the \textit{torso} of a bag \(B_x\) of \(T\) is the graph \(G\langle B_x \rangle\), with vertex set $B_x$ and edge set where \(vw\) is an edge in \(G\langle B_x \rangle\) if and only if $vw \in E(G)$ or \(v,w \in B_x \cap B_y\), where \(y\) is any neighbour of \(x\) in \(T\). The edge $uv$ where $uv \in B_x \cap B_y$ are called \textit{torso edges}. The set \(B_x \cap B_y\) for all neighbours \(y\) of \(x\) in \(T\) is a clique in \(G\langle B_x \rangle\).
The \textit{adhesion set} is the set \(B_x \cap B_y\). 
The \textit{adhesion} of a tree is defined as \(\max(|B_x \cap B_y|)\) where \(xy\) is an edge in \(T\).

Given \(G\) and a tree-decomposition \(\tree\), \(G\) is a clique-sum of the torsos \(G\langle B_x \rangle\) where the size of the cliques that we paste over is at most the adhesion of $\tree$. This holds for any arbitrary tree-decomposition.
We will discuss decomposing graphs in the language of tree-decompositions, rather than clique-sums. This is because we can discuss the structure of the tree-decomposition.

\subsection{Vortices}\label{sssec:vortices}
Let \(G\) be embedded on a surface \(\Sigma\), and let \(F\) be a face on \(G\). A disc $D$ is \textit{$G$-clean} if $D$ is an open subset of $F$ and $G \cap D$ is a tuple of vertices \(\Lambda = (x_1, x_2, \ldots, x_b)\). No vertex appears more than once in $\Lambda$. The ordering of $\Lambda$ is around the boundary of $D$. 
\par
Let $G$ be a graph embedded on $\Sigma$. Let $D$ be a $G$-clean disc with $G \cap D = \Lambda = (x_1, x_2, \ldots, x_k)$. A \textit{$D$-vortex} is a graph $H$ such that $V(G) \cap V(H) = \Lambda$ and there is a \textit{path-decomposition} of \(H\) of bags \(B_1, B_2, \ldots B_k\) such that \(x_i \in B_i\) for all \(i\). The \textit{depth} of the vortex $H$ is the path-width of $H$. 
\par
The following figure, \cref{fig:tenniscourt} demonstrates the necessity of vortices. $G_n$ is $K_8$-minor free. However, $G_n$ has around $\frac{n}{3}$ $K_{3,3}$ copies, so has genus around $\frac{2n}{3}$. As there is an $n \times n$ grid minor, $G$ has treewidth at least $n$. As $G$ can be arbitrarily large, the number of apex vertices to remove to bound the treewidth and genus is arbitrarily large. However, there is a decomposition of $G_n$ into two graphs $G_0$ and $G_1$ where $G_0$ can be embedded on a surface and $G_1$ is a vortex on $G_0$ with depth 6. $G_0$ is the $n \times n - 1$ grid and $G_1$ is the $n \times 2$ grid in the back plus the apex vertices. 

\begin{figure}[h]
	\centering
	\includesvg[width = 0.8\textwidth]{figures/tenniscourt}
	\caption[Tennis-Court graph]{An example of an $n \times n$ \textit{tennis-court} graph $G_n$ which is \(K_8\) minor free.}
	\label{fig:tenniscourt}
\end{figure}
\subsection{Robertson-Seymour Graph Minor Structure Theorem}\label{ssec:Robertson_Seymour_Graph_Structure}
Given integers \(g, p, a \geq 0\), \(k \geq 1\), a graph \(G\) is \((g, p, k, a)\)- almost-embeddable if there exists an \(A \subseteq V(G)\) with \(|A| \leq a\), and there exists subgraphs \(G_0, G_1, \ldots,  G_{p'}\) of \(G\) such that:
\begin{itemize}
	\item \(G - A = G_0 \cup G_1 \cup G_2 \cup \ldots \cup G_{p'}\),
	\item \(p' \leq p\),
	\item there is an embedding of \(G_0\) onto a surface \(\Sigma\) of genus \(\leq g\),
	\item there exists pairwise disjoint \(G_0\)-clean discs \(D_1, D_2, \ldots, D_{p'}\) in \(\Sigma\),
	\item \(G_i\) is a \(D_i\)-vortex of depth at most \(k\).
\end{itemize}

If we restrict $G_0$ to live only on orientable surfaces, then the graph $G$ is ${(g, p, k, a)}^+$-almost embeddable. If the apex set $A$ is empty, then the graph $G$ is $(g, p, k)$-almost embeddable. 

\begin{theorem}[Graph Minor Structure Theorem \cite{robertsonGraphMinorsXVI2003}]\label{thm:gmst}\todo{what is $\ell$ with respect to the other constants?}
	For all \(t\), there exists \(g, p, a \geq 0\) and \(k, \ell \geq 1\) such that every \(K_t\)-minor-free graph has a tree-decomposition of adhesion \(\leq \ell\) and each torso is \((g, p, k, a)\)-almost-embeddable. The  family of graphs with tree-decomposition of adhesion $\leq \ell$ with torsos $(g, p, k, a)$-almost-embeddable is \(\mathcal{G}(g, p, k, a)\). 
\end{theorem}
There exists a function \(t(g, p, k, a)\) such that if a graph has a tree-decomposition of adhesion \(\leq \ell\) and each torso is \((g, p, k, a)\)-almost embeddable, then \(G\) has no \(K_t\) minor.

\textcite{kawarabayashiQuicklyExcludingNonplanar2021} found upper bounds for $g, p, k, a$. 
\begin{theorem}[\textcite{kawarabayashiQuicklyExcludingNonplanar2021}]
	Let $t \geq 1$ be a positive integer. Let $G$ be a $K_t$-minor free graph. Then let $\alpha = t^{18 \cdot 10^{7} t^{26} + c}$ for a constant $c$, which is defined in the paper. Then setting $g = t(t+1)$, $p = 2t^2$, $k = \alpha$, $\ell = 4\alpha$, and $a = 3\alpha$, $G \in \mathcal{G}(g,p,k,a)$. 
\end{theorem}

\textcite{joretCompleteGraphMinors2013} studies the question of the maximum order of a complete graph minor of a graph in $\mathcal{G}(g, p, k, a)$. 
\begin{theorem}[\textcite{joretCompleteGraphMinors2013}]\label{thm:graph_structure_bound_theorem}
	For all graphs \(G \in \mathcal{G}(g, p, k, a)\),
	\(\had(G) \leq a + 48(k + 1)\sqrt{g + p} + \sqrt{6g} + 5\). Moreover, for some constant $c$, for all $g, a \geq 0$, $p \geq 1, k \geq 2$, there exists a graph $G$ in \(\mathcal{G}(g, p, k, a)\) such that in \(n \geq a + c k\sqrt{p + g}\) such that \(K_n\) is a minor of $G$.
\end{theorem}

\section{Graph Minor Theorem}\label{sec:Graph Minor Theorem}
We move on to one of the most important and deepest theorems in graph theory, the Graph Minor Theorem. This was proven in a series of 23 papers by Robertson and Seymour, from 1983 to 2004. As part of the proof, the Graph Minor Structure Theorem was developed. For the case when the infinite set is the family of trees, this is Kruskal's tree theorem \cite{kruskalWellQuasiOrderingTreeTheorem1960}. 
\begin{theorem}[\textcite{robertsonGraphMinorsXX2004} Graph Minor Theorem]
	Every infinite set of graphs contains two distinct graphs \(G\) and \(H\) such that \(H\) is a minor of \(G\).
\end{theorem}
%However, this infinite family can be extremely large. Let $T_1, \ldots T_m$ be a sequence of rooted trees from the labels $\{1, 2, 3\}$ where each $T_i$ has at most $i$ vertices. By Kruskal's theorem, when $m$ becomes large enough, there exists a $i, j$ such that $1 \leq i < j$ such that $T_i$ is a label-preserving minor of $T_j$. $TREE(3)$ is the largest $m$ such that there is no label-preserving minor. 
Let $\mathcal{F}$ be a minor-closed graph family. A graph $H$ is a \textit{forbidden minor} of $\mathcal{F}$ if every graph $G$ in $\mathcal{F}$ is $H$-minor free and every proper minor $H'$ of $H$ (meaning $H'$ is not $H$) is in $\mathcal{F}$. 
The Graph Minor Theorem is equivalent to the statement:
\begin{theorem}
	Every minor-closed graph family $\mathcal{F}$ is characterised by a finite set of forbidden minors $\mathcal{H}$. A graph $G$ is in $\mathcal{F}$ if and only if $G$ is $H$-minor free for every $H$ in $\mathcal{H}$. 
\end{theorem}
The family of graphs that can be embedded on a fixed surface $\Sigma$ is minor-closed. 
For planar graphs, the two forbidden minors are \(K_5\) and \(K_{3,3}\), from \textcite{wagnerUeberEigenschaftEbenen1937}. 
The family of graphs that can be embedded on a torus are the toroidal graphs. 17,523 forbidden toroidal minors have been found, with a database maintained by \textcite{myrvoldLargeSetTorus2018}. A complete enumeration of forbidden minors has not been found. 

A graph $G$ is \textit{linkless} if $G$ has an embedding in $\mathbb{R}^3$ such that no two cycles are linked. If no embedding of $G$ has this property, then $G$ is \textit{inherently linked}. The family of linkless graphs is minor-closed. If $G$ is linkless, then contracting any edge maintains the linkless property. Linkless graphs only have seven forbidden minors, including $K_6$ and the Petersen graph, which was found by \textcite{robertsonSachsLinklessEmbedding1995}. 

A graph $G$ is \textit{knotless} if $G$ can be embedded in $\mathbb{R}^3$ such that every cycle of $G$ is the unknot. $G$ is \textit{inherently knotted} if this is not the case.\ \textcite{conwayKnotsLinksSpatial1983} showed that $K_7$ is inherently knotted. The family of knotless graphs is also minor-closed, since contracting any edge preserves the knotless property. $K_7$ is an example of a forbidden minor. \textcite{conwayKnotsLinksSpatial1983,foisyNewlyRecognizedIntrinsically2003,foisyIntrinsicallyKnottedGraphs2002} found that there exists at least three minimal minors, but a complete enumeration of forbidden minors have not been found. 

%This chapter is a brief look into some of the deepest theorems in graph theory and a discussion of some of the theorems that were used to prove the Graph Minor Theorem. This chapter discusses the Graph Minor Structure Theorem, which is an important theorem that will be used throughout this thesis to prove some theorems. 

\section{Minor-closed graph families}\label{sec:minor_closed_families}
This section is a compilation of important information on small $K_t$ minor free graphs and some important minor-closed families. There is no known good characterisation of $K_6$ minor free graphs. 

\begin{table}[h!]
    \centering
    \begin{tabular*}{\textwidth}{@{}lll@{}}
        \toprule
        $K_n$ minor free graph  & Characterisation  & Reference \\
        1                       & Empty graph       &           \\
        2                       & Edgeless graph    &           \\
        3                       & Forests           &           \\
        4                       & Treewidth $\leq 2$&  {\textcite{norinMath599GraphMinors2017}}         \\
        5                       & Clique-sums of planar and Wagner graphs & {\textcite{wagnerUeberEigenschaftEbenen1937}}
    \end{tabular*}
\end{table}


\cref{tab:minor-closed families} compiles a list of minor-closed graph families and its forbidden minors. \# MFM stands for ``Number of forbidden inors''.

\begin{table}[h!]
    
    \centering
    \begin{tabular*}{\textwidth}{@{}llll@{}}
    \toprule
    Family name                  & \# MFM & List/Partial list of minors                      & Refs \\ \midrule
    Forests                      & 1                                  & $K_3$                                            &            \\
    Planar Graphs                & 2                                  & $K_5$, $K_{3,3}$                                 & \tablefootnote{\textcite{wagnerUeberEigenschaftEbenen1937}}           \\
    Toroidal Graphs              & $\geq 17523$                       & $K_8$                                            & \tablefootnote{\textcite{myrvoldLargeSetTorus2018}}           \\
    Projective-Planar Graphs     & 35                                 & Petersen Family                                  & \tablefootnote{\textcite{moharGraphsSurfaces2001}}           \\
    Graphs embedded on a fixed surface & & &\\
    $\tw(G) \leq 2$              & 1                                  & $K_4$                                            &            \\
    $\tw(G) \leq 3$              & 4                                  & $K_5$, $K_{2,2,2}$, $P_2 \square C_5$, Wagner graph & \tablefootnote{\textcite{arnborgForbiddenMinorsCharacterization1990}}           \\
    $\tw(G) \leq 4$              & $\geq 75$                          & $K_6$                                            & \tablefootnote{\textcite{sandersLinearAlgorithmsGraphs1993}}           \\
    $\tw(G) \leq k$              &                                    &                                                  &            \\
    Linklessly Embeddable graphs & 7                                  & $K_6$, Petersen Family                           & \tablefootnote{\textcite{conwayKnotsLinksSpatial1983}} \\
    Knotlessly Embeddable graphs & $\geq 3$                           &                                                  & \tablefootnote{\textcite{conwayKnotsLinksSpatial1983,foisyIntrinsicallyKnottedGraphs2002,foisyNewlyRecognizedIntrinsically2003}}          
    \end{tabular*}
    \caption[Table of Minor-Closed Families of graphs]{A table of minor-closed families of graphs. This is listed in no particular order. There are infinitely many minor-closed graph families. As an example, a family $\mathcal{F}_G$ could be the family of graphs that do not contain graph $G$ as a minor. This is minor-closed, and there are infinitely many of these families.}\label{tab:minor-closed families}
    \end{table}
