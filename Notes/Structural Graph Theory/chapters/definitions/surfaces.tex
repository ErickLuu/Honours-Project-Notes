\section{Surfaces and graphs on surfaces}
This section discusses surfaces, and graphs on surfaces. Graphs on surfaces is a natural extension to graphs on planes. Furthermore, graphs on fixed surfaces are an essential ingredient to the Graph Minor Structure Theorem.

\subsection{Important definitions from topology}
A \textit{topological space} is a set $(X, \mathcal{F})$ where $\mathcal{F} \subseteq 2^X$, with the following properties:
\begin{itemize}
    \item $\emptyset, X \in \mathcal{F}$,
    \item $\mathcal{F}$ is closed under arbitrary union,
    \item $\mathcal{F}$ is closed under finite intersection.
\end{itemize} 
If $U \in \mathcal{F}$, then $U$ is open in $X$. A function $f : X \rightarrow Y$ between two topological spaces is \textit{continuous} if the preimage of any open set in $Y$ is open in $X$. Furthermore, if a function $f : X \rightarrow Y$ is continuous, bijective and the inverse $f^{-1}$ is also continuous, then $X$ and $Y$ are \textit{homeomorphic} with homeomorphism $f$. 

Let $X$ be a set. $S \subseteq 2^X$ \textit{generates} a topology $\tau(S)$ on $X$ such that $\tau(S)$ is the intersection of all topologies on $X$ where $S$ is open. 

The spaces we work over are \textit{locally homeomorphic} to $\mathbb{R}^n$, which has a topology generated by open balls $\{x \in \mathbb{R}^n: \| x - r \| < \varepsilon\}$ for $r \in \mathbb{R}^n$ and $\varepsilon > 0$.

Let $I = [0, 1]$.
A \textit{loop} is a continuous function $\gamma : I \rightarrow X$ where $\gamma(0) = \gamma(1) = x_0$. The point $x_0$ is the \textit{base point}. A \textit{homotopy} between two loops $\alpha, \beta$ is a continuous map $h : I \times I \rightarrow (x)$ where $h(0, t) = h(1, t) = x$ for all $t$, $h(\cdot, 0) = \alpha$, $h(\cdot, 1) = \beta$. A \textit{null-homotopic} loop is a loop homotopy to the constant map at $x_0$. Homotopic and null-homotopic loops come up in our discussion of graphs on surfaces as they can be used to classify edges embedded on a surface when the graph is a single point $x_0$. 

\subsection{Surfaces}

This section comes from Mohar and Thomassen's\cite{moharGraphsSurfaces2001} book on graphs on surfaces. A \textit{surface} \(\Sigma\) is a topological space which, at every point, has a neighbourhood homeomorphic to an open disk. There are four important surfaces to know- the sphere \(S^2\), the torus \(T\), the projective plane \(P\) and the Klein bottle $K$. 

\textit{Handles} are added to a surface \(\Sigma\) by removing two disks in \(\Sigma\) and identifying the boundaries such that one goes clockwise and the other goes counter-clockwise. \textit{Crosscaps} are added to a surface $\Sigma$ by removing a disk in \(\Sigma\) and identifying opposite points on the boundary. 
\par
\begin{definition}[Euler Genus]
	The \textit{Euler genus} of a surface \(\Sigma\), obtained from a sphere by adding \(h\) handles and \(c\) crosscaps, is \(2h + c\).
\end{definition}
A sphere with one handle and one crosscap is homeomorphic to a sphere with three crosscaps. 

\begin{example}
	This is the Euler genus of some surfaces.
	\begin{enumerate}
		\item The Euler genus of the sphere is \(0\).
		\item The Euler genus of the torus is \(2\).
		\item The Euler genus of the projective plane is \(1\). 
		\item The Euler genus of the Klein bottles is \(2\). 
	\end{enumerate}
\end{example}

Note that ``genus'' and ``Euler genus'' are two distinct concepts in topology. In this paper, when we discuss genus, we will always discuss \underline{Euler genus}.

A surface \(\Sigma\) is \textit{orientable} if \(\Sigma\) can be obtained from \(S^2\) by only adding handles. An example of an orientable surface is the torus.

A surface \(\Sigma\) is \textit{non-orientable} if \(\Sigma\) can be obtained from \(S^2\) by adding at least one crosscap or twisted handle. An example of a non-orientable surface is the projective plane or the Klein bottle. 

The \textit{Euler Genus} of a \textit{graph} \(G\) is the smallest Euler genus \(g\) surface \(\Sigma\) such that \(G\) can be embedded on \(\Sigma\) without crossings (note that \(\Sigma\) can be orientable or nonorientable). 


\subsection{Graphs on surfaces}

We say that $G$ is \textit{2-cell embedded} on a surface $\Sigma$ if every connected component of $\Sigma - G$ is homeomorphic to a 2-disk. 

\begin{theorem}[Euler's formula on surfaces]\label{thm:Euler_surfaces}
	Let \(|F(G)|\) be the number of faces in a graph \(G\). Then \(|V(G)| - |E(G)| + |F(G)| = \chi = 2 - g\). 
\end{theorem}
The value $\chi$ is known as the \textit{Euler characteristic} of a topological space, in this case a surface. Calculating the Euler characteristic of any space is done through \textit{homological algebra}. 

Graphs that can be embedded on the plane are called \textit{planar} graphs. Graphs that can be 2-cell embedded on the torus are called \textit{toroidal} graphs, and graphs that can be 2-cell embedded on the projective plane are called \textit{projective-planar} graphs.
If $G$ is embedded on a surface $\Sigma$ and every face in $G$ has three distinct vertices on its boundary, then $G$ is a \textit{triangulation} of $\Sigma$. 

Given graphs $G$ and $H$ with genus $g_1, g_2$, it is useful to construct a new graph with genus $g_1 + g_2$. 
\begin{theorem}[\textcite{millerAdditivityTheoremGenus1987}]\label{thm:additivity_genus}
	Let graphs $G$ and $H$ have genus $g_1$, $g_2$. Then the graph obtained from identifying a vertex in $G$ to a vertex in $H$ has genus $g_1 + g_2$. 
\end{theorem}

\begin{theorem}\label{thm:K5_Free_Planar}
	If \(G\) is a planar graph, then \(G\) is \(K_5\)-minor-free.
\end{theorem}
\begin{proof}
	If \(G\) is planar with \(n\) vertices and \(m\) edges where $n \geq 3$, then \(m \leq 3n -6\).
	However \(K_5\) has \(5\) vertices and \(10\) edges, but  \( 10 > 3 \times 5 - 6\), so \(K_5\) is not planar. As the family of planar graphs is minor-closed, if \(G\) is planar, then $G$ is \(K_5\)-minor free.
\end{proof}

\begin{theorem}[Bounded genus]\label{thm:bounded_genus_kt_free}
	If \(G\) is a genus \(g\) graph, then \(G\) is \(K_t\)-minor free, where \(t > \sqrt{6g} + 4\). 
\end{theorem}
\begin{proof}
	This proof mimics the above proof for planarity, but on surfaces of higher genus. 
	Suppose \(G\) has \(n\) vertices and \(m\) edges and of genus $g$. Then \(n - m + f = \chi = 2-g\), from \cref{thm:Euler_surfaces}. As at least three vertices bound each face and each edge touches exactly two faces, then \(f \leq 2m/3\). Therefore, \(m \leq 3(n + g - 2)\). If \(K_t\) is embeddable on a genus \(g\) graph, then \(\binom{t}{2} \leq 3 (t + g - 2)\). Thus \(t \leq \sqrt{6g} + 4\). So if a graph has genus \(g\), then it is \(K_t\)-minor-free, where \(t > \sqrt{6g} + 4\). 
\end{proof}

\subsection{Colouring graphs on surfaces}
The problem of colouring planar graphs has been studied extensively for the past hundred years. This has led to many generalisations, including Hadwiger's conjecture \cref{conj:Hadwiger's Conjecture}. One particular generalisation was by \textcite{heawoodMapcolourTheorem1890} who proved that:

\begin{theorem}
	Every graph embedded on a surface of genus $g \geq 1$ is
	\begin{equation*}
		\left\lfloor 
		\frac{7 + \sqrt{1 + 24g}}{2}
		\right\rfloor
	\end{equation*}
	-colourable. 
\end{theorem}

Equality in the case when the graph $G$ is a map was shown by \textcite{ringelMapColorTheorem1974}.

\todo{expand out!}