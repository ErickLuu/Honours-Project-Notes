
\section{Basic definitions}\label{sec: Basic definitions}
A graph $G$ is a pair of sets; a vertex set $V(G)$ and an edge set $E(G)$. $E(G)$ is a set that contains two-element subsets of $V(G)$. An edge $ \{v, w\}$ \textit{joins} vertices $v$ and $w$. A graph is \textit{simple} if all edges join two distinct vertices, with at most one edge between any two vertices. In this paper, all graphs are simple unless stated. Furthermore, all graphs $G$ are finite, so $|V(G)| < \infty$. The graph with all possible edges on $n$ vertices is the complete graph $K_n$. Graphs are defined up to isomorphism, or up to relabelling of the vertices. 

\subsection{Graphs and subgraphs}
Let $G$ be a graph. A \textit{subgraph} $H$ in $G$ is a graph with vertex set $V(H) \subseteq V(G)$ and edge set $E(H)$ with the property that if $vw$ is an edge in $E(H)$, then $vw$ is an edge in $E(G)$.
Let $G$ be a graph and let $S$ be a non-empty subset of the vertex set of $G$. The \textit{induced subgraph} of $S$ in $G$ is the graph $G[S]$ with vertex set $S$ and edge set containing precisely all edges in $G$ incident to two vertices in $S$. Removing a set of vertices $S \subseteq V(G)$ from $G$ forms the induced subgraph $G - S := G[V(G) - S]$. 
$H$ is a \textit{spanning subgraph} of $G$ if $H$ is a subgraph of $G$ and $V(H) = V(G)$. 
The neighbourhood of a set of vertices $A \subseteq V(G)$ are precisely all vertices that are adjacent to a vertex in $A$ and not in $A$ and is denoted as $N_G(A)$. Below are a list of useful subgraphs.

\begin{itemize}
	\item A \textit{path} in a graph \(G\) is a sequence of edges \(e_1, e_2, \ldots, e_{\ell- 1}\) which join a sequence of vertices \(v_1, v_2, \ldots, v_{\ell}\) such that \(e_i = v_i v_{i + 1}\), and all the vertices are distinct.
	\item A \textit{cycle} \(C\) in a graph \(G\) is a sequence of edges \(e_1, e_2, \ldots, e_{\ell}\) which join a sequence of distinct vertices \(v_1, v_2, \ldots, v_{\ell}\) such that \(e_i = v_i v_{i + 1}\) for \(1 \leq i \leq \ell - 1\) and \(e_\ell = v_\ell v_1\).
	\item Let $G$ be a graph and $C$ be a cycle in $G$. A \textit{chord} in $C$ is an edge $e$ that joins two vertices in $C$ that are not adjacent in $C$. 
	\item A \textit{Hamiltonian cycle} in a graph \(G\) is a cycle \(C\) such that all the vertices in \(G\) appear in \(C\). If a graph $G$ has a Hamiltonian cycle, then $G$ is a Hamiltonian graph. 
\end{itemize}

For any path $P$, the subpath from $u$ to $v$ in $P$ is $P[u, v]$. The subpath from $u$ to $v$, then removing $v$ is $P[u, v)$, and the internal subpath from $u$ to $v$ is $P(u, v)$. 

The disjoint union of two graphs $G_1$ and $G_2$ is the graph $G_1 \sqcup G_2$ with vertex set $V(G_1) \sqcup V(G_2)$ and edge set $V(G_1) \sqcup V(G_2)$. 

\subsection{Connected graphs}
A graph $G$ is \textit{connected} if between any two distinct vertices $x$ and $y$ there exists a path which starts and ends at $x$ and $y$ respectively. 
A connected graph \(G\) is \textit{\(k\)-connected} if \(G\) has more than \(k\) vertices and for any vertex set $S \subseteq V(G)$ where $|S| \leq k - 1$ it holds that $G - S$ is connected.\ \textit{Biconnected} graphs are $2$-connected graphs. 

Throughout this report, the set $\lbrace 1\ldots n \rbrace$ is notated as $[n]$. 
A graph \(G\) is \(k\)-colourable if there exists a function \(f: V(G) \rightarrow [k]\) such that if $f(v) = f(w)$, then $v$ and $w$ do not share an edge. The \textit{chromatic number} \(\chi(G)\) is the smallest \(k\) such that \(G\) is \(k\)-colourable.
$G$ is $k$-degenerate if every subgraph $H$ of $G$ has a vertex of degree at most $k$. Every $k$-degenerate graph is $(k + 1)$-colourable. 

Menger's theorem \cite{mengerZurAllgemeinenKurventheorie1927} is an important theorem which is used throughout the report.
Let \(G\) be a graph and \(A, B \subseteq V(G)\). An \(AB\)-path is a path in \(G\) which starts in \(A\) and ends in \(B\) with no internal vertices in \(A \cup B\). An \(AB\)-connector is a set of disjoint \(AB\)-paths. An \(AB\)-separator is a set \(S \subseteq V(G)\) such that \(G - S\) contains no \(AB\)-path. Then:
\begin{theorem}[Menger's theorem]\label{thm:Menger}
	Let $G$ be a graph and let $A, B \subseteq V(G)$. Then the size of the smallest \(AB\)-separator of \(G\) is equal to the size of the largest \(AB\)-connector.
\end{theorem}
Now take two distinct vertices \(x, y\). Let \(A = N_G(x) \cup \{x\} \) and \(B = N_G(y) \cup \{y\} \). Then \cref{thm:Menger} implies that:
\begin{theorem}[Menger's theorem, vertex-connectivity version]\label{thm:Menger_Vertex}
	A graph \(G\) is \(k\)-connected if and only if for any two distinct vertices $u,v \in V(G)$, there are at least \(k\) internally disjoint paths between $u$ and $v$.
\end{theorem}
As a corollary, all Hamiltonian graphs are biconnected. For any two distinct vertices in a Hamiltonian graph, there are two internally disjoint paths by traversing the Hamiltonian cycle.
\subsection{Block-cut diagram}
A \textit{cut-vertex} $v$ in a connected graph $G$ has the property that $G - v$ is disconnected.
A subgraph $H$ of a graph $G$ is a \textit{block} if:
\begin{itemize}
	\item $H$ is a maximal biconnected component of $G$,
	\item $H$ is a bridge edge of $G$ with its two endpoints or
	\item $H$ is an isolated vertex.
\end{itemize}

\begin{theorem}
	For all graphs $G$, blocks partition $E(G)$.
\end{theorem}

A \textit{block-cut tree} is a tree $T$ whose vertices are the blocks and cut-vertices of $G$, and $uv$ is an edge in $T$ whenever one block is adjacent to a cut-vertex. 

\begin{theorem}\label{thm:block-cut tree}
	Every graph has a block-cut tree.
\end{theorem}