
\section{Graph minors}\label{sec:Graph Minors}
A graph \(H\) is a \textit{minor} of a graph \(G\) if a graph isomorphic to \(H\) can be obtained from \(G\) by deleting vertices, deleting edges, and \textit{contracting} edges. Let $G$ be a graph and let $uv$ be an edge in $E(G)$. To \textit{contract} \(uv\), we delete both \(u\) and \(v\) and create a new vertex \(w\) with neighbourhood \(N(w) = N_G(u) \cup N_G(v)\). The graph obtained after contracting the edge \(uv\) in $G$ is written as \(G\setminus uv\).
The statement ``\(H\) is a minor of \(G\)'' is written as \(H \leq G\). A graph \(G\) is \textit{\(H\)-minor-free} if $H$ is not a minor of $G$. A family of graphs \(\mathcal{F}\) is \textit{minor-closed} if when $G$ is in \(\mathcal{F}\) and \(H \leq G\), then $H$ is in \(\mathcal{F}\).
An example of a minor-closed class is the class of planar graphs.
An important class of graph families are the \(K_t\)-minor free graphs. For a graph \(G\), we define \(\had(G)\) to be the largest \(t\) such that \(K_t\) is a minor of \(G\). This is named after Hugo Hadwiger and his most famous conjecture.

\begin{conjecture}[Hadwiger's conjecture]
	For all graphs \(G\), \(\chi(G) \leq \had(G)\)\cite{hadwigerUeberKlassifikationStreckenkomplexe1943}.
\end{conjecture}
Much work has been done on solving Hadwiger's conjecture, with a document by \textcite{seymourHadwigerConjecture2016} on the latest progress. However, it remains unsolved. 

A \textit{model} of a graph \(H\) in a graph \(G\) is a function $\rho$ which assigns to \(H\) vertex-disjoint connected subgraphs of \(G\). If $uv$ is an edge in \(E(H)\), then some edge in \(G\) joins the two subgraphs \(\rho(u)\) and \(\rho(v)\). A description of a model is in \cref{fig:model_of_P5}.
\begin{figure}[h!]\label{fig:model_of_P5}
	\centering
	\includesvg[width = 0.8\textwidth]{figures/model.svg}
	\caption{Illustration of a model $H$ in a graph $G$. The coloured boxes are the connected subgraphs contracted to a single vertex on the right.}
\end{figure}

\begin{theorem}
	\(H\) is a model of \(G\) if and only if $H$ is a minor of $G$.
\end{theorem}

\begin{proof}
	From \textcite{norinMath599GraphMinors2017}. Suppose \(H\) is a model of \(G\). Then for all \(x\) in \(V(H)\), contract \(\rho(x)\) in \(G\) to a single vertex. This is a well-defined operation as the image $\rho(x)$ is connected and disjoint from all $\rho(y)$ where $y$ is a distinct vertex in $H$. Then delete edges to form \(H\).

	Suppose \(H \leq G\). Use induction to show that \(H\) has a model in \(G\). Suppose \(H\) is obtained from \(G\) by contraction operations only. We can assume this by taking a subgraph of \(G\) if necessary. Let \(uv\) be the first contracted edge and let \(G' := G \setminus uv\). Let \(w\) be the vertex obtained after contracting \(uv\). Then by induction, there is a model \(\rho\) of \(H\) in \(G'\). Then find $x \in V(H)$ such that $w \in V(\rho(x))$. If there is no such $x$, then it is obvious that $\rho$ is a model of $H$ in $G$. Otherwise, 
	delete \(w\) from \(V(\rho(x)) \) and add $u, v$ to $V(\rho(x))$, the edge $uv$, and the edges from $u$ and $v$ to the neighbours in $w$ in $\rho(x)$. Then this is a model of \(H\) in \(G\). 
\end{proof}

 Much of structural graph theory involves graph minors in some way. Many of the theorems that we will discuss throughout this report discuss graph minors. 

\section{Book embedding}\label{sec:Book Embedding}
A \textit{book} with \(k\) \textit{pages} consists of \(k\) half-planes glued together on a common boundary. We refer to the boundary as the \textit{spine}, and the individual half-planes as \textit{pages}. In topology, these are referred to as \textit{fans} of half-planes.\ \textcite{persingerSubsetsNbooksE31966,atneosenOnedimensionalNleavedContinua1972} described fans in the 1960s.
A \textit{book-embedding} of a graph \(G\) is an embedding of \(G\) on a book. We place the vertices of \(G\) on the spine, and we place each edge on a single page such that no two edges cross.
The \textit{pagenumber} of a graph \(G\) is the minimum number of pages required to embed \(G\) on a book. This is also referred to as \textit{book-thickness}, or \textit{stack-number}. An embedding of $K_5$ in three pages is in \cref{fig:book-embedding}.
\begin{figure}[h!]\label{fig:book-embedding}
	\centering
	\includesvg[height = 0.5\textheight]{figures/3page_K5.svg}
	\caption{Book-embedding of $K_5$ on three pages. Image by \textcite{eppsteinBookEmbedding2014}}
\end{figure}
\par
There is an equivalent combinatorial definition. A \textit{book embedding} of a graph \(G\) is an arrangement of the vertices of \(G\) in a total ordering \(v_1 < v_2 < \cdots < v_n\). We then \textit{colour} the edges \(E(G)\) such that if there are vertices with ordering \(v_a < v_b < v_c < v_d\) and edges \(v_a v_c\) and \(v_b v_d\) in $E(G)$, then $v_a v_c$ and $v_b v_d$ are assigned different colours.
We refer to the total ordering of \(V(G)\) in the book embedding as \((<)\) or as \((\leq)\). For a book-embedding \((v_1, v_2, \ldots, v_{|G|})\), we refer to the edges \( \left\{ v_1 v_2, v_2 v_3, \ldots, v_{|G| - 1}v_{|G|}, v_{|G|}v_{1} \right\} \) as \textit{spine edges}.
We may use a \textit{circular ordering} of the vertices rather than a linear ordering. This means that we order the vertices in a circle rather than on a straight line. The book-embedding of a circular ordering is exactly the same as for a linear ordering, and we can convert between a circular and linear ordering by choosing a vertex to be at the start of the sequence.
Book-embeddings were introduced by \textcite{kainenRecentResultsTopological1974, ollmannBookThicknessVarious1973} in the 1970s. A paper by \textcite{bernhartBookThicknessGraph1979} calculated the book-thickness of complete and bipartite graphs.
\begin{lemma}\label{lem:Pagenumber_1}
	A graph \(G\) can be embedded on a single page if and only if \(G\) is an outerplanar graph.
\end{lemma}
\begin{proof}
	Suppose $G$ is outerplanar, and embedded in $\mathbb{R}^2$. Then we choose a single vertex $v_0$, and traverse anticlockwise around the outerface to form an ordering. If a vertex $v_i$ appears more than once, then add $v_i$ the first time we see it in the traversal and no other times. Then this is a one-page book embedding, as when $v_a < v_b < v_c < v_d$ and edges $v_a v_c$, $v_b v_d$ in $G$, then $v_a v_c$ or $v_b v_d$ have to lie on the outerface, which breaks the condition that $G$ is outerplanar. This is because either $v_b$ or $v_c$ is not on the outerface. If $G$ has a $1$-page book-embedding, then embedding the page in $\mathbb{R}^2$ through the inclusion map is an outerplanar embedding of $G$. 
\end{proof}
\begin{lemma}\label{lem:Pagenumber_2}
	A graph \(G\) can be embedded on two pages if and only if \(G\) is a subgraph of a planar graph with a Hamiltonian cycle.
\end{lemma}

\begin{proof}
	Suppose $G$ is a subgraph of a planar graph $G'$ with a Hamiltonian cycle $C$. By the Jordan curve theorem, $\mathbb{R}^2 - C$ has two connected components $F_1$ and $F_2$. Choose a vertex $x_0$ and order the vertices with respect to the Hamiltonian cycle $C$ where $x_0$ is the first vertex. Give edges on $C$ colour $1$. For all edges which are a chord of $C$ that lies in $F_1$, give these edges colour $1$. For all edges which are a chord of $C$ that lies in $F_2$, give these edges colour $2$. This is a book-embedding of $G'$ on two pages. 
	\par
	Suppose $G$ has pagenumber $2$, and embedded in a book with two pages. Add all remaining spine edges to one of the pages. Then embed the two pages in $\mathbb{R}^2$ through the homeomorphism of two pages to $\mathbb{R}^2$, by flipping one page over. Then this is a planar graph with a Hamiltonian cycle, so $G$ is a subgraph of a graph with a Hamiltonian cycle.
\end{proof}
\subsection{Properties of pagenumber}\label{ssec:Related_Properties}
\cref{lem:Edge_Bound} comes from \textcite{bernhartBookThicknessGraph1979}.
\begin{lemma}\label{lem:Edge_Bound}
	If an \(n\)-vertex graph \(G\) can be embedded on $k$ pages, then \(G\) has at most \(n + k(n-3)\) edges.
\end{lemma}
\begin{proof}
	Given a vertex ordering \(v_1 \leq v_2 \leq \cdots \leq v_n\), the spine edges can appear on any page. Furthermore, there are at most \(n-3\) non-spine edges on each page. The maximum number of edges in an outerplanar graph is \(2n - 3\) from \cref{thm:outerplanar_bound}, but we remove the spine edges, with \(n\) edges on the spine. Therefore, \(m\), the number of edges, satisfies \(m \leq n + k (n - 3)\).
\end{proof}
\begin{theorem}[]\label{thm:Pagenumber_Complete_Graph}
	The complete graph $K_n$ has $\pn(K_n) = \lceil \frac{n}{2} \rceil$ when $n \geq 4$.
\end{theorem}
\begin{proof}
	To show the upper bound, arrange the vertices in any circular ordering $v_1 < v_2 < \cdots < v_n$. Then colour edges $v_1 v_2, v_2 v_{n}, v_{n} v_{3}, v_{3} v_{n-1}, \ldots$ in a zigzag pattern. As an example, refer to \cref{fig:k8 coloured with colours} for a description of a zig-zagging pattern. We rotate this pattern $\lceil n/2 \rceil$ times. 
	\par
	To show the lower bound, we use \cref{lem:Edge_Bound}. \(K_n\) has \(n\) vertices and \(\binom{n}{2}\) edges. Then \(\pn(K_n) \geq \frac{\binom{n}{2} - n}{n - 3} = \frac{n}{2}\) when \(n \geq 4\). As \(\pn(K_n)\) is an integer, we take the ceiling of \(\frac{n}{2}\). This concludes the equality.
\end{proof}
\begin{figure}[ht]
	\caption{Circular embedding of \(K_8\) with 4 colours, the minimum possible.}
	\centering
	\usetikzlibrary{graphs,graphs.standard}

\tikz
	\graph[nodes={circle, draw}] { 
		subgraph K_n [n=8,clockwise,radius=2cm];
		
		{[induced path, edges= red] 1,2,8,3,7,4,6,5},
		{[induced path, edges= blue] 8,1,7,2,6,3,5,4},
		{[induced path, edges= green] 7,8,6,1,5,2,4,3},
		{[induced path, edges= yellow] 6,7,5,8,4,1,3,2},
 };\label{fig:k8 coloured with colours}
\end{figure}
The proof of \cref{thm:Pagenumber_Complete_Graph} is from \textcite{bernhartBookThicknessGraph1979}
This is an upper bound of any graph \(G\) with \(n\) vertices.
Therefore for any graph \(G\) on \(n\) vertices, \(n \geq 4\), \(\pn(G) \leq \lceil n/2 \rceil\). The next theorem bounds the chromatic number, from \textcite{bernhartBookThicknessGraph1979}
\begin{theorem}\label{thm:Colouring_Bound}
	For all graphs \(G\), \(\chi(G) \leq 2 \pn(G) + 2\).
\end{theorem}
\begin{proof}
	Let \(\pn(G) = k\) and suppose \(G\) has \(n\) vertices and \(m\) edges. Then the average degree of \(G\), \(d(G) = 2m/n\) by the handshaking lemma. So \(2\frac{m}{n} \leq 2 \frac{n + k(n-3)}{n} = 2 + 2k \frac{n-3}{n} < 2k + 2\). But this implies that \(G\) has a vertex of degree \(\leq 2k + 1\), and as if \(G'\) is a subgraph of \(G\), then \(G'\) also has \(\pn(G') \leq k\), thus \(G'\) has a vertex of degree at most \(2k + 1\). However, this implies \(G\) is \((2k + 1)\)-degenerate, thus \(\chi(G) \leq 2k + 2\).
\end{proof}

Let $G$ be a graph. A \textit{subdivision} of an edge $uv \in E(G)$ deletes $uv$ and adds a new vertex $w$ with edges $uw$ and $wv$. A graph subdivision of $G$ is to do this for all edges in $G$. A $k$-subdivision of $G$ is to subdivide each edge $k$ times in $G$, so the edge $e$ is replaced with a path $P$ of length $k$.\ \textcite{atneosenEmbeddabilityCompactaNbooks} proved that all graphs can be subdivided a finite number of times such the subdivision has pagenumber 3.\ \textcite{dujmovicLayoutsGraphSubdivisions2005} showed that the number of subdivision necessary is $O(\log\pn(G))$.

\begin{theorem}
	There exists a family of 2-colourable graphs with arbitrarily pagenumber.
\end{theorem}
\begin{proof}
	Let $G_n$ be the complete graph $K_n$ with every edge subdivided once. Then $G_n$ is bipartite, so is $2$-colourable. However, from \textcite{eppsteinSeparatingThicknessGeometric2002}, for every $t$ there exists an $n$ such that $G_n$ cannot be embedded in $t-1$ pages. This proof comes from \textit{Ramsey theory}, and explicit values of $n$ are difficult to find. 
\end{proof}

An \textit{expander graph} is a sparse, highly connected graph. Expander graphs share many properties with random graphs, but are constructed explicitly. One type of expander graph is a \textit{bipartite \varepsilon-expander}, where $\varepsilon \in (0, 1]$. We say a graph $G$ is a bipartite \varepsilon-expander if there exists a bipartition $ \{A, B\}$ of $V(G)$ such that $|A| = |B|$ and for all subsets $S \subset A$ where $|S| \leq \frac{|A|}{2}$, $|N(S)| \geq (1 + \varepsilon) |S|$. 
\textcite{dujmovicLayoutsExpanderGraphs2016} showed that all bipartite \varepsilon-expander graphs can be embedded in 3 pages. 


Book-embeddings of graphs were has applications in VLSI and processor designs, bioinformatics by \textcite{haslingerRNAStructuresPseudoknots1999}, and in graph drawings by \textcite{woodBoundedDegreeBook2002}. 
The project of finding upper and lower bounds of the pagenumber of planar graphs was started by \textcite{bernhartBookThicknessGraph1979} when they conjectured that planar graphs had unbounded pagenumber. However, \textcite{bussPagenumberPlanarGraphs1984} showed that all graphs could be embedded in nine pages, and \textcite{heathEmbeddingPlanarGraphs1984} brought down the number of needed pages to seven.\ \textcite{yannakakisEmbeddingPlanarGraphs1989} devised an algorithm to embed a graph in four pages. Yannakakis, in this paper, claimed that there exists planar graphs which cannot be embedded in three pages. However, his proof was incomplete and the full proof was left unpublished. In 2020, Yannanakis published his full proof \cite{yannakakisPlanarGraphsThat2020}. At around the same time, \textcite{kaufmannFourPagesAre2020} published the same lower bound.

\textcite{malitzGraphsEdgesHave1994} proved that any graph with $e$ edges has pagenumber $O(\sqrt{e})$. Additionally, he proved that random $d$-regular graphs $G$ with $n$ vertices have the property that $\pn(G) \in \Omega(\sqrt{d} n^{1/2 - 1/d})$. For random 3-regular graphs $G$ with $n$ vertices, $\pn(G) \in \Omega(n^{1/6})$. These constructions of $\Omega(n^d)$ pagenumber graphs are not explicit.\ \textcite{eppsteinThreeDimensionalGraphProducts2024} showed that $\pn(P_n \boxtimes P_n \boxtimes P_n) \in \Theta(n^{1/3})$. This is an explicit construction of a graph which has pagenumber in $\Theta(n^{d})$. 
\section{Treewidth}\label{sec:treewidth}
The \textit{treewidth} of a graph \(G\) measures how similar $G$ is to a forest.