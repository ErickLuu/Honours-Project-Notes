\chapter{Definitions}\label{chap:Definitions}
This chapter presents important notions in structural graph theory.
\begin{itemize}
	\item \cref{sec: Basic definitions} are basic graph theory definitions and notations.
	\item \cref{sec:Planar graphs} discusses important information involving planar graphs.
	\item \cref{sec:Graph Minors} discusses graph minors.
	\item \cref{sec:Book Embedding} defines book-embeddings and discusses its importance.
	\item \cref{sec:treewidth} defines \textit{treewidth}.
	\item \cref{sec:Pathwidth} defines a related notion \textit{pathwidth}.
\end{itemize}
These definitions will be useful in discussing the Graph Minor Structure Theorem and our conjecture.
\section{Basic definitions}\label{sec: Basic definitions}
A graph $G$ is a pair of sets; a vertex set $V(G)$ and an edge set $E(G)$. $E(G)$ is a set that contains two-element subsets of $V(G)$. An edge $ \{v, w\}$ \textit{joins} vertices $v$ and $w$. In this paper, all graphs are simple unless stated. A graph is \textit{simple} if all edges join two distinct vertices and there is at most one edge between any two vertices. Furthermore, all graphs $G$ are finite, so $|V(G)| < \infty$. The graph with all possible edges on $n$ vertices is the complete graph $K_n$.

Let $G$ be a graph. A \textit{subgraph} $H$ in $G$ is a graph with vertex set $V(H) \subseteq V(G)$ and edge set $E(H)$ with the property that if $vw$ is an edge in $E(H)$, then $vw$ is an edge in $E(G)$.
Let $G$ be a graph and let $S$ be a non-empty subset of the vertex set of $G$. The \textit{induced subgraph} of $S$ in $G$ is the graph $G[S]$ with vertex set $S$ and edge set containing precisely all edges in $G$ incident to two vertices in $S$. Removing a set of vertices $S \subseteq V(G)$ from $G$ forms the induced subgraph $G - S := G[V(G) - S]$. 
$H$ is a \textit{spanning subgraph} of $G$ if $H$ is a subgraph of $G$ and $V(H) = V(G)$. 
The neighborhood of a set of vertices $A \subseteq V(G)$ are precisely all vertices that are adjacent to a vertex in $A$ and not in $A$ and is denoted as $N_G(A)$. 

\begin{itemize}
	\item A \textit{path} in a graph \(G\) is a sequence of edges \(e_1, e_2, \ldots, e_{\ell- 1}\) which join a sequence of vertices \(v_1, v_2, \ldots, v_{\ell}\) such that \(e_i = v_i v_{i + 1}\), and all the vertices are distinct.
	\item A \textit{cycle} \(C\) in a graph \(G\) is a sequence of edges \(e_1, e_2, \ldots, e_{\ell}\) which join a sequence of distinct vertices \(v_1, v_2, \ldots, v_{\ell}\) such that \(e_i = v_i v_{i + 1}\) for \(1 \leq i \leq \ell - 1\) and \(e_\ell = v_\ell v_1\).
	\item Let $G$ be a graph and $C$ be a cycle in $G$. A \textit{chord} in $C$ is an edge $e$ that joins two vertices in $C$ that are not adjacent in $C$. 
	\item A \textit{Hamiltonian cycle} in a graph \(G\) is a cycle \(C\) such that all the vertices in \(G\) appear in \(C\). If a graph $G$ has a Hamiltonian cycle, then $G$ is a Hamiltonian graph. 
\end{itemize}

A graph $G$ is \textit{connected} if between any two distinct vertices $x$ and $y$ there exists a path which starts and ends at $x$ and $y$ respectively. 
A connected graph \(G\) is \textit{\(k\)-connected} if \(G\) has more than \(k\) vertices and for any vertex set $S \subseteq V(G)$ where $|S| \leq k - 1$ it holds that $G - S$ is connected.\ \textit{Biconnected} graphs are $2$-connected graphs. 

Throughout this report, the set $\lbrace 1\ldots n \rbrace$ is notated as $[n]$. 
A graph \(G\) is \(k\)-colourable if there exists a function \(f: V(G) \rightarrow [k]\) such that if $f(v) = f(w)$, then $v$ and $w$ do not share an edge. The \textit{chromatic number} \(\chi(G)\) is the smallest \(k\) such that \(G\) is \(k\)-colourable.
$G$ is $k$-degenerate if every subgraph $H$ of $G$ has a vertex of degree at most $k$. Every $k$-degenerate graph is $(k + 1)$-colourable. 

Menger's theorem \cite{mengerZurAllgemeinenKurventheorie1927} is an important theorem which is used throughout the report.
Let \(G\) be a graph and \(A, B \subseteq V(G)\). An \(AB\)-path is a path in \(G\) which starts in \(A\) and ends in \(B\) with no internal vertices in \(A \cup B\). An \(AB\)-connector is a set of disjoint \(AB\)-paths. An \(AB\)-separator is a set \(S \subseteq V(G)\) such that \(G - S\) contains no \(AB\)-path. Then:
\begin{theorem}[Menger's theorem]\label{thm:Menger}
	Let $G$ be a graph and let $A, B \subseteq V(G)$. Then the size of the smallest \(AB\)-separator of \(G\) is equal to the size of the largest \(AB\)-connector.
\end{theorem}
Now take two distinct vertices \(x, y\). Let \(A = N_G(x) \cup \{x\} \) and \(B = N_G(y) \cup \{y\} \). Then \cref{thm:Menger} implies that:
\begin{theorem}[Menger's theorem, vertex-connectivity version]\label{thm:Menger_Vertex}
	A graph \(G\) is \(k\)-connected if and only if for any two distinct vertices, there are at least \(k\) internally disjoint paths between the two vertices.
\end{theorem}
As a corollary, all Hamiltonian graphs are biconnected. For any two distinct vertices in a Hamiltonian graph, there are two internally disjoint paths by traversing the Hamiltonian cycle.

\section{Planar graphs}\label{sec:Planar graphs}
A graph \(G\) is \textit{planar} if \(G\) can be drawn in the Euclidean plane \( \mathbb{R}^2 \) such that no two edges cross. A drawing of $G$ in $\mathbb{R}^2$ is referred to as an \textit{embedding} of $G$ in $\mathbb{R}^2$. If \(G\) is embedded in \(\mathbb{R}^2 \), then we can topologically remove the set $G$ from $\mathbb{R}^2$. The connected components in $\mathbb{R}^2 - G$ are called \textit{faces}. The \textit{outerface} is the unbounded face in $\mathbb{R}^2$, meaning that the outerface goes to infinity. The \textit{internal faces} of $G$ are all the faces which are not the outerface. A set of vertices $S$ in $V(G)$ \textit{lie} on a face $F$, or \textit{bound} $F$ if $S$ is in the closure of $F$. A set of edges $T$ in $V(G)$ also bound a face $F$ if $T$ is in the closure of $F$. If an edge $e$ bounds a face $F$, then $e$ \textit{touches} $F$. \(G\) is \textit{outerplanar} if \(G\) is planar and there exists an embedding such that all vertices in \(G\) lie on the outerface.
Let \(F(G)\) be the set of faces of \(G\) embedded on \(\mathbb{R}^2\). Then
\begin{theorem}[Euler's formula]\label{thm:Euler_planar}
	For all connected graphs $G$,
	\begin{equation}
		|V(G)| - |E(G)| + |F(G)| = 2.
	\end{equation}
\end{theorem}

We can use this result to bound the number of edges in an outerplanar graph.
\begin{theorem}\label{thm:outerplanar_bound}
	If \(G\) is an outerplanar graph with \(n\) vertices and \(m\) edges, then \(m \leq 2n - 3\).
\end{theorem}

\begin{proof}
	Suppose \(G\) is \textit{maximal outerplanar}, meaning adding any edge will break the outerplanar property. Let there be \(f\) internal faces in $G$. Then the outerface has \(n\) edges on the boundary as every vertex in $G$ is on the boundary of the outerface. Each internal face will have exactly \(3\) edges on the boundary. However, each edge is touching exactly two distinct faces. By counting the number of edges on every face,
	\begin{equation*}
		3 f + n = 2m.
	\end{equation*}
	Combining with \cref{thm:Euler_planar},
	\begin{equation*}
		n - m + (f + 1) = 2
	\end{equation*}
	we have, after some rearrangement,
	\begin{equation*}
		2n = 3 + m.
	\end{equation*}
	Therefore, \(m = 2n - 3\). Since every outerplanar graph is a spanning subgraph of a maximal planar graph, \(m \leq 2n - 3\).
\end{proof}

\begin{theorem}
	Let $G$ be a planar graph with $n$ vertices and $m$ edges. Then $m \leq 3n - 6$.
\end{theorem}
\begin{proof}
	Suppose $G$ is embedded on $\mathbb{R}^2$. Then $G$ has $f$ faces, where every edge touches exactly two faces. Every face touches at least three edges. Therefore, $3f \leq 2m$. From Euler's formula, $n - m + f = 2$, so $m \leq 3n - 6$. 
\end{proof}
\section{Graph minors}\label{sec:Graph Minors}
A graph \(H\) is a \textit{minor} of a graph \(G\) if a graph isomorphic to \(H\) can be obtained from \(G\) by deleting vertices, deleting edges, and \textit{contracting} edges. Let $G$ be a graph and let $uv$ be an edge in $E(G)$. To \textit{contract} \(uv\), we delete both \(u\) and \(v\) and create a new vertex \(w\) with neighbourhood \(N(w) = N_G(u) \cup N_G(v)\). The graph obtained after contracting the edge \(uv\) in $G$ is written as \(G\setminus uv\).
The statement ``\(H\) is a minor of \(G\)'' is written as \(H \leq G\). A graph \(G\) is \textit{\(H\)-minor-free} if $H$ is not a minor of $G$. A family of graphs \(\mathcal{F}\) is \textit{minor-closed} if when $G$ is in \(\mathcal{F}\) and \(H \leq G\), then $H$ is in \(\mathcal{F}\).
An example of a minor-closed class is the class of planar graphs.
An important class of graph families are the \(K_t\)-minor free graphs. For a graph \(G\), we define \(\had(G)\) to be the largest \(t\) such that \(K_t\) is a minor of \(G\). This is named after Hugo Hadwiger and his most famous conjecture.

\begin{conjecture}[Hadwiger's conjecture]
	For all graphs \(G\), \(\chi(G) \leq \had(G)\)\cite{hadwigerUeberKlassifikationStreckenkomplexe1943}.
\end{conjecture}
Much work has been done on solving Hadwiger's conjecture, with a document by \textcite{seymourHadwigerConjecture2016} on the latest progress. However, it remains unsolved. 

A \textit{model} of a graph \(H\) in a graph \(G\) is a function $\rho$ which assigns to \(H\) vertex-disjoint connected subgraphs of \(G\). If $uv$ is an edge in \(E(H)\), then some edge in \(G\) joins the two subgraphs \(\rho(u)\) and \(\rho(v)\). 

\begin{theorem}
	\(H\) is a model of \(G\) if and only if $H$ is a minor of $G$.
\end{theorem}

\begin{proof}
	From \textcite{norinMath599GraphMinors2017}. Suppose \(H\) is a model of \(G\). Then for all \(x\) in \(V(H)\), contract \(\rho(x)\) in \(G\) to a single vertex. This is a well-defined operation as the image $\rho(x)$ is connected and disjoint from all $\rho(y)$ where $y$ is a distinct vertex in $H$. Then delete edges to form \(H\).

	Suppose \(H \leq G\). Use induction to show that \(H\) has a model in \(G\). Suppose \(H\) is obtained from \(G\) by contraction operations only. We can assume this by taking a subgraph of \(G\) if necessary. Let \(uv\) be the first contracted edge and let \(G' := G \setminus uv\). Let \(w\) be the vertex obtained after contracting \(uv\). Then by induction, there is a model \(\rho\) of \(H\) in \(G'\). Then find $x \in V(H)$ such that $w \in V(\rho(x))$. If there is no such $x$, then it is obvious that $\rho$ is a model of $H$ in $G$. Otherwise, 
	delete \(w\) from \(V(\rho(x)) \) and add $u, v$ to $V(\rho(x))$, the edge $uv$, and the edges from $u$ and $v$ to the neighbours in $w$ in $\rho(x)$. Then this is a model of \(H\) in \(G\).
\end{proof}
\todo{Add pictures}

\section{Book embedding}\label{sec:Book Embedding}
A \textit{book} with \(k\) \textit{pages} consists of \(k\) half-planes glued together on a common boundary. We refer to the boundary as the \textit{spine}, and we refer to the individual half-planes as \textit{pages}. In topology, these are referred to as \textit{fans} of half-planes.\ \textcite{persingerSubsetsNbooksE31966,atneosenOnedimensionalNleavedContinua1972} described fans in the 1960s.
A \textit{book-embedding} of a graph \(G\) is an embedding of \(G\) on a book. We place the vertices of \(G\) on the spine, and we place each edge on a single page such that no two edges cross.
The \textit{pagenumber} of a graph \(G\) is the minimum number of pages required to embed \(G\) on a book. This is also referred to as \textit{book-thickness}, or \textit{stack-number}. An embedding of $K_5$ in three pages is in \cref{fig:book-embedding}.
\begin{figure}[h!]\label{fig:book-embedding}
	\centering
	\includesvg[height = 0.5\textheight]{figures/3page_K5.svg}
	\caption{Book-embedding of $K_5$ on three pages. Image by \textcite{eppsteinBookEmbedding2014}}.
\end{figure}
\par
There is an equivalent combinatorial definition. A \textit{book embedding} of a graph \(G\) is an arrangement of the vertices of \(G\) in a total ordering \(v_1 < v_2 < \cdots < v_n\). We then \textit{colour} the edges \(E(G)\) such that if there are vertices with ordering \(v_a < v_b < v_c < v_d\) and edges \(v_a v_c\) and \(v_b v_d\) in $E(G)$, then $v_a v_c$ and $v_b v_d$ are assigned different colours.
We refer to the total ordering of \(V(G)\) in the book embedding as \((<)\) or as \((\leq)\). For a book-embedding \((v_1, v_2, \ldots, v_{|G|})\), we refer to the edges \( \left\{ v_1 v_2, v_2 v_3, \ldots, v_{|G| - 1}v_{|G|}, v_{|G|}v_{1} \right\} \) as \textit{spine edges}.
We may use a \textit{circular ordering} of the vertices rather than a linear ordering. This means that we order the vertices in a circle rather than on a straight line. The book-embedding of a circular ordering is exactly the same as for a linear ordering, and we can convert between a circular and linear ordering by choosing a vertex to be at the start of the sequence.
Book-embeddings were introduced by \textcite{kainenRecentResultsTopological1974, ollmannBookThicknessVarious1973} in the 1970s. A paper by \textcite{bernhartBookThicknessGraph1979} calculated the book-thickness of complete and bipartite graphs.
\begin{lemma}\label{lem:Pagenumber_1}
	A graph \(G\) can be embedded on a single page if and only if \(G\) is an outerplanar graph.
\end{lemma}
\begin{proof}
	Suppose $G$ is outerplanar, and embedded in $\mathbb{R}^2$. Then we choose a single vertex $v_0$, and traverse anticlockwise around the outerface to form an ordering. If a vertex $v_i$ appears more than once, then add $v_i$ the first time we see it in the traversal and no other times. Then this is a one-page book embedding, as when $v_a < v_b < v_c < v_d$ and edges $v_a v_c$, $v_b v_d$ in $G$, then $v_a v_c$ or $v_b v_d$ have to lie on the outerface, which breaks the condition that $G$ is outerplanar. This is because either $v_b$ or $v_c$ is not on the outerface. If $G$ has a $1$-page book-embedding, then embedding the page in $\mathbb{R}^2$ through the inclusion map is an outerplanar embedding of $G$. 
\end{proof}
\begin{lemma}\label{lem:Pagenumber_2}
	A graph \(G\) can be embedded on two pages if and only if \(G\) is a subgraph of a planar graph with a Hamiltonian cycle.
\end{lemma}

\begin{proof}
	Suppose $G$ is a subgraph of a planar graph $G'$ with a Hamiltonian cycle $C$. By the Jordan curve theorem, $\mathbb{R}^2 - C$ has two connected components $F_1$ and $F_2$. Choose a vertex $x_0$ and order the vertices with respect to the Hamiltonian cycle $C$ where $x_0$ is the first vertex. Give edges on $C$ colour $1$. For all edges which are a chord of $C$ that lies in $F_1$, give these edges colour $1$. For all edges which are a chord of $C$ that lies in $F_2$, give these edges colour $2$. This is a book-embedding of $G'$ on two pages. 
	\par
	Suppose $G$ has pagenumber $2$, and embedded in a book with two pages. Add all remaining spine edges to one of the pages. Then embed the two pages in $\mathbb{R}^2$ through the homeomorphism of two pages to $\mathbb{R}^2$, by flipping one page over. Then this is a planar graph with a Hamiltonian cycle, so $G$ is a subgraph of a graph with a Hamiltonian cycle.
\end{proof}
\subsection{Properties of pagenumber}\label{ssec:Related_Properties}
\cref{lem:Edge_Bound} comes from \textcite{bernhartBookThicknessGraph1979}.
\begin{lemma}\label{lem:Edge_Bound}
	If an \(n\)-vertex graph \(G\) can be embedded on $k$ pages, then \(G\) has at most \(n + k(n-3)\) edges.
\end{lemma}
\begin{proof}
	Given a vertex ordering \(v_1 \leq v_2 \leq \cdots \leq v_n\), the spine edges can appear on any page. Furthermore we have there are at most \(n-3\) non-spine edges on each page. The maximum number of edges in an outerplanar graph is \(2n - 3\) from \cref{thm:outerplanar_bound}, but we remove the spine edges, with \(n\) edges on the spine. Therefore, \(m\), the number of edges, satisfies \(m \leq n + k (n - 3)\).
\end{proof}
\begin{theorem}[]\label{thm:Pagenumber_Complete_Graph}
	The complete graph $K_n$ has $\pn(K_n) = \lceil \frac{n}{2} \rceil$ when $n \geq 4$.
\end{theorem}
\begin{proof}
	To show the upper bound, arrange the vertices in any circular ordering $v_1 < v_2 < \cdots < v_n$. Then we colour edges $v_1 v_2, v_2 v_{n}, v_{n} v_{3}, v_{3} v_{n-1}, \ldots$ in a zigzag pattern. As an example, we refer to \cref{fig:k8 coloured with colours} to show what zig-zagging pattern we are referring to. We rotate this pattern $\lceil n/2 \rceil$ times. 
	\par
	To show the lower bound, we use \cref{lem:Edge_Bound}. \(K_n\) has \(n\) vertices and \(\binom{n}{2}\) edges. Then \(\pn(K_n) \geq \frac{\binom{n}{2} - n}{n - 3} = \frac{n}{2}\) when \(n \geq 4\). As \(\pn(K_n)\) is an integer, we take the ceiling of \(\frac{n}{2}\). This concludes the equality.
\end{proof}
\begin{figure}[ht]
	\caption{Circular embedding of \(K_8\) with 4 colours, the minimum possible.}
	\centering
	\usetikzlibrary{graphs,graphs.standard}

\tikz
	\graph[nodes={circle, draw}] { 
		subgraph K_n [n=8,clockwise,radius=2cm];
		
		{[induced path, edges= red] 1,2,8,3,7,4,6,5},
		{[induced path, edges= blue] 8,1,7,2,6,3,5,4},
		{[induced path, edges= green] 7,8,6,1,5,2,4,3},
		{[induced path, edges= yellow] 6,7,5,8,4,1,3,2},
 };\label{fig:k8 coloured with colours}
\end{figure}
The proof of \cref{thm:Pagenumber_Complete_Graph} is from \textcite{bernhartBookThicknessGraph1979}
This is an upper bound of any graph \(G\) with \(n\) vertices.
Therefore for any graph \(G\) on \(n\) vertices, \(n \geq 4\), \(\pn(G) \leq \lceil n/2 \rceil\). The next theorem bounds the chromatic number, from \textcite{bernhartBookThicknessGraph1979}
\begin{theorem}\label{thm:Colouring_Bound}
	For all graphs \(G\), \(\chi(G) \leq 2 \pn(G) + 2\).
\end{theorem}
\begin{proof}
	Let \(\pn(G) = k\) and suppose \(G\) has \(n\) vertices and \(m\) edges. Then the average degree of \(G\), \(d(G) = 2m/n\) by the handshaking lemma. So \(2\frac{m}{n} \leq 2 \frac{n + k(n-3)}{n} = 2 + 2k \frac{n-3}{n} < 2k + 2\). But this implies that \(G\) has a vertex of degree \(\leq 2k + 1\), and as if \(G'\) is a subgraph of \(G\), then \(G'\) also has \(\pn(G') \leq k\), thus \(G'\) has a vertex of degree at most \(2k + 1\). However, this implies \(G\) is \((2k + 1)\)-degenerate, thus \(\chi(G) \leq 2k + 2\).
\end{proof}

There exists graphs with bounded chromatic number but bunbounded pagenumber. 
\todo{ find graphs $G$ such that $\chi(G) \geq \pn(G)$}. 

Let $G$ be a graph. A \textit{subdivision} of an edge $uv \in E(G)$ deletes $uv$ and adds a new vertex $w$ with edges $uw$ and $wv$. A graph subdivision of $G$ is to do this for all edges in $G$. A $k$-subdivision of $G$ is to subdivide each edge $k$ times in $G$, so the edge $e$ is replaced with a path $P$ of length $k$.\ \textcite{atneosenEmbeddabilityCompactaNbooks} proved that all graphs can be subdivided a finite number of times such the subdivision has pagenumber 3.\ \textcite{dujmovicLayoutsGraphSubdivisions2005} showed that the number of subdivision necessary is $O(\log\pn(G))$.

An \textit{expander graph} is a sparse, highly connected graph. Expander graphs share many properties with random graphs, but are constructed explicitly. One type of expander graph is a \textit{bipartite \varepsilon-expander}, where $\varepsilon \in (0, 1]$. We say a graph $G$ is a bipartite \varepsilon-expander if there exists a bipartition $ \{A, B\}$ of $V(G)$ such that $|A| = |B|$ and for all subsets $S \subset A$ where $|S| \leq \frac{|A|}{2}$, $|N(S)| \geq (1 + \varepsilon) |S|$. 
\textcite{dujmovicLayoutsExpanderGraphs2016} showed that all bipartite \varepsilon-expander graphs can be embedded in 3 pages. 


Book-embeddings of graphs were developed for VLSI and processor designs, but has been used in bioinformatics by \textcite{haslingerRNAStructuresPseudoknots1999}.\todo{find applications in graph drawings}
The project of finding upper and lower bounds of the pagenumber of planar graphs was started by \textcite{bernhartBookThicknessGraph1979} when they conjectured that planar graphs had unbounded pagenumber. However, \textcite{bussPagenumberPlanarGraphs1984} showed that all graphs could be embedded in nine pages, and \textcite{heathEmbeddingPlanarGraphs1984} brought down the number of needed pages to seven.\ \textcite{yannakakisEmbeddingPlanarGraphs1989} devised an algorithm to embed a graph in four pages. Yannakakis, in this paper, claimed that there exists planar graphs which cannot be embedded in three pages. However, his proof was incomplete and the full proof was left unpublished. In 2020, Yannanakis published his full proof \cite{yannakakisPlanarGraphsThat2020}. At around the same time, \textcite{kaufmannFourPagesAre2020} published the same lower bound.

\textcite{malitzGraphsEdgesHave1994} proved that any graph with $e$ edges has pagenumber $O(\sqrt{e})$. Additionally, he proved that random $d$-regular graphs $G$ with $n$ vertices have the property that $\pn(G) \in \Omega(\sqrt{d} n^{1/2 - 1/d})$. For random 3-regular graphs $G$ with $n$ vertices, $\pn(G) \in \Omega(n^{1/6})$. These constructions of $\Omega(n^d)$ pagenumber graphs are not explicit.\ \textcite{eppsteinThreeDimensionalGraphProducts2024} showed that $\pn(P_n \boxtimes P_n \boxtimes P_n) \in \Theta(n^{1/3})$. This is an explicit construction of a graph which has pagenumber in $\Theta(n^{d})$. 
\section{Treewidth}\label{sec:treewidth}
The \textit{treewidth} of a graph \(G\) measures how similar $G$ is to a forest.

\begin{definition}[Tree-decomposition]\label{def:tree-decomposition}
	A tree-decomposition \(\tree\) of a graph \(G\) is defined as a tree \(T\) with associated \textit{bags} \(\lbrace B_x : x \in V(T) \rbrace\) such that:
	\begin{itemize}
		\item $\bigcup_{x \in V(T)} B_x = V(G)$.
		\item For all \(v \in V(G)\), the subset of vertices \(\left\lbrace x \in V(T): v \in B_x \right\rbrace\) induces a connected subtree in \(V(T)\).
		\item For all edges \(vw \in E(G)\), there exists a bag \(B_x\) such that both \(v\) and \(w\) are in \(B_x\).
	\end{itemize}
\end{definition}
We refer to the vertices of the tree \(T\) as \textit{nodes}.
The \textit{width} of the tree decomposition \(\tree\) is defined as \(\max \lbrace |B_x| - 1 : x \in V(T) \rbrace\).
The treewidth of a graph \(G\), denoted as \(\tw(G)\), is defined to be the smallest width for all tree decompositions of the graph \(G\).

\begin{lemma}[Helly Property]\label{lem:Helly}
	Let \(T_1, \ldots, T_k\) be subtrees of a tree \(T\) such that for every pair of trees $T_i$, $T_j \in T_1, \ldots, T_k$, $V(T_i) \cap V(T_j) \neq \emptyset$. Then there exists a vertex which is common to all trees.
\end{lemma}
\begin{proof}
	This proof is by induction on the number of vertices of $T$. Suppose $T$ has a single vertex. Then it is obvious that the Helly property holds. By induction, suppose the Helly property holds for all trees with at most $n$ vertices. Suppose $T$ has $n + 1$ vertices and \(T_1, \ldots, T_k\) are subtrees which satisfy the property above. Let $v$ be a leaf vertex of $T$ with neighbour $w$. If one of the subtrees $T_i = \{v\}$, then by nonempty intersection, all trees contain $v$. $v$ is the common intersection. Otherwise, consider $T - v$ and the subtrees $(T_1 - v, \ldots, T_k - v)$. If $v \in T_i \cap T_j$, then as none of the subtrees is the single vertex $\{v\}$, $w \in T_i \cap T_j$. Therefore, $T_i - v \cap T_j - v$ is nonempty. By the induction hypothesis, $T - v$ has a vertex common to all $(T_1 - v, \ldots, T_k - v)$, so \(T_1, \ldots, T_k\) has a common vertex in $T$. 
\end{proof}

\begin{lemma}[Clique lemma]\label{lem:clique}
	For every graph $G$, in any tree-decomposition of \(G\), for every clique \(C\) in \(G\), there exists a node \(x \in V(T)\) such that \(C \subseteq B_x\).
\end{lemma}

\begin{proof}
	Let \(\tree\) be a tree-decomposition. Every vertex \(v\) induces a connected subtree in \(T\), call it \(T_v\). Then for any two vertices \(x, y\) in \(C\), \(T_x\) and \(T_y\) must intersect as the edge \(xy\) is inside a bag \(B_z\) corresponding to a node \(z\). Then by the Helly property, there exists a node \(v\) such that \(C \subseteq B_v\).
\end{proof}

\begin{corollary}\label{cor:complete_tw}
	\(\tw(K_n) = n-1\).
\end{corollary}
\begin{proof}
	By \cref{lem:clique}, $\tw(K_n)\geq n-1$. Placing all vertices of $K_n$ in a single bag is a tree-decomposition of width $n-1$. Therefore, $\tw(K_n) = n-1$. 
\end{proof}

\begin{theorem}\label{thm:tw_minor_closure}
	If \(H\) is a minor of \(G\), then \(\tw(H) \leq \tw(G)\).
\end{theorem}
\begin{proof}
	Let \((B_x : x \in V(T))\) be a tree-decomposition of \(G\). Remove an edge $e$ from $G$. Then \((B_x : x \in V(T))\) is a tree-decomposition of $G - e$. Remove a vertex $v$ from $G$. Then \((B_x - v : x \in V(T))\) is a tree-decompsition of $G - v$. Contract an edge $vw$ in $G$ to $u$. Define a new tree-decomposition $\tree'$ by relabeling \(v\) and \(w\) in all $B_x$ to \(u\). $\tree'$ is a valid tree-decomposition of $G \setminus uv$. The induced subtree of \(u\) is the union of the induced subtrees of \(v\) and \(w\), which is a subtree. As $v$ and $w$ share the edge $vw$, then there exists a bag $B_x$ such that $v, w \in B_x$. Every neighbor of \(v\) or \(w\) is a neighbor of \(u\). The edges in the neighborhood do not change. Notice that the size of each bag in each operation does not increase. Therefore, if $H \leq G$ by a series of vertex deletions, edge deletions, and edge contractions, the tree-decomposition \((B_x : x \in V(T))\) of $G$ can have the algorithm applied above to build a tree-decomposition of $H$ with width at most the tree-decomposition of $G$. Then by the minimality of the treewidth, \(\tw(H) \leq \tw(G)\). 
\end{proof}

\begin{lemma}\label{lem:treewidth_forest}
	\(\tw(G) = 1\) if and only if \(G\) is a forest.
\end{lemma}

\begin{proof}
	Suppose \(G\) is a tree. Root the graph \(G\) at the vertex \(r\). Then let \(T = G\) and \(B_x:= \lbrace x, p \rbrace\) where \(p\) is the parent of \(x\) and $x \neq r$. The bag \(B_r\) will just contain \(r\). Then all edges \(vw\) will be between parent \(v\) and child \(w\), so it will be in bag \(B_w\). Finally, the subgraph induced by vertex \(x\) in \(T\) will be \(B_x\) and the children of \(B_x\), which is a connected subtree.
	\par
	If \(G\) is a forest, then we perform this operation on every connected component of \(G\) and connect the roots to form a new tree. Then this tree is a tree-decomposition. This forms a tree-decomposition of width at most 1. An example is in \cref{fig:tree-treedecomp}.
	\par
	If \(G\) has a cycle \(C\), then $G$ has a $K_3$-minor. By \cref{cor:complete_tw}, $\tw(K_3) = 2$. By \cref{thm:tw_minor_closure}, $2 \leq \tw(G)$. Therefore, $G$ has treewidth at least two. 
	\begin{figure}[ht]
		\centering
		\usetikzlibrary {graphs,graphdrawing}
 \usegdlibrary {trees}
\tikz [subgraph text bottom=text centered,
subgraph nodes={font=\itshape}]
\graph [tree layout] {
	1 -> { 2 -> {3, 4}, 5 -> {6, 7 -> 8} };
	left [draw] // { b, c, d };
	right [draw] // { e, f, g, h};
};
		\usetikzlibrary {graphs,graphdrawing}
\usegdlibrary {trees}
\tikz 
\graph [tree layout] {
	1 -> { 2 -> {3, 4}, 5 -> {6, 7 -> 8} };
};
		\caption{A tree and its tree-decomposition. Every non-root bag consists of a vertex and its parent.}\label{fig:tree-treedecomp}
	\end{figure}
\end{proof}

\begin{lemma}\label{ex:tw_outerplanar}
	The treewidth of an outerplanar graph is at most 2.
\end{lemma}
\begin{proof}
	Let \(G\) be an outerplanar graph, and let \(G'\) be a triangulation of \(G\). Since \(G\) is a minor of \(G'\), \(\tw(G) \leq \tw(G')\). We look at the \textit{weak dual} of \(G'\). This is a tree \(T\), where every node \(v_f\) in \(T\) corresponds to an internal face \(f\) in \(G'\). Then let \(B_{v_f}\) be the bag of the tree-decomposition, where \(B_{v_f}\) is the set of vertices on the boundary of the face \(f\). Then the tree \(T\) with bags \(B_{v_f}\) is a valid tree-decomposition of \(G'\). Every vertex is on the boundary of some internal face, so every vertex is in some bag.  where every bag has at most 3 vertices. Furthermore, every edge is on the boundary of some internal face, so every edge is in some bag. Finally, let $v$ be a vertex. Then the bags that contain $v$ must be connected in $T$ as there is a sequence of internal faces which are adjacent to $v$ and are connected in $T$. Thus, \(\tw(G) \leq 2\).
	\todo{add picture!}
\end{proof}

Define a \(k\)-tree inductively. The complete graph \(K_k\) is a \(k\)-tree, and if \(G\) is a \(k\)-tree, then add a new vertex to \(G\) that is adjacent to \(k\) vertices that form a clique of size \(k\) in \(G\) results in a \(k\)-tree.
A \(k\)-tree is a maximal graph with treewidth \(k\).
\begin{theorem}
	For all graphs $G$, \(\tw(G) \leq k\) if and only if \(G\) is a subgraph of a \(k\)-tree.
\end{theorem}


\begin{theorem}\label{thm:treewidth_clique-minor-free}
	For all graphs $G$, if \(\tw(G) \leq k\), then \(G\) is \(K_{k+2}\)-minor-free.
\end{theorem}
\begin{proof}
	We shall prove the contrapositive: If \(K_t\) is a minor of \(G\), then \(\tw(G) \geq t-1\).
	If \(K_t\) is a minor of \(G\), then from \cref{thm:tw_minor_closure} that \(\tw(K_t) \leq \tw(G)\). As \(\tw(K_t) = t-1\), then \(\tw(G) \geq t - 1\).
\end{proof}

Treewidth was introduced by \textcite{berteleChapterEliminationVariables1972} with applications to dynamic programming under the name ``dimension''. It was then rediscovered by \textcite{halinSfunctionsGraphs1976}. Neither of the papers above discuss treewidth with an explicit construction.

\textcite{robertsonGraphMinorsIII1984} introduced tree-decompositions as defined in \cref{def:tree-decomposition}. This definition is concrete and could be calculated explicitly. They showed that if $\mathcal{F}$ is a graph family with bounded treewidth, then there exists a planar graph $H$ such that $H$ is a forbidden minor of $\mathcal{F}$. This was used to prove the Graph Minor Theorem.


\section{Path-width}\label{sec:Pathwidth}
Similar to treewidth, the pathwidth of a graph \(G\) defines how similar $G$ is from a path.

Define the path-decomposition of a graph \(G\) to be a sequence of bags \(B_i\) such that the subsequence of bags containing a vertex \(v\) induces a nontrivial subpath and each edge \(vw\) is in a bag \(B_i\). Then define the width of a path-decomposition as \(\max_i \lbrace |B_i| \rbrace -1\), same as treewidth.

If a graph has a path-decomposition \({(B_i)}_i\), then it has a tree-decomposition \(\left({(B_i)}_i, P\right)\). Therefore,
\begin{equation*}
	\pw(G) \geq \tw(G).
\end{equation*}
The pathwidth of \(G\) is the largest pathwidth over all connected components.

A graph \(G\) is a \textit{caterpillar} if \(G\) is a tree and $G$ has a path \(P\) where every vertex not in $P$ is adjacent to a vertex on the path \(P\). Alternatively, a tree \(G\) is a caterpillar if removing every leaf yields a path. We refer to the path where every leaf is connected to as the \textit{central path}.
\begin{theorem}
	Graphs have pathwidth at most 1 if and only if every connected component is a caterpillar.
\end{theorem}
\begin{proof}
	Suppose \(G\) is a caterpillar.
	Denote \(P =\left( p_1, p_2, \dots, p_n\right)\) as the central path. The leaves of vertex \(p_i\) are denoted as \(v_{i, 1}, v_{i, 2} \dots, v_{i, k}\). Define the bags as \((v_{1, 1}, v_1)\), \((v_{1, 2}, v_1)\) \dots \((v_{1, j}, v_1)\),  \((v_1, v_2)\), \((v_{2, 1}, v_2)\), \((v_{2,2}, v_2,)\) \dots. We can see that each leaf appears once and each vertex on the central path is on a subpath of the path. Therefore, the pathwidth of \(G\) is 1. We can repeat this for every connected component.
	\par
	Suppose \(G\) has pathwidth 1. Then for each connected component of \(G\), we choose a vertex \(v\) in \(B_1\) and a vertex \(w\) in \(B_n\), the final bag, and look at a path from \(v\) to \(w\). This path must go through every bag, thus the non-path vertices must have as their neighbour the path vertex in the bag and thus the graph is a caterpillar. An example of this is in \cref{fig:caterpillar}.
\end{proof}
\begin{figure}[ht]
	\centering
	\includesvg[pretex=\small, width = 0.8\textwidth]{figures/caterpillar}
	\caption{A caterpillar graph with central path \((v_1, v_2, v_3, v_4, v_5, v_6)\).}\label{fig:caterpillar}
\end{figure}

\begin{example}
	Recall that $K_n$ is the complete graph on $n$ vertices. It holds that \(\pw(K_n) = \tw(K_n) = n - 1\).
\end{example}
\begin{proof}
	The pathwidth of \(K_n\) is at least the treewidth of \(K_n\). But the pathwidth is at most \(n- 1\) (where all the vertices are in the same bag), but the treewidth of \(K_n\) is \(n - 1\). Therefore, \(\pw(K_n) = n - 1\).
\end{proof}

\begin{theorem}
	The pathwidth of a tree \(T\) equals \(\min_{P \subseteq T} \left\lbrace 1 + \pw(T - V(P))\right\rbrace \) where \(P\) is a path.
\end{theorem}

\begin{proof}[Proof]
	Start by showing the upper bound, \(\pw(T) \leq 1 + \pw(T - V(P))\). If \(P\) is a path in \(T\) with vertices \(v_1, v_2, \ldots v_i\), then consider the subtrees hanging off \(v_i\) for all \(i\). \(T - V(P)\) will have a minimal width path-decomposition. We can order each connected component such that they appear in the order of their parents on the paths. Then adding \(v_i\) to the bags of subtrees connected to \(v_i\), and the bag \((v_i, v_{i+1})\) between the subtrees \(v_i\) and \(v_{i + 1}\) will yield a path-decomposition of width \(1 + \pw(T - V(P))\).
	\par
	To show there exists a path \(P\) such that \(\pw(T) \geq 1 + \pw(T - V(P))\), we proceed by induction. Let \(B_1, \ldots B_n\) be a path-decomposition of \(T\). Let \(x\) live in bag \(B_1\) and \(y\) live in bag \(B_n\), the final bag. Then let \(P\) be the unique path from \(x\) to \(y\). Then \(P\) traverses through every bag in the path-decomposition. Then \(\tw(T) \geq 1 + \tw(T - P)\) by induction.
\end{proof}
