\chapter{Definitions}\label{chap:Definitions}
We lay out this chapter to build towards important notions in structural graph theory.
\begin{itemize}
	\item \cref{sec: Basic definitions} defines the basics of graph theory and introduces useful notation.
	\item \cref{sec:Planar graphs} discusses important information involving planar graphs. 
	\item \cref{sec:Graph Minors} discusses graph minors.
	\item \cref{sec:Book Embedding} defines what a book-embedding is and its importance. 
	\item \cref{sec:treewidth} defines the \textit{treewidth}.
	\item \cref{sec:Pathwidth} defines a related notion, the \textit{pathwidth}.
\end{itemize}
These notions will be useful in discussing the Graph Minor Structure Theorem and our conjecture. 
\section{Basic definitions}\label{sec: Basic definitions}
For a graph \(G\), we define the \textit{vertex} and \textit{edge} sets of \(G\) to be \(V(G)\) and \(E(G)\) respectively.
For a subset of vertices \(A \subseteq V(G)\), we denote the \textit{induced subgraph} on \(G\) with vertex set \(A\) as \(G[A]\). 

\begin{itemize}
	\item A \textit{path} in a graph \(G\) is a sequence of edges \(e_1, e_2, \ldots, e_{\ell- 1}\) which join a sequence of vertices \(v_1, v_2, \ldots, v_{\ell}\) such that \(e_i = v_iv_{i + 1}\), and all the vertices are distinct. 
	\item A \textit{simple cycle} \(C\) in a graph \(G\) is a sequence of edges \(e_1, e_2, \ldots, e_{\ell}\) which join a sequence of distinct vertices \(v_1, v_2, \ldots, v_{\ell}\) such that \(e_i = v_iv_{i + 1}\) for \(1 \leq i \leq \ell - 1\) and \(e_\ell = v_\ell v_1\). 
	\item A \textit{Hamiltonian cycle} in a graph \(G\) is a simple cycle \(C\) such that all vertices in \(G\) appear in \(C\).
\end{itemize}

We say that a connected graph \(G\) is \textit{\(k\)-connected} if \(G\) has more than \(k\) vertices and removing up to \(k-1\) vertices from \(G\) maintains the connected property. For the case that \(k = 2\), the graph is \textit{biconnected}. Note that all graphs \(G\) with a Hamiltonian cycle is biconnected, due to Menger's theorem.\cite{mengerZurAllgemeinenKurventheorie1927a}

We say a graph \(G\) is \(k\)-colourable if there exists a function \(f: V(G) \rightarrow [k]\) such that for all \(vw \in E(G)\), \(f(v)\) and \(f(w)\) are assigned different colours. The \textit{chromatic number} \(\chi(G)\) is the smallest \(k\) such that \(G\) is \(k\)-colourable. 

Throughout this report, we use as shorthand for the set \([n] = \lbrace 1\ldots n \rbrace\). 

\subsection{Menger's theorem}
Menger's theorem is an important theorem which we use throughout the report, so it is worth dedicating some time discussing its significance.

Let \(G\) be a graph and \(A, B \subseteq V(G)\). An \(AB\)-path is a path in \(G\) which starts in \(A\) and ends in \(B\) with no internal vertices in \(A\) or \(B\). An \(AB\)-connector is a set of interally-disjoint \(AB\)-paths. An \(AB\)-separator is a set \(S\) such that \(G - S\) contains no \(AB\) path. Then:

\begin{theorem}[Menger's theorem]
	\label{thm:Menger}
	The size of the smallest \(AB\) separator of \(G\) is equal to the largest \(AB\)-connector. 
\end{theorem}

Now take vertices \(x, y\). Now let \(A = N_G(x) \cup \{x\}\) and \(B = N_G(y) \cup \{y\}\). Then the size of the smallest \(AB\) separator implies that:
\begin{theorem}[Menger's theorem, vertex-connectivity version]
	\label{thm:Menger_Vertex}
	\(G\) is \(k\)-connected iff every pair of vertices has at least \(k\) internally disjoint paths between. 
\end{theorem}

\section{Planar graphs}\label{sec:Planar graphs}
We say a graph \(G\) is \textit{planar} if \(G\) can be drawn on the Euclidean plane \(\Sigma\) such that no two edges cross. Say \(G\) is drawn on a plane. If \(G\) is embedded on \(\Sigma\), then \(\Sigma\) is divided into regions where no edges cross. These regions are called \textit{faces}. The \textit{outerface} is the face on the outside of the graph. We say that a set of vertices \textit{lie} on a face if they are on the boundary of the face. We also say that the vertices and edges \textit{bound} a face. \(G\) is \textit{outerplanar} if \(G\) is planar and all vertices in \(G\) lie on the outerface. 
Let \(F(G)\) be the set of faces of \(G\) embedded on \(\Sigma\). Then we have that:
\begin{theorem}[Euler's formula]\label{thm:Euler_planar}
	\begin{equation}
		|V(G)| - |E(G)| + |F(G)| = 2
	\end{equation}
\end{theorem}

We can use this result to bound the number of edges in an outerplanar graph.
\begin{theorem}\label{thm:outerplanar_bound}
	If \(G\) is outerplanar with \(n\) vertices and \(m\) edges, then \(m \leq 2n - 3\).
\end{theorem}

\begin{proof}[Proof of theorem]
	Suppose \(G\) is \textit{maximal outerplanar}, meaning adding any edge will break the outerplanar property. Let the \textit{internal faces} be all the faces which are not the outerface, and let there be \(f\) internal faces. Then the outerface has \(n\) edges on the boundary, but each internal face will have exactly \(3\) edges on the boundary. However, each edge is touching two distinct faces. Therefore we have that
	\begin{equation}
		3 f - 3 + n = 2m
	\end{equation}
	Combining with \cref{thm:Euler_planar} given by
	\begin{equation}
		n - m + f = 2
	\end{equation}
	
	we have, after some rearrangment, 
	\begin{equation}
		2n = 3 + m
	\end{equation}
	Therefore, \(m = 2n - 3\). As every outerplanar graph is a subgraph of a maximal planar graph, then we have that \(m \leq 2n - 3\). 
\end{proof}
\section{Graph minors}\label{sec:Graph Minors}
We say that a graph \(H\) is a \textit{minor} of another graph \(G\) if a graph isomorphic to \(H\) can be obtained from \(G\) by deleting vertices, edges, and \textit{contracting} edges. To \textit{contract} an edge \(uv\), we delete both \(u\) and \(v\) and create a new vertex \(w\) such that \(N_G(w) = N_G(u) \cup N_G(v)\), where \(N_G(v)\) is the neighbourhood of \(v\). We notate an edge contraction on \(uv\) as \(G\setminus uv\).
We notate the statement ``\(H\) is a minor of \(G\)'' as \(H \leq G\). 
We define minors up to isomorphism, and we typically omit the isomorphism relation in our definitions. We say that a graph \(G\) is \textit{\(H\)-minor-free} if there is no sequence of deletions and contractions of \(G\) that yield a graph isomorphic to \(H\). We say that a family of graphs \(\mathcal{F}\) is minor-closed if for all \(G \in \mathcal{F}\), all minors of \(G\) is also in \(\mathcal{F}\). 

\begin{example}
	An example of a minor-closed class is the class of planar graphs. 
\end{example}

An important class of graph families are the \(K_t\)-minor free graphs. For a graph \(G\), we define \textit{Hadwiger's number} \(\had(G)\) to be the largest \(t\) such that \(K_t\) is a minor of \(G\). This is named after Hugo Hadwiger and his most famous conjecture.

\begin{conjecture}[Hadwiger's conjecture]\cite{hadwigerUeberKlassifikationStreckenkomplexe1943}
	For all graphs \(G\), \(\chi(G) \leq \had(G)\), where \(\chi(G)\) is the \textit{chromatic number} of \(G\). 
\end{conjecture}
Much work has been done on solving Hadwiger's conjecture but as of writing this report it has not been solved. 
\subsection{Minors and models}
A \textit{model} of a graph \(H\) to a graph \(G\) is a function \(\rho\) which assigns to \(H\) vertex disjoint connected subgraphs of \(G\), such that if \(uv \in E(H)\), then some edge in \(G\) connects a vertex in \(\rho(u)\) to \(\rho(v)\). This definition comes from Sergey Norin in his notes on graph minors \cite{norinMath599GraphMinors2017}. 

\begin{theorem}
	\(H\) is a model of \(G\) iff \( H \leq G\). 
\end{theorem}

\begin{proof}[From Norin \cite{norinMath599GraphMinors2017}]
	Suppose \(H\) is a model of \(G\). Then for all \(x\) in \(V(H)\), we contract \(\rho^{-1}(x)\) in \(G\) to a single vertex. We can do this as each preimages are disjoint and connected. Then we delete edges to form \(H\). 
	
	Suppose \(H \leq G\). We will use induction to show that \(H\) has a model in \(G\). Suppose \(H\) is obtained from \(G\) by contraction operations only. We can assume this by taking a subgraph of \(G\) if necessary. Let \(uv\) be the edge that is contracted first and let \(G' = G \setminus uv\). Let \(w\) be the vertex obtained after contracting \(uv\). Then by induction, there is a model \(\mu'\) of \(H\) in \(G'\). Then we look at the block \(B\) where \(H\) lives in, and we delete \(w\) from \(H\) and add back in \(u\), \(v\), and \(uv\). Then this is a model of \(H\) in \(G\). 
\end{proof}
\todo{Add pictures}
\section{Book embedding}\label{sec:Book Embedding}
A \textit{book} of \textit{thickness} \(k\) are \(k\) half-planes glued together on a common boundary. We refer to the boundary as the \textit{spine} and we refer to the individual half-planes as \textit{pages}. In topology, these are referred to as \textit{fans} of half-planes. Books were described by Persinger and Atnosen in the 1960s \cite{persingerSubsetsNbooksE31966, atneosenOnedimensionalNleavedContinua1972}. 
A \textit{book-embedding} of a graph \(G\) is an embedding of \(G\) on a book. We place the vertices of \(G\) on the mutual boundary of all half-planes, and we place the edges on each half-plane such that no two edges cross.

The \textit{pagenumber} of a graph \(G\) is the minimum number of pages required to embed \(G\) on a book. This is also referred to as \textit{book-thickness}, or \textit{stack-number}. 

We have another definition which abstracts the underlying topology and focuses on the graph. This definition is more combinatorial in nature. 
A \textit{book embedding} of a graph \(G\) is an arrangement of the vertices of \(G\) in a total ordering \(v_1 < v_2 < \ldots < v_n\). We then colour the edges \(E(G)\) such that if we have \(v_a < v_b < v_c < v_d\) and edges \(v_a v_c\) and \(v_b v_d\), then the edges are each assigned different colours.
We refer to the total ordering of \(V(G)\) in the book embedding as \((<)\) or as \((\leq)\). For a book-embedding \((v_1, v_2, \ldots, v_{|G|})\), we refer to the edges \(\left\{v_1 v_2, v_2 v_3, \ldots, v_{|G| - 1}, v_{|G|}v_{1}\right\}\) as \textit{spine edges}.
We may use a \textit{circular ordering} of the vertices rather than a linear ordering. This means that we order the vertices in a circle rather than on a straight line. The book-embedding of a circular ordering is exactly the same as for a linear ordering, and we can convert between a circular and linear ordering by choosing a vertex to be at the start of the sequence. 

Book-embeddings were introduced by Kainen and Ollmann in the 1970s.\cite{kainenRecentResultsTopological1974, ollmannBookThicknessVarious1973} It was developed further in a paper by Bernhart and Kainen \cite{bernhartBookThicknessGraph1979}. 
\begin{lemma}\label{lem:Pagenumber_1}
	A graph \(G\) has page-number at most 1 iff \(G\) is outerplanar.
\end{lemma}
\begin{proof}
	We can choose an ordering of the vertices \(V(G)\) to go anticlockwise around the outer face. Suppose we have two edges \(uv\), \(xy\) that cross, so that \(u < x < v < y\). Then in the original graph embedding, we will have that \(uv\) and \(xy\) will cross inside the circle, thus \(G\) is not outerplanar. 
\end{proof}
\begin{lemma}\label{lem:Pagenumber_2}
	A graph \(G\) has pagenumber at most 2 iff \(G\) is a subgraph of a planar graph with a Hamiltonian cycle.
\end{lemma}
This is because we can embed the graph on a sphere with the vertices and Hamiltonian cycle on the equator, and edges on each hemisphere forms the interior and exterior edges of the cycle respectively.

\subsection{Pagenumber of complete graphs}\label{ssec:Pagenumber_Complete_Graphs}
This is an upper bound of any graph \(G\) with vertices \(n\). 
\begin{theorem}[\cite{bernhartBookThicknessGraph1979}]\label{thm:Pagenumber_Complete_Graph}
	The complete graph \(K_n\) on \(n\) vertices where \(n \geq 4\), has pagenumber \( \leq \lceil n/2 \rceil\). 
\end{theorem}
\begin{proof}
	Arrange the vertices in any circular ordering, and colour the edges like in \cref{fig:k8 coloured with colours}. To show the other equality, we use \cref{lem:Edge_Bound}.
	\begin{figure}[ht]
		\caption{Circular embedding of \(K_8\) with 4 colours, the minimum possible.}
		\centering
		\usetikzlibrary{graphs,graphs.standard}

\tikz
	\graph[nodes={circle, draw}] { 
		subgraph K_n [n=8,clockwise,radius=2cm];
		
		{[induced path, edges= red] 1,2,8,3,7,4,6,5},
		{[induced path, edges= blue] 8,1,7,2,6,3,5,4},
		{[induced path, edges= green] 7,8,6,1,5,2,4,3},
		{[induced path, edges= yellow] 6,7,5,8,4,1,3,2},
 };
		\label{fig:k8 coloured with colours}
	\end{figure}
\end{proof}
Therefore for any graph \(G\) on \(n\) vertices, \(n \geq 4\), \(\pn(G) \leq \lceil n/2 \rceil\). 
\subsection{Related graph properties}\label{ssec:Related_Properties}

\begin{lemma}[Bound on number of edges \cite{bernhartBookThicknessGraph1979}]\label{lem:Edge_Bound}
	If an \(n\)-vertex graph \(G\) has \(\pn(G) = k\), then \(G\) has at most \(n + k(n-3)\) edges.
\end{lemma}
\begin{proof}
	Given a vertex ordering \(v_1 \leq v_2 \leq \ldots \leq v_n\), we have that the spine edges \(v_i v_{i + 1}\), \(v_1 v_n\)  can appear on any page. Furthermore we have there are at most \(n-3\) non outer-cycle edges on each page as the maximum number of edges in an outerplanar graph is \(2n - 3\) from \cref{thm:outerplanar_bound}, but we remove the outer cycle (with \(n\) edges on the cycle) to have at most \(n-3\) edges on each page. Therefore, \(m\), the number of edges, satisfies \(m \leq n + k (n - 3)\). 
\end{proof}

\begin{corollary}
	The pagenumber of \(K_n\) is \(\lceil \frac{n}{2} \rceil\) when \(n \geq 4\).
\end{corollary}
We have shown one direction of the inequality, and we use \cref{lem:Edge_Bound} to prove the other direction. We have that \(K_n\) has \(n\) vertices and \(\binom{n}{2}\) edges. Then \(\pn(K_n) \geq \frac{\binom{n}{2} - n}{n - 3} = \frac{n}{2}\) when \(n \geq 4\), from the lemma. As \(\pn(K_n)\) is an integer, we take the ceiling of \(\frac{n}{2}\). This concludes the equality. 
\begin{theorem}[Chromatic number bound\cite{bernhartBookThicknessGraph1979}]\label{thm:Colouring_Bound}
	For all graphs \(G\), \(\chi(G) \leq 2 \pn(G) + 2\).
\end{theorem}
\begin{proof}
	Let \(\pn(G) = k\) and suppose \(G\) has \(n\) vertices and \(m\) edges. Then we have that the average degree of \(G\), \(d(G) = 2m/n\) by the handshaking lemma. But \(2m/n \leq 2 \frac{n + k(n-3)}{n} < 2k + 2\). But this implies that \(G\) has a vertex of degree \(\leq 2k + 1\), and as if \(G'\) is a subgraph of \(G\), then \(G'\) also has \(\pn(G') \leq k\), thus \(G'\) has a vertex of degree at most \(2k + 1\). However, this implies \(G\) is \(2k + 1\)-degenerate, thus \(\chi(G) \leq 2k + 2\). 
\end{proof}
We may note that this bound is not tight when \(\pn(G) = 1\) or \(2\). When \(\pn(G) = 1\), then \(G\) is outerplanar, but can be coloured with 3 colours. When \(\pn(G) = 2\), then \(G\) is Hamiltonian and planar, but is 4-colourable. 

\subsection{Historical interest}\label{ssec:Pagenumber_History}
Pagenumbers were developed for VLSI and processor designs, but more recently have been used in bioinformatics \cite{haslingerRNAStructuresPseudoknots1999}. 
The project of finding upper and lower bounds of the pagenumber of planar graphs was started by Bernhart and Kainen when they conjectured that planar graphs had unbounded page-number. However, Buss and Shor\cite{bussPagenumberPlanarGraphs1984} found that the pagenumber of planar graphs was at most 9, and Heath \cite{heathEmbeddingPlanarGraphs1984} found that the pagenumber of planar graphs is at most 7. Yannakakis' devised an algorithm to bound the pagenumber to at most 4 \cite{yannakakisEmbeddingPlanarGraphs1989}. Yannakakis, in this paper, claimed that there exist planar graphs with pagenumber 4. However, his proof was incomplete and the full proof was left unpublished. In 2020, Yannanakis published his full proof. \cite{yannakakisPlanarGraphsThat2020} At around the same time, Kaufmann, Bekos, Klute, Pupyrev, Raftopoulu and Ueckerdt published a similar result\cite{kaufmannFourPagesAre2020}. 

\section{Treewidth}\label{sec:treewidth}

The \textit{treewidth} of a graph \(G\) measures how far \(G\) is from being a forest \cite{diestelGraphMinors2017}. 

\begin{definition}[Tree-decomposition]\label{def:tree-decomposition}
	The tree-decomposition \(\tree\) of a graph \(G\) is defined as a tree \(T\) with associated \textit{bags} \(\lbrace B_x : x \in V(T) \rbrace\) such that:
	\begin{itemize}
		\item Every vertex in \(G\) is in at least one bag \(B_x\). 
		\item for all \(v \in V(G)\), the subset of vertices \(\left\lbrace x \in V(T): v \in B_x \right\rbrace\) in \(V(T)\) induces a connected subtree in \(V(T)\).
		\item For all edges \(vw \in E(G)\), there exists a bag \(B_x\) such that both \(v\) and \(w\) are in the bag \(B_x\).
	\end{itemize}
\end{definition}
We refer to the vertices of the tree \(T\) as \textit{nodes}. 
The \textit{width} of the tree decomposition \(\tree\) is defined as \(\max \lbrace |B_x| - 1 : x \in V(T) \rbrace\). An important tree-decomposition 

\begin{definition}\label{def:treewidth}
	The treewidth of a graph \(G\), denoted as \(\tw(G)\), is defined to be the smallest width for all tree decompositions of the graph \(G\).
\end{definition}


\begin{example}\label{ex:treewidth_forest}
	\(\tw(G) = 1\) iff \(G\) is a forest.
	\begin{lemma}
		If \(G\) is a forest, then \(\tw(G) = 1\).
	\end{lemma}
	\begin{proof}
		Suppose \(G\) is a tree. Root the graph \(G\) at the vertex \(r\). Then let \(T = G\) and \(B_x:= \lbrace x, p \rbrace\) where \(p\) is the parent of \(x\). The bag \(B_r\) will just contain \(r\). Then all edges \(vw\) will be between parent \(v\) and child \(w\), so it will be in bag \(B_w\). Finally, the subgraph induced by vertex \(x\) in \(T\) will be \(x\) and the children of \(x\), which is a connected subtree.
		
		If \(G\) is a forest, then we perform this operation on every connected component of \(G\) and connect the roots to form a new tree. Then this tree is a tree-decomposition. This forms a tree-decomposition of width at most 1. An example is in \cref{fig:tree-treedecomp}. 
		\begin{figure}[ht]
			\centering
			\usetikzlibrary {graphs,graphdrawing}
 \usegdlibrary {trees}
\tikz [subgraph text bottom=text centered,
subgraph nodes={font=\itshape}]
\graph [tree layout] {
	1 -> { 2 -> {3, 4}, 5 -> {6, 7 -> 8} };
	left [draw] // { b, c, d };
	right [draw] // { e, f, g, h};
};
			\usetikzlibrary {graphs,graphdrawing}
\usegdlibrary {trees}
\tikz 
\graph [tree layout] {
	1 -> { 2 -> {3, 4}, 5 -> {6, 7 -> 8} };
};
			\caption{A tree and its tree-decomposition. We root at the vertex \(1\) and every bag consists of a vertex and its parent.}
			\label{fig:tree-treedecomp}
		\end{figure}
	\end{proof}
	\begin{lemma}
		If \(\tw(G) = 1\), then \(G\) has no cycles.
	\end{lemma}
	\begin{proof}
		If \(G\) has a cycle \(C\), then the treewidth cannot be 1. This is because if there is a tree decomposition \(\tree\) where the size of each bag is at most 2, then as the graph must have every edge, then every edge in \(C\) is in separate bags. However, we have that for any vertex \(v\) in \(C\) to have an induced connected subgraph in \(T\), then it follows that the cycle \(C\) is also in \(T\). Thus \(T\) is not a tree, and this is not a valid tree-decomposition. 
	\end{proof}
\end{example}

\begin{lemma}[Helly Property]\label{lem:Helly}
	Let \(T_1, \ldots, T_k\) be subtrees of a tree \(T\) such that for every pair of trees, there is a vertex in common. Then there exists a vertex which is common to all trees.
\end{lemma}
\begin{proof}[Helly property]
	If \(T_1\), \(T_2\) and \(T_3\) are subtrees of \(T\) such that the vertex sets are pairwise nonempty, then there is a common vertex in all three subtrees. If this is not the case, denote \(v_1\) as a vertex in the intersection of \(T_1\) and \(T_2\), \(v_2\) as the vertex in \(T_1 \cap T_3\), and \(v_3\) as the vertex in \(T_2\) and \(T_3\). Then there exists a unique path \(P\) in \(T_1\) from \(v_1\) to \(v_2\).
\end{proof}

\begin{theorem}[Clique theorem]\label{thm:clique}
	In any tree-decomposition of \(G\), for every clique \(C\) in \(G\), there exists a node \(x \in V(T)\) such that \(C \subseteq B_x\). 
\end{theorem}

\begin{proof}
	Let \(\tree\) be a tree-decomposition. Every vertex \(v\) induces a connected subtree in \(T\), call it \(T_v\). Then for any two vertices \(x, y\) in \(C\), we have that \(T_x\) and \(T_y\) must intersect as the edge \(xy\) is inside a bag \(B_z\) corresponding to a node \(z\). Then by the Helly property, there exists a node \(v\) such that \(C \subseteq B_v\).
\end{proof}

\begin{corollary}\label{cor:complete_tw}
	\(\tw(K_n)\) is \(n-1\). 
\end{corollary}

\begin{theorem}\label{thm:tw_minor_closure}
	If \(H\) is a minor of \(G\), then \(\tw(H) \leq \tw(G)\). 
\end{theorem}
\begin{proof}[Proof of theorem]
	Suppose we have a tree-decomposition \(\tree\) of \(G\). If we delete an edge in \(G\), then \(\tree\) remains a valid tree-decomposition. If we delete a vertex \(v\), then \(\tree\) where we remove \(v\) from every bag in \(\tree\) is also a valid tree-decomposition. If we contract an edge \(vw\), creating a new vertex \(u\), then relabeling \(v\) and \(w\) in all bags to \(u\) is a valid tree-decomposition as the induced subtree of \(u\) is the union of the induced subtrees of \(v\) and \(w\), and every neighbor of \(v\) or \(w\) is a neighbor of \(u\). But the edges in the neighborhood do not change. Thus this is a valid tree-decomposition of \(H\), with width at most the width of \(\tree\).
\end{proof}

Recall that an outerplanar graph is a planar graph where there exists a face such that all vertices lie on the boundary of that face. 
\begin{example}\label{ex:tw_outerplanar}
	The treewidth of an outerplanar graph is at most 2.
\end{example}
\begin{proof}[Proof of outerplanar treewidth.]
	Let \(G\) be an outerplanar graph, and let \(G'\) be the triangulation of \(G\). As \(G\) is a minor of \(G'\), \(\tw(G) \leq \tw(G')\). We look at the \textit{weak dual} of \(G'\). This is a tree \(T\), where every node \(v_f\) in \(T\) corresponds to a face \(f\) in \(G'\). Then let \(B_{v_f}\) be the bag of the tree-decomposition, where \(B_{v_f}\) is the set of vertices on the boundary of the face \(f\). Then the tree \(T\) with bags \(B_{v_f}\) is a valid tree-decomposition of \(G'\), where every bag has at most 3 vertices. Thus, \(\tw(G) \leq 2\). 
\end{proof}

\subsection{Characteristics of treewidth}\label{ssec:characterising_Treewidth}
\subsubsection{\(k\)-trees}\label{sssec:k-trees}
We define a \(k\)-tree inductively. We have that the complete graph \(K_k\) is a \(k\)-tree, and if \(G\) is a \(k\)-tree, then we add a new vertex to \(G\) that is adjacent to \(k\) vertices that form a clique of size \(k\) in \(G\) results in a \(k\)-tree. 
A \(k\)-tree is a maximal graph with treewidth \(k\). 
\begin{theorem}[K-tree theorem]
	\(\tw(G) \leq k\) iff \(G\) is a subgraph of a \(k\)-tree. 
\end{theorem}


\subsubsection{Bounded treewidth graphs}\label{sssec:Graph_treewidth_Bounded}
\begin{theorem}\label{thm:treewidth_clique-minor-free}
	If \(\tw(G) \leq k\), then \(G\) is \(K_{k+2}\)-minor-free. 
\end{theorem}
\begin{proof}
	We shall prove the contrapositive: If \(K_t\) is a minor of \(G\), then \(\tw(G) \geq t-1\).
	If \(K_t\) is a minor of \(G\), then we have that from \cref{thm:tw_minor_closure} that \(\tw(K_t) \leq \tw(G)\). As \(\tw(K_t) = t-1\), then we have that \(\tw(G) \geq t - 1\). 
\end{proof}

\subsection{Historical discussion}\label{ssec:tw_historical}
Treewidth was introduced in \cite{berteleChapterEliminationVariables1972} with applications to dynamic programming under the name ``dimension''. It was then rediscovered by Halin \cite{halinSfunctionsGraphs1976} before being used by Robertson and Seymour in \cite{robertsonGraphMinorsIII1984}, which was introduced to prove the Graph Minor Theorem\cite{robertsonGraphMinorsXX2004}.


\section{Path-width}\label{sec:Pathwidth}
Similarly to treewidth, the pathwidth of a graph \(G\) is how far a graph is from being a path. 

We define the path-decomposition of a graph \(G\) to be a sequence of bags \(B_i\) such that the subsequence of bags containing a vertex \(v\) induces a nontrivial subpath and each edge \(vw\) is in a bag \(B_i\). Then we define the width of a path-decomposition as \(\max_i \lbrace |B_i| \rbrace -1\), same as treewidth.

If a graph has a path-decomposition \((B_i)_i\), then it has a tree-decomposition \(\left((B_i)_i, P\right)\). Therefore,
\begin{equation}
	\pw(G) \geq \tw(G).
\end{equation}

Similarly to treewidth, we have the following observation.
\begin{lemma}
	The pathwidth of \(G\) is the largest pathwidth over all connected components.
\end{lemma}
We say a graph \(G\) is a \textit{caterpillar} if \(G\) has a path \(P\) and every vertex is adjacent to a vertex on the path \(P\). Alternatively, \(G\) is a caterpillar if removing every leaf yields a path. We refer to the path where every leaf is connected to as the \textit{central path}.
\begin{theorem}[Caterpillars]
	Graphs have pathwidth at most 11 iff every connected component is a caterpillar.
\end{theorem}
\begin{proof}[Caterpillars]
	Suppose \(G\) is a caterpillar. 
	Denote \(P =\left( p_1, p_2, \dots, p_n\right)\) as the central path. The leaves of vertex \(p_i\) are denoted as \(v_{i, 1}, v_{i, 2} \dots, v_{i, k}\). Define the bags as \((v_{1, 1}, v_1)\), \((v_{1, 2}, v_1)\) \dots \((v_{1, j}, v_1)\),  \((v_1, v_2)\), \((v_{2, 1}, v_2)\), \((v_{2,2}, v_2,)\) \dots. We can see that each leaf appears once and each vertex on the central path is on a subpath of the path. Therefore, the pathwidth of \(G\) is 1. We can repeat this for every connected component.
	Suppose \(G\) has pathwidth 1. Then for each connected component of \(G\), we choose a vertex \(v\) in \(B_1\) and a vertex \(w\) in \(B_n\), the final bag, and look at a path from \(v\) to \(w\). This path must go through every bag, thus the non-path vertices must have neighbour only of the other one in the bag and thus the graph is a caterpillar. An example of this is in \cref{fig:caterpillar}.
\end{proof}
	\begin{figure}[ht]
		\centering
		\includesvg{chapters/figures/caterpillar}
		\caption{A caterpillar graph. Notice that the path \(v_1, v_2, v_3, v_4, v_5, v_6\) is the central path.}
		\label{fig:caterpillar}
	\end{figure}

\begin{example}[Complete graphs]
	\(\pw(K_n) = \tw(K_n) = n - 1\). 
\end{example}
\begin{proof}
	We have that the pathwidth of \(K_n\) is at least the treewidth of \(K_n\). But we have that the pathwidth is at most \(n- 1\) (where all the vertices are in the same bag), but the treewidth of \(K_n\) is \(n - 1\). Therefore, \(\pw(K_n) = n - 1\). 
\end{proof}

\begin{theorem}
	The pathwidth of a tree \(T\) is \(\min_{P \subset T} \left\lbrace 1 + \pw(T - V(P))\right\rbrace \) where \(P\) is a path.
\end{theorem}

\begin{proof}[Proof of pathwidth of tree]
	\item
	\paragraph{Showing \(\leq\)}
	To show \(\pw(T) \leq 1 + \pw(T - V(P))\), we have that if \(P\) is a path in \(T\) with vertices \(v_1, v_2, \ldots v_i\), then consider the subtrees hanging off \(v_i\) for all \(i\). \(T - V(P)\) will have a path-width and we can order each connected component such that they appear in the order of the trees. Then we have that adding \(v_i\) to the bags of subtrees connected to \(v_i\), and the bag \((v_i, v_{i+1})\) between the subtrees \(v_i\) and \(v_{i + 1}\) will yield a path-decomposition of width \(1 + \pw(T - V(P))\). 
	\item
	\paragraph{Showing \(\geq\)}
	To show there exists a path \(P\) such that \(\pw(T) \geq 1 + \pw(T - V(P))\), we proceed by induction. Let \(B_1, \ldots B_n\) be a path-decomposition of \(T\). Let \(x\) live in bag \(B_1\) and \(y\) live in bag \(B_n\), the final bag. Then let \(P\) be the unique path from \(x\) to \(y\). Then \(P\) traverses through every bag in the path-decomposition. Then \(\tw(T) \geq 1 + \tw(T - P)\) by induction. 
\end{proof}
