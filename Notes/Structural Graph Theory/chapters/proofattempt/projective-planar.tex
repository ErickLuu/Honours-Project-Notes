% !TEX root = ./thesis.tex

\section{Partial result for Projective-planar graphs}

\begin{theorem}
	Every $G \in \mathcal{G}(g, p, k)$ where $g = 1$ can be embedded on a finite number of pages.
\end{theorem}
Recall the definition of almost-embeddable from \cref{thm:gmst}. 
\begin{theorem}
	Every graph almost-embeddable on a projective-plane can be embedded on a finite number of pages. 
\end{theorem}

\begin{proof}
	Let $G$ be an almost-projective-planar graph, so there are subgraphs $G_0, \ldots, G_p$ where $G_0$ is a projective-planar graph, $G_1, \ldots, G_p$ are vortices on $G_0$ of width $\leq k$. 
	Run Nakamoto's algorithm on a triangulation of $G_0$, but instead of implementing Yannakakis's algorithm on the planar section, use \cref{thm:Planar Graph Hickingbotham Bound}. Then $G_0$ can be embedded in at most $23$ pages. However, every face of this embedding has at most $24$ monochromatic paths, from \cref{corr:orientable_nonplanar_faces}. Therefore, from \label{lem:orientablesurfaces_monochromatic_edges}, every book-embedding requires $23 + 24k$ pages. Thus shown. 
\end{proof}