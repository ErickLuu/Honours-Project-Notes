
\chapter{Potential proof techniques}\label{chap:Proving_The_Theorem}
Our aim is to show that graphs with bounded genus $g$ containing $p$ vortices of bounded width $k$ have bounded pagenumber $f(g, p, k)$.
Using Robert's theorem \cref{thm:clique_sum_pagenumber_bound}, we can build clique-sums with adhesion $\leq \ell$ and from a theorem below, we can add $a$ apex vertices. 
Thus we can show that for fixed $t$, all $K_t$-minor free graphs have bounded pagenumber $f(g, p, k, a, \ell)$. 
We wish to find a book-embedding of a graph $G$ of bounded genus $g$ with vortices $G_1, ..., G_p$ of adhesion $k$ such that the pagenumber of $G$ is at most $f(g, p, k)$ for some constants $g$ and $p$. It is trivial to show that apex vertices only increase the number of pages by $f(a)$ for a fixed function $f$. 
\subsection{Apex vertices}
In this section, we will prove that apex vertices can be added with a bounded increase to the number of pages. Typically, apex vertices are the problematic component when applying the Graph Minor Structure theorem, but in this case they are a trivial addition.
\begin{theorem}
	If $G$ is a graph with partition $(G', A)$ such that $G'$ is a graph with pagenumber $s$, $|V(A)| \leq a$, then $G$ has pagenumber $s + \left\lceil \frac{3a}{2}\right\rceil$. 
\end{theorem}
\begin{proof}
	Let $G'$ have book-embedding $(<, \rho)$. Then place the vertices of $A$ at the very start of $(<)$ and for every edge $u_iv$, $u_i \in A$, $v \in G'$, we colour $\rho(uv) = i$. Then for any edge $e \in E(G')$, we maintain the same colour as before. Then for the edges between vertices in $A$, we have that the number of colours is bounded above by $\left\lceil \frac{a}{2} \right\rceil$ from \cref{thm:Pagenumber_Complete_Graph}. Therefore, we have that $\pn(G) \leq \pn(G') + a + \left\lceil \frac{a}{2} \right\rceil =s + \left\lceil \frac{3a}{2}\right\rceil$. 
\end{proof}

Therefore, adding $ \leq a$ apex vertices still maintains a bounded pagenumber.

\section{Vortices on graphs}
The most problematic section is dealing with vortices on surfaces. Recall that vortices are on graphs on surfaces. 

From Heath and Istrail, we can form a planar-nonplanar decomposition of $G$ of bounded genus $g$. Therefore, it makes sense to think about planar graphs first before thinking about graphs with bounded genus $g$.
Let $G$ be a graph. We say that a vertex ordering $<$ \textit{preserves} a face $F$ if there is a vertex $v_0$ on the boundary of $F$ and a vertex ordering $(v_0, v_1, ..., v_k)$ around the boundary of $F$ such that $v_0 < v_1 < ... < v_k$. We say that a circular ordering $<$ preserves a face $F$ if we can start at any point in the circular ordering and have the condition above. 
As a consequence of Tutte's theorem, we have that every 4-connected planar graph is Hamiltonian and also every face of the embedding is preserved.
\begin{theorem}
	Let $G$ be a 4-connected planar graph embedded on some surface. 
\end{theorem}

\begin{proof}
	Let $G$ be a 4-connected planar graph embedded on some surface. From Tutte's theorem, we have that $G$ has a Hamiltonian cycle $C$. Let $D$ be the natural circular ordering of these vertices by traversing $C$ clockwise. Now as $G$ is planar, $C$ splits the surface into an interior region and an exterior region, by the Jordan curve theorem. So every face is inside either the interior or exterior of $C$. But this means that every face must be preserved in $D$, as the surface we are dealing with is orientable and we can affix an orientation to every face $F$ such that the order of the vertices in the orientation is the same order as the orientation in $D$. Thus every face in this embedding is preserved. 
	\input{"chapters/figures/4-connected-planar-graph with cycle"}
\end{proof}
\todo{complete proof}

We wish to prove the following:

\begin{conjecture}\label{conj:4-planar graphs}
	For all graphs $G$ embedded on a surface $\Sigma$ of genus $g$, where there are $p$ distinguished faces on $G$, we have that $G$ can be embedded on $\leq f(g, p)$ pages.
\end{conjecture}

We first do the planar-nonplanar decomposition of $G$ first. 

Then we look at the planar subgraph $G_p$ and decompose $G_p$ into 4-connected components with adhesion at most 3.

From Tutte's theorem on planar graphs \cite{tutteTheoremPlanarGraphs1956}, we have that if $G$ is a 4-connected planar graph, then the vertex ordering of the Hamiltonian cycle $(\leq)$ preserves all faces on $G$.

In Robert's proof, we have to move three vertices to the start of the decomposition. This will be a problem, but we claim that for every distinguished face $F$ which touches these vertices, the number of pages needed to embed $F$ is bounded. 

\subsection{Graphs of bounded genus}

\begin{lemma}
	Suppose $G$ is a graph with components $G_0$ and $G'$. Suppose $G_0$ is embedded on a surface $\Sigma$ of genus $g$ and let $F$ be a face on $G_0$. Let $v_1, v_2, ..., v_k$ be the vertices bordering $F$, and let $C$ be the cycle bordering $F$. Let $D$ be a $d$-clean disk on $F$. Now suppose $G'$ is a vortex on $D$ with a path-decomposition $(B_0, ... B_l)$. Suppose $G_0$ has a book-embedding $(<, \phi)$. Then partition the edges $e_i = v_i v_{i + 1}$ (modulo $k$) such that the edges form a maximal $\phi$-monochromatic path on $C$. Suppose there are $m$ paths (alternatively, the number of transitions is $m$ ). Then $G$ has a book-embedding with $pn(G) + f(m)$ pages.
\end{lemma}
We shall prove this auxillary lemma for each $l$. 
\begin{lemma}
	Let $(B_1, ..., B_n)$ be a path-decomposition of $G$ with path-width $k$. Let $x_1, ..., x_n$ be vertices in $G$ such that $x_i \in B_i$ for all $i$. Then we have that for any one-page embedding of $x_1, ..., x_n$, $G$ has a $k + 1$-page embedding. 
\end{lemma}

Now we will use the auxillary lemma to prove the above lemma.
\begin{proof}
	Suppose $G$ has this structure above, and every edge $e$ has a colour $\phi(e)$. 
	Now we give every vertex in $B_i$ the colour $(b(v),  \phi(e))$. Then we have that when we have the book-embedding, the monochromatic $v_i$ form a 1-page book embedding. Therefore, we use the lemma above to form a bounded book-embedding, which is what we need. However, this relies on this unproven conjecture.
\end{proof}

\begin{theorem}
	For all graphs $G$ of genus $g$, for all faces $F_1, ..., F_k$ of $G$, there exists a $f(g, k)$-page book embedding such that $F_i$ has $\leq h(g, k)$ transition points. 
\end{theorem}

For faces in the planar decomposition, this is from applying Robert's theorem and Tutte's theorem. As we only move at most three vertices in every face, the vortex $G_i$ on the face $F$ has a bounded increase in the number of pages needed to move around. 

\subsection{Non-planar decomposition}
Now consider $E_n$, the edges not in the planar decomposition. Consider $F_n$, the faces $F$ such that there exists an edge $e$ in $E_n$ which bounds $F$. Then we claim that the number of edges that bound $F$ is bounded. We first need an auxillary topological conjecture. We say a loop is \textit{trivial} if it is homotopy equivalent to a constant loop, and \textit{nontrivial} if this is not the case. A \textit{facial walk} is a sequence of edges $e_1, ..., e_n$ that bound a face such that $e_i$ is incident to $e_{i + 1}$ modulo $n$ for all $i$. The length of the facial walk is $n$. 

\begin{lemma}
	Let $\Sigma$ be a surface of Euler genus $g$ and let $x_0$ be a point on the surface. Then let $L$ be nontrivial loops that start and end at $x_0$ on $\Sigma$. Then let $F_1, ..., F_j$ be faces on $\Sigma + L$, such that each $F_i$ is homeomorphic to a disk. Then the length of the facial walk for all $F$ is at most $2g$. 
\end{lemma}

Parts of the proof was motivated by a discussion with Corbin Reid. 

\begin{proof}
	Let $G'$ be the dual graph of this graph, where the vertex set is $F_1, ..., F_j$ and the edge set are the loops, where two faces are incident if there is a loop that is touching both faces. Then take a spanning tree $T$ of $G'$, and let $L'$ be the loops that are not in $T$ and $L''$ be the loops that are in $T$. 
	Then consider embedding $L'$ on the surface $\Sigma$. Now this bounds a disk, call it $F_0$. This is a disk as we glue together $F_1, ..., F_j$ according to the spanning tree $T$, meaning that we maintain the property that the surface is contractible.
	Now we have that there is one vertex, $|L'|$ edges, and one face $F_0$. Then by Euler's formula, we have that:
	\begin{equation}
		n - m + f = 2 - g
	\end{equation}
	therefore, we have that $1 - |L'| + 1 = 2 - g$, or that $|L'| = g$. As every edge is traversed twice on either side on the facial walk, we have that the length of the facial walk is $2g$. 
	Now let us add each edge from $L''$, one at a time. We shall show that every face after adding all edges from $L''$ to $L'$ has a facial walk length of $\leq 2g$. 
	
	Before adding any edge, we have $F_0$ has $\leq 2g$ on the facial walk. Now suppose we have added loops $\gamma_1, ..., \gamma_{i - 1}$, and suppose every face has a facial walk length of $\leq 2g$. 
	Then suppose loop $\gamma_i$ splits face $F$ into faces $H$ and $J$. We have that the facial walk length $|H|$ and $|J|$ are at least 2, but bounded above by the facial walk length $F + 2$. Then we have that: 
	\begin{equation}
		2 + |J| \leq |H| + |J| \leq |F| + 2 \leq 2g + 2
	\end{equation}
	so $|J| \leq 2g$ and by symmetry so does $|H|$. Thus shown.
\end{proof}

\begin{corollary}
	From the theorem above, every nonplanar face alternates between having a planar trace and nonplanar edge at most $4g$ times. Therefore, we are bounded.
\end{corollary}

By the theorem above, then we have that the planar-nonplanar decomposition has a bounded number of pages. Therefore, we are done for orientable surfaces. 