
\chapter{Potential proof techniques}\label{chap:Proving_The_Theorem}
Our aim is to show that graphs with bounded genus $g$ containing $p$ vortices of bounded width $k$ have bounded pagenumber $f(g, p, k)$. Thus we can show that for fixed $t$, all $K_t$-minor free graphs have bounded pagenumber $f(g, p, k, a, \ell)$. However, from \cite{hickingbothamStackNumberCliqueSum2023}, we can show that the clique-sums of graphs of bounded genus also have bounded genus.

We wish to find a book-embedding of a graph $G$ of bounded genus $g$ with vortices $G_1, ..., G_p$ of adhesion $k$ such that the pagenumber of $G$ is at most $f(g, p, k)$ for some constants $g$ and $p$. It is trivial to show that apex vertices only increase the number of pages by $f(a)$ for a fixed function $f$. 
\subsection{Apex vertices}
In this section, we will prove that apex vertices can be added with a bounded increase to the number of pages.
\begin{theorem}
	If $G$ is a graph with partition $(G', A)$ such that $G'$ is a graph with pagenumber $s$, $|V(A)| \leq a$, then $G$ has pagenumber $s + \left\lceil \frac{3a}{2}\right\rceil$. 
\end{theorem}
\begin{proof}
	Let $G'$ have book-embedding $(<, \rho)$. Then place the vertices of $A$ at the very start of $(<)$ and for every edge $u_iv$, $u_i \in A$, $v \in G'$, we colour $\rho(uv) = i$. Then for any edge $e \in E(G')$, we maintain the same colour as before. Then for the edges between vertices in $A$, we have that the number of colours is bounded above by $\left\lceil \frac{a}{2} \right\rceil$ from \cref{thm:Pagenumber_Complete_Graph}. Therefore, we have that $\pn(G) \leq \pn(G') + a + \left\lceil \frac{a}{2} \right\rceil =s + \left\lceil \frac{3a}{2}\right\rceil$. 
\end{proof}

\section{Planar graphs}
From Heath and Istrail, we can form a planar-nonplanar decomposition of $G$ of bounded genus $g$. Therefore, it makes sense to think about planar graphs first before thinking about graphs with bounded genus $g$.
Let $G$ be a graph. We say that a vertex ordering $<$ preserves a face $F$ if there is a vertex $v_0$ on the boundary of $F$ and an orientation $(v_0, v_1, ..., v_k)$ around the boundary of $F$ such that $v_0 < v_1 < ... < v_k$. 
As a consequence of Tutte's theorem, we have that every 4-connected planar graph is Hamiltonian and also every face of the embedding is preserved. \todo{prove this!}

We wish to prove the following:

\begin{conjecture}\label{conj:4-planar graphs}
	For all graphs $G$ embedded on a surface $\Sigma$ of genus $g$, where there are $p$ distinguished faces on $G$, we have that $G$ can be embedded on $\leq f(g, p)$ pages.
\end{conjecture}

We first do the planar-nonplanar decomposition of $G$ first. 

Then we look at the planar subgraph $G_p$ and decompose $G_p$ into 4-connected components with adhesion at most 3. We form Hamiltonian cycles which preserve every face on the embedding.
In Robert's proof, we have to move three vertices to the start of the decomposition. This will be a problem, but we claim that for every distinguished face $F$ which touches these vertices, the number of pages needed to embed $F$ is bounded because of a lemma below. 

\subsection{Graphs of bounded genus}

\begin{lemma}
	Suppose $G$ is a graph with components $G_0$ and $G'$. Suppose $G_0$ is embedded on a surface $\Sigma$ of genus $g$ and let $F$ be a face on $G_0$. Let $v_1, v_2, ..., v_k$ be the vertices bordering $F$, and let $C$ be the cycle bordering $F$. Let $D$ be a $d$-clean disk on $F$. Now suppose $G'$ is a vortex on $D$ with a path-decomposition $(B_0, ... B_l)$. Suppose $G_0$ has a book-embedding $(<, \phi)$. Then partition the edges $e_i = v_i v_{i + 1}$ (modulo $k$) such that the edges form a maximal $\phi$-monochromatic path on $C$. Suppose there are $m$ paths (alternatively, the number of transitions is $m$ ). Then $G$ has a book-embedding with $pn(G) + f(m)$ pages.
\end{lemma}
We shall prove this auxillary lemma for each $l$. 
\begin{lemma}
	Let $(B_1, ..., B_n)$ be a path-decomposition of $G$ with path-width $k$. Let $x_1, ..., x_n$ be vertices in $G$ such that $x_i \in B_i$ for all $i$. Then we have that for any one-page embedding of $x_1, ..., x_n$, $G$ has a $k + 1$-page embedding. 
\end{lemma}

\begin{proof}
	For each $v \in V(G)$, let $b(v) := \min \left\{ i : v \in B_i \right\}$. This partitions $V(G)$. Let $(\leq)$ be the one-page embedding of $x_1, ..., x_n$ and suppose we have a circular ordering. Colour the vertices such that every vertex in each bag is assigned a different colour. Now for all edges $uv$ in $E(G)$ where $b(u) \leq b(v) $, set $\phi(uv) = col(u)$. 
	
	Now for all $i$, place $b^{-1}(i)$ immediately clockwise from $x_i$. We claim that this is a book-embedding. 
	
	Suppose edges $uv$ and $xy$ are assigned the same colour, $u < v$, $u \leq x$ and $x \leq y$. Suppose $h(v) = i$. Then we can draw a line such that $B_i, B_1$ and $x_1$ is on one side and $B_{i + 1}, B_{i + 2}, ..., B_k$ from the fact that $x_1$ form a planar partition of the graph $G$. This is possible from the Jordan curve theorem and the fact that the line that goes through $x_1 ... x_k$ is planar. Then this implies that $uv$ and $xy$ do not cross. 
	If $u = x$, then we can follow the path $P$ from $x_1$ to $x_k$ to reach $u$ with no crossings. (Draw picture!)
\end{proof}

Now we will use the auxillary lemma to prove the above lemma.
\begin{proof}
	Suppose $G$ has this structure above, and every edge $e$ has a colour $\phi(e)$. 
	Now we give every vertex in $B_i$ the colour $(b(v),  \phi(e))$. Then we have that when we have the book-embedding, the monochromatic $v_i$ form a 1-page book embedding. Therefore, we use the lemma above to form a bounded book-embedding, which is what we need. However, this relies on this unproven conjecture.
\end{proof}

\begin{conjecture}
	For all graphs $G$ of genus $g$, for all faces $F_1, ..., F_k$ of $G$, there exists a $f(g, k)$-page book embedding such that $F_i$ has $\leq h(g, k)$ transition points. 
\end{conjecture}

\subsection{Non-planar decomposition}
Now consider $E_n$, the edges not in the planar decomposition. Consider $F_n$, the faces $F$ such that there exists an edge $e$ in $E_n$ which bounds $F$. Then we claim that the number of edges that bound $F$ is bounded. We first need an auxillary topological conjecture. 

\begin{conjecture}
	Let $\Sigma$ be a surface and let $x_0$ be a point on the surface. Then let $\gamma_1, ..., \gamma_k$ be loops that start and end at $x_0$ which are nontrivial. Then let $F_1, ..., F_j$ be faces on $\Sigma$ derived from the loops, meaning that $\Sigma - \gamma_1, ..., \gamma_k$ has path-connected components $F_1, ..., F_j$. Then for all $i$, $F_i$ has at most $f(g)$ loops on its boundary.
\end{conjecture}

If we can prove this conjecture, then we can prove the above theorem. 


