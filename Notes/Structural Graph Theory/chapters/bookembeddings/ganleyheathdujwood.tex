% !TEX root = ./thesis.tex

\section{Bounded treewidth and page number}\label{sec:Bounded_Treewidth}
\begin{theorem}[\textcite{ganleyPagenumberTrees2001}]\label{thm:bded_treewidth_bded_pagenumber}
	Every graph \(G\) with \(\tw(G) \leq k\) can be embedded on $k + 1$ pages.
\end{theorem}
\textcite{ganleyPagenumberTrees2001} considered the case where \(G\) is a \(k\)-tree. $k$-trees are not used in this proof, and a tree-decomposition of \(G\) is used instead. 

Let $G$ be a graph. Recall a clique is a subgraph that is a complete graph. The clique-number $\omega(G)$ is the largest clique in $G$. A \textit{perfect} graph $G$ is one where $\chi(G) = \omega(G)$. An \textit{intersection} graph $G$ of a tree $T$ with a set of subtrees $\mathcal{T}$ has vertex set $V(G) = \mathcal{T}$ and edge set $T_u T_v \in E(G)$ if and only if $T_u$ and $T_v$ have a vertex in common. Intersection graphs are perfect graphs. 

\begin{proof}
	Consider a tree-decomposition of \(G\), $\{B_x: x \in V(T)\}$. Perform a depth-first search on \(T\), starting at an arbitrary root node \(r\). Let \(\sigma(v)\) of a vertex \(v\) in \(V(G)\) be the first time \(v \in B_x\), $x \in T$ appears in the depth-first search. $\sigma(v)$ determines when $v$ appears in the book-embedding. Within each bag $B_x$, if two vertices $v,w$ first appear in $B_x$, then $v$ and $w$ are ordered arbitrarily in the book-embedding. Now consider the subtree \(T_v\) induced by the bags \(B_x\) containing \(v\). Let \(H\) be the intersection graph of the subtrees. \(V(H) = \lbrace T_v : v \in G \rbrace\) and \(T_u T_v \in E(H)\) if there exists a common bag containing both $u$ and $v$. \(H\) is perfect as all intersection graphs of subtrees are perfect. Therefore, \(\chi(H) = \omega(H)\). Then as \(\tw(G) \leq k + 1\), then the size of a clique in \(H\) is at most \(k + 1\). Thus \(H\) is \(k + 1\)-colourable. Let $z: V(H) \rightarrow [k + 1]$ be a proper colouring of $H$. 
	\paragraph{}
	Then we assign the edges of \(G\) a colour. Colour each edge \(uv \in E(G)\) as follows:
	\begin{equation}
		c(uv) =
		\begin{cases}
			z(T_u) & \text{ if } \sigma(u) \leq \sigma(v), \\
			z(T_v) & \text{ if } \sigma(v) \leq \sigma(u)
		\end{cases}
	\end{equation}
	This is a proper book-embedding of \(G\). Suppose that edges \(uv\), \(xy\) cross, so \(\sigma(u) \leq \sigma(x) \leq \sigma(v) \leq \sigma(y)\). However, this implies that there exists a bag \(B\) such that \(u, x, v \in B\), as \(uv\) is an edge in \(B\) and we do a depth-first search to establish the ordering of $\sigma(u)$. So \(u, x, v\) they are in the same bags. However, this implies that the trees \(T_u\) and \(T_x\) intersect, meaning that \(c(uv) \neq c(xy)\). Therefore, all crossing edges are assigned different colours. Finally, the number of pages used is \(\chi(H) = k + 1\), so \(\pn(G) \leq k + 1\). Thus shown.
\end{proof}
\textcite{dujmovicGraphTreewidthGeometric2007} showed that when $k \leq 2$, all graphs with treewidth $k$ can be embedded on $k$ pages. However, when $k \geq 3$, there exist graphs with treewidth $k$ that cannot be embedded on $k$ pages. Therefore, when $k \geq 3$, the bound from \textcite{ganleyPagenumberTrees2001} is tight. 
For a simpler example, consider the case when $T$ is a path. The proof given below is identical in spirit to \cref{thm:bded_treewidth_bded_pagenumber}, but omits many of the details that are missed. 

\begin{theorem}
	Every graph \(G\) with \(\pw(G) \leq k\) can be embedded on $k + 1$ pages.
\end{theorem}

\begin{proof}[Proof sketch]
	Let $G$ have path-decomposition $\{B_1, B_2, \ldots , B_k\}$. For each $v$, let $\sigma(v)$ be the smallest $i$ where $v \in B_i$. Colour each vertex so that in each bag, no two vertices are assigned the same colour. This is possible to do in $k+1$ colours as interval intersection graphs are perfect. Then colour edges so that edge $uv$ has the same colour as $u$ if $u \leq v$ in the book-embedding. This is a proper book-embedding.
\end{proof}