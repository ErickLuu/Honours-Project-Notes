\section{Bounds of pagenumber of graphs}\label{sec:BoundedPagenumber}
\subsection{Tree-decomposition into bounded page number torsos}\label{ssec:Clique_sum_Pagenumber_bound}

This proof has been adapted into the language of tree-decompositions.
\begin{theorem}[\textcite{hickingbothamStackNumberCliqueSum2023}]\label{thm:clique_sum_pagenumber_bound}
	Let \(G\) be a graph with a tree-decomposition \((B_x: x \in V(T))\). Suppose every torso \(G \langle B_x \rangle\) has pagenumber \(\leq s\) and is \(c\)-colourable. Further suppose the adhesion of the tree-decomposition is at most \(\ell\).
	Then $G$ can be embedded in \(\leq cs + \ell \) pages.
\end{theorem}

\subsubsection{Proof of above theorem}
This proof involves gluing torsos along cliques of size at most \( \ell \).

Let \(C\) be a clique in \(G\) and let \(\sigma_C = (u_1, \ldots , u_k)\) be a vertex ordering of \(V(C)\), and let \(C \leq \ell \). Let $J$ be a clique in $G$. A vertex $v$ is rainbow in $J$ in a book-embedding $(<, \psi)$ if the set of edges $\{u_i v | u_i < v, u_i \in J\}$ each have distinct colours. The structure of the book-embedding will look like this: \((\underbrace{u_1, u_2, \ldots, u_k}_{\text{Vertices in } C}, \underbrace{v_1, v_2, \ldots, v_l}_{\text{Vertices not in }C})\).

To prove this theorem, we use a common technique in graph theory. We strengthen the lemma so that we may use induction to prove the statement.
\begin{lemma}\label{lem:Hickingbotham_Lemma}
	Let \(G\) be a graph where \(\pn(G) \leq s\) and \(\chi(G) \leq c\), and a clique \(C\) with an ordering \(\sigma_C\). Let \(|C| \leq \ell\). There exists a \(cs + \ell\)-page layout \((\leq, \psi)\) of \(G\) where:
	\begin{enumerate}
		\item The vertex ordering has the structure \((\underbrace{u_1, u_2, \ldots, u_k}_{\text{Vertices in } C}, \underbrace{v_1, v_2, \ldots, v_l}_{\text{Vertices not in }C})\).
		\item For every \(u \in V(C)\), the edges \(\lbrace uv \in E(G) : u \leq v \rbrace\) are a monochromatic star.
		\item For every clique \(J\), the last vertex of \(J\) is a rainbow-vertex.
	\end{enumerate}
\end{lemma}
\begin{proof}
	Let \((\leq_a, \psi_a)\) be a \(s\)-page layout of \(G\) and let \(\rho: V(G) \rightarrow [c]\) be a proper colouring of \(V(G)\).

	Let \(u_1, \ldots, u_k\) be the vertices of \(C\) ordered by \(\sigma_C\). Note that \(k \leq \ell\). Then the new ordering starts with \(u_1 \leq u_2 \leq \ldots, \leq u_k\), and all vertices not in \(K\) are placed after, according to \(\leq_a\).
	The edge-colouring \(\psi\) is defined as follows. For every edge \(u_i v\) where \(u_i \in V(C)\) and \(u_i \leq v\), \(\psi(u_i v) = i\). If neither \(u\) nor \(v\) are in \(V(C)\), and \(u \leq v\), then let \(\psi(uv) = (\rho(u), \psi_a(uv))\). Then this edge-colouring requires \(|\rho| |\psi_a| + k \leq cs + \ell\) pages.

	Now we show \((\leq, \psi)\) is a proper book-embedding. Suppose there exists edges \(uv\) and \(xy\) where \(\psi(uv) = \psi(xy)\). Suppose that \(u\) is the smallest vertex in the ordering \(\leq\). If \(u \in V(C)\), then the edge \(uv\) is assigned the page consisting of only edges adjacent to $u$. So \(x = u\), but this is a star. Therefore, the edges do not cross. Therefore \(u, v, x, y\) are not in \(V(C)\). But \((\leq, \psi)\) restricted to the subgraph $G - C$ looks like \((\leq_a, \psi_a)\), if pages with different colours where identified. Therefore, \((\leq, \psi)\) is a proper book-embedding.  
	\par
	Properties 1 and 2 are immediate from the definition \((\leq, \psi)\). For property 3, consider a clique \(J\) in \(G\). Then we must show the last vertex of \(J\) is rainbow. Suppose the last vertex of \(J\) is \(w\), and let \(u, v\) be earlier vertices. Since \(u\) and \(v\) necessarily are assigned different colours in the colouring, then \(\psi(uw) = (\rho(u), \psi_a(uw))\) and \(\psi(vw) = (\rho(v), \psi_a(vw))\). Therefore, the two edges are on different pages. Thus \(w\) is a rainbow vertex.
\end{proof}

\begin{theorem}[\textcite{hickingbothamStackNumberCliqueSum2023}]
	Suppose a graph \(G\) has a tree-decomposition \((B_x: x \in V(T))\) with torsos \(G \langle B_x \rangle\) and adhesion at most \(\ell\). Order the vertices \(v_0, \ldots, v_k\) in $T$ with respect to a breath-first search. Let $B_i = B_{v_i}$. Suppose that for all torsos $G\langle B_i \rangle$, \(\pn(G\langle B_i \rangle) \leq s\) and \(\chi(G\langle B_i \rangle) \leq c\). Then there is a book-embedding of \(G\) with at most \(cs + \ell\) pages.
\end{theorem}
For a breadth-first search, we maintain the property that for all \(i\), \(T[v_0, \ldots, v_{i}]\) is a tree and \(v_i\) is a leaf in \(T[v_0, \ldots, v_{i}]\).
\begin{proof}
	We prove the stronger statement that there exists a book-embedding with the property that the last vertex of any clique \(J\) is a rainbow vertex. For short, this property is the \textit{rainbow-clique} property. 

	Suppose $G$ has a tree-decomposition consisting of a single torso with the properties above. Then \(G\langle B_0 \rangle\) is a single graph with \(\pn(G) \leq s\). Choose \(C\) to be an arbitrary vertex \(v\) in \(G\langle B_0 \rangle\). Then by \cref{lem:Hickingbotham_Lemma}, there is a book-embedding with pagenumber at most \(cs + 1\) and every last vertex in a clique is a rainbow vertex.

	Suppose $G$ has a tree-decomposition $(B_x: x \in V(T))$ with the properties above. Take a breadth-first search of $T$, with vertex ordering $v_0, \ldots, v_n$. For the induction hypothesis, suppose that the induced subgraph $G' := G[B_0 \cup B_1 \cup \ldots \cup B_{n-1}]$ maintains the rainbow-clique property with pagenumber of at most \(cs + \ell\).  
	Let \(C\) be the adhesion clique between \(G \langle B_n \rangle\) and $G'$. Then let \((\leq_n, \psi_n)\) be the \(cs + \ell\)-pagenumber book-embedding of \(G \langle B_n \rangle\) that starts with \(V(C)\). Let \((\leq_{n-1}, \psi_{n-1})\) be the book-embedding of \(G'\). By the induction hypothesis, \((\leq_{n-1}, \psi_{n-1})\) is a \(cs + \ell\)-page book-embedding with the rainbow-clique property.

	\paragraph{Construction of new book-embedding}
	We construct a new book-embedding \((\leq, \psi)\).
	Let \(w \in V(C)\) be the last vertex of \(C\) with respect to \(\leq_{n-1}\). Then insert \(V(G \langle B_n \rangle) - C\) between \(w\) and its successor in $G'$ with the order of \(\leq_{n-1}\) to make $\leq$. For the page assignment \(\psi\), if \(uv \in E(G')\), then \(\psi(uv) = \psi_{n-1}(uv)\). For edges in $G \langle B_n \rangle$, permute the edge assignments of \(\psi_n\) such that for all \(u \in V(C)\), \(\psi(uv) = \psi_n(uw)\) for $v \in C$ and $u \leq_n v$. This is possible as \(w\) is a rainbow vertex and the edges \(\{uv : v \in C, u \leq_n v\}\) are assigned to a unique page in \(\psi_n\). Finally, let \(\psi(uv) = \psi_n(uv)\) for all remaining edges in $G \langle B_n \rangle$. 
	\paragraph{Proof that this is a valid book-embedding}
	We claim that \((\leq , \psi)\) is a stack layout that satisfies the induction hypothesis. Suppose that \(\psi(uv) = \psi(xy)\). If \(uv, xy \in E(G')\), then by the induction hypothesis, they do not cross. Similarly, if \(uv, xy \in E(G \langle B_n \rangle)\), then they do not cross as well. If \(uv\) is in \(E(G')\) and \(xy \in E(G \langle B_n \rangle)\), then they will go over each other or be sequential and therefore will not cross.
	Finally, if \(u, v, x, y \in C\), then by the induction hypothesis on \(G'\), they do not cross either. The final case is if \(uv \in E(G\langle B_{n} \rangle)\) and \(u \in V(C)\), \(v \notin V(C)\), \(xy \in E(G')\). If \(uv\) and \(xy\) cross, then \(xy\) and \(uw\) will cross. But this will contradict the page-embedding of \(G'\) as $u, w, x, y$ are in $G'$.

	Let \(J\) be a clique in \(G\). Then $J$ is either only in $G'$, only in $G\langle B_n \rangle$, or shares vertices with $C$ This is because $V(G') \cap V(G\langle B_n \rangle) = C$, so $C$ is a separator between $G'$ and $G \langle B_n \rangle$. If $J$ is only in $G'$ or only in $G\langle B_n \rangle$, then the last vertex of $J$ is a rainbow vertex by hypothesis. If $J$ shares vertices with $C$, then the last vertex of $J$ with $\leq$ is in $G\langle B_n \rangle$ by construction. However, this vertex is a rainbow vertex, as the last vertex in $J$ with respect to $\leq$ is the last vertex in $G\langle B_n \rangle$ with respect to $\leq_n$. 
\end{proof}

We have some bounds in terms of pagenumber on some of these constants. However, these bounds are not tight, in the case of planar graphs. 

\begin{lemma}
	Suppose a graph $G$ can be embedded on $s$ pages where $s$ is at least $2$. Then \(G\) does not contain any cliques on more than \(2s+1\) vertices.
\end{lemma}

\begin{proof}
	If \(G\) has a clique $K$ of size \(k\), then embedding $K$ requires at least \(\lceil \frac{k}{2} \rceil\) pages, from \cref{thm:Pagenumber_Complete_Graph}. Therefore, if we can embed \(G\) in \(s\) pages, then every clique in $G$ has at most \(2s + 1\) vertices.
\end{proof}
Therefore, \(\ell \leq 2s + 1\).

Let $G$ be a graph with the properties above. As \(\chi(G) \leq 2 \pn(G) + 2\), from \cref{thm:Colouring_Bound}, there exists a bound that does not depend on the chromatic number or largest clique of \(G\).
\begin{corollary}[\textcite{hickingbothamStackNumberCliqueSum2023}]\label{corr:bded_pn_tree_decomp}
	Let \(G\) be a graph with a tree-decomposition \((B_x: x \in V(T))\) where each torso \(G \langle B_x \rangle\) can be embedded on $s$ pages. Then from \cref{thm:clique_sum_pagenumber_bound}, with $\ell \leq 2s + 1$ and $\chi(G) \leq 2 s + 2$, \(G\) can be embedded on \(2s^2 + 4s + 1\) pages.
\end{corollary}

This section finds a bound on the pagenumber of planar graphs. This upper bound is much worse than the tight upper bound found by Yannakakis \cite{yannakakisEmbeddingPlanarGraphs1989}. 

\section{Decomposing planar graphs}
In this section, we decompose planar graphs into smaller components that are more useful to applying \cref{thm:clique_sum_pagenumber_bound}. We start with Menger's theorem, then move onto proving a decomposition of planar graphs into smaller subgraphs. 

Menger's theorem \cite{mengerZurAllgemeinenKurventheorie1927} is an important theorem which is used throughout this section. 
Let \(G\) be a graph and \(A, B \subseteq V(G)\). An \(AB\)-path is a path in \(G\) which starts in \(A\) and ends in \(B\) with no internal vertices in \(A \cup B\). An \(AB\)-connector is a set of disjoint \(AB\)-paths. An \(AB\)-separator is a set \(S \subseteq V(G)\) such that \(G - S\) contains no \(AB\)-path. Then:
\begin{theorem}[Menger's theorem]\label{thm:Menger}
	Let $G$ be a graph and let $A, B \subseteq V(G)$. Then the size of the smallest \(AB\)-separator of \(G\) is equal to the size of the largest \(AB\)-connector.
\end{theorem}
Now take two distinct vertices \(x, y\). Let \(A = N_G(x) \cup \{x\} \) and \(B = N_G(y) \cup \{y\} \). Then \cref{thm:Menger} implies that:
\begin{theorem}[Menger's theorem, vertex-connectivity version]\label{thm:Menger_Vertex}
	A graph \(G\) is \(k\)-connected if and only if for any two distinct vertices, there are at least \(k\) internally disjoint paths between the two vertices.
\end{theorem}

Let $G$ be a plane graph. A \textit{separating triangle} on $G$ is a triangle $u,v,w$ on $G$ such that the triangle $uvw$ separates $\mathbb{R}^2$ into two components $U, V$ and there are vertices in the interior of $U$ and $V$. All nonseparating triangles bound faces. 
\todo{draw picture!}

\begin{proposition}\label{lem:planar_graphs_4_connected_cliqesums}
	Every planar graph has a tree-decomposition such that each torso is a \(4\)-connected planar graph or $K_{\leq 4}$ with adhesion at most \(3\). Furthermore, adhesion sets between $4$-connected torsos are separating triangles.
\end{proposition}

To prove this proposition, we need the following lemma. A \textit{rooted triangle} on a set $u, v, w$ on a graph $G$ is a model of $K_3$ in $G$ where $u, v, w$ are in separate components. 

\begin{lemma}\label{lem:rooted_triangle}
	Every graph $H$ with vertices $u, v, w$ has a rooted $K_3$ on $u, v, w$ if and only if there exists a cut-set $S$ where $|S| \leq 1$ and $H - S$ places the set $\{u,v,w\} \setminus S$ each in different components. 
\end{lemma}
\todo{prove this!}

This lemma proves the following lemma.

\begin{lemma}\label{thm:cutset_added_edges}
	Suppose $G$ is a $3$-connected planar graph with cut-set $u,v,w$. Then $G + uv, uw, vw$ is planar. 
\end{lemma}

\begin{proof}
	Let $G_1$ and $G_2$ be the two subgraphs that are disconnected in $G - \{u,v,w\}$. Then if $G_1 + \{u,v,w\}$ has a rooted $K_3$ minor on $u,v,w$, then contract this minor to $K_3$ and delete all other vertices in $G_1$. Repeat for $G_2$. Then as for each subgraph $G_i + uv + uw + vw$ is a planar graph, then because $uvw$ is not a separating triangle in $G_i$, then $uvw$ bounds a face in $G_1$ and $G_2$. Draw $G_1$ and $G_2$ on the sphere and stereographically project $G_2$ so that $uvw$ bounds the outerface. Then draw $G_2$ on $uvw$ in $G_1$ for a graph drawing of $G + uv, uw, vw$. 
	
	Suppose $G_2 + \{u,v,w\}$ has no rooted $K_3$ minor. By \cref{lem:rooted_triangle}, $G_2$ has a cut-set $S$, $|S| \leq 1$, where $G_2 - S$ places $u,v,w - S$ in different components. Call the components of $G_2 - S$ that contain $u,v,w$, $G_u, G_v, G_w$ respectively. If $S$ is empty, then $G_2$ has components $G_u, G_v, G_w$. But this implies that $u$ is a cut vertex from $G_u$ to $G$, so $G$ is not $3$-connected. Suppose $S = x$ distinct from $u,v,w$. Without loss of generality, suppose $G_u$ contains a vertex distinct from $u$. Then $x, u$ disconnects $G_u$ from $G$, so there is a cutset of size 2. Now suppose $G_u, G_v, G_w$ only contain $u,v,w$ respectively. Then we can add the edges $uv, uw, vw$ without any crossings around $x$. \todo{draw pictures!}. Suppose without loss of generality that $S = v$. Then suppose without loss of generalisation $G_u$ contains another vertex distinct from $u$. Then $G - u - v$ separates $G_u$ from $G$, which is a small cutset. Finally, suppose $G_u, G_w$ only contain $u$ and $w$ respectively. Then $v$ is a separator, so $G$ is not $3$-connected, or $G - u,v,w$ separates nothing. Then $u,v,w$ is not a separator. 
\end{proof}
\todo{prove this!}

Now to prove the theorem. 
\begin{proof}
	We do induction on the number of vertices of $G$. 
	
	Suppose $|V(G)| \leq 3$. Then $\mathcal{T}$ has nodes $x$, $y$ and edge $xy$, and $B_x = B_y = V(G)$. Then $G \langle B_x \rangle = G \langle B_y \rangle = K_3$. \todo{draw pictures!}
	
	Suppose $|V(G)| = 4$, say $V(G) = \{a,b,c,d\}$.  Then $\mathcal{T}$ has nodes $u, i,j,k,l$ and edges $ui, uj, uk, ul$. Let $B_u = \{a,b,c,d\}$, $B_i = \{b,c,d\}$, $B_j = \{a,c,d\}$, $B_k = \{a,b,d\}$, $B_l = \{a,b,c\}$. Then $G\langle B_u \rangle = K_4$, $G\langle B_i \rangle =G\langle B_j\rangle = G\langle B_k \rangle = G\langle B_l \rangle = K_3$. Then this also satisfies the proposition. 

	Now suppose $G$ is $4$-connected. Then place the entirety of $G$ in a single bag.

	Suppose $G$ is disconnected. Then run this algorithm on every component and add arbitrary edges between each subtree to make a tree-decomposition.

	Suppose $G$ is connected but not $2$-connected, so there exists a cut vertex $v$. Now let $G_1, G_2$ be the two disconnected subgraphs of $G - v$, where both $G_1$ and $G_2$ are nonempty subgraphs. $G_1 + v$ and $G_2 + v$ are planar graphs as they are planar subgraphs. Then $|V(G_i + v)| < |V(G)|$, so use induction on $G_1 + v$ and $G_2 + v$. This yields a tree-decomposition $\tree_1, \tree_2$ with the properties above. Add an edge between a bag $B_1$ in $\tree_1$ containing $v$ and a bag $B_2$ in $\tree_2$ containing $v$. Then the adhesion of these two bags is $1$, so it satisfies the property above. 

	Suppose $G$ is $2$-connected but not $3$-connected, so there exists a cut-set $\{u,v\}$. Let $G_1, G_2$ be the two subgraphs that are separated by $G - \{u, v\}$. Firstly, $G_1' := G_1 + \{u,v\} + uv$ is a planar graph. As $G$ is $2$-connected, there is a cycle of $u$ and $v$, so there either exists an edge $uv$ or there exists a path in $G_2$ from $u$ to $v$. Contracting this path to an edge yields another planar graph $G_1'$ where $uv$ is an edge. Repeating this, $G_2' := G_2 + \{u,v\} + uv$ is also a planar graph. Then use induction on $G_1'$ and $G_2'$ with tree-decomposition $\tree_1, \tree_2$ respectively. As $uv$ is an edge in both subgraphs, there exists a bag $B_1, B_2$ in $\tree_1, \tree_2$ that contains both $u$ and $v$. Add an edge between both bags. Finally, we want that the edge $uv$ is in every torso $G\langle B_x \rangle$ where $u, v \in B_x$. Without loss of generality, suppose $B_x$ was originally in $\tree_1$. If $u$ and $v$ are in a bag $B_x$, then there is a unique path of nodes from $B_x$ to $B_1$ containing $u$ and $v$. This is because of the property that the bags that contain $u$ are connected, and between any two vertices on a tree there is a unique path. Then $B_x$ is adjacent to a bag that contains $u$ and $v$, so $G \langle B_x \rangle$ contains $uv$. 

	To show that $G$ has a graph drawing with $uv$ as an edge, take an embedding of $G_1'$ on the sphere and project this embedding to the plane such that $uv$ is adjacent to the outerface. Do the same for $G_2'$. Then glue these two embeddings along $uv$ both on the outerface. 

	Suppose that $G$ is $3$-connected but not $4$-connected. Then $G$ has a cut-set $u,v,w$. Then let $G' = G + \{uv, uw, vw\}$. Then $G'$ is planar, from \cref{thm:cutset_added_edges}. 
	
	If $uvw$ does not bound a face, then $uvw$ is a separating triangle. Use the Jordan curve theorem to split $\mathbb{R}^2 - uvw$ into two regions, an interior and an exterior. Let $G_1$ be the interior of $uvw$ and the triangle $uvw$ and let $G_2$ be the exterior of $uvw$ and the triangle $uvw$. 

	If $uvw$ bounds a face, take $G_1$ and $G_2$ to be the nonempty separating components of $G - \{u,v,w\}$. Then $G_1' = G_1 + uv + uw + vw$ is a planar graph where $uvw$ bounds a face, and $G_2' = G_2 + uv + uw + vw$ is a planar graph where $uvw$ bounds a face. Embed $G_2'$ on a sphere. Then there is a stereographic projection of $G_2'$ where $uvw$ is the outerface, and place this embedding in $G_1'$. 

	Use induction on $G_1$ and $G_2$ with tree-decompositions $\tree_1$ and $\tree_2$ respectively. As $uvw$ is a clique, there exists a bag $B_1$ and a bag $B_2$ that contains $uvw$. Add an edge between bags $uvw$. However, the edge must propagate in every torso. To show that every torso $G\langle B_x \rangle$ that contains $u$ and $v$ must contain $uv$, we repeat the same argument for the $2$-connected case. As $B_x$ must have a neighbouring bag that contains $u$ and $v$, where this neighbour lies on the unique path from $B_x$ to $B_1$, then $G\langle B_x \rangle$ must contain $u, v$. Repeat this argument for $w$. Then every torso is $4$-connected as no edges are deleted in the torso operation and every torso is planar as adding $uv, uw, vw$ preserves planarity. 

	Since all torso edges are edges which preserve planarity in $G$, then $G$ with all torso edges is also planar. Therefore, every torso of $G$ in this tree-decomposition is planar.
	As this handles every case, then by induction, $G \langle B_x \rangle$ is a 4-connected planar graph or $K_{\leq 4}$. 
\end{proof}
Then use a theorem of Tutte to prove a fact for all $4$-connected planar graphs. 

\begin{theorem}[Tutte\cite{tutteTheoremPlanarGraphs1956}]\label{thm:4-connected_planar_ham_cycle}
	All 4-connected planar graphs are Hamiltonian.
\end{theorem}

As a corollary to \textcite{hickingbothamStackNumberCliqueSum2023}, the pagenumber of planar graphs are bounded.

\begin{theorem}\label{thm:Planar Graph Hickingbotham Bound}
	Let \(G\) be a 2-connected planar graph. Then $G$ can be embedded on $11$ pages, with book-embedding $(<, \rho)$. $<$ restricted to the outer cycle $C$ is $C$. Furthermore, for every face cycle $C$, $<|_{V(C) - \{u, v, w\}} = C - \{u, v, w\}$ for some vertices $u$, $v$, $w$. 
\end{theorem}
\begin{proof}
	From \cref{thm:clique_sum_pagenumber_bound} with tree-decomposition from \cref{lem:planar_graphs_4_connected_cliqesums}, the pagenumber is at most \(2 \cdot 4 + 3 = 11\).

	Furthermore, from the construction given in \cref{lem:planar_graphs_4_connected_cliqesums}, every $4$-connected class are glued on faces. Therefore, every face only changes by $3$ vertices, from \cref{thm:clique_sum_pagenumber_bound}. Therefore removing $3$ vertices from every face preserves the cyclic ordering of every face.
\end{proof}

We will discuss the \(K_5\)-minor free case. If \(G\) is \(K_5\)-minor free, then we can use Wagner's theorem.
\begin{theorem}[Wagner's theorem\cite{wagnerUeberEigenschaftEbenen1937}]\label{thm:WagnersTheorem}
	Let \(G\) be a \(K_5\)-minor-free graph. Then \(G\) has a tree-decomposition of adhesion $\leq 3$ where every torso is either a planar graph or the Wagner graph \(V_8\).
\end{theorem}
A description of the Wagner graph is in \cref{fig:wagner}. The edges are coloured such that the internal edges are on different pages. The spine edges (the edges that are on the outerface) are the ones which can go on any page.
\begin{figure}[h!]
	\centering
	\begin{tikzpicture}[thick,scale=2, every node/.style={scale=2}]
		\tikz \graph [nodes = {draw, circle}, clockwise, empty nodes] {
	subgraph C_n [n=8];
	1 --[red] 5;
	2 -- 6;
	3 -- 7;
	4 -- 8;
};

	\end{tikzpicture}
	\caption[Wagner graph]{The Wagner graph $V_8$. Notice how the clockwise circular ordering of the vertices of the Wagner graph needs 4 pages to embed the graph. }\label{fig:wagner}
\end{figure}

\begin{theorem}
	Let \(G\) be a \(K_5\)-minor free graph. Then \(G\) has pagenumber \(\leq 19\).
\end{theorem}

\begin{proof}
	Suppose \(G\) is \(K_5\)-minor free. Then by Wagner's theorem \cite{wagnerUeberEigenschaftEbenen1937}, \(G\) has a tree-decomposition of adhesion at most 3 where every torso is either a planar graph or the Wagner graph.
	Planar graphs are \(4\)-colourable and can be embedded on four pages. The Wagner graph is \(3\)-colourable and can be embedded on four pages. Therefore, if \(G\) is \(K_5\)-minor free, then \(G\) has pagenumber at most \(4 \cdot 4 + 3 = 19\) from \cref{thm:clique_sum_pagenumber_bound}.
\end{proof}