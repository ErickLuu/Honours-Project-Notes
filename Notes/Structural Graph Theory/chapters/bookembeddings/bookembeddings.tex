\section{Book-Embeddings and Pagenumber}\label{sec:Book Embedding}
A \textit{book} with \(k\) \textit{pages} consists of \(k\) half-planes glued together on a common boundary. The boundary is referred to as a \textit{spine}, and the individual half-planes as \textit{pages}. In topology, these are referred to as \textit{fans} of half-planes.\ \textcite{persingerSubsetsNbooksE31966,atneosenOnedimensionalNleavedContinua1972} described fans in the 1960s.
A \textit{book-embedding} of a graph \(G\) is an embedding of \(G\) on a book. The vertices of \(G\) are placed on the spine, and the edges of $G$ are placed on a single page such that no two edges cross. 
The \textit{pagenumber} of a graph \(G\) is the minimum number of pages required to embed \(G\) on a book. This is also referred to as \textit{book-thickness}, or \textit{stack-number}. An embedding of $K_5$ in three pages is in \cref{fig:book-embedding}.

\begin{figure}[h!]
	\centering
	\includesvg[height = 0.5\textheight]{figures/3page_K5.svg}
	\caption[Three-page book-embedding of $K_5$]{Book-embedding of $K_5$ on three pages. Image by \textcite{eppsteinBookEmbedding2014}}\label{fig:book-embedding}
\end{figure}

Book-embeddings have an equivalent combinatorial definition. A \textit{book embedding} of a graph \(G\) is an arrangement of the vertices of \(G\) in a total ordering \(v_1 < v_2 < \cdots < v_n\). The edges in \(E(G)\) are coloured such that if there are vertices with ordering \(v_a < v_b < v_c < v_d\) and edges \(v_a v_c\) and \(v_b v_d\) in $E(G)$, then $v_a v_c$ and $v_b v_d$ are assigned different colours.
The ordering of \(V(G)\) in the book embedding is denoted as \((<)\) or as \((\leq)\). For a book-embedding \((v_1, v_2, \ldots, v_{|G|})\), the edges \( \left\{ v_1 v_2, v_2 v_3, \ldots, v_{|G| - 1}v_{|G|}, v_{|G|}v_{1} \right\} \) are called \textit{spine edges}.

A circular ordering of a set $S$ is an ordering where the elements of $S$ are in a circle rather than on a straight line. A circular ordering of a book-embedding is equivalent to a linear ordering of a book-embedding. The book-embedding of a circular ordering is exactly the same as for a linear ordering. By choosing a vertex to be at the start of the sequence, there is a way to convert from a circular ordering to a linear ordering. Another book-embedding of $K_5$, this time with a circular embedding, is in \cref{fig:circular_book-embedding}.

\begin{figure}[h!]
	\centering
	\includesvg[width = 0.3\textwidth]{figures/3page_K5_circular.svg}
	\caption[Three-page circular book-embedding of $K_5$]{Another book-embedding of $K_5$ on three pages, this time with a circular book-embedding.}\label{fig:circular_book-embedding}
\end{figure}

Book-embeddings were introduced by \textcite{kainenRecentResultsTopological1974, ollmannBookThicknessVarious1973} in the 1970s. A paper by \textcite{bernhartBookThicknessGraph1979} calculated the book-thickness of complete and bipartite graphs.
\begin{proposition}\label{lem:Pagenumber_1}
	A graph \(G\) can be embedded on a single page if and only if \(G\) is an outerplanar graph.
\end{proposition}
\begin{proof}
	Suppose $G$ is outerplanar, and embedded in $\mathbb{R}^2$. Then select a single vertex $v_0$, and traverse anticlockwise around the outerface to form an ordering. If a vertex $v_i$ appears more than once, then add $v_i$ the first time $v_i$ appears in the traversal. Then this is a one-page book embedding, as when $v_a < v_b < v_c < v_d$ and edges $v_a v_c$, $v_b v_d$ in $G$, then $v_a v_c$ or $v_b v_d$ have to lie on the outerface, breaking the condition that $G$ is outerplanar. This is because either $v_b$ or $v_c$ is not on the outerface. If $G$ has a $1$-page book-embedding, then embedding the page in $\mathbb{R}^2$ through the inclusion map is an outerplanar embedding of $G$. 
\end{proof}
\begin{proposition}\label{lem:Pagenumber_2}
	A graph \(G\) can be embedded on two pages if and only if \(G\) is a subgraph of a planar graph with a Hamiltonian cycle.
\end{proposition}

\begin{proof}
	Suppose $G$ is a subgraph of a planar graph $G'$ with a Hamiltonian cycle $C$. By the Jordan curve theorem, $\mathbb{R}^2 - C$ has two connected components $F_1$ and $F_2$. Choose a vertex $x_0$ and order the vertices with respect to the Hamiltonian cycle $C$ where $x_0$ is the first vertex. Give edges on $C$ colour $1$ or $2$. For all chord edges of $C$ that lie on $F_1$, give these edges colour $1$. For all chord edges of $C$ that lie on $F_2$, give these edges colour $2$. This is a book-embedding of $G'$ on two pages. 
	\par
	Suppose $G$ has pagenumber $2$, and embedded in a book with two pages. Add all remaining spine edges to one of the pages. Then embed the two pages in $\mathbb{R}^2$ through the homeomorphism of two pages to $\mathbb{R}^2$, by flipping one page over. Then this is a planar graph with a Hamiltonian cycle, so $G$ is a subgraph of a graph with a Hamiltonian cycle.
\end{proof}
\subsection{Properties of pagenumber}\label{ssec:Related_Properties}
\cref{lem:Edge_Bound} comes from \textcite{bernhartBookThicknessGraph1979}.
\begin{proposition}\label{lem:Edge_Bound}
	If an \(n\)-vertex graph \(G\) can be embedded on $k$ pages, then \(G\) has at most \(n + k(n-3)\) edges.
\end{proposition}
\begin{proof}
	Given a vertex ordering \(v_1 \leq v_2 \leq \cdots \leq v_n\), the spine edges can appear on any page. Furthermore, at most \(n-3\) non-spine edges appear on each page. The maximum number of edges in an outerplanar graph is \(2n - 3\) from \cref{thm:outerplanar_bound}, but spine edges are counted multiple times. Excluding spine edges, there are $n-3$ edges per page. Adding back on $n$ spine edges, \(m\), the number of edges, satisfies \(m \leq n + k (n - 3)\).
\end{proof}
\begin{proposition}\label{thm:Pagenumber_Complete_Graph}
	The complete graph $K_n$ has pagenumber equal to  $\lceil \frac{n}{2} \rceil$ when $n \geq 4$.
\end{proposition}
\begin{proof}
	$\pn(K_n) \leq \lceil \frac{n}{2} \rceil$: Arrange the vertices of $K_n$ in any circular ordering $v_1 < v_2 < \cdots < v_n$. Then colour edges $v_1 v_2, v_2 v_{n}, v_{n} v_{3}, v_{3} v_{n-1}, \ldots$ in a zigzag pattern. As an example, refer to \cref{fig:k8 coloured with colours} for a description of a zigzagging pattern. Rotate this pattern $\lceil n/2 \rceil$ times. 
	\par
	$\pn(K_n) \geq \lceil \frac{n}{2} \rceil$: Use \cref{lem:Edge_Bound}. \(K_n\) has \(n\) vertices and \(\binom{n}{2}\) edges. Then \(\pn(K_n) \geq \frac{\binom{n}{2} - n}{n - 3} = \frac{n}{2}\) when \(n \geq 4\). As \(\pn(K_n)\) is an integer, take the ceiling of \(\frac{n}{2}\). This concludes the equality.
\end{proof}
\begin{figure}[ht]
	\centering
	\usetikzlibrary{graphs,graphs.standard}

\tikz
	\graph[nodes={circle, draw}] { 
		subgraph K_n [n=8,clockwise,radius=2cm];
		
		{[induced path, edges= red] 1,2,8,3,7,4,6,5},
		{[induced path, edges= blue] 8,1,7,2,6,3,5,4},
		{[induced path, edges= green] 7,8,6,1,5,2,4,3},
		{[induced path, edges= yellow] 6,7,5,8,4,1,3,2},
 };
	\caption[Embedding $K_8$ on four pages]{Circular embedding of \(K_8\) with four pages, the minimum possible.}\label{fig:k8 coloured with colours}
\end{figure}
The proof of \cref{thm:Pagenumber_Complete_Graph} is from \textcite{bernhartBookThicknessGraph1979}
This is an upper bound of any graph \(G\) with \(n\) vertices.
Therefore, for any graph \(G\) on \(n\) vertices, \(n \geq 4\), \(\pn(G) \leq \lceil n/2 \rceil\). The next theorem bounds the chromatic number, from \textcite{bernhartBookThicknessGraph1979}.

The next proof uses degeneracy. A graph $G$ is \textit{$k$-degenerate} if every subgraph $H \subseteq G$ has a vertex of degree at most $k$. 
A simple lemma connecting $k$-degenerate graphs and chromatic number helps prove this theorem. 
\begin{lemma}
	Every $k$-degenerate graph is $k + 1$-colourable.
\end{lemma}
\begin{proof}
	Let $G$ be a $k$-degenerate graph. If $G$ has a single vertex, then $G$ can be coloured in $k + 1$ colours. Now suppose this holds for all $k$-degenerate graphs with $|V(G)|- 1 \geq 1$ vertices. Let $v$ be a vertex in $G$ with degree $k$. Then $G - v$ can be coloured inductively with $k + 1$ colours as $G - v$ is also $k$-degenerate. Now look at the neighbours of $v$ and colour $v$ with a colour that is not used by one of its neighbours. This can be done as $v$ has $n$ neighbours but $n + 1$ colours can be used. 
\end{proof}

This lemma is applied to \cref{thm:Colouring_Bound}.

\begin{proposition}\label{thm:Colouring_Bound}
	For all graphs \(G\), \(\chi(G) \leq 2 \pn(G) + 2\).
\end{proposition}
\begin{proof}
	Let \(\pn(G) = k\) and suppose \(G\) has \(n\) vertices and \(m\) edges. Then the average degree of \(G\), \(d(G) = 2m/n\) by the handshaking lemma. So \(2\frac{m}{n} \leq 2 \frac{n + k(n-3)}{n} = 2 + 2k \frac{n-3}{n} < 2k + 2\). But this implies that \(G\) has a vertex of degree \(\leq 2k + 1\), and as if \(G'\) is a subgraph of \(G\), then \(G'\) also has \(\pn(G') \leq k\), thus \(G'\) has a vertex of degree at most \(2k + 1\). However, this implies \(G\) is \((2k + 1)\)-degenerate, thus \(\chi(G) \leq 2k + 2\).
\end{proof}

Let $G$ be a graph. A \textit{subdivision} of an edge $uv \in E(G)$ deletes $uv$ and adds a new vertex $w$ with edges $uw$ and $wv$. A graph subdivision of $G$ is to do this for all edges in $G$. A $k$-subdivision of $G$ is to subdivide each edge $k$ times in $G$, so the edge $e$ is replaced with a path $P$ of length $k$.\ \textcite{atneosenEmbeddabilityCompactaNbooks} proved that all graphs can be subdivided a finite number of times such the subdivision has pagenumber 3.\ \textcite{dujmovicLayoutsGraphSubdivisions2005} showed that the number of subdivision necessary is $O(\log\pn(G))$. Subdivisions of graphs can also show that the chromatic number can remain constant but the pagenumber of the graph can be arbitrarily large.

\begin{theorem}
	There exists a family of 2-colourable graphs with arbitrarily large pagenumber.
\end{theorem}
\begin{proof}
	Let $G_n$ be the complete graph $K_n$ with every edge subdivided once. Then $G_n$ is bipartite, so is $2$-colourable. However, from \textcite{eppsteinSeparatingThicknessGeometric2002}, for every $t$ there exists an $n$ such that $G_n$ cannot be embedded in $t-1$ pages. This proof comes from \textit{Ramsey theory}, and explicit values of $n$ are difficult to find. 
\end{proof}

However, the class of graphs with pagenumber $\leq p$ is not a minor-closed class.
This does not say that $\pn(H) \leq \pn(G)$ when $H$ is a minor of $G$. Subdivide $K_n$ $n$ times. From \textcite{atneosenEmbeddabilityCompactaNbooks}, the subdivision of $K_n$ can be embedded on three pages. But $K_n$ is a minor of its subdivision, and from \cref{thm:Pagenumber_Complete_Graph}, $\pn(K_n) = \lceil \frac{n}{2} \rceil$. Therefore, pagenumber is not a minor-closed property. Proving \cref{lem:Minor-Closed_Pagenumber} will imply that linklessly embeddable graphs or knotlessly embeddable graphs have bounded pagenumber, which is difficult to prove directly. 

An \textit{expander graph} is a sparse, highly connected graph. Expander graphs share many properties with random graphs, but are constructed explicitly. One type of expander graph is a \textit{bipartite \varepsilon-expander}, where $\varepsilon \in (0, 1]$. A graph $G$ is a bipartite \varepsilon-expander if there exists a bipartition $ \{A, B\}$ of $V(G)$ such that $|A| = |B|$ and for all subsets $S \subset A$ where $|S| \leq \frac{|A|}{2}$, $|N(S)| \geq (1 + \varepsilon) |S|$. 
\textcite{dujmovicLayoutsExpanderGraphs2016} showed that every bipartite \varepsilon-expander graphs can be embedded in 3 pages. 


Book-embeddings of graphs were has applications in VLSI and processor designs, bioinformatics by \textcite{haslingerRNAStructuresPseudoknots1999}, and in graph drawings by \textcite{woodBoundedDegreeBook2002}. 
The project of finding upper and lower bounds of the pagenumber of planar graphs was started by \textcite{bernhartBookThicknessGraph1979} when they conjectured that planar graphs had unbounded pagenumber. However, \textcite{bussPagenumberPlanarGraphs1984} showed that all graphs could be embedded in nine pages, and \textcite{heathEmbeddingPlanarGraphs1984} brought down the number of needed pages to seven.\ \textcite{yannakakisEmbeddingPlanarGraphs1989} devised an algorithm to embed a graph in four pages. Yannakakis, in this paper, claimed that there exists planar graphs which cannot be embedded in three pages. However, his proof was incomplete, and the full proof was left unpublished. In 2020, \textcite{yannakakisPlanarGraphsThat2020} published his full proof. At around the same time, \textcite{kaufmannFourPagesAre2020} also proved that there exists a planar graph requiring four pages.

\textcite{malitzGraphsEdgesHave1994} proved that any graph with $e$ edges has pagenumber $O(\sqrt{e})$. Additionally, he proved that random $d$-regular graphs $G$ with $n$ vertices have the property that $\pn(G) \in \Omega(\sqrt{d} n^{1/2 - 1/d})$. For random 3-regular graphs $G$ with $n$ vertices, $\pn(G) \in \Omega(n^{1/6})$. These constructions of $\Omega(n^d)$ pagenumber graphs are not explicit.\ \textcite{eppsteinThreeDimensionalGraphProducts2024} showed that $\pn(P_n \boxtimes P_n \boxtimes P_n) \in \Theta(n^{1/3})$. This is an explicit construction of a graph with pagenumber in $\Theta(n^{d})$. 