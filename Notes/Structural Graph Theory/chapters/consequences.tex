\chapter{consequences}\label{chap:consequences}

We will finish by discussing some consequences of the above theorem. 
We say a family of graphs $\mathcal{F}$ is nontrivial if it is not the empty set of graphs or all graphs. 

\begin{lemma}\label{lem:Minor-Closed_Pagenumber}
    Let $\mathcal{F}$ be a nontrivial, minor-closed graph family. Then $\mathcal{F}$ has bounded pagenumber.
\end{lemma}
\begin{proof}
    From \textcite{robertsonGraphMinorsXX2004}, we have that every minor-closed graph family has a finite forbidden minor characterisation. Let $\mathcal{H}$ be the finite forbidden minor characterisation of $\mathcal{F}$. Let $H \in \mathcal{H}$ be the forbidden minor with the largest number of vertices, say $|V(H)| = \ell$. Then $\mathcal{F}$ is als $K_\ell$-minor free, as if $K_\ell$ appears as a minor in a graph in $\mathcal{F}$, then the subgraph $H$ also appears in $F$. As $H$ is the largest forbidden minor, all other graphs in $\mathcal{H}$ are also minors of $K_\ell$. Therefore, every graph $G$ in $\mathcal{F}$ is $K_\ell$-minor free. Thus $\pn(G) \leq f(\ell)$. Thus shown.
\end{proof}

This does not say that $\pn(H) \leq \pn(G)$ when $H$ is a minor of $G$. We may subdivide a graph $G$ such that every edge has $O(|V(G)|)$ vertices to make $G'$, and we have that $G'$ has pagenumber at most 3. Then $G$ is a minor of $G'$ but $G$ has unbounded pagenumber.

\subsection{Similarities with Blankenship's PhD}
This is an apt time to discuss the similarity of the proof given above with Blankenship's PhD work. 
Blankenship also uses \textcite{heathPagenumberGenusGraphs1992} to do a planar-nonplanar decomposition. Apex vertices are handled the same. However, Blankenship deals with vortices very differently. They use a ``cap edges'' solution to deal with vortices to embed a graph. This is much different to our approach using monochromatic paths on vortices, and using a tree-decomposition. 
Blankenship also uses a similar theme of having some vertices being moved to the front of a book-embedding, with extra pages needed. However, her lemma was much simpler than Robert's theorem. 