\section{Consequences}
We will finish by discussing some consequences of the above theorem. 
We say a family of graphs $\mathcal{F}$ is \textit{proper} if $\mathcal{F}$ is not the set of all graphs. 

\begin{lemma}\label{lem:minor-closed-Kt}
    Every proper minor-closed graph family $\mathcal{F}$ has a fixed $t$ such that $\mathcal{F}$ is $K_t$-minor free. 
\end{lemma}

\begin{proof}
    From \textcite{robertsonGraphMinorsXX2004} Graph Minor Theorem, we have that every proper minor-closed graph family has a finite forbidden minor characterisation. Let $\mathcal{H}$ be the finite forbidden minor characterisation of $\mathcal{F}$. Let $H \in \mathcal{H}$ be the forbidden minor with the largest number of vertices, say $|V(H)| = t$. Then $\mathcal{F}$ is also $K_t$-minor free, as if $K_t$ appears as a minor in $G$ in $\mathcal{F}$, then the subgraph $H$ also appears in $G$. As $H$ is the largest forbidden minor, all other graphs in $\mathcal{H}$ are also minors of $K_t$. Therefore, every graph $G$ in $\mathcal{F}$ is $K_t$-minor free.
\end{proof}

\begin{lemma}\label{lem:Minor-Closed_Pagenumber}
    Every proper minor-closed graph family has bounded pagenumber.
\end{lemma}
\begin{proof}
    From \cref{lem:minor-closed-Kt}, every proper minor-closed graph family is also $K_t$-minor free. Therefore, $\pn(G) \leq f(\ell)$.
\end{proof}

This does not say that $\pn(H) \leq \pn(G)$ when $H$ is a minor of $G$. We may subdivide a graph $G$ such that every edge has $O(|V(G)|)$ vertices to make $G'$, and we have that $G'$ has pagenumber at most 3. Then $G$ is a minor of $G'$ but $G$ has unbounded pagenumber.
\todo{reference this!}

\subsection{Similarities with Blankenship's PhD}
This is an apt time to discuss the similarity of the proof given above with Blankenship's PhD work. 
Blankenship also uses \textcite{heathPagenumberGenusGraphs1992} to do a planar-nonplanar decomposition. Apex vertices are handled the same. However, Blankenship deals with vortices very differently. They use a ``cap edges'' solution to deal with vortices to embed a graph. This is much different to our approach using monochromatic paths on vortices, and using a tree-decomposition. 
Blankenship also uses a similar theme of having some vertices being moved to the front of a book-embedding, with extra pages needed. However, her lemma was much simpler than Robert's theorem. 