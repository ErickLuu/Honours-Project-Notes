\section{Layout of report}
The following list is how the rest of the report is laid out. 
\begin{itemize}
	\item \cref{sec:background} gives an overview of some basic graph-theoretic definitions. We discuss some basic terminology of graphs. We also discuss graphs embedded on the plane and on surfaces. Additionally, we also discuss the important topics of book-embeddings and graph minors and how they fit to prove \cref{conj:bded_had_pn}. 
	\item \cref{chap:gmst} discusses every component of the Graph Minor Structure Theorem and how these components combine to form the Graph Minor Structure Theorem. The components that are discussed in detail are treewidth and graphs on surfaces. We also discuss the Graph Minor Theorem. 
	\item \cref{chap:book-embeddings} discusses book-embeddings and book-embeddings of graphs of bounded treewidth. There is also a discussion of book-embeddings of graphs with a tree-decomposition with torsos of bounded pagenumber. The paper that is discussed in this section is by \textcite{hickingbothamStackNumberCliqueSum2023} and by \textcite{ganleyPagenumberTrees2001}. There is no original research in this section but much of the technology in both papers is used. 
	\item \cref{chap:orientable} discusses graphs embedded on orientable surface and a book-embedding of graphs on orientable surfaces. Additionally, graphs on orientable surfaces with vortices is also discussed with a theorem by \textcite{heathPagenumberGenusGraphs1992} that allows vortices to be added to graphs embedded on a surface with a bounded pagenumber.  We extend the results in \textcite{heathPagenumberGenusGraphs1992} to include graphs embedded on orientable surfaces with vortices attached. 
	\item \cref{chap:nonorientable} discusses graphs embedded on nonorientable surfaces. There is a discussion of a proof by \textcite{nakamotoBookEmbeddingProjectiveplanar2015} with embedding projective planar graphs with a bounded number of pages. We extend this result to graphs embedded on projective planes with vortices attached. There is also a discussion of the Klein Bottle case and some discussion on its difficulty. Finally, we discuss a conjecture involving non-orientable surfaces that can be used to embed any $K_t$-minor free graph on a bounded number of pages. 
	\item \cref{chap:conclusion} discusses some consequences of the conjecture if it was proven, and discusses Blankenship's PhD in more detail. There are many similarities between the concepts that she used and the ones used in this thesis, although we did not read her thesis in our proof. This chapter finishes with a conclusion to the thesis. 
\end{itemize}

Readers are expected to have at least an undergraduate understanding in graph theory and point-set topology. 
