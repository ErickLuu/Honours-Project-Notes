The following list is how the rest of the report is laid out. 
\begin{itemize}
	\item \cref{chap:Definitions} contains definitions and concepts that will be used throughout the rest of the report. Some of these concepts are part of any undergraduate graph theory unit. Some other concepts, like book-embeddings and treewidth, are unlikely to appear in an undergraduate graph theory unit.
	\item \cref{chap:Known results} discusses some known results from graph theory, including the Graph Minor Structure Theorem. We discuss some proofs related to bounded pagenumber that can be used to prove \cref{conj:bded_had_pn}. The results we discuss are the Graph Minor Structure Theorem itself, a result from \textcite{heathPagenumberGenusGraphs1992}, a result from \textcite{ganleyPagenumberTrees2001} and a result from \textcite{hickingbothamStackNumberCliqueSum2023}. \cref{chap:Definitions} and \cref{chap:Known results} form the literature review section of the report.

	\item \cref{chap:Proving_The_Theorem} is an attempt at a proof to \cref{conj:bded_had_pn}. The main bulk of the argument is showing that the construction given by the Graph Minor Structure Theorem can be used to bound the pagenumber of the graph. Concepts and constructions in the literature introduced in the previous sections is used to show this result. 
\end{itemize}

Readers are expected to have at least an undergraduate understanding in graph theory and point-set topology. 
