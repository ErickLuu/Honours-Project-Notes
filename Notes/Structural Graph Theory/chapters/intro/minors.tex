\subsection{Graph minors}\label{sec:Graph Minors}
A graph \(H\) is a \textit{minor} of a graph \(G\) if a graph isomorphic to \(H\) can be obtained from \(G\) by deleting vertices, deleting edges, and \textit{contracting} edges. Let $G$ be a graph and let $uv$ be an edge in $E(G)$. To \textit{contract} \(uv\), we delete both \(u\) and \(v\) and create a new vertex \(w\) with neighbourhood \(N(w) = N_G(u) \cup N_G(v)\). The graph obtained after contracting the edge \(uv\) in $G$ is written as \(G\setminus uv\).
The statement ``\(H\) is a minor of \(G\)'' is written as \(H \leq G\). A graph \(G\) is \textit{\(H\)-minor-free} if $H$ is not a minor of $G$. A family of graphs \(\mathcal{F}\) is \textit{minor-closed} if when $G$ is in \(\mathcal{F}\) and \(H \leq G\), then $H$ is in \(\mathcal{F}\).
An example of a minor-closed class is the class of planar graphs.
An important class of graph families are the \(K_t\)-minor free graphs. For a graph \(G\), \(\had(G)\) is the the largest \(t\) such that \(K_t\) is a minor of \(G\). $\had(G)$ is named after Hugo Hadwiger due to his conjecture below.

\begin{conjecture}[Hadwiger's conjecture]\label{conj:Hadwiger's Conjecture}
	For all graphs \(G\), \(\chi(G) \leq \had(G)\)\cite{hadwigerUeberKlassifikationStreckenkomplexe1943}.
\end{conjecture}
Much work has been done on solving Hadwiger's conjecture, with a document by \textcite{seymourHadwigerConjecture2016} on the latest progress. However, \cref{conj:Hadwiger's Conjecture} remains unsolved. 

 Much of structural graph theory involves graph minors in some way. Many of the theorems that we will discuss throughout this report discuss graph minors. 

 \subsection{Graph Minor Theorem}\label{sec:Graph Minor Theorem}
The Graph Minor Theorem is one of the deepest and most important theorems in graph theory, and was proven by \textcite{robertsonGraphMinorsXX2004}. It states that every infinite family of graphs contains two distinct graphs \(G\) and \(H\) such that \(H\) is a minor of \(G\).
The Graph Minor Theorem is equivalent to the statement that every minor-closed graph family $\mathcal{F}$ is characterised by a finite set of minimal forbidden minors $\mathcal{H}$. A graph $G$ is in $\mathcal{F}$ if and only if $G$ is $\mathcal{H}$-minor free.
For planar graphs, the two minimal forbidden minors are \(K_5\) and \(K_{3,3}\), from \textcite{wagnerUeberEigenschaftEbenen1937}. 
The family of graphs that can be embedded on a torus are the toroidal graphs. There are at least 17,523 graphs which are forbidden minors, with a database maintained by \textcite{myrvoldLargeSetTorus2018}. A complete enumeration of minimal forbidden minors has not been found yet. $K_7$ is a toroidal graph but $K_8$ is not.