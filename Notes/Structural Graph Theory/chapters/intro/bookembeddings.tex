\section{Book-Embeddings and Pagenumber}\label{sec:Book Embedding}
A \textit{book} with \(k\) \textit{pages} consists of \(k\) half-planes glued together on a common boundary. We refer to the boundary as the \textit{spine}, and the individual half-planes as \textit{pages}. In topology, these are referred to as \textit{fans} of half-planes.\ \textcite{persingerSubsetsNbooksE31966,atneosenOnedimensionalNleavedContinua1972} described fans in the 1960s.
A \textit{book-embedding} of a graph \(G\) is an embedding of \(G\) on a book. We place the vertices of \(G\) on the spine, and we place each edge on a single page such that no two edges cross.
The \textit{pagenumber} of a graph \(G\) is the minimum number of pages required to embed \(G\) on a book. This is also referred to as \textit{book-thickness}, or \textit{stack-number}. An embedding of $K_5$ in three pages is in \cref{fig:book-embedding}.
\begin{figure}[h!]\label{fig:book-embedding}
	\centering
	\includesvg[height = 0.5\textheight]{figures/3page_K5.svg}
	\caption{Book-embedding of $K_5$ on three pages. Image by \textcite{eppsteinBookEmbedding2014}}
\end{figure}
\par
There is an equivalent combinatorial definition. A \textit{book embedding} of a graph \(G\) is an arrangement of the vertices of \(G\) in a total ordering \(v_1 < v_2 < \cdots < v_n\). We then \textit{colour} the edges \(E(G)\) such that if there are vertices with ordering \(v_a < v_b < v_c < v_d\) and edges \(v_a v_c\) and \(v_b v_d\) in $E(G)$, then $v_a v_c$ and $v_b v_d$ are assigned different colours.
We refer to the total ordering of \(V(G)\) in the book embedding as \((<)\) or as \((\leq)\). For a book-embedding \((v_1, v_2, \ldots, v_{|G|})\), we refer to the edges \( \left\{ v_1 v_2, v_2 v_3, \ldots, v_{|G| - 1}v_{|G|}, v_{|G|}v_{1} \right\} \) as \textit{spine edges}.
We may use a \textit{circular ordering} of the vertices rather than a linear ordering. This means that we order the vertices in a circle rather than on a straight line. The book-embedding of a circular ordering is exactly the same as for a linear ordering, and we can convert between a circular and linear ordering by choosing a vertex to be at the start of the sequence.
Book-embeddings were introduced by \textcite{kainenRecentResultsTopological1974, ollmannBookThicknessVarious1973} in the 1970s. A paper by \textcite{bernhartBookThicknessGraph1979} calculated the book-thickness of complete and bipartite graphs.

An \textit{expander graph} is a sparse, highly connected graph. Expander graphs share many properties with random graphs, but are constructed explicitly. One type of expander graph is a \textit{bipartite \varepsilon-expander}, where $\varepsilon \in (0, 1]$. We say a graph $G$ is a bipartite \varepsilon-expander if there exists a bipartition $ \{A, B\}$ of $V(G)$ such that $|A| = |B|$ and for all subsets $S \subset A$ where $|S| \leq \frac{|A|}{2}$, $|N(S)| \geq (1 + \varepsilon) |S|$. 
\textcite{dujmovicLayoutsExpanderGraphs2016} showed that all bipartite \varepsilon-expander graphs can be embedded in 3 pages. 


Book-embeddings of graphs were has applications in VLSI and processor designs, bioinformatics by \textcite{haslingerRNAStructuresPseudoknots1999}, and in graph drawings by \textcite{woodBoundedDegreeBook2002}. 
The project of finding upper and lower bounds of the pagenumber of planar graphs was started by \textcite{bernhartBookThicknessGraph1979} when they conjectured that planar graphs had unbounded pagenumber. However, \textcite{bussPagenumberPlanarGraphs1984} showed that all graphs could be embedded in nine pages, and \textcite{heathEmbeddingPlanarGraphs1984} brought down the number of needed pages to seven.\ \textcite{yannakakisEmbeddingPlanarGraphs1989} devised an algorithm to embed a graph in four pages. Yannakakis, in this paper, claimed that there exists planar graphs which cannot be embedded in three pages. However, his proof was incomplete and the full proof was left unpublished. In 2020, Yannanakis published his full proof \cite{yannakakisPlanarGraphsThat2020}. At around the same time, \textcite{kaufmannFourPagesAre2020} published the same lower bound.

\textcite{malitzGraphsEdgesHave1994} proved that any graph with $e$ edges has pagenumber $O(\sqrt{e})$. Additionally, he proved that random $d$-regular graphs $G$ with $n$ vertices have the property that $\pn(G) \in \Omega(\sqrt{d} n^{1/2 - 1/d})$. For random 3-regular graphs $G$ with $n$ vertices, $\pn(G) \in \Omega(n^{1/6})$. These constructions of $\Omega(n^d)$ pagenumber graphs are not explicit.\ \textcite{eppsteinThreeDimensionalGraphProducts2024} showed that $\pn(P_n \boxtimes P_n \boxtimes P_n) \in \Theta(n^{1/3})$. This is an explicit construction of a graph which has pagenumber in $\Theta(n^{d})$. 