% !TEX root = ./thesis.tex
\section{Problem Statement}

A \textit{book-embedding} of a graph $G$ arranges the vertices of $G$ on the ``spine'' of a book and arranges the edges of $G$ on ``pages'' of a book. The \textit{pagenumber} of a graph \(G\) is the minimum number of pages necessary in a book-embedding of \(G\). The concept of the \textit{pagenumber} of a graph was introduced by Ollmann \cite{ollmannBookThicknessVarious1973} in the context of VLSI design and integrated circuitry. 
The driving question of this report is the following:
\begin{conjecture}\label{conj:bded_had_pn}
	Suppose a graph $G$ is $K_t$-minor-free. Then the pagenumber of \(G\) is bounded by a function of \(t\).
\end{conjecture}

In her PhD thesis, \textcite{Blankenship-PhD03} claimed to prove \cref{conj:bded_had_pn}. However, this result has not been published and has not been independently verified. Furthermore, a key proof used by Blankenship, from \textcite{heathPagenumberGenusGraphs1992}, was found to be missing crucial parts. \textcite{nakamotoBookEmbeddingProjectiveplanar2015} found that a crucial case was missed by Heath and Istrail.

We begin this report by discussing the relevant literature. In their seminal work, \textcite{robertsonGraphMinorsXVI2003} proved the Graph Minor Theorem. In their papers, they introduced many important concepts and theorems that are still used to this day. The result we use is the Graph Minor Structure Theorem, which coarsely describes $K_t$-minor free graphs.

Robertson and Seymour showed that graphs with no \(K_t\) minor can be built from smaller building blocks. This is a rough overview of the building blocks. We first start with a graph \(G\) embedded on a genus \(g\) surface. Then we add on \(p\) \textit{vortices} to \(G\), with \textit{pathwidth} at most \(k\). Then we add on \(a\) \textit{apex vertices} to \(G\). We say that \(G\) is \((g, p, k, a)\)-\textit{almost embeddable}. Robertson and Seymour \cite{robertsonGraphMinorsXVI2003} proved that all graphs with no \(K_t\) minor has a \textit{tree-decomposition} where every \textit{torso} is a \((g, p, k, a)\) almost-embeddable graph, with \((g, p, k, a)\) bounded by a function of \(t\).

This honours project has two goals. The first goal is to investigate and learn more about structural graph theory. We will discuss some important machinery in structural graph theory, the main ones being the Graph Minor Theorem and the Graph Minor Structure Theorem. The second goal is to address an open problem within this field. To this end, an entire chapter building on the techniques discussed in previous chapters discusses this open problem. 
