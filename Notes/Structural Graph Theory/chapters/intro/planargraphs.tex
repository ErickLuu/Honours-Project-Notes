\subsection{Planar graphs}
A graph \(G\) is \textit{planar} if \(G\) can be drawn in the Euclidean plane \( \mathbb{R}^2 \) such that no two edges cross. A drawing of $G$ in $\mathbb{R}^2$ is referred to as an \textit{embedding} of $G$ in $\mathbb{R}^2$, or a \textit{plane graph}. 

Planar graphs are of exceptional interest in graph theory, and have been studied extensively. Planar graphs model many real-world scenarios, such as roads between towns or cities, or borders between regions.
Let $G$ be a connected planar graph, and let $F(G)$ be the set of faces on $G$, or the connected components of $\mathbb{R}^2 - G$. Then:
\begin{equation}
	|V(G)| - |E(G)| + |F(G)| = 2. 
\end{equation}

The most famous theorem involving planar graphs is the Map Colouring Problem,  which is that every planar graph is 4-colourable. 
This theorem was one of the most famous open problems in graph theory.
This was proven by \textcite{appelEveryPlanarMap1989,robertsonEfficientlyFourcoloringPlanar1996}. The proofs given by both papers relies on computer analysis of a large number of planar graphs. Additionally, both used an algorithm to reduce large planar graphs to one of the solved cases. The proof given by \textcite{appelEveryPlanarMap1989} was flawed in some key reduction steps, and was revised extensively by \textcite{robertsonEfficientlyFourcoloringPlanar1996}. Famously, this was one of the earliest examples of computers being used in proof verification.
