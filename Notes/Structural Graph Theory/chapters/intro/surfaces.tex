\subsection{Surfaces and graphs on surfaces}
This section discusses surfaces, and graphs on surfaces. Graphs on surfaces are a natural extension to graphs on planes. Readers are expected to be familiar with point-set topology. This section is based on \textcite{moharGraphsSurfaces2001}.

An $n$-manifold $M$ is a second-countable Hausdorff space where every point in $M$ has an open neighbourhood homeomorphic to an open ball in $\mathbb{R}^n$.  
A \textit{surface} is a $2$-manifold. Surfaces are typically denoted as $\Sigma$. Examples of surfaces are the sphere $S^2$, the torus $T^2$, the real projective plane $\mathbb{R}P^2$, and the Klein bottle $K$. 

\textit{Handles} are added to a surface \(\Sigma\) by removing two disks in \(\Sigma\) and identifying the boundaries such that one goes clockwise and the other goes counter-clockwise. \textit{Crosscaps} are added to a surface $\Sigma$ by removing a disk in \(\Sigma\) and identifying opposite points on the boundary. 

\begin{theorem}[Classification of surfaces]
	Every surface is homeomorphic to a sphere with some added crosscaps or handles. 
\end{theorem}

\begin{definition}[Euler Genus]
	The \textit{Euler genus} of a surface \(\Sigma\), obtained from a sphere by adding \(h\) handles and \(c\) crosscaps, is \(2h + c\).
\end{definition}

The \textit{Euler Genus} of a \textit{graph} \(G\) is the smallest Euler genus \(g\) surface \(\Sigma\) such that \(G\) can be embedded on \(\Sigma\) without crossings (note that \(\Sigma\) can be orientable or nonorientable). 

An extension of Euler's formula to surfaces is given below. 
\begin{theorem}[Euler's formula on surfaces]\label{thm:Euler_surfaces}
	Let \(|F(G)|\) be the number of faces in a graph \(G\). Then \(|V(G)| - |E(G)| + |F(G)| = 2 - g = \chi\). 
\end{theorem}
The value $\chi$ is known as the \textit{Euler characteristic} of a topological space, in this case a surface. Calculating the Euler characteristic of any space is done through \textit{homological algebra}. 

Graphs that can be embedded on the plane are called \textit{planar} graphs. Graphs that can be 2-cell embedded on the torus are called \textit{toroidal} graphs, and graphs that can be 2-cell embedded on the projective plane are called \textit{projective-planar} graphs.


Graphs on surfaces have been studied extensively. A famous conjecture involving graphs on surfaces is Heawood's conjecture, which states that:
\begin{theorem}
	The minimum number of colours sufficient to colour all Euler genus $g$ graphs, $g \geq 0$, is
	\begin{equation*}
		\gamma(g) := \left\lfloor 
		\frac{7 + \sqrt{1 + 24g}}{2}
		\right\rfloor.
	\end{equation*}
\end{theorem}
\textcite{ringelMapColorTheorem1974} showed that for almost every case, $\gamma(g)$ is also necessary. The case where this does not hold is the Klein bottle case. There exists a 6-colourable Klein bottle graph, but $\gamma(g) = 7$. 