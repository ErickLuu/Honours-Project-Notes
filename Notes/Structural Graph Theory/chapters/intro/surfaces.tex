\subsection{Surfaces and graphs on surfaces}
Graphs on surfaces are a natural extension to graphs on planes. This section is an introduction to surfaces and graphs on surfaces. Readers are expected to be familiar with point-set topology. This section is based on \textcite{moharGraphsSurfaces2001}.

An \textit{$n$-manifold} $M$ is a second-countable Hausdorff space where every point in $M$ has an open neighbourhood homeomorphic to an open ball in $\mathbb{R}^n$.  
A \textit{surface} is a $2$-manifold. Surfaces are typically denoted as $\Sigma$. Examples of surfaces are the sphere $S^2$, the torus $T^2$, the real projective plane $\mathbb{R}P^2$, and the Klein bottle $K$. 

\textit{Handles} are added to a surface \(\Sigma\) by removing two disks in \(\Sigma\) and identifying the boundaries such that one goes clockwise and the other goes counter-clockwise. \textit{Crosscaps} are added to a surface $\Sigma$ by removing a disk in \(\Sigma\) and identifying opposite points on the boundary. Every surface is homeomorphic to a sphere with $m$ handles and $n$ crosscaps. The \textit{Euler genus} of a surface \(\Sigma\) with $m$ handles and $n$ crosscaps is $2m + n$. In fact, a sphere with a mix of crosscaps and handles is homeomorphic to a sphere with all crosscaps, as a sphere with a handle and crosscap is homeomorphic to three crosscaps.

An \textit{embedding} of $G$ on a surface $\Sigma$ is a drawing of $G$ on $\Sigma$ such that no two edges cross. 
A \textit{$2$-cell embedding} of a graph $G$ on a surface $\Sigma$ is an embedding of $G$ in $\Sigma$ such that $\Sigma - G$ is homeomorphic to a finite number of disks. The \textit{Euler Genus} of a \textit{graph} \(G\) is the smallest Euler genus \(g\) surface \(\Sigma\) such that \(G\) can be $2$-cell embedded on $\Sigma$.

An extension for Euler's formula is below. Suppose $G$ is $2$-cell embedded on a surface $\Sigma$ of genus $g$. Let \(|F(G)|\) be the number of faces in a graph \(G\). Then \(|V(G)| - |E(G)| + |F(G)| = 2 - g = \chi\). When $g = 0$, then $\Sigma$ is a $2$-sphere and this is the original Euler's formula. 
The value $\chi$ is known as the \textit{Euler characteristic} of a topological space, in this case a surface. The Euler characteristic is invariant under homeomorphism. Calculating the Euler characteristic of any space is done through \textit{homological algebra}, specifically by looking at the free rank of homology groups. 

Graphs that can be embedded on the plane are called \textit{planar} graphs. Graphs that can be 2-cell embedded on the torus are called \textit{toroidal} graphs, and graphs that can be 2-cell embedded on the projective plane are called \textit{projective-planar} graphs. Graphs that can be 2-cell embedded on a surface of genus $g$ are called \textit{genus $g$} graphs. Similarly to plane graphs, graph drawings on the torus are called torus graphs, and graphs drawings on the projective plane are called projective-plane graphs. 


Graphs on surfaces have been studied extensively. A famous conjecture involving graphs on surfaces is Heawood's conjecture, from \textcite{heawoodMapcolourTheorem1890}. The conjecture states that the minimum number of colours sufficient to colour all Euler genus $g$ graphs when $g \geq 0$ is
	\begin{equation*}
		\gamma(g) := \left\lfloor 
		\frac{7 + \sqrt{1 + 24g}}{2}
		\right\rfloor.
	\end{equation*}\todo{is this right?}
\textcite{ringelMapColorTheorem1974} showed that for almost every case, $\gamma(g)$ is also necessary. The case where this does not hold is the Klein bottle case. There exists a 6-colourable Klein bottle graph, but $\gamma(g) = 7$. 