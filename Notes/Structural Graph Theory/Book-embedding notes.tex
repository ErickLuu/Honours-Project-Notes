\documentclass[]{article}
\usepackage[margin = 1in]{geometry}

\usepackage{amsmath}
\usepackage{amssymb}
\usepackage{amsthm}
\usepackage{url}

% Commands
\newcommand{\tree}{\mathcal{T}}
\newcommand{\tw}{\text{tw}}
\newcommand{\had}{\text{had}}
\newcommand{\pw}{\text{pw}}
\newcommand{\td}{\text{td}}
\newcommand{\pn}{\text{pn}}
% Environments

\newtheorem{theorem}{Theorem}
\newtheorem{proposition}[theorem]{Proposition}
\newtheorem{corollary}[theorem]{Corollary}
\newtheorem{lemma}[theorem]{Lemma}
\newtheorem{definition}[theorem]{Definition}
\newtheorem{conjecture}[theorem]{Conjecture}

\theoremstyle{definition}
\newtheorem{example}[theorem]{Example}

\numberwithin{theorem}{section}
\numberwithin{equation}{section}

%opening
\title{Structural Graph theory - Book embeddings}
\author{Eric Luu}

\begin{document}

\maketitle

\section{Book embedding}
A \textit{book embedding} of a graph $G$ is an arrangement of the vertices of $G$ in an ordered sequence $v_1, v_2, ..., v_n$ and a colouring of the edges of $G$ into partitions such that each edge $v_av_b, v_c v_d$ (where $a < b$ and $c < d$), if they are assigned the same colour, does not have the property that $v_a < v_c < v_b < v_d$ or $v_c < v_a < v_d < v_b$, i.e. the edges $v_a v_b$ and $v_c v_d$ do not cross. Equivalently, we can treat the vertices as lying on the "spine" of a series of half-planes in $\mathbb{R}$ and each edge is drawn on the half plane such that no two edges cross on each sheet. There is also an alternative definition where we think about a \textit{circular ordering} of the vertices rather than a linear ordering, so each vertex lies on a cycle. Then we can view the page-embedding as stacking discs on top of each other such that the edges on each disc is a planar subgraph. We wish to minimise the number of pages used given a particular graph embedding. We refer to the minimum number of pages necessary to draw a graph $G$ in this way as the \textit{page-number} of $G$, or alternatively the \textit{book-thickness}, or \textit{stack-number} of the graph $G$. We denote the page-number of a graph $G$ as $\pn(G)$. Page-numbers were introduced by Bernhart and Kainen \url{https://www.sciencedirect.com/science/article/pii/0095895679900212?via%3Dihub}.

\subsection{Properties of page numbers}
We have that if $H$ is a subgraph of $G$, then $\pn(H)$ is a subgraph of $\pn(G)$. (CHECK IF PAGENUMBER IS MINOR-CLOSED)
\subsection{Graphs with bounded page number}
\begin{lemma}
	A graph $G$ has page-number at most 1 iff $G$ is an outerplanar graph.
\end{lemma}

\begin{lemma}
	A graph $G$ has pagenumber at most 2 iff $G$ is a subgraph of a planar graph with a Hamiltonian cycle. This is because we can embed the graph on a sphere with the vertices and Hamiltonian cycle on the equator, and the edges forming the interior and exterior edges of the cycle respectively.
\end{lemma}
This is because we can embed the graph on a sphere with the vertices and Hamiltonian cycle on the equator, and the edges forming the interior and exterior edges of the cycle respectively.
\subsection{Planar graphs}
\begin{theorem}
	Planar graphs can be embedded on at most four pages.
\end{theorem}
The proof of this will be shown below.
\subsubsection{Yannakakis' Algorithm}
DO LATER


\subsection{Historical interest}
Pagenumbers were developed for processor designs, but more recently have been used in bioinformatics and the like. 
The project of finding upper and lower bounds of the pagenumber of planar graphs was started by Bernhart and Kainen when they conjectured that planar graphs had unbounded page-number. However, Buss and Shor \url{https://users.monash.edu.au/~davidwo/PAPERS/StacksQueues/BussShor-PagenumberPlanar.pdf} found that the pagenumber of planar graphs was at most 9, and Heath \url{https://ieeexplore.ieee.org/document/715903} found that the pagenumber of planar graphs is at most 7. Yannakakis' devised an algorithm to bound the pagenumber to at most 4 \url{https://users.monash.edu.au/~davidwo/PAPERS/StacksQueues/Yannakakis-4pages-JCSS89.pdf}. It was conjectured by Yannakakis that there exist planar graphs with pagenumber 4, but only recently were planar graphs found with pagenumber 4. 
\subsection{Other graphs}
We have that $K_n$ complete graphs can be embedded on $\lceil n/2 \rceil$ pages. Therefore for any graph $G$ on $n$ vertices, $\pn(G) \leq \lceil n/2 \rceil$. 

\section{Characterisations of $K_t$-minor free graphs}
What is the structure of $K_t$-minor free graphs? We shall show that we can roughly characterise all $K_t$-minor free graphs as graphs that are products of a series of operations. 
\subsection{$K_t$-minor free minor-closed families}
We define $\had(G)$ to be the largest $t$ such that $G$ has a $K_t$ minor. 
\subsubsection{Planar graphs}
\begin{theorem}
	If $G$ is a planar graph, then $G$ is $K_5$-minor-free.
\end{theorem}
\begin{proof}
	If $G$ is planar with $n$ vertices and $m$ edges, then we have that $m \leq 3n -6$. However, we have that $K_5$ has $5$ vertices and $10$edges, but we have that $ 10 > 3 \times 5 - 6$, so $K_5$ is not planar. As the family of planar graphs is minor-closed, then if $G$ is planar, then $K_5$ is minor-free.
\end{proof}

\subsubsection{Genus-g graphs}
We define the genus $g$ of a surface to be 2 times the number of handles + the number of crosscaps. From topology, we have that we can add a handle to crosscaps to form 3 crosscaps. Therefore, the Euler characteristic $\chi = 2 - g$ for both orientable and non-orientable surfaces. Note that the genus is defined slightly differently from topology. We do this to allow non-orientable and orientable surfaces to coincide in definition.

We can show that if $G$ has genus $g$, then if $G$ has $n$ vertices and $m$ edges, then $n - m + f = \chi = 2-g$, then as each face has at most 3 vertices and each edge is incident to two faces, we have that $f \leq 2m/3$. Therefore, $m \leq 3(n + g - 2)$, and if $K_t$ is embeddable on a genus $g$ graph, then $\binom{t}{2} \leq 3 (t + g - 2)$. Thus $t \leq \sqrt{6g} + 4$. So if a graph has genus $g$, then it is $K_t$-minor-free, where $t > \sqrt{6g} + 4$. 

\subsubsection{Bounded treewidth graphs}
\begin{theorem}
	If $\tw(G) \leq k$, then $G$ is $K_{k+2}$-minor-free. 
\end{theorem}
\begin{proof}
	We shall prove the contrapositive: If $K_t$ is a minor of $G$, then $tw(G) \geq t-1$.
	If $K_t$ is a minor of $G$, and $\tw(G) \leq k$, then we have that $\tw(K_t) \leq \tw(G) \leq k$, but $\tw(K_t) = t-1 \leq k$, so $t \leq k + 1$. Thus shown a family of minor-closed which are $K_t$-minor free. 
\end{proof}
\subsubsection{Apex vertices}
An apex vertex $v$ is added to a graph $G$ such that it has arbitrary edges. As such, it can simply dominate all other vertices in $G$. Then if $G$ is $K_t$-minor free, $G$ with the apex vertex $v$ is $K_{t+1}$- minor free. 
\subsubsection{Clique-sums}
The \textit{$k$-clique-sum} of two graphs $G$ and $H$, denoted as $G \# H$, is the graph obtained by performing a series of operation on the cliques of $G$ and $H$. We find cliques in $G$ and $H$, $C_G$ and $C_H$ respectively, such that $C_G$ and $C_H$ have size $k$. Then we identify the vertices in $C_G$ and $C_H$ so that $G$ and $H$ are connected to each other on this clique. 

\begin{lemma}
	If $G = G_1 \# G_2$,then $\had(G) = \max(\had(G_1), \had(G_2))$ and $\tw(G) = \max(\tw(G_1), \tw(G_2))$.
\end{lemma}

\begin{example}
	If $G$ is the clique-sum of Euler genus $g$ graphs, then $G$ is $K_{\sqrt{6g} + 5}$-minor-free, but has unbounded genus.
\end{example}

\begin{theorem}[Wagner's theorem]
	If $G$ is $K_5$-minor-free, then $G$ can be obtained from $\leq 3$-clique-sums of planar graphs and the Wagner graph $W_8$.
\end{theorem}


\subsection{Torsos and adhesion}
Given a graph $G$ and a tree-decomposition $\tree$, the \textit{torso} of a bag $B_x$ of $T$ is the graph $G\langle B_x \rangle$, obtained from $G[B_x]$ where $vw$ is a vertex in $G\langle B_x \rangle$ iff $v,w \in B_x \cap B_y$, where $y$ is a neighbour of $x$ in $T$. So the set $B_x \cap B_y$ for all neighbours $y$ of $x$ in $T$ is a clique in $G\langle B_x \rangle$. 
The \textit{adhesion} of a tree is defined as $\max(|B_x \cap B_y|)$ where $xy$ is an edge in $T$.

\subsubsection{Vortices}
Let $G$ be embedded on a surface $\Sigma$, and let $F$ be a face on $G$. Let $D$ be a disc in $\Sigma$ such that $D$ only intersects $G$ only on vertices on the boundary of $F$. We denote these discs as $G$-clean. 

Then let $\Lambda = (x_1, x_2, ..., x_b)$ be a tuple of vertices on the boundary of $F$ such that they intersect $D$. Then we define a new graph $H$ such that $V(G) \cap V(H) = \Lambda$, and there is a path-decomposition of $H$ of bags $B_1, B_2, ... B_b$ such that $x_i \in B_i$ for all $i$. $H$ is denoted as a \textit{$D$-vortex} of $G$. The width of a $D$-vortex is the width of the path above, or $\max_i(|B_i| - 1)$. 

Vortices were created to solve the problem of grid-like graphs with large treewidth, torsos and adhesion, yet are all $K_t$-free for bounded $t$. 
\subsection{Robertson-Seymour theorem}
Given $g, p, a \geq 0$, $k \geq 1$, a graph $G$ is $(g, p, k, a)$- almost embeddable if there exists an $A \subseteq V(G)$ with $|A| \leq a$, and there exists subgraphs $G_0, G_1, ...,  G_{p'}$ of $G$ such that:
\begin{itemize}
	\item $G - A = G_0 \cup G_1 \cup G_2 ... G_{p'}$
	\item $p' \leq p$
	\item There is an embedding of $G_0$ onto a surface $\Sigma$ of genus $\leq g$
	\item There exists pairwise disjoint $G_0$-clean discs $D_1, D_2, ..., D_{p'}$ in $\Sigma$
	\item $G_i$ is a $D_i$-vortex of width at most $k$.
\end{itemize}

\begin{theorem}[Robertson-Seymour graph structure theorem for $K_t$-minor-free graphs]
	For all $t$, there exists $g, p, a \geq 0$, $k \ell \geq 1$, such that every $K_t$-minor-free graph has a tree-decomposition of adhesion $\leq \ell$ and each torso is $(g, p, k, a)$-embeddable. 
\end{theorem}
In fact, there exists a function $t(g, p, k, a)$ such that if a graph has a tree-decomposition of adhesion $\leq \ell$ and each torso is $(g, p, k, a)$-almost embeddable, then $G$ has no $K_t$ minor. We can show that such a function is $t(g, p, k, a) = a + ck \sqrt{g + p}$. 

\section{Bounds of pagenumbers of graphs}
\begin{theorem}
	Let $g$ be the genus of a graph $G$. We have that for all graphs $G$, $\pn(G) \leq O(g)$ for some $g$.
\end{theorem}
\begin{proof}[Heath and Istrail]
	
\end{proof}
Note that this bound extends the one found by Yannakakis. 
The best known bound is $\sqrt{g}$, found by Malitz. (cite here).

We additionally have that $\pn(G) \leq \tw(G) + 1$, from \url{https://core.ac.uk/download/pdf/82019214.pdf}. 
We also have that bounded clique-sums of graphs of bounded treewidth also have bounded path-number.
What is a sticking-point is vortices of graphs on bounded genus, and cliquesums of vortices on bounded genus as well. 
\subsection{Tree-decomposition into bounded page number components}
We denote $G(\mathcal{G}, T)$ as a clique-sum tree, where each vertex $v_i$ in the tree $T$ is a graph $G_i$ in $\mathcal{G}$, and the edge $v_iv_j$ corresponds with the clique sum of graphs $G_i$ and $G_j$. We refer to the stack layout as a tuple $(\leq , \psi)$ where $\leq$ is an ordering of the vertices and $\psi : E(G) \rightarrow X$ assigns to each edge the page. The page number is the smallest set $X$ on which this is defined. 
\begin{theorem}[Hickingbotham and Wood, private communications]
	Let $G(\mathcal{G}, T)$ be a clique-sum tree where $\pn(G_i) \leq s$ for all $G_i \in \mathcal{G}$. Then $\pn(G(\mathcal{G}, T)) \leq 2s^2 + 4s + 1$.  
\end{theorem}

\subsubsection{Proof of above theorem.}
Let $C$ be a clique in $G$ and let $\sigma_C = (u_1, ... , u_k)$ be a vertex ordering of $V(C)$. For any arbitrary clique $J$, we define a rainbow-vertex $w \in V(J)$ as a vertex where for any $x, y \in V(J)$, the edges $wx$ and $wy$ are on different pages. We want the book embedding to have the structure $(\underbrace{u_1, u_2, ..., u_k}_{\text{Vertices in } C}, \underbrace{v_1, v_2, ..., v_l}_{\text{Vertices not in }C})$.

We will use these two results.
\begin{lemma}
	If $pn(G) \leq s$, then $G$ does not contain any cliques on more than $2s-1$ vertices.
\end{lemma}
\begin{lemma}
	For any graph $G$, $\chi(G) \leq 2 \pn(G) + 2$.
\end{lemma}

To prove this theorem, we will use a common technique in graph theory. We will strengthen the lemma so that we may use induction to prove the statement.
\begin{lemma}
	Let $G$ be a graph where $\pn(G) \leq s$, and a clique $C$ with an ordering $\sigma_C$. There exists a $(2s^2 + 4s + 1)$-stack layout $(\leq, \psi)$ of $G$ where:
	\begin{enumerate}
		\item The vertex ordering has the structure $(\underbrace{u_1, u_2, ..., u_k}_{\text{Vertices in } C}, \underbrace{v_1, v_2, ..., v_l}_{\text{Vertices not in }C})$
		\item For every $u \in V(C)$, the edges $\lbrace uv \in E(G) : u \leq v \rbrace$ are in a single page with no other edges on it
		\item For every clique $J$, the last vertex of $J$ is a rainbow-vertex. 
	\end{enumerate}
\end{lemma}
Let $(\leq_a, \psi_a)$ be a $s$-stack layout of $G$ and let $\rho: V(G) \rightarrow \lbrace 1, 2, ..., 2s + 2 \rbrace$ be a proper colouring of $V(G)$. Then:

Let $u_1, ..., u_k$ be the vertices of $C$ ordered by $\sigma_C$. Note that $k \leq 2s + 1$. Then the new ordering starts with $u_1 \leq u_2 \leq ..., \leq u_k$, and all vertices not in $K$ are placed after. according to $\leq_a$.
Then the stack assignment $\psi$ is now defined. For every edge $u_i v$ where $u_i \in V(C)$ and $u_i \leq v$, define $u_i v = i$. Otherwise, if neither $u$ or $v$ are in $V(C)$, and $u \leq v$, then let $\psi(uv) = (\rho(u), \psi_a(uv))$. Then we have at most $|\rho| |\psi_a| + k \leq (2s + 2) s + (2s + 1) = 2s^2 + 4s + 1$ pages.

We shall show that $(\leq, \psi)$ is a proper book-embedding. Suppose we have a pair of edges $uv$ and $xy$ which cross, and $\phi(uv) = \phi(xy)$. Suppose that $u$ is the smallest vertex in the ordering $\leq$. If $u \in V(C)$, then the edge $uv$ is assigned to its own page, meaning that it cannot cross $xy$. So $x = u$, but we can draw $uv$ and $uy$ on a single page. Thus they do not cross. Therefore we have that $u, v, x, y$ are not in $V(C)$, and as we preserve the original book-embedding, then these edges do not cross. Thus shown.
We have that property 1 and 2 are immediate. For property 3, consider a clique $J$ in $G$. Then we must show the last vertex of $J$ is rainbow. Suppose the last vertex of $J$ is $w$, and let $u, v$ be earlier vertices. Since $u$ and $v$ necessarily are assigned different colours in the colouring, then $\psi(uw) = (\rho(u), \psi_a(uw))$ and $\psi(vw) = (\rho(v), \psi_a(vw))$. Therefore, the two edges are on different pages. Thus $w$ is a rainbow vertex. 

\subsubsection{Full proof}
\begin{theorem}
	Suppose $G = G(\mathcal{G}, T)$ be the clique-sum tree, with vertices $G_0, G_1, ..., G_k$, and for all $i \in \lbrace 0, 1, ..., k \rbrace$, we have that $\pn(G) \leq s$. Then there is a book-embedding of $G$ with pagenumber of at most $2s^2 + 4s + 1$. 
\end{theorem}

\begin{proof}
	To prove the statement, we shall prove the stronger statement that there exists a book-embedding with the properties described with the lemma above using induction. In particular, we will have that the last vertex of any clique $J$ is a rainbow vertex.
	
	Suppose $k = 0$. Then $G_0$ is a single graph with $\pn(G) \leq s$. Then by the lemma above, there is a book-embedding with pagenumber at most $2s^2 + 4s + 1$ with the property that it starts with $K$ and every last vertex in a clique is a rainbow vertex.
	
	Suppose $k = n$. Let $C$ be the clique between $G_n$ and the rest of $G$, where $G_n$ is a leaf of the tree $T$. Denote the induced subgraph $G[V(G) - V(G_n - C)]$ as $G'$. Then let $(\leq_n, \psi_n)$ be the $2s^2 + 4s + 1$-page-number book-embedding of $G_n$ that starts with $V(C)$, and let $\sigma_C$ be the ordering of $V(C)$ by $\leq_n$. Let $(\leq_{n-1}, \psi_{n-1})$ be the stack-embedding of $G'$. By induction, this has a $2s^2 + 4s + 1)$ book-embedding with the properties described above.
	
	\paragraph{Construction  of new book-embedding}
	Let $w \in V(K)$ be the last vertex of $K$ with respect to $\leq_{n-1}$. Then insert $V(G_n - C)$ between $w$ and its successor in the order of $\leq_{n}$. For the page assignment $\psi$, we have that if $uv \in E(G')$, then $\psi(uv) = \psi_{n-1}(uv)$. For the remaining edges, we can permute the edge assignments of $\psi_n$ such that for all $u \in V(K)$, we have that $\psi(E_u) = \psi_n(uw)$. We can do this as $w$ is a rainbow vertex and the edges $E_u$ are assigned to a unique page in $\psi_n$. Finally, let $\psi(uv) = \psi_n(uv)$ for the remainder of the edges. Denote the new ordering and assignment as $(\leq, \psi)$. 
	
	We claim that $(\leq , \psi)$ is a stack layout that satisfies the induction hypothesis. Suppose that $\psi(uv) = \psi(xy)$. If $uv, xy \in E(G')$, then by the induction hypothesis, they do not cross. Similarly, if $uv, xy \in E(G_n)$, then they will not cross, by the above lemma. If $uv$ is in $E(G')$ and $xy \in E(G_n)$, then they will go over each other or be sequential and therefore will not cross. 
	Finally, if $u, v, x, y \in C$, then by the induction hypothesis on $G'$, they do not cross either. The final case is if $uv \in E(G_{i + 1})$ and $u \in V(C)$, $v \notin V(C)$, $xy \in E(G')$. If $uv$ and $xy$ cross, then we have that $xy$ and $uw$ will cross. But this will contradict the page-embedding of $G'$.
	
	Let $J$ be a clique in $G$, and $w$ is its final vertex. Then if $J$ is in $G'$, then $w$ is a rainbow-vertex. Otherwise, the last vertex is contained in $G_n$. By the above lemma, $w$ must also be a rainbow vertex. Thus shown.
\end{proof}

\subsection{Bounded treewidth and page number}
\begin{theorem}{Ganley + Heath}
	Every graph $G$ with $\tw(G) \leq k$ has $\pn(G) \leq k + 1$. 
\end{theorem}

We use the characterisation of treewidth as a subgraph of a $k$-tree. We will show that all $k$-trees have pagenumber at most $k + 1$, which suffices to show the above theorem.

\begin{theorem}
	If $G$ is a $k$-tree, then $\pn(G) \leq k + 1$. 
\end{theorem}
\begin{proof}
	Consider a tree-decomposition of $G$, $\tree = (B, T)$, and perform a depth-first search on $T$, starting at an arbitrary root $r$. Let the ordering of the book-embedding $\sigma(v)$ of a vertex $v$ in $V(G)$ be determined by the first time $x \in T$ appears, where $v \in B_x$. For the vertices in the root bag $B_r$, we order arbitrarily. By the definition of a $k$-tree, only one vertex will be added in the depth-first search. Now consider the subtree $T_v$ induced by the bags $B_x$ containing $v$. We now consider colouring the subtrees $T_v$ for all $v \in G$ such that no overlapping subtrees have the same colour. Let $H$ be the intersection graph of the subtrees, where $V(H) = \lbrace T_v : v \in G \rbrace$ and $T_u T_v \in E(H)$ if there exists a bag $B_x$ such that $u, b \in B_x$. We have that $H$ is perfect, and thus $\chi(H) = \omega(H)$. Since $|B_x| = k + 1$ for all $x \in V(T)$, then there is a clique of size $k + 1$ in $H$. If there is a clique in $H$ with more than $k+ 1$ vertices, then this implies that there exists some $B_x$ such that $|B_x| > k + 1$. Thus $\chi(H) = k + 1$. 
	\paragraph{}
	We now use this to assign the edges of $G$ a page. Let $c(T_v)$ be the colour assigned to $T_v$. Colour each edge $uv \in E(G)$ as follows:
	\begin{equation}
		c(uv) = 
		\begin{cases}
			c(T_u) &\text{ if } \sigma(u) \leq \sigma(v),\\
			c(T_v) &\text{ if } \sigma(v) \leq \sigma(u)
		\end{cases}
	\end{equation}
	Then we claim that this is a proper book-embedding of $G$. Suppose we have that edges $uv$, $xy$ cross, so $\sigma(u) \leq \sigma(x) \leq \sigma(v) \leq \sigma(y)$. However, this implies that there exists a bag $B$ such that $u, x, v \in B$, as we have that $uv$ is an edge in $B$ and we do a depth-first search to establish the ordering, meaning that $u, v, x$ are in a clique. Therefore, they are in the same bags. However, this implies that the trees $T_u$ and $T_x$ intersect, meaning that $c(uv) \neq c(xy)$. Finally, the number of pages used is $\chi(H) = k + 1$, so $\pn(G) \leq k + 1$. Thus shown.
\end{proof}
\end{document}
