\documentclass[]{report}
\usepackage[margin = 1in]{geometry}

\usepackage{amsmath}
\usepackage{amssymb}
\usepackage{amsthm}
\usepackage[english]{babel}
\usepackage{url}
\usepackage{todonotes}
\usepackage{csquotes}

\usepackage{hyperref}
\usepackage[noabbrev, capitalise]{cleveref}

\usepackage[style = numeric,
isbn=false,
url=false,
eprint = false,
maxbibnames=99
]{biblatex}
\renewbibmacro{in:}{}
\DeclareSourcemap{
	\maps[datatype=bibtex]{
		\map{
			\step[fieldset=url, null]
			\step[fieldset=extra, null]
			\step[fieldset=urldate, null]
		}
	}
}
\AtEveryBibitem{%
	\clearfield{day}%
	\clearfield{month}%
	\clearfield{endday}%
	\clearfield{endmonth}%
}

\addbibresource{Book-Embeddings.bib}
% Commands
\newcommand{\tree}{\mathcal{T}}
\newcommand{\tw}{\text{tw}}
\newcommand{\had}{\text{had}}
\newcommand{\pw}{\text{pw}}
\newcommand{\td}{\text{td}}
\newcommand{\pn}{\text{pn}}
% Environments

\newtheorem{theorem}{Theorem}
\newtheorem{proposition}[theorem]{Proposition}
\newtheorem{corollary}[theorem]{Corollary}
\newtheorem{lemma}[theorem]{Lemma}
\newtheorem{definition}[theorem]{Definition}
\newtheorem{conjecture}[theorem]{Conjecture}

\theoremstyle{definition}
\newtheorem{example}[theorem]{Example}

\numberwithin{theorem}{section}
\numberwithin{equation}{section}

%opening
\title{Towards a proof that all $K_t$-minor-free graphs have bounded pagenumber}
\author{Eric Luu}

\begin{document}

\maketitle
\chapter{Abstract}\label{abstract}
In this report we outline our current progress in proving that $K_t$-minor free graphs have bounded pagenumber. Proving this bound will connect two important concepts in structural graph theory that have been studied extensively for the past 40 years. In this report we will outline the most important theorem in structural graph theory related to $K_t$-minor free graphs, the Graph Minor Structure Theorem. We will also introduce some other proofs which will be of use proving this bounded pagenumber conjecture. 
\chapter{Introduction}\label{sec:introduction}
Structural graph theory is a fundamental topic in graph theory and its study has led to a deeper understanding of graphs. Many results from structural graph theory decompose graphs into ones with bounded parameters. One of the most important theorems in structural graph theory is Robertson and Seymour's Graph Minor Theorem \cite{robertsonGraphMinorsXX2004} which states that proper minor-closed families of graphs are characterised by a finite set of forbidden minors. 
\par
The concept of the \textit{pagenumber} of a graph was introduced by L. Taylor Ollmann \cite{ollmannBookThicknessVarious1973} in the context of VLSI design and integrated circuitry, but was studied deeply by Bernhardt and Kainen\cite{bernhartBookThicknessGraph1979} in 1979. A \textit{book-embedding} of a graph is a way to arrange the vertices on the ``spine'' of a book and arrange the edges on ``pages'' of a book, or half-planes. The \textit{pagenumber} of a graph $G$ is the smallest number of pages necessary in a book-embedding of $G$. 

The driving question of this report is the following: 
\begin{conjecture}\label{conj:bded_had_pn}
	Given a graph $G$ with no $K_t$ minor, is the pagenumber of $G$ bounded by a function on $t$, so $\pn(G) \leq f(t)$ for some $t$?
\end{conjecture}
Answering this question will yield a link between the pagenumber of a graph and the global structure of the graph. This report currently lays out the literature related to this question. We use a result in one of the papers of the Graph Minor Theorem, which is the Graph Minor Structure Theorem \cite{robertsonGraphMinorsXVI2003}.

In a PhD thesis, Blankenship claimed to have a proof of \cref{conj:bded_had_pn}.\cite{Blankenship-PhD03} However, this result has not been published in any journal and has not been independently verified. We aim to fill this gap in our knowledge. 
\subsection{Plan for solving the problem}
We aim to solve the question using the Graph Minor Structure Theorem \cite{robertsonGraphMinorsXVI2003}, which describes the structure of graphs which do not contain a $K_t$ minor.
\par
Robertson and Seymour showed that we can build graphs with no $K_t$ minor from smaller building blocks. We first start with a graph $G$ embedded on a genus $g$ surface. Then we add on $p$ \textit{vortices} to $G$, with \textit{pathwidth} at most $k$. Then we add on $a$ \textit{apex vertices} to $G$. We say that $G$ is $(g, p, k, a)$-\textit{almost embeddable}. Robertson and Seymour \cite{robertsonGraphMinorsXVI2003} proved that all graphs with no $K_t$ minor are \textit{clique-sums} of $(g, p, k, a)$ almost-embeddable graphs, with $(g, p, k, a)$ depending on only $k$. 
\par
We have some useful results that can be paired with the Graph Minor Structure Theorem to prove this result.
\begin{itemize}
	\item From Heath and Istrail, all graphs of bounded genus have bounded pagenumber \cite{heathPagenumberGenusGraphs1992}.
	\item From Ganley and Heath \cite{ganleyPagenumberTrees2001}, and Dujmovic and Wood \cite{dujmovicGraphTreewidthGeometric2007}, all graphs of bounded treewidth have bounded pagenumber.
	\item From Hickingbotham and Wood\cite{hickingbothamStackNumberCliqueSum2023}, \textit{tree-decompositions} of graphs where every \textit{torso} has bounded pagenumber has bounded pagenumber. 
\end{itemize}

\subsection{Layout of report}
\begin{itemize}
	\item \cref{chap:Definitions} formally describes important definitions and concepts that will be used throughout the rest of the report.
	\item \cref{sec:BoundedPagenumber} formally defines a book-embedding and pagenumber of a graph $G$, and describes some cases where the pagenumber is completely defined for a family of graphs $\mathcal{G}$. Additionally, we describe some related graph parameters to pagenumber.
	\item \cref{sec:treewidth} describes treewidth and tree-decompositions, which are important concepts in the Graph Minor Structure Theorem. We formally define treewidth and give the treewidth of some family of graphs. 
	\item \cref{sec:Kt_Minor_Free} describes the Graph Minor Structure Theorem, but we do not discuss the proof.
	\item \cref{sec:BoundedPagenumber} describes some results about graphs with bounded pagenumber. One of the most important results is Yannakakis's proof that planar graphs have pagenumber at most  4 \cite{yannakakisEmbeddingPlanarGraphs1989}. This proof forms the basis for many other proofs in this paper. We also show that clique-sums of graphs of bounded pagenumber also have bounded pagenumber, and graphs of bounded treewidth have bounded pagenumber.
\end{itemize}

\chapter{Definitions}\label{chap:Definitions}
For a graph $G$, we define the \textit{vertex} and \textit{edge} sets of $G$ to be $V(G)$ and $E(G)$ respectively.

For a subset of vertices $A \subseteq V(G)$, we denote the \textit{induced subgraph} on $G$ with vertex set $A$ as $G[A]$. 
\par
A \textit{path} in a graph $G$ is a sequence of edges $e_1, e_2, ..., e_{\ell- 1}$ which join a sequence of vertices $v_1, v_2, ..., v_{\ell}$ such that $e_i = v_iv_{i + 1}$, and all the vertices are distinct. 
A \textit{simple cycle} $C$ in a graph $G$ is a sequence of edges $e_1, e_2, ..., e_{\ell}$ which join a sequence of vertices $v_1, v_2, ..., v_{\ell}$ such that $e_i = v_iv_{i + 1}$ for $1 \leq i \leq \ell - 1$ and $e_\ell = v_\ell v_1$. 
A \textit{Hamiltonian cycle} in a graph $G$ is a simple cycle $C$ such that all vertices in $G$ appear in $C$.
\par
We say that a graph $G$ is \textit{$k$-connected} if between any two vertices $v, w$ in $G$, there are $k$ vertex-disjoint paths between $v$ and $w$. For the case that $k = 2$, the graph is \textit{biconnected}. A graph $G$ with a Hamiltonian cycle is biconnected. 
\section{Planar graphs}
We say a graph $G$ is \textit{planar} if $G$ can be drawn on the Euclidean plane $\Sigma$ such that no two edges cross. Say $G$ is drawn on a plane. If $G$ is embedded on $\Sigma$, then $\Sigma$ is divided into regions where no edges cross. The \textit{outerface} is the face on the outside of the graph. We say that a set of vertices \textit{lie} on a face if they are on the boundary of the face. We also say that the vertices \textit{bound} a face. $G$ is \textit{outerplanar} if $G$ is planar and all vertices in $G$ lie on the outerface. 
Let $F(G)$ be the set of faces of $G$ embedded on $\Sigma$. Then we have that:
\begin{theorem}[Euler's formula]\label{thm:Euler_planar}
	\begin{equation}
		|V(G)| - |E(G)| + |F(G)| = 2
	\end{equation}
\end{theorem}

We can use this result to bound the number of edges in an outerplanar graph.
\begin{theorem}\label{thm:outerplanar_bound}
	If $G$ is outerplanar with $n$ vertices and $m$ edges, then $m \leq 2n - 3$.
\end{theorem}

\begin{proof}[Proof of theorem]
	Suppose $G$ is maximal outerplanar, meaning adding any edge will break the outerplanar property. Let the internal faces be all the faces which are not the outerface, and let there be $f$ faces. Then the outerface has $n$ edges on the boundary, but each internal face will have exactly $3$ edges on the boundary. However, each edge is touching two distinct faces. Therefore we have that
	\begin{equation}
		3 f - 3 + n = 2m
	\end{equation}
	Combining with \cref{thm:Euler_planar} given by
	\begin{equation}
		n - m + f = 2
	\end{equation}
	
	we have, after some rearrangment, 
	\begin{equation}
		2n = 3 + m
	\end{equation}
	Therefore, $m = 2n - 3$. As every outerplanar graph is a subgraph of a maximal planar graph, then we have that $m \leq 2n - 3$. 
\end{proof}
\section{Graph minors}
We say that a graph $H$ is a \textit{minor} of another graph $G$ if a graph isomorphic to $H$ can be obtained from $G$ by deleting vertices, edges, and \textit{contracting} edges. To \textit{contract} an edge $uv$, we delete both $u$ and $v$ and create a new vertex $w$ such that $N_G(w) = N_G(u) \cup N_G(v)$, where $N_G(v)$ is the neighbourhood of $v$. We define minors up to isomorphism, and we typically omit the isomorphism relation in our definitions. We say that a graph $G$ is \textit{$H$-minor-free} if there is no sequence of deletions and contractions of $G$ that yield a graph isomorphic to $H$. We say that a family of graphs $\mathcal{F}$ is minor-closed if $G \in \mathcal{F}$, then a graph isomorphic to the minor $G'$ of $G$ is also in $\mathcal{F}$. 

\begin{example}
	An example of a minor-closed class is the class of planar graphs. 
\end{example}

An important class of graph families are the $K_t$-minor free graphs. For a graph $G$, we define \textit{Hadwiger's number} $\had(G)$ to be the largest $t$ such that $K_t$ is a minor of $G$. This is named after Hugo Hadwiger and one of his most famous conjectures.

\begin{conjecture}[Hadwiger's conjecture]\todo{Add citation}
	For all graphs $G$, $\chi(G) \leq \had(G)$.
\end{conjecture}
where $\chi(G)$ is the \textit{chromatic number} of $G$. 

\subsection{Minors and models}
A \textit{model} of a graph $H$ to a graph $G$ is a function $\rho$ which assigns to $H$ vertex disjoint connected subgraphs of $G$, such that if $uv \in E(H)$, then some edge in $G$ connects a vertex in $\rho(u)$ to $\rho(v)$. This definition comes from Sergey Norin in his notes on graph minors \cite{norinMath599GraphMinors2017}. 

\begin{theorem}
	$H$ is a model of $G$ iff $H$ is a minor of $G$. 
\end{theorem}
\todo{Prove this!}
\section{Book embedding}\label{ssec:Book Embedding}
A \textit{book} of \textit{thickness} $k$ are $k$ half-planes glued together on a common boundary. We refer to the boundary as the \textit{spine} and we refer to the individual half-planes as \textit{pages}. In topology, these are referred to as \textit{fans} of half-pages. Books were described by Persinger and Atnosen in the 1960s \cite{persingerSubsetsNbooksE31966} \cite{atneosenOnedimensionalNleavedContinua1972}. 
A \textit{book-embedding} of a graph $G$ is an embedding of $G$ on a book. We place the vertices of $G$ on the mutual boundary of all half-planes, and we place the edges on each half-plane such that no two edges cross.

The \textit{pagenumber} of a graph $G$ is the minimum number of pages required to embed $G$ on a book. This is also referred to as \textit{book-thickness}, or \textit{stack-number}. 

We have another, more combinatorial definition, which abstracts the underlying topology and focuses on the graph.
A \textit{book embedding} of a graph $G$ is an arrangement of the vertices of $G$ in a total ordering $v_1 < v_2 < ... < v_n$. We then colour the edges $E(G)$ such that if we have $v_a < v_b < v_c < v_d$ and edges $v_a v_c$ and $v_b v_d$, then they are each assigned different colours.
We refer to the total ordering of $V(G)$ in the book embedding as $(<)$ or as $(\leq)$. For a book-embedding $(v_1, v_2, ..., v_{|G|})$, we refer to the edges $\left\{v_1 v_2, v_2 v_3, ..., v_{|G| - 1}, v_{|G|}v_{1}\right\}$ as \textit{spine edges}.
We may use a \textit{circular ordering} of the vertices rather than a linear ordering, so we ignore the start and end vertices.

Book-embeddings were introduced by Kainen and Ollmann in the 1970s. \cite{kainenRecentResultsTopological1974} \cite{ollmannBookThicknessVarious1973}. It was developed further in a paper by \cite{bernhartBookThicknessGraph1979}. 
\begin{lemma}\label{lem:Pagenumber_1}
	A graph $G$ has page-number at most 1 iff $G$ is outerplanar.
\end{lemma}
\begin{proof}
	We can choose an ordering of the vertices $V(G)$ to go anticlockwise around the outer face. Suppose we have two edges $uv$, $xy$ that cross, so that $u < x < v < y$. Then in the original graph embedding, we will have that $uv$ and $xy$ will cross inside the circle, thus $G$ is not outerplanar. 
\end{proof}
\begin{lemma}\label{lem:Pagenumber_2}
	A graph $G$ has pagenumber at most 2 iff $G$ is a subgraph of a planar graph with a Hamiltonian cycle.
\end{lemma}
This is because we can embed the graph on a sphere with the vertices and Hamiltonian cycle on the equator, and the edges forming the interior and exterior edges of the cycle respectively.

\subsection{Pagenumber of complete graphs}\label{ssec:Pagenumber_Complete_Graphs}
This is an upper bound of any graph $G$ with vertices $n$. 
\begin{theorem}[\cite{bernhartBookThicknessGraph1979}]\label{thm:Pagenumber_Complete_Graph}
	The complete graph $K_n$ on $n$ vertices, $n \geq 4$, has pagenumber $\lceil n/2 \rceil$. 
\end{theorem}
 Therefore for any graph $G$ on $n$ vertices, $n \geq 4$, $\pn(G) \leq \lceil n/2 \rceil$. 
\todo{Add picture of proof}
\subsection{Related graph properties}\label{ssec:Related_Properties}

\begin{lemma}[Bound on number of edges \cite{bernhartBookThicknessGraph1979}]\label{lem:Edge_Bound}
	If an $n$-vertex graph $G$ has $\pn(G) = k$, then $G$ has at most $n + k(n-3)$ edges.
\end{lemma}
\begin{proof}
	Given a vertex ordering $v_1 \leq v_2 \leq ... \leq v_n$, we have that the spine edges $v_i v_i+1$, $v_1 v_n$  can appear on any page. Furthermore we have there are at most $n-3$ non outer-cycle edges on each page as the maximum number of edges in an outerplanar graph is $2n - 3$ from \cref{thm:outerplanar_bound}, but we remove the outer cycle (with $n$ edges on the cycle) to have at most $n-3$ edges on each page. Therefore, $m$, the number of edges, satisfies $m \leq n + k (n - 3)$. 
\end{proof}
\begin{theorem}[Chromatic number bound\cite{bernhartBookThicknessGraph1979}]\label{thm:Colouring_Bound}
	For all graphs $G$, $\chi(G) \leq 2 \pn(G) + 2$.
\end{theorem}
\begin{proof}
	Let $\pn(G) = k$ and suppose $G$ has $n$ vertices and $m$ edges. Then we have that the average degree of $G$, $d(G) = 2m/n$ by the handshaking lemma. But $2m/n \leq 2 \frac{n + k(n-3)}{n} < 2k + 2$. But this implies that $G$ has a vertex of degree $\leq 2k + 1$, and as if $G'$ is a subgraph of $G$, then $G'$ also has $\pn(G') \leq k$, thus $G'$ has a vertex of degree at most $2k + 1$. However, this implies $G$ is $2k + 1$-degenerate, thus $\chi(G) \leq 2k + 2$. 
\end{proof}
We may note that this bound is not tight. \todo{Find example of graph where this is not tight. Not $K_t$.}

\subsection{Historical interest}\label{ssec:Pagenumber_History}
Pagenumbers were developed for processor designs, but more recently have been used in bioinformatics and the like. 
The project of finding upper and lower bounds of the pagenumber of planar graphs was started by Bernhart and Kainen when they conjectured that planar graphs had unbounded page-number. However, Buss and Shor\cite{bussPagenumberPlanarGraphs1984} found that the pagenumber of planar graphs was at most 9, and Heath \cite{heathEmbeddingPlanarGraphs1984} found that the pagenumber of planar graphs is at most 7. Yannakakis' devised an algorithm to bound the pagenumber to at most 4 \cite{yannakakisEmbeddingPlanarGraphs1989}. Yannakakis, in this paper, claimed that there exist planar graphs with pagenumber 4. However, his proof was incomplete and the full proof was left unpublished. However in 2020, Yannanakis published his full proof. \cite{yannakakisPlanarGraphsThat2020} At around the same time, Kaufmann, Bekos, Klute, Pupyrev, Raftopoulu and Ueckerdt published a similar result\cite{kaufmannFourPagesAre2020}. 

\section{Treewidth}\label{sec:treewidth}

The \textit{treewidth} of a graph $G$ measures how far $G$ is from being a forest \cite{diestelGraphMinors2017}. 

\begin{definition}[Tree-decomposition]\label{def:tree-decomposition}
	The tree-decomposition $\tree$ of a graph $G$ is defined as a tree $T$ with associated \textit{bags} $\lbrace B_x : x \in V(T) \rbrace$ such that:
	\begin{itemize}
		\item $\bigcup_{x \in V(T)} B_x = V(G)$
		\item for all $v \in V(G)$, the subset of vertices $\lbrace x \in V(T): v \in B_x \rbrace$ in $V(T)$ induces a connected subtree in $V(T)$.
		\item For all edges $vw \in E(G)$, there exists a bag $B_x$ such that both $v$ and $w$ are in the bag $B_x$.
	\end{itemize}
\end{definition}
We refer to the vertices of the tree $T$ as \textit{nodes}. 
The \textit{width} of the tree decomposition $\tree$ is defined as $\max \lbrace |B_x| - 1 : x \in V(T) \rbrace$. We define the \textit{treewidth} of a graph $G$ as such:


\begin{definition}\label{def:treewidth}
	The treewidth of a graph $G$, denoted as $\tw(G)$, is defined to be the smallest width for all tree decompositions of the graph $G$.
\end{definition}


\begin{example}\label{ex:treewidth_forest}
	$\tw(G) = 1$ iff $G$ is a forest.
	\begin{lemma}
		If $G$ is a forest, then $\tw(G) = 1$.
	\end{lemma}
	\begin{proof}
		Suppose $G$ is a tree. Root the graph $G$ at the vertex $r$. Then let $T = G$ and $B_x:= \lbrace x, p \rbrace$ where $p$ is the parent of $x$. The bag $B_r$ will just contain $r$. Then all edges $vw$ will be between parent $v$ and child $w$, so it will be in bag $B_w$. Finally, the subgraph induced by vertex $x$ in $T$ will be $x$ and the children of $x$, which is a connected subtree.
		
		If $G$ is a forest, then we perform this operation on every connected component of $G$ and connect the roots to form a new tree. Then this tree is a tree-decomposition. This forms a tree-decomposition of width at most 1. 
	\end{proof}
	\begin{lemma}
		If $\tw(G) = 1$, then $G$ has no cycles.
	\end{lemma}
	\begin{proof}
		If $G$ has a cycle $C$, then the treewidth cannot be 1. This is because if there is a tree decomposition $\tree$ where the size of each bag is at most 2, then as the graph must have every edge, then every edge in $C$ is in separate bags. However, we have that for any vertex $v$ in $C$ to have an induced connected subgraph in $T$, then it follows that the cycle $C$ is also in $T$. Thus $T$ is not a tree, and this is not a valid tree-decomposition. 
	\end{proof}
\end{example}

\begin{lemma}[Helly Property]\label{lem:Helly}
	Let $T_1, ..., T_k$ be subtrees of a tree $T$ such that for every pair of trees, there is a vertex in common. Then there exists a vertex which is common to all trees.
\end{lemma}
\begin{proof}[Helly property]
	If $T_1$, $T_2$ and $T_3$ are subtrees of $T$ such that the vertex sets are pairwise nonempty, then there is a common vertex in all three subtrees. If this is not the case, denote $v_1$ as a vertex in the intersection of $T_1$ and $T_2$, $v_2$ as the vertex in $T_1 \cap T_3$, and $v_3$ as the vertex in $T_2$ and $T_3$. Then there exists a unique path $P$ in $T_1$ from $v_1$ to $v_2$.
\end{proof}

\begin{theorem}[Clique theorem]\label{thm:clique}
	In any tree-decomposition of $G$, for every clique $C$ in $G$, there exists a node $x \in V(T)$ such that $C \subseteq B_x$. 
\end{theorem}

\begin{proof}
	Let $\tree$ be a tree-decomposition. Every vertex $v$ induces a connected subtree in $T$, call it $T_v$. Then for any two vertices $x, y$ in $C$, we have that $T_x$ and $T_y$ must intersect as the edge $xy$ is inside a bag $B_z$ corresponding to a node $z$. Then by the Helly property, there exists a node $v$ such that $C \subseteq B_v$.
\end{proof}

\begin{corollary}\label{cor:complete_tw}
	$\tw(K_n)$ is $n-1$. 
\end{corollary}

\begin{theorem}\label{thm:tw_minor_closure}
	If $H$ is a minor of $G$, then $\tw(H) \leq \tw(G)$. 
\end{theorem}
\begin{proof}[Proof of minor]
	Suppose we have a tree-decomposition $\tree$ of $G$. If we delete an edge in $G$, then $\tree$ remains a valid tree-decomposition. If we delete a vertex $v$, then $\tree$ where we remove $v$ from every bag in $\tree$ is also a valid tree-decomposition. If we contract an edge $vw$, creating a new vertex $u$, then relabeling $v$ and $w$ in all bags to $u$ is a valid tree-decomposition as the induced subtree of $u$ is the union of the induced subtrees of $v$ and $w$, and every neighbor of $v$ or $w$ is a neighbor of $u$. But the edges in the neighborhood do not change. Thus this is a valid tree-decomposition, with width at most the width of $\tree$.
\end{proof}

Recall that an outerplanar graph is a planar graph where there exists a face such that all vertices lie on the boundary of that face. 
\begin{example}\label{ex:tw_outerplanar}
	The treewidth of an outerplanar graph is at most 2.
\end{example}
\begin{proof}[Proof of outerplanar treewidth.]
	Let $G$ be an outerplanar graph, and let $G'$ be the triangulation of $G$. As $G$ is a minor of $G'$, $\tw(G) \leq \tw(G')$. We look at the \textit{weak dual} of $G'$. This is a tree $T$, where every node $v_f$ in $T$ corresponds to a face $f$ in $G'$. Then let $B_{v_f}$ be the bag of the tree-decomposition, where $B_{v_f}$ is the set of vertices on the boundary of the face $f$. Then the tree $T$ with bags $B_{v_f}$ is a valid tree-decomposition of $G'$, where every bag has at most 3 vertices. Thus, $\tw(G) \leq 2$. 
\end{proof}

\subsection{Characteristics of treewidth}\label{ssec:characterising_Treewidth}
\subsubsection{$k$-trees}\label{sssec:k-trees}
We define a $k$-tree inductively. We have that the complete graph $K_k$ is a $k$-tree, and if $G$ is a $k$-tree, then we add a new vertex to $G$ that is adjacent to $k$ vertices that form a clique of size $k$ in $G$ results in a $k$-tree. 
A $k$-tree is a maximal graph with treewidth $k$. $\tw(G) \leq k$ iff $G$ is a subgraph of a $k$-tree. 


\subsubsection{Bounded treewidth graphs}\label{sssec:Graph_treewidth_Bounded}
\begin{theorem}\label{thm:treewidth_clique-minor-free}
	If $\tw(G) \leq k$, then $G$ is $K_{k+2}$-minor-free. 
\end{theorem}
\begin{proof}
	We shall prove the contrapositive: If $K_t$ is a minor of $G$, then $\tw(G) \geq t-1$.
	If $K_t$ is a minor of $G$, then we have that from \cref{thm:tw_minor_closure} that $\tw(K_t) \leq \tw(G)$. As $\tw(K_t) = r-1$, then we have that $\tw(G) \geq t - 1$. 
\end{proof}

\subsection{Historical discussion}\label{ssec:tw_historical}
Treewidth was introduced in \cite{berteleChapterEliminationVariables1972} with applications to dynamic programming under the name ``dimension''. It was then rediscovered by Halin \cite{halinSfunctionsGraphs1976} before most famously used in \cite{robertsonGraphMinorsIII1984}, which was introduced to prove the Graph Minor Theorem\cite{robertsonGraphMinorsXX2004}.


\section{Path-width}
Similarly to treewidth, the pathwidth of a graph $G$ is how far a graph is from being a path. 

We define the path-decomposition of a graph $G$ to be a sequence of bags $B_i$ such that the subsequence of bags containing a vertex $v$ induces a nontrivial subpath and each edge $vw$ is in a bag $B_i$. Then we define the width of a path-decomposition as $\max_i \lbrace |B_i| \rbrace -1$, same as treewidth.

If a graph has a path-decomposition $(B_i)_i$, then it has a tree-decomposition $\left((B_i)_i, P\right)$. Therefore,
\begin{equation}
	\pw(G) \geq \tw(G).
\end{equation}

Similarly to treewidth, we have the following observation.
\begin{lemma}
	The pathwidth of $G$ is the largest pathwidth over all connected components.
\end{lemma}
We say a graph $G$ is a caterpillar if $G$ has a path $P$ and every vertex is adjacent to a vertex on the path $P$. Alternatively, $G$ is a caterpillar if removing every leaf yields a path.
\begin{theorem}[Caterpillars]
	Graphs have pathwidth at most 11 iff every connected component is a caterpillar.
\end{theorem}
\begin{proof}[Caterpillars]
	\par($\Leftarrow$)
	Suppose $G$ is a caterpillar. Denote $P =\left( p_1, p_2, ..., p_n\right)$ as the central path. The leaves of vertex $p_i$ are denoted as $v_{i, 1}, v_{i, 2} ..., v_{i, k}$. Define the bags as $(v_{1, 1}, v_1), (v_{1, 2}, v_1)... (v_{1, j}, v_1), (v_1, v_2), (v_{2, 1}, v_2), (v_{2,2}, v_2,)... $. We can see that each leaf appears once and each vertex on the central path is on a subpath of the path. Therefore, the pathwidth of $G$ is 1.
	\paragraph{$\Rightarrow$}
	If $G$ has pathwidth 1, then for each connected component, we choose a vertex $v$ in $B_1$ and a vertex $w$ in $B_n$, the final bag, and look at a path from $v$ to $w$. This path must go through every bag, thus the non-path vertices must have neighbour only of the other one in the bag and thus the graph is a caterpillar. 
\end{proof}
\begin{example}[Complete graphs]
	$\pw(K_n) = \tw(K_n) = n - 1$. 
\end{example}
\begin{proof}
	We have that the pathwidth of $K_n$ is at least the treewidth of $K_n$. But we have that the pathwidth is at most $n- 1$ (where all the vertices are in the same bag), but the treewidth of $K_n$ is $n - 1$. Therefore, $\pw(K_n) = n - 1$. 
\end{proof}

\begin{example}
	The pathwidth of a tree $T$ is $\min_{P \subset T} \left\lbrace 1 + \pw(T - V(P))\right\rbrace $ where $P$ is a path. 
\end{example}

\chapter{Known results from structural graph theory}\label{chap:Known results}
In this chapter of the report, we outline important known results that will help us solve \cref{conj:bded_had_pn}.

\begin{itemize}
	\item \cref{sec:Kt_Minor_Free} is the Graph Minor Structure Theorem, and the full explanation of the Graph Minor Structure Theorem.
	\item \cref{sec}
	\item \cref{sec:BoundedPagenumber} are a series of proofs that can be used with the Graph Minor Structure Theorem to prove that each individual component of the structure theorem has bounded pagenumber.
	\begin{itemize}
		\item \cref{ssec:Clique_sum_Pagenumber_bound} proves that clique sums of bounded adhesion where each component has bounded pagenumber also has bounded pagenumber.
		\item \cref{ssec:Bounded_Treewidth} proves that graphs with bounded treewidth also have bounded pagenumber.
		\item \cref{ssec:pagenumber_bounded_genus} proves that all graphs with bounded genus have bounded pagenumber as well.
	\end{itemize}
\end{itemize}

\section{Graph Minor Structure Theorem}\label{sec:Kt_Minor_Free}
What is the structure of $K_t$-minor free graphs? We shall show that we can roughly characterise all $K_t$-minor free graphs as graphs that are products of a series of operations. This classification comes from \cite{robertsonGraphMinorsXVI2003}.
\subsection{Clique-minor-free minor-closed families}\label{ssec:Kt_Minor_Closed_families}
We define $\had(G)$ to be the largest $t$ such that $G$ has a $K_t$ minor. 
\subsubsection{Planar graphs}\label{sssec:K_5-free_Planar}
\begin{theorem}\label{thm:K5_Free_Planar}
	If $G$ is a planar graph, then $G$ is $K_5$-minor-free.
\end{theorem}
\begin{proof}
	If $G$ is planar with $n$ vertices and $m$ edges, then we have that $m \leq 3n -6$. However, we have that $K_5$ has $5$ vertices and $10$ edges, but we have that $ 10 > 3 \times 5 - 6$, so $K_5$ is not planar. As the family of planar graphs is minor-closed, then if $G$ is planar, then $K_5$ is minor-free.
\end{proof}

\subsubsection{A short discussion of topology}\label{sssec:topology}
A surface $\Sigma$ is a topological space which, at every point, has a neighbourhood homeomorphic to a disk. There are three important surfaces to know- the sphere $S^2$, the torus $T$, and the real projective plane $P$.
\par
We \textit{add} a \textit{handle} to a surface $\Sigma$ by removing two disks in $\Sigma$ and identifying the boundaries such that one goes clockwise and the other goes counterclockwise. We add a \textit{crosscap} by removing a disk in $\Sigma$ and identifying opposite points on the boundary. We add a \textit{twisted handle} to a surface $\Sigma$ by removing two disks in $\Sigma$ and identifying the boundaries such that both go clockwise.
\par
The \textit{Euler genus} of a surface $\Sigma$ obtained from a sphere by adding $h$ handles, $c$ crosscaps and $t$ twisted handles is $2h + 2t + c$.

\begin{example}
	Here are the Euler genus of some important surfaces.
	\begin{enumerate}
		\item The Euler genus of the sphere is $0$.
		\item The Euler genus of the torus is $2$.
		\item The Euler genus of the projective plane is $1$. 
		\item The Euler genus of Klein bottles is $2$. 
	\end{enumerate}
\end{example}

Note that ``genus'' and ``Euler genus'' are two distinct concepts in topology. In this paper, when we discuss genus, we will always discuss \underline{Euler genus}.

We say a surface $\Sigma$ is \textit{orientable} if $\Sigma$ can be obtained from $S^2$ by only adding handles. An example of an orientable surface is the torus.

We say a surface $\Sigma$ is \textit{non-orientable} if $\Sigma$ can be obtained from $S^2$ by adding at least one crosscap or twisted handle. An example of a non-orientable surface is the projective plane or the Klein bottle. 

\subsubsection{Genus-g graphs}\label{sssec:Graph_genus}

We define the \textit{Euler Genus} of a \textit{graph} $G$ to be the smallest Euler genus $g$ surface $\Sigma$ such that $G$ can be embedded on $\Sigma$ without crossings (note that $\Sigma$ can be orientable or nonorientable).

Let $|F(G)|$ be the number of faces in a graph $G$. Then we have that $|V(G)| - |E(G)| + |F(G)| = \chi = 2 - g$. 

\begin{theorem}[Bounded genus]\label{thm:bounded_genus_kt_free}
	If $G$ is a genus $g$ graph, then $G$ is $K_t$-minor free, where $t > \sqrt{6g} + 4$. 
\end{theorem}
This proof mimics the above proof for planarity, but with higher dimensions. 
We can show that if $G$ has genus $g$, then if $G$ has $n$ vertices and $m$ edges, then $n - m + f = \chi = 2-g$, then as each face has at most 3 vertices and each edge is incident to two faces, we have that $f \leq 2m/3$. Therefore, $m \leq 3(n + g - 2)$, and if $K_t$ is embeddable on a genus $g$ graph, then $\binom{t}{2} \leq 3 (t + g - 2)$. Thus $t \leq \sqrt{6g} + 4$. So if a graph has genus $g$, then it is $K_t$-minor-free, where $t > \sqrt{6g} + 4$. 

\subsubsection{Apex vertices}\label{sssec:Apex_Vertices}
An apex vertex $a$ is added to a graph $G$ such that it has arbitrary edges. As such, $a$ can dominate all other vertices in $G$.
\begin{theorem}
	If $G$ is $K_t$-minor free, $G$ with the apex vertex $a$ is $K_{t+1}$- minor free. 
\end{theorem}
\begin{proof}
	We shall prove the contrapositive. Suppose $G + a$ has a $K_{t + 1}$ minor. Then $K_{t + 1}$ has a model in $G + a$ and denote the model function as $\rho$. Now let $v$ be the vertex in $K_{t + 1}$ such that $\rho(v)$ contains $a$. Then if we delete $v$ from $K_{t + 1}$ and delete all the vertices in $\rho(v)$, then we have that $K_t$ is a minor of $G'$, where $G'$ is $G + a - \rho(v)$. But $G'$ is a minor of $G$, as $G'$ does not contain $A$. But this means that $G$ has a $K_t$ minor. 
\end{proof}
\subsubsection{Clique-sums}\label{sssec:Clique_Sums}
The \textit{$k$-clique-sum} of two graphs $G$ and $H$, denoted as $G \# H$, is the graph obtained by performing a series of operation on the cliques of $G$ and $H$. We find cliques in $G$ and $H$, $C_G$ and $C_H$ respectively, such that $C_G$ and $C_H$ have size $k$. Then we identify the vertices in $C_G$ and $C_H$ so that $G$ and $H$ are connected to each other on this clique. 

\begin{lemma}
	If $G = G_1 \# G_2$,then $\had(G) = \max(\had(G_1), \had(G_2))$ and $\tw(G) = \max(\tw(G_1), \tw(G_2))$.
\end{lemma}

\begin{example}\label{ex:clique_sum_genus}
	If $G$ is the clique-sum of Euler genus $g$ graphs, then $G$ is $K_{\sqrt{6g} + 5}$-minor-free, but $G$ possibly has unbounded genus.
\end{example}

\begin{theorem}[Wagner's theorem\cite{wagnerUeberEigenschaftEbenen1937}]\label{thm:WagnersTheorem}
	If $G$ is $K_5$-minor-free, then $G$ can be obtained from $(\leq 3)$-clique-sums of planar graphs and the Wagner graph $V_8$.
\end{theorem}

The Wagner graph is given below.
\todo{Draw picture of Wagner graph.}


\subsubsection{Torsos and adhesion}\label{sssec:Torsos and Adhesion}
Given a graph $G$ and a tree-decomposition $\tree$, the \textit{torso} of a bag $B_x$ of $T$ is the graph $G\langle B_x \rangle$, obtained from $G[B_x]$ where $vw$ is a vertex in $G\langle B_x \rangle$ iff $v,w \in B_x \cap B_y$, where $y$ is any neighbour of $x$ in $T$. So the set $B_x \cap B_y$ for all neighbours $y$ of $x$ in $T$ is a clique in $G\langle B_x \rangle$. 
The \textit{adhesion} of a tree is defined as $\max(|B_x \cap B_y|)$ where $xy$ is an edge in $T$. Intuitively, it is the largest overlap between bags of a tree-decomposition

\subsubsection{Vortices}\label{sssec:vortices}
Let $G$ be embedded on a surface $\Sigma$, and let $F$ be a face on $G$. Let $D$ be a disc in $\Sigma$ such that $D$ only intersects $G$ only on vertices on the boundary of $F$. We denote these discs as $G$-clean. 

Then let $\Lambda = (x_1, x_2, ..., x_b)$ be a tuple of vertices on the boundary of $F$ such that they intersect $D$. Then we define a new graph $H$ such that $V(G) \cap V(H) = \Lambda$, and there is a \textit{path-decomposition} of $H$ of bags $B_1, B_2, ... B_b$ such that $x_i \in B_i$ for all $i$. $H$ is denoted as a \textit{$D$-vortex} of $G$. The width of a $D$-vortex is the width of the path above, or $\max_i(|B_i| - 1)$. 

The reason why vortices are important is because of the graph below. 

\todo{Draw graph with large treewidth, torso and adhesion, but with bounded $K_t$.}
\subsection{Robertson-Seymour Graph Minor Structure Theorem\cite{robertsonGraphMinorsXVI2003}}\label{ssec:Robertson_Seymour_Graph_Structure}
Given $g, p, a \geq 0$, $k \geq 1$, a graph $G$ is $(g, p, k, a)$- almost embeddable if there exists an $A \subseteq V(G)$ with $|A| \leq a$, and there exists subgraphs $G_0, G_1, ...,  G_{p'}$ of $G$ such that:
\begin{itemize}
	\item $G - A = G_0 \cup G_1 \cup G_2 ... G_{p'}$
	\item $p' \leq p$
	\item There is an embedding of $G_0$ onto a surface $\Sigma$ of genus $\leq g$
	\item There exists pairwise disjoint $G_0$-clean discs $D_1, D_2, ..., D_{p'}$ in $\Sigma$
	\item $G_i$ is a $D_i$-vortex of width at most $k$.
\end{itemize}

\begin{theorem}[Robertson-Seymour Graph Minor Structure Theorem for $K_t$-minor-free graphs]
	For all $t$, there exists $g, p, a \geq 0$, $k, \ell \geq 1$, such that every $K_t$-minor-free graph has a tree-decomposition of adhesion $\leq \ell$ and each torso is $(g, p, k, a)$-embeddable. We refer to the family of graphs which satisfy these constants as $\mathcal{G}(g, p, k, a)$. 
\end{theorem}
In fact, there exists a function $t(g, p, k, a)$ such that if a graph has a tree-decomposition of adhesion $\leq \ell$ and each torso is $(g, p, k, a)$-almost embeddable, then $G$ has no $K_t$ minor. Joret and Wood\cite{joretCompleteGraphMinors2013} found that
\begin{theorem}[Bounds on Graph Minor Structure Theorem\cite{joretCompleteGraphMinors2013}]\label{thm:graph_structure_bound_theorem}
	Then 
	$\had(G) \leq a + 48(k + 1)\sqrt{g + p} + \sqrt{6g} + 5$. Moreover, there exists an integer $n \geq a + 1 4 k\sqrt{p + g}$ such that $K_n$ is a minor of some $\mathcal{G}(g, p, k, a)$ graph.
\end{theorem}

\section{Graph Minor Theorem}
We move on to the most important and deepest theorem in structural graph theory, the Graph Minor Theorem. This proof was proven in a series of 20 papers by Robertson and Seymour, from 1983 to 2004. As part of the proof, the Graph Minor Structure Theorem was developed, as we have outlined above. 
\begin{theorem}[Graph Minor Theorem \cite{robertsonGraphMinorsXX2004}]
	Every infinite family of graphs contains two distinct graphs $G$ and $H$ such that $H$ is a minor of $G$.
\end{theorem}
Equivalently, this theorem states that:
\begin{theorem}
	Every minor-closed family of graphs can be characterised by a finite set of forbidden minors.
\end{theorem}
Importantly, graphs of bounded genus can be characterised as a set of forbidden minors.
For planar graphs, the two forbidden minors are $K_5$ and $K_{3,3}$. This was proven by Wagner \cite{wagnerUeberEigenschaftEbenen1937}. 
For graphs that can be embedded on a torus, there are at least 17,523 graphs which are forbidden minors, with a database maintained by Myrvold and Woodcock\cite{myrvoldLargeSetTorus2018}. 

\section{Bounds of pagenumbers of graphs}\label{sec:BoundedPagenumber} 
\subsection{Tree-decomposition into bounded page number components}\label{ssec:Clique_sum_Pagenumber_bound}
We denote $G(\mathcal{G}, T)$ as a clique-sum tree, where each vertex $v_i$ in the tree $T$ is a graph $G_i$ in $\mathcal{G}$, and the edge $v_iv_j$ corresponds with the clique sum of graphs $G_i$ and $G_j$. We refer to the stack layout as a tuple $(\leq , \psi)$ where $\leq$ is an ordering of the vertices and $\psi : E(G) \rightarrow X$ assigns to each edge the page. The page number is the smallest set $X$ on which this is defined. 
\begin{theorem}[Hickingbotham and Wood \cite{hickingbothamStackNumberCliqueSum2023}]\label{thm:clique_sum_pagenumber_bound}
	Let $G(\mathcal{G}, T)$ be a clique-sum tree where $\pn(G_i) \leq s$ for all $G_i \in \mathcal{G}$. Then $\pn(G(\mathcal{G}, T)) \leq 2s^2 + 4s + 1$.  
\end{theorem}

\subsubsection{Proof of above theorem.}
Let $C$ be a clique in $G$ and let $\sigma_C = (u_1, ... , u_k)$ be a vertex ordering of $V(C)$. For any arbitrary clique $J$, we define a rainbow-vertex $w \in V(J)$ as a vertex where for any $x, y \in V(J)$, the edges $wx$ and $wy$ are on different pages. We want the book embedding to have the structure $(\underbrace{u_1, u_2, ..., u_k}_{\text{Vertices in } C}, \underbrace{v_1, v_2, ..., v_l}_{\text{Vertices not in }C})$.

We will use these two results.
\begin{lemma}[Pagenumber]
	If $pn(G) \leq s$, then $G$ does not contain any cliques on more than $2s-1$ vertices, from \cref{thm:treewidth_clique-minor-free}
\end{lemma}

To prove this theorem, we will use a common technique in graph theory. We will strengthen the lemma so that we may use induction to prove the statement.
\begin{lemma}\label{lem:Hickingbotham_Lemma}
	Let $G$ be a graph where $\pn(G) \leq s$, and a clique $C$ with an ordering $\sigma_C$. There exists a $(2s^2 + 4s + 1)$-stack layout $(\leq, \psi)$ of $G$ where:
	\begin{enumerate}
		\item The vertex ordering has the structure $(\underbrace{u_1, u_2, ..., u_k}_{\text{Vertices in } C}, \underbrace{v_1, v_2, ..., v_l}_{\text{Vertices not in }C})$. 
		\item For every $u \in V(C)$, the edges $\lbrace uv \in E(G) : u \leq v \rbrace$ are a monochromatic star. 
		\item For every clique $J$, the last vertex of $J$ is a rainbow-vertex. 
	\end{enumerate}
\end{lemma}
\begin{proof}[Proof of \cref{lem:Hickingbotham_Lemma}]
	Let $(\leq_a, \psi_a)$ be a $s$-stack layout of $G$ and let $\rho: V(G) \rightarrow \lbrace 1, 2, ..., 2s + 2 \rbrace$ be a proper colouring of $V(G)$. This colouring exists because of \cref{thm:Colouring_Bound}.

Let $u_1, ..., u_k$ be the vertices of $C$ ordered by $\sigma_C$. Note that $k \leq 2s + 1$. Then the new ordering starts with $u_1 \leq u_2 \leq ..., \leq u_k$, and all vertices not in $K$ are placed after, according to $\leq_a$.
Then the stack assignment $\psi$ is now defined. For every edge $u_i v$ where $u_i \in V(C)$ and $u_i \leq v$, define $u_i v = i$. Otherwise, if neither $u$ or $v$ are in $V(C)$, and $u \leq v$, then let $\psi(uv) = (\rho(u), \psi_a(uv))$. Then we have at most $|\rho| |\psi_a| + k \leq (2s + 2) s + (2s + 1) = 2s^2 + 4s + 1$ pages.

We shall show that $(\leq, \psi)$ is a proper book-embedding. Suppose we have a pair of edges $uv$ and $xy$ which cross, and $\phi(uv) = \phi(xy)$. Suppose that $u$ is the smallest vertex in the ordering $\leq$. If $u \in V(C)$, then the edge $uv$ is assigned to its own page, meaning that it cannot cross $xy$. So $x = u$, but we can draw $uv$ and $uy$ on a single page. Thus they do not cross. Therefore we have that $u, v, x, y$ are not in $V(C)$, and as we preserve the original book-embedding, then these edges do not cross. Thus shown.
We have that property 1 and 2 are immediate. For property 3, consider a clique $J$ in $G$. Then we must show the last vertex of $J$ is rainbow. Suppose the last vertex of $J$ is $w$, and let $u, v$ be earlier vertices. Since $u$ and $v$ necessarily are assigned different colours in the colouring, then $\psi(uw) = (\rho(u), \psi_a(uw))$ and $\psi(vw) = (\rho(v), \psi_a(vw))$. Therefore, the two edges are on different pages. Thus $w$ is a rainbow vertex. 
\end{proof}

\subsubsection{Full proof}
\begin{theorem}
	Suppose $G = G(\mathcal{G}, T)$ be the clique-sum tree, with vertices $G_0, G_1, ..., G_k$, and for all $i \in \lbrace 0, 1, ..., k \rbrace$, we have that $\pn(G) \leq s$. Then there is a book-embedding of $G$ with pagenumber of at most $2s^2 + 4s + 1$. 
\end{theorem}

\begin{proof}
	To prove the statement, we shall prove the stronger statement that there exists a book-embedding with the properties described with the lemma above using induction. In particular, we will have that the last vertex of any clique $J$ is a rainbow vertex.
	
	Suppose $k = 0$. Then $G_0$ is a single graph with $\pn(G) \leq s$. Then by the lemma above, there is a book-embedding with pagenumber at most $2s^2 + 4s + 1$ with the property that it starts with $K$ and every last vertex in a clique is a rainbow vertex.
	
	Suppose $k = n$. Let $C$ be the clique between $G_n$ and the rest of $G$, where $G_n$ is a leaf of the tree $T$. Denote the induced subgraph $G[V(G) - V(G_n - C)]$ as $G'$. Then let $(\leq_n, \psi_n)$ be the $2s^2 + 4s + 1$-page-number book-embedding of $G_n$ that starts with $V(C)$, and let $\sigma_C$ be the ordering of $V(C)$ by $\leq_n$. Let $(\leq_{n-1}, \psi_{n-1})$ be the stack-embedding of $G'$. By induction, this has a $2s^2 + 4s + 1)$ book-embedding with the properties described above.
	
	\paragraph{Construction  of new book-embedding}
	Let $w \in V(K)$ be the last vertex of $K$ with respect to $\leq_{n-1}$. Then insert $V(G_n - C)$ between $w$ and its successor in the order of $\leq_{n}$. For the page assignment $\psi$, we have that if $uv \in E(G')$, then $\psi(uv) = \psi_{n-1}(uv)$. For the remaining edges, we can permute the edge assignments of $\psi_n$ such that for all $u \in V(K)$, we have that $\psi(E_u) = \psi_n(uw)$. We can do this as $w$ is a rainbow vertex and the edges $E_u$ are assigned to a unique page in $\psi_n$. Finally, let $\psi(uv) = \psi_n(uv)$ for the remainder of the edges. Denote the new ordering and assignment as $(\leq, \psi)$. 
	
	We claim that $(\leq , \psi)$ is a stack layout that satisfies the induction hypothesis. Suppose that $\psi(uv) = \psi(xy)$. If $uv, xy \in E(G')$, then by the induction hypothesis, they do not cross. Similarly, if $uv, xy \in E(G_n)$, then they will not cross, by the above lemma. If $uv$ is in $E(G')$ and $xy \in E(G_n)$, then they will go over each other or be sequential and therefore will not cross. 
	Finally, if $u, v, x, y \in C$, then by the induction hypothesis on $G'$, they do not cross either. The final case is if $uv \in E(G_{i + 1})$ and $u \in V(C)$, $v \notin V(C)$, $xy \in E(G')$. If $uv$ and $xy$ cross, then we have that $xy$ and $uw$ will cross. But this will contradict the page-embedding of $G'$.
	
	Let $J$ be a clique in $G$, and $w$ is its final vertex. Then if $J$ is in $G'$, then $w$ is a rainbow-vertex. Otherwise, the last vertex is contained in $G_n$. By the above lemma, $w$ must also be a rainbow vertex. Thus shown.
\end{proof}


\subsection{Bounded treewidth and page number}\label{ssec:Bounded_Treewidth}
\begin{theorem}[Ganley + Heath\cite{ganleyPagenumberTrees2001}]\label{thm:bded_treewidth_bded_pagenumber}
	Every graph $G$ with $\tw(G) \leq k$ has $\pn(G) \leq k + 1$. 
\end{theorem}

We use the characterisation of treewidth as a subgraph of a $k$-tree. We will show that all $k$-trees have pagenumber at most $k + 1$, which suffices to show the above theorem.

\begin{theorem}
	If $G$ is a $k$-tree, then $\pn(G) \leq k + 1$. 
\end{theorem}
\begin{proof}
	Consider a tree-decomposition of $G$, $\tree = (B, T)$, and perform a depth-first search on $T$, starting at an arbitrary root $r$. Let the ordering of the book-embedding $\sigma(v)$ of a vertex $v$ in $V(G)$ be determined by the first time $x \in T$ appears, where $v \in B_x$. For the vertices in the root bag $B_r$, we order arbitrarily. By the definition of a $k$-tree, only one vertex will be added in the depth-first search. Now consider the subtree $T_v$ induced by the bags $B_x$ containing $v$. We now consider colouring the subtrees $T_v$ for all $v \in G$ such that no overlapping subtrees have the same colour. Let $H$ be the intersection graph of the subtrees, where $V(H) = \lbrace T_v : v \in G \rbrace$ and $T_u T_v \in E(H)$ if there exists a bag $B_x$ such that $u, b \in B_x$. We have that $H$ is perfect, and thus $\chi(H) = \omega(H)$. Since $|B_x| = k + 1$ for all $x \in V(T)$, then there is a clique of size $k + 1$ in $H$. If there is a clique in $H$ with more than $k+ 1$ vertices, then this implies that there exists some $B_x$ such that $|B_x| > k + 1$. Thus $\chi(H) = k + 1$. 
	\paragraph{}
	We now use this to assign the edges of $G$ a page. Let $c(T_v)$ be the colour assigned to $T_v$. Colour each edge $uv \in E(G)$ as follows:
	\begin{equation}
		c(uv) = 
		\begin{cases}
			c(T_u) &\text{ if } \sigma(u) \leq \sigma(v),\\
			c(T_v) &\text{ if } \sigma(v) \leq \sigma(u)
		\end{cases}
	\end{equation}
	Then we claim that this is a proper book-embedding of $G$. Suppose we have that edges $uv$, $xy$ cross, so $\sigma(u) \leq \sigma(x) \leq \sigma(v) \leq \sigma(y)$. However, this implies that there exists a bag $B$ such that $u, x, v \in B$, as we have that $uv$ is an edge in $B$ and we do a depth-first search to establish the ordering, meaning that $u, v, x$ are in a clique. Therefore, they are in the same bags. However, this implies that the trees $T_u$ and $T_x$ intersect, meaning that $c(uv) \neq c(xy)$. Finally, the number of pages used is $\chi(H) = k + 1$, so $\pn(G) \leq k + 1$. Thus shown.
\end{proof}

\subsection{Planar graphs}\label{ssec:Planar_Graphs}
\begin{theorem}[Yannakakis \cite{yannakakisEmbeddingPlanarGraphs1989}] \label{thm:4Pages_Planar}
	Planar graphs can be embedded on at most four pages.
\end{theorem}

\subsection{Graphs embedded on a surface of bounded genus}\label{ssec:pagenumber_bounded_genus}
\begin{theorem}[Heath and Istrail\cite{heathPagenumberGenusGraphs1992}]\label{thm:Genus_pagenumber_bound}
	Let $g$ be the genus of a graph $G$. We have that for all graphs $G$, $\pn(G) \leq O(g)$ for some $g$.
\end{theorem}
Note that this bound extends the one found by Yannakakis \cite{yannakakisEmbeddingPlanarGraphs1989} to graph families of bounded genus. 
The best known bound is $\sqrt{g}$, found by Malitz\cite{malitzGenusGraphsHave1994}.

It was shown by Heath and Istrail that the family of graphs of bounded genus have bounded page-number. 
We refer to the ``layout'' of the graph as the book-embedding of the graph and ``embedding'' as the surface-embedding. We refer to orientable surfaces as genus $g$ as a sphere with $g$ handles, and a nonorientable surface of genus $g$ as a sphere with $g$ cross-caps. We define the orientable genus of a graph $G$, denoted $\gamma(G)$, as the minimum orientable surface genus that $G$ can be embedded on. The nonorientable genus of a graph $G$, denoted $\tilde{\gamma}(G)$, is the minimum nonorientable genus surface that $G$ can be embedded on. Mohar\cite{moharOrientableGenusGraphs1998} claims that $\tilde{\gamma}(G) \leq 2 \gamma(G) + 1$ for all graphs, meaning that if the orientable genus is bounded, then the non-orientable genus is bounded. Note that, Auslander et al.\cite{auslanderImbeddingGraphsManifolds1963} showed that there exists graphs which are embeddable on the projective plane who has arbitrarily large orientable genus. 
\paragraph{Proof}
We say that the embedding is $2$-cell if every face is homeomorphic to an open disc in $\mathbb{R}^2$. Any embedding of $G$ onto an orientable surface is a 2-cell embedding, but this does not hold for nonorientable surfaces, but we assume there exists a $2$-cell embedding.
Heath and Istrail rely on decomposing the graph $G$ of genus $\gamma(G)$ into a planar spanning subgraph $G_p$ of $G$ such that:
\begin{enumerate}
	\item The edges in $E(G) - E(G_p)$ attach to the boundary vertices of $V(G_p)$. 
	\item Adding an edge from $E(G) - E(G_p)$ to $G_p$ breaks the above condition. 
\end{enumerate}
To talk about graphs embedded in surfaces, we assign to each face a cyclic permutation $\sigma_v$ which represents the sequence of vertices encountered when traversing the boundary of a face in counterclockwise order.

This is enough to represent any graph in an orientable surface, but not enough for a non-orientable surface. We have to attach on an orientation to each edge, where each edge is either orientation-preserving or orientation-reversing. 

We have that a planar-nonplanar decomposition of $G$ is a triple $(R, G_P, E_N)$ where $R$ is a rotation of $G$ representing the surface embedding on the surface $S$, $G$ is a spanning planar graph, and $E_N = E - E(G_P)$. 
This satisfies a list of properties:
\begin{enumerate}
	\item The subrotation induces a planar embedding of $G_p$, so we can arrange $G$ on the surface $S$ such that the embedding of $G_p$ is planar. 
	\item For each $vw \in E_N$, we have that $v$ and $w$ live on the outerface $F_0$.
	\item $E(G_P)$ is maximal, so we cannot add edges from $E_N$ to $G_P$ without breaking properties 1 and 2. 
\end{enumerate}

\subsubsection{Decomposing graphs on surfaces}\label{sssec:Planar_nonplanar_decomp}
We first have to know that the planar-nonplanar decomposition exists. 

Suppose $G$ is embedded on an surface $\Sigma$. Then we wish to triangulate $G$ to form $G_T$. We choose a single triangle as the starting point and we add traces to the planar part incrementally. At each step, we set $G_P$ to be the current planar part and $E_N$ as the edges that are outside the planar part. There are two types of edges in $E_N$: edges which have both endpoints in $V(G_P)$, so cannot become edges of $G_P$, and edges that have either one or no endpoints in $V(G_P)$. 

We want to maintain the property that if $v \in G_P$, and edge $vw \in E_n$, then $v$ is a vertex on the boundary of $G_p$. 
\paragraph{Adding vertices to biconnected block}
For a current boundary of the outerface of $G_P$, if $v_i \rightarrow v_j \rightarrow v_k$ is trace with no edge of $E_N$ incident to $v_j$, then $v_iv_k \in E(G_T)$ is called a safe edge. If $v_i \rightarrow v_j$ is on the boundary of $G_P$, and $v_k \notin V(G_P)$, and $v_i,v_j,v_k$ is the boundary of a face, then $v_k$ is a safe vertex and we can add it to $G_P$. 
\paragraph{Creating new biconnected block}
If no $v_k$ exists, then we find a $w'$ which is the newest vertex in $V(G_P)$ adjacent to a vertex not in $V(G_P)$. We have that there exists a triangle $(x, w', z)$ on the boundary of $G_P$. Then we have that $z$ is unsafe and $xz$ and $w'z$ are essentially nonplanar. Then we let $w'w$ be the edge such that $w$ is not in $V(G_P)$ and we let $(w', y)$ be the next nonplanar edge encountered. 
We then add all safe edges to the picture, as we are finished with the edges inside.
\todo{Add pictures! this proof needs lots of pictures}

\subsubsection{Level sets and cycles}
On a planar graph $G$, we want to separate out vertices depending on how far away they are from the outerface. We fix a single outerface $F_0$ and define the first level set $V_0$ as the vertices adjacent to $F_0$. We then define the $i$-th level set, $V_i$ inductively. Consider the induced graph on $V(G) - \cup_{k = 0}^{i-1} V_k$. Then we define the vertices adjacent to $F_0$ in this induced graph, where we expand $F_0$ to include the vertices. This partitions $V(G)$.

We then define $C_0$ to be the edges adjacent to $F_0$ in this decomposition. Then we want $C_i$ to be the edges adjacent to $F_0$ in this decomposition. We define the chord edges $K_i$ to be the edges between vertices in $V_i$ that are not edges in $C_i$. Finally, we define the edges between faces, $E_i$ as the edges that are between vertices on level $V_i$ and $V_{i + 1}$.

\begin{lemma}
	For all faces $F$ in $G$, the vertices around $F$ are either all in one $C_i$ or they are in $C_i$ and $C_{i + 1}$ for some $i$.
\end{lemma}

\begin{proof}
	Let $i$ be the smallest value such that $v \in V_i$ is on the boundary of $F$. Now we have that $G[V(G) - \cup_{j = 1}^{i} V_i]$ will also remove $v$. However, this removes all the edges next to $v$, therefore all vertices that are on the boundary of $F$ will either be in $V_i$ or $V_{i + 1}$.
\end{proof}
We refer to the faces that have vertices in only $V_i$ as chordal and the faces that are between $V_i$ and $V_{i + 1}$ as non-chordal.

We define a weak triangulation of $G$ to be a triangulation $G'$ such that all faces except for the outerface is a triangulation.
\begin{lemma}
	There exists a weak triangulation of $G$, $G'$ which preserves the level sets $V_i$ and edge sets $E_i$, $C_i$, $K_i$ for all $i$. 
\end{lemma}

\begin{proof}
	If $F$ is a chordal face of $G$, then any triangulation maintains the property. If $F$ is nonchordal and the boundary has edges in $V_i$ and $V_{i + 1}$, then add edges that are only between vertices in $V_i$ and $V_{i + 1}$. This will suffice to build a new triangulated graph $G'$ where all vertices and edges are in the correct place. 
\end{proof}

\subsubsection{Classifying nonplanar edges according to homotopy class}

We can then form a directed cycle $C_0$ induced by $F_0$. Each vertex on the boundary of $F_0$ appears at least once, and twice if it is an articulation point, and each edge on the boundary of $F_0$ is encountered at least once on this cycle. Heath and Istrail refer to a directed subpath of the cycle $C_0$ as a trace, so trace $T = v_1 \rightarrow v_2 \rightarrow ... \rightarrow v_t$. The inverse trace is $T^{-1} = v_t \rightarrow v_{t-1} \rightarrow ... \rightarrow v_1$. We now wish to partition $E_N$ into equivalence classes. Suppose that $u_1v_1, u_2v_2 \in E_N$ are part of the boundary of the same face $F$ on the embedding of $G$. Then $u_1v_1$ and $u_2v_2$ are homotopic (with respect to $F$) if:
\begin{enumerate}
	\item $u_1v_1$ and $u_2v_2$ are the only edges of $E_N$ on the boundary of $F$
	\item There exist traces $T_u = u_1 \rightarrow ... \rightarrow u_2$ and $T_v = v_1 \rightarrow ... \rightarrow v_2$ such that $T_u$ and $T_v$ are on the boundary of $F$.
\end{enumerate}
We may think of $G_n$ as living on a locally flat part of $S$ and the homotopy class $u_1v_1$ and $u_2 v_2$ living on a handle (alternatively, passing through a crosscap such that they bound a face) of the surface such that if we take $G_n$ to a point, there exists a homotopy from $u_1v_1$ to $u_2v_2$. These form equivalence classes of the vertices.

\begin{lemma}
	If $C$ is a homotopy class of edges $u_1v_1, ..., u_kv_k$ with a natural order, then we can build traces $T_1$ and $T_2$ by building the trace from $u_1$ to $u_k$ passing through all $u_i$, and $v_1$ to $v_k$ passing through all $v_i$. 
\end{lemma}
We refer to a homotopy class as orientable if $T_1$ and $T_2$ go in opposite directions, and non-orientable if $T_1$ and $T_2$ go in the same direction.

\begin{lemma}
	We have that if $G$ is embedded in an orientable surface, then every homotopy class is orientable.
\end{lemma}
\begin{proof}[Sketch]
	We have that if a homotopy class is non-orientable, then on the handle the class sits on, the edges must cross. However, we have the graph is embedded on the surface, therefore this cannot happen. Thus shown. 
\end{proof}

\begin{lemma}
	If $G$ is $2$-cell embedded on an orientable surface of genus $g$, then any planar-nonplanar decomposition has at most $6g-3$ homotopy classes. 
\end{lemma}
\begin{proof}
	Decompose $G$ to a $(R, G_P, E_N)$ decomposition of $G$. Suppose $E_N \neq \emptyset$. Then identify $G_P$ to a single point, and identify each homotopy class to a single edge. Then draw a circle around the point $G_P$, and place vertices where the circle intersects all edges. Then delete the vertex $G_P$, and call the new graph $H$. We have that $n = |V(H)|$, $m = |E(H)|$, $h$ is the number of homotopy classes, and $f$ is the number of faces. We have that $v - e + f = 2 - 2g$. Since $H$ is cubic as every vertex has two edges on the circle and one on the homotopy class, then $3v = 2e$ by the handshaking lemma. Since there is only one nonplanar edge for each homotopy class, $v = 2h$. The interior face of $H$ has $v$ incident edges, and the remaining $f-1$ faces have at least 6 incident edges each, as we can identify the two homotopy classes bordering a face with four edges together. Therefore, we have that $6(f-1) + v \leq 2e$, by double counting faces and edges. Thus, we have that $6g - e \geq v/2 = h$ by manipulating the inequalities. 
\end{proof}

\begin{lemma}
	If $G$ is $2$-cell embedded on an non-orientable surface of genus $g$, then any planar-nonplanar decomposition has at most $3g-3$ homotopy classes. 
\end{lemma}
\begin{proof}[Similar to above]
	Use Euler's identity $v - e + f = 2-g$. 
\end{proof}
\subsubsection{Proving graphs with bounded number of homotopy classes have bounded pagenumber}\label{sssec:bounded_pagenumber_homotopy}
\begin{lemma}\label{lem:planar_nonplanar_orientable}
	Suppose $G$ has a planar-nonplanar decomposition $(R, G_P, E_N)$ on an orientable surface $\Sigma$. Then $G$ can be embedded on at most $18g - 5$ pages.
\end{lemma}
\begin{proof}
	We use Yannikakis' proof \cref{ssec:Planar_Graphs} to lay out the nonplanar spanning subgraph $G_P$ on four pages, maintaining the cyclic order of vertices. Then we can combine each blocks to form a 4 page layout of the graph. For each homotopy class in $E_P$, we allocate three pages. One page is for vertices in the same block, and the other two pages are used for edges between blocks, the biconnected components of $G$. We need two as we could have some which span blocks in a way that forces them to be on different pages. Therefore, we need at most $4 + 3(6g - 3) = 18g-5$ pages if $G$ has a planar-nonplanar decomposition. 
\end{proof}

\begin{lemma}\label{lem:planar_nonplanar_nonorientable}
	Suppose $G$ has a planar-nonplanar decomposition $(R, G_P, E_N)$ on an non-orientable surface $\Sigma$. Then $G$ can be embedded on at most $9g - 1$ pages.
\end{lemma}
\begin{proof}[Proof sketch]
\todo{Flesh out details completely}
We want to add edges in a controlled way so that the traces that are reversed become unreversed. This is done by adding edges between vertices so that we can invert the ordering on the circle such that we have that the vertices in one homotopy class have a non-crossing page embedding. However, an issue is chords that go between traces that are inverted. We go around this problem by removing chords and adding them to a separate page where there are finitely many pages wrt to the genus of the surface. Let $\mathcal{C}_{i,j}$ be a chord class when it is the set of chords that go between traces $T_i$ and $T_j$, where $T_i$ and $T_j$ both go clockwise or counterclockwise. Note that the number of chord classes $\mathcal{C}_{i,j}$ is bounded by the genus of the graph, and we can embed the chord classes that share a trace onto a single page. As there is a bounded number of chord classes, it must hold that the number of pages is finite. 
\end{proof}


\chapter{Potential proof techniques}\label{chap:Proving_The_Theorem}
Our aim is to show that graphs with bounded genus $g$ containing $p$ vortices of bounded width $k$ have bounded pagenumber $f(g, p, k)$. Thus we can show that for fixed $t$, all $K_t$-minor free graphs have bounded pagenumber $f(g, p, k, a, \ell)$. However, from \cite{hickingbothamStackNumberCliqueSum2023}, we can show that the clique-sums of graphs of bounded genus also have bounded genus.

We wish to find a book-embedding of a graph $G$ of bounded genus $g$ with vortices $G_1, ..., G_p$ of adhesion $k$ such that the pagenumber of $G$ is at most $f(g, p, k)$ for some constants $g$ and $p$. We do not worry about the $a$-case now.

\section{Case $g = 0$, planar graphs}
From Heath and Istrail, we can form a planar-nonplanar decomposition of $G$ of bounded genus $g$. Therefore, it makes sense to think about planar graphs first before thinking about graphs with bounded genus $g$.
\begin{theorem}\label{thm:clique_sum_connected}
	All graphs are clique-sum trees of $k$-connected subgraphs with adhesion at most $k-1$.
\end{theorem}
\begin{proof}
	We have that from Menger's theorem that if a graph $G$ is not $k$-connected, we can find a separator of size at most $k-1$ such that we have sets $A$ and $B$. Then we repeat this operation on $A$ and $B$ and we can construct a clique-sum tree where every component is $k$-connected and the adhesion size is at most $k-1$. 
\end{proof}

\begin{corollary}\label{corr:planar_graphs_4_connected_cliqesums}
	All planar graphs $G$ are clique-sum trees of $4$-connected planar graphs with adhesion at most $k-1$.
\end{corollary}
Planarity is a little harder to show, but it must be the case that the separator from Menger's theorem must lie on the outerface. Therefore, we can add edges between each vertices in the separator. 

We wish to prove the following:

\begin{conjecture}\label{conj:4-planar graphs}
	For all 4-connected planar graphs $G$ with an embedding $\Sigma$, with $i$ distinguished faces $\left\lbrace F_1, ..., F_i \right\rbrace$ and a distinguished clique $C$ of size at most $3$, there exists a book-embedding $(<, \psi)$ such that all edges bounding the faces $F$ are almost monochromatic and $C$ is at the start of the embedding with pagenumber at most $f(i)$.
\end{conjecture}
We plan to use a classic result from Tutte.
\begin{theorem}[Tutte\cite{tutteTheoremPlanarGraphs1956}]\label{thm:4-connected_planar_ham_cycle}
	All 4-connected planar graphs are Hamiltonian.
\end{theorem}

We also have a result by Thommassen, which strengthens Tutte's theorem. A graph $G$ is Hamiltonian-connected if for any distinct $x, y \in V(G)$, there exists a Hamiltonian path between $x$ and $y$.
\begin{theorem}[Thomassen \cite{thomassenTheoremPathsPlanar1983}]\label{thm:4 Connected Planar Ham-Connected}
	All 4-connected planar graphs are Hamiltonian connected.
\end{theorem}
Distinguished cliques of size 1 are easy to handle, and all faces will be on one page or another page sans the spine edges. If a face $F$ uses an edge $e$ on the spine, then we can put $e$ in its own special face. As there is a finite number of faces, then we have that the number pages needed only depends on $i$, the number of faces.

If we consider a distinguished clique $C$ of size 2 with vertices $uv \in C$, then we can use Thomassen's theorem to find a Hamiltonian path between $u$ and $v$, and we add the edge $uv$ to form a Hamiltonian cycle with the edge $uv$. Then this means that there exists a vertex ordering $<$ such that all faces are monochromatic. By the same argument above, the number of pages needed only depends on $i$.

The real problem are distinguished cliques of size $3$, $uvw$. 

\section{Apex vertices}
\begin{theorem}
	If $G$ is a graph with components $G'$ and $A$ apex vertices and $G'$ is a graph with pagenumber $s$, $a = |A|$, then $G$ has pagenumber $s + \left\lceil \frac{3a}{2}\right\rceil$. 
\end{theorem}
\begin{proof}
	Let $G'$ have book-embedding $(<, \rho)$. Then place the vertices of $A$ at the very start of $(<)$ and for every edge $u_iv$, $u_i \in A$, $v \in G'$, we colour $\rho(uv) = i$. Then for any edge $e \in E(G')$, we maintain the same colour as before. Then for the edges between vertices in $A$, we have that the number of colours is bounded above by $\left\lceil \frac{a}{2} \right\rceil$ from \cref{thm:Pagenumber_Complete_Graph}. Therefore, we have that $\pn(G) \leq \pn(G') + a + \left\lceil \frac{a}{2} \right\rceil =s + \left\lceil \frac{3a}{2}\right\rceil$. 
\end{proof}

\chapter{Conclusion}\label{chap:conclusion}
We conclude this report by explaining our current progress and our future plans. We have explained the Graph Minor Structure Theorem \cite{robertsonGraphMinorsXVI2003} and its application in our particular problem involving $K_t$-minor free graphs. We apply the Graph Minor Structure Theorem to the problem and explain some partial results which prove some components of $K_t$-minor free graphs, which are clique-sums, graphs of bounded treewidth, and graphs embeddable on surfaces. What will be the future area of research are graphs which are almost-embeddable on surfaces. We have to show that adding $p$ vortices of adhesion $k$ to surfaces of genus $g$ still has a bounded pagenumber, with the bounds depending only on $p, k$ and $g$. 
\printbibliography
\end{document}
