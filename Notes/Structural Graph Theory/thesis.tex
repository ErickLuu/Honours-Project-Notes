\documentclass[]{report}
\usepackage[margin = 1in]{geometry}

\usepackage{amsmath}
\usepackage{amssymb}
\usepackage{amsthm}
\usepackage[english]{babel}
\usepackage{url}
\usepackage{todonotes}
\usepackage{csquotes}

\usepackage{hyperref}
\usepackage[noabbrev, capitalise]{cleveref}

\usepackage{tikz}
\usetikzlibrary{graphs,graphdrawing, graphs.standard}
\usegdlibrary{trees}
\usepackage{svg}
\svgsetup{inkscapeexe="C:/Program Files/Inkscape/bin/inkscape.exe"}

\usepackage{graphicx}

\usepackage[style = numeric,
isbn=false,
url=false,
eprint = false,
maxbibnames=99
]{biblatex}
\renewbibmacro{in:}{}
\DeclareSourcemap{
	\maps[datatype=bibtex]{
		\map{
			\step[fieldset=url, null]
			\step[fieldset=extra, null]
			\step[fieldset=urldate, null]
			\step[fieldset=location, null]
		}
	}
}
\AtEveryBibitem{%
	\clearfield{day}%
	\clearfield{month}%
	\clearfield{endday}%
	\clearfield{endmonth}%
}

\addbibresource{Book-Embeddings.bib}
% Commands
\newcommand{\tree}{\mathcal{T}}
\DeclareMathOperator{\tw}{tw}
\DeclareMathOperator{\had}{had}
\DeclareMathOperator{\pw}{pw}
\DeclareMathOperator{\td}{td}
\DeclareMathOperator{\pn}{pn}
% Environments

\newtheorem{theorem}{Theorem}
\newtheorem{proposition}[theorem]{Proposition}
\newtheorem{corollary}[theorem]{Corollary}
\newtheorem{lemma}[theorem]{Lemma}
\newtheorem{definition}[theorem]{Definition}
\newtheorem{conjecture}[theorem]{Conjecture}

\theoremstyle{definition}
\newtheorem{example}[theorem]{Example}

\numberwithin{theorem}{section}
\numberwithin{equation}{section}

%includeonly
\includeonly{
chapters/introduction,
chapters/definitions,
chapters/knownresults,
	chapters/knownresults/gmst,
	chapters/knownresults/robertsproof,
	chapters/knownresults/heathandistrail,
chapters/proofattempt
}

%opening
\title{Towards a proof that all \(K_t\)-minor-free graphs have bounded pagenumber}
\author{Eric Luu}

\begin{document}

\maketitle
\listoftodos
\tableofcontents
\chapter{Abstract}\label{abstract}
In this report we outline our current progress in proving that \(K_t\)-minor free graphs have bounded pagenumber. Proving this bound will connect two important concepts in structural graph theory that have been studied extensively for the past 40 years. We will outline the most important theorem in structural graph theory related to \(K_t\)-minor free graphs, the Graph Minor Structure Theorem. We will also introduce some other proofs which will be of use proving this bounded pagenumber conjecture. 
\todo{Move basic definitions to start before the main body of introduction}

% !TEX root = ./thesis.tex
\chapter{Introduction}\label{sec:introduction}
Structural graph theory is a fundamental topic in graph theory. Many results from structural graph theory decompose graphs, or families of graphs, into smaller graphs with bounded parameters. One of the most important theorems in structural graph theory is \textcite{robertsonGraphMinorsXX2004} Graph Minor Theorem which states that every proper minor-closed graph family is characterised by a finite set of forbidden minors. Furthermore, structural graph theory is a field of interest, with connections to topological graph theory, extremal graph theory and algorithmic complexity. 

Topological graph theory is also a fundamental topic in graph theory. The main questions topological graph theory aims to solve are graphs embedded on topological spaces. Topological graph theory is used in data science, optimisation, and computer science. The topological spaces that this report is concerned on are surfaces and quotient spaces of surfaces. 

% !TEX root = ./thesis.tex
\section{Problem Statement}

A \textit{book-embedding} of a graph $G$ arranges the vertices of $G$ on the ``spine'' of a book and arranges the edges of $G$ on ``pages'' of a book. The \textit{pagenumber} of a graph \(G\) is the minimum number of pages necessary in a book-embedding of \(G\). The concept of the \textit{pagenumber} of a graph was introduced by Ollmann \cite{ollmannBookThicknessVarious1973} in the context of VLSI design and integrated circuitry. 
The driving question of this report is the following:
\begin{conjecture}\label{conj:bded_had_pn}
	There exists a function $f$ such that for all integers $t \geq 1$, every $K_t$ minor free graph $G$ can be embedded on $f(g)$ pages.
\end{conjecture}

In her PhD thesis, \textcite{Blankenship-PhD03} claimed to prove \cref{conj:bded_had_pn}. However, this result has not been published and has not been independently verified. Furthermore, a key proof used by Blankenship, from \textcite{heathPagenumberGenusGraphs1992}, was found to be missing crucial parts. \textcite{nakamotoBookEmbeddingProjectiveplanar2015} found that a crucial case was missed by Heath and Istrail.

We begin this report by discussing some background to the topics in our literature, which includes structural graph theory and topological graph theory. We introduce some basic concepts and definitions. We discuss planar graphs, graphs on surfaces, graph minors and the Graph Minor Structure Theorem, with a discussion on some famous theorems and conjectures connected to each topic. 

Robertson and Seymour showed that graphs with no \(K_t\) minor can be built from smaller building blocks. This is a rough overview of the building blocks. We first start with a graph \(G\) embedded on a genus \(g\) surface. Then we add on \(p\) \textit{vortices} to \(G\), with \textit{pathwidth} at most \(k\). Then we add on \(a\) \textit{apex vertices} to \(G\). We say that \(G\) is \((g, p, k, a)\)-\textit{almost embeddable}. Robertson and Seymour \cite{robertsonGraphMinorsXVI2003} proved that all graphs with no \(K_t\) minor has a \textit{tree-decomposition} where every \textit{torso} is a \((g, p, k, a)\) almost-embeddable graph, with \((g, p, k, a)\) bounded by a function of \(t\).

This honours project has two goals. The first goal is to investigate and learn more about structural graph theory. We will discuss some important machinery in structural graph theory, the main ones being the Graph Minor Theorem and the Graph Minor Structure Theorem. The second goal is to address an open problem within this field. To this end, an entire chapter building on the techniques discussed in previous chapters discusses this open problem. 

We prove that graphs that are almost-embeddable on a surface of genus $g$ with $p$ vortices of depth $k$ on some faces is embeddable in $f(g, p, k)$ pages, when the surface is orientable or the projective plane. Proving graphs embedded on nonorientable surfaces with higher genus remains a conjecture. 

%\subsection{Support for conjecture}
We have good reason to believe \cref{conj:bded_had_pn} is true. Firstly, \textcite{yannakakisEmbeddingPlanarGraphs1989} showed that every planar graph can be embedded on 4 pages. \textcite{heathPagenumberGenusGraphs1992} then showed that every graph of orientable genus $g$ can be embedded on $O(g)$ pages. Finally, \textcite{ganleyPagenumberTrees2001} showed that graphs with bounded treewidth have bounded pagenumber. \textcite{dujmovicGraphTreewidthGeometric2007} showed that the bound given by \citeauthor{ganleyPagenumberTrees2001} is tight.
We discuss some relevant papers that are used to prove \cref{conj:bded_had_pn}.
We aim to solve this question using the Graph Minor Structure Theorem \cite{robertsonGraphMinorsXVI2003}, which describes the structure of graphs that do not contain a \(K_t\) minor. 
We have some useful results that can be paired with the Graph Minor Structure Theorem to prove \cref{conj:bded_had_pn}.
\begin{itemize}
	\item From \textcite{heathPagenumberGenusGraphs1992}, every graph of bounded orientable genus have bounded pagenumber.
	\item From \textcite{ganleyPagenumberTrees2001}, and \textcite{dujmovicGraphTreewidthGeometric2007}, every graph of bounded treewidth have bounded pagenumber.
	\item From \textcite{hickingbothamStackNumberCliqueSum2023}, if a graph \(G\) has a \textit{tree-decomposition} where every \textit{torso} has bounded pagenumber, then \(G\) has bounded pagenumber.
	\item From \textcite{nakamotoBookEmbeddingProjectiveplanar2015}, all planar-projective graphs have bounded pagenumber.
\end{itemize}
These results individually show that the constituent ingredients of the Graph Minor Structure Theorem, except non-orientable surfaces of genus at least 2, have bounded pagenumber. We summarise some relevant technology that will be used to obtain some partial results for \cref{conj:bded_had_pn}. 
The biggest hurdle is showing that adding vortices on surfaces will not blow up the pagenumber. To address this issue, we introduce a new concept when considering faces on surfaces with a fixed book-embedding, monochromatic paths. 
The following list is how the rest of the report is laid out. 
\begin{itemize}
	\item \cref{chap:Definitions} contains definitions and concepts that will be used throughout the rest of the report. Some of these concepts are part of any undergraduate graph theory unit. Some other concepts, like book-embeddings and treewidth, are unlikely to appear in an undergraduate graph theory unit.
	\item \cref{chap:Known results} discusses some known results from graph theory, including the Graph Minor Structure Theorem. We discuss some proofs related to bounded pagenumber that can be used to prove \cref{conj:bded_had_pn}. The results we discuss are the Graph Minor Structure Theorem itself, a result from \textcite{heathPagenumberGenusGraphs1992}, a result from \textcite{ganleyPagenumberTrees2001} and a result from \textcite{hickingbothamStackNumberCliqueSum2023}. \cref{chap:Definitions} and \cref{chap:Known results} form the literature review section of the report.

	\item \cref{chap:Proving_The_Theorem} is an attempt at a proof to \cref{conj:bded_had_pn}. The main bulk of the argument is showing that the construction given by the Graph Minor Structure Theorem can be used to bound the pagenumber of the graph. Concepts and constructions in the literature introduced in the previous sections is used to show this result. 
\end{itemize}

Readers are expected to have at least an undergraduate understanding in graph theory and point-set topology. 

\newpage
\section{Background}\label{sec:background}
This section is a brief overview of some concepts in topological graph theory and structural graph theory. The concepts discussed are planar graphs, graphs embedded on surfaces, graphs on books, graph minors and the Graph Minor Structure Theorem. The Four Colour Theorem and the Map Colour Theorem are also discussed. These were conjectures for a very long time before being solved in the 1970s. The attempts to prove these theorems helped motivate the study of graphs on surfaces.

\subsection{Basic definitions}\label{sec: Basic definitions}
A graph $G$ is a pair of sets; a vertex set $V(G)$ and an edge set $E(G)$. $E(G)$ is a set that contains two-element subsets of $V(G)$. An edge $ \{v, w\}$ \textit{joins} vertices $v$ and $w$. A graph is \textit{simple} if all edges join two distinct vertices and there is at most one edge between any two vertices. In this paper, all graphs are simple unless stated. Furthermore, all graphs $G$ are finite, so $|V(G)| < \infty$. The graph with all possible edges on $n$ vertices is the \textit{complete graph} $K_n$. Graphs are defined up to isomorphism, or up to relabelling of the vertices.
Throughout this report, the set $\lbrace 1\ldots n \rbrace$ is notated as $[n]$. 
A graph \(G\) is \(k\)-colourable if there exists a function \(f: V(G) \rightarrow [k]\) such that if $f(v) = f(w)$, then $v$ and $w$ do not share an edge. The \textit{chromatic number} \(\chi(G)\) is the smallest \(k\) such that \(G\) is \(k\)-colourable. 

Let $G$ be a graph. A \textit{subgraph} $H$ in $G$ is a graph with vertex set $V(H) \subseteq V(G)$ and edge set $E(H)$ with the property that if $vw$ is an edge in $E(H)$, then $vw$ is an edge in $E(G)$.
Let $G$ be a graph and let $S$ be a non-empty subset of the vertex set of $G$. The \textit{induced subgraph} of $S$ in $G$ is the graph $G[S]$ with vertex set $S$ and edge set containing precisely all edges in $G$ incident to two vertices in $S$. Removing a set of vertices $S \subseteq V(G)$ from $G$ forms the induced subgraph $G - S := G[V(G) - S]$. 
$H$ is a \textit{spanning subgraph} of $G$ if $H$ is a subgraph of $G$ and $V(H) = V(G)$. 
The \textit{neighbourhood} of a set of vertices $A \subseteq V(G)$ are precisely all vertices that are adjacent to a vertex in $A$ and not in $A$ and is denoted as $N_G(A)$. A \textit{clique} is a subgraph isomorphic to a complete graph. 

\subsection{Planar graph bounds}
This subsection uses \cref{lem:planar_graphs_4_connected_cliqesums} and \cref{thm:clique_sum_pagenumber_bound} to find a book-embedding of 4-connected planar graphs. This proof is different from previous proofs as it does not require a triangulation of a planar graph. Because of this fact, this proof is used in future sections with respect to adding vortices on faces. 
Then use a theorem of Tutte to prove a fact for all $4$-connected planar graphs. 

\begin{theorem}[Tutte\cite{tutteTheoremPlanarGraphs1956}]\label{thm:4-connected_planar_ham_cycle}
	All 4-connected planar graphs are Hamiltonian.
\end{theorem}

As a corollary to \textcite{hickingbothamStackNumberCliqueSum2023}, the pagenumber of planar graphs are bounded.

\begin{corollary}\label{thm:Planar Graph Hickingbotham Bound}
	Let \(G\) be a 2-connected planar graph. Then $G$ can be embedded on $11$ pages, with book-embedding $(<, \rho)$. $<$ restricted to the outer cycle $C$ is $C$. Furthermore, for every face cycle $C$, $<_{V(C) - \{u, v, w\}} = C - \{u, v, w\}$ for some vertices $u$, $v$, $w$. 
\end{corollary}
\begin{proof}
	From \cref{thm:clique_sum_pagenumber_bound} with tree-decomposition from \cref{lem:planar_graphs_4_connected_cliqesums}, the pagenumber is at most \(2 \cdot 4 + 3 = 11\).

	Furthermore, from the construction given in \cref{lem:planar_graphs_4_connected_cliqesums}, every $4$-connected class are glued on faces. Therefore, every face only changes by $3$ vertices, from \cref{thm:clique_sum_pagenumber_bound}. Therefore removing $3$ vertices from every face preserves the cyclic ordering of every face.
\end{proof}

We will discuss the \(K_5\)-minor free case. If \(G\) is \(K_5\)-minor free, then we can use Wagner's theorem.
\begin{theorem}[Wagner's theorem\cite{wagnerUeberEigenschaftEbenen1937}]\label{thm:WagnersTheorem}
	Let \(G\) be a \(K_5\)-minor-free graph. Then \(G\) has a tree-decomposition of adhesion $\leq 3$ where every torso is either a planar graph or the Wagner graph \(V_8\).
\end{theorem}
A description of the Wagner graph is in \cref{fig:wagner}. The edges are coloured such that the internal edges are on different pages. The spine edges (the edges that are on the outerface) are the ones which can go on any page.
\begin{figure}[h!]
	\centering
	\begin{tikzpicture}[thick,scale=2, every node/.style={scale=2}]
		\tikz \graph [nodes = {draw, circle}, clockwise, empty nodes] {
	subgraph C_n [n=8, red];
	1 --[red] 5;
	2 --[blue] 6;
	3 --[green] 7;
	4 --[yellow] 8;
};

	\end{tikzpicture}
	\caption[Wagner graph]{The Wagner graph $V_8$. Notice how the clockwise circular ordering of the vertices of the Wagner graph needs 4 pages to embed the graph. }\label{fig:wagner}
\end{figure}

\begin{theorem}
	Let \(G\) be a \(K_5\)-minor free graph. Then \(G\) has pagenumber \(\leq 19\).
\end{theorem}

\begin{proof}
	Suppose \(G\) is \(K_5\)-minor free. Then by Wagner's theorem \cite{wagnerUeberEigenschaftEbenen1937}, \(G\) has a tree-decomposition of adhesion at most 3 where every torso is either a planar graph or the Wagner graph.
	Planar graphs are \(4\)-colourable and can be embedded on four pages. The Wagner graph is \(3\)-colourable and can be embedded on four pages. Therefore, if \(G\) is \(K_5\)-minor free, then \(G\) has pagenumber at most \(4 \cdot 4 + 3 = 19\) from \cref{thm:clique_sum_pagenumber_bound}.
\end{proof}

\subsection{Surfaces and graphs on surfaces}
Graphs on surfaces are a natural extension to graphs on planes. This section is an introduction to surfaces and graphs on surfaces. Readers are expected to be familiar with point-set topology. This section is based on \textcite{moharGraphsSurfaces2001}.

An \textit{$n$-manifold} $M$ is a second-countable Hausdorff space where every point in $M$ has an open neighbourhood homeomorphic to an open ball in $\mathbb{R}^n$.  
A \textit{surface} is a $2$-manifold. Surfaces are typically denoted as $\Sigma$. Examples of surfaces are the sphere $S^2$, the torus $T^2$, the real projective plane $\mathbb{R}P^2$, and the Klein bottle $K$. 

\textit{Handles} are added to a surface \(\Sigma\) by removing two disks in \(\Sigma\) and identifying the boundaries such that one goes clockwise and the other goes counter-clockwise. \textit{Crosscaps} are added to a surface $\Sigma$ by removing a disk in \(\Sigma\) and identifying opposite points on the boundary. Every surface is homeomorphic to a sphere with $m$ handles and $n$ crosscaps. The \textit{Euler genus} of a surface \(\Sigma\) with $m$ handles and $n$ crosscaps is $2m + n$. In fact, a sphere with a mix of crosscaps and handles is homeomorphic to a sphere with all crosscaps, as a sphere with a handle and crosscap is homeomorphic to three crosscaps.

An \textit{embedding} of $G$ on a surface $\Sigma$ is a drawing of $G$ on $\Sigma$ such that no two edges cross. 
A \textit{$2$-cell embedding} of a graph $G$ on a surface $\Sigma$ is an embedding of $G$ in $\Sigma$ such that $\Sigma - G$ is homeomorphic to a finite number of disks. The \textit{Euler Genus} of a \textit{graph} \(G\) is the smallest Euler genus \(g\) surface \(\Sigma\) such that \(G\) can be $2$-cell embedded on $\Sigma$.

An extension for Euler's formula is below. Suppose $G$ is $2$-cell embedded on a surface $\Sigma$ of genus $g$. Let \(|F(G)|\) be the number of faces in a graph \(G\). Then \(|V(G)| - |E(G)| + |F(G)| = 2 - g = \chi\). When $g = 0$, then $\Sigma$ is a $2$-sphere and this is the original Euler's formula. 
The value $\chi$ is known as the \textit{Euler characteristic} of a topological space, in this case a surface. The Euler characteristic is invariant under homeomorphism. Calculating the Euler characteristic of any space is done through \textit{homological algebra}, specifically by looking at the free rank of homology groups. 

Graphs that can be embedded on the plane are called \textit{planar} graphs. Graphs that can be 2-cell embedded on the torus are called \textit{toroidal} graphs, and graphs that can be 2-cell embedded on the projective plane are called \textit{projective-planar} graphs. Graphs that can be 2-cell embedded on a surface of genus $g$ are called \textit{genus $g$} graphs. Similarly to plane graphs, graph drawings on the torus are called torus graphs, and graphs drawings on the projective plane are called projective-plane graphs. 


Graphs on surfaces have been studied extensively. A famous conjecture involving graphs on surfaces is Heawood's conjecture, from \textcite{heawoodMapcolourTheorem1890}. The conjecture states that the minimum number of colours sufficient to colour all Euler genus $g$ graphs when $g \geq 0$ is
	\begin{equation*}
		\gamma(g) := \left\lfloor 
		\frac{7 + \sqrt{1 + 24g}}{2}
		\right\rfloor.
	\end{equation*}\todo{is this right?}
\textcite{ringelMapColorTheorem1974} showed that for almost every case, $\gamma(g)$ is also necessary. The case where this does not hold is the Klein bottle case. There exists a 6-colourable Klein bottle graph, but $\gamma(g) = 7$. 

\section{Book-Embeddings and Pagenumber}\label{sec:Book Embedding}
A \textit{book} with \(k\) \textit{pages} consists of \(k\) half-planes glued together on a common boundary. We refer to the boundary as the \textit{spine}, and the individual half-planes as \textit{pages}. In topology, these are referred to as \textit{fans} of half-planes.\ \textcite{persingerSubsetsNbooksE31966,atneosenOnedimensionalNleavedContinua1972} described fans in the 1960s.
A \textit{book-embedding} of a graph \(G\) is an embedding of \(G\) on a book. We place the vertices of \(G\) on the spine, and we place each edge on a single page such that no two edges cross.
The \textit{pagenumber} of a graph \(G\) is the minimum number of pages required to embed \(G\) on a book. This is also referred to as \textit{book-thickness}, or \textit{stack-number}. An embedding of $K_5$ in three pages is in \cref{fig:book-embedding}.
\begin{figure}[h!]\label{fig:book-embedding}
	\centering
	\includesvg[height = 0.5\textheight]{figures/3page_K5.svg}
	\caption{Book-embedding of $K_5$ on three pages. Image by \textcite{eppsteinBookEmbedding2014}}
\end{figure}
\par
There is an equivalent combinatorial definition. A \textit{book embedding} of a graph \(G\) is an arrangement of the vertices of \(G\) in a total ordering \(v_1 < v_2 < \cdots < v_n\). We then \textit{colour} the edges \(E(G)\) such that if there are vertices with ordering \(v_a < v_b < v_c < v_d\) and edges \(v_a v_c\) and \(v_b v_d\) in $E(G)$, then $v_a v_c$ and $v_b v_d$ are assigned different colours.
We refer to the total ordering of \(V(G)\) in the book embedding as \((<)\) or as \((\leq)\). For a book-embedding \((v_1, v_2, \ldots, v_{|G|})\), we refer to the edges \( \left\{ v_1 v_2, v_2 v_3, \ldots, v_{|G| - 1}v_{|G|}, v_{|G|}v_{1} \right\} \) as \textit{spine edges}.
We may use a \textit{circular ordering} of the vertices rather than a linear ordering. This means that we order the vertices in a circle rather than on a straight line. The book-embedding of a circular ordering is exactly the same as for a linear ordering, and we can convert between a circular and linear ordering by choosing a vertex to be at the start of the sequence.
Book-embeddings were introduced by \textcite{kainenRecentResultsTopological1974, ollmannBookThicknessVarious1973} in the 1970s. A paper by \textcite{bernhartBookThicknessGraph1979} calculated the book-thickness of complete and bipartite graphs.

An \textit{expander graph} is a sparse, highly connected graph. Expander graphs share many properties with random graphs, but are constructed explicitly. One type of expander graph is a \textit{bipartite \varepsilon-expander}, where $\varepsilon \in (0, 1]$. We say a graph $G$ is a bipartite \varepsilon-expander if there exists a bipartition $ \{A, B\}$ of $V(G)$ such that $|A| = |B|$ and for all subsets $S \subset A$ where $|S| \leq \frac{|A|}{2}$, $|N(S)| \geq (1 + \varepsilon) |S|$. 
\textcite{dujmovicLayoutsExpanderGraphs2016} showed that all bipartite \varepsilon-expander graphs can be embedded in 3 pages. 


Book-embeddings of graphs were has applications in VLSI and processor designs, bioinformatics by \textcite{haslingerRNAStructuresPseudoknots1999}, and in graph drawings by \textcite{woodBoundedDegreeBook2002}. 
The project of finding upper and lower bounds of the pagenumber of planar graphs was started by \textcite{bernhartBookThicknessGraph1979} when they conjectured that planar graphs had unbounded pagenumber. However, \textcite{bussPagenumberPlanarGraphs1984} showed that all graphs could be embedded in nine pages, and \textcite{heathEmbeddingPlanarGraphs1984} brought down the number of needed pages to seven.\ \textcite{yannakakisEmbeddingPlanarGraphs1989} devised an algorithm to embed a graph in four pages. Yannakakis, in this paper, claimed that there exists planar graphs which cannot be embedded in three pages. However, his proof was incomplete and the full proof was left unpublished. In 2020, Yannanakis published his full proof \cite{yannakakisPlanarGraphsThat2020}. At around the same time, \textcite{kaufmannFourPagesAre2020} published the same lower bound.

\textcite{malitzGraphsEdgesHave1994} proved that any graph with $e$ edges has pagenumber $O(\sqrt{e})$. Additionally, he proved that random $d$-regular graphs $G$ with $n$ vertices have the property that $\pn(G) \in \Omega(\sqrt{d} n^{1/2 - 1/d})$. For random 3-regular graphs $G$ with $n$ vertices, $\pn(G) \in \Omega(n^{1/6})$. These constructions of $\Omega(n^d)$ pagenumber graphs are not explicit.\ \textcite{eppsteinThreeDimensionalGraphProducts2024} showed that $\pn(P_n \boxtimes P_n \boxtimes P_n) \in \Theta(n^{1/3})$. This is an explicit construction of a graph which has pagenumber in $\Theta(n^{d})$. 


\chapter{Definitions}\label{chap:Definitions}
We lay out this chapter to build towards important notions in structural graph theory.
\begin{itemize}
	\item \cref{sec: Basic definitions} defines the basics of graph theory and introduces useful notation.
	\item \cref{sec:Planar graphs} discusses important information involving planar graphs.
	\item \cref{sec:Graph Minors} discusses graph minors.
	\item \cref{sec:Book Embedding} defines what a book-embedding is and its importance.
	\item \cref{sec:treewidth} defines the \textit{treewidth}.
	\item \cref{sec:Pathwidth} defines a related notion, the \textit{pathwidth}.
\end{itemize}
These notions will be useful in discussing the Graph Minor Structure Theorem and our conjecture.
\section{Basic definitions}\label{sec: Basic definitions}
For a graph \(G\), we define the \textit{vertex} and \textit{edge} sets of \(G\) to be \(V(G)\) and \(E(G)\) respectively.
For a subset of vertices \(A \subseteq V(G)\), we denote the \textit{induced subgraph} on \(G\) with vertex set \(A\) as \(G[A]\).

\begin{itemize}
	\item A \textit{path} in a graph \(G\) is a sequence of edges \(e_1, e_2, \ldots, e_{\ell- 1}\) which join a sequence of vertices \(v_1, v_2, \ldots, v_{\ell}\) such that \(e_i = v_i v_{i + 1}\), and all the vertices are distinct.
	\item A \textit{simple cycle} \(C\) in a graph \(G\) is a sequence of edges \(e_1, e_2, \ldots, e_{\ell}\) which join a sequence of distinct vertices \(v_1, v_2, \ldots, v_{\ell}\) such that \(e_i = v_i v_{i + 1}\) for \(1 \leq i \leq \ell - 1\) and \(e_\ell = v_\ell v_1\).
	\item A \textit{Hamiltonian cycle} in a graph \(G\) is a simple cycle \(C\) such that all vertices in \(G\) appear in \(C\).
\end{itemize}

We say that a connected graph \(G\) is \textit{\(k\)-connected} if \(G\) has more than \(k\) vertices and removing up to \(k-1\) vertices from \(G\) maintains the connected property. For the case that \(k = 2\), the graph is \textit{biconnected}. Note that all graphs \(G\) with a Hamiltonian cycle is biconnected, due to Menger's theorem.\cite{mengerZurAllgemeinenKurventheorie1927}

We say a graph \(G\) is \(k\)-colourable if there exists a function \(f: V(G) \rightarrow [k]\) such that for all \(vw \in E(G)\), \(f(v)\) and \(f(w)\) are assigned different colours. The \textit{chromatic number} \(\chi(G)\) is the smallest \(k\) such that \(G\) is \(k\)-colourable.

Throughout this report, we use as shorthand for the set \( [n] = \lbrace 1\ldots n \rbrace \).

\subsection{Menger's theorem}
Menger's theorem \cite{mengerZurAllgemeinenKurventheorie1927} is an important theorem which we use throughout the report, so it is worth dedicating some time discussing its significance.

Let \(G\) be a graph and \(A, B \subseteq V(G)\). An \(AB\)-path is a path in \(G\) which starts in \(A\) and ends in \(B\) with no internal vertices in \(A\) or \(B\). An \(AB\)-connector is a set of interally-disjoint \(AB\)-paths. An \(AB\)-separator is a set \(S\) such that \(G - S\) contains no \(AB\) path. Then:

\begin{theorem}[Menger's theorem]\label{thm:Menger}
	The size of the smallest \(AB\) separator of \(G\) is equal to the largest \(AB\)-connector.
\end{theorem}

Now take vertices \(x, y\). Now let \(A = N_G(x) \cup \{x\} \) and \(B = N_G(y) \cup \{y\} \). Then the size of the smallest \(AB\) separator implies that:
\begin{theorem}[Menger's theorem, vertex-connectivity version]\label{thm:Menger_Vertex}
	\(G\) is \(k\)-connected iff every pair of vertices has at least \(k\) internally disjoint paths between.
\end{theorem}

\section{Planar graphs}\label{sec:Planar graphs}
We say a graph \(G\) is \textit{planar} if \(G\) can be drawn on the Euclidean plane \(\Sigma \) such that no two edges cross. Say \(G\) is drawn on a plane. If \(G\) is embedded on \(\Sigma \), then \(\Sigma \) is divided into regions where no edges cross. These regions are called \textit{faces}. The \textit{outerface} is the face on the outside of the graph. We say that a set of vertices \textit{lie} on a face if they are on the boundary of the face. We also say that the vertices and edges \textit{bound} a face. \(G\) is \textit{outerplanar} if \(G\) is planar and all vertices in \(G\) lie on the outerface.
Let \(F(G)\) be the set of faces of \(G\) embedded on \(\Sigma\). Then we have that:
\begin{theorem}[Euler's formula]\label{thm:Euler_planar}
	\begin{equation}
		|V(G)| - |E(G)| + |F(G)| = 2
	\end{equation}
\end{theorem}

We can use this result to bound the number of edges in an outerplanar graph.
\begin{theorem}\label{thm:outerplanar_bound}
	If \(G\) is outerplanar with \(n\) vertices and \(m\) edges, then \(m \leq 2n - 3\).
\end{theorem}

\begin{proof}[Proof of theorem]
	Suppose \(G\) is \textit{maximal outerplanar}, meaning adding any edge will break the outerplanar property. Let the \textit{internal faces} be all the faces which are not the outerface, and let there be \(f\) internal faces. Then the outerface has \(n\) edges on the boundary, but each internal face will have exactly \(3\) edges on the boundary. However, each edge is touching two distinct faces. Therefore we have that
	\begin{equation}
		3 f - 3 + n = 2m
	\end{equation}
	Combining with \cref{thm:Euler_planar} given by
	\begin{equation}
		n - m + f = 2
	\end{equation}

	we have, after some rearrangment,
	\begin{equation}
		2n = 3 + m
	\end{equation}
	Therefore, \(m = 2n - 3\). As every outerplanar graph is a subgraph of a maximal planar graph, then we have that \(m \leq 2n - 3\).
\end{proof}
\section{Graph minors}\label{sec:Graph Minors}
We say that a graph \(H\) is a \textit{minor} of another graph \(G\) if a graph isomorphic to \(H\) can be obtained from \(G\) by deleting vertices, edges, and \textit{contracting} edges. To \textit{contract} an edge \(uv\), we delete both \(u\) and \(v\) and create a new vertex \(w\) such that \(N_G(w) = N_G(u) \cup N_G(v)\), where \(N_G(v)\) is the neighbourhood of \(v\). We notate an edge contraction on \(uv\) as \(G\setminus uv\).
We notate the statement ``\(H\) is a minor of \(G\)'' as \(H \leq G\).
We define minors up to isomorphism, and we typically omit the isomorphism relation in our definitions. We say that a graph \(G\) is \textit{\(H\)-minor-free} if there is no sequence of deletions and contractions of \(G\) that yield a graph isomorphic to \(H\). We say that a family of graphs \(\mathcal{F}\) is minor-closed if for all \(G \in \mathcal{F}\), all minors of \(G\) is also in \(\mathcal{F}\).
An example of a minor-closed class is the class of planar graphs.
An important class of graph families are the \(K_t\)-minor free graphs. For a graph \(G\), we define \textit{Hadwiger's number} \(\had(G)\) to be the largest \(t\) such that \(K_t\) is a minor of \(G\). This is named after Hugo Hadwiger and his most famous conjecture.

\begin{conjecture}[Hadwiger's conjecture]\cite{hadwigerUeberKlassifikationStreckenkomplexe1943}
	For all graphs \(G\), \(\chi(G) \leq \had(G)\), where \(\chi(G)\) is the \textit{chromatic number} of \(G\).
\end{conjecture}
Much work has been done on solving Hadwiger's conjecture but as of writing this report it has not been solved.
\subsection{Minors and models}
A \textit{model} of a graph \(H\) to a graph \(G\) is a function \(\rho\) which assigns to \(H\) vertex disjoint connected subgraphs of \(G\), such that if \(uv \in E(H)\), then some edge in \(G\) connects a vertex in \(\rho(u)\) to \(\rho(v)\). This definition comes from Sergey Norin in his notes on graph minors \cite{norinMath599GraphMinors2017}.

\begin{theorem}
	\(H\) is a model of \(G\) iff \( H \leq G\).
\end{theorem}

\begin{proof}[From Norin \cite{norinMath599GraphMinors2017}]
	Suppose \(H\) is a model of \(G\). Then for all \(x\) in \(V(H)\), we contract \(\rho^{-1}(x)\) in \(G\) to a single vertex. We can do this as each preimages are disjoint and connected. Then we delete edges to form \(H\).

	Suppose \(H \leq G\). We will use induction to show that \(H\) has a model in \(G\). Suppose \(H\) is obtained from \(G\) by contraction operations only. We can assume this by taking a subgraph of \(G\) if necessary. Let \(uv\) be the edge that is contracted first and let \(G' = G \setminus uv\). Let \(w\) be the vertex obtained after contracting \(uv\). Then by induction, there is a model \(\mu'\) of \(H\) in \(G'\). Then we look at the block \(B\) where \(H\) lives in, and we delete \(w\) from \(H\) and add back in \(u\), \(v\), and \(uv\). Then this is a model of \(H\) in \(G\).
\end{proof}
\todo{Add pictures}
\todo{Subdivisions of edges}
\todo{Layouts of graph subdivisions}
\todo{Layouts of expander graphs}
\todo{section on Kt minor free graphs and some properties of graphs}
\todo{linklessly embedded graphs}
\todo{fundamentally knotted graphs}
\section{Book embedding}\label{sec:Book Embedding}
A \textit{book} of \textit{thickness} \(k\) are \(k\) half-planes glued together on a common boundary. We refer to the boundary as the \textit{spine} and we refer to the individual half-planes as \textit{pages}. In topology, these are referred to as \textit{fans} of half-planes. Books were described by Persinger and Atnosen in the 1960s \cite{persingerSubsetsNbooksE31966, atneosenOnedimensionalNleavedContinua1972}.
A \textit{book-embedding} of a graph \(G\) is an embedding of \(G\) on a book. We place the vertices of \(G\) on the mutual boundary of all half-planes, and we place the edges on each half-plane such that no two edges cross.

The \textit{pagenumber} of a graph \(G\) is the minimum number of pages required to embed \(G\) on a book. This is also referred to as \textit{book-thickness}, or \textit{stack-number}.

We have another definition which abstracts the underlying topology and focuses on the graph. This definition is more combinatorial in nature.
A \textit{book embedding} of a graph \(G\) is an arrangement of the vertices of \(G\) in a total ordering \(v_1 < v_2 < \cdots < v_n\). We then colour the edges \(E(G)\) such that if we have \(v_a < v_b < v_c < v_d\) and edges \(v_a v_c\) and \(v_b v_d\), then the edges are each assigned different colours.
We refer to the total ordering of \(V(G)\) in the book embedding as \((<)\) or as \((\leq)\). For a book-embedding \((v_1, v_2, \ldots, v_{|G|})\), we refer to the edges \( \left\{ v_1 v_2, v_2 v_3, \ldots, v_{|G| - 1}, v_{|G|}v_{1} \right\} \) as \textit{spine edges}.
We may use a \textit{circular ordering} of the vertices rather than a linear ordering. This means that we order the vertices in a circle rather than on a straight line. The book-embedding of a circular ordering is exactly the same as for a linear ordering, and we can convert between a circular and linear ordering by choosing a vertex to be at the start of the sequence.

Book-embeddings were introduced by Kainen and Ollmann in the 1970s.\cite{kainenRecentResultsTopological1974, ollmannBookThicknessVarious1973} It was developed further in a paper by Bernhart and Kainen \cite{bernhartBookThicknessGraph1979}.
\begin{lemma}\label{lem:Pagenumber_1}
	A graph \(G\) has pagenumber at most 1 iff \(G\) is outerplanar.
\end{lemma}
\begin{proof}
	We can choose an ordering of the vertices \(V(G)\) to go anticlockwise around the outer face. Suppose we have two edges \(uv\), \(xy\) that cross, so that \(u < x < v < y\). Then in the original graph embedding, we will have that \(uv\) and \(xy\) will cross inside the circle, thus \(G\) is not outerplanar.
\end{proof}
\begin{lemma}\label{lem:Pagenumber_2}
	A graph \(G\) has pagenumber at most 2 iff \(G\) is a subgraph of a planar graph with a Hamiltonian cycle.
\end{lemma}
This is because we can embed the graph on a sphere with the vertices and Hamiltonian cycle on the equator, and edges on each hemisphere forms the interior and exterior edges of the cycle respectively.
\subsection{Properties of pagenumber}\label{ssec:Related_Properties}
\begin{lemma}[Bound on number of edges, from \textcite{bernhartBookThicknessGraph1979}]\label{lem:Edge_Bound}
	If an \(n\)-vertex graph \(G\) has \(\pn(G) = k\), then \(G\) has at most \(n + k(n-3)\) edges.
\end{lemma}
\begin{proof}
	Given a vertex ordering \(v_1 \leq v_2 \leq \cdots \leq v_n\), we have that the spine edges \(v_i v_{i + 1}\), \(v_1 v_n\)  can appear on any page. Furthermore we have there are at most \(n-3\) non outer-cycle edges on each page as the maximum number of edges in an outerplanar graph is \(2n - 3\) from \cref{thm:outerplanar_bound}, but we remove the outer cycle (with \(n\) edges on the cycle) to have at most \(n-3\) edges on each page. Therefore, \(m\), the number of edges, satisfies \(m \leq n + k (n - 3)\).
\end{proof}
\begin{theorem}[\cite{bernhartBookThicknessGraph1979}]\label{thm:Pagenumber_Complete_Graph}
	The complete graph \(K_n\) on \(n\) vertices where \(n \geq 4\), has pagenumber \( \leq \lceil n/2 \rceil \).
\end{theorem}
\begin{proof}
	Arrange the vertices in any circular ordering, and colour the edges in a zig-zag spanning path. As an example, we refer to \cref{fig:k8 coloured with colours} to show what zig-zagging pattern we are referring to. We rotate this pattern $\lceil n/2 \rceil$ times. To show equality, we use \cref{lem:Edge_Bound}.
	\begin{figure}[ht]
		\caption{Circular embedding of \(K_8\) with 4 colours, the minimum possible.}
		\centering
		\usetikzlibrary{graphs,graphs.standard}

\tikz
	\graph[nodes={circle, draw}] { 
		subgraph K_n [n=8,clockwise,radius=2cm];
		
		{[induced path, edges= red] 1,2,8,3,7,4,6,5},
 };
		\label{fig:k8 coloured with colours}
	\end{figure}

	We have shown one direction of the inequality, and we use \cref{lem:Edge_Bound} to prove the other direction. We have that \(K_n\) has \(n\) vertices and \(\binom{n}{2}\) edges. Then \(\pn(K_n) \geq \frac{\binom{n}{2} - n}{n - 3} = \frac{n}{2}\) when \(n \geq 4\), from the lemma. As \(\pn(K_n)\) is an integer, we take the ceiling of \(\frac{n}{2}\). This concludes the equality.
\end{proof}
This is an upper bound of any graph \(G\) with \(n\) vertices.
Therefore for any graph \(G\) on \(n\) vertices, \(n \geq 4\), \(\pn(G) \leq \lceil n/2 \rceil\).
\begin{theorem}[Chromatic number bound\cite{bernhartBookThicknessGraph1979}]\label{thm:Colouring_Bound}
	For all graphs \(G\), \(\chi(G) \leq 2 \pn(G) + 2\).
\end{theorem}
\begin{proof}
	Let \(\pn(G) = k\) and suppose \(G\) has \(n\) vertices and \(m\) edges. Then we have that the average degree of \(G\), \(d(G) = 2m/n\) by the handshaking lemma. But \(2m/n \leq 2 \frac{n + k(n-3)}{n} < 2k + 2\). But this implies that \(G\) has a vertex of degree \(\leq 2k + 1\), and as if \(G'\) is a subgraph of \(G\), then \(G'\) also has \(\pn(G') \leq k\), thus \(G'\) has a vertex of degree at most \(2k + 1\). However, this implies \(G\) is \(2k + 1\)-degenerate, thus \(\chi(G) \leq 2k + 2\).
\end{proof}
We may note that this bound is not tight when \(\pn(G) = 1\) or \(2\). When \(\pn(G) = 1\), then \(G\) is outerplanar, but can be coloured with 3 colours. When \(\pn(G) = 2\), then \(G\) is Hamiltonian and planar, but is 4-colourable.

\subsection{Historical interest}\label{ssec:Pagenumber_History}
Pagenumbers were developed for VLSI and processor designs, but more recently have been used in bioinformatics~\cite{haslingerRNAStructuresPseudoknots1999}.
The project of finding upper and lower bounds of the pagenumber of planar graphs was started by Bernhart and Kainen when they conjectured that planar graphs had unbounded pagenumber. However, Buss and Shor\cite{bussPagenumberPlanarGraphs1984} found that the pagenumber of planar graphs was at most 9, and Heath \cite{heathEmbeddingPlanarGraphs1984} found that the pagenumber of planar graphs is at most 7. Yannakakis' devised an algorithm to bound the pagenumber to at most 4 \cite{yannakakisEmbeddingPlanarGraphs1989}. Yannakakis, in this paper, claimed that there exist planar graphs with pagenumber 4. However, his proof was incomplete and the full proof was left unpublished. In 2020, Yannanakis published his full proof.\cite{yannakakisPlanarGraphsThat2020} At around the same time, Kaufmann, Bekos, Klute, Pupyrev, Raftopoulu and Ueckerdt published a similar result\cite{kaufmannFourPagesAre2020}.

\todo{Add Malitz square root proof}
\todo{Add random 3-regular graph upper bound Malitz}
\todo{Add paper: Three-dimensional graph products with unbounded stack-number }
\section{Treewidth}\label{sec:treewidth}

The \textit{treewidth} of a graph \(G\) measures how far \(G\) is from being a forest \cite{diestelGraphMinors2017}.

\begin{definition}[Tree-decomposition]\label{def:tree-decomposition}
	The tree-decomposition \(\tree\) of a graph \(G\) is defined as a tree \(T\) with associated \textit{bags} \(\lbrace B_x : x \in V(T) \rbrace\) such that:
	\begin{itemize}
		\item Every vertex in \(G\) is in at least one bag \(B_x\).
		\item for all \(v \in V(G)\), the subset of vertices \(\left\lbrace x \in V(T): v \in B_x \right\rbrace\) in \(V(T)\) induces a connected subtree in \(V(T)\).
		\item For all edges \(vw \in E(G)\), there exists a bag \(B_x\) such that both \(v\) and \(w\) are in the bag \(B_x\).
	\end{itemize}
\end{definition}
We refer to the vertices of the tree \(T\) as \textit{nodes}.
The \textit{width} of the tree decomposition \(\tree\) is defined as \(\max \lbrace |B_x| - 1 : x \in V(T) \rbrace\). An important tree-decomposition

\begin{definition}\label{def:treewidth}
	The treewidth of a graph \(G\), denoted as \(\tw(G)\), is defined to be the smallest width for all tree decompositions of the graph \(G\).
\end{definition}


\begin{lemma}\label{lem:treewidth_forest}
	\(\tw(G) = 1\) iff \(G\) is a forest.
\end{lemma}
\begin{proof}
	Suppose \(G\) is a tree. Root the graph \(G\) at the vertex \(r\). Then let \(T = G\) and \(B_x:= \lbrace x, p \rbrace\) where \(p\) is the parent of \(x\). The bag \(B_r\) will just contain \(r\). Then all edges \(vw\) will be between parent \(v\) and child \(w\), so it will be in bag \(B_w\). Finally, the subgraph induced by vertex \(x\) in \(T\) will be \(x\) and the children of \(x\), which is a connected subtree.

	If \(G\) is a forest, then we perform this operation on every connected component of \(G\) and connect the roots to form a new tree. Then this tree is a tree-decomposition. This forms a tree-decomposition of width at most 1. An example is in \cref{fig:tree-treedecomp}.
	If \(G\) has a cycle \(C\), then the treewidth cannot be 1. This is because if there is a tree decomposition \(\tree\) where the size of each bag is at most 2, then as the graph must have every edge, then every edge in \(C\) is in separate bags. However, we have that for any vertex \(v\) in \(C\) to have an induced connected subgraph in \(T\), then it follows that the cycle \(C\) is also in \(T\). Thus \(T\) is not a tree, and this is not a valid tree-decomposition.
	\begin{figure}[ht]
		\centering
		\usegdlibrary {circular,trees}
\tikz \graph [simple necklace layout] {
	tree 1[draw, circle] // [tree layout] { a -> {1, 2}; }
	-> b
	-> c
	-> tree 2[draw] // [tree layout] { d -> {3, 4 -> {5, 6} } }
	-> e
	-> f
	-> tree 1;
};
		\tikz 
\graph [tree layout, nodes={draw,circle}] {
	1 -- {1 2 -- {2 3 , 2 4},1 5 -- {5 6, 5 7 -- 5 8}};
};
		\caption{A tree and its tree-decomposition. We root at the vertex \(1\) and every bag consists of a vertex and its parent.}
		\label{fig:tree-treedecomp}
	\end{figure}
\end{proof}

\begin{lemma}[Helly Property]\label{lem:Helly}
	Let \(T_1, \ldots, T_k\) be subtrees of a tree \(T\) such that for every pair of trees, there is a vertex in common. Then there exists a vertex which is common to all trees.
\end{lemma}
\begin{proof}[Helly property]
	If \(T_1\), \(T_2\) and \(T_3\) are subtrees of \(T\) such that the vertex sets are pairwise nonempty, then there is a common vertex in all three subtrees. If this is not the case, denote \(v_1\) as a vertex in the intersection of \(T_1\) and \(T_2\), \(v_2\) as the vertex in \(T_1 \cap T_3\), and \(v_3\) as the vertex in \(T_2\) and \(T_3\). Then there exists a unique path \(P\) in \(T_1\) from \(v_1\) to \(v_2\).
\end{proof}

\begin{theorem}[Clique theorem]\label{thm:clique}
	In any tree-decomposition of \(G\), for every clique \(C\) in \(G\), there exists a node \(x \in V(T)\) such that \(C \subseteq B_x\).
\end{theorem}

\begin{proof}
	Let \(\tree\) be a tree-decomposition. Every vertex \(v\) induces a connected subtree in \(T\), call it \(T_v\). Then for any two vertices \(x, y\) in \(C\), we have that \(T_x\) and \(T_y\) must intersect as the edge \(xy\) is inside a bag \(B_z\) corresponding to a node \(z\). Then by the Helly property, there exists a node \(v\) such that \(C \subseteq B_v\).
\end{proof}

\begin{corollary}\label{cor:complete_tw}
	\(\tw(K_n)\) is \(n-1\).
\end{corollary}

\begin{theorem}\label{thm:tw_minor_closure}
	If \(H\) is a minor of \(G\), then \(\tw(H) \leq \tw(G)\).
\end{theorem}
\begin{proof}[Proof of theorem]
	Suppose we have a tree-decomposition \(\tree\) of \(G\). If we delete an edge in \(G\), then \(\tree\) remains a valid tree-decomposition. If we delete a vertex \(v\), then \(\tree\) where we remove \(v\) from every bag in \(\tree\) is also a valid tree-decomposition. If we contract an edge \(vw\), creating a new vertex \(u\), then relabeling \(v\) and \(w\) in all bags to \(u\) is a valid tree-decomposition as the induced subtree of \(u\) is the union of the induced subtrees of \(v\) and \(w\), and every neighbor of \(v\) or \(w\) is a neighbor of \(u\). But the edges in the neighborhood do not change. Thus this is a valid tree-decomposition of \(H\), with width at most the width of \(\tree\).
\end{proof}

Recall that an outerplanar graph is a planar graph where there exists a face such that all vertices lie on the boundary of that face.
\begin{lemma}\label{ex:tw_outerplanar}
	The treewidth of an outerplanar graph is at most 2.
\end{lemma}
\begin{proof}
	Let \(G\) be an outerplanar graph, and let \(G'\) be the triangulation of \(G\). As \(G\) is a minor of \(G'\), \(\tw(G) \leq \tw(G')\). We look at the \textit{weak dual} of \(G'\). This is a tree \(T\), where every node \(v_f\) in \(T\) corresponds to a face \(f\) in \(G'\). Then let \(B_{v_f}\) be the bag of the tree-decomposition, where \(B_{v_f}\) is the set of vertices on the boundary of the face \(f\). Then the tree \(T\) with bags \(B_{v_f}\) is a valid tree-decomposition of \(G'\), where every bag has at most 3 vertices. Thus, \(\tw(G) \leq 2\).
\end{proof}

\subsection{Characteristics of treewidth}\label{ssec:characterising_Treewidth}
\subsubsection{\(k\)-trees}\label{sssec:k-trees}
We define a \(k\)-tree inductively. We have that the complete graph \(K_k\) is a \(k\)-tree, and if \(G\) is a \(k\)-tree, then we add a new vertex to \(G\) that is adjacent to \(k\) vertices that form a clique of size \(k\) in \(G\) results in a \(k\)-tree.
A \(k\)-tree is a maximal graph with treewidth \(k\).
\begin{theorem}[K-tree theorem]
	\(\tw(G) \leq k\) iff \(G\) is a subgraph of a \(k\)-tree.
\end{theorem}


\subsubsection{Bounded treewidth graphs}\label{sssec:Graph_treewidth_Bounded}
\begin{theorem}\label{thm:treewidth_clique-minor-free}
	If \(\tw(G) \leq k\), then \(G\) is \(K_{k+2}\)-minor-free.
\end{theorem}
\begin{proof}
	We shall prove the contrapositive: If \(K_t\) is a minor of \(G\), then \(\tw(G) \geq t-1\).
	If \(K_t\) is a minor of \(G\), then we have that from \cref{thm:tw_minor_closure} that \(\tw(K_t) \leq \tw(G)\). As \(\tw(K_t) = t-1\), then we have that \(\tw(G) \geq t - 1\).
\end{proof}

\subsection{Historical discussion}\label{ssec:tw_historical}
Treewidth was introduced in \cite{berteleChapterEliminationVariables1972} with applications to dynamic programming under the name ``dimension''. It was then rediscovered by Halin \cite{halinSfunctionsGraphs1976} before being used by Robertson and Seymour in \cite{robertsonGraphMinorsIII1984}, which was introduced to prove the Graph Minor Theorem\cite{robertsonGraphMinorsXX2004}.


\section{Path-width}\label{sec:Pathwidth}
Similarly to treewidth, the pathwidth of a graph \(G\) is how far a graph is from being a path.

We define the path-decomposition of a graph \(G\) to be a sequence of bags \(B_i\) such that the subsequence of bags containing a vertex \(v\) induces a nontrivial subpath and each edge \(vw\) is in a bag \(B_i\). Then we define the width of a path-decomposition as \(\max_i \lbrace |B_i| \rbrace -1\), same as treewidth.

If a graph has a path-decomposition \((B_i)_i\), then it has a tree-decomposition \(\left((B_i)_i, P\right)\). Therefore,
\begin{equation}
	\pw(G) \geq \tw(G).
\end{equation}

Similarly to treewidth, we have the following observation.
\begin{lemma}
	The pathwidth of \(G\) is the largest pathwidth over all connected components.
\end{lemma}
We say a graph \(G\) is a \textit{caterpillar} if \(G\) has a path \(P\) and every vertex is adjacent to a vertex on the path \(P\). Alternatively, \(G\) is a caterpillar if removing every leaf yields a path. We refer to the path where every leaf is connected to as the \textit{central path}.
\begin{theorem}[Caterpillars]
	Graphs have pathwidth at most 1 iff every connected component is a caterpillar.
\end{theorem}
\begin{proof}
	Suppose \(G\) is a caterpillar.
	Denote \(P =\left( p_1, p_2, \dots, p_n\right)\) as the central path. The leaves of vertex \(p_i\) are denoted as \(v_{i, 1}, v_{i, 2} \dots, v_{i, k}\). Define the bags as \((v_{1, 1}, v_1)\), \((v_{1, 2}, v_1)\) \dots \((v_{1, j}, v_1)\),  \((v_1, v_2)\), \((v_{2, 1}, v_2)\), \((v_{2,2}, v_2,)\) \dots. We can see that each leaf appears once and each vertex on the central path is on a subpath of the path. Therefore, the pathwidth of \(G\) is 1. We can repeat this for every connected component.
	Suppose \(G\) has pathwidth 1. Then for each connected component of \(G\), we choose a vertex \(v\) in \(B_1\) and a vertex \(w\) in \(B_n\), the final bag, and look at a path from \(v\) to \(w\). This path must go through every bag, thus the non-path vertices must have neighbour only of the other one in the bag and thus the graph is a caterpillar. An example of this is in \cref{fig:caterpillar}.
\end{proof}
\begin{figure}[ht]
	\centering
	\includesvg{figures/caterpillar}
	\caption{A caterpillar graph. Notice that the path \(v_1, v_2, v_3, v_4, v_5, v_6\) is the central path.}
	\label{fig:caterpillar}
\end{figure}

\begin{example}[Complete graphs]
	\(\pw(K_n) = \tw(K_n) = n - 1\).
\end{example}
\begin{proof}
	We have that the pathwidth of \(K_n\) is at least the treewidth of \(K_n\). But we have that the pathwidth is at most \(n- 1\) (where all the vertices are in the same bag), but the treewidth of \(K_n\) is \(n - 1\). Therefore, \(\pw(K_n) = n - 1\).
\end{proof}

\begin{theorem}
	The pathwidth of a tree \(T\) is \(\min_{P \subset T} \left\lbrace 1 + \pw(T - V(P))\right\rbrace \) where \(P\) is a path.
\end{theorem}

\begin{proof}[Proof of pathwidth of tree]
	To show \(\pw(T) \leq 1 + \pw(T - V(P))\), we have that if \(P\) is a path in \(T\) with vertices \(v_1, v_2, \ldots v_i\), then consider the subtrees hanging off \(v_i\) for all \(i\). \(T - V(P)\) will have a path-width and we can order each connected component such that they appear in the order of the trees. Then we have that adding \(v_i\) to the bags of subtrees connected to \(v_i\), and the bag \((v_i, v_{i+1})\) between the subtrees \(v_i\) and \(v_{i + 1}\) will yield a path-decomposition of width \(1 + \pw(T - V(P))\).
	To show there exists a path \(P\) such that \(\pw(T) \geq 1 + \pw(T - V(P))\), we proceed by induction. Let \(B_1, \ldots B_n\) be a path-decomposition of \(T\). Let \(x\) live in bag \(B_1\) and \(y\) live in bag \(B_n\), the final bag. Then let \(P\) be the unique path from \(x\) to \(y\). Then \(P\) traverses through every bag in the path-decomposition. Then \(\tw(T) \geq 1 + \tw(T - P)\) by induction.
\end{proof}

% !TEX root = ./thesis.tex

\chapter{Known results from structural graph theory}\label{chap:Known results}
In this chapter, we outline important known results that will help us solve \cref{conj:bded_had_pn}.

\begin{itemize}
	\item \cref{sec:Kt_Minor_Free} is an explanation of the Graph Minor Structure Theorem.
	\item \cref{sec:Graph Minor Theorem} discusses the Graph Minor Theorem. 
	\item \cref{sec:BoundedPagenumber} are a series of proofs that can be used with the Graph Minor Structure Theorem to prove that each individual component of the structure theorem has bounded pagenumber.
	\begin{itemize}
		\item \cref{ssec:Clique_sum_Pagenumber_bound} proves that tree-decompositions of bounded adhesion where each torso has bounded pagenumber also has bounded pagenumber.
		\item \cref{ssec:Bounded_Treewidth} proves that graphs with bounded treewidth also have bounded pagenumber.
		\item \cref{sec:pagenumber_bounded_genus} proves that all graphs with bounded genus have bounded pagenumber.
	\end{itemize}
\end{itemize}
\subsection{Graph Minor Structure Theorem}
\textcite{robertsonGraphMinorsXVII1999} provides a rough characterisation of all \(K_t\)-minor-free graphs. 

Every graph that is $K_t$-minor-free can be constructed from the following ingredients. This is a coarse characterisation of $K_t$-minor-free graphs, meaning that a subset, or a single one of these ingredients constitutes a $K_t$-minor-free graph. 
\begin{itemize}
	\item Graphs of bounded Euler genus.
	\item Sets of apex vertices.
	\item Graphs of bounded treewidth.
	\item Sets of vortices on graphs.
\end{itemize}
\textcite{robertsonGraphMinorsXVII1999} showed that every \(K_t\)-minor-free graph can be built up from smaller graphs with the above ingredients.
\section{Bounds of pagenumbers of graphs}\label{sec:BoundedPagenumber}
\subsection{Tree-decomposition into bounded page number torsos}\label{ssec:Clique_sum_Pagenumber_bound}

This proof has been adapted into the language of tree-decompositions.
\begin{theorem}[Hickingbotham and Wood \cite{hickingbothamStackNumberCliqueSum2023}]\label{thm:clique_sum_pagenumber_bound}
	Let \(G\) be a graph with a tree-decomposition \((B_x: x \in V(T))\) where each torso \(G \langle B_x \rangle\) has pagenumber \(\leq s\) and every torso \(G \langle B_x \rangle\) is \(c\)-colourable. Additionally, we have that the adhesion of this tree is at most \(\ell\).
	Then \(\pn(G) \leq cs + \ell \).
\end{theorem}

\subsubsection{Proof of above theorem.}
This proof will involve gluing torsos along cliques of size at most \( \ell \).

Let \(C\) be a clique in \(G\) and let \(\sigma_C = (u_1, \ldots , u_k)\) be a vertex ordering of \(V(C)\), and let \(C \leq \ell \). For any arbitrary clique \(J\), we define a rainbow-vertex \(w \in V(J)\) as a vertex where for any \(x, y \in V(J)\), the edges \(wx\) and \(wy\) are on different pages. We want the book embedding to have the structure \((\underbrace{u_1, u_2, \ldots, u_k}_{\text{Vertices in } C}, \underbrace{v_1, v_2, \ldots, v_l}_{\text{Vertices not in }C})\).

To prove this theorem, we will use a common technique in graph theory. We will strengthen the lemma so that we may use induction to prove the statement.
\begin{lemma}\label{lem:Hickingbotham_Lemma}
	Let \(G\) be a graph where \(\pn(G) \leq s\) and \(\chi(G) \leq c\), and a clique \(C\) with an ordering \(\sigma_C\). Let \(|C| \leq \ell\). There exists a \(cs + \ell\)-stack layout \((\leq, \psi)\) of \(G\) where:
	\begin{enumerate}
		\item The vertex ordering has the structure \((\underbrace{u_1, u_2, \ldots, u_k}_{\text{Vertices in } C}, \underbrace{v_1, v_2, \ldots, v_l}_{\text{Vertices not in }C})\).
		\item For every \(u \in V(C)\), the edges \(\lbrace uv \in E(G) : u \leq v \rbrace\) are a monochromatic star.
		\item For every clique \(J\), the last vertex of \(J\) is a rainbow-vertex.
	\end{enumerate}
\end{lemma}
\begin{proof}[Proof of \cref{lem:Hickingbotham_Lemma}]
	Let \((\leq_a, \psi_a)\) be a \(s\)-stack layout of \(G\) and let \(\rho: V(G) \rightarrow \lbrace 1, 2, \ldots, c\rbrace\) be a proper colouring of \(V(G)\).

	Let \(u_1, \ldots, u_k\) be the vertices of \(C\) ordered by \(\sigma_C\). Note that \(k \leq \ell\). Then the new ordering starts with \(u_1 \leq u_2 \leq \ldots, \leq u_k\), and all vertices not in \(K\) are placed after, according to \(\leq_a\).
	Then the stack assignment \(\psi\) is now defined. For every edge \(u_i v\) where \(u_i \in V(C)\) and \(u_i \leq v\), define \(u_i v = i\). Otherwise, if neither \(u\) or \(v\) are in \(V(C)\), and \(u \leq v\), then let \(\psi(uv) = (\rho(u), \psi_a(uv))\). Then we have at most \(|\rho| |\psi_a| + k \leq cs + \ell\) pages.

	We shall show that \((\leq, \psi)\) is a proper book-embedding. Suppose we have a pair of edges \(uv\) and \(xy\) which cross, and \(\phi(uv) = \phi(xy)\). Suppose that \(u\) is the smallest vertex in the ordering \(\leq\). If \(u \in V(C)\), then the edge \(uv\) is assigned to its own page, meaning that it cannot cross \(xy\). So \(x = u\), but we can draw \(uv\) and \(uy\) on a single page. Thus they do not cross. Therefore we have that \(u, v, x, y\) are not in \(V(C)\), and as we preserve the original book-embedding, then these edges do not cross. Thus shown.
	\par
	We have that properties 1 and 2 are immediate from the definition of the book-embedding. For property 3, consider a clique \(J\) in \(G\). Then we must show the last vertex of \(J\) is rainbow. Suppose the last vertex of \(J\) is \(w\), and let \(u, v\) be earlier vertices. Since \(u\) and \(v\) necessarily are assigned different colours in the colouring, then \(\psi(uw) = (\rho(u), \psi_a(uw))\) and \(\psi(vw) = (\rho(v), \psi_a(vw))\). Therefore, the two edges are on different pages. Thus \(w\) is a rainbow vertex.
\end{proof}

\subsubsection{Full proof}
\begin{theorem}
	Suppose \(G\) has a tree-decomposition \((B_x: x \in V(T))\) with torsos \(G \langle B_x \rangle\) and adhesion at most \(\ell\). Order the vertices of \(T\) with a breath-first search, and relabel the vertices \(v_0, \ldots, v_k\) with respect to the bfs ordering. Define \(G_i := G \langle B_{v_i} \rangle \). Suppose that for all torsos \(G_i\), \(i \in [k]\), we have that \(\pn(G_i) \leq s\) and \(\chi(G_i) \leq c\). Then there is a book-embedding of \(G\) with pagenumber of at most \(cs + \ell\).
\end{theorem}
Recall that for a breadth-first search, we maintain the property that for all \(i\), \(T[v_0, \ldots, v_{i - 1}]\) is also a tree and \(v_i\) is adjacent to one of \(v_0, \ldots, v_{i}\).
\begin{proof}
	To prove the statement, we shall prove the stronger statement that there exists a book-embedding with the properties described with the lemma above using induction. In particular, we will have that the last vertex of any clique \(J\) is a rainbow vertex.

	Suppose \(k = 0\). Then \(G_0\) is a single graph with \(\pn(G) \leq s\). Choose \(C\) to be an arbitrary vertex \(v\) in \(G_0\). Then by the lemma above, there is a book-embedding with pagenumber at most \(cs + 1\) and every last vertex in a clique is a rainbow vertex.

	Suppose \(k = n\). Let \(C\) be the adhesion clique between \(G_n\) and the rest of \(G\), where \(G_n\) is a leaf of the tree \(T\). Denote the induced subgraph \(G[V(G) - V(G_n - C)]\) as \(G'\). Then let \((\leq_n, \psi_n)\) be the \(cs + \ell\)-pagenumber book-embedding of \(G_n\) that starts with \(V(C)\), and let \(\sigma_C\) be the ordering of \(V(C)\) by \(\leq_n\). Let \((\leq_{n-1}, \psi_{n-1})\) be the stack-embedding of \(G'\). By induction, this has a \(cs + \ell\)-page book-embedding with the properties described above.

	\paragraph{Construction of new book-embedding}
	Let \(w \in V(K)\) be the last vertex of \(K\) with respect to \(\leq_{n-1}\). Then insert \(V(G_n - C)\) between \(w\) and its successor in the order of \(\leq_{n}\). For the page assignment \(\psi\), we have that if \(uv \in E(G')\), then \(\psi(uv) = \psi_{n-1}(uv)\). For the remaining edges, we can permute the edge assignments of \(\psi_n\) such that for all \(u \in V(K)\), we have that \(\psi(E_u) = \psi_n(uw)\). We can do this as \(w\) is a rainbow vertex and the edges \(E_u\) are assigned to a unique page in \(\psi_n\). Finally, let \(\psi(uv) = \psi_n(uv)\) for the remainder of the edges. Denote the new ordering and assignment as \((\leq, \psi)\).
	\paragraph{Proof that this is a valid book-embedding}
	We claim that \((\leq , \psi)\) is a stack layout that satisfies the induction hypothesis. Suppose that \(\psi(uv) = \psi(xy)\). If \(uv, xy \in E(G')\), then by the induction hypothesis, they do not cross. Similarly, if \(uv, xy \in E(G_n)\), then they will not cross, by the above lemma. If \(uv\) is in \(E(G')\) and \(xy \in E(G_n)\), then they will go over each other or be sequential and therefore will not cross.
	Finally, if \(u, v, x, y \in C\), then by the induction hypothesis on \(G'\), they do not cross either. The final case is if \(uv \in E(G_{i + 1})\) and \(u \in V(C)\), \(v \notin V(C)\), \(xy \in E(G')\). If \(uv\) and \(xy\) cross, then we have that \(xy\) and \(uw\) will cross. But this will contradict the page-embedding of \(G'\).

	Let \(J\) be a clique in \(G\), and \(w\) is its final vertex. Then if \(J\) is in \(G'\), then \(w\) is a rainbow-vertex. Otherwise, the last vertex is contained in \(G_n\). By the above lemma, \(w\) must also be a rainbow vertex. Thus shown.
\end{proof}

\subsubsection{Bounds on pagenumber of \(\ell\) and \(c\)}

We have some bounds in terms of pagenumber on some of these constants, however these constants are not always tight. In particular, we can get better bounds for planar graphs.

\begin{lemma}[Bounded pagenumber implies bounded clique-number]
	If \(pn(G) \leq s\), then \(G\) does not contain any cliques on more than \(2s+1\) vertices.
\end{lemma}

\begin{proof}
	If \(G\) has a clique of size \(k\), then embedding the clique of size \(k\) and therefore \(G\) requires at least \(\lfloor \frac{k}{2} \rfloor\) pages, from \cref{thm:Pagenumber_Complete_Graph}. Therefore, if we can embed \(G\) in \(s\) pages, then the largest clique of \(G\) is at most \(2s + 1\). Therefore, \(\ell \leq 2s + 1\).
\end{proof}
As we have the bound of \(\chi(G) \leq 2 \pn(G) + 2\), from \cref{thm:Colouring_Bound}, we have a bound that does not depend on the chromatic number of \(G\).
\begin{corollary}[Bounded pagenumber of tree-decompositions]\label{corr:bded_pn_tree_decomp}
	Let \(G\) be a graph with a tree-decomposition \((B_x: x \in V(T))\) where each torso \(G \langle B_x \rangle\) has pagenumber \(\leq s\). Then from \cref{thm:clique_sum_pagenumber_bound}, \(G\) has pagenumber at most \(2s^2 + 4s + 1\).
\end{corollary}

\subsubsection{Bounds on pagenumbers of planar graphs}
This theorem also tells us that pagenumbers of planar graphs are bounded.

\begin{theorem}\label{thm:Planar Graph Hickingbotham Bound}
	Let \(G\) be a planar graph. Then \(\pn(G) \leq 11\).
\end{theorem}

We will use \cref{corr:planar_graphs_4_connected_cliqesums}. We will also use the fact that all planar graphs are \(4\)-colourable \cite{appelEveryPlanarMap1989} and the fact that all 4-connected planar graphs are Hamiltonian, from Tutte \cite{tutteTheoremPlanarGraphs1956}.

\begin{theorem}[Tutte\cite{tutteTheoremPlanarGraphs1956}]\label{thm:4-connected_planar_ham_cycle}
	All 4-connected planar graphs are Hamiltonian.
\end{theorem}

\begin{lemma}\label{lem:clique_sum_connected}
	All graphs have a tree-decomposition where every torso is a \(k\)-connected graph, or has at most $k-1$ vertices, with adhesion set size at most \(k-1\).
\end{lemma}
\begin{proof}
	If a graph \(G\) is not \(k\)-connected, we can find a set \(S\) of size at most \(k-1\) such that \(G - S\) is disconnected. Then we repeat this operation on the connected components of \(G - S\), until each component either has \(k-1\) vertices or is \(k\)-connected. So we can construct a tree-decomposition where every torso is \(k\)-connected and the adhesion size is at most \(k-1\).
\end{proof}

\begin{corollary}\label{corr:planar_graphs_4_connected_cliqesums}
	All planar graphs \(G\) have tree-decompositions with the torsos being \(4\)-connected planar graphs and adhesion at most \(3\).
\end{corollary}


\begin{proof}[Proof \cref{thm:Planar Graph Hickingbotham Bound}]
	Let \(G\) be planar. Then \(G\) has a tree-decomposition of adhesion size at most \(3\) with the torsos being \(4\)-connected, from \cref{lem:clique_sum_connected}. However, this implies that the torsos are Hamiltonian, from Tutte \cite{tutteTheoremPlanarGraphs1956}, thus the number of pages needed for each torso is \(2\). Therefore from \cref{thm:clique_sum_pagenumber_bound}, we have that the pagenumber is at most \(2 \times 4 + 3 = 11\).
\end{proof}
This upper bound is much worse than the tight upper bound found by Yannakakis \cite{yannakakisEmbeddingPlanarGraphs1989}. However, the proof given above is more general and will be used throughout to prove several bounds.
We will discuss the \(K_5\)-minor free case. If \(G\) is \(K_5\)-minor free, then we do not need to worry about the GMST to form an upper bound. We will use Wagner's theorem.
\begin{theorem}[Wagner's theorem\cite{wagnerUeberEigenschaftEbenen1937}]\label{thm:WagnersTheorem}
	If \(G\) is \(K_5\)-minor-free, then \(G\) has a tree-decomposition of adhesion $\leq 3$ where every torso is eithe a planar graph or the Wagner graph \(V_8\).
\end{theorem}
A description of the Wagner graph is in \cref{fig:wagner}. The edges are coloured such that the internal edges are on different pages. The spine edges (the edges that are on the outerface) are the ones which can go on any page.
\begin{figure}[h]
	\centering
	\begin{tikzpicture}
		\tikz \graph [nodes = {draw, circle}, clockwise, empty nodes] {
	subgraph C_n [n=8, red];
	1 --[red] 5;
	2 --[blue] 6;
	3 --[green] 7;
	4 --[yellow] 8;
};

	\end{tikzpicture}
	\caption{The Wagner graph. Notice how the clockwise circular ordering of the vertices of the Wagner graph needs 4 pages to embed the graph. }\label{fig:wagner}
\end{figure}
Equivalently, if \(G\) is \(K_5\) minor free, then \(G\) has a tree-decomposition \(\tree\) where \(\tree\) has adhesion at most 3, and every torso of \(\tree\) is either planar or the Wagner graph. We will use this to prove the theorem below.
\begin{theorem}
	If \(G\) is \(K_5\)-minor free, then \(G\) has pagenumber \(\leq 19\).
\end{theorem}

\begin{proof}
	Suppose \(G\) is \(K_5\)-minor free. Then by Wagner's theorem \cite{wagnerUeberEigenschaftEbenen1937}, \(G\) has a tree-decomposition of adhesion at most 3 where every torso is either a planar graph or the Wagner graph.
	We have that planar graphs are \(4\)-colourable and have pagenumber \(\leq 4\). We have that the Wagner graph is \(3\)-colourable and has pagenumber \(\leq 4\). Therefore, we have that if \(G\) is \(K_5\)-minor free, then \(G\) has pagenumber at most \(4 \times 4 + 3 = 19\).
\end{proof}
We refer to \cref{fig:wagner} for a description of a circular ordering of a Wagner graph and a book embedding.

\subsection{Bounded treewidth and page number}\label{ssec:Bounded_Treewidth}
\begin{theorem}[Ganley + Heath\cite{ganleyPagenumberTrees2001}]\label{thm:bded_treewidth_bded_pagenumber}
	Every graph \(G\) with \(\tw(G) \leq k\) can be embedded on $k + 1$ pages.
\end{theorem}
In the original proof, they considered the case where \(G\) is a \(k\)-tree. We will bypass using \(k\)-trees and consider a tree-decomposition of \(G\) directly.

\begin{proof}
	Consider a tree-decomposition of \(G\) with bags $B_x$ and tree $T$. Perform a depth-first search on \(T\), starting at an arbitrary root node \(r\). Let the ordering of the book-embedding \(\sigma(v)\) of a vertex \(v\) in \(V(G)\) be determined by the first time \(x \in T\) appears, where \(v \in B_x\). Within each bag, if two vertices appears in the bag first, then they are ordered arbitrarily in the book-embedding. Now consider the subtree \(T_v\) induced by the bags \(B_x\) containing \(v\). We now consider colouring the subtrees \(T_v\) for all \(v \in G\) such that no overlapping subtrees have the same colour. Let \(H\) be the intersection graph of the subtrees, where \(V(H) = \lbrace T_v : v \in G \rbrace\) and \(T_u T_v \in E(H)\) if there exists a bag \(B_x\) such that \(u, b \in B_x\). We have that \(H\) is perfect, and thus \(\chi(H) = \omega(H)\). Then as \(\tw(G) \leq k + 1\), then the size of a clique in \(H\) is at most \(k + 1\). Thus \(H\) is \(k + 1\)-colourable.
	\paragraph{}
	We now use this to assign the edges of \(G\) a page. Let \(c(T_v)\) be the colour assigned to \(T_v\). Colour each edge \(uv \in E(G)\) as follows:
	\begin{equation}
		c(uv) =
		\begin{cases}
			c(T_u) & \text{ if } \sigma(u) \leq \sigma(v), \\
			c(T_v) & \text{ if } \sigma(v) \leq \sigma(u)
		\end{cases}
	\end{equation}
	Then we claim that this is a proper book-embedding of \(G\). Suppose we have that edges \(uv\), \(xy\) cross, so \(\sigma(u) \leq \sigma(x) \leq \sigma(v) \leq \sigma(y)\). However, this implies that there exists a bag \(B\) such that \(u, x, v \in B\), as we have that \(uv\) is an edge in \(B\) and we do a depth-first search to establish the ordering. So \(u, x, v\) they are in the same bags. However, this implies that the trees \(T_u\) and \(T_x\) intersect, meaning that \(c(uv) \neq c(xy)\). Finally, the number of pages used is \(\chi(H) \leq k + 1\), so \(\pn(G) \leq k + 1\). Thus shown.
\end{proof}

We have a simpler proof if the tree is a path. We go from one end of the path to the other, and add vertices to the book-embedding in the order of the first time they appear. Then we colour each vertex such that in each bag, no two vertices are assigned the same colour. We can do this as this is the same intersection graph as above. The rest of the proof follows.

\subsection{Planar graphs}\label{ssec:Planar_Graphs}
\begin{theorem}[Yannakakis \cite{yannakakisEmbeddingPlanarGraphs1989}]\label{thm:4Pages_Planar}
	Planar graphs can be embedded on at most four pages.
\end{theorem}
We have shown above a proof that the number of pages necessary to embed a planar graph is bounded. However, the proof given by Yannakakis is tight, as there exist planar graphs that need four pages \cite{yannakakisPlanarGraphsThat2020} \cite{kaufmannFourPagesAre2020}. We need the fact that the number of pages to embed a planar graph is bounded for proving that graphs embedded on a surface of bounded genus has bounded pagenumber.

\section{Graphs embedded on a surface of bounded genus}\label{sec:pagenumber_bounded_genus}

\begin{theorem}[Heath and Istrail\cite{heathPagenumberGenusGraphs1992}]\label{thm:Genus_pagenumber_bound}
	Let \(g\) be the genus of a graph \(G\). For all graphs \(G\), \(\pn(G) \leq O(g)\).
\end{theorem}
Note that this bound extends the one found by Yannakakis \cite{yannakakisEmbeddingPlanarGraphs1989} to graph families of bounded genus.
The best known bound is \(O(\sqrt{g})\), found by Malitz\cite{malitzGenusGraphsHave1994}.

It was shown by Heath and Istrail that the family of graphs of bounded genus have bounded pagenumber.
We refer to the ``layout'' of the graph as the book-embedding of the graph and ``embedding'' as the surface-embedding. We refer to orientable surfaces as genus \(g\) as a sphere with \(g\) handles, and a nonorientable surface of genus \(g\) as a sphere with \(g\) cross-caps. We define the orientable genus of a graph \(G\), denoted \(\gamma(G)\), as the minimum orientable surface genus that \(G\) can be embedded on. The nonorientable genus of a graph \(G\), denoted \(\tilde{\gamma}(G)\), is the minimum nonorientable genus surface that \(G\) can be embedded on. Mohar\cite{moharOrientableGenusGraphs1998} claims that \(\tilde{\gamma}(G) \leq 2 \gamma(G) + 1\) for all graphs, meaning that if the orientable genus is bounded, then the non-orientable genus is bounded. Note that, Auslander et al.\cite{auslanderImbeddingGraphsManifolds1963} showed that there exists graphs which are embeddable on the projective plane who has arbitrarily large orientable genus.
\paragraph{Proof}
We say that the embedding is \(2\)-cell if every face is homeomorphic to an open disc in \(\mathbb{R}^2\). Any embedding of \(G\) onto an orientable surface is a 2-cell embedding, but this does not hold for nonorientable surfaces, but we assume there exists a \(2\)-cell embedding.
Heath and Istrail rely on decomposing the graph \(G\) of genus \(\gamma(G)\) into a planar spanning subgraph \(G_p\) of \(G\) such that:
\begin{enumerate}
	\item The edges in \(E(G) - E(G_p)\) attach to the boundary vertices of \(V(G_p)\).
	\item Adding an edge from \(E(G) - E(G_p)\) to \(G_p\) breaks the above condition.
\end{enumerate}
To talk about graphs embedded in surfaces, we assign to each face a cyclic permutation \(\sigma_v\) which represents the sequence of vertices encountered when traversing the boundary of a face in counterclockwise order.

This is enough to represent any graph in an orientable surface, but not enough for a non-orientable surface. We have to attach on an orientation to each edge, where each edge is either orientation-preserving or orientation-reversing.

We have that a planar-nonplanar decomposition of \(G\) is a triple \((R, G_P, E_N)\) where \(R\) is a rotation of \(G\) representing the surface embedding on the surface \(S\), \(G\) is a spanning planar graph, and \(E_N = E - E(G_P)\).
This satisfies a list of properties:
\begin{enumerate}
	\item The subrotation induces a planar embedding of \(G_p\), so we can arrange \(G\) on the surface \(S\) such that the embedding of \(G_p\) is planar.
	\item For each \(vw \in E_N\), we have that \(v\) and \(w\) live on the outerface \(F_0\).
	\item \(E(G_P)\) is maximal, so we cannot add edges from \(E_N\) to \(G_P\) without breaking properties 1 and 2.
\end{enumerate}

\subsubsection{Decomposing graphs on surfaces}\label{sssec:Planar_nonplanar_decomp}
We first have to know that the planar-nonplanar decomposition exists.

Suppose \(G\) is embedded on an surface \(\Sigma\). Then we wish to triangulate \(G\) to form \(G_T\). We choose a single triangle as the starting point and we add traces to the planar part incrementally. At each step, we set \(G_P\) to be the current planar part and \(E_N\) as the edges that are outside the planar part. There are two types of edges in \(E_N\): edges which have both endpoints in \(V(G_P)\), so cannot become edges of \(G_P\), and edges that have either one or no endpoints in \(V(G_P)\).

We want to maintain the property that if \(v \in G_P\), and edge \(vw \in E_n\), then \(v\) is a vertex on the boundary of \(G_p\).
\paragraph{Adding vertices to biconnected block}
For a current boundary of the outerface of \(G_P\), if \(v_i \rightarrow v_j \rightarrow v_k\) is trace with no edge of \(E_N\) incident to \(v_j\), then \(v_i v_k \in E(G_T)\) is called a safe edge. If \(v_i \rightarrow v_j\) is on the boundary of \(G_P\), and \(v_k \notin V(G_P)\), and \(v_i,v_j,v_k\) is the boundary of a face, then \(v_k\) is a safe vertex and we can add it to \(G_P\).
\paragraph{Creating new biconnected block}
If no \(v_k\) exists, then we find a \(w'\) which is the newest vertex in \(V(G_P)\) adjacent to a vertex \(w\) not in \(V(G_P)\). We then add the vertex \(w\) and the edge \(w w'\) to \(G_P\). Then we add all safe edges. This is so that every edge in \(E_p\) maintains the property that both endpoints are on the boundary.

We claim that after repeating this operation, then we have that every edge in \(E_N\) (edges not in \(G_P\)) satisfy the two properties above. If we have that an edge \(vw\) has \(v\) added, then we should be able to add \(w\) as a safe vertex or biconnected block. If an edge \(vw\) has neither \(v\) or \(w\) added to \(G_P\), then the algorithm has not finished yet. By connectivity, we can add \(v\) and \(w\) at some stage and therefore go to part 1. This has the corollary that every vertex is in \(G_P\).

Now we have that \(E_N\) has every edge which cannot be added to \(G_P\) without crossing over another edge, and that \(G_P\) is maximal. Then we have that all edges in \(E_N\) satisfy the conditions lised above.
\todo{Add pictures! this proof needs lots of pictures}

\subsubsection{Level sets and cycles}
On a planar graph \(G\), we want to separate out vertices depending on how far away they are from the outerface. We fix a single outerface \(F_0\) and define the first level set \(V_0\) as the vertices adjacent to \(F_0\). We then define the \(i\)-th level set, \(V_i\) inductively. Consider the induced graph on \(V(G) - \cup_{k = 0}^{i-1} V_k\). Then we define the vertices adjacent to \(F_0\) in this induced graph, where we expand \(F_0\) to include the vertices. This partitions \(V(G)\).

We then define \(C_0\) to be the edges adjacent to \(F_0\) in this decomposition. Then we want \(C_i\) to be the edges adjacent to \(F_0\) in this decomposition. We define the chord edges \(K_i\) to be the edges between vertices in \(V_i\) that are not edges in \(C_i\). Finally, we define the edges between faces, \(E_i\) as the edges that are between vertices on level \(V_i\) and \(V_{i + 1}\).

\begin{lemma}
	For all faces \(F\) in \(G\), the vertices around \(F\) are either all in one \(C_i\) or they are in \(C_i\) and \(C_{i + 1}\) for some \(i\).
\end{lemma}

\begin{proof}
	Let \(i\) be the smallest value such that \(v \in V_i\) is on the boundary of \(F\). Now we have that \(G[V(G) - \cup_{j = 1}^{i} V_i]\) will also remove \(v\). However, this removes all the edges next to \(v\), therefore all vertices that are on the boundary of \(F\) will either be in \(V_i\) or \(V_{i + 1}\).
\end{proof}
We refer to the faces that have vertices in only \(V_i\) as chordal and the faces that are between \(V_i\) and \(V_{i + 1}\) as non-chordal.

We define a weak triangulation of \(G\) to be a triangulation \(G'\) such that all faces except for the outerface is a triangulation.
\begin{lemma}
	There exists a weak triangulation of \(G\), \(G'\) which preserves the level sets \(V_i\) and edge sets \(E_i\), \(C_i\), \(K_i\) for all \(i\).
\end{lemma}

\begin{proof}
	If \(F\) is a chordal face of \(G\), then any triangulation maintains the property. If \(F\) is nonchordal and the boundary has edges in \(V_i\) and \(V_{i + 1}\), then add edges that are only between vertices in \(V_i\) and \(V_{i + 1}\). This will suffice to build a new triangulated graph \(G'\) where all vertices and edges are in the correct place.
\end{proof}

\subsubsection{Classifying nonplanar edges according to homotopy class}

We can then form a directed cycle \(C_0\) induced by \(F_0\). Each vertex on the boundary of \(F_0\) appears at least once, and twice if it is an \textit{articulation point}. Each edge on the boundary of \(F_0\) is encountered at least once on this cycle. Heath and Istrail refer to a directed subpath of the cycle \(C_0\) as a trace, so trace \(T = v_1 \rightarrow v_2 \rightarrow \cdots \rightarrow v_t\). The inverse trace is \(T^{-1} = v_t \rightarrow v_{t-1} \rightarrow \cdots \rightarrow v_1\). We now wish to partition \(E_N\) into equivalence classes. Suppose that \(u_1v_1, u_2v_2 \in E_N\) are part of the boundary of the same face \(F\) on the embedding of \(G\). Then \(u_1v_1\) and \(u_2v_2\) are \textit{homotopic} (with respect to \(F\)) if:
\begin{enumerate}
	\item \(u_1v_1\) and \(u_2v_2\) are the only edges of \(E_N\) on the boundary of \(F\)
	\item There exist traces \(T_u = u_1 \rightarrow \cdots \rightarrow u_2\) and \(T_v = v_1 \rightarrow \cdots \rightarrow v_2\) such that \(T_u\) and \(T_v\) are on the boundary of \(F\).
\end{enumerate}
We may think of \(G_n\) as living on a locally flat part of \(S\) and the homotopy class \(u_1v_1\) and \(u_2 v_2\) living on a handle (alternatively, passing through a crosscap such that they bound a face). Then if we take \(G_n\) to a point, there exists a \textit{homotopy} (in the topological sense) from \(u_1v_1\) to \(u_2v_2\). These form equivalence classes of the nonplanar edges.

\begin{lemma}
	If \(C\) is a homotopy class of edges \(u_1 v_1, \ldots, u_k v_k\) with a natural order, then we can build traces \(T_1\) and \(T_2\) by building the trace from \(u_1\) to \(u_k\) passing through all \(u_i\), and \(v_1\) to \(v_k\) passing through all \(v_i\).
\end{lemma}
We refer to a homotopy class as orientable if \(T_1\) and \(T_2\) go in opposite directions, and non-orientable if \(T_1\) and \(T_2\) go in the same direction.

\begin{lemma}
	We have that if \(G\) is embedded in an orientable surface, then every homotopy class is orientable.
\end{lemma}
\begin{proof}[Sketch]
	We have that if a homotopy class is non-orientable, then on the handle the class sits on, the edges must cross. However, we have the graph is embedded on the surface, therefore this cannot happen. Thus shown.
\end{proof}

\begin{lemma}
	If \(G\) is \(2\)-cell embedded on a surface of Euler genus \(g\), then any planar-nonplanar decomposition has at most \(3g-3\) homotopy classes.
\end{lemma}
\begin{proof}
	Decompose \(G\) to a \((R, G_P, E_N)\) decomposition of \(G\). Suppose \(E_N \neq \emptyset\). Then identify \(G_P\) to a single point, and identify each homotopy class to a single edge. Then draw a circle around the point \(G_P\), and place vertices where the circle intersects all edges. Then delete the vertex \(G_P\), and call the new graph \(H\). We have that \(n = |V(H)|\), \(m = |E(H)|\), \(h\) is the number of homotopy classes, and \(f\) is the number of faces. We have that \(n - m + f = 2 - g\). Since \(H\) is cubic as every vertex has two edges on the circle and one on the homotopy class, then \(3n = 2m\) by the handshaking lemma. Since there is only one nonplanar edge for each homotopy class, \(n = 2h\). The interior face of \(H\) has \(v\) incident edges, and the remaining \(f-1\) faces have at least 3 incident edges each, as we can identify the two homotopy classes bordering a face with four edges together. Therefore, we have that \(3(f-1) + n \leq 2m\), by double counting faces and edges. Thus, we have that
	\begin{align*}
		3n  & \geq 6(f - 1) + n         \\
		2n  & \geq 6f + 6               \\
		4h  & \geq 5 f - 6              \\
		4h  & \geq 5(2 - g + m - n) - 6 \\
		4h  & \geq 6 - 6g + 3n          \\
		4h  & \geq 6 - 6g + 6h          \\
		-2h & \geq 6 - 6g               \\
		h   & \leq 3g - 3
	\end{align*}
	\(3g - 3 \geq h\) by manipulating the inequalities.
\end{proof}

\subsubsection{Proving graphs with bounded number of homotopy classes have bounded pagenumber}\label{sssec:bounded_pagenumber_homotopy}
\begin{lemma}\label{lem:planar_nonplanar_orientable}
	Suppose \(G\) has a planar-nonplanar decomposition \((R, G_P, E_N)\) on an orientable surface \(\Sigma\). Then \(G\) can be embedded on at most \(18g - 5\) pages.
\end{lemma}
\begin{proof}
	We use Yannikakis' proof in \cref{ssec:Planar_Graphs} to lay out the nonplanar spanning subgraph \(G_P\) on four pages, maintaining the cyclic order of vertices. Then we can combine each blocks to form a 4 page layout of the graph. For each homotopy class in \(E_P\), we allocate three pages. One page is for vertices in the same block, and the other two pages are used for edges between blocks, the biconnected components of \(G\). We need two as we could have some which span blocks in a way that forces them to be on different pages. Therefore, we need at most \(4 + 3(6g - 3) = 18g-5\) pages if \(G\) has a planar-nonplanar decomposition.
\end{proof}

\begin{lemma}\label{lem:planar_nonplanar_nonorientable}
	Suppose \(G\) has a planar-nonplanar decomposition \((R, G_P, E_N)\) on a non-orientable surface \(\Sigma\). Then \(G\) can be embedded on at most \(9g - 1\) pages.
\end{lemma}
\begin{proof}[Proof sketch]
	\todo{Flesh out details completely}
	We want to add edges in a controlled way so that the traces that are reversed become non-reversed. This is done by adding edges between vertices so that we can invert the ordering on the circle such that we have that the vertices in one homotopy class have a non-crossing page embedding. However, an issue is chords that go between traces that are inverted. We go around this problem by removing chords and adding them to a separate page where there are finitely many pages wrt to the genus of the surface. Let \(\mathcal{C}_{i,j}\) be a chord class when it is the set of chords that go between traces \(T_i\) and \(T_j\), where \(T_i\) and \(T_j\) both go clockwise or counterclockwise. Note that the number of chord classes \(\mathcal{C}_{i,j}\) is bounded by the genus of the graph, and we can embed the chord classes that share a trace onto a single page. As there is a bounded number of chord classes, it must hold that the number of pages is finite.

	Consider the scenario where the graph $G_p$ is a cycle with a cyclic ordering. Then every trace on the boundary is a path on the cycle. Recall that a trace $T_1$ and $T_2$ are paired if there is a nonplanar edge between the two traces. For every two pairs of traces that are non-orientable, we label one to be non-reversed and one to be reversed, with arbitrary allocation. We then merge all the consecutive reversed and non-reversed traces together. Then we add an edge around each reversed trace, called bypassing edges. We then use Heath's algorithm to find an edge colouring with 7 colours.

	Now consider the case where $G_p$ is outerplanar. We cannot do the unreversing trick anymore, because of edges inside of the graph. However, we can form a path with these edges and colour them with a finite number of colours as well.
	\todo{finish this case!}
	We bundle chords together and the number of chords that exist is $O(g)$ as there are at most $g$ reversed-unreversed trace changes on the boundary of the surface. As no chords cross, then we have that there are $O(g)$ chords. 

	Finally, consider the case where $G_p$ has a block-cut tree. We still have the same chords from before, and now we have that chords can be used to draw a path that goes back and forth between traces.
	\todo{finish this proof!}

	We first merge traces, and then build a boundary between two vertices. Then we have two subgraphs and two traces. 
\end{proof}

% !TEX root = ./thesis.tex

\section{Pagenumber of projective-planar graphs}
This section discusses graphs embedded on non-orientable surfaces. Heath and Istrail claimed to prove \cref{lem:planar_nonplanar_nonorientable}.
\begin{conjecture}\label{lem:planar_nonplanar_nonorientable}
	Suppose a graph \(G\) has a planar-nonplanar decomposition \((G_P, E_N)\) on a non-orientable surface \(\Sigma\) of non-orientable genus $g$. Then \(G\) can be embedded on \( O(g)\) pages.
\end{conjecture}
However, the outline given in their paper is insufficient to prove \cref{lem:planar_nonplanar_nonorientable}. In fact, their proof method fails when a graph is projective-planar, as stated by \textcite{nakamotoBookEmbeddingProjectiveplanar2015} and \textcite{ozekiBookEmbeddingGraphs2019}. We came to the same conclusion after attempting to apply their proof to some simple projective-planar graphs. An example of a projective-planar graph is in \cref{fig:projectiveplanar}.
As \textcite{Blankenship-PhD03} relies on Heath and Istrail's result, we are confident that \cref{conj:bded_had_pn} is still an open problem. 

However, \textcite{nakamotoBookEmbeddingProjectiveplanar2015} prove the following theorem:

\begin{figure}[h]
    \centering
    \includesvg[width = 0.5 \textwidth]{figures/projective_planar_graph.svg}
    \caption[Projective planar graph]{Graph embedded on the projective plane, the purple circle. This graph is not a triangulation of the projective plane as the edges that pass through the projective plane do not bound a triangle.}\label{fig:projectiveplanar}\end{figure}

\begin{theorem}[\textcite{nakamotoBookEmbeddingProjectiveplanar2015}]\label{thm:proj_planar_graphs_9pages}
	Every graph embedded on the projective plane has a book-embedding with nine pages.
\end{theorem}
Firstly, a planar spanning disk is found with nonplanar edges. Then a non-contactable cycle is found which intersects the boundary of the disk exactly twice. Finally, a book-embedding is found.

This proof relies on triangulations of surfaces. Recall a graph $G$ is a \textit{triangulation} of a surface $\Sigma$ if every face has three distinct vertices on its boundary. Note that unlike graphs embedded on the sphere, there exists graphs embedded on surfaces which are edge-maximal but not a triangulation. \textcite{hararyMaximalToroidalGraph1973} gives an example of such a graph embedded on a torus. 

\begin{theorem}\label{thm:triangulation_subgraph}
    Every graph embedded on a surface $\Sigma$ is a subgraph of a triangulation of $\Sigma$.
\end{theorem}

\begin{proof}
    Let $G$ be a graph embedded on $\Sigma$. If a face is bounded by a simple cycle $C$ of length $k \geq 4$, then add a vertex $v$ in the centre of the face and have every vertex in $C$ be adjacent to $v$. Suppose a face $F$ has reappearing vertices on its boundary walk, $v_0, \ldots, v_{k-1}$. Then as $G$ is simple and therefore loopless, the reappearing vertices are non-consecutive. Add a maximal outerplanar graph with $k$ vertices on the interior of $F$ with boundary $w_0, \ldots, w_{k-1}$ and add edges $v_i w_i$ and $v_i w_{i + 1}$ for $i = 0, \ldots, {k-1}$ modulo $k$. Doing this operation on every face of $G$ is a triangulation of $\Sigma$, with $G$ a subgraph of this graph. An example of such a triangulation is \cref{fig:triangulation}.
\end{proof}
\begin{figure}[h!]
    \centering
    \includesvg[width = 0.5 \textwidth]{figures/facetriangulation.svg}
    \caption[Face triangulation]{A triangulation of a face with five vertices on its boundary, where there exists a duplicate green vertex.}\label{fig:triangulation}
\end{figure}
As a consequence, every projective-planar graph is a subgraph of a triangulation of the projective plane. 

Recall a cycle $C$ in $G$ embedded on $\Sigma$ is \textit{contractible} if $C$ in $\Sigma$ as a loop is null-homotopic. Otherwise, $C$ is \textit{ non-contractible}. A subpath of a path $P$ in a graph $G$ between vertices $x$ and $y$ inclusive is $P[x, y]$. A subpath between $x$ and $y$ exclusive is $P(x, y)$. 
A \textit{link} of a vertex $v$ is the cycle that goes around the boundary of the union of the faces incident to $v$. This coincides with the definition of a link in a simplicial complex from topology. 

To prove this theorem, \cref{lem:proj_planar triangulation} is used to find a planar spanning subgraph.

\begin{lemma}\label{lem:proj_planar triangulation}
    Let $G$ be a triangulated projective-planar graph. Then there exists a planar spanning subgraph $G_P$ with outer cycle $B$ that is contractible, nonplanar edges $E_N$, and a non-contractible cycle $C$. Furthermore, there exists two vertices $x, y$ such that $\{x, y \} = V(C) \cap V(B)$, $xy \in E_N$ and $C - xy$ has no edges in $E_N$. 
\end{lemma}

\begin{proof}
     Firstly, there exists a non-contractible cycle in $G$. If every cycle of $G$ is contractible, then $G$ is planar. Then $G$ is not projective-planar. Therefore, $G$ has a non-contractible cycle. Now let $C$ be the shortest non-contractible cycle. Let $xy$ be an edge in $C$. Let $P = C - xy$ be a path starting at $x$ and ending at $y$. 
    Number the vertices of $P = v_1, \ldots, v_m$. 

    Locally define the left-hand side and right-hand side of $P$. Take $r_i$ to be the vertex on the right hand of $P$ such that $v_i v_{i + 1} r_i$ bounds a face in $G$. Let $R_i$ be the right-hand path from $r_i$ to $r_{i + 1}$ on the link of $v_i$, disjoint from $P$. Let 
    \begin{equation*}
        R' = v_1 r_1 + \bigcup_{i = 1}^n R_i + r_{m-1} v_m
    \end{equation*}
    be a walk from $x$ to $y$ disjoint from $P$. Take $R$ to be a path from $x$ to $y$, $R \subseteq R'$. Repeat for $L$, the left-hand side path from $x$ to $y$. Now $P, R, L$ are internally-disjoint paths. Suppose $R$ and $L$ have a common inner vertex $k$. Then this means that for two vertices $x_i, x_j$, $k$ is a right-hand neighbour of $x_i$ and a left-hand neighbour of $x_j$. Now the cycle $(k v_i) (v_i v_{i + 1}), \ldots , (v_j k)$ is non-contractible, as the cycle passes through $C$ and thus lies on a crosscap. However, $i, j$ is not $1$ or $m$. The only neighbour of $x$ is $r_2$. The only neighbour of $y$ is $r_{m-1}$. $r_1$ and $r_2$ are distinct except when $C$ has three vertices, the shortest cycle possible. In this case, $k$ cannot exist as this implies that $G$ is not simple. Therefore, $C'$ is shorter than $C$, breaking the assumption that $C$ is the smallest non-contractible cycle.

    Now $P, R, L$ are three-internally disjoint paths from $x$ to $y$ where $P \cup R$ and $P \cup L$ are null-homotopic cycles that bound disks $D_1, D_2$. As $P$ is a contractible path, $D_1 \cup D_2$ is also a disk, therefore $R \cup L$ bounds a disk $D$. If every vertex is contained in $D$, then a planar subgraph has been found. Otherwise, find a vertex $v$ not in $D$. Since triangulations are 3-connected, there are three internally-disjoint paths $P_1, P_2, P_3$ from $v$ to $D$, ending at vertices $v_1, v_2, v_3$ respectively. Then two vertices will be distinct, suppose they are $v_1, v_2$. Then there is a path $R'$ from $v_1$ to $v_2$ on the boundary of $D$ such that $R' \cup P_1 \cup P_2$ is a contractible cycle. This is because if $P_1 \cup P_2 \cup R'$ contains a crosscap, then either $P_3 \cup P_2 \cup R'$ contains no crosscap and is a disk, or $P_1 \cup P_3 \cup R'$ is a disk. Relabel if necessary. Then add this disk to $D$ to grow $D$. As this procedure can be repeated for every vertex, there exists a spanning planar subgraph $G_p$ of $G$, where $G_P = D$, the subgraph contained in $D$. Finally, $C$ is a noncontractible cycle where $C \cap \partial D = \{x, y\}$ by construction of the disk $D$. 
\end{proof}
A description of what this construction looks like is in \cref{fig:projectiveplanardecomp}.
\begin{figure}[h]
    \centering
    \includesvg[width = 0.6 \textwidth]{figures/projectiveplanar_decomposition.svg}
    \caption[Projective-planar decomposition]{A decomposition of a projective-planar graph into components in \cref{lem:proj_planar triangulation}.}\label{fig:projectiveplanardecomp}
\end{figure}

Now we will prove \cref{thm:proj_planar_graphs_9pages}. This is done by breaking the disk into two halves and finding a book-embedding where the non-orientable edges are nested.
\begin{proof}
    Let $H$ be a projective-planar graph. Let $G$ be a triangulation of the projective plane, where $H$ is a subgraph.
    Now apply \cref{lem:proj_planar triangulation} to $G$. Let $B_1, B_2$ be the two paths from $x$ to $y$ on the boundary of $B$. Let $D_1$ be the planar subgraph bounded by $P \cup B_1$, similarly for $D_2$. Finally, let $s$ be the vertex clockwise to $x$ on $B$, and let $t$ be the vertex clockwise to $y$ on $B$.

    From \cref{thm:4Pages_Planar}, there exists a four-page embedding $(<_1, \sigma_1)$ of $D_1$ which preserves the cycle $P \cup B_1$. Similarly, there exists a four-page embedding $(<_2, \sigma_2)$ of $D_2$ which preserves the cycle $P \cup B_2$.
    Let $t$ be at the start of the vertex ordering of $<_1$ and $s$ be at the end of the vertex ordering of $<_2$. 
    Combine these two embeddings along $P$ (interlace $<_1, <_2$ along $P$) to form a book-embedding $(<, \sigma)$ of $G_P$ with eight pages. This is a vertex ordering of every edge in $G$ as $G_P$ is spanning. 

    Now for edges in $E_N$. Let $W = B_1[t, x] xy B_2[y, s]$ be a walk which starts at $t$ and ends at $s$. Note that all edges of $E_N$ goes from $B_1$ to $B_2$ because they pairwise cross. Traversing from $t$ to $x$ and from $s$ to $y$ in $<$, every edge in $E_N$ not adjacent to $x$ is nested. Therefore, all edges can be added to a new page. For the inner pages from $B_1$ to $x$, or from $B_2$ to $y$, add to an old page in $D_2$ or $D_1$ respectively. 

    An example of the book-embedding is in \cref{fig:projectiveplanarbookembedding}. The path given is the book-embedding of $B \cup P$, the coloured regions are the book-embeddings of $D_1$ and $D_2$, and the other edges are edges in $E_N$. 
    This embeds a projective-planar graph in nine pages.
\end{proof}


\begin{figure}[h]
    \centering
    \includesvg[width = 0.8 \textwidth]{figures/projectiveplanar_bookembedding.svg}
    \caption[Projective-planar book-embedding]{A book-embedding of a projective plane in nine pages. The blue regions are a book-embedding of $D_1$ and $D_2$ respectively. The central path is the path $B_1 xy B_2$, with the vertices in $D_1$ and $D_2$ in between. The long edge are edges that pass through the crosscap. These edges are nested, so do not pairwise cross in this embedding. These edges can be embedded on a single page. }\label{fig:projectiveplanarbookembedding}
\end{figure}
 This upper bound was improved by \textcite{ozekiBookEmbeddingGraphs2019}. They showed that every projective-planar graph can be embedded in six pages. Their proof used \textit{Tutte paths}, which are paths in planar graphs with certain properties. They tie Tutte paths with the observation in \cref{thm:4-connected_planar_ham_cycle} to find better book-embeddings. 

	\subsection{Graph Minor Structure Theorem}
\textcite{robertsonGraphMinorsXVII1999} provides a rough characterisation of all \(K_t\)-minor-free graphs. 

Every graph that is $K_t$-minor-free can be constructed from the following ingredients. This is a coarse characterisation of $K_t$-minor-free graphs, meaning that a subset, or a single one of these ingredients constitutes a $K_t$-minor-free graph. 
\begin{itemize}
	\item Graphs of bounded Euler genus.
	\item Sets of apex vertices.
	\item Graphs of bounded treewidth.
	\item Sets of vortices on graphs.
\end{itemize}
\textcite{robertsonGraphMinorsXVII1999} showed that every \(K_t\)-minor-free graph can be built up from smaller graphs with the above ingredients.
	\section{Bounds of pagenumbers of graphs}\label{sec:BoundedPagenumber}
\subsection{Tree-decomposition into bounded page number torsos}\label{ssec:Clique_sum_Pagenumber_bound}

This proof has been adapted into the language of tree-decompositions.
\begin{theorem}[Hickingbotham and Wood \cite{hickingbothamStackNumberCliqueSum2023}]\label{thm:clique_sum_pagenumber_bound}
	Let \(G\) be a graph with a tree-decomposition \((B_x: x \in V(T))\) where each torso \(G \langle B_x \rangle\) has pagenumber \(\leq s\) and every torso \(G \langle B_x \rangle\) is \(c\)-colourable. Additionally, we have that the adhesion of this tree is at most \(\ell\).
	Then \(\pn(G) \leq cs + \ell \).
\end{theorem}

\subsubsection{Proof of above theorem.}
This proof will involve gluing torsos along cliques of size at most \( \ell \).

Let \(C\) be a clique in \(G\) and let \(\sigma_C = (u_1, \ldots , u_k)\) be a vertex ordering of \(V(C)\), and let \(C \leq \ell \). For any arbitrary clique \(J\), we define a rainbow-vertex \(w \in V(J)\) as a vertex where for any \(x, y \in V(J)\), the edges \(wx\) and \(wy\) are on different pages. We want the book embedding to have the structure \((\underbrace{u_1, u_2, \ldots, u_k}_{\text{Vertices in } C}, \underbrace{v_1, v_2, \ldots, v_l}_{\text{Vertices not in }C})\).

To prove this theorem, we will use a common technique in graph theory. We will strengthen the lemma so that we may use induction to prove the statement.
\begin{lemma}\label{lem:Hickingbotham_Lemma}
	Let \(G\) be a graph where \(\pn(G) \leq s\) and \(\chi(G) \leq c\), and a clique \(C\) with an ordering \(\sigma_C\). Let \(|C| \leq \ell\). There exists a \(cs + \ell\)-stack layout \((\leq, \psi)\) of \(G\) where:
	\begin{enumerate}
		\item The vertex ordering has the structure \((\underbrace{u_1, u_2, \ldots, u_k}_{\text{Vertices in } C}, \underbrace{v_1, v_2, \ldots, v_l}_{\text{Vertices not in }C})\).
		\item For every \(u \in V(C)\), the edges \(\lbrace uv \in E(G) : u \leq v \rbrace\) are a monochromatic star.
		\item For every clique \(J\), the last vertex of \(J\) is a rainbow-vertex.
	\end{enumerate}
\end{lemma}
\begin{proof}[Proof of \cref{lem:Hickingbotham_Lemma}]
	Let \((\leq_a, \psi_a)\) be a \(s\)-stack layout of \(G\) and let \(\rho: V(G) \rightarrow \lbrace 1, 2, \ldots, c\rbrace\) be a proper colouring of \(V(G)\).

	Let \(u_1, \ldots, u_k\) be the vertices of \(C\) ordered by \(\sigma_C\). Note that \(k \leq \ell\). Then the new ordering starts with \(u_1 \leq u_2 \leq \ldots, \leq u_k\), and all vertices not in \(K\) are placed after, according to \(\leq_a\).
	Then the stack assignment \(\psi\) is now defined. For every edge \(u_i v\) where \(u_i \in V(C)\) and \(u_i \leq v\), define \(u_i v = i\). Otherwise, if neither \(u\) or \(v\) are in \(V(C)\), and \(u \leq v\), then let \(\psi(uv) = (\rho(u), \psi_a(uv))\). Then we have at most \(|\rho| |\psi_a| + k \leq cs + \ell\) pages.

	We shall show that \((\leq, \psi)\) is a proper book-embedding. Suppose we have a pair of edges \(uv\) and \(xy\) which cross, and \(\phi(uv) = \phi(xy)\). Suppose that \(u\) is the smallest vertex in the ordering \(\leq\). If \(u \in V(C)\), then the edge \(uv\) is assigned to its own page, meaning that it cannot cross \(xy\). So \(x = u\), but we can draw \(uv\) and \(uy\) on a single page. Thus they do not cross. Therefore we have that \(u, v, x, y\) are not in \(V(C)\), and as we preserve the original book-embedding, then these edges do not cross. Thus shown.
	\par
	We have that properties 1 and 2 are immediate from the definition of the book-embedding. For property 3, consider a clique \(J\) in \(G\). Then we must show the last vertex of \(J\) is rainbow. Suppose the last vertex of \(J\) is \(w\), and let \(u, v\) be earlier vertices. Since \(u\) and \(v\) necessarily are assigned different colours in the colouring, then \(\psi(uw) = (\rho(u), \psi_a(uw))\) and \(\psi(vw) = (\rho(v), \psi_a(vw))\). Therefore, the two edges are on different pages. Thus \(w\) is a rainbow vertex.
\end{proof}

\subsubsection{Full proof}
\begin{theorem}
	Suppose \(G\) has a tree-decomposition \((B_x: x \in V(T))\) with torsos \(G \langle B_x \rangle\) and adhesion at most \(\ell\). Order the vertices of \(T\) with a breath-first search, and relabel the vertices \(v_0, \ldots, v_k\) with respect to the bfs ordering. Define \(G_i := G \langle B_{v_i} \rangle \). Suppose that for all torsos \(G_i\), \(i \in [k]\), we have that \(\pn(G_i) \leq s\) and \(\chi(G_i) \leq c\). Then there is a book-embedding of \(G\) with pagenumber of at most \(cs + \ell\).
\end{theorem}
Recall that for a breadth-first search, we maintain the property that for all \(i\), \(T[v_0, \ldots, v_{i - 1}]\) is also a tree and \(v_i\) is adjacent to one of \(v_0, \ldots, v_{i}\).
\begin{proof}
	To prove the statement, we shall prove the stronger statement that there exists a book-embedding with the properties described with the lemma above using induction. In particular, we will have that the last vertex of any clique \(J\) is a rainbow vertex.

	Suppose \(k = 0\). Then \(G_0\) is a single graph with \(\pn(G) \leq s\). Choose \(C\) to be an arbitrary vertex \(v\) in \(G_0\). Then by the lemma above, there is a book-embedding with pagenumber at most \(cs + 1\) and every last vertex in a clique is a rainbow vertex.

	Suppose \(k = n\). Let \(C\) be the adhesion clique between \(G_n\) and the rest of \(G\), where \(G_n\) is a leaf of the tree \(T\). Denote the induced subgraph \(G[V(G) - V(G_n - C)]\) as \(G'\). Then let \((\leq_n, \psi_n)\) be the \(cs + \ell\)-pagenumber book-embedding of \(G_n\) that starts with \(V(C)\), and let \(\sigma_C\) be the ordering of \(V(C)\) by \(\leq_n\). Let \((\leq_{n-1}, \psi_{n-1})\) be the stack-embedding of \(G'\). By induction, this has a \(cs + \ell\)-page book-embedding with the properties described above.

	\paragraph{Construction of new book-embedding}
	Let \(w \in V(K)\) be the last vertex of \(K\) with respect to \(\leq_{n-1}\). Then insert \(V(G_n - C)\) between \(w\) and its successor in the order of \(\leq_{n}\). For the page assignment \(\psi\), we have that if \(uv \in E(G')\), then \(\psi(uv) = \psi_{n-1}(uv)\). For the remaining edges, we can permute the edge assignments of \(\psi_n\) such that for all \(u \in V(K)\), we have that \(\psi(E_u) = \psi_n(uw)\). We can do this as \(w\) is a rainbow vertex and the edges \(E_u\) are assigned to a unique page in \(\psi_n\). Finally, let \(\psi(uv) = \psi_n(uv)\) for the remainder of the edges. Denote the new ordering and assignment as \((\leq, \psi)\).
	\paragraph{Proof that this is a valid book-embedding}
	We claim that \((\leq , \psi)\) is a stack layout that satisfies the induction hypothesis. Suppose that \(\psi(uv) = \psi(xy)\). If \(uv, xy \in E(G')\), then by the induction hypothesis, they do not cross. Similarly, if \(uv, xy \in E(G_n)\), then they will not cross, by the above lemma. If \(uv\) is in \(E(G')\) and \(xy \in E(G_n)\), then they will go over each other or be sequential and therefore will not cross.
	Finally, if \(u, v, x, y \in C\), then by the induction hypothesis on \(G'\), they do not cross either. The final case is if \(uv \in E(G_{i + 1})\) and \(u \in V(C)\), \(v \notin V(C)\), \(xy \in E(G')\). If \(uv\) and \(xy\) cross, then we have that \(xy\) and \(uw\) will cross. But this will contradict the page-embedding of \(G'\).

	Let \(J\) be a clique in \(G\), and \(w\) is its final vertex. Then if \(J\) is in \(G'\), then \(w\) is a rainbow-vertex. Otherwise, the last vertex is contained in \(G_n\). By the above lemma, \(w\) must also be a rainbow vertex. Thus shown.
\end{proof}

\subsubsection{Bounds on pagenumber of \(\ell\) and \(c\)}

We have some bounds in terms of pagenumber on some of these constants, however these constants are not always tight. In particular, we can get better bounds for planar graphs.

\begin{lemma}[Bounded pagenumber implies bounded clique-number]
	If \(pn(G) \leq s\), then \(G\) does not contain any cliques on more than \(2s+1\) vertices.
\end{lemma}

\begin{proof}
	If \(G\) has a clique of size \(k\), then embedding the clique of size \(k\) and therefore \(G\) requires at least \(\lfloor \frac{k}{2} \rfloor\) pages, from \cref{thm:Pagenumber_Complete_Graph}. Therefore, if we can embed \(G\) in \(s\) pages, then the largest clique of \(G\) is at most \(2s + 1\). Therefore, \(\ell \leq 2s + 1\).
\end{proof}
As we have the bound of \(\chi(G) \leq 2 \pn(G) + 2\), from \cref{thm:Colouring_Bound}, we have a bound that does not depend on the chromatic number of \(G\).
\begin{corollary}[Bounded pagenumber of tree-decompositions]\label{corr:bded_pn_tree_decomp}
	Let \(G\) be a graph with a tree-decomposition \((B_x: x \in V(T))\) where each torso \(G \langle B_x \rangle\) has pagenumber \(\leq s\). Then from \cref{thm:clique_sum_pagenumber_bound}, \(G\) has pagenumber at most \(2s^2 + 4s + 1\).
\end{corollary}

\subsubsection{Bounds on pagenumbers of planar graphs}
This theorem also tells us that pagenumbers of planar graphs are bounded.

\begin{theorem}\label{thm:Planar Graph Hickingbotham Bound}
	Let \(G\) be a planar graph. Then \(\pn(G) \leq 11\).
\end{theorem}

We will use \cref{corr:planar_graphs_4_connected_cliqesums}. We will also use the fact that all planar graphs are \(4\)-colourable \cite{appelEveryPlanarMap1989} and the fact that all 4-connected planar graphs are Hamiltonian, from Tutte \cite{tutteTheoremPlanarGraphs1956}.

\begin{theorem}[Tutte\cite{tutteTheoremPlanarGraphs1956}]\label{thm:4-connected_planar_ham_cycle}
	All 4-connected planar graphs are Hamiltonian.
\end{theorem}

\begin{lemma}\label{lem:clique_sum_connected}
	All graphs have a tree-decomposition where every torso is a \(k\)-connected graph, or has at most $k-1$ vertices, with adhesion set size at most \(k-1\).
\end{lemma}
\begin{proof}
	If a graph \(G\) is not \(k\)-connected, we can find a set \(S\) of size at most \(k-1\) such that \(G - S\) is disconnected. Then we repeat this operation on the connected components of \(G - S\), until each component either has \(k-1\) vertices or is \(k\)-connected. So we can construct a tree-decomposition where every torso is \(k\)-connected and the adhesion size is at most \(k-1\).
\end{proof}

\begin{corollary}\label{corr:planar_graphs_4_connected_cliqesums}
	All planar graphs \(G\) have tree-decompositions with the torsos being \(4\)-connected planar graphs and adhesion at most \(3\).
\end{corollary}


\begin{proof}[Proof \cref{thm:Planar Graph Hickingbotham Bound}]
	Let \(G\) be planar. Then \(G\) has a tree-decomposition of adhesion size at most \(3\) with the torsos being \(4\)-connected, from \cref{lem:clique_sum_connected}. However, this implies that the torsos are Hamiltonian, from Tutte \cite{tutteTheoremPlanarGraphs1956}, thus the number of pages needed for each torso is \(2\). Therefore from \cref{thm:clique_sum_pagenumber_bound}, we have that the pagenumber is at most \(2 \times 4 + 3 = 11\).
\end{proof}
This upper bound is much worse than the tight upper bound found by Yannakakis \cite{yannakakisEmbeddingPlanarGraphs1989}. However, the proof given above is more general and will be used throughout to prove several bounds.
We will discuss the \(K_5\)-minor free case. If \(G\) is \(K_5\)-minor free, then we do not need to worry about the GMST to form an upper bound. We will use Wagner's theorem.
\begin{theorem}[Wagner's theorem\cite{wagnerUeberEigenschaftEbenen1937}]\label{thm:WagnersTheorem}
	If \(G\) is \(K_5\)-minor-free, then \(G\) has a tree-decomposition of adhesion $\leq 3$ where every torso is eithe a planar graph or the Wagner graph \(V_8\).
\end{theorem}
A description of the Wagner graph is in \cref{fig:wagner}. The edges are coloured such that the internal edges are on different pages. The spine edges (the edges that are on the outerface) are the ones which can go on any page.
\begin{figure}[h]
	\centering
	\begin{tikzpicture}
		\tikz \graph [nodes = {draw, circle}, clockwise, empty nodes] {
	subgraph C_n [n=8, red];
	1 --[red] 5;
	2 --[blue] 6;
	3 --[green] 7;
	4 --[yellow] 8;
};

	\end{tikzpicture}
	\caption{The Wagner graph. Notice how the clockwise circular ordering of the vertices of the Wagner graph needs 4 pages to embed the graph. }\label{fig:wagner}
\end{figure}
Equivalently, if \(G\) is \(K_5\) minor free, then \(G\) has a tree-decomposition \(\tree\) where \(\tree\) has adhesion at most 3, and every torso of \(\tree\) is either planar or the Wagner graph. We will use this to prove the theorem below.
\begin{theorem}
	If \(G\) is \(K_5\)-minor free, then \(G\) has pagenumber \(\leq 19\).
\end{theorem}

\begin{proof}
	Suppose \(G\) is \(K_5\)-minor free. Then by Wagner's theorem \cite{wagnerUeberEigenschaftEbenen1937}, \(G\) has a tree-decomposition of adhesion at most 3 where every torso is either a planar graph or the Wagner graph.
	We have that planar graphs are \(4\)-colourable and have pagenumber \(\leq 4\). We have that the Wagner graph is \(3\)-colourable and has pagenumber \(\leq 4\). Therefore, we have that if \(G\) is \(K_5\)-minor free, then \(G\) has pagenumber at most \(4 \times 4 + 3 = 19\).
\end{proof}
We refer to \cref{fig:wagner} for a description of a circular ordering of a Wagner graph and a book embedding.

	\subsection{Bounded treewidth and page number}\label{ssec:Bounded_Treewidth}
\begin{theorem}[Ganley + Heath\cite{ganleyPagenumberTrees2001}]\label{thm:bded_treewidth_bded_pagenumber}
	Every graph \(G\) with \(\tw(G) \leq k\) can be embedded on $k + 1$ pages.
\end{theorem}
In the original proof, they considered the case where \(G\) is a \(k\)-tree. We will bypass using \(k\)-trees and consider a tree-decomposition of \(G\) directly.

\begin{proof}
	Consider a tree-decomposition of \(G\) with bags $B_x$ and tree $T$. Perform a depth-first search on \(T\), starting at an arbitrary root node \(r\). Let the ordering of the book-embedding \(\sigma(v)\) of a vertex \(v\) in \(V(G)\) be determined by the first time \(x \in T\) appears, where \(v \in B_x\). Within each bag, if two vertices appears in the bag first, then they are ordered arbitrarily in the book-embedding. Now consider the subtree \(T_v\) induced by the bags \(B_x\) containing \(v\). We now consider colouring the subtrees \(T_v\) for all \(v \in G\) such that no overlapping subtrees have the same colour. Let \(H\) be the intersection graph of the subtrees, where \(V(H) = \lbrace T_v : v \in G \rbrace\) and \(T_u T_v \in E(H)\) if there exists a bag \(B_x\) such that \(u, b \in B_x\). We have that \(H\) is perfect, and thus \(\chi(H) = \omega(H)\). Then as \(\tw(G) \leq k + 1\), then the size of a clique in \(H\) is at most \(k + 1\). Thus \(H\) is \(k + 1\)-colourable.
	\paragraph{}
	We now use this to assign the edges of \(G\) a page. Let \(c(T_v)\) be the colour assigned to \(T_v\). Colour each edge \(uv \in E(G)\) as follows:
	\begin{equation}
		c(uv) =
		\begin{cases}
			c(T_u) & \text{ if } \sigma(u) \leq \sigma(v), \\
			c(T_v) & \text{ if } \sigma(v) \leq \sigma(u)
		\end{cases}
	\end{equation}
	Then we claim that this is a proper book-embedding of \(G\). Suppose we have that edges \(uv\), \(xy\) cross, so \(\sigma(u) \leq \sigma(x) \leq \sigma(v) \leq \sigma(y)\). However, this implies that there exists a bag \(B\) such that \(u, x, v \in B\), as we have that \(uv\) is an edge in \(B\) and we do a depth-first search to establish the ordering. So \(u, x, v\) they are in the same bags. However, this implies that the trees \(T_u\) and \(T_x\) intersect, meaning that \(c(uv) \neq c(xy)\). Finally, the number of pages used is \(\chi(H) \leq k + 1\), so \(\pn(G) \leq k + 1\). Thus shown.
\end{proof}

We have a simpler proof if the tree is a path. We go from one end of the path to the other, and add vertices to the book-embedding in the order of the first time they appear. Then we colour each vertex such that in each bag, no two vertices are assigned the same colour. We can do this as this is the same intersection graph as above. The rest of the proof follows.

\subsection{Planar graphs}\label{ssec:Planar_Graphs}
\begin{theorem}[Yannakakis \cite{yannakakisEmbeddingPlanarGraphs1989}]\label{thm:4Pages_Planar}
	Planar graphs can be embedded on at most four pages.
\end{theorem}
We have shown above a proof that the number of pages necessary to embed a planar graph is bounded. However, the proof given by Yannakakis is tight, as there exist planar graphs that need four pages \cite{yannakakisPlanarGraphsThat2020} \cite{kaufmannFourPagesAre2020}. We need the fact that the number of pages to embed a planar graph is bounded for proving that graphs embedded on a surface of bounded genus has bounded pagenumber.

\section{Graphs embedded on a surface of bounded genus}\label{sec:pagenumber_bounded_genus}

\begin{theorem}[Heath and Istrail\cite{heathPagenumberGenusGraphs1992}]\label{thm:Genus_pagenumber_bound}
	Let \(g\) be the genus of a graph \(G\). For all graphs \(G\), \(\pn(G) \leq O(g)\).
\end{theorem}
Note that this bound extends the one found by Yannakakis \cite{yannakakisEmbeddingPlanarGraphs1989} to graph families of bounded genus.
The best known bound is \(O(\sqrt{g})\), found by Malitz\cite{malitzGenusGraphsHave1994}.

It was shown by Heath and Istrail that the family of graphs of bounded genus have bounded pagenumber.
We refer to the ``layout'' of the graph as the book-embedding of the graph and ``embedding'' as the surface-embedding. We refer to orientable surfaces as genus \(g\) as a sphere with \(g\) handles, and a nonorientable surface of genus \(g\) as a sphere with \(g\) cross-caps. We define the orientable genus of a graph \(G\), denoted \(\gamma(G)\), as the minimum orientable surface genus that \(G\) can be embedded on. The nonorientable genus of a graph \(G\), denoted \(\tilde{\gamma}(G)\), is the minimum nonorientable genus surface that \(G\) can be embedded on. Mohar\cite{moharOrientableGenusGraphs1998} claims that \(\tilde{\gamma}(G) \leq 2 \gamma(G) + 1\) for all graphs, meaning that if the orientable genus is bounded, then the non-orientable genus is bounded. Note that, Auslander et al.\cite{auslanderImbeddingGraphsManifolds1963} showed that there exists graphs which are embeddable on the projective plane who has arbitrarily large orientable genus.
\paragraph{Proof}
We say that the embedding is \(2\)-cell if every face is homeomorphic to an open disc in \(\mathbb{R}^2\). Any embedding of \(G\) onto an orientable surface is a 2-cell embedding, but this does not hold for nonorientable surfaces, but we assume there exists a \(2\)-cell embedding.
Heath and Istrail rely on decomposing the graph \(G\) of genus \(\gamma(G)\) into a planar spanning subgraph \(G_p\) of \(G\) such that:
\begin{enumerate}
	\item The edges in \(E(G) - E(G_p)\) attach to the boundary vertices of \(V(G_p)\).
	\item Adding an edge from \(E(G) - E(G_p)\) to \(G_p\) breaks the above condition.
\end{enumerate}
To talk about graphs embedded in surfaces, we assign to each face a cyclic permutation \(\sigma_v\) which represents the sequence of vertices encountered when traversing the boundary of a face in counterclockwise order.

This is enough to represent any graph in an orientable surface, but not enough for a non-orientable surface. We have to attach on an orientation to each edge, where each edge is either orientation-preserving or orientation-reversing.

We have that a planar-nonplanar decomposition of \(G\) is a triple \((R, G_P, E_N)\) where \(R\) is a rotation of \(G\) representing the surface embedding on the surface \(S\), \(G\) is a spanning planar graph, and \(E_N = E - E(G_P)\).
This satisfies a list of properties:
\begin{enumerate}
	\item The subrotation induces a planar embedding of \(G_p\), so we can arrange \(G\) on the surface \(S\) such that the embedding of \(G_p\) is planar.
	\item For each \(vw \in E_N\), we have that \(v\) and \(w\) live on the outerface \(F_0\).
	\item \(E(G_P)\) is maximal, so we cannot add edges from \(E_N\) to \(G_P\) without breaking properties 1 and 2.
\end{enumerate}

\subsubsection{Decomposing graphs on surfaces}\label{sssec:Planar_nonplanar_decomp}
We first have to know that the planar-nonplanar decomposition exists.

Suppose \(G\) is embedded on an surface \(\Sigma\). Then we wish to triangulate \(G\) to form \(G_T\). We choose a single triangle as the starting point and we add traces to the planar part incrementally. At each step, we set \(G_P\) to be the current planar part and \(E_N\) as the edges that are outside the planar part. There are two types of edges in \(E_N\): edges which have both endpoints in \(V(G_P)\), so cannot become edges of \(G_P\), and edges that have either one or no endpoints in \(V(G_P)\).

We want to maintain the property that if \(v \in G_P\), and edge \(vw \in E_n\), then \(v\) is a vertex on the boundary of \(G_p\).
\paragraph{Adding vertices to biconnected block}
For a current boundary of the outerface of \(G_P\), if \(v_i \rightarrow v_j \rightarrow v_k\) is trace with no edge of \(E_N\) incident to \(v_j\), then \(v_i v_k \in E(G_T)\) is called a safe edge. If \(v_i \rightarrow v_j\) is on the boundary of \(G_P\), and \(v_k \notin V(G_P)\), and \(v_i,v_j,v_k\) is the boundary of a face, then \(v_k\) is a safe vertex and we can add it to \(G_P\).
\paragraph{Creating new biconnected block}
If no \(v_k\) exists, then we find a \(w'\) which is the newest vertex in \(V(G_P)\) adjacent to a vertex \(w\) not in \(V(G_P)\). We then add the vertex \(w\) and the edge \(w w'\) to \(G_P\). Then we add all safe edges. This is so that every edge in \(E_p\) maintains the property that both endpoints are on the boundary.

We claim that after repeating this operation, then we have that every edge in \(E_N\) (edges not in \(G_P\)) satisfy the two properties above. If we have that an edge \(vw\) has \(v\) added, then we should be able to add \(w\) as a safe vertex or biconnected block. If an edge \(vw\) has neither \(v\) or \(w\) added to \(G_P\), then the algorithm has not finished yet. By connectivity, we can add \(v\) and \(w\) at some stage and therefore go to part 1. This has the corollary that every vertex is in \(G_P\).

Now we have that \(E_N\) has every edge which cannot be added to \(G_P\) without crossing over another edge, and that \(G_P\) is maximal. Then we have that all edges in \(E_N\) satisfy the conditions lised above.
\todo{Add pictures! this proof needs lots of pictures}

\subsubsection{Level sets and cycles}
On a planar graph \(G\), we want to separate out vertices depending on how far away they are from the outerface. We fix a single outerface \(F_0\) and define the first level set \(V_0\) as the vertices adjacent to \(F_0\). We then define the \(i\)-th level set, \(V_i\) inductively. Consider the induced graph on \(V(G) - \cup_{k = 0}^{i-1} V_k\). Then we define the vertices adjacent to \(F_0\) in this induced graph, where we expand \(F_0\) to include the vertices. This partitions \(V(G)\).

We then define \(C_0\) to be the edges adjacent to \(F_0\) in this decomposition. Then we want \(C_i\) to be the edges adjacent to \(F_0\) in this decomposition. We define the chord edges \(K_i\) to be the edges between vertices in \(V_i\) that are not edges in \(C_i\). Finally, we define the edges between faces, \(E_i\) as the edges that are between vertices on level \(V_i\) and \(V_{i + 1}\).

\begin{lemma}
	For all faces \(F\) in \(G\), the vertices around \(F\) are either all in one \(C_i\) or they are in \(C_i\) and \(C_{i + 1}\) for some \(i\).
\end{lemma}

\begin{proof}
	Let \(i\) be the smallest value such that \(v \in V_i\) is on the boundary of \(F\). Now we have that \(G[V(G) - \cup_{j = 1}^{i} V_i]\) will also remove \(v\). However, this removes all the edges next to \(v\), therefore all vertices that are on the boundary of \(F\) will either be in \(V_i\) or \(V_{i + 1}\).
\end{proof}
We refer to the faces that have vertices in only \(V_i\) as chordal and the faces that are between \(V_i\) and \(V_{i + 1}\) as non-chordal.

We define a weak triangulation of \(G\) to be a triangulation \(G'\) such that all faces except for the outerface is a triangulation.
\begin{lemma}
	There exists a weak triangulation of \(G\), \(G'\) which preserves the level sets \(V_i\) and edge sets \(E_i\), \(C_i\), \(K_i\) for all \(i\).
\end{lemma}

\begin{proof}
	If \(F\) is a chordal face of \(G\), then any triangulation maintains the property. If \(F\) is nonchordal and the boundary has edges in \(V_i\) and \(V_{i + 1}\), then add edges that are only between vertices in \(V_i\) and \(V_{i + 1}\). This will suffice to build a new triangulated graph \(G'\) where all vertices and edges are in the correct place.
\end{proof}

\subsubsection{Classifying nonplanar edges according to homotopy class}

We can then form a directed cycle \(C_0\) induced by \(F_0\). Each vertex on the boundary of \(F_0\) appears at least once, and twice if it is an \textit{articulation point}. Each edge on the boundary of \(F_0\) is encountered at least once on this cycle. Heath and Istrail refer to a directed subpath of the cycle \(C_0\) as a trace, so trace \(T = v_1 \rightarrow v_2 \rightarrow \cdots \rightarrow v_t\). The inverse trace is \(T^{-1} = v_t \rightarrow v_{t-1} \rightarrow \cdots \rightarrow v_1\). We now wish to partition \(E_N\) into equivalence classes. Suppose that \(u_1v_1, u_2v_2 \in E_N\) are part of the boundary of the same face \(F\) on the embedding of \(G\). Then \(u_1v_1\) and \(u_2v_2\) are \textit{homotopic} (with respect to \(F\)) if:
\begin{enumerate}
	\item \(u_1v_1\) and \(u_2v_2\) are the only edges of \(E_N\) on the boundary of \(F\)
	\item There exist traces \(T_u = u_1 \rightarrow \cdots \rightarrow u_2\) and \(T_v = v_1 \rightarrow \cdots \rightarrow v_2\) such that \(T_u\) and \(T_v\) are on the boundary of \(F\).
\end{enumerate}
We may think of \(G_n\) as living on a locally flat part of \(S\) and the homotopy class \(u_1v_1\) and \(u_2 v_2\) living on a handle (alternatively, passing through a crosscap such that they bound a face). Then if we take \(G_n\) to a point, there exists a \textit{homotopy} (in the topological sense) from \(u_1v_1\) to \(u_2v_2\). These form equivalence classes of the nonplanar edges.

\begin{lemma}
	If \(C\) is a homotopy class of edges \(u_1 v_1, \ldots, u_k v_k\) with a natural order, then we can build traces \(T_1\) and \(T_2\) by building the trace from \(u_1\) to \(u_k\) passing through all \(u_i\), and \(v_1\) to \(v_k\) passing through all \(v_i\).
\end{lemma}
We refer to a homotopy class as orientable if \(T_1\) and \(T_2\) go in opposite directions, and non-orientable if \(T_1\) and \(T_2\) go in the same direction.

\begin{lemma}
	We have that if \(G\) is embedded in an orientable surface, then every homotopy class is orientable.
\end{lemma}
\begin{proof}[Sketch]
	We have that if a homotopy class is non-orientable, then on the handle the class sits on, the edges must cross. However, we have the graph is embedded on the surface, therefore this cannot happen. Thus shown.
\end{proof}

\begin{lemma}
	If \(G\) is \(2\)-cell embedded on a surface of Euler genus \(g\), then any planar-nonplanar decomposition has at most \(3g-3\) homotopy classes.
\end{lemma}
\begin{proof}
	Decompose \(G\) to a \((R, G_P, E_N)\) decomposition of \(G\). Suppose \(E_N \neq \emptyset\). Then identify \(G_P\) to a single point, and identify each homotopy class to a single edge. Then draw a circle around the point \(G_P\), and place vertices where the circle intersects all edges. Then delete the vertex \(G_P\), and call the new graph \(H\). We have that \(n = |V(H)|\), \(m = |E(H)|\), \(h\) is the number of homotopy classes, and \(f\) is the number of faces. We have that \(n - m + f = 2 - g\). Since \(H\) is cubic as every vertex has two edges on the circle and one on the homotopy class, then \(3n = 2m\) by the handshaking lemma. Since there is only one nonplanar edge for each homotopy class, \(n = 2h\). The interior face of \(H\) has \(v\) incident edges, and the remaining \(f-1\) faces have at least 3 incident edges each, as we can identify the two homotopy classes bordering a face with four edges together. Therefore, we have that \(3(f-1) + n \leq 2m\), by double counting faces and edges. Thus, we have that
	\begin{align*}
		3n  & \geq 6(f - 1) + n         \\
		2n  & \geq 6f + 6               \\
		4h  & \geq 5 f - 6              \\
		4h  & \geq 5(2 - g + m - n) - 6 \\
		4h  & \geq 6 - 6g + 3n          \\
		4h  & \geq 6 - 6g + 6h          \\
		-2h & \geq 6 - 6g               \\
		h   & \leq 3g - 3
	\end{align*}
	\(3g - 3 \geq h\) by manipulating the inequalities.
\end{proof}

\subsubsection{Proving graphs with bounded number of homotopy classes have bounded pagenumber}\label{sssec:bounded_pagenumber_homotopy}
\begin{lemma}\label{lem:planar_nonplanar_orientable}
	Suppose \(G\) has a planar-nonplanar decomposition \((R, G_P, E_N)\) on an orientable surface \(\Sigma\). Then \(G\) can be embedded on at most \(18g - 5\) pages.
\end{lemma}
\begin{proof}
	We use Yannikakis' proof in \cref{ssec:Planar_Graphs} to lay out the nonplanar spanning subgraph \(G_P\) on four pages, maintaining the cyclic order of vertices. Then we can combine each blocks to form a 4 page layout of the graph. For each homotopy class in \(E_P\), we allocate three pages. One page is for vertices in the same block, and the other two pages are used for edges between blocks, the biconnected components of \(G\). We need two as we could have some which span blocks in a way that forces them to be on different pages. Therefore, we need at most \(4 + 3(6g - 3) = 18g-5\) pages if \(G\) has a planar-nonplanar decomposition.
\end{proof}

\begin{lemma}\label{lem:planar_nonplanar_nonorientable}
	Suppose \(G\) has a planar-nonplanar decomposition \((R, G_P, E_N)\) on a non-orientable surface \(\Sigma\). Then \(G\) can be embedded on at most \(9g - 1\) pages.
\end{lemma}
\begin{proof}[Proof sketch]
	\todo{Flesh out details completely}
	We want to add edges in a controlled way so that the traces that are reversed become non-reversed. This is done by adding edges between vertices so that we can invert the ordering on the circle such that we have that the vertices in one homotopy class have a non-crossing page embedding. However, an issue is chords that go between traces that are inverted. We go around this problem by removing chords and adding them to a separate page where there are finitely many pages wrt to the genus of the surface. Let \(\mathcal{C}_{i,j}\) be a chord class when it is the set of chords that go between traces \(T_i\) and \(T_j\), where \(T_i\) and \(T_j\) both go clockwise or counterclockwise. Note that the number of chord classes \(\mathcal{C}_{i,j}\) is bounded by the genus of the graph, and we can embed the chord classes that share a trace onto a single page. As there is a bounded number of chord classes, it must hold that the number of pages is finite.

	Consider the scenario where the graph $G_p$ is a cycle with a cyclic ordering. Then every trace on the boundary is a path on the cycle. Recall that a trace $T_1$ and $T_2$ are paired if there is a nonplanar edge between the two traces. For every two pairs of traces that are non-orientable, we label one to be non-reversed and one to be reversed, with arbitrary allocation. We then merge all the consecutive reversed and non-reversed traces together. Then we add an edge around each reversed trace, called bypassing edges. We then use Heath's algorithm to find an edge colouring with 7 colours.

	Now consider the case where $G_p$ is outerplanar. We cannot do the unreversing trick anymore, because of edges inside of the graph. However, we can form a path with these edges and colour them with a finite number of colours as well.
	\todo{finish this case!}
	We bundle chords together and the number of chords that exist is $O(g)$ as there are at most $g$ reversed-unreversed trace changes on the boundary of the surface. As no chords cross, then we have that there are $O(g)$ chords. 

	Finally, consider the case where $G_p$ has a block-cut tree. We still have the same chords from before, and now we have that chords can be used to draw a path that goes back and forth between traces.
	\todo{finish this proof!}

	We first merge traces, and then build a boundary between two vertices. Then we have two subgraphs and two traces. 
\end{proof}



\newcommand{\gpk}{12gpk + 12gp + 18g + 12kp + 12p + 11}
\chapter{Towards a proof of \cref{conj:bded_had_pn}}\label{chap:Proving_The_Theorem}
Our aim is to prove the following conjecture. Recall the definition of $\mathcal{G}(g, p, k, a)$ from \cref{thm:gmst}. 
\begin{conjecture}\label{conj:gmst_conjecture_pagenumber}
	Every graph $G$ in $\mathcal{G}(g, p, k, a)$ can be embedded on \(f(g, p, k, a)\) pages.
\end{conjecture}
Our aim is to show that every graph of Euler genus \(g\) containing \(p\) vortices of width \(k\) can be embedded on \(f(g, p, k)\) pages. If $G$ is a graph with a tree-decomposition where every torso is $(g, p, k)$-almost embeddable, then from \cref{thm:clique_sum_pagenumber_bound}, $G$ has bounded pagenumber. If there exists a set $A \subseteq V(G)$ such that $G - A$ has the tree-decomposition above, then from \cref{thm:apex_vertices_pagenumber}, the pagenumber of $G$ is bounded.

\section{Apex vertices}
In this section, we prove that apex vertices can be added with a bounded increase to the number of pages. Typically, apex vertices are the most problematic ingredient when applying the Graph Minor Structure theorem, but in this case they are a trivial addition.
\begin{theorem}\label{thm:apex_vertices_pagenumber}
	If \(G\) is a graph with vertex partition \((A, B)\) such that \(G' = G[B]\) is embeddable on $s$ pages, then \(G\) is embeddable on \(s + |A| \) pages.
\end{theorem}
\begin{proof}
	Let \(G'\) have a book-embedding \((<, \rho)\) on $s$ pages. Place the vertices of \(A\) at the start of \((<)\), with order $u_1, u_2, \ldots, u_a$. For every edge \(u_i v \), \(u_i \in A\), \(v \in G'\), colour \(\rho(u_i v) = i\), and for any edge \(u_i u_j\) where $u_i, u_j \in A$, $i < j$, colour $u_i u_j$ with colour $i$. Then this is a proper book-embedding. If two edges $uv$, $xy$ cross, and if $uv, xy \in E(G')$, then $uv, xy$ are given different colours as $(< \rho)$ is a book-embedding of $G'$. If $u$ or $x$ are in $A$, then $uv, xy$ have different colours because their leftmost vertices are different. The number of pages necessary is $s + |A|$. 
\end{proof}

Therefore, a fixed number of apex vertices do not affect the pagenumber.

\section{Monochromatic paths}

We wish to find a book-embedding of an almost-embeddable graph on some orientable surface. To do so, we need to introduce some new terminology to work with vortices on surfaces. 
Recall the definition of $(g, p, k, a)$-almost-embeddable from \cref{ssec:Robertson_Seymour_Graph_Structure}. A graph $G$ is $(g, p, k)$-almost-embeddable if $G$ is $(g, p, k, 0)$-almost-embeddable, so $G$ has no apex set.
What we plan to show is this:
\begin{theorem}\label{thm:bounded_almost_embeddable}
	Suppose $G$ is $(g, p, k)$-almost embeddable on an orientable surface. Then $G$ can be embedded on \(f(g, p, k)\) pages for some function $f$.
\end{theorem}

The most problematic section is dealing with vortices on surfaces.
To work with vortices, consider how an ordering affects the face that the vortex is sitting on. Then see what happens when the vortex is added onto the face. 

Let \(F\) be a face on \(G\). Let \( (<, \varphi) \) be a book-embedding of \(G\). A \textit{monochromatic path} $P$ on the boundary of $F$ is a maximal path where every edge is given the same colour from $\varphi$. $F$ being a preserved face implies that $F$ has a single monochromatic path, but not the other way around. 

\begin{figure}[h!]
	\centering
	\includesvg[width = 0.8\textwidth]{figures/monochromatic_paths.svg}
	\caption[Monochromatic paths]{An example of a monochromatic path on a face. Above is a face embedded on $\mathbb{R}^2$. The circular book-embedding has edges coloured red and blue. This face has four monochromatic paths. Below is a circular ordering of the vertices on the boundary of the face. Every edge coloured red is embedded on a single page.}
\end{figure}


The list below are the steps taken to prove \cref{thm:bounded_almost_embeddable}.
\begin{enumerate}
	\item Suppose $G$ is a \(4\)-connected planar graph. There is a book-embedding where every face is monochromatic, by \cref{thm:4-connected_planar_ham_cycle}. 
	\item Suppose \(G\) is a connected planar graph. Faces are not preserved, but a fixed number of vertices are moved around on every face. Additional pages are necessary to embed $G$, but the number of monochromatic paths is bounded. 
	\item Suppose \(G\) is $2$-cell embedded on an orientable surface. \textcite{heathPagenumberGenusGraphs1992} gives a planar-nonplanar decomposition of $G$. Then apply the previous steps to the spanning planar subgraph and add vortices.
\end{enumerate}

Preserved faces allow us to embed vortices with a bounded number of pages. 

\begin{lemma}[Vortex on preserved faces]\label{lem:preserved_faces_pagenumber}
	Suppose a graph \(G = G_0 \cup G_1\), where \(G_0\) is embedded on a surface $\Sigma$ and \(G_1\) is a vortex on a face $F$ of $G_0$ with depth \(k\). Suppose $(<, \varphi)$ is a $p$-page book-embedding of \(G_0\) which preserves \(F\). Then $G$ can be embedded on \(p + k + 1\) pages. Furthermore, the new ordering restricted to $G_0$ is $(<, \varphi)$.
\end{lemma}

\begin{proof}
	We repeat a similar argument to \cref{thm:bded_treewidth_bded_pagenumber}. Let \(B_1, \ldots, B_i\) be a path-decomposition of \(G_1\). Let \(\sigma(v)\) be the first time \(v\) appears in the path-decomposition. Colour the edges of \(G_1\) as such. If \(uv \in E(G_1)\), then:
	\begin{equation}
		c(uv) =
		\begin{cases}
			c(T_u) & \text{ if } \sigma(u) \leq \sigma(v), \\
			c(T_v) & \text{ if } \sigma(v) \leq \sigma(u).
		\end{cases}
	\end{equation}
	This is a book-embedding of \(G_1\) with \(k+1\) colours. Since the intersection graphs of every graph is perfect, we can colour with $k + 1$ colours. Furthermore, suppose that edges \(uv\), \(xy\) cross, so \(\sigma(u) \leq \sigma(x) \leq \sigma(v) \leq \sigma(y)\). However, this implies that $u,x,v$ are in some bag $B$. \(uv\) is an edge in \(B\), and we do a depth-first search to establish the ordering of $\sigma(u)$. So \(u, x, v\) are in the same bags. However, this implies that the trees \(T_u\) and \(T_x\) intersect, meaning that \(c(uv) \neq c(xy)\). Therefore, all crossing edges are assigned different colours. 

	To add this book-embedding to \(G_0\), add the vertices that appear first in \(B_i\) after the associated vertex \(v_i\) in \(G_0\) such that \(v_i\) is on the face \(F\) and \(v_i \in B_i\). This is a book-embedding of \(G\) requiring at most \(\pn(G_0) + k + 1\) colours. An illustration of such a book-embedding is in \cref{fig:preserved_face}.
\end{proof}

\begin{figure}[h!]
	\centering
	\includesvg[pretex=\tiny, width = 0.8\textwidth]{figures/bookembedding_preserved_face.svg}
	\caption[Book-embedding of a preserved face]{A book-embedding of a preserved face  with a vortex on the face. The green vertices are the boundary of the face and the other coloured vertices are distinct vertices in each bag. The bag is the red circle. This graph can be embedded on $3$ pages as the depth of the vortex attached is 2. }\label{fig:preserved_face}
\end{figure}

\begin{lemma}\label{lem:vortices_mono_paths}
	Suppose \(G\) is a graph where $G = G_0 \cup G_1$, where \(G_0\) is embedded on a surface \(\Sigma \) of genus \(g\). Let \(F\) be a face on \(G_0\). Let \(v_1, v_2, \ldots, v_k\) be the vertices bordering \(F\). Let \(D\) be a \(G\)-clean disk on \(F\). Now suppose \(G_1\) is a vortex of depth $k$ on \(D\) with a path-decomposition \((B_0, \ldots, B_l)\) and \(G_0\) has a book-embedding \((<, \varphi)\) on $s$ pages. Suppose there are at most \(m\) monochromatic paths for every face on $G_0$. Then \(G\) can be embedded on \(s + m(k+1)\) pages.
\end{lemma}

\cref{lem:vortices_mono_paths} implies \cref{corr:vortices_paths_pn}.

\begin{corollary}\label{corr:vortices_paths_pn}
	Suppose $G$ is a graph where $G = G_0 \cup G_1 \cup \ldots \cup G_P$, $G_0$ is embedded on a surface $\Sigma$ of genus $g$, and $G_1, \ldots ,G_P$ are vortices on $G_0$. Suppose further that $G_0$ has $m$ monochromatic paths on each of its faces on $\Sigma$. Then $G$ can be embedded on $s + pm(k+1)$ pages.
\end{corollary}

This is proven by adding $G_1, \ldots, G_P$ to $G_0$ one at a time and applying \cref{lem:vortices_mono_paths} each time. To prove \cref{lem:vortices_mono_paths}, we prove \cref{lem:one_page_decomposition}.
\begin{lemma}\label{lem:one_page_decomposition}
	Let \((B_1, \ldots, B_n)\) be a path-decomposition of \(G\) with path-width \(k\). Let \(x_1, \ldots, x_n\) be vertices in \(G\) such that \(x_i \in B_i\) for all \(i\), and suppose \(P\) is an induced path \((x_1, x_2, \ldots, x_n)\) in \(G\). Then for every one-page embedding of \(P\), \(G\) has a \((k + 1)\)-page embedding where $P$ is preserved in the book-embedding.
\end{lemma}
\begin{proof}
	The proof given is similar to the one in \cref{lem:preserved_faces_pagenumber}. 
	Suppose \(G\) has the structure as described in \cref{lem:vortices_mono_paths}. For each vertex \(v\) in \(G\), let \(\sigma(v)\) be the index of the first bag \(v\) appears in. Then in the book-embedding of \(G\), place all bags of \(v_i\) after \(x_i\) in the book-embedding, following the orientation of the path with the book-embedding. Colour the edges like so:
	\begin{equation}
		c(uv) =
		\begin{cases}
			c(T_u) & \text{ if } \sigma(u) \leq \sigma(v), \\
			c(T_v) & \text{ if } \sigma(v) \leq \sigma(u).
		\end{cases}
	\end{equation}

	Then if two edges cross in the book-embedding, then they have different colours. If two edges cross, then that implies that in a book-embedding of the path \(P\) with \((B_1, \ldots, B_n)\) added in like \cref{lem:preserved_faces_pagenumber}, then they will cross as well. Two examples are given in \cref{fig:preserving_pages}.
\end{proof}

\begin{figure}[h!]
	\centering
	\includesvg[pretex=\tiny, width = 0.8\textwidth]{figures/one_page_embedding.svg}
	\caption[One-page decomposition]{Description of \cref{lem:one_page_decomposition}. \(x_1, \ldots, x_n\) are the vertices with a path that is a single book-embedding and \(B_1, \ldots, B_n\) are the bags of the embedding. Notice that there are two different ways that the \(n + 1\)-th bag can end up, but both ways still maintain the property that this is a book-embedding. This diagram is a circular ordering of \(x_1, \ldots, x_n\).}\label{fig:preserving_pages}
\end{figure}

We now prove \cref{lem:vortices_mono_paths}.
\begin{proof}
	Use the path-decomposition on \(G'\) as the set \((B_1, \ldots , B_n)\) in proving \cref{lem:vortices_mono_paths}. Then apply \cref{lem:one_page_decomposition} for the monochromatic \(v_i\) to each of the monochromatic paths. From the construction of the vortices in \cref{lem:preserved_faces_pagenumber}, add on the faces in the exact order. Then the monochromatic paths are preserved in the ordering, and in fact from \cref{lem:one_page_decomposition}, they are bounded.
\end{proof}
\begin{lemma}\label{lem:Hamiltonian_preserved_faces}
	Let \(G\) be a Hamiltonian planar graph with Hamiltonian cycle $C$ Then there exists a book-embedding $(\leq, \varphi)$ on two pages where every face of $G$ is preserved, and the circular ordering of $\leq$ is $C$.
\end{lemma}

\begin{proof}
	Let \(<\) be the circular ordering of these vertices by traversing \(C\). Now as \(G\) is planar, \(C\) splits the surface into an interior region and an exterior region, by the Jordan curve theorem. So every edge in $G - C$ is inside either the interior or exterior of \(C\). Colour every edge in $C$ and on the interior of $\mathbb{R}^2 - C$ red and colour the other edges blue. This is a book-embedding as both regions can be embedded on a single page with the same cyclic ordering. Furthermore, every face is preserved. Because $\mathbb{R}$ is orientable, affix an orientation to every face \(F\) on $G$ such that the order of the vertices in the orientation is the same order as the orientation in \(D\).
	\begin{figure}[h!]
		\centering
		\includesvg[pretex=\tiny, width=0.3\linewidth]{figures/hamiltonian-planar}
		\caption[Hamiltonian planar graph]{This is a Hamiltonian planar graph with a book-embedding of 2. The linear ordering is $v_1, v_2, \ldots,  v_8$. Let $C$ be the Hamiltonian cycle. $C$ splits the plane into two regions. Then edges in the interior region and on $C$ are coloured red, and the remaining edges are blue. Now choosing any face and restricting the linear ordering to be around the face preserves each face. }\label{fig:hamiltonian_planar}
	\end{figure}
\end{proof}
From \cref{lem:Hamiltonian_preserved_faces}, if \(G\) is a Hamiltonian planar graph, then the vertex ordering of the Hamiltonian cycle \((\leq)\) preserves all faces on \(G\). As a consequence of \cref{thm:4-connected_planar_ham_cycle}, every 4-connected planar graph has a circular ordering which preserves every face.

Then this can be extended to every connected planar graph, adding a constant number of monochromatic paths, as shown by \cref{thm:planar_graph__be_mono_paths}.

\begin{lemma}\label{lemma:decomposition_faces}
	Let $G$ be a planar graph embedded on $\mathbb{R}^2$, and let $F$ be a face. The vertex set of $F$ has a tree-decomposition $(T, (B_x)_x)$ of adhesion 1 where each torso is a cycle.
\end{lemma}

\begin{proof}
	Suppose the facial walk of $F$ is a cycle $C$. Then there are no cut-vertices, so $C$ is a valid cycle.

	Now suppose $F$ has a cut vertex.Let $W$ be a facial walk of $F$, and suppose the sum times duplicate vertices appear is $n$. Suppose for all faces with walks with $n-1$ duplicate vertices, the lemma holds. Take a cut-vertex $v$ and look at the first and last time $v$ appears in $W$, call these positions $a, b$. Then let $W_1$ be the subwalk $W[a, b-1]$ and let $W_2$ be the subwalk of $W - W[a, b-1]$, removing $[a, b-1]$. Now both $W_1$ and $W_2$ bound subfaces of $F$. Then applying the inductive hypothesis, $W_1$ and $W_2$ have the decomposition above. Then there is a bag containing $v$ in the tree-decomposition of $W_1$ and $W_2$, so join the two bags. As $W_1$ and $W_2$ only have $v$ in common, then $w_1, w_2$ contain $w$. 
\end{proof}

We want to show that there exists a book-embedding of every planar graph where every face has a bounded number of monochromatic paths. We were able to prove it in the special case where there is no cut-set of size 3.

\begin{theorem}\label{thm:planar_graph__be_mono_paths}
	Let \( G \) be a planar graph embedded in $\mathbb{R}^2$, with $k$ sets of graphs with adhesion $3$. Then $G$ can be embedded on 11 pages where every face has at most $8k$ monochromatic paths. 
\end{theorem}

$ ((T,r), (B_x)_x)$ is a tree-decomposition with $r$ as the root node. 

\begin{proof}
	Use \cref{lem:planar_graphs_4_connected_cliqesums} to build a tree-decomposition $((T,r), (B_x)_x)$ of $G$ with adhesion $3$. Every torso is either a $4$-connected planar graph or $K_t$ where $t \leq 4$.
	
	Every torso $G \langle B_x \rangle$ has a tree-decomposition where every face is preserved, from \cref{lem:Hamiltonian_preserved_faces}. Then apply \cref{thm:clique_sum_pagenumber_bound} to $((T,r), (B_x)_x)$ to embed $G$ in a book. The number of pages used is $2 \cdot 4 + 3 = 11$. 

	If $v$ is a cut-vertex of $G$, then $v$ is in an adhesion set of torsos $G \langle B_x \rangle$ and $G \langle B_y \rangle$. Assume that $B_y$ is the parent of $B_x$. Then from \cref{thm:clique_sum_pagenumber_bound}, the ordering does not change, so $v$ is not moved anywhere. Then any face $F$ where every vertex on the boundary of $F$ is in $B_x$ is preserved. Since the outerface of $B_x$ is preserved, the colour of the outerface can be changed so that it matches the colour of the face that $B_x$ is a part of. 

	Now suppose $S$ is an adhesion set between bags $B_x$ and the parent $B_y$. Suppose $ S = \{a,b\}$. Let $F$ be a face in $B_x$ and let $F_v$ be the set of vertices that touch $F$. Assume that $F_v \subseteq B_x$. If $F_v \cap S$ is nonempty, then $S$ vertices are moved to the start of $F$. 
	
	No vertex in $S$ is a cut vertex, because that would imply from \cref{lem:planar_graphs_4_connected_cliqesums} that $|S| = 1$. Let $(<_x, \psi_x)$ be a two-page book-embedding of $G \langle B_x \rangle$, from \cref{lem:Hamiltonian_preserved_faces}. Now $F$ has a single monochromatic face.
	From \cref{thm:4-connected_planar_ham_cycle}, there is a Hamiltonian cycle in $B_x$ with edge $ab$ on the cycle. From \cref{lem:Hamiltonian_preserved_faces}, this implies that $ab$ is consecutive on the cycle. From \cref{thm:clique_sum_pagenumber_bound}, this cycle is the same in $B_x$. If $ab$ is the boundary on a face $F$, then every face in $G[B_x \cup B_y]$ is contained in either $B_x$ or $B_y$. So every face is monochromatic. If $ab$ is a chord on a face $F$, $F$ lies in the subtree rooted at $B_x$ and $B_y$. Then $F$ can be coloured with a single monochromatic as we can cycle through colours of $B_x$ so that the colour of $F$ in $B_x$ and the colour of $F$ in $B_y$ matches. Then every face is preserved in the book-embedding. 

	If $S = \{u,v,w\}$, and if $u$ appears on a face $F$, then no vertices are moved around. If $u,v$ appear on a face $F$, then the edge $uv$ is in $B_x$. Then there is a face $F'$ in $B_x$ with the edge $uv$ on its boundary so that $F'$ is a part of $F$. But this means that there is a monochromatic path from $u$ to $v$. In this book-embedding, $u,v,w$ are moved to the very front. Then the edges of $u,v,w$ break the preserved face at $F$ and $F'$. Only the edges that $u$ and $v$ is adjacent to change colours, meaning that the number of monochromatic paths at the end is $8$. Repeat for all cut-sets of size $3$ on the boundary of $F$, and as there are a bounded number of cutsets in $G$, then $F$ has $8k$ monochromatic paths.

\end{proof}	
There is no restriction on the number of planar graphs with cut-sets of size $3$.  There may be an unbounded number of vertices moved from \cref{thm:clique_sum_pagenumber_bound}, meaning that the number of monochromatic paths that are generated is unbounded.



\subsection{Surfaces and graphs on surfaces}
Graphs on surfaces are a natural extension to graphs on planes. This section is an introduction to surfaces and graphs on surfaces. Readers are expected to be familiar with point-set topology. This section is based on \textcite{moharGraphsSurfaces2001}.

An \textit{$n$-manifold} $M$ is a second-countable Hausdorff space where every point in $M$ has an open neighbourhood homeomorphic to an open ball in $\mathbb{R}^n$.  
A \textit{surface} is a $2$-manifold. Surfaces are typically denoted as $\Sigma$. Examples of surfaces are the sphere $S^2$, the torus $T^2$, the real projective plane $\mathbb{R}P^2$, and the Klein bottle $K$. 

\textit{Handles} are added to a surface \(\Sigma\) by removing two disks in \(\Sigma\) and identifying the boundaries such that one goes clockwise and the other goes counter-clockwise. \textit{Crosscaps} are added to a surface $\Sigma$ by removing a disk in \(\Sigma\) and identifying opposite points on the boundary. Every surface is homeomorphic to a sphere with $m$ handles and $n$ crosscaps. The \textit{Euler genus} of a surface \(\Sigma\) with $m$ handles and $n$ crosscaps is $2m + n$. In fact, a sphere with a mix of crosscaps and handles is homeomorphic to a sphere with all crosscaps, as a sphere with a handle and crosscap is homeomorphic to three crosscaps.

An \textit{embedding} of $G$ on a surface $\Sigma$ is a drawing of $G$ on $\Sigma$ such that no two edges cross. 
A \textit{$2$-cell embedding} of a graph $G$ on a surface $\Sigma$ is an embedding of $G$ in $\Sigma$ such that $\Sigma - G$ is homeomorphic to a finite number of disks. The \textit{Euler Genus} of a \textit{graph} \(G\) is the smallest Euler genus \(g\) surface \(\Sigma\) such that \(G\) can be $2$-cell embedded on $\Sigma$.

An extension for Euler's formula is below. Suppose $G$ is $2$-cell embedded on a surface $\Sigma$ of genus $g$. Let \(|F(G)|\) be the number of faces in a graph \(G\). Then \(|V(G)| - |E(G)| + |F(G)| = 2 - g = \chi\). When $g = 0$, then $\Sigma$ is a $2$-sphere and this is the original Euler's formula. 
The value $\chi$ is known as the \textit{Euler characteristic} of a topological space, in this case a surface. The Euler characteristic is invariant under homeomorphism. Calculating the Euler characteristic of any space is done through \textit{homological algebra}, specifically by looking at the free rank of homology groups. 

Graphs that can be embedded on the plane are called \textit{planar} graphs. Graphs that can be 2-cell embedded on the torus are called \textit{toroidal} graphs, and graphs that can be 2-cell embedded on the projective plane are called \textit{projective-planar} graphs. Graphs that can be 2-cell embedded on a surface of genus $g$ are called \textit{genus $g$} graphs. Similarly to plane graphs, graph drawings on the torus are called torus graphs, and graphs drawings on the projective plane are called projective-plane graphs. 


Graphs on surfaces have been studied extensively. A famous conjecture involving graphs on surfaces is Heawood's conjecture, from \textcite{heawoodMapcolourTheorem1890}. The conjecture states that the minimum number of colours sufficient to colour all Euler genus $g$ graphs when $g \geq 0$ is
	\begin{equation*}
		\gamma(g) := \left\lfloor 
		\frac{7 + \sqrt{1 + 24g}}{2}
		\right\rfloor.
	\end{equation*}\todo{is this right?}
\textcite{ringelMapColorTheorem1974} showed that for almost every case, $\gamma(g)$ is also necessary. The case where this does not hold is the Klein bottle case. There exists a 6-colourable Klein bottle graph, but $\gamma(g) = 7$. 

\chapter{Book-embeddings and orientable surfaces}

This chapter explores book-embeddings involving graphs on orientable surfaces.We first discuss a paper by \textcite{heathPagenumberGenusGraphs1992} on book-embeddings of graphs of bounded orientable genus. We then discuss embedding graphs of bounded nonorientable genus. 

\subsection{Bounded treewidth and page number}\label{ssec:Bounded_Treewidth}
\begin{theorem}[Ganley + Heath\cite{ganleyPagenumberTrees2001}]\label{thm:bded_treewidth_bded_pagenumber}
	Every graph \(G\) with \(\tw(G) \leq k\) can be embedded on $k + 1$ pages.
\end{theorem}
In the original proof, they considered the case where \(G\) is a \(k\)-tree. We will bypass using \(k\)-trees and consider a tree-decomposition of \(G\) directly.

\begin{proof}
	Consider a tree-decomposition of \(G\) with bags $B_x$ and tree $T$. Perform a depth-first search on \(T\), starting at an arbitrary root node \(r\). Let the ordering of the book-embedding \(\sigma(v)\) of a vertex \(v\) in \(V(G)\) be determined by the first time \(x \in T\) appears, where \(v \in B_x\). Within each bag, if two vertices appears in the bag first, then they are ordered arbitrarily in the book-embedding. Now consider the subtree \(T_v\) induced by the bags \(B_x\) containing \(v\). We now consider colouring the subtrees \(T_v\) for all \(v \in G\) such that no overlapping subtrees have the same colour. Let \(H\) be the intersection graph of the subtrees, where \(V(H) = \lbrace T_v : v \in G \rbrace\) and \(T_u T_v \in E(H)\) if there exists a bag \(B_x\) such that \(u, b \in B_x\). We have that \(H\) is perfect, and thus \(\chi(H) = \omega(H)\). Then as \(\tw(G) \leq k + 1\), then the size of a clique in \(H\) is at most \(k + 1\). Thus \(H\) is \(k + 1\)-colourable.
	\paragraph{}
	We now use this to assign the edges of \(G\) a page. Let \(c(T_v)\) be the colour assigned to \(T_v\). Colour each edge \(uv \in E(G)\) as follows:
	\begin{equation}
		c(uv) =
		\begin{cases}
			c(T_u) & \text{ if } \sigma(u) \leq \sigma(v), \\
			c(T_v) & \text{ if } \sigma(v) \leq \sigma(u)
		\end{cases}
	\end{equation}
	Then we claim that this is a proper book-embedding of \(G\). Suppose we have that edges \(uv\), \(xy\) cross, so \(\sigma(u) \leq \sigma(x) \leq \sigma(v) \leq \sigma(y)\). However, this implies that there exists a bag \(B\) such that \(u, x, v \in B\), as we have that \(uv\) is an edge in \(B\) and we do a depth-first search to establish the ordering. So \(u, x, v\) they are in the same bags. However, this implies that the trees \(T_u\) and \(T_x\) intersect, meaning that \(c(uv) \neq c(xy)\). Finally, the number of pages used is \(\chi(H) \leq k + 1\), so \(\pn(G) \leq k + 1\). Thus shown.
\end{proof}

We have a simpler proof if the tree is a path. We go from one end of the path to the other, and add vertices to the book-embedding in the order of the first time they appear. Then we colour each vertex such that in each bag, no two vertices are assigned the same colour. We can do this as this is the same intersection graph as above. The rest of the proof follows.

\subsection{Planar graphs}\label{ssec:Planar_Graphs}
\begin{theorem}[Yannakakis \cite{yannakakisEmbeddingPlanarGraphs1989}]\label{thm:4Pages_Planar}
	Planar graphs can be embedded on at most four pages.
\end{theorem}
We have shown above a proof that the number of pages necessary to embed a planar graph is bounded. However, the proof given by Yannakakis is tight, as there exist planar graphs that need four pages \cite{yannakakisPlanarGraphsThat2020} \cite{kaufmannFourPagesAre2020}. We need the fact that the number of pages to embed a planar graph is bounded for proving that graphs embedded on a surface of bounded genus has bounded pagenumber.

\section{Graphs embedded on a surface of bounded genus}\label{sec:pagenumber_bounded_genus}

\begin{theorem}[Heath and Istrail\cite{heathPagenumberGenusGraphs1992}]\label{thm:Genus_pagenumber_bound}
	Let \(g\) be the genus of a graph \(G\). For all graphs \(G\), \(\pn(G) \leq O(g)\).
\end{theorem}
Note that this bound extends the one found by Yannakakis \cite{yannakakisEmbeddingPlanarGraphs1989} to graph families of bounded genus.
The best known bound is \(O(\sqrt{g})\), found by Malitz\cite{malitzGenusGraphsHave1994}.

It was shown by Heath and Istrail that the family of graphs of bounded genus have bounded pagenumber.
We refer to the ``layout'' of the graph as the book-embedding of the graph and ``embedding'' as the surface-embedding. We refer to orientable surfaces as genus \(g\) as a sphere with \(g\) handles, and a nonorientable surface of genus \(g\) as a sphere with \(g\) cross-caps. We define the orientable genus of a graph \(G\), denoted \(\gamma(G)\), as the minimum orientable surface genus that \(G\) can be embedded on. The nonorientable genus of a graph \(G\), denoted \(\tilde{\gamma}(G)\), is the minimum nonorientable genus surface that \(G\) can be embedded on. Mohar\cite{moharOrientableGenusGraphs1998} claims that \(\tilde{\gamma}(G) \leq 2 \gamma(G) + 1\) for all graphs, meaning that if the orientable genus is bounded, then the non-orientable genus is bounded. Note that, Auslander et al.\cite{auslanderImbeddingGraphsManifolds1963} showed that there exists graphs which are embeddable on the projective plane who has arbitrarily large orientable genus.
\paragraph{Proof}
We say that the embedding is \(2\)-cell if every face is homeomorphic to an open disc in \(\mathbb{R}^2\). Any embedding of \(G\) onto an orientable surface is a 2-cell embedding, but this does not hold for nonorientable surfaces, but we assume there exists a \(2\)-cell embedding.
Heath and Istrail rely on decomposing the graph \(G\) of genus \(\gamma(G)\) into a planar spanning subgraph \(G_p\) of \(G\) such that:
\begin{enumerate}
	\item The edges in \(E(G) - E(G_p)\) attach to the boundary vertices of \(V(G_p)\).
	\item Adding an edge from \(E(G) - E(G_p)\) to \(G_p\) breaks the above condition.
\end{enumerate}
To talk about graphs embedded in surfaces, we assign to each face a cyclic permutation \(\sigma_v\) which represents the sequence of vertices encountered when traversing the boundary of a face in counterclockwise order.

This is enough to represent any graph in an orientable surface, but not enough for a non-orientable surface. We have to attach on an orientation to each edge, where each edge is either orientation-preserving or orientation-reversing.

We have that a planar-nonplanar decomposition of \(G\) is a triple \((R, G_P, E_N)\) where \(R\) is a rotation of \(G\) representing the surface embedding on the surface \(S\), \(G\) is a spanning planar graph, and \(E_N = E - E(G_P)\).
This satisfies a list of properties:
\begin{enumerate}
	\item The subrotation induces a planar embedding of \(G_p\), so we can arrange \(G\) on the surface \(S\) such that the embedding of \(G_p\) is planar.
	\item For each \(vw \in E_N\), we have that \(v\) and \(w\) live on the outerface \(F_0\).
	\item \(E(G_P)\) is maximal, so we cannot add edges from \(E_N\) to \(G_P\) without breaking properties 1 and 2.
\end{enumerate}

\subsubsection{Decomposing graphs on surfaces}\label{sssec:Planar_nonplanar_decomp}
We first have to know that the planar-nonplanar decomposition exists.

Suppose \(G\) is embedded on an surface \(\Sigma\). Then we wish to triangulate \(G\) to form \(G_T\). We choose a single triangle as the starting point and we add traces to the planar part incrementally. At each step, we set \(G_P\) to be the current planar part and \(E_N\) as the edges that are outside the planar part. There are two types of edges in \(E_N\): edges which have both endpoints in \(V(G_P)\), so cannot become edges of \(G_P\), and edges that have either one or no endpoints in \(V(G_P)\).

We want to maintain the property that if \(v \in G_P\), and edge \(vw \in E_n\), then \(v\) is a vertex on the boundary of \(G_p\).
\paragraph{Adding vertices to biconnected block}
For a current boundary of the outerface of \(G_P\), if \(v_i \rightarrow v_j \rightarrow v_k\) is trace with no edge of \(E_N\) incident to \(v_j\), then \(v_i v_k \in E(G_T)\) is called a safe edge. If \(v_i \rightarrow v_j\) is on the boundary of \(G_P\), and \(v_k \notin V(G_P)\), and \(v_i,v_j,v_k\) is the boundary of a face, then \(v_k\) is a safe vertex and we can add it to \(G_P\).
\paragraph{Creating new biconnected block}
If no \(v_k\) exists, then we find a \(w'\) which is the newest vertex in \(V(G_P)\) adjacent to a vertex \(w\) not in \(V(G_P)\). We then add the vertex \(w\) and the edge \(w w'\) to \(G_P\). Then we add all safe edges. This is so that every edge in \(E_p\) maintains the property that both endpoints are on the boundary.

We claim that after repeating this operation, then we have that every edge in \(E_N\) (edges not in \(G_P\)) satisfy the two properties above. If we have that an edge \(vw\) has \(v\) added, then we should be able to add \(w\) as a safe vertex or biconnected block. If an edge \(vw\) has neither \(v\) or \(w\) added to \(G_P\), then the algorithm has not finished yet. By connectivity, we can add \(v\) and \(w\) at some stage and therefore go to part 1. This has the corollary that every vertex is in \(G_P\).

Now we have that \(E_N\) has every edge which cannot be added to \(G_P\) without crossing over another edge, and that \(G_P\) is maximal. Then we have that all edges in \(E_N\) satisfy the conditions lised above.
\todo{Add pictures! this proof needs lots of pictures}

\subsubsection{Level sets and cycles}
On a planar graph \(G\), we want to separate out vertices depending on how far away they are from the outerface. We fix a single outerface \(F_0\) and define the first level set \(V_0\) as the vertices adjacent to \(F_0\). We then define the \(i\)-th level set, \(V_i\) inductively. Consider the induced graph on \(V(G) - \cup_{k = 0}^{i-1} V_k\). Then we define the vertices adjacent to \(F_0\) in this induced graph, where we expand \(F_0\) to include the vertices. This partitions \(V(G)\).

We then define \(C_0\) to be the edges adjacent to \(F_0\) in this decomposition. Then we want \(C_i\) to be the edges adjacent to \(F_0\) in this decomposition. We define the chord edges \(K_i\) to be the edges between vertices in \(V_i\) that are not edges in \(C_i\). Finally, we define the edges between faces, \(E_i\) as the edges that are between vertices on level \(V_i\) and \(V_{i + 1}\).

\begin{lemma}
	For all faces \(F\) in \(G\), the vertices around \(F\) are either all in one \(C_i\) or they are in \(C_i\) and \(C_{i + 1}\) for some \(i\).
\end{lemma}

\begin{proof}
	Let \(i\) be the smallest value such that \(v \in V_i\) is on the boundary of \(F\). Now we have that \(G[V(G) - \cup_{j = 1}^{i} V_i]\) will also remove \(v\). However, this removes all the edges next to \(v\), therefore all vertices that are on the boundary of \(F\) will either be in \(V_i\) or \(V_{i + 1}\).
\end{proof}
We refer to the faces that have vertices in only \(V_i\) as chordal and the faces that are between \(V_i\) and \(V_{i + 1}\) as non-chordal.

We define a weak triangulation of \(G\) to be a triangulation \(G'\) such that all faces except for the outerface is a triangulation.
\begin{lemma}
	There exists a weak triangulation of \(G\), \(G'\) which preserves the level sets \(V_i\) and edge sets \(E_i\), \(C_i\), \(K_i\) for all \(i\).
\end{lemma}

\begin{proof}
	If \(F\) is a chordal face of \(G\), then any triangulation maintains the property. If \(F\) is nonchordal and the boundary has edges in \(V_i\) and \(V_{i + 1}\), then add edges that are only between vertices in \(V_i\) and \(V_{i + 1}\). This will suffice to build a new triangulated graph \(G'\) where all vertices and edges are in the correct place.
\end{proof}

\subsubsection{Classifying nonplanar edges according to homotopy class}

We can then form a directed cycle \(C_0\) induced by \(F_0\). Each vertex on the boundary of \(F_0\) appears at least once, and twice if it is an \textit{articulation point}. Each edge on the boundary of \(F_0\) is encountered at least once on this cycle. Heath and Istrail refer to a directed subpath of the cycle \(C_0\) as a trace, so trace \(T = v_1 \rightarrow v_2 \rightarrow \cdots \rightarrow v_t\). The inverse trace is \(T^{-1} = v_t \rightarrow v_{t-1} \rightarrow \cdots \rightarrow v_1\). We now wish to partition \(E_N\) into equivalence classes. Suppose that \(u_1v_1, u_2v_2 \in E_N\) are part of the boundary of the same face \(F\) on the embedding of \(G\). Then \(u_1v_1\) and \(u_2v_2\) are \textit{homotopic} (with respect to \(F\)) if:
\begin{enumerate}
	\item \(u_1v_1\) and \(u_2v_2\) are the only edges of \(E_N\) on the boundary of \(F\)
	\item There exist traces \(T_u = u_1 \rightarrow \cdots \rightarrow u_2\) and \(T_v = v_1 \rightarrow \cdots \rightarrow v_2\) such that \(T_u\) and \(T_v\) are on the boundary of \(F\).
\end{enumerate}
We may think of \(G_n\) as living on a locally flat part of \(S\) and the homotopy class \(u_1v_1\) and \(u_2 v_2\) living on a handle (alternatively, passing through a crosscap such that they bound a face). Then if we take \(G_n\) to a point, there exists a \textit{homotopy} (in the topological sense) from \(u_1v_1\) to \(u_2v_2\). These form equivalence classes of the nonplanar edges.

\begin{lemma}
	If \(C\) is a homotopy class of edges \(u_1 v_1, \ldots, u_k v_k\) with a natural order, then we can build traces \(T_1\) and \(T_2\) by building the trace from \(u_1\) to \(u_k\) passing through all \(u_i\), and \(v_1\) to \(v_k\) passing through all \(v_i\).
\end{lemma}
We refer to a homotopy class as orientable if \(T_1\) and \(T_2\) go in opposite directions, and non-orientable if \(T_1\) and \(T_2\) go in the same direction.

\begin{lemma}
	We have that if \(G\) is embedded in an orientable surface, then every homotopy class is orientable.
\end{lemma}
\begin{proof}[Sketch]
	We have that if a homotopy class is non-orientable, then on the handle the class sits on, the edges must cross. However, we have the graph is embedded on the surface, therefore this cannot happen. Thus shown.
\end{proof}

\begin{lemma}
	If \(G\) is \(2\)-cell embedded on a surface of Euler genus \(g\), then any planar-nonplanar decomposition has at most \(3g-3\) homotopy classes.
\end{lemma}
\begin{proof}
	Decompose \(G\) to a \((R, G_P, E_N)\) decomposition of \(G\). Suppose \(E_N \neq \emptyset\). Then identify \(G_P\) to a single point, and identify each homotopy class to a single edge. Then draw a circle around the point \(G_P\), and place vertices where the circle intersects all edges. Then delete the vertex \(G_P\), and call the new graph \(H\). We have that \(n = |V(H)|\), \(m = |E(H)|\), \(h\) is the number of homotopy classes, and \(f\) is the number of faces. We have that \(n - m + f = 2 - g\). Since \(H\) is cubic as every vertex has two edges on the circle and one on the homotopy class, then \(3n = 2m\) by the handshaking lemma. Since there is only one nonplanar edge for each homotopy class, \(n = 2h\). The interior face of \(H\) has \(v\) incident edges, and the remaining \(f-1\) faces have at least 3 incident edges each, as we can identify the two homotopy classes bordering a face with four edges together. Therefore, we have that \(3(f-1) + n \leq 2m\), by double counting faces and edges. Thus, we have that
	\begin{align*}
		3n  & \geq 6(f - 1) + n         \\
		2n  & \geq 6f + 6               \\
		4h  & \geq 5 f - 6              \\
		4h  & \geq 5(2 - g + m - n) - 6 \\
		4h  & \geq 6 - 6g + 3n          \\
		4h  & \geq 6 - 6g + 6h          \\
		-2h & \geq 6 - 6g               \\
		h   & \leq 3g - 3
	\end{align*}
	\(3g - 3 \geq h\) by manipulating the inequalities.
\end{proof}

\subsubsection{Proving graphs with bounded number of homotopy classes have bounded pagenumber}\label{sssec:bounded_pagenumber_homotopy}
\begin{lemma}\label{lem:planar_nonplanar_orientable}
	Suppose \(G\) has a planar-nonplanar decomposition \((R, G_P, E_N)\) on an orientable surface \(\Sigma\). Then \(G\) can be embedded on at most \(18g - 5\) pages.
\end{lemma}
\begin{proof}
	We use Yannikakis' proof in \cref{ssec:Planar_Graphs} to lay out the nonplanar spanning subgraph \(G_P\) on four pages, maintaining the cyclic order of vertices. Then we can combine each blocks to form a 4 page layout of the graph. For each homotopy class in \(E_P\), we allocate three pages. One page is for vertices in the same block, and the other two pages are used for edges between blocks, the biconnected components of \(G\). We need two as we could have some which span blocks in a way that forces them to be on different pages. Therefore, we need at most \(4 + 3(6g - 3) = 18g-5\) pages if \(G\) has a planar-nonplanar decomposition.
\end{proof}

\begin{lemma}\label{lem:planar_nonplanar_nonorientable}
	Suppose \(G\) has a planar-nonplanar decomposition \((R, G_P, E_N)\) on a non-orientable surface \(\Sigma\). Then \(G\) can be embedded on at most \(9g - 1\) pages.
\end{lemma}
\begin{proof}[Proof sketch]
	\todo{Flesh out details completely}
	We want to add edges in a controlled way so that the traces that are reversed become non-reversed. This is done by adding edges between vertices so that we can invert the ordering on the circle such that we have that the vertices in one homotopy class have a non-crossing page embedding. However, an issue is chords that go between traces that are inverted. We go around this problem by removing chords and adding them to a separate page where there are finitely many pages wrt to the genus of the surface. Let \(\mathcal{C}_{i,j}\) be a chord class when it is the set of chords that go between traces \(T_i\) and \(T_j\), where \(T_i\) and \(T_j\) both go clockwise or counterclockwise. Note that the number of chord classes \(\mathcal{C}_{i,j}\) is bounded by the genus of the graph, and we can embed the chord classes that share a trace onto a single page. As there is a bounded number of chord classes, it must hold that the number of pages is finite.

	Consider the scenario where the graph $G_p$ is a cycle with a cyclic ordering. Then every trace on the boundary is a path on the cycle. Recall that a trace $T_1$ and $T_2$ are paired if there is a nonplanar edge between the two traces. For every two pairs of traces that are non-orientable, we label one to be non-reversed and one to be reversed, with arbitrary allocation. We then merge all the consecutive reversed and non-reversed traces together. Then we add an edge around each reversed trace, called bypassing edges. We then use Heath's algorithm to find an edge colouring with 7 colours.

	Now consider the case where $G_p$ is outerplanar. We cannot do the unreversing trick anymore, because of edges inside of the graph. However, we can form a path with these edges and colour them with a finite number of colours as well.
	\todo{finish this case!}
	We bundle chords together and the number of chords that exist is $O(g)$ as there are at most $g$ reversed-unreversed trace changes on the boundary of the surface. As no chords cross, then we have that there are $O(g)$ chords. 

	Finally, consider the case where $G_p$ has a block-cut tree. We still have the same chords from before, and now we have that chords can be used to draw a path that goes back and forth between traces.
	\todo{finish this proof!}

	We first merge traces, and then build a boundary between two vertices. Then we have two subgraphs and two traces. 
\end{proof}

\section{Monochromatic paths}

We wish to find a book-embedding of an almost-embeddable graph on some orientable surface. To do so, we need to introduce some new terminology to work with vortices on surfaces. 
Recall the definition of $(g, p, k, a)$-almost-embeddable from \cref{ssec:Robertson_Seymour_Graph_Structure}. A graph $G$ is $(g, p, k)$-almost-embeddable if $G$ is $(g, p, k, 0)$-almost-embeddable, so $G$ has no apex set.
What we plan to show is this:
\begin{theorem}\label{thm:bounded_almost_embeddable}
	Suppose $G$ is $(g, p, k)$-almost embeddable on an orientable surface. Then $G$ can be embedded on \(f(g, p, k)\) pages for some function $f$.
\end{theorem}

The most problematic section is dealing with vortices on surfaces.
To work with vortices, consider how an ordering affects the face that the vortex is sitting on. Then see what happens when the vortex is added onto the face. 

Let \(F\) be a face on \(G\). Let \( (<, \varphi) \) be a book-embedding of \(G\). A \textit{monochromatic path} $P$ on the boundary of $F$ is a maximal path where every edge is given the same colour from $\varphi$. $F$ being a preserved face implies that $F$ has a single monochromatic path, but not the other way around. 

\begin{figure}[h!]
	\centering
	\includesvg[width = 0.8\textwidth]{figures/monochromatic_paths.svg}
	\caption[Monochromatic paths]{An example of a monochromatic path on a face. Above is a face embedded on $\mathbb{R}^2$. The circular book-embedding has edges coloured red and blue. This face has four monochromatic paths. Below is a circular ordering of the vertices on the boundary of the face. Every edge coloured red is embedded on a single page.}
\end{figure}


The list below are the steps taken to prove \cref{thm:bounded_almost_embeddable}.
\begin{enumerate}
	\item Suppose $G$ is a \(4\)-connected planar graph. There is a book-embedding where every face is monochromatic, by \cref{thm:4-connected_planar_ham_cycle}. 
	\item Suppose \(G\) is a connected planar graph. Faces are not preserved, but a fixed number of vertices are moved around on every face. Additional pages are necessary to embed $G$, but the number of monochromatic paths is bounded. 
	\item Suppose \(G\) is $2$-cell embedded on an orientable surface. \textcite{heathPagenumberGenusGraphs1992} gives a planar-nonplanar decomposition of $G$. Then apply the previous steps to the spanning planar subgraph and add vortices.
\end{enumerate}

Preserved faces allow us to embed vortices with a bounded number of pages. 

\begin{lemma}[Vortex on preserved faces]\label{lem:preserved_faces_pagenumber}
	Suppose a graph \(G = G_0 \cup G_1\), where \(G_0\) is embedded on a surface $\Sigma$ and \(G_1\) is a vortex on a face $F$ of $G_0$ with depth \(k\). Suppose $(<, \varphi)$ is a $p$-page book-embedding of \(G_0\) which preserves \(F\). Then $G$ can be embedded on \(p + k + 1\) pages. Furthermore, the new ordering restricted to $G_0$ is $(<, \varphi)$.
\end{lemma}

\begin{proof}
	We repeat a similar argument to \cref{thm:bded_treewidth_bded_pagenumber}. Let \(B_1, \ldots, B_i\) be a path-decomposition of \(G_1\). Let \(\sigma(v)\) be the first time \(v\) appears in the path-decomposition. Colour the edges of \(G_1\) as such. If \(uv \in E(G_1)\), then:
	\begin{equation}
		c(uv) =
		\begin{cases}
			c(T_u) & \text{ if } \sigma(u) \leq \sigma(v), \\
			c(T_v) & \text{ if } \sigma(v) \leq \sigma(u).
		\end{cases}
	\end{equation}
	This is a book-embedding of \(G_1\) with \(k+1\) colours. Since the intersection graphs of every graph is perfect, we can colour with $k + 1$ colours. Furthermore, suppose that edges \(uv\), \(xy\) cross, so \(\sigma(u) \leq \sigma(x) \leq \sigma(v) \leq \sigma(y)\). However, this implies that $u,x,v$ are in some bag $B$. \(uv\) is an edge in \(B\), and we do a depth-first search to establish the ordering of $\sigma(u)$. So \(u, x, v\) are in the same bags. However, this implies that the trees \(T_u\) and \(T_x\) intersect, meaning that \(c(uv) \neq c(xy)\). Therefore, all crossing edges are assigned different colours. 

	To add this book-embedding to \(G_0\), add the vertices that appear first in \(B_i\) after the associated vertex \(v_i\) in \(G_0\) such that \(v_i\) is on the face \(F\) and \(v_i \in B_i\). This is a book-embedding of \(G\) requiring at most \(\pn(G_0) + k + 1\) colours. An illustration of such a book-embedding is in \cref{fig:preserved_face}.
\end{proof}

\begin{figure}[h!]
	\centering
	\includesvg[pretex=\tiny, width = 0.8\textwidth]{figures/bookembedding_preserved_face.svg}
	\caption[Book-embedding of a preserved face]{A book-embedding of a preserved face  with a vortex on the face. The green vertices are the boundary of the face and the other coloured vertices are distinct vertices in each bag. The bag is the red circle. This graph can be embedded on $3$ pages as the depth of the vortex attached is 2. }\label{fig:preserved_face}
\end{figure}

\begin{lemma}\label{lem:vortices_mono_paths}
	Suppose \(G\) is a graph where $G = G_0 \cup G_1$, where \(G_0\) is embedded on a surface \(\Sigma \) of genus \(g\). Let \(F\) be a face on \(G_0\). Let \(v_1, v_2, \ldots, v_k\) be the vertices bordering \(F\). Let \(D\) be a \(G\)-clean disk on \(F\). Now suppose \(G_1\) is a vortex of depth $k$ on \(D\) with a path-decomposition \((B_0, \ldots, B_l)\) and \(G_0\) has a book-embedding \((<, \varphi)\) on $s$ pages. Suppose there are at most \(m\) monochromatic paths for every face on $G_0$. Then \(G\) can be embedded on \(s + m(k+1)\) pages.
\end{lemma}

\cref{lem:vortices_mono_paths} implies \cref{corr:vortices_paths_pn}.

\begin{corollary}\label{corr:vortices_paths_pn}
	Suppose $G$ is a graph where $G = G_0 \cup G_1 \cup \ldots \cup G_P$, $G_0$ is embedded on a surface $\Sigma$ of genus $g$, and $G_1, \ldots ,G_P$ are vortices on $G_0$. Suppose further that $G_0$ has $m$ monochromatic paths on each of its faces on $\Sigma$. Then $G$ can be embedded on $s + pm(k+1)$ pages.
\end{corollary}

This is proven by adding $G_1, \ldots, G_P$ to $G_0$ one at a time and applying \cref{lem:vortices_mono_paths} each time. To prove \cref{lem:vortices_mono_paths}, we prove \cref{lem:one_page_decomposition}.
\begin{lemma}\label{lem:one_page_decomposition}
	Let \((B_1, \ldots, B_n)\) be a path-decomposition of \(G\) with path-width \(k\). Let \(x_1, \ldots, x_n\) be vertices in \(G\) such that \(x_i \in B_i\) for all \(i\), and suppose \(P\) is an induced path \((x_1, x_2, \ldots, x_n)\) in \(G\). Then for every one-page embedding of \(P\), \(G\) has a \((k + 1)\)-page embedding where $P$ is preserved in the book-embedding.
\end{lemma}
\begin{proof}
	The proof given is similar to the one in \cref{lem:preserved_faces_pagenumber}. 
	Suppose \(G\) has the structure as described in \cref{lem:vortices_mono_paths}. For each vertex \(v\) in \(G\), let \(\sigma(v)\) be the index of the first bag \(v\) appears in. Then in the book-embedding of \(G\), place all bags of \(v_i\) after \(x_i\) in the book-embedding, following the orientation of the path with the book-embedding. Colour the edges like so:
	\begin{equation}
		c(uv) =
		\begin{cases}
			c(T_u) & \text{ if } \sigma(u) \leq \sigma(v), \\
			c(T_v) & \text{ if } \sigma(v) \leq \sigma(u).
		\end{cases}
	\end{equation}

	Then if two edges cross in the book-embedding, then they have different colours. If two edges cross, then that implies that in a book-embedding of the path \(P\) with \((B_1, \ldots, B_n)\) added in like \cref{lem:preserved_faces_pagenumber}, then they will cross as well. Two examples are given in \cref{fig:preserving_pages}.
\end{proof}

\begin{figure}[h!]
	\centering
	\includesvg[pretex=\tiny, width = 0.8\textwidth]{figures/one_page_embedding.svg}
	\caption[One-page decomposition]{Description of \cref{lem:one_page_decomposition}. \(x_1, \ldots, x_n\) are the vertices with a path that is a single book-embedding and \(B_1, \ldots, B_n\) are the bags of the embedding. Notice that there are two different ways that the \(n + 1\)-th bag can end up, but both ways still maintain the property that this is a book-embedding. This diagram is a circular ordering of \(x_1, \ldots, x_n\).}\label{fig:preserving_pages}
\end{figure}

We now prove \cref{lem:vortices_mono_paths}.
\begin{proof}
	Use the path-decomposition on \(G'\) as the set \((B_1, \ldots , B_n)\) in proving \cref{lem:vortices_mono_paths}. Then apply \cref{lem:one_page_decomposition} for the monochromatic \(v_i\) to each of the monochromatic paths. From the construction of the vortices in \cref{lem:preserved_faces_pagenumber}, add on the faces in the exact order. Then the monochromatic paths are preserved in the ordering, and in fact from \cref{lem:one_page_decomposition}, they are bounded.
\end{proof}
\begin{lemma}\label{lem:Hamiltonian_preserved_faces}
	Let \(G\) be a Hamiltonian planar graph with Hamiltonian cycle $C$ Then there exists a book-embedding $(\leq, \varphi)$ on two pages where every face of $G$ is preserved, and the circular ordering of $\leq$ is $C$.
\end{lemma}

\begin{proof}
	Let \(<\) be the circular ordering of these vertices by traversing \(C\). Now as \(G\) is planar, \(C\) splits the surface into an interior region and an exterior region, by the Jordan curve theorem. So every edge in $G - C$ is inside either the interior or exterior of \(C\). Colour every edge in $C$ and on the interior of $\mathbb{R}^2 - C$ red and colour the other edges blue. This is a book-embedding as both regions can be embedded on a single page with the same cyclic ordering. Furthermore, every face is preserved. Because $\mathbb{R}$ is orientable, affix an orientation to every face \(F\) on $G$ such that the order of the vertices in the orientation is the same order as the orientation in \(D\).
	\begin{figure}[h!]
		\centering
		\includesvg[pretex=\tiny, width=0.3\linewidth]{figures/hamiltonian-planar}
		\caption[Hamiltonian planar graph]{This is a Hamiltonian planar graph with a book-embedding of 2. The linear ordering is $v_1, v_2, \ldots,  v_8$. Let $C$ be the Hamiltonian cycle. $C$ splits the plane into two regions. Then edges in the interior region and on $C$ are coloured red, and the remaining edges are blue. Now choosing any face and restricting the linear ordering to be around the face preserves each face. }\label{fig:hamiltonian_planar}
	\end{figure}
\end{proof}
From \cref{lem:Hamiltonian_preserved_faces}, if \(G\) is a Hamiltonian planar graph, then the vertex ordering of the Hamiltonian cycle \((\leq)\) preserves all faces on \(G\). As a consequence of \cref{thm:4-connected_planar_ham_cycle}, every 4-connected planar graph has a circular ordering which preserves every face.

Then this can be extended to every connected planar graph, adding a constant number of monochromatic paths, as shown by \cref{thm:planar_graph__be_mono_paths}.

\begin{lemma}\label{lemma:decomposition_faces}
	Let $G$ be a planar graph embedded on $\mathbb{R}^2$, and let $F$ be a face. The vertex set of $F$ has a tree-decomposition $(T, (B_x)_x)$ of adhesion 1 where each torso is a cycle.
\end{lemma}

\begin{proof}
	Suppose the facial walk of $F$ is a cycle $C$. Then there are no cut-vertices, so $C$ is a valid cycle.

	Now suppose $F$ has a cut vertex.Let $W$ be a facial walk of $F$, and suppose the sum times duplicate vertices appear is $n$. Suppose for all faces with walks with $n-1$ duplicate vertices, the lemma holds. Take a cut-vertex $v$ and look at the first and last time $v$ appears in $W$, call these positions $a, b$. Then let $W_1$ be the subwalk $W[a, b-1]$ and let $W_2$ be the subwalk of $W - W[a, b-1]$, removing $[a, b-1]$. Now both $W_1$ and $W_2$ bound subfaces of $F$. Then applying the inductive hypothesis, $W_1$ and $W_2$ have the decomposition above. Then there is a bag containing $v$ in the tree-decomposition of $W_1$ and $W_2$, so join the two bags. As $W_1$ and $W_2$ only have $v$ in common, then $w_1, w_2$ contain $w$. 
\end{proof}

We want to show that there exists a book-embedding of every planar graph where every face has a bounded number of monochromatic paths. We were able to prove it in the special case where there is no cut-set of size 3.

\begin{theorem}\label{thm:planar_graph__be_mono_paths}
	Let \( G \) be a planar graph embedded in $\mathbb{R}^2$, with $k$ sets of graphs with adhesion $3$. Then $G$ can be embedded on 11 pages where every face has at most $8k$ monochromatic paths. 
\end{theorem}

$ ((T,r), (B_x)_x)$ is a tree-decomposition with $r$ as the root node. 

\begin{proof}
	Use \cref{lem:planar_graphs_4_connected_cliqesums} to build a tree-decomposition $((T,r), (B_x)_x)$ of $G$ with adhesion $3$. Every torso is either a $4$-connected planar graph or $K_t$ where $t \leq 4$.
	
	Every torso $G \langle B_x \rangle$ has a tree-decomposition where every face is preserved, from \cref{lem:Hamiltonian_preserved_faces}. Then apply \cref{thm:clique_sum_pagenumber_bound} to $((T,r), (B_x)_x)$ to embed $G$ in a book. The number of pages used is $2 \cdot 4 + 3 = 11$. 

	If $v$ is a cut-vertex of $G$, then $v$ is in an adhesion set of torsos $G \langle B_x \rangle$ and $G \langle B_y \rangle$. Assume that $B_y$ is the parent of $B_x$. Then from \cref{thm:clique_sum_pagenumber_bound}, the ordering does not change, so $v$ is not moved anywhere. Then any face $F$ where every vertex on the boundary of $F$ is in $B_x$ is preserved. Since the outerface of $B_x$ is preserved, the colour of the outerface can be changed so that it matches the colour of the face that $B_x$ is a part of. 

	Now suppose $S$ is an adhesion set between bags $B_x$ and the parent $B_y$. Suppose $ S = \{a,b\}$. Let $F$ be a face in $B_x$ and let $F_v$ be the set of vertices that touch $F$. Assume that $F_v \subseteq B_x$. If $F_v \cap S$ is nonempty, then $S$ vertices are moved to the start of $F$. 
	
	No vertex in $S$ is a cut vertex, because that would imply from \cref{lem:planar_graphs_4_connected_cliqesums} that $|S| = 1$. Let $(<_x, \psi_x)$ be a two-page book-embedding of $G \langle B_x \rangle$, from \cref{lem:Hamiltonian_preserved_faces}. Now $F$ has a single monochromatic face.
	From \cref{thm:4-connected_planar_ham_cycle}, there is a Hamiltonian cycle in $B_x$ with edge $ab$ on the cycle. From \cref{lem:Hamiltonian_preserved_faces}, this implies that $ab$ is consecutive on the cycle. From \cref{thm:clique_sum_pagenumber_bound}, this cycle is the same in $B_x$. If $ab$ is the boundary on a face $F$, then every face in $G[B_x \cup B_y]$ is contained in either $B_x$ or $B_y$. So every face is monochromatic. If $ab$ is a chord on a face $F$, $F$ lies in the subtree rooted at $B_x$ and $B_y$. Then $F$ can be coloured with a single monochromatic as we can cycle through colours of $B_x$ so that the colour of $F$ in $B_x$ and the colour of $F$ in $B_y$ matches. Then every face is preserved in the book-embedding. 

	If $S = \{u,v,w\}$, and if $u$ appears on a face $F$, then no vertices are moved around. If $u,v$ appear on a face $F$, then the edge $uv$ is in $B_x$. Then there is a face $F'$ in $B_x$ with the edge $uv$ on its boundary so that $F'$ is a part of $F$. But this means that there is a monochromatic path from $u$ to $v$. In this book-embedding, $u,v,w$ are moved to the very front. Then the edges of $u,v,w$ break the preserved face at $F$ and $F'$. Only the edges that $u$ and $v$ is adjacent to change colours, meaning that the number of monochromatic paths at the end is $8$. Repeat for all cut-sets of size $3$ on the boundary of $F$, and as there are a bounded number of cutsets in $G$, then $F$ has $8k$ monochromatic paths.

\end{proof}	
There is no restriction on the number of planar graphs with cut-sets of size $3$.  There may be an unbounded number of vertices moved from \cref{thm:clique_sum_pagenumber_bound}, meaning that the number of monochromatic paths that are generated is unbounded.


\subsection{Surfaces and graphs on surfaces}
Graphs on surfaces are a natural extension to graphs on planes. This section is an introduction to surfaces and graphs on surfaces. Readers are expected to be familiar with point-set topology. This section is based on \textcite{moharGraphsSurfaces2001}.

An \textit{$n$-manifold} $M$ is a second-countable Hausdorff space where every point in $M$ has an open neighbourhood homeomorphic to an open ball in $\mathbb{R}^n$.  
A \textit{surface} is a $2$-manifold. Surfaces are typically denoted as $\Sigma$. Examples of surfaces are the sphere $S^2$, the torus $T^2$, the real projective plane $\mathbb{R}P^2$, and the Klein bottle $K$. 

\textit{Handles} are added to a surface \(\Sigma\) by removing two disks in \(\Sigma\) and identifying the boundaries such that one goes clockwise and the other goes counter-clockwise. \textit{Crosscaps} are added to a surface $\Sigma$ by removing a disk in \(\Sigma\) and identifying opposite points on the boundary. Every surface is homeomorphic to a sphere with $m$ handles and $n$ crosscaps. The \textit{Euler genus} of a surface \(\Sigma\) with $m$ handles and $n$ crosscaps is $2m + n$. In fact, a sphere with a mix of crosscaps and handles is homeomorphic to a sphere with all crosscaps, as a sphere with a handle and crosscap is homeomorphic to three crosscaps.

An \textit{embedding} of $G$ on a surface $\Sigma$ is a drawing of $G$ on $\Sigma$ such that no two edges cross. 
A \textit{$2$-cell embedding} of a graph $G$ on a surface $\Sigma$ is an embedding of $G$ in $\Sigma$ such that $\Sigma - G$ is homeomorphic to a finite number of disks. The \textit{Euler Genus} of a \textit{graph} \(G\) is the smallest Euler genus \(g\) surface \(\Sigma\) such that \(G\) can be $2$-cell embedded on $\Sigma$.

An extension for Euler's formula is below. Suppose $G$ is $2$-cell embedded on a surface $\Sigma$ of genus $g$. Let \(|F(G)|\) be the number of faces in a graph \(G\). Then \(|V(G)| - |E(G)| + |F(G)| = 2 - g = \chi\). When $g = 0$, then $\Sigma$ is a $2$-sphere and this is the original Euler's formula. 
The value $\chi$ is known as the \textit{Euler characteristic} of a topological space, in this case a surface. The Euler characteristic is invariant under homeomorphism. Calculating the Euler characteristic of any space is done through \textit{homological algebra}, specifically by looking at the free rank of homology groups. 

Graphs that can be embedded on the plane are called \textit{planar} graphs. Graphs that can be 2-cell embedded on the torus are called \textit{toroidal} graphs, and graphs that can be 2-cell embedded on the projective plane are called \textit{projective-planar} graphs. Graphs that can be 2-cell embedded on a surface of genus $g$ are called \textit{genus $g$} graphs. Similarly to plane graphs, graph drawings on the torus are called torus graphs, and graphs drawings on the projective plane are called projective-plane graphs. 


Graphs on surfaces have been studied extensively. A famous conjecture involving graphs on surfaces is Heawood's conjecture, from \textcite{heawoodMapcolourTheorem1890}. The conjecture states that the minimum number of colours sufficient to colour all Euler genus $g$ graphs when $g \geq 0$ is
	\begin{equation*}
		\gamma(g) := \left\lfloor 
		\frac{7 + \sqrt{1 + 24g}}{2}
		\right\rfloor.
	\end{equation*}\todo{is this right?}
\textcite{ringelMapColorTheorem1974} showed that for almost every case, $\gamma(g)$ is also necessary. The case where this does not hold is the Klein bottle case. There exists a 6-colourable Klein bottle graph, but $\gamma(g) = 7$. 
\chapter{Book-embeddings and orientable surfaces}

This chapter explores book-embeddings involving graphs on orientable surfaces.We first discuss a paper by \textcite{heathPagenumberGenusGraphs1992} on book-embeddings of graphs of bounded orientable genus. We then discuss embedding graphs of bounded nonorientable genus. 

\input{chapters/orientable_surfaces/heathandistrail.tex}
\input{chapters/orientable_surfaces/aegraphs.tex}
\input{chapters/orientable_surfaces/surfaces.tex}
\input{chapters/orientable_surfaces/orientable.tex}



% !TEX root = ./thesis.tex

\section{Further developments}

\textcite{heathPagenumberGenusGraphs1992} did not adequately prove that graphs embedded on non-orientable surfaces of genus $g$ can be bounded on $k$ pages where $k = k(g)$. For the case when the surface is of genus $1$, \textcite{nakamotoBookEmbeddingProjectiveplanar2015} showed that every projective-planar graph can be embedded in $9$ pages, which was improved to 6 by \textcite{ozekiBookEmbeddingGraphs2019}. However, we want to find a book-embedding for all non-orientable surfaces. 

\textcite{ozekiBookEmbeddingGraphs2019} states that they were unable to prove that all Klein bottle graphs can be embedded on a bounded number of pages. \cref{sec:kleinbottle} discusses some strategies to embed Klein bottle graphs, but the problem remains open. The Klein bottle case is just one example in the much broader class of non-orientable surfaces, which is stated in \cref{conj:nonorientable}. For the case that the surface is a projective plane, then $G$ can be embedded in 6 pages, from \textcite{ozekiBookEmbeddingGraphs2019}. We believe the easiest conjecture currently to prove (or possibly disprove) is \cref{conj:nonorientable}. 
\begin{conjecture}\label{conj:nonorientable}
	Every non-orientable genus $g$ graph can be embedded in $f(g)$ pages for some function $f$. 
\end{conjecture}

% !TEX root = ./thesis.tex

\section{Almost-embeddable graphs on the projective plane}

\begin{theorem}\label{thm:projective_planar_be}
	Every $G \in \mathcal{G}(g, p, k)$ where $g = 1$ can be embedded on $1151 + 9600 k p + 20000 k^2 p^2$ pages.
\end{theorem}
Recall the definition of almost-embeddable from \cref{thm:gmst}. 
\begin{lemma}\label{lem:proj_planar vortices}
	Every projective-planar graph with $p$ vortices of depth $k$ can be embedded on $23 + 100pk$ pages.
\end{lemma}
\begin{proof}
	Let $G$ be a graph almost-embeddable on the projective plane, so there are subgraphs $G_0, \ldots, G_P$ where $G_0$ is a projective-planar graph, $G_1, \ldots, G_P$ are vortices on $G_0$ of depth $\leq k$. Use \cref{thm:proj_planar_graphs_23pages}. Then $G_0$ can be embedded in at most $23$ pages, where each face has at most 100 monochromatic paths. Therefore, from \cref{lem:orientablesurfaces_monochromatic_edges}, every book-embedding requires $23 + 100pk$ pages.
\end{proof}

Using \cref{lem:proj_planar vortices}, we can now prove \cref{thm:projective_planar_be}.
\begin{proof}
	Let $G \in \mathcal{G}(g, p, k)$ where $g = 1$. Then every torso has at most $p$ vortices of depth $k$, so every torso can be embedded in $23 + 100 pk$ pages. Then from \cref{corr:bded_pn_tree_decomp}, $G$ can be embedded on $2s^2 + 4s + 1$ pages, whee $s = 23 + 100pk$. Therefore, $G$ can be embedded on $1151 + 9600 k p + 20000 k^2 p^2$ pages.
\end{proof}


\section{Book-embeddings of Klein Bottle-graphs}
The Klein Bottle is written as $K$.
This is an attempt at proving that all Klein bottle-graphs can be embedded on a book with a bounded number of pages.

\subsection{Irreducible triangulations}
Recall a \textit{triangulation} of a surface $\Sigma$ is a simple graph embedded on $\Sigma$ such that every face is bounded by a triangle. A triangulation $G$ of $\Sigma$ is \textit{reducible} if there exists an edge $vw$ such that $G \setminus vw$ is a triangulation of $\Sigma$. A triangulation $G$ of $\Sigma$ is \textit{irreducible} if there exists no such edge. A large corpus of work on graphs on surfaces depends on understanding the properties of irreducible triangulations. An important fact to know about triangulations, and what makes them computationally useful, is this:

\begin{theorem}[\textcite{barnetteAll2manifoldsHave1989}]
    Every surface $\Sigma$ has a finite number of irreducible triangulations.
\end{theorem}

. An upper bound on the size of an irreducible triangulation is given by \textcite{joretIrreducibleTriangulationsAre2010}.

\begin{theorem}\textcite{joretIrreducibleTriangulationsAre2010}
    Every irreducible triangulation of a surface with Euler genus $g \geq 1$ has at most $13g - 4$ vertices. 
\end{theorem}

Irreducible triangulations are useful in proving properties on graphs embedded on a surface. From \todo{insert your proof here}, every graph $G$ embedded on a surface $\Sigma$ is a subgraph of a triangulation $G'$ embedded on $\Sigma$. Furthermore, every triangulation of a surface $\Sigma$ can be edge contracted to one of finitely many irreducible triangulations. Note that the choice of edge contraction matters in determining an irreducible triangulation. If a list of irreducible triangulations is known, then it is possible to use an argument by induction under edge contraction to prove that every embedding of a graph has this property. We will use this strategy to show that every graph embedded on a Klein bottle has a book-embedding with a bounded number of pages. 

In the case of the Klein bottle, the full list of irreducible triangulations of the Klein bottle is known. 
\begin{theorem}[\textcite{sulankeNoteIrreducibleTriangulations2006}]
    There are 29 irreducible triangulations of the Klein bottle. 
\end{theorem} 
In a previous paper, \textcite{lawrencenkoIrreducibleTriangulationsKlein1997} claimed that there were 25. However, Sulanke found 4 more graphs, a modification of one of the original irreducible triangulations that was missed. 

\begin{theorem}
    Every Klein-Bottle graph has a book-embedding with 500 pages.
\end{theorem}

To prove this theorem, we need some other lemmas. Let $G$ be a triangulation of the Klein bottle. Let a \textit{$\phi$-structure} of $G$ be a decomposition of the vertices and edges of $G$ with the following properties:
\begin{itemize}
    \item $G$ has vertex set $A \cup \{v\}$ where $A$ is a set of vertices and $v$ is a single vertex in $G$,
    \item $G_P$ is a spanning planar subgraph of $G[A]$ which is edge-maximal, meaning that adding any edge not in $G_P$ to $G_P$ breaks the planarity condition,
    \item $G_P$ has a boundary cycle $B$,
    \item There exists a noncontractible cycle $C$ in $G$ such that $\{x, y\} = C \cap B$ and the edge $xy$ is the only edge in $C - G_P$. 
    \item Six edges from $v$ to $G$ divide $B$ into regions such that all edges that pass through one region preserve their orientation, and edges that pass through another region pass through a crosscap. 
    \item The boundary can be partitioned paths $a, b, c, d, e, f$ like in \cref{fig:phiembedding}. 
\end{itemize}

A description of the figure is in \cref{fig:phiembedding}.

\begin{figure}[h]
    \centering
    \includesvg{figures/kleinbottlegraph.svg}
    \caption{$\phi$-embedding on a Klein Bottle graph}\label{fig:phiembedding}
\end{figure}

A description of this arrangement is seen on the Klein bottle's fundamental polygon. 
\begin{claim}
    Every Klein-bottle graph $G$ with the $\phi$-structure can be embedded on $11$ pages. 
\end{claim}
\begin{proof}
    Take a $\phi$-structure on $G$. Then there is a book-embedding of $G_P$ on eight pages, using Yannakakis's algorithm on $B_1$ and $B_2$. However, this book-embedding has the property that sides $b$ and $e$ are oriented the opposite way, and $a, c$ and $d, f$ are oriented correctly. $b$ and $e$ go on another book, and $d, a$ and $c, f$ go on another book. Then add every edge adjacent to $v$ on its own page. Then this is a book-embedding on $11$ pages. 
\end{proof}

\begin{lemma}
    Every irreducible Klein bottle graph has a $\phi$-structure. 
\end{lemma}

\begin{proof}
    This is done by looking at every irreducible Klein bottle graph and identifying a $\phi$-structure. This is handled by case analysis below.
    \todo{are you sure about this? try finding another structure that's more amenable}
\end{proof}

\begin{lemma}
    Every Klein-bottle graph has a $\phi$-structure. 
\end{lemma}

\begin{proof}
    Proof by induction on the number of edge contractions. Let $G$ be a Klein Bottle-graph. If $G$ is irreducible, then there exists a $\phi$-structure of $G$. 
\end{proof}

\section{Consequences}
We will finish by discussing the importance of \cref{conj:bded_had_pn}. We discuss some consequences of \cref{conj:bded_had_pn} if it is proven. 
A family of graphs $\mathcal{F}$ is \textit{proper} if $\mathcal{F}$ is not the set of all graphs. 

\begin{lemma}\label{lem:minor-closed-Kt}
    Every proper minor-closed graph family $\mathcal{F}$ has a fixed $t$ such that $\mathcal{F}$ is $K_t$-minor free. 
\end{lemma}

\begin{proof}
    From \textcite{robertsonGraphMinorsXX2004} Graph Minor Theorem, every proper minor-closed graph family has a finite forbidden minor characterisation. Let $\mathcal{H}$ be the finite forbidden minor characterisation of $\mathcal{F}$. Let $H \in \mathcal{H}$ be the forbidden minor with the largest number of vertices, say $|V(H)| = t$. Then $\mathcal{F}$ is also $K_t$-minor free, as if $K_t$ appears as a minor in $G$ in $\mathcal{F}$, then the subgraph $H$ also appears in $G$. As $H$ is the largest forbidden minor, all other graphs in $\mathcal{H}$ are also minors of $K_t$. Therefore, every graph $G$ in $\mathcal{F}$ is $K_t$-minor free.
\end{proof}

\begin{lemma}\label{lem:Minor-Closed_Pagenumber}
    If \cref{conj:bded_had_pn} is true, then every proper minor-closed graph family can be embedded on a bounded number of pages.
\end{lemma}
\begin{proof}
    From \cref{lem:minor-closed-Kt}, every proper minor-closed graph family is also $K_t$-minor free. Therefore, every graph in a proper minor-closed graph family can be embedded in a bounded number of pages.
\end{proof}

This does not say that $\pn(H) \leq \pn(G)$ when $H$ is a minor of $G$. Subdivide $K_n$ $n$ times. From \textcite{atneosenEmbeddabilityCompactaNbooks}, the subdivision of $K_n$ can be embedded on three pages. But $K_n$ is a minor of its subdivision, and from \cref{thm:Pagenumber_Complete_Graph}, $\pn(K_n) = \lceil \frac{n}{2} \rceil$. Therefore, pagenumber is not a minor-closed property. 

This will imply that linklessly embeddable graphs or knotlessly embeddable graphs have bounded pagenumber. 

\subsection{Similarities with Blankenship's PhD}
As stated previously, we did not use Blankenship's PhD at all during the thesis. However, there are some peculiar similarities between her thesis and our work. 
Blankenship also uses \textcite{heathPagenumberGenusGraphs1992} to do a planar-nonplanar decomposition. Apex vertices are handled the same. However, Blankenship deals with vortices very differently. They use a ``cap edges'' solution to deal with vortices to embed a graph. This is much different to our approach using monochromatic paths on vortices, and using a tree-decomposition. 
Blankenship also uses a similar theme of having some vertices being moved to the front of a book-embedding, with extra pages needed. However, her lemma was much simpler than Robert's theorem. Note that her proof relied on \textcite{heathPagenumberGenusGraphs1992} embedding graphs on surfaces of genus $g$, which we have shown in this paper to be a flawed approach. 


\chapter{Conclusion}\label{chap:conclusion}
We conclude this report by explaining our current progress and our future plans. 
The main bulk of the report is defining some important concepts and discussing some important results associated with them.
The main important result explained is the Graph Minor Structure Theorem~\cite{robertsonGraphMinorsXVI2003} and its application in our particular problem involving \(K_t\)-minor free graphs. We apply the Graph Minor Structure Theorem to the problem and explain some partial results.
We also go through some other important results in \(K_t\)-minor free graphs and pagenumbers, including Heath and Istrail's \cite{heathPagenumberGenusGraphs1992} paper on graphs on surfaces and bounding the pagenumber.
The main result we have attempted to show is the case where graphs are almost embedded on an orientable surface. We can show that if a graph is almost-embeddable on an orientable surface, then the pagenumber of the graph is bounded.
What remains to show is if a graph is almost-embeddable on a non-orientable surface. This will be much more difficult as we do not have the same tools for orientable surfaces. Heath and Istrail's proof for non-orientable surface page numbers is a lot more complex than the orientable case, and we will need to look at that proof in depth and extend their results for our case.
\todo{Add the conjecture we want to prove!} 
\printbibliography{}
\end{document}
