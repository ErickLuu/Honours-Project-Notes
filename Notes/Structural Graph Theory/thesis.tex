\documentclass[]{report}
\usepackage[margin = 1in]{geometry}

\usepackage{amsmath}
\usepackage{amssymb}
\usepackage{amsthm}

\usepackage[english]{babel}
\usepackage{url}
\usepackage{todonotes}
\usepackage{csquotes}

\usepackage{parskip}

\usepackage{hyperref}
\usepackage[noabbrev, capitalise]{cleveref}

\usepackage{tikz}
\usetikzlibrary{graphs,graphdrawing, graphs.standard}
\usegdlibrary{trees}
\usepackage{svg}
\svgsetup{inkscapeexe="C:/Program Files/Inkscape/bin/inkscape.exe"}

\usepackage{graphicx}

\usepackage[style = numeric,
isbn=false,
url=false,
eprint = false,
maxbibnames=99
]{biblatex}
\renewbibmacro{in:}{}
\DeclareSourcemap{
	\maps[datatype=bibtex]{
		\map{
			\step[fieldset=url, null]
			\step[fieldset=extra, null]
			\step[fieldset=urldate, null]
			\step[fieldset=location, null]
		}
	}
}
\AtEveryBibitem{%
	\clearfield{day}%
	\clearfield{month}%
	\clearfield{endday}%
	\clearfield{endmonth}%
}

\addbibresource{Book-Embeddings.bib}
% Commands
\newcommand{\tree}{\mathcal{T}}
\DeclareMathOperator{\tw}{tw}
\DeclareMathOperator{\had}{had}
\DeclareMathOperator{\pw}{pw}
\DeclareMathOperator{\td}{td}
\DeclareMathOperator{\pn}{pn}
% Environments

\newtheorem{theorem}{Theorem}
\newtheorem{proposition}[theorem]{Proposition}
\newtheorem{corollary}[theorem]{Corollary}
\newtheorem{lemma}[theorem]{Lemma}
\newtheorem{definition}[theorem]{Definition}
\newtheorem{conjecture}[theorem]{Conjecture}

\theoremstyle{definition}
\newtheorem{example}[theorem]{Example}

\numberwithin{theorem}{section}
\numberwithin{equation}{section}

%includeonly
\includeonly{
chapters/introduction,
chapters/definitions,
chapters/knownresults,
	chapters/knownresults/gmst,
	chapters/knownresults/robertsproof,
	chapters/knownresults/heathandistrail,
chapters/proofattempt
}

%opening
\title{Towards a proof that all \(K_t\)-minor-free graphs have bounded pagenumber}
\author{Eric Luu}

\begin{document}

\maketitle
In this report we outline our current progress in proving that \(K_t\)-minor free graphs have bounded pagenumber. Proving this bound will connect two important concepts in structural graph theory that have been studied extensively for the past 40 years. We will outline the most important theorem in structural graph theory related to \(K_t\)-minor free graphs, the Graph Minor Structure Theorem. We will also introduce some other proofs which will be of use proving this bounded pagenumber conjecture.
\todo{Move basic definitions to start before the main body of introduction}

\listoftodos

\setcounter{tocdepth}{1}
\tableofcontents

% !TEX root = ./thesis.tex
\chapter{Introduction}\label{sec:introduction}
Structural graph theory is a fundamental topic in graph theory. Many results from structural graph theory decompose graphs, or families of graphs, into smaller graphs with bounded parameters. One of the most important theorems in structural graph theory is Robertson and Seymour's Graph Minor Theorem \cite{robertsonGraphMinorsXX2004} which states that every proper minor-closed graph family is characterised by a finite set of forbidden minors.
A \textit{book-embedding} of a graph $G$ arranges the vertices of $G$ on the ``spine'' of a book and arranges the edges of $G$ on ``pages'' of a book. The \textit{pagenumber} of a graph \(G\) is the minimum number of pages necessary in a book-embedding of \(G\). The concept of the \textit{pagenumber} of a graph was introduced by Ollmann \cite{ollmannBookThicknessVarious1973} in the context of VLSI design and integrated circuitry. 
The driving question of this report is the following:
\begin{conjecture}\label{conj:bded_had_pn}
	Suppose a graph $G$ is $K_t$-minor-free. Then the pagenumber of \(G\) is bounded by a function of \(t\).
\end{conjecture}
In her PhD thesis, \textcite{Blankenship-PhD03} claimed to prove \cref{conj:bded_had_pn}. However, this result has not been published and has not been independently verified. We aim to fill this gap in knowledge. 
We begin this report by discussing the relevant literature. In their seminal work, \textcite{robertsonGraphMinorsXVI2003} proved the Graph Minor Theorem. In their papers, they introduced many important concepts and theorems that are still used to this day. The result we use is the Graph Minor Structure Theorem, which coarsely describes $K_t$-minor free graphs.

Robertson and Seymour showed that graphs with no \(K_t\) minor can be built from smaller building blocks. This is a rough overview of the building blocks. We first start with a graph \(G\) embedded on a genus \(g\) surface. Then we add on \(p\) \textit{vortices} to \(G\), with \textit{pathwidth} at most \(k\). Then we add on \(a\) \textit{apex vertices} to \(G\). We say that \(G\) is \((g, p, k, a)\)-\textit{almost embeddable}. Robertson and Seymour \cite{robertsonGraphMinorsXVI2003} proved that all graphs with no \(K_t\) minor has a \textit{tree-decomposition} where every \textit{torso} is a \((g, p, k, a)\) almost-embeddable graph, with \((g, p, k, a)\) bounded by a function of \(t\).

We have good reason to believe \cref{conj:bded_had_pn} is true. Firstly, \textcite{yannakakisEmbeddingPlanarGraphs1989} showed that all planar graphs can be embedded on 4 pages.\ \textcite{malitzGenusGraphsHave1994} then showed that all graphs of Euler genus $g$ can be embedded on $O(\sqrt{g})$ pages. Finally, graphs with bounded treewidth have bounded pagenumber, from \textcite{ganleyPagenumberTrees2001} and \textcite{dujmovicGraphTreewidthGeometric2007}.
We discuss some relevant papers that are used to prove \cref{conj:bded_had_pn}.
We aim to solve this question using the Graph Minor Structure Theorem \cite{robertsonGraphMinorsXVI2003}, which describes the structure of graphs that do not contain a \(K_t\) minor. 
We have some useful results that can be paired with the Graph Minor Structure Theorem to prove \cref{conj:bded_had_pn}.
\begin{itemize}
	\item From \textcite{heathPagenumberGenusGraphs1992}, all graphs of bounded genus have bounded pagenumber.
	\item From \textcite{ganleyPagenumberTrees2001}, and \textcite{dujmovicGraphTreewidthGeometric2007}, all graphs of bounded treewidth have bounded pagenumber.
	\item From \textcite{hickingbothamStackNumberCliqueSum2023}, if a graph \(G\) has a \textit{tree-decomposition} where every \textit{torso} has bounded pagenumber, then \(G\) has bounded pagenumber.
\end{itemize}
These results individually show that the constituent ingredients of the Graph Minor Structure Theorem have bounded pagenumber. Once stating the requisite technology, we may proceed with the main aim of the thesis of proving \cref{conj:bded_had_pn}. 
The biggest hurdle is showing that adding vortices on surfaces will not blow up the pagenumber. To address this issue, we introduce a new concept when considering faces on surfaces with a fixed book-embedding. 

This honours project has two goals. The first goal is to investigate and learn more about structural graph theory. We will discuss some important machinery in structural graph theory, the main ones being the Graph Minor Theorem and the Graph Minor Structure Theorem. The second goal is to address an open problem within this field. To this end, an entire chapter building on the techniques discussed in previous chapters discusses this open proble. 

The following list is how the rest of the report is laid out. 
\begin{itemize}
	\item \cref{chap:Definitions} contains definitions and concepts that will be used throughout the rest of the report. Some of these concepts are part of any undergraduate graph theory unit. Some other concepts, like book-embeddings and treewidth, are unlikely to appear in an undergraduate graph theory unit.
	\item \cref{chap:Known results} discusses some known results from graph theory, including the Graph Minor Structure Theorem. We discuss some proofs related to bounded pagenumber that can be used to prove \cref{conj:bded_had_pn}. The results we discuss are the Graph Minor Structure Theorem itself, a result from \textcite{heathPagenumberGenusGraphs1992}, a result from \textcite{ganleyPagenumberTrees2001} and a result from \textcite{hickingbothamStackNumberCliqueSum2023}. \cref{chap:Definitions} and \cref{chap:Known results} form the literature review section of the report.

	\item \cref{chap:Proving_The_Theorem} is an attempt at a proof to \cref{conj:bded_had_pn}. The main bulk of the argument is showing that the construction given by the Graph Minor Structure Theorem can be used to bound the pagenumber of the graph. Concepts and constructions in the literature introduced in the previous sections is used to show this result. 
\end{itemize}

Readers are expected to have at least an undergraduate understanding in graph theory.

\chapter{Definitions}\label{chap:Definitions}
This chapter presents important notions in structural graph theory.
\begin{itemize}
	\item \cref{sec: Basic definitions} are basic graph theory definitions and notations. These include \textit{graphs}, \textit{subgraphs} and types of subgraphs, including \textit{induced subgraphs,} \textit{spanning subgraphs}, \textit{paths}, \textit{cycles} and \textit{Hamiltonian cycles}. 
	\item \cref{sec:Planar graphs} discusses \textit{planar graphs}. The concepts of graphs \textit{embedded} on $\mathbb{R}^2$, \textit{faces} and \textit{Euler's formula}.
	\item \cref{sec:Graph Minors} discusses \textit{graph minors}.
	\item \cref{sec:Book Embedding} defines book-embeddings and discusses its importance.
	\item \cref{sec:treewidth} defines \textit{treewidth}.
	\item \cref{sec:Pathwidth} defines a related notion \textit{pathwidth}.
\end{itemize}
These definitions will be useful in discussing the Graph Minor Structure Theorem and our conjecture.
\section{Basic definitions}\label{sec: Basic definitions}
A graph $G$ is a pair of sets; a vertex set $V(G)$ and an edge set $E(G)$. $E(G)$ is a set that contains two-element subsets of $V(G)$. An edge $ \{v, w\}$ \textit{joins} vertices $v$ and $w$. In this paper, all graphs are simple unless stated. A graph is \textit{simple} if all edges join two distinct vertices and there is at most one edge between any two vertices. Furthermore, all graphs $G$ are finite, so $|V(G)| < \infty$. The graph with all possible edges on $n$ vertices is the complete graph $K_n$.

Let $G$ be a graph. A \textit{subgraph} $H$ in $G$ is a graph with vertex set $V(H) \subseteq V(G)$ and edge set $E(H)$ with the property that if $vw$ is an edge in $E(H)$, then $vw$ is an edge in $E(G)$.
Let $G$ be a graph and let $S$ be a non-empty subset of the vertex set of $G$. The \textit{induced subgraph} of $S$ in $G$ is the graph $G[S]$ with vertex set $S$ and edge set containing precisely all edges in $G$ incident to two vertices in $S$. Removing a set of vertices $S \subseteq V(G)$ from $G$ forms the induced subgraph $G - S := G[V(G) - S]$. 
$H$ is a \textit{spanning subgraph} of $G$ if $H$ is a subgraph of $G$ and $V(H) = V(G)$. 
The neighborhood of a set of vertices $A \subseteq V(G)$ are precisely all vertices that are adjacent to a vertex in $A$ and not in $A$ and is denoted as $N_G(A)$. Below are a list of useful subgraphs.

\begin{itemize}
	\item A \textit{path} in a graph \(G\) is a sequence of edges \(e_1, e_2, \ldots, e_{\ell- 1}\) which join a sequence of vertices \(v_1, v_2, \ldots, v_{\ell}\) such that \(e_i = v_i v_{i + 1}\), and all the vertices are distinct.
	\item A \textit{cycle} \(C\) in a graph \(G\) is a sequence of edges \(e_1, e_2, \ldots, e_{\ell}\) which join a sequence of distinct vertices \(v_1, v_2, \ldots, v_{\ell}\) such that \(e_i = v_i v_{i + 1}\) for \(1 \leq i \leq \ell - 1\) and \(e_\ell = v_\ell v_1\).
	\item Let $G$ be a graph and $C$ be a cycle in $G$. A \textit{chord} in $C$ is an edge $e$ that joins two vertices in $C$ that are not adjacent in $C$. 
	\item A \textit{Hamiltonian cycle} in a graph \(G\) is a cycle \(C\) such that all the vertices in \(G\) appear in \(C\). If a graph $G$ has a Hamiltonian cycle, then $G$ is a Hamiltonian graph. 
\end{itemize}

The disjoint union of two graphs $G_1$ and $G_2$ is the graph $G_1 \sqcup G_2$ with vertex set $V(G_1) \sqcup V(G_2)$ and edge set $V(G_1) \sqcup V(G_2)$. 

A graph $G$ is \textit{connected} if between any two distinct vertices $x$ and $y$ there exists a path which starts and ends at $x$ and $y$ respectively. 
A connected graph \(G\) is \textit{\(k\)-connected} if \(G\) has more than \(k\) vertices and for any vertex set $S \subseteq V(G)$ where $|S| \leq k - 1$ it holds that $G - S$ is connected.\ \textit{Biconnected} graphs are $2$-connected graphs. 

Throughout this report, the set $\lbrace 1\ldots n \rbrace$ is notated as $[n]$. 
A graph \(G\) is \(k\)-colourable if there exists a function \(f: V(G) \rightarrow [k]\) such that if $f(v) = f(w)$, then $v$ and $w$ do not share an edge. The \textit{chromatic number} \(\chi(G)\) is the smallest \(k\) such that \(G\) is \(k\)-colourable.
$G$ is $k$-degenerate if every subgraph $H$ of $G$ has a vertex of degree at most $k$. Every $k$-degenerate graph is $(k + 1)$-colourable. 

Menger's theorem \cite{mengerZurAllgemeinenKurventheorie1927} is an important theorem which is used throughout the report.
Let \(G\) be a graph and \(A, B \subseteq V(G)\). An \(AB\)-path is a path in \(G\) which starts in \(A\) and ends in \(B\) with no internal vertices in \(A \cup B\). An \(AB\)-connector is a set of disjoint \(AB\)-paths. An \(AB\)-separator is a set \(S \subseteq V(G)\) such that \(G - S\) contains no \(AB\)-path. Then:
\begin{theorem}[Menger's theorem]\label{thm:Menger}
	Let $G$ be a graph and let $A, B \subseteq V(G)$. Then the size of the smallest \(AB\)-separator of \(G\) is equal to the size of the largest \(AB\)-connector.
\end{theorem}
Now take two distinct vertices \(x, y\). Let \(A = N_G(x) \cup \{x\} \) and \(B = N_G(y) \cup \{y\} \). Then \cref{thm:Menger} implies that:
\begin{theorem}[Menger's theorem, vertex-connectivity version]\label{thm:Menger_Vertex}
	A graph \(G\) is \(k\)-connected if and only if for any two distinct vertices, there are at least \(k\) internally disjoint paths between the two vertices.
\end{theorem}
As a corollary, all Hamiltonian graphs are biconnected. For any two distinct vertices in a Hamiltonian graph, there are two internally disjoint paths by traversing the Hamiltonian cycle.

\section{Planar graphs}\label{sec:Planar graphs}
A graph \(G\) is \textit{planar} if \(G\) can be drawn in the Euclidean plane \( \mathbb{R}^2 \) such that no two edges cross. A drawing of $G$ in $\mathbb{R}^2$ is referred to as an \textit{embedding} of $G$ in $\mathbb{R}^2$. If \(G\) is embedded in \(\mathbb{R}^2 \), then we can topologically remove the set $G$ from $\mathbb{R}^2$. The connected components in $\mathbb{R}^2 - G$ are called \textit{faces}. The \textit{outerface} is the unbounded face in $\mathbb{R}^2$, meaning that the outerface goes to infinity. The \textit{internal faces} of $G$ are all the faces which are not the outerface. A set of vertices $S$ in $V(G)$ \textit{lie} on a face $F$, or \textit{bound} $F$ if $S$ is in the closure of $F$. A set of edges $T$ in $V(G)$ also bound a face $F$ if $T$ is in the closure of $F$. If an edge $e$ bounds a face $F$, then $e$ \textit{touches} $F$. \(G\) is \textit{outerplanar} if \(G\) is planar and there exists an embedding such that all vertices in \(G\) lie on the outerface.
Let \(F(G)\) be the set of faces of \(G\) embedded on \(\mathbb{R}^2\). Then
\begin{theorem}[Euler's formula]\label{thm:Euler_planar}
	For all connected graphs $G$,
	\begin{equation}
		|V(G)| - |E(G)| + |F(G)| = 2.
	\end{equation}
\end{theorem}

We can use this result to bound the number of edges in an outerplanar graph.
\begin{theorem}\label{thm:outerplanar_bound}
	If \(G\) is an outerplanar graph with \(n\) vertices and \(m\) edges, then \(m \leq 2n - 3\).
\end{theorem}

\begin{proof}
	Suppose \(G\) is \textit{maximal outerplanar}, meaning adding any edge will break the outerplanar property. Let there be \(f\) internal faces in $G$. Then the outerface has \(n\) edges on the boundary as every vertex in $G$ is on the boundary of the outerface. Each internal face will have exactly \(3\) edges on the boundary. However, each edge is touching exactly two distinct faces. By counting the number of edges on every face,
	\begin{equation*}
		3 f + n = 2m.
	\end{equation*}
	Combining with \cref{thm:Euler_planar},
	\begin{equation*}
		n - m + (f + 1) = 2
	\end{equation*}
	we have, after some rearrangement,
	\begin{equation*}
		2n = 3 + m.
	\end{equation*}
	Therefore, \(m = 2n - 3\). Since every outerplanar graph is a spanning subgraph of a maximal planar graph, \(m \leq 2n - 3\).
\end{proof}

\begin{theorem}\label{thm:planar_graph_edge_bound}
	Let $G$ be a planar graph with $n$ vertices and $m$ edges. Then $m \leq 3n - 6$.
\end{theorem}
\begin{proof}
	Suppose $G$ is embedded on $\mathbb{R}^2$. Then $G$ has $f$ faces, where every edge touches exactly two faces. Every face touches at least three edges. Therefore, $3f \leq 2m$. From Euler's formula, $n - m + f = 2$, so $m \leq 3n - 6$. 
\end{proof}
\section{Graph minors}\label{sec:Graph Minors}
A graph \(H\) is a \textit{minor} of a graph \(G\) if a graph isomorphic to \(H\) can be obtained from \(G\) by deleting vertices, deleting edges, and \textit{contracting} edges. Let $G$ be a graph and let $uv$ be an edge in $E(G)$. To \textit{contract} \(uv\), we delete both \(u\) and \(v\) and create a new vertex \(w\) with neighbourhood \(N(w) = N_G(u) \cup N_G(v)\). The graph obtained after contracting the edge \(uv\) in $G$ is written as \(G\setminus uv\).
The statement ``\(H\) is a minor of \(G\)'' is written as \(H \leq G\). A graph \(G\) is \textit{\(H\)-minor-free} if $H$ is not a minor of $G$. A family of graphs \(\mathcal{F}\) is \textit{minor-closed} if when $G$ is in \(\mathcal{F}\) and \(H \leq G\), then $H$ is in \(\mathcal{F}\).
An example of a minor-closed class is the class of planar graphs.
An important class of graph families are the \(K_t\)-minor free graphs. For a graph \(G\), we define \(\had(G)\) to be the largest \(t\) such that \(K_t\) is a minor of \(G\). This is named after Hugo Hadwiger and his most famous conjecture.

\begin{conjecture}[Hadwiger's conjecture]
	For all graphs \(G\), \(\chi(G) \leq \had(G)\)\cite{hadwigerUeberKlassifikationStreckenkomplexe1943}.
\end{conjecture}
Much work has been done on solving Hadwiger's conjecture, with a document by \textcite{seymourHadwigerConjecture2016} on the latest progress. However, it remains unsolved. 

A \textit{model} of a graph \(H\) in a graph \(G\) is a function $\rho$ which assigns to \(H\) vertex-disjoint connected subgraphs of \(G\). If $uv$ is an edge in \(E(H)\), then some edge in \(G\) joins the two subgraphs \(\rho(u)\) and \(\rho(v)\). A description of a model is in \cref{fig:model_of_P5}.
\begin{figure}[h!]\label{fig:model_of_P5}
	\centering
	\includesvg[width = 0.8\textwidth]{figures/model.svg}
	\caption{Illustration of a model $H$ in a graph $G$. The coloured boxes are the connected subgraphs contracted to a single vertex on the right.}
\end{figure}

\begin{theorem}
	\(H\) is a model of \(G\) if and only if $H$ is a minor of $G$.
\end{theorem}

\begin{proof}
	From \textcite{norinMath599GraphMinors2017}. Suppose \(H\) is a model of \(G\). Then for all \(x\) in \(V(H)\), contract \(\rho(x)\) in \(G\) to a single vertex. This is a well-defined operation as the image $\rho(x)$ is connected and disjoint from all $\rho(y)$ where $y$ is a distinct vertex in $H$. Then delete edges to form \(H\).

	Suppose \(H \leq G\). Use induction to show that \(H\) has a model in \(G\). Suppose \(H\) is obtained from \(G\) by contraction operations only. We can assume this by taking a subgraph of \(G\) if necessary. Let \(uv\) be the first contracted edge and let \(G' := G \setminus uv\). Let \(w\) be the vertex obtained after contracting \(uv\). Then by induction, there is a model \(\rho\) of \(H\) in \(G'\). Then find $x \in V(H)$ such that $w \in V(\rho(x))$. If there is no such $x$, then it is obvious that $\rho$ is a model of $H$ in $G$. Otherwise, 
	delete \(w\) from \(V(\rho(x)) \) and add $u, v$ to $V(\rho(x))$, the edge $uv$, and the edges from $u$ and $v$ to the neighbours in $w$ in $\rho(x)$. Then this is a model of \(H\) in \(G\). 
\end{proof}

 Much of structural graph theory involves graph minors in some way. Many of the theorems that we will discuss throughout this report discuss graph minors. 

\section{Book embedding}\label{sec:Book Embedding}
A \textit{book} with \(k\) \textit{pages} consists of \(k\) half-planes glued together on a common boundary. We refer to the boundary as the \textit{spine}, and we refer to the individual half-planes as \textit{pages}. In topology, these are referred to as \textit{fans} of half-planes.\ \textcite{persingerSubsetsNbooksE31966,atneosenOnedimensionalNleavedContinua1972} described fans in the 1960s.
A \textit{book-embedding} of a graph \(G\) is an embedding of \(G\) on a book. We place the vertices of \(G\) on the spine, and we place each edge on a single page such that no two edges cross.
The \textit{pagenumber} of a graph \(G\) is the minimum number of pages required to embed \(G\) on a book. This is also referred to as \textit{book-thickness}, or \textit{stack-number}. An embedding of $K_5$ in three pages is in \cref{fig:book-embedding}.
\begin{figure}[h!]\label{fig:book-embedding}
	\centering
	\includesvg[height = 0.5\textheight]{figures/3page_K5.svg}
	\caption{Book-embedding of $K_5$ on three pages. Image by \textcite{eppsteinBookEmbedding2014}}
\end{figure}
\par
There is an equivalent combinatorial definition. A \textit{book embedding} of a graph \(G\) is an arrangement of the vertices of \(G\) in a total ordering \(v_1 < v_2 < \cdots < v_n\). We then \textit{colour} the edges \(E(G)\) such that if there are vertices with ordering \(v_a < v_b < v_c < v_d\) and edges \(v_a v_c\) and \(v_b v_d\) in $E(G)$, then $v_a v_c$ and $v_b v_d$ are assigned different colours.
We refer to the total ordering of \(V(G)\) in the book embedding as \((<)\) or as \((\leq)\). For a book-embedding \((v_1, v_2, \ldots, v_{|G|})\), we refer to the edges \( \left\{ v_1 v_2, v_2 v_3, \ldots, v_{|G| - 1}v_{|G|}, v_{|G|}v_{1} \right\} \) as \textit{spine edges}.
We may use a \textit{circular ordering} of the vertices rather than a linear ordering. This means that we order the vertices in a circle rather than on a straight line. The book-embedding of a circular ordering is exactly the same as for a linear ordering, and we can convert between a circular and linear ordering by choosing a vertex to be at the start of the sequence.
Book-embeddings were introduced by \textcite{kainenRecentResultsTopological1974, ollmannBookThicknessVarious1973} in the 1970s. A paper by \textcite{bernhartBookThicknessGraph1979} calculated the book-thickness of complete and bipartite graphs.
\begin{lemma}\label{lem:Pagenumber_1}
	A graph \(G\) can be embedded on a single page if and only if \(G\) is an outerplanar graph.
\end{lemma}
\begin{proof}
	Suppose $G$ is outerplanar, and embedded in $\mathbb{R}^2$. Then we choose a single vertex $v_0$, and traverse anticlockwise around the outerface to form an ordering. If a vertex $v_i$ appears more than once, then add $v_i$ the first time we see it in the traversal and no other times. Then this is a one-page book embedding, as when $v_a < v_b < v_c < v_d$ and edges $v_a v_c$, $v_b v_d$ in $G$, then $v_a v_c$ or $v_b v_d$ have to lie on the outerface, which breaks the condition that $G$ is outerplanar. This is because either $v_b$ or $v_c$ is not on the outerface. If $G$ has a $1$-page book-embedding, then embedding the page in $\mathbb{R}^2$ through the inclusion map is an outerplanar embedding of $G$. 
\end{proof}
\begin{lemma}\label{lem:Pagenumber_2}
	A graph \(G\) can be embedded on two pages if and only if \(G\) is a subgraph of a planar graph with a Hamiltonian cycle.
\end{lemma}

\begin{proof}
	Suppose $G$ is a subgraph of a planar graph $G'$ with a Hamiltonian cycle $C$. By the Jordan curve theorem, $\mathbb{R}^2 - C$ has two connected components $F_1$ and $F_2$. Choose a vertex $x_0$ and order the vertices with respect to the Hamiltonian cycle $C$ where $x_0$ is the first vertex. Give edges on $C$ colour $1$. For all edges which are a chord of $C$ that lies in $F_1$, give these edges colour $1$. For all edges which are a chord of $C$ that lies in $F_2$, give these edges colour $2$. This is a book-embedding of $G'$ on two pages. 
	\par
	Suppose $G$ has pagenumber $2$, and embedded in a book with two pages. Add all remaining spine edges to one of the pages. Then embed the two pages in $\mathbb{R}^2$ through the homeomorphism of two pages to $\mathbb{R}^2$, by flipping one page over. Then this is a planar graph with a Hamiltonian cycle, so $G$ is a subgraph of a graph with a Hamiltonian cycle.
\end{proof}
\subsection{Properties of pagenumber}\label{ssec:Related_Properties}
\cref{lem:Edge_Bound} comes from \textcite{bernhartBookThicknessGraph1979}.
\begin{lemma}\label{lem:Edge_Bound}
	If an \(n\)-vertex graph \(G\) can be embedded on $k$ pages, then \(G\) has at most \(n + k(n-3)\) edges.
\end{lemma}
\begin{proof}
	Given a vertex ordering \(v_1 \leq v_2 \leq \cdots \leq v_n\), the spine edges can appear on any page. Furthermore we have there are at most \(n-3\) non-spine edges on each page. The maximum number of edges in an outerplanar graph is \(2n - 3\) from \cref{thm:outerplanar_bound}, but we remove the spine edges, with \(n\) edges on the spine. Therefore, \(m\), the number of edges, satisfies \(m \leq n + k (n - 3)\).
\end{proof}
\begin{theorem}[]\label{thm:Pagenumber_Complete_Graph}
	The complete graph $K_n$ has $\pn(K_n) = \lceil \frac{n}{2} \rceil$ when $n \geq 4$.
\end{theorem}
\begin{proof}
	To show the upper bound, arrange the vertices in any circular ordering $v_1 < v_2 < \cdots < v_n$. Then we colour edges $v_1 v_2, v_2 v_{n}, v_{n} v_{3}, v_{3} v_{n-1}, \ldots$ in a zigzag pattern. As an example, we refer to \cref{fig:k8 coloured with colours} to show what zig-zagging pattern we are referring to. We rotate this pattern $\lceil n/2 \rceil$ times. 
	\par
	To show the lower bound, we use \cref{lem:Edge_Bound}. \(K_n\) has \(n\) vertices and \(\binom{n}{2}\) edges. Then \(\pn(K_n) \geq \frac{\binom{n}{2} - n}{n - 3} = \frac{n}{2}\) when \(n \geq 4\). As \(\pn(K_n)\) is an integer, we take the ceiling of \(\frac{n}{2}\). This concludes the equality.
\end{proof}
\begin{figure}[ht]
	\caption{Circular embedding of \(K_8\) with 4 colours, the minimum possible.}
	\centering
	\usetikzlibrary{graphs,graphs.standard}

\tikz
	\graph[nodes={circle, draw}] { 
		subgraph K_n [n=8,clockwise,radius=2cm];
		
		{[induced path, edges= red] 1,2,8,3,7,4,6,5},
		{[induced path, edges= blue] 8,1,7,2,6,3,5,4},
		{[induced path, edges= green] 7,8,6,1,5,2,4,3},
		{[induced path, edges= yellow] 6,7,5,8,4,1,3,2},
 };\label{fig:k8 coloured with colours}
\end{figure}
The proof of \cref{thm:Pagenumber_Complete_Graph} is from \textcite{bernhartBookThicknessGraph1979}
This is an upper bound of any graph \(G\) with \(n\) vertices.
Therefore for any graph \(G\) on \(n\) vertices, \(n \geq 4\), \(\pn(G) \leq \lceil n/2 \rceil\). The next theorem bounds the chromatic number, from \textcite{bernhartBookThicknessGraph1979}
\begin{theorem}\label{thm:Colouring_Bound}
	For all graphs \(G\), \(\chi(G) \leq 2 \pn(G) + 2\).
\end{theorem}
\begin{proof}
	Let \(\pn(G) = k\) and suppose \(G\) has \(n\) vertices and \(m\) edges. Then the average degree of \(G\), \(d(G) = 2m/n\) by the handshaking lemma. So \(2\frac{m}{n} \leq 2 \frac{n + k(n-3)}{n} = 2 + 2k \frac{n-3}{n} < 2k + 2\). But this implies that \(G\) has a vertex of degree \(\leq 2k + 1\), and as if \(G'\) is a subgraph of \(G\), then \(G'\) also has \(\pn(G') \leq k\), thus \(G'\) has a vertex of degree at most \(2k + 1\). However, this implies \(G\) is \((2k + 1)\)-degenerate, thus \(\chi(G) \leq 2k + 2\).
\end{proof}

Let $G$ be a graph. A \textit{subdivision} of an edge $uv \in E(G)$ deletes $uv$ and adds a new vertex $w$ with edges $uw$ and $wv$. A graph subdivision of $G$ is to do this for all edges in $G$. A $k$-subdivision of $G$ is to subdivide each edge $k$ times in $G$, so the edge $e$ is replaced with a path $P$ of length $k$.\ \textcite{atneosenEmbeddabilityCompactaNbooks} proved that all graphs can be subdivided a finite number of times such the subdivision has pagenumber 3.\ \textcite{dujmovicLayoutsGraphSubdivisions2005} showed that the number of subdivision necessary is $O(\log\pn(G))$.

\begin{theorem}
	There exists a family of 2-colourable graphs with arbitrarily pagenumber.
\end{theorem}
\begin{proof}
	Let $G_n$ be the complete graph $K_n$ with every edge subdivided once. Then $G_n$ is bipartite, so is $2$-colourable. However, from \textcite{eppsteinSeparatingThicknessGeometric2002}, for every $t$ there exists an $n$ such that $G_n$ cannot be embedded in $t-1$ pages. This proof comes from \textit{Ramsey theory}, and explicit values of $n$ are difficult to find. 
\end{proof}

An \textit{expander graph} is a sparse, highly connected graph. Expander graphs share many properties with random graphs, but are constructed explicitly. One type of expander graph is a \textit{bipartite \varepsilon-expander}, where $\varepsilon \in (0, 1]$. We say a graph $G$ is a bipartite \varepsilon-expander if there exists a bipartition $ \{A, B\}$ of $V(G)$ such that $|A| = |B|$ and for all subsets $S \subset A$ where $|S| \leq \frac{|A|}{2}$, $|N(S)| \geq (1 + \varepsilon) |S|$. 
\textcite{dujmovicLayoutsExpanderGraphs2016} showed that all bipartite \varepsilon-expander graphs can be embedded in 3 pages. 


Book-embeddings of graphs were has applications in VLSI and processor designs, bioinformatics by \textcite{haslingerRNAStructuresPseudoknots1999}, and in graph drawings by \textcite{woodBoundedDegreeBook2002}. 
The project of finding upper and lower bounds of the pagenumber of planar graphs was started by \textcite{bernhartBookThicknessGraph1979} when they conjectured that planar graphs had unbounded pagenumber. However, \textcite{bussPagenumberPlanarGraphs1984} showed that all graphs could be embedded in nine pages, and \textcite{heathEmbeddingPlanarGraphs1984} brought down the number of needed pages to seven.\ \textcite{yannakakisEmbeddingPlanarGraphs1989} devised an algorithm to embed a graph in four pages. Yannakakis, in this paper, claimed that there exists planar graphs which cannot be embedded in three pages. However, his proof was incomplete and the full proof was left unpublished. In 2020, Yannanakis published his full proof \cite{yannakakisPlanarGraphsThat2020}. At around the same time, \textcite{kaufmannFourPagesAre2020} published the same lower bound.

\textcite{malitzGraphsEdgesHave1994} proved that any graph with $e$ edges has pagenumber $O(\sqrt{e})$. Additionally, he proved that random $d$-regular graphs $G$ with $n$ vertices have the property that $\pn(G) \in \Omega(\sqrt{d} n^{1/2 - 1/d})$. For random 3-regular graphs $G$ with $n$ vertices, $\pn(G) \in \Omega(n^{1/6})$. These constructions of $\Omega(n^d)$ pagenumber graphs are not explicit.\ \textcite{eppsteinThreeDimensionalGraphProducts2024} showed that $\pn(P_n \boxtimes P_n \boxtimes P_n) \in \Theta(n^{1/3})$. This is an explicit construction of a graph which has pagenumber in $\Theta(n^{d})$. 
\section{Treewidth}\label{sec:treewidth}
The \textit{treewidth} of a graph \(G\) measures how similar $G$ is to a forest.

\begin{definition}[Tree-decomposition]\label{def:tree-decomposition}
	A tree-decomposition \(\tree\) of a graph \(G\) is defined as a tree \(T\) with associated \textit{bags} \(\lbrace B_x : x \in V(T) \rbrace\) such that:
	\begin{itemize}
		\item $\bigcup_{x \in V(T)} B_x = V(G)$.
		\item For all \(v \in V(G)\), the subset of vertices \(\left\lbrace x \in V(T): v \in B_x \right\rbrace\) induces a connected subtree in \(V(T)\).
		\item For all edges \(vw \in E(G)\), there exists a bag \(B_x\) such that both \(v\) and \(w\) are in \(B_x\).
	\end{itemize}
\end{definition}
We refer to the vertices of the tree \(T\) as \textit{nodes}.
The \textit{width} of the tree decomposition \(\tree\) is defined as \(\max \lbrace |B_x| - 1 : x \in V(T) \rbrace\).
The treewidth of a graph \(G\), denoted as \(\tw(G)\), is defined to be the smallest width for all tree decompositions of the graph \(G\).

\begin{lemma}[Helly Property]\label{lem:Helly}
	Let \(T_1, \ldots, T_k\) be subtrees of a tree \(T\) such that for every pair of trees $T_i$, $T_j \in T_1, \ldots, T_k$, $V(T_i) \cap V(T_j) \neq \emptyset$. Then there exists a vertex which is common to all trees.
\end{lemma}
\begin{proof}
	This proof is by induction on the number of vertices of $T$. Suppose $T$ has a single vertex. Then it is obvious that the Helly property holds. By induction, suppose the Helly property holds for all trees with at most $n$ vertices. Suppose $T$ has $n + 1$ vertices and \(T_1, \ldots, T_k\) are subtrees which satisfy the property above. Let $v$ be a leaf vertex of $T$ with neighbour $w$. If one of the subtrees $T_i = \{v\}$, then by nonempty intersection, all trees contain $v$. $v$ is the common intersection. Otherwise, consider $T - v$ and the subtrees $(T_1 - v, \ldots, T_k - v)$. If $v \in T_i \cap T_j$, then as none of the subtrees is the single vertex $\{v\}$, $w \in T_i \cap T_j$. Therefore, $T_i - v \cap T_j - v$ is nonempty. By the induction hypothesis, $T - v$ has a vertex common to all $(T_1 - v, \ldots, T_k - v)$, so \(T_1, \ldots, T_k\) has a common vertex in $T$. 
\end{proof}

\begin{lemma}[Clique lemma]\label{lem:clique}
	For every graph $G$, in any tree-decomposition of \(G\), for every clique \(C\) in \(G\), there exists a node \(x \in V(T)\) such that \(C \subseteq B_x\).
\end{lemma}

\begin{proof}
	Let \(\tree\) be a tree-decomposition. Every vertex \(v\) induces a connected subtree in \(T\), call it \(T_v\). Then for any two vertices \(x, y\) in \(C\), \(T_x\) and \(T_y\) must intersect as the edge \(xy\) is inside a bag \(B_z\) corresponding to a node \(z\). Then by the Helly property, there exists a node \(v\) such that \(C \subseteq B_v\).
\end{proof}

\begin{corollary}\label{cor:complete_tw}
	\(\tw(K_n) = n-1\).
\end{corollary}
\begin{proof}
	By \cref{lem:clique}, $\tw(K_n)\geq n-1$. Placing all vertices of $K_n$ in a single bag is a tree-decomposition of width $n-1$. Therefore, $\tw(K_n) = n-1$. 
\end{proof}

\begin{theorem}\label{thm:tw_minor_closure}
	If \(H\) is a minor of \(G\), then \(\tw(H) \leq \tw(G)\).
\end{theorem}
\begin{proof}
	Let \((B_x : x \in V(T))\) be a tree-decomposition of \(G\). Remove an edge $e$ from $G$. Then \((B_x : x \in V(T))\) is a tree-decomposition of $G - e$. Remove a vertex $v$ from $G$. Then \((B_x - v : x \in V(T))\) is a tree-decompsition of $G - v$. Contract an edge $vw$ in $G$ to $u$. Define a new tree-decomposition $\tree'$ by relabeling \(v\) and \(w\) in all $B_x$ to \(u\). $\tree'$ is a valid tree-decomposition of $G \setminus uv$. The induced subtree of \(u\) is the union of the induced subtrees of \(v\) and \(w\), which is a subtree. As $v$ and $w$ share the edge $vw$, then there exists a bag $B_x$ such that $v, w \in B_x$. Every neighbor of \(v\) or \(w\) is a neighbor of \(u\). The edges in the neighborhood do not change. Notice that the size of each bag in each operation does not increase. Therefore, if $H \leq G$ by a series of vertex deletions, edge deletions, and edge contractions, the tree-decomposition \((B_x : x \in V(T))\) of $G$ can have the algorithm applied above to build a tree-decomposition of $H$ with width at most the tree-decomposition of $G$. Then by the minimality of the treewidth, \(\tw(H) \leq \tw(G)\). 
\end{proof}

\begin{lemma}\label{lem:treewidth_forest}
	\(\tw(G) = 1\) if and only if \(G\) is a forest.
\end{lemma}

\begin{proof}
	Suppose \(G\) is a tree. Root the graph \(G\) at the vertex \(r\). Then let \(T = G\) and \(B_x:= \lbrace x, p \rbrace\) where \(p\) is the parent of \(x\) and $x \neq r$. The bag \(B_r\) will just contain \(r\). Then all edges \(vw\) will be between parent \(v\) and child \(w\), so it will be in bag \(B_w\). Finally, the subgraph induced by vertex \(x\) in \(T\) will be \(B_x\) and the children of \(B_x\), which is a connected subtree.
	\par
	If \(G\) is a forest, then we perform this operation on every connected component of \(G\) and connect the roots to form a new tree. Then this tree is a tree-decomposition. This forms a tree-decomposition of width at most 1. An example is in \cref{fig:tree-treedecomp}.
	\par
	If \(G\) has a cycle \(C\), then $G$ has a $K_3$-minor. By \cref{cor:complete_tw}, $\tw(K_3) = 2$. By \cref{thm:tw_minor_closure}, $2 \leq \tw(G)$. Therefore, $G$ has treewidth at least two. 
	\begin{figure}[ht]
		\centering
		\usetikzlibrary {graphs,graphdrawing}
 \usegdlibrary {trees}
\tikz [subgraph text bottom=text centered,
subgraph nodes={font=\itshape}]
\graph [tree layout] {
	1 -> { 2 -> {3, 4}, 5 -> {6, 7 -> 8} };
	left [draw] // { b, c, d };
	right [draw] // { e, f, g, h};
};
		\usetikzlibrary {graphs,graphdrawing}
\usegdlibrary {trees}
\tikz 
\graph [tree layout] {
	1 -> { 2 -> {3, 4}, 5 -> {6, 7 -> 8} };
};
		\caption{A tree and its tree-decomposition. Every non-root bag consists of a vertex and its parent.}\label{fig:tree-treedecomp}
	\end{figure}
\end{proof}

\begin{lemma}\label{ex:tw_outerplanar}
	The treewidth of an outerplanar graph is at most 2.
\end{lemma}
\begin{proof}
	Let \(G\) be an outerplanar graph, and let \(G'\) be a triangulation of \(G\). Since \(G\) is a minor of \(G'\), \(\tw(G) \leq \tw(G')\). We look at the \textit{weak dual} of \(G'\). This is a tree \(T\), where every node \(v_f\) in \(T\) corresponds to an internal face \(f\) in \(G'\). Then let \(B_{v_f}\) be the bag of the tree-decomposition, where \(B_{v_f}\) is the set of vertices on the boundary of the face \(f\). Then the tree \(T\) with bags \(B_{v_f}\) is a valid tree-decomposition of \(G'\). Every vertex is on the boundary of some internal face, so every vertex is in some bag. Every bag has at most 3 vertices. Furthermore, every edge is on the boundary of some internal face, so every edge is in some bag. Finally, let $v$ be a vertex. Then the bags that contain $v$ must be connected in $T$ as there is a sequence of internal faces which are adjacent to $v$ and are connected in $T$. Thus, \(\tw(G) \leq 2\). Refer to \cref{fig:outerplanar_treedecomp} for an example of a tree-decomposition. 
	\begin{figure}[h]\label{fig:outerplanar_treedecomp}
		\includesvg[width = 0.8\textwidth]{figures/outerplanar_tree_decomposition.svg}
		\caption{Tree-decomposition of outerplanar graph. The green vertices and black edges are an outerplanar graph. The red vertices and blue edges are the weak dual. The magenta circles around green vertices are examples of bags in the tree-decomposition.}
	\end{figure}
\end{proof}

Define a \(k\)-tree inductively. The complete graph \(K_k\) is a \(k\)-tree, and if \(G\) is a \(k\)-tree, then add a new vertex to \(G\) that is adjacent to \(k\) vertices that form a clique of size \(k\) in \(G\) results in a \(k\)-tree.
A \(k\)-tree is a maximal graph with treewidth \(k\).
\begin{theorem}
	For all graphs $G$, \(\tw(G) \leq k\) if and only if \(G\) is a subgraph of a \(k\)-tree.
\end{theorem}


\begin{theorem}\label{thm:treewidth_clique-minor-free}
	For all graphs $G$, if \(\tw(G) \leq k\), then \(G\) is \(K_{k+2}\)-minor-free.
\end{theorem}
\begin{proof}
	We shall prove the contrapositive: If \(K_t\) is a minor of \(G\), then \(\tw(G) \geq t-1\).
	If \(K_t\) is a minor of \(G\), then from \cref{thm:tw_minor_closure} that \(\tw(K_t) \leq \tw(G)\). As \(\tw(K_t) = t-1\), then \(\tw(G) \geq t - 1\).
\end{proof}

Treewidth was introduced by \textcite{berteleChapterEliminationVariables1972} with applications to dynamic programming under the name ``dimension''. It was then rediscovered by \textcite{halinSfunctionsGraphs1976}. Neither of the papers above discuss treewidth with an explicit construction.

\textcite{robertsonGraphMinorsIII1984} introduced tree-decompositions as defined in \cref{def:tree-decomposition}. This definition is concrete and could be calculated explicitly. They showed that if $\mathcal{F}$ is a graph family with bounded treewidth, then there exists a planar graph $H$ such that $H$ is a forbidden minor of $\mathcal{F}$. This was used to prove the Graph Minor Theorem. Furthermore, \textcite{robertsonQuicklyExcludingPlanar1994} refined this theorem. They showed that if a graph $G$ has large treewidth, then $G$ contains a large $n \times n$ grid as a minor. This is the Grid Minor Theorem.


\section{Path-width}\label{sec:Pathwidth}
Similar to treewidth, the pathwidth of a graph \(G\) defines how similar $G$ is from a path.

Define the path-decomposition of a graph \(G\) to be a sequence of bags \(B_i\) such that the subsequence of bags containing a vertex \(v\) induces a nontrivial subpath and each edge \(vw\) is in a bag \(B_i\). Then define the width of a path-decomposition as \(\max_i \lbrace |B_i| \rbrace -1\), same as treewidth.

If a graph has a path-decomposition \({(B_i)}_i\), then it has a tree-decomposition \(\left({(B_i)}_i, P\right)\). Therefore,
\begin{equation*}
	\pw(G) \geq \tw(G).
\end{equation*}
The pathwidth of \(G\) is the largest pathwidth over all connected components.

A graph \(G\) is a \textit{caterpillar} if \(G\) is a tree and $G$ has a path \(P\) where every vertex not in $P$ is adjacent to a vertex on the path \(P\). Alternatively, a tree \(G\) is a caterpillar if removing every leaf yields a path. We refer to the path where every leaf is connected to as the \textit{central path}.
\begin{theorem}
	Graphs have pathwidth at most 1 if and only if every connected component is a caterpillar.
\end{theorem}
\begin{proof}
	Suppose \(G\) is a caterpillar.
	Denote \(P =\left( p_1, p_2, \dots, p_n\right)\) as the central path. The leaves of vertex \(p_i\) are denoted as \(v_{i, 1}, v_{i, 2} \dots, v_{i, k}\). Define the bags as \((v_{1, 1}, v_1)\), \((v_{1, 2}, v_1)\) \dots \((v_{1, j}, v_1)\),  \((v_1, v_2)\), \((v_{2, 1}, v_2)\), \((v_{2,2}, v_2,)\) \dots. We can see that each leaf appears once and each vertex on the central path is on a subpath of the path. Therefore, the pathwidth of \(G\) is 1. We can repeat this for every connected component.
	\par
	Suppose \(G\) has pathwidth 1. Then for each connected component of \(G\), we choose a vertex \(v\) in \(B_1\) and a vertex \(w\) in \(B_n\), the final bag, and look at a path from \(v\) to \(w\). This path must go through every bag, thus the non-path vertices must have as their neighbour the path vertex in the bag and thus the graph is a caterpillar. An example of this is in \cref{fig:caterpillar}.
\end{proof}
\begin{figure}[ht]
	\centering
	\includesvg[pretex=\small, width = 0.8\textwidth]{figures/caterpillar}
	\caption{A caterpillar graph with central path \((v_1, v_2, v_3, v_4, v_5, v_6)\).}\label{fig:caterpillar}
\end{figure}

\begin{example}
	Recall that $K_n$ is the complete graph on $n$ vertices. It holds that \(\pw(K_n) = \tw(K_n) = n - 1\).
\end{example}
\begin{proof}
	The pathwidth of \(K_n\) is at least the treewidth of \(K_n\). But the pathwidth is at most \(n- 1\) (where all the vertices are in the same bag), but the treewidth of \(K_n\) is \(n - 1\). Therefore, \(\pw(K_n) = n - 1\).
\end{proof}

\begin{theorem}
	The pathwidth of a tree \(T\) equals \(\min_{P \subseteq T} \left\lbrace 1 + \pw(T - V(P))\right\rbrace \) where \(P\) is a path.
\end{theorem}

\begin{proof}[Proof]
	Start by showing the upper bound, \(\pw(T) \leq 1 + \pw(T - V(P))\). If \(P\) is a path in \(T\) with vertices \(v_1, v_2, \ldots v_i\), then consider the subtrees hanging off \(v_i\) for all \(i\). \(T - V(P)\) will have a minimal width path-decomposition. We can order each connected component such that they appear in the order of their parents on the paths. Then adding \(v_i\) to the bags of subtrees connected to \(v_i\), and the bag \((v_i, v_{i+1})\) between the subtrees \(v_i\) and \(v_{i + 1}\) will yield a path-decomposition of width \(1 + \pw(T - V(P))\).
	\par
	To show there exists a path \(P\) such that \(\pw(T) \geq 1 + \pw(T - V(P))\), we proceed by induction. Let \(B_1, \ldots B_n\) be a path-decomposition of \(T\). Let \(x\) live in bag \(B_1\) and \(y\) live in bag \(B_n\), the final bag. Then let \(P\) be the unique path from \(x\) to \(y\). Then \(P\) traverses through every bag in the path-decomposition. Then \(\tw(T) \geq 1 + \tw(T - P)\) by induction.
\end{proof}

% !TEX root = ./thesis.tex

\chapter{Known results from structural graph theory}\label{chap:Known results}
In this chapter of the report, we outline important known results that will help us solve \cref{conj:bded_had_pn}.

\begin{itemize}
	\item \cref{sec:Kt_Minor_Free} is the Graph Minor Structure Theorem, and the full explanation of the Graph Minor Structure Theorem.
	\item \cref{sec:Graph Minor Theorem} discusses the Graph Minor Theorem. 
	\item \cref{sec:BoundedPagenumber} are a series of proofs that can be used with the Graph Minor Structure Theorem to prove that each individual component of the structure theorem has bounded pagenumber.
	\begin{itemize}
		\item \cref{ssec:Clique_sum_Pagenumber_bound} proves that clique sums of bounded adhesion where each component has bounded pagenumber also has bounded pagenumber.
		\item \cref{ssec:Bounded_Treewidth} proves that graphs with bounded treewidth also have bounded pagenumber.
		\item \cref{ssec:pagenumber_bounded_genus} proves that all graphs with bounded genus have bounded pagenumber as well.
	\end{itemize}
\end{itemize}

\section{Bounds of pagenumbers of graphs}\label{sec:BoundedPagenumber} 

\section{Graph Minor Structure Theorem}\label{sec:Kt_Minor_Free}
\textcite{robertsonGraphMinorsXVII1999} proved a rough characterisation of all \(K_t\)-minor free graphs. 

Every graph that is $K_t$-minor-free can be constructed from the following ingredients. This is a coarse characterisation of $K_t$-minor free graphs, meaning that a subset, or a single one of these ingredients constitutes a $K_t$-minor free graph. 
\begin{itemize}
	\item Graphs of bounded Euler genus.
	\item Sets of apex vertices.
	\item Graphs of bounded treewidth.
	\item Sets of vortices on graphs.
\end{itemize}
\textcite{robertsonGraphMinorsXVII1999} showed that every \(K_t\)-minor free graph can be built up from smaller graphs with the above ingredients.

\subsection{Graphs of bounded Euler genus}

Graphs embeddable on a surface of Euler genus $g$ are $K_t$ minor-free, where \(t > \sqrt{6g} + 4\). This comes from \cref{thm:bounded_genus_kt_free}. 

In the case when the surface is a torus, $K_7$ is a toroidal graph but $K_8$ is not. An example of an embedding of $K_7$ on a torus is in \cref{fig:k7_on_torus}.

\begin{figure}[h!]
	\centering
	\includesvg[width = 0.8\textwidth]{figures/k7 on torus.svg}
	\caption[Toroidal graph]{An example of a toroidal graph $K_7$ embedded on a torus.}\label{fig:k7_on_torus}
\end{figure}

\begin{proposition}
	$K_8$ is not embeddable on the torus.
\end{proposition}
\begin{proof}
	A torus has genus 2. By Euler's equation, if a graph $G$ is embedded on a torus, then $|V(G)| - |E(G)| + |F(G)| = 2 - 2 = 0$, where $|F(G)|$ counts the number of faces on the surface. Every face bounds at least three vertices and every edge touches two faces. Therefore, $|F(G)| \leq 2|E(G)|/3$. Suppose $K_8$ is embeddable on the torus. Then $|V(G)| = 8$ and $|E(G)| = 28$. Therefore, $|F(G)| = 20$. But $|F(G)| \leq 2 (28)/3 \leq 19$. Therefore, $K_8$ is not embeddable on the torus.
\end{proof}


\subsection{Apex sets}\label{sssec:Apex_Vertices}
Let $G$ be a graph. A set of vertices $A \subseteq V(G)$ is an apex set if $G - A$ has some bounded parameter. Common parameters are planarity or bounded genus. 
\begin{proposition}
	Let $G$ be a graph. If \(G-a\) is \(K_{t}\)-minor free, then $G$ is $K_{t+1}$-minor free. 
\end{proposition}
\begin{proof}
	We shall prove the contrapositive. Suppose \(G\) has a \(K_{t + 1}\) minor. Then \(K_{t + 1}\) has a model $\rho$ in \(G\). Now let \(v\) be the vertex in \(K_{t + 1}\) such that \(\rho(v)\) contains \(a\). Then delete \(v\) from \(K_{t + 1}\) to form $K_t$. \(K_t\) is a minor of \(G - \rho(v)\). But \(G - \rho(v)\) is a minor of \(G - a\), as \(G - \rho(v)\) does not contain \(a\). So \(G - a\) has a \(K_t\) minor. 
\end{proof}
\subsection{Treewidth and clique-sums}\label{sssec:Clique_Sums}
The \textit{\(k\)-clique-sum} of two graphs \(G\) and \(H\) is a new graph formed from both $G$ and $H$ by identifying two cliques together. The clique-sum of $G$ and $H$ is \(G \oplus_k H\), and is defined as follows. Find cliques in \(G\) and \(H\), \(C_G\) and \(C_H\) respectively, such that both \(C_G\) and \(C_H\) have size \(k\). Identify the vertices in \(C_G\) and \(C_H\) to glue \(G\) and \(H\) together, and possibly delete edges in $C_G$. An illustration can be found in \cref{fig:clique-sum}. 

\begin{figure}[h]
	\centering
	\includesvg[width=0.7 \textwidth]{figures/Clique-sum}
	\caption[Clique-sum]{Figure of clique-sum. Public domain image from David Eppstein \cite{eppsteinCliquesum2023}.}
	\label{fig:clique-sum}
\end{figure}


\begin{proposition}
	Let $t$ be an integer $\geq 1$. Let $G_1, G_2$ be two graphs with treewidth $t$. Then for all $k \leq t + 1$, $G_1 \oplus_k G_2$ has treewidth $t$. 
\end{proposition}
\begin{proof}
	Suppose $C = V(G_1) \cap V(G_2)$ be the clique that is glued over, where $|C| = k$. Let $(B_x: x \in T_1)$ be a tree-decomposition of $G_1$ of minimum width. Let $(B_x : x \in T_2)$ be a tree-decomposition of $G_2$ of minimum width. Then $C$ must appear in some bag $A_x$ and $B_y$ by \cref{lem:clique}. Let $T = T_1 \sqcup T_2$. Add a new node $u$ to $T$ and let $B_u = C$. Then add edges $ux, uy$ to $E(T)$ to form a new tree. Every vertex not in $C$ has a subtree in $T$. If $v \in C$, then the induced subgraph in $T$ is the graph $T_1(v) \cup T_2(v) \cup u$. $T_1(v) \cup T_2(v) \cup u$ is a subtree of $T$. Finally, every edge in $G_1 \cup G_2$ remains in $T$. Therefore, $T$ is a tree-decomposition of $G_1 \oplus_k G_2$. The size of each bag in $T$ is still at most $t + 1$, so the treewidth of $G_1 \oplus_k G_2 \leq t$.
\end{proof}

\begin{proposition}
	Let $t$ be an integer $\geq 1$. Suppose $G$ and $H$ are $K_t$-minor-free graphs. Then $G \oplus_k H$ is $K_{t}$-minor free, $k < t$.  
\end{proposition}
\begin{proof}
	Let $C = V(G) \cap V(H)$ be the clique that is being pasted over. As $G$ and $H$ are $K_t$-minor free, then $|C| \leq k - 1$. Suppose $G \oplus_k H$ is not $K_t$-minor free. Then there exists a model $\rho: V(K_t) \rightarrow G\oplus_k H$ of $K_t$ in $G \oplus_k H$. $\rho$ cannot have its image only in $G$ or only in $H$, as that would mean that $G$ or $H$ has a $K_t$ minor. Therefore, every connected subgraph of $\rho$ must use a vertex in $C$. But $C$ has only $k$ vertices, and $k < t$. Since $\rho$ has disjoint subgraphs, every vertex in $C$ belongs in at most one subgraph in $\rho$. Therefore, $\rho$ cannot have $t$ subgraphs, which is a contradiction. Therefore, $G \oplus_k H$ is also $K_t$ minor free. 
\end{proof}

\begin{corollary}\label{corr:clique_sum_genus}
	If \(G\) is the clique-sum of Euler genus \(g\) graphs, then \(G\) is \(K_{\lceil \sqrt{6g} + 5 \rceil}\)-minor-free.
\end{corollary}
The reverse does not hold. 
\begin{proposition}
	There exists graphs $G$ where \(G\) has arbitrarily large genus, but $G$ is \(K_{6}\)-minor-free.
\end{proposition}

\begin{proof}
	Consider $n$ copies of $K_5$ and identify one vertex in every $K_5$ to a single vertex $v$ to form $G$. Then $G$ is $K_6$-minor free. However, from \cref{thm:additivity_genus}, $G$ has genus $n$. Thus, $G$ has unbounded genus. 
\end{proof}

\subsubsection{Torsos and adhesion}\label{sssec:Torsos and Adhesion}
Given a graph \(G\) and a tree-decomposition \(\tree\), the \textit{torso} of a bag \(B_x\) of \(T\) is the graph \(G\langle B_x \rangle\), with vertex set $B_x$ and edge set where \(vw\) is an edge in \(G\langle B_x \rangle\) if and only if $vw \in E(G)$ or \(v,w \in B_x \cap B_y\), where \(y\) is any neighbour of \(x\) in \(T\). The edge $uv$ where $uv \in B_x \cap B_y$ are called \textit{torso edges}. The set \(B_x \cap B_y\) for all neighbours \(y\) of \(x\) in \(T\) is a clique in \(G\langle B_x \rangle\).
The \textit{adhesion set} is the set \(B_x \cap B_y\). 
The \textit{adhesion} of a tree is defined as \(\max(|B_x \cap B_y|)\) where \(xy\) is an edge in \(T\).

Given \(G\) and a tree-decomposition \(\tree\), \(G\) is a clique-sum of the torsos \(G\langle B_x \rangle\) where the size of the cliques that we paste over is at most the adhesion of $\tree$. This holds for any arbitrary tree-decomposition.
We will discuss decomposing graphs in the language of tree-decompositions, rather than clique-sums. This is because we can discuss the structure of the tree-decomposition.

\subsection{Vortices}\label{sssec:vortices}
Let \(G\) be embedded on a surface \(\Sigma\), and let \(F\) be a face on \(G\). A disc $D$ is \textit{$G$-clean} if $D$ is an open subset of $F$ and $G \cap D$ is a tuple of vertices \(\Lambda = (x_1, x_2, \ldots, x_b)\). No vertex appears more than once in $\Lambda$. The ordering of $\Lambda$ is around the boundary of $D$. 
\par
Let $G$ be a graph embedded on $\Sigma$. Let $D$ be a $G$-clean disc with $G \cap D = \Lambda = (x_1, x_2, \ldots, x_k)$. A \textit{$D$-vortex} is a graph $H$ such that $V(G) \cap V(H) = \Lambda$ and there is a \textit{path-decomposition} of \(H\) of bags \(B_1, B_2, \ldots B_k\) such that \(x_i \in B_i\) for all \(i\). The \textit{depth} of the vortex $H$ is the path-width of $H$. 
\par
The following figure, \cref{fig:tenniscourt} demonstrates the necessity of vortices. $G_n$ is $K_8$-minor free. However, $G_n$ has around $\frac{n}{3}$ $K_{3,3}$ copies, so has genus around $\frac{2n}{3}$. As there is an $n \times n$ grid minor, $G$ has treewidth at least $n$. As $G$ can be arbitrarily large, the number of apex vertices to remove to bound the treewidth and genus is arbitrarily large. However, there is a decomposition of $G_n$ into two graphs $G_0$ and $G_1$ where $G_0$ can be embedded on a surface and $G_1$ is a vortex on $G_0$ with depth 6. $G_0$ is the $n \times n - 1$ grid and $G_1$ is the $n \times 2$ grid in the back plus the apex vertices. 

\begin{figure}[h]
	\centering
	\includesvg[width = 0.8\textwidth]{figures/tenniscourt}
	\caption[Tennis-Court graph]{An example of an $n \times n$ \textit{tennis-court} graph $G_n$ which is \(K_8\) minor free.}
	\label{fig:tenniscourt}
\end{figure}
\subsection{Robertson-Seymour Graph Minor Structure Theorem}\label{ssec:Robertson_Seymour_Graph_Structure}
Given integers \(g, p, a \geq 0\), \(k \geq 1\), a graph \(G\) is \((g, p, k, a)\)- almost-embeddable if there exists an \(A \subseteq V(G)\) with \(|A| \leq a\), and there exists subgraphs \(G_0, G_1, \ldots,  G_{p'}\) of \(G\) such that:
\begin{itemize}
	\item \(G - A = G_0 \cup G_1 \cup G_2 \cup \ldots \cup G_{p'}\),
	\item \(p' \leq p\),
	\item there is an embedding of \(G_0\) onto a surface \(\Sigma\) of genus \(\leq g\),
	\item there exists pairwise disjoint \(G_0\)-clean discs \(D_1, D_2, \ldots, D_{p'}\) in \(\Sigma\),
	\item \(G_i\) is a \(D_i\)-vortex of depth at most \(k\).
\end{itemize}

If we restrict $G_0$ to live only on orientable surfaces, then the graph $G$ is ${(g, p, k, a)}^+$-almost embeddable. If the apex set $A$ is empty, then the graph $G$ is $(g, p, k)$-almost embeddable. 

\begin{theorem}[Graph Minor Structure Theorem \cite{robertsonGraphMinorsXVI2003}]\label{thm:gmst}\todo{what is $\ell$ with respect to the other constants?}
	For all \(t\), there exists \(g, p, a \geq 0\) and \(k, \ell \geq 1\) such that every \(K_t\)-minor-free graph has a tree-decomposition of adhesion \(\leq \ell\) and each torso is \((g, p, k, a)\)-almost-embeddable. The  family of graphs with tree-decomposition of adhesion $\leq \ell$ with torsos $(g, p, k, a)$-almost-embeddable is \(\mathcal{G}(g, p, k, a)\). 
\end{theorem}
There exists a function \(t(g, p, k, a)\) such that if a graph has a tree-decomposition of adhesion \(\leq \ell\) and each torso is \((g, p, k, a)\)-almost embeddable, then \(G\) has no \(K_t\) minor.

\textcite{kawarabayashiQuicklyExcludingNonplanar2021} found upper bounds for $g, p, k, a$. 
\begin{theorem}[\textcite{kawarabayashiQuicklyExcludingNonplanar2021}]
	Let $t \geq 1$ be a positive integer. Let $G$ be a $K_t$-minor free graph. Then let $\alpha = t^{18 \cdot 10^{7} t^{26} + c}$ for a constant $c$, which is defined in the paper. Then setting $g = t(t+1)$, $p = 2t^2$, $k = \alpha$, $\ell = 4\alpha$, and $a = 3\alpha$, $G \in \mathcal{G}(g,p,k,a)$. 
\end{theorem}

\textcite{joretCompleteGraphMinors2013} studies the question of the maximum order of a complete graph minor of a graph in $\mathcal{G}(g, p, k, a)$. 
\begin{theorem}[\textcite{joretCompleteGraphMinors2013}]\label{thm:graph_structure_bound_theorem}
	For all graphs \(G \in \mathcal{G}(g, p, k, a)\),
	\(\had(G) \leq a + 48(k + 1)\sqrt{g + p} + \sqrt{6g} + 5\). Moreover, for some constant $c$, for all $g, a \geq 0$, $p \geq 1, k \geq 2$, there exists a graph $G$ in \(\mathcal{G}(g, p, k, a)\) such that in \(n \geq a + c k\sqrt{p + g}\) such that \(K_n\) is a minor of $G$.
\end{theorem}
\section{Bounds of pagenumber of graphs}\label{sec:BoundedPagenumber}
\subsection{Tree-decomposition into bounded page number torsos}\label{ssec:Clique_sum_Pagenumber_bound}

This proof has been adapted into the language of tree-decompositions.
\begin{theorem}[\textcite{hickingbothamStackNumberCliqueSum2023}]\label{thm:clique_sum_pagenumber_bound}
	Let \(G\) be a graph with a tree-decomposition \((B_x: x \in V(T))\). Suppose every torso \(G \langle B_x \rangle\) has pagenumber \(\leq s\) and is \(c\)-colourable. Further suppose the adhesion of the tree-decomposition is at most \(\ell\).
	Then $G$ can be embedded in \(\leq cs + \ell \) pages.
\end{theorem}

\subsubsection{Proof of above theorem}
This proof involves gluing torsos along cliques of size at most \( \ell \).

Let \(C\) be a clique in \(G\) and let \(\sigma_C = (u_1, \ldots , u_k)\) be a vertex ordering of \(V(C)\), and let \(C \leq \ell \). Let $J$ be a clique in $G$. A vertex $v$ is rainbow in $J$ in a book-embedding $(<, \psi)$ if the set of edges $\{u_i v | u_i < v, u_i \in J\}$ each have distinct colours. The structure of the book-embedding will look like this: \((\underbrace{u_1, u_2, \ldots, u_k}_{\text{Vertices in } C}, \underbrace{v_1, v_2, \ldots, v_l}_{\text{Vertices not in }C})\).

To prove this theorem, we use a common technique in graph theory. We strengthen the lemma so that we may use induction to prove the statement.
\begin{lemma}\label{lem:Hickingbotham_Lemma}
	Let \(G\) be a graph where \(\pn(G) \leq s\) and \(\chi(G) \leq c\), and a clique \(C\) with an ordering \(\sigma_C\). Let \(|C| \leq \ell\). There exists a \(cs + \ell\)-page layout \((\leq, \psi)\) of \(G\) where:
	\begin{enumerate}
		\item The vertex ordering has the structure \((\underbrace{u_1, u_2, \ldots, u_k}_{\text{Vertices in } C}, \underbrace{v_1, v_2, \ldots, v_l}_{\text{Vertices not in }C})\).
		\item For every \(u \in V(C)\), the edges \(\lbrace uv \in E(G) : u \leq v \rbrace\) are a monochromatic star.
		\item For every clique \(J\), the last vertex of \(J\) is a rainbow-vertex.
	\end{enumerate}
\end{lemma}
\begin{proof}
	Let \((\leq_a, \psi_a)\) be a \(s\)-page layout of \(G\) and let \(\rho: V(G) \rightarrow [c]\) be a proper colouring of \(V(G)\).

	Let \(u_1, \ldots, u_k\) be the vertices of \(C\) ordered by \(\sigma_C\). Note that \(k \leq \ell\). Then the new ordering starts with \(u_1 \leq u_2 \leq \ldots, \leq u_k\), and all vertices not in \(K\) are placed after, according to \(\leq_a\).
	The edge-colouring \(\psi\) is defined as follows. For every edge \(u_i v\) where \(u_i \in V(C)\) and \(u_i \leq v\), \(\psi(u_i v) = i\). If neither \(u\) nor \(v\) are in \(V(C)\), and \(u \leq v\), then let \(\psi(uv) = (\rho(u), \psi_a(uv))\). Then this edge-colouring requires \(|\rho| |\psi_a| + k \leq cs + \ell\) pages.

	Now we show \((\leq, \psi)\) is a proper book-embedding. Suppose there exists edges \(uv\) and \(xy\) where \(\psi(uv) = \psi(xy)\). Suppose that \(u\) is the smallest vertex in the ordering \(\leq\). If \(u \in V(C)\), then the edge \(uv\) is assigned the page consisting of only edges adjacent to $u$. So \(x = u\), but this is a star. Therefore, the edges do not cross. Therefore \(u, v, x, y\) are not in \(V(C)\). But \((\leq, \psi)\) restricted to the subgraph $G - C$ looks like \((\leq_a, \psi_a)\), if pages with different colours where identified. Therefore, \((\leq, \psi)\) is a proper book-embedding.  
	\par
	Properties 1 and 2 are immediate from the definition \((\leq, \psi)\). For property 3, consider a clique \(J\) in \(G\). Then we must show the last vertex of \(J\) is rainbow. Suppose the last vertex of \(J\) is \(w\), and let \(u, v\) be earlier vertices. Since \(u\) and \(v\) necessarily are assigned different colours in the colouring, then \(\psi(uw) = (\rho(u), \psi_a(uw))\) and \(\psi(vw) = (\rho(v), \psi_a(vw))\). Therefore, the two edges are on different pages. Thus \(w\) is a rainbow vertex.
\end{proof}

\begin{theorem}[\textcite{hickingbothamStackNumberCliqueSum2023}]
	Suppose a graph \(G\) has a tree-decomposition \((B_x: x \in V(T))\) with torsos \(G \langle B_x \rangle\) and adhesion at most \(\ell\). Order the vertices \(v_0, \ldots, v_k\) in $T$ with respect to a breath-first search. Let $B_i = B_{v_i}$. Suppose that for all torsos $G\langle B_i \rangle$, \(\pn(G\langle B_i \rangle) \leq s\) and \(\chi(G\langle B_i \rangle) \leq c\). Then there is a book-embedding of \(G\) with at most \(cs + \ell\) pages.
\end{theorem}
For a breadth-first search, we maintain the property that for all \(i\), \(T[v_0, \ldots, v_{i}]\) is a tree and \(v_i\) is a leaf in \(T[v_0, \ldots, v_{i}]\).
\begin{proof}
	We prove the stronger statement that there exists a book-embedding with the property that the last vertex of any clique \(J\) is a rainbow vertex. For short, this property is the \textit{rainbow-clique} property. 

	Suppose $G$ has a tree-decomposition consisting of a single torso with the properties above. Then \(G\langle B_0 \rangle\) is a single graph with \(\pn(G) \leq s\). Choose \(C\) to be an arbitrary vertex \(v\) in \(G\langle B_0 \rangle\). Then by \cref{lem:Hickingbotham_Lemma}, there is a book-embedding with pagenumber at most \(cs + 1\) and every last vertex in a clique is a rainbow vertex.

	Suppose $G$ has a tree-decomposition $(B_x: x \in V(T))$ with the properties above. Take a breadth-first search of $T$, with vertex ordering $v_0, \ldots, v_n$. For the induction hypothesis, suppose that the induced subgraph $G' := G[B_0 \cup B_1 \cup \ldots \cup B_{n-1}]$ maintains the rainbow-clique property with pagenumber of at most \(cs + \ell\).  
	Let \(C\) be the adhesion clique between \(G \langle B_n \rangle\) and $G'$. Then let \((\leq_n, \psi_n)\) be the \(cs + \ell\)-pagenumber book-embedding of \(G \langle B_n \rangle\) that starts with \(V(C)\). Let \((\leq_{n-1}, \psi_{n-1})\) be the book-embedding of \(G'\). By the induction hypothesis, \((\leq_{n-1}, \psi_{n-1})\) is a \(cs + \ell\)-page book-embedding with the rainbow-clique property.

	\paragraph{Construction of new book-embedding}
	We construct a new book-embedding \((\leq, \psi)\).
	Let \(w \in V(C)\) be the last vertex of \(C\) with respect to \(\leq_{n-1}\). Then insert \(V(G \langle B_n \rangle) - C\) between \(w\) and its successor in $G'$ with the order of \(\leq_{n-1}\) to make $\leq$. For the page assignment \(\psi\), if \(uv \in E(G')\), then \(\psi(uv) = \psi_{n-1}(uv)\). For edges in $G \langle B_n \rangle$, permute the edge assignments of \(\psi_n\) such that for all \(u \in V(C)\), \(\psi(uv) = \psi_n(uw)\) for $v \in C$ and $u \leq_n v$. This is possible as \(w\) is a rainbow vertex and the edges \(\{uv : v \in C, u \leq_n v\}\) are assigned to a unique page in \(\psi_n\). Finally, let \(\psi(uv) = \psi_n(uv)\) for all remaining edges in $G \langle B_n \rangle$. 
	\paragraph{Proof that this is a valid book-embedding}
	We claim that \((\leq , \psi)\) is a stack layout that satisfies the induction hypothesis. Suppose that \(\psi(uv) = \psi(xy)\). If \(uv, xy \in E(G')\), then by the induction hypothesis, they do not cross. Similarly, if \(uv, xy \in E(G \langle B_n \rangle)\), then they do not cross as well. If \(uv\) is in \(E(G')\) and \(xy \in E(G \langle B_n \rangle)\), then they will go over each other or be sequential and therefore will not cross.
	Finally, if \(u, v, x, y \in C\), then by the induction hypothesis on \(G'\), they do not cross either. The final case is if \(uv \in E(G\langle B_{n} \rangle)\) and \(u \in V(C)\), \(v \notin V(C)\), \(xy \in E(G')\). If \(uv\) and \(xy\) cross, then \(xy\) and \(uw\) will cross. But this will contradict the page-embedding of \(G'\) as $u, w, x, y$ are in $G'$.

	Let \(J\) be a clique in \(G\). Then $J$ is either only in $G'$, only in $G\langle B_n \rangle$, or shares vertices with $C$ This is because $V(G') \cap V(G\langle B_n \rangle) = C$, so $C$ is a separator between $G'$ and $G \langle B_n \rangle$. If $J$ is only in $G'$ or only in $G\langle B_n \rangle$, then the last vertex of $J$ is a rainbow vertex by hypothesis. If $J$ shares vertices with $C$, then the last vertex of $J$ with $\leq$ is in $G\langle B_n \rangle$ by construction. However, this vertex is a rainbow vertex, as the last vertex in $J$ with respect to $\leq$ is the last vertex in $G\langle B_n \rangle$ with respect to $\leq_n$. 
\end{proof}

We have some bounds in terms of pagenumber on some of these constants. However, these bounds are not tight, in the case of planar graphs. 

\begin{lemma}
	Suppose a graph $G$ can be embedded on $s$ pages where $s$ is at least $2$. Then \(G\) does not contain any cliques on more than \(2s+1\) vertices.
\end{lemma}

\begin{proof}
	If \(G\) has a clique $K$ of size \(k\), then embedding $K$ requires at least \(\lceil \frac{k}{2} \rceil\) pages, from \cref{thm:Pagenumber_Complete_Graph}. Therefore, if we can embed \(G\) in \(s\) pages, then every clique in $G$ has at most \(2s + 1\) vertices.
\end{proof}
Therefore, \(\ell \leq 2s + 1\).

Let $G$ be a graph with the properties above. As \(\chi(G) \leq 2 \pn(G) + 2\), from \cref{thm:Colouring_Bound}, there exists a bound that does not depend on the chromatic number or largest clique of \(G\).
\begin{corollary}[\textcite{hickingbothamStackNumberCliqueSum2023}]\label{corr:bded_pn_tree_decomp}
	Let \(G\) be a graph with a tree-decomposition \((B_x: x \in V(T))\) where each torso \(G \langle B_x \rangle\) can be embedded on $s$ pages. Then from \cref{thm:clique_sum_pagenumber_bound}, with $\ell \leq 2s + 1$ and $\chi(G) \leq 2 s + 2$, \(G\) can be embedded on \(2s^2 + 4s + 1\) pages.
\end{corollary}

This section finds a bound on the pagenumber of planar graphs. This upper bound is much worse than the tight upper bound found by Yannakakis \cite{yannakakisEmbeddingPlanarGraphs1989}. 

\begin{lemma}\label{lem:planar_graphs_4_connected_cliqesums}
	Every planar graph with at least 5 vertices has a tree-decomposition such that each torso is a \(4\)-connected planar graph (or a subgraph of $K_4$) with adhesion at most \(3\). Furthermore, there exists a rooting of the tree-decomposition such that every torso is contained within a face of another torso. 
\end{lemma}

\begin{proof}
	Suppose that $G$ is not $4$-connected, meaning there exists at most $3$ vertices $u, v, w$ in $G$ that separate $G$. There exists graphs $G_1$, $G_2$ such that $G = G_1 \oplus_3 G_2$, $\{u, v, w\} = V(G_1 \cap G_2)$ and $G$ embedded on the plane has that $G_2$ is a subset of a face on $G_1$ with boundary $\{u, v, w\}$. If this is not the case, then $G_1$ or $G_2$ is not $4$-connected. This comes from the fact that $G$ is planar. This is the only arrangement as every separator from a cut-set $A$ to $B$ can be pushed along until the separator bounds a face, which is $u, v, w$. Therefore, repeating this operation for every face of $G$ yields a tree-decomposition with the property listed above. 
	\todo{Finish proof!}
\end{proof}

\begin{theorem}[Tutte\cite{tutteTheoremPlanarGraphs1956}]\label{thm:4-connected_planar_ham_cycle}
	All 4-connected planar graphs are Hamiltonian.
\end{theorem}

As a corollary to \textcite{hickingbothamStackNumberCliqueSum2023}, the pagenumber of planar graphs are bounded.

\begin{theorem}\label{thm:Planar Graph Hickingbotham Bound}
	Let \(G\) be a 2-connected planar graph. Then $G$ can be embedded on $11$ pages, with book-embedding $(<, \rho)$. $<$ restricted to the outer cycle $C$ is $C$. Furthermore, for every face cycle $C$, $<|_{V(C) - \{u, v, w\}} = C - \{u, v, w\}$ for some vertices $u$, $v$, $w$. 
\end{theorem}
\begin{proof}
	From \cref{thm:clique_sum_pagenumber_bound} with tree-decomposition from \cref{lem:planar_graphs_4_connected_cliqesums}, the pagenumber is at most \(2 \cdot 4 + 3 = 11\).

	Furthermore, from the construction given in \cref{lem:planar_graphs_4_connected_cliqesums}, every $4$-connected class are glued on faces. Therefore, every face only changes by $3$ vertices, from \cref{thm:clique_sum_pagenumber_bound}. Therefore removing $3$ vertices from every face preserves the cyclic ordering of every face.
\end{proof}

We will discuss the \(K_5\)-minor free case. If \(G\) is \(K_5\)-minor free, then we can use Wagner's theorem.
\begin{theorem}[Wagner's theorem\cite{wagnerUeberEigenschaftEbenen1937}]\label{thm:WagnersTheorem}
	Let \(G\) be a \(K_5\)-minor-free graph. Then \(G\) has a tree-decomposition of adhesion $\leq 3$ where every torso is either a planar graph or the Wagner graph \(V_8\).
\end{theorem}
A description of the Wagner graph is in \cref{fig:wagner}. The edges are coloured such that the internal edges are on different pages. The spine edges (the edges that are on the outerface) are the ones which can go on any page.
\begin{figure}[h!]
	\centering
	\begin{tikzpicture}[thick,scale=2, every node/.style={scale=2}]
		\tikz \graph [nodes = {draw, circle}, clockwise, empty nodes] {
	subgraph C_n [n=8];
	1 --[red] 5;
	2 -- 6;
	3 -- 7;
	4 -- 8;
};

	\end{tikzpicture}
	\caption[Wagner graph]{The Wagner graph $V_8$. Notice how the clockwise circular ordering of the vertices of the Wagner graph needs 4 pages to embed the graph. }\label{fig:wagner}
\end{figure}

\begin{theorem}
	Let \(G\) be a \(K_5\)-minor free graph. Then \(G\) has pagenumber \(\leq 19\).
\end{theorem}

\begin{proof}
	Suppose \(G\) is \(K_5\)-minor free. Then by Wagner's theorem \cite{wagnerUeberEigenschaftEbenen1937}, \(G\) has a tree-decomposition of adhesion at most 3 where every torso is either a planar graph or the Wagner graph.
	Planar graphs are \(4\)-colourable and can be embedded on four pages. The Wagner graph is \(3\)-colourable and can be embedded on four pages. Therefore, if \(G\) is \(K_5\)-minor free, then \(G\) has pagenumber at most \(4 \cdot 4 + 3 = 19\) from \cref{thm:clique_sum_pagenumber_bound}.
\end{proof}
% !TEX root = ./thesis.tex

\section{Graphs embedded on an orientable surface of bounded genus}\label{sec:pagenumber_bounded_genus}

\textcite{yannakakisEmbeddingPlanarGraphs1989} showed that every planar graph can be embedded on four pages.

From \cref{thm:Planar Graph Hickingbotham Bound}, every planar graph can be embedded on $11$ pages. However, the proof given by Yannakakis is tight, as there exist planar graphs that need four pages, from \textcite{yannakakisPlanarGraphsThat2020,bekosFourPagesAre2020}. 

\textcite{heathPagenumberGenusGraphs1992} proved the following theorem:
\begin{theorem}[\textcite{heathPagenumberGenusGraphs1992}]\label{thm:Genus_pagenumber_bound}
	When $g \geq 1$, every orientable genus $g$ graph can be embedded in $18g+2$ pages.
\end{theorem}
Note that this bound extends the one found by \textcite{yannakakisEmbeddingPlanarGraphs1989} to graph families of bounded orientable genus.
The best known bound is \(O(\sqrt{g})\), found by \textcite{malitzGenusGraphsHave1994}. This is best possible as complete graphs $K_n$ have pagenumber $O(n)$ but have Euler genus $O(n^2)$, as shown by \textcite{ringelMapColorTheorem1974}. Therefore, if $K_n$ has Euler genus $g$ then the pagenumber of $K_n$ is $\Theta(\sqrt{g})$.

%In fact, there exists a family of graphs such that $\tilde{\gamma}(G) \in \Omega(|V(G)|)$ for every graph $G$ in the family. Consider $n$ copies of $K_6$ with a single vertex in each copy identified to a single vertex. Then each $K_6$ will occupy a single crosscap with no room for another copy of $K_6$ to fit. Therefore, this graph is of non-orientable genus $2n$. 

This section \cref{thm:Genus_pagenumber_bound}. We first decompose a graph $G$ on an orientable surface into a planar spanning subgraph and some `nonplanar' edges. This decomposition can be used to embed a graph on a book with a bounded number of pages.

Let $G$ be a connected graph embedded on a surface $\Sigma$. A subgraph $H$ of $G$ inherits an embedding on $\Sigma$ if the embedding of $H$ is a subembedding of $G$ on $\Sigma$. Heath and Istrail partition the edge-set of \(G\) to a planar spanning connected subgraph \(G_p\) embedded on $\Sigma$, with the same embedding as $G$, and set of `nonplanar' edges $E_N$. This decomposition of $G$ has the following properties:
\begin{enumerate}
	\item Every edge in \(E_N\) are between vertices on the boundary of $F_0$, the outerface of $G_p$.
	\item Adding any edge from $E_N$ to \(G_p\) breaks conditions 1.
\end{enumerate}
$G_p$ is a maximal planar subgraph with respect to the conditions above. The set $(G_p, E_N)$ is a \textit{planar-nonplanar decomposition} of $G$. 
% To talk about graphs embedded in surfaces, we assign to each face a cyclic permutation \(\sigma_v\) which represents the sequence of vertices encountered when traversing the boundary of a face in counterclockwise order.

% $\sigma_v$ enough to represent any graph in an orientable surface, but not enough for a non-orientable surface. We have to attach on an orientation to each edge, where each edge is either orientation-preserving or orientation-reversing.

% A planar-nonplanar decomposition of \(G\) is a triple \((R, G_P, E_N)\) where \(R\) is a rotation of \(G\) representing the surface embedding on the surface \(S\), \(G\) is a spanning planar graph, and \(E_N = E - E(G_P)\).
% This satisfies a list of properties:
% \begin{enumerate}
% 	\item The subrotation induces a planar embedding of \(G_p\), so we can arrange \(G\) on the surface \(S\) such that the embedding of \(G_p\) is planar.
%	\item For each \(vw \in E_N\), \(v\) and \(w\) live on the outerface \(F_0\).
%	\item \(E(G_P)\) is maximal, so we cannot add edges from \(E_N\) to \(G_P\) without breaking properties 1 and 2.
%\end{enumerate}
We build an equivalence class of nonplanar edges in a planar-nonplanar decomposition according to homotopy class. Let $G$ be a graph embedded on $\Sigma$ and let $(G_P, E_n)$ be a planar-nonplanar decomposition. Let \(C_0\) be the walk on the boundary of \(F_0\), the outerface of $G_P$. Each vertex on the boundary of \(F_0\) appears at least once on $C_0$. If a vertex $v$ appears more than once on $C_0$, then $v$ is an \textit{articulation point}, meaning $v$ separates biconnected components. Heath and Istrail refer to a directed subpath of \(C_0\) as a \textit{trace}. Let \(T = v_1 \rightarrow v_2 \rightarrow \cdots \rightarrow v_t\) be a trace. The inverse trace is \(T^{-1} = v_t \rightarrow v_{t-1} \rightarrow \cdots \rightarrow v_1\).

Nonplanar edges need to be classified so that they can be embedded on a book. Suppose that \(u_1v_1, u_2v_2 \in E_N\) are part of the boundary of the same face \(F\) on the embedding of \(G\). Then \(u_1v_1\) and \(u_2v_2\) are \textit{homotopic} if both conditions hold.
\begin{enumerate}
	\item \(u_1v_1\) and \(u_2v_2\) are the only edges of \(E_N\) on the boundary of \(F\).
	\item There exist traces \(T_u = u_1 \rightarrow \cdots \rightarrow u_2\) and \(T_v = v_1 \rightarrow \cdots \rightarrow v_2\) such that \(T_u\) and \(T_v\) are on the boundary of \(F\).
\end{enumerate}
Then extend homotopy to be an equivalence relation, with homotopy classes being the classes of loops that are homotopic to each other. 
Think of \(G_n\) as living on a disk on \(S\) and the loops \(u_1v_1\) and \(u_2 v_2\) living on a handle or passing through a crosscap. Taking \(G_n\) to a point, \(u_1v_1\) and \(u_2v_2\) are homotopic if and only if the two loops are homotopic in a topological sense.

We now do a planar-nonplanar decomposition of graphs embedded on surfaces.
\begin{claim}\cref{claim:planar_nonplanar_decomp}
	There exists a planar-nonplanar decomposition of any graph 2-cell embedded on any surface, and the trace of any homotopy class lies on the same block, except for two vertices adjacent to special endpoints $x$ and $y$. 
\end{claim}
\begin{proof}
	Suppose \(G\) is a graph 2-cell embedded on a surface \(\Sigma\). Start at a single face and define this face as $G_p$. Add edges and vertices to the planar part incrementally. At each step, set \(G_P\) to be the current planar part and \(E_N\) to be the edges that are outside the planar part. There are two types of edges in \(E_N\): edges which have both endpoints on the boundary \(V(G_P)\), so cannot become edges of \(G_P\), and edges that have either one or no endpoints in \(V(G_P)\). We maintain the property that for any \(v \in V(G_P)\), and edge \(vw \in E_n\), then \(v\) is a vertex on the boundary of \(F_0\).
	\paragraph{Adding vertices to biconnected block:}
	Suppose $F$ is the outerface of \(G_P\). Define an oriented walk around the boundary of $F$ to be a \textit{trace}. If \(v_i \rightarrow v_j \rightarrow v_k\) is a walk with no edge of \(E_N\) incident to \(v_j\), then \(v_i v_k \in E(G_T)\) is called a \textit{safe} edge. If \(v_i \rightarrow v_j\) is on the boundary of \(G_P\), and \(v_k \notin V(G_P)\), and \(v_i,v_k,v_j\) is the boundary of a face, then \(v_k\) is a \textit{safe vertex} and can be added to \(G_P\). This face is biconnected as there are no cut-vertices, as triangles are added at every step. Add as many vertices $v_k$ as possible. 

	\paragraph{Creating a new biconnected block:}
	If no \(v_k\) exists, then find a \(w'\) which is the newest vertex in \(V(G_P)\) adjacent to a vertex \(w\) not in \(V(G_P)\). Add the vertex \(w\) and the edge \(w w'\) to \(G_P\). Then add all safe edges. This is so that every edge $xy$ in \(E_p\) has $x$ and $y$ on the boundary of $F_0$. 
	
	After repeating this operation, every edge in \(E_N\) are between two vertices on the boundary of $F_0$ and adding any edge from $E_N$ to $G_p$ breaks the first condition. If an edge \(vw\) has \(v\) in $G_p$ and $w$ not in $G_p$, then at some step \(w\) is added as a safe vertex or biconnected block. If an edge \(vw\) has neither \(v\) nor \(w\) added to \(G_P\), then the algorithm has not finished yet. This algorithm always finds some vertex $v$ which is neighbouring $G_P$ and adds $v$. Since $G$ is connected, then every vertex is added to $G_P$. Thus, $G_P$ is spanning.
	
	Now every edge in \(E_N\) cannot be added to \(G_P\) without crossing over another edge, and \(G_P\) is maximal. Therefore, every edge in \(E_N\) are between two vertices on the boundary of $F_0$ and adding any edge from $E_N$ to $G_p$ breaks the first condition.

	Suppose $v_k$ was just added to $G_P$ and $v_k, z$ is a nonplanar edge where $z$ is not in the same block as $v_k$. Then no other edge incident to $v_k$ is homotopic to $v_k, z$ otherwise $v_k$ would have been a safe vertex and added to $z$. Then the homotopy class will just have $z$. 

	Suppose a new homotopy class $H$ has multiple edges added when $v_k$ is added. Then the endpoints of $H$ must be in the same block as $v_k$, as edges in a single block is added. Let $T_1 = u_1, \ldots, u_s$ and $T_2 = v_1, \ldots, v_t$ be the trace of $H$. The only vertices that are not in the same block must be the endpoints of $H$. If $w$ extends $H$, then either $u_1 w$ and $v_1 w$ are edges or $u_s w$, $v_t w$ are edges. Then $T_1$ and $T_2$ cannot be both extended as $v_t w$ or $u_s w$ is now a cut edge. If $v_t$ is $w'$ and $w$ is its neighbour, then $w$ is cut off. Then every new edge in $H$ must be adjacent to $w$. Define $x := w$. This argument can be repeated for $u_1, v_1$, and define $y := u_1$ when $u_1$ is the vertex that is cut off. 
\end{proof}


Safe vertices and unsafe vertices are illustrated in \cref{fig:safe_vertices}.

\begin{figure}[h!]
	\centering
	\includesvg[width = 0.8\textwidth]{figures/safe_vertices.svg}
	\caption[Safe and unsafe vertices]{On the left is a safe vertex $v_k$. Notice how this is not on $G_P$ but can be added without breaking the biconnected planarity condition. On the right is an unsafe vertex $w$. This cannot be added to a block in $G_P$ as there is a bridge edge. Therefore, $w$ is added as on its own separate block.}\label{fig:safe_vertices}
\end{figure}

% \subsubsection{Level sets and cycles}
% On a planar graph \(G\), we want to separate out vertices depending on how far away they are from the outerface. Fix a single outerface \(F_0\) and define the first level set \(V_0\) as the vertices adjacent to \(F_0\). Define the \(i\)-th level set, \(V_i\) inductively. Consider the induced graph on \(V(G) - \cup_{k = 0}^{i-1} V_k\). Define the vertices adjacent to \(F_0\) in this induced graph, where we expand \(F_0\) after deleting the vertices. This partitions \(V(G)\).

% We then define \(C_0\) to be the edges adjacent to \(F_0\) in this decomposition. Then we want \(C_i\) to be the edges adjacent to \(F_0\) in this decomposition. We define the chord edges \(K_i\) to be the edges between vertices in \(V_i\) that are not edges in \(C_i\). Finally, we define the edges between faces, \(E_i\) as the edges that are between vertices on level \(V_i\) and \(V_{i + 1}\).

% \begin{claim}
% 	For all faces \(F\) in \(G\), the vertices around \(F\) are either all in one \(C_i\) or they are in \(C_i\) and \(C_{i + 1}\) for some \(i\).
% \end{claim}

% \begin{proof}
% 	Let \(i\) be the smallest value such that \(v \in V_i\) is on the boundary of \(F\). Now \(G[V(G) - \cup_{j = 1}^{i} V_i]\) will also remove \(v\). However, this removes all the edges next to \(v\), therefore all vertices that are on the boundary of \(F\) will either be in \(V_i\) or \(V_{i + 1}\).
% \end{proof}
% We refer to the faces that have vertices in only \(V_i\) as chordal and the faces that are between \(V_i\) and \(V_{i + 1}\) as non-chordal.

% Recall a weak triangulation of \(G\) is a triangulation \(G'\) where all faces except for the outerface is a triangulation.
% \begin{claim}
% 	There exists a weak triangulation $G'$ of \(G\) that preserves the level sets \(V_i\) and edge sets \(E_i\), \(C_i\), \(K_i\) for all \(i\).
% \end{claim}

% \begin{proof}
% 	If \(F\) is a chordal face of \(G\), then any triangulation maintains the property. If \(F\) is non-chordal and the boundary has edges in \(V_i\) and \(V_{i + 1}\), then add edges that are only between vertices in \(V_i\) and \(V_{i + 1}\). This will suffice to build a new triangulated graph \(G'\) where all vertices and edges are in the correct place.
% \end{proof}


Now homotopy classes need to be embedded on a book, and the orientability of the surface plays an important role. Let \(u_1 v_1, \ldots, u_k v_k\) be a homotopy class of edges such that $u_1, \ldots u_k$ are in cyclic order around $C_0$. Let trace \(T_1\) be the concatenation of traces from \(u_1\) to \(u_2\), $u_2$ to $u_3$, up to $u_k$. Finally, let trace \(T_2\) be the concatenation of traces from \(v_1\) to \(v_2\), $v_2$ to $v_3$, up to $v_k$.
Two traces $(T_1, T_2)$ are \textit{orientable} if \(T_1\) and \(T_2\) go in opposite directions on $C_0$, and \textit{non-orientable} if \(T_1\) and \(T_2\) go in the same direction on $C_0$.

\begin{lemma}\label{lem:orientable_traces}
	Let $G$ be a graph embedded in an orientable surface $\Sigma$. Let $(G_P, E_N)$ be a planar-nonplanar decomposition of $G$. Then every homotopy class of $E_N$ is orientable. 
\end{lemma}
\begin{proof}
	Let $H$ be a homotopy class of edges in $E_N$. Since $H$ is a homotopy class, there exists a closed disk $D_1$ on $\Sigma$ of which every edge in $E_N$ is on. Furthermore, the maximal traces $T_1, T_2$ of $H$ are on the boundary of $D_1$. Add a loop $L$ around $G_P$ which bounds a closed disk $D_0$ containing $G_P$. $L$ intersects every edge in $E_N$ twice. Add vertices at every intersection point, and add edges between consecutive vertices on $L$. Now add an orientation to $D_0$, suppose this orientation goes anticlockwise. Then this forces $D_1$ to go anticlockwise as well, meaning that $T_1$ is going in the opposite direction to $T_2$ on the boundary of $T_1$. Then $T_1$ and $T_2$ are orientable. As this holds for every homotopy class of $H$, $T_1$ and $T_2$ are orientable. 
\end{proof}

\begin{lemma}
	Let $G$ be a graph. If \(G\) is \(2\)-cell embedded on a surface of Euler genus \(g\), then any planar-nonplanar decomposition has at most \(3g-3\) homotopy classes. 
\end{lemma}
\begin{proof}
	Decompose \(G\) to a planar-nonplanar decomposition of \(G\), \((G_P, E_N)\). Suppose \(E_N \neq \emptyset\). Identify \(G_P\) to a single point, and identify each homotopy class to a single edge. Draw a circle $S$ that bounds a disk around \(G_P\), and place new vertices where $S$ intersects an edge in $E_N$. Add edges between vertices on $S$. Delete \(G_P\), and call the new graph \(H\) embedded on $\Sigma$. Let \(n = |V(H)|\), \(m = |E(H)|\). Let \(h\) be the number of homotopy classes, and \(f\) is the number of faces on this embedding of $H$. Since $H$ is connected, we can apply Euler's formula. So, \(n - m + f = 2 - g\). Since \(H\) is cubic as every vertex has two edges on $S$ and one on the homotopy class, then \(3n = 2m\) by the handshaking lemma. Since there is only one nonplanar edge for each homotopy class, and every homotopy class is incident to two edges, \(n = 2h\). The interior face of \(H\) has \(v\) incident edges, and the remaining \(f-1\) faces have at least 3 incident edges each. If a face has four edges, then the two nonplanar edges are homotopy equivalent. Then every face has at least six edges. Every edge is incident to two faces. Place a 'dot' on the left and right side of each edge. $f-1$ faces get at least $6$ dots and $1$ face gets $n$ dots. Each edge has two dots. Therefore, \(  6(f-1) + n \leq 2m \). Use the previous facts that $n = 2h$, $3n = 2m$, and $n - m + f = 2 - g$. After an algebraic manipulation, \(h \leq 3g - 3 \). 
	\begin{align*}
		3n  & \geq 6(f - 1) + n         && \text{from } 2n \leq 3m \\
		2n  & \geq 6f - 6               \\
		4h  & \geq 6(2 - g + m - n) - 6 && \text{from } n = 2h \text{ and } f = 2 - g + m - n\\
		-2h & \geq 6 - 6g               \\
		h   & \leq 3g - 3
	\end{align*}
\end{proof}

Let \(G\) be a graph embedded on some surface $\Sigma$. Let $F$ be a face on $\Sigma$ and let $S$ be a walk on the boundary of $F$. $S$ induces many circular orderings on the boundary of $F$. If a vertex $v$ appears multiple times in $S$, then $v$ is a cut vertex. A sub-ordering of $S$ is a sub-walk of $S$ where if a vertex $v$ appears multiple times in $S$, then $v$ appears exactly once restricted to where $v$ previously appeared. A vertex ordering \((<)\) \textit{preserves} \(F\) if $<$ restricted to the boundary of $F$ is equal to some sub-ordering of $S$.

\begin{lemma}\label{lem:planar_nonplanar_orientable}
	Suppose a graph \(G\) has a planar-nonplanar decomposition \((G_P, E_N)\) on an Euler genus $g \geq 1$ orientable surface \(\Sigma\). Then \(G\) can be embedded on \(18g + 2\) pages.
\end{lemma}
\begin{proof}
	Use \cref{thm:Planar Graph Hickingbotham Bound} to embed $G_P$ on four pages with the property that the outer-cycle is preserved. Let $H$ be a homotopy class in $E_P$.
	Take a block-cut tree $T$ of $G_P$. Edges in $H$ that go between the same block go on the same page. These edges do not cross because of \cref{lem:orientable_traces}. Now use the property from \cref{claim:planar_nonplanar_decomp} that every edge in $H$ that is not on the same block is adjacent to either $x$ or $y$. Colour all the edges adjacent to $x$ a new colour and all adjacent to $y$ the same colour. In total, there are three colours. 
	Therefore, \(11 + 3(6g - 3) = 18g + 2\) pages suffice if \(G\) has a planar-nonplanar decomposition.
\end{proof}

% !TEX root = ./thesis.tex

\chapter{Potential proof techniques}\label{chap:Proving_The_Theorem}
Our aim is to show that graphs with bounded genus $g$ containing $p$ vortices of bounded width $k$ have bounded pagenumber $f(g, p, k)$. Thus we can show that for fixed $t$, all $K_t$-minor free graphs have bounded pagenumber $f(g, p, k, a, \ell)$. However, from \cite{hickingbothamStackNumberCliqueSum2023}, we can show that the clique-sums of graphs of bounded genus also have bounded genus.

We wish to find a book-embedding of a graph $G$ of bounded genus $g$ with vortices $G_1, ..., G_p$ of adhesion $k$ such that the pagenumber of $G$ is at most $f(g, p, k)$ for some constants $g$ and $p$. We do not worry about the $a$-case now.

\section{Planar graphs}
From Heath and Istrail, we can form a planar-nonplanar decomposition of $G$ of bounded genus $g$. Therefore, it makes sense to think about planar graphs first before thinking about graphs with bounded genus $g$.

We wish to prove the following:

\begin{conjecture}\label{conj:4-planar graphs}
	For all 4-connected planar graphs $G$ with an embedding $\Sigma$, with $i$ distinguished faces $\left\lbrace F_1, ..., F_i \right\rbrace$ and a distinguished clique $C$ of size at most $3$, there exists a book-embedding $(<, \psi)$ such that all edges bounding the faces $F_i$ are ``almost'' monochromatic and $C$ is at the start of the embedding with pagenumber at most $f(i)$.
\end{conjecture}

\subsection{Graphs of bounded genus}

However, this conjecture may be too difficult to prove. Then we can consider a book-embedding of graphs $g$ of bounded genus, and consider what happens if the disks were manipulated only slightly, with a bounded number of vertices on the boundary of the disks moved around. Then from using Heath and Istrail's algorithm to build the homotopy classes, we can add on the disks to the graphs of bounded genus. However, we do not know any good way of making this work properly. 
\section{Apex vertices}
In this section, we will prove that apex vertices can be added with a bounded increase to the number of pages.
\begin{theorem}
	If $G$ is a graph with components $G'$ and $A$ apex vertices and $G'$ is a graph with pagenumber $s$, $a = |A|$, then $G$ has pagenumber $s + \left\lceil \frac{3a}{2}\right\rceil$. 
\end{theorem}
\begin{proof}
	Let $G'$ have book-embedding $(<, \rho)$. Then place the vertices of $A$ at the very start of $(<)$ and for every edge $u_iv$, $u_i \in A$, $v \in G'$, we colour $\rho(uv) = i$. Then for any edge $e \in E(G')$, we maintain the same colour as before. Then for the edges between vertices in $A$, we have that the number of colours is bounded above by $\left\lceil \frac{a}{2} \right\rceil$ from \cref{thm:Pagenumber_Complete_Graph}. Therefore, we have that $\pn(G) \leq \pn(G') + a + \left\lceil \frac{a}{2} \right\rceil =s + \left\lceil \frac{3a}{2}\right\rceil$. 
\end{proof}

\chapter{Conclusion}\label{chap:conclusion}
We conclude this report by explaining our current progress and our future plans.
The main important result explained is the Graph Minor Structure Theorem~\cite{robertsonGraphMinorsXVI2003} and its application in our particular problem involving \(K_t\)-minor free graphs. We apply the Graph Minor Structure Theorem to the problem and explain some partial results.
We also go through some other important results in \(K_t\)-minor free graphs and pagenumbers, including Heath and Istrail's \cite{heathPagenumberGenusGraphs1992} paper on graphs on surfaces and bounding the pagenumber.
The main result we have attempted to show is the case where graphs are almost embedded on an orientable surface. We can show that if a graph is almost-embeddable on an orientable surface, then the pagenumber of the graph is bounded.
What remains to show is if a graph is almost-embeddable on a non-orientable surface.
We aim to prove this conjecture:
\begin{conjecture}
	Let $G$ be a graph where all torsos are $k$ almost-embeddable on a surface of genus at most $g$. Now suppose there are $p$ vortices which are embedded on $G$. Then $G$ has bounded pagenumber.
\end{conjecture}

The non-orientable analogue of the orientable case is:
\begin{conjecture}
	Let $G$ be a graph embedded on a non-orientable surface of Euler genus $g$. Let $F_1, \ldots, F_p$ be faces on $G$ which are homeomorphic to a disk. Then there exists a book-embedding $(<, \leq)$ such that for all $F_i$, $1 \leq i \leq p$, we have that $F_i$ has at most $f(g)$ monochromatic paths.
\end{conjecture}
We aim to replicate the result from \cref{lem:orientablesurfaces_monochromatic_edges} in the non-orientable case and remove the word ``orientable'' from the statement.
This will be much more difficult as we do not have the same tools for orientable surfaces. Heath and Istrail's proof for non-orientable surface page numbers is much different than the orientable case, and we will need to look at that proof in depth and extend their results for our case.

\printbibliography{}
\end{document}
