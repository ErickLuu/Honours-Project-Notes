\chapter{Introduction}\label{sec:introduction}
Structural graph theory is a fundamental topic in graph theory and its study has led to a deeper understanding of graphs. Many results from structural graph theory decompose graphs into ones with bounded parameters. One of the most important theorems in structural graph theory is Robertson and Seymour's Graph Minor Theorem \cite{robertsonGraphMinorsXX2004} which states that proper minor-closed families of graphs are characterised by a finite set of forbidden minors. 
\par
The concept of the \textit{pagenumber} of a graph was introduced by Ollmann \cite{ollmannBookThicknessVarious1973} in the context of VLSI design and integrated circuitry. A \textit{book-embedding} of a graph is a way to arrange the vertices on the ``spine'' of a book and arrange the edges on ``pages'' of a book, or half-planes. The \textit{pagenumber} of a graph $G$ is the smallest number of pages necessary in a book-embedding of $G$. 

The driving question of this report is the following: 
\begin{conjecture}\label{conj:bded_had_pn}
	Given a graph $G$ with no $K_t$ minor, is the pagenumber of $G$ bounded by a function on $t$, so $\pn(G) \leq f(t)$ for some $t$?
\end{conjecture}
Answering this question will yield a link between the pagenumber of a graph and the global structure of the graph. This report currently lays out the literature related to this question. We use a result in one of the papers of the Graph Minor Theorem, which is the Graph Minor Structure Theorem \cite{robertsonGraphMinorsXVI2003}.

In a PhD thesis, Blankenship claimed to have a proof of \cref{conj:bded_had_pn}.\cite{Blankenship-PhD03} However, this result has not been published in any journal and has not been independently verified. We aim to fill this gap in our knowledge. 
\subsection{Plan for solving the problem}
We aim to solve the question using the Graph Minor Structure Theorem \cite{robertsonGraphMinorsXVI2003}, which describes the structure of graphs which do not contain a $K_t$ minor.
\par
Robertson and Seymour showed that we can build graphs with no $K_t$ minor from smaller building blocks. We first start with a graph $G$ embedded on a genus $g$ surface. Then we add on $p$ \textit{vortices} to $G$, with \textit{pathwidth} at most $k$. Then we add on $a$ \textit{apex vertices} to $G$. We say that $G$ is $(g, p, k, a)$-\textit{almost embeddable}. Robertson and Seymour \cite{robertsonGraphMinorsXVI2003} proved that all graphs with no $K_t$ minor are \textit{clique-sums} of $(g, p, k, a)$ almost-embeddable graphs, with $(g, p, k, a)$ depending on only $k$. 
\par
We have some useful results that can be paired with the Graph Minor Structure Theorem to prove this result.
\begin{itemize}
	\item From Heath and Istrail, all graphs of bounded genus have bounded pagenumber \cite{heathPagenumberGenusGraphs1992}.
	\item From Ganley and Heath \cite{ganleyPagenumberTrees2001}, and Dujmovic and Wood \cite{dujmovicGraphTreewidthGeometric2007}, all graphs of bounded treewidth have bounded pagenumber.
	\item From Hickingbotham and Wood\cite{hickingbothamStackNumberCliqueSum2023}, if a graph $G$ has a \textit{tree-decomposition} where every \textit{torso} has bounded pagenumber, then $G$ has bounded pagenumber. 
\end{itemize}

\subsection{Layout of report}
\begin{itemize}
	\item \cref{chap:Definitions} formally describes important definitions and concepts that will be used throughout the rest of the report.
	\item \cref{chap:Known results} discusses some known results from graph theory, including the Graph Minor Structure Theorem. We discuss some proofs related to bounded pagenumber that can be used to prove \cref{conj:bded_had_pn}.
	
	\item \cref{chap:Proving_The_Theorem} discusses some new avenues towards proving \cref{conj:bded_had_pn} and some relatively easy proofs of some aspects of the conjecture. 
\end{itemize}