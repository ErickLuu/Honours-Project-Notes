\documentclass[]{beamer}
\usepackage{amsmath}
\usepackage{amssymb}
\usepackage{svg}
\usepackage{todonotes}
\usepackage{tikz}
\usepackage{svg}

\newtheorem{conjecture}[theorem]{Conjecture}

\usepackage[mode=buildnew,subpreambles=true]{standalone}
\makeatletter
\setbeamertemplate{navigation symbols}{}
\setbeamertemplate{footline}
{%
	\leavevmode%
	\hbox{%
		\begin{beamercolorbox}[wd=1\paperwidth,ht=2.25ex,dp=1ex,center]{author in head/foot}%
		\end{beamercolorbox}%
	}%
	\vskip0pt%
}
\makeatother

%\setbeameroption{show notes on second screen}
\setbeamertemplate{page number in head/foot}{}
\setbeamertemplate{title in head/foot}{}
\setbeamertemplate{author in head/foot}{}
%opening
\title{Book embeddings of $K_t$-minor free graphs}
\author{Eric Luu}
\newcommand{\ops}{\overset{\text{ops}}{\leftrightarrow}}
\begin{document}

\frame{\titlepage}

%\begin{frame}{$K_t$-minor-free graphs, $ 1 \leq t \leq 5$}
%	\begin{itemize}
%		\item $K_1$-minor-free graph is the empty graph
%		\item $K_2$-minor-free graphs have no edges
%		\item $K_3$-minor-free graphs are forests
%		\item $K_4$-minor-free graphs are graphs whose biconnected components are series parallel graphs
%		\item $K_5$-minor-free graphs are almost planar graphs.
%	\end{itemize}
%\end{frame}
\begin{frame}[allowframebreaks]{Book-Embedding}
	\begin{figure}[h]
		\begin{minipage}{0.5\textwidth}
			\includestandalone[width = \linewidth]{Drawings/Book_Embedding}
		\end{minipage}%
		\begin{minipage}{0.5\textwidth}
			\begin{block}{Books}
				\begin{itemize}
					\item A \textit{book} is a collection of pages glued together on a common boundary.
					\item A \textit{book-embedding} of a graph is where the vertices are on the common boundary and the edges are planar.
				\end{itemize}
			\end{block}
		\end{minipage}
	\end{figure}
	
\note{	Embed graph $G$ on series of half-planes identified on edge where:
	\begin{itemize}
		\item $V(G)$ lies on boundary of half-planes
		\item $E(G)$ lies on a single half plane so that no two edges cross
	\end{itemize}
	Half planes are \textit{pages}, vertices lie on spines of a book. 
	So book embedding has a vertex ordering $(\leq)$ and a function $\varphi: E(G) \rightarrow \lbrace 1, 2, ..., n \rbrace$ where we assign edges to pages.
	Can use cyclic ordering to not worry about the first vertex. Many proofs are agnostic to the first starting vertex.}
\end{frame}

\begin{frame}
	\begin{itemize}
		\item Pagenumber of $G$ is the least number of pages needed to embed $G$.
		\item $G$ has pagenumber 1 iff $G$ is outerplanar ( all vertices on one face )
		\item $G$ has pagenumber 2 iff $G$ is Hamiltonian planar.
	\end{itemize}
\end{frame}

\begin{frame}{Usage of pagenumber}
	\begin{itemize}
		\item Book-thickness was investigated in 1970s
		\item Applications to chip design, knot theory, abstract algebra
	\end{itemize}
\end{frame}

\begin{frame}{Embedding graph on surfaces}
	Genus of a surface is the number of holes and crosscaps. 
	\begin{theorem}[Heath and Istrail]
		Graphs embedded on surfaces of bounded genus have bounded pagenumber.
	\end{theorem}
	\begin{figure}[h]
		\includesvg{Drawings/TorusGridGraph3DEmbeddings_1000.svg}
		\caption{Torus graph (Wolfram Mathworld)}
	\end{figure}
\end{frame}

\begin{frame}{Graph minors}
	$H$ is a \textit{graph minor} of $G$ if $H$ can be obtained from $G$ by contracting edges, deleting edges, and deleting vertices.
	
	$G$ is $H$-free if $H$ is not a minor of $G$. 
	\note{Draw picture of a minor on whiteboard}
\end{frame}

\begin{frame}
	$K_t$ is the complete graph on $t$ vertices.
	\begin{definition}[$K_t$-minor free]
		A graph $G$ is $K_t$-minor free if $K_t$ is not a minor of $G$.
	\end{definition}
	\begin{figure}[h]
		\centering
		\includestandalone[width=0.4\textwidth]{Drawings/P10}
		\includestandalone[width=0.4\textwidth]{Drawings/P10_Contraction_To_K5}
		\caption{The Petersen graph has a $K_5$ minor.}\label{P10contraction}
	\end{figure}
\end{frame}

\begin{frame}{Research question}
	\begin{conjecture}
		Is there a function $f(t)$ such that if $G$ is $K_t$-minor free, $G$ is embeddable on $f(t)$ pages?
	\end{conjecture}
	There are reasons to believe the answer is yes!
	\begin{itemize}
		\item Graphs which are $K_t$-minor free are almost like graphs with bounded genus.
	\end{itemize}
\end{frame}

%\begin{frame}{$K_t$-minor free graphs}
%	\begin{conjecture}
%		$K_t$-minor free graphs have bounded pagenumber.
%	\end{conjecture}
%\end{frame}


%\begin{frame}{Conclusion}
%	Book thickness is a useful graph parameter
%	Generalises planarity
%	Bounded book thickness- interesting problems
%\end{frame}
\note{Add notes on Heath + Istrail paper}

\note{Do not discuss Robertson-Seymour, will take up 40 minutes}
\end{document}
