\documentclass[]{beamer}
\usepackage{amsmath}
\usepackage{amssymb}
\usepackage{svg}
\usepackage{todonotes}
\makeatletter
\setbeamertemplate{navigation symbols}{}
\setbeamertemplate{footline}
{%
	\leavevmode%
	\hbox{%
		\begin{beamercolorbox}[wd=1\paperwidth,ht=2.25ex,dp=1ex,center]{author in head/foot}%
		\end{beamercolorbox}%
	}%
	\vskip0pt%
}
\makeatother

%\setbeameroption{show notes on second screen}
\setbeamertemplate{page number in head/foot}{}
\setbeamertemplate{title in head/foot}{}
\setbeamertemplate{author in head/foot}{}
%opening
\title{Book embeddings of $K_t$-minor free graphs}
\author{Eric Luu}
\newcommand{\ops}{\overset{\text{ops}}{\leftrightarrow}}
\begin{document}

\frame{\titlepage}
\begin{frame}{Graph minors}
	$H$ is a \textit{graph minor} of $G$ if $H$ can be obtained from $G$ by contracting edges, deleting edges, and deleting vertices.
	\begin{definition}[$K_t$-minor free]
		A graph $G$ is $K_t$-minor free if $K_t$ is not a minor of $G$.
	\end{definition}
	\begin{figure}[h]
		\centering
		\includesvg[scale=0.2]{Drawings/P10.svg}
		\includesvg[scale=0.2]{Drawings/P10 contraction to K5}
		\caption{The Petersen graph has a $K_5$ minor.}\label{P10contraction}
	\end{figure}
	\note{The Petersen graph is $K_6$-minor-free as there is no way to contract edges such that there is a $K_6$ minor. Note the Petersen graph has 10 vertices, so if there was a $K_6$ minor, there would have to be a vertex which does not get contracted. But all vertices have degree 3 so it would not be $K_6$.}
\end{frame}

\begin{frame}{$K_t$-minor-free graphs, $ 1 \leq t \leq 5$}
	\begin{itemize}
		\item $K_1$-minor-free graph is the empty graph
		\item $K_2$-minor-free graphs have no edges
		\item $K_3$-minor-free graphs are forests
		\item $K_4$-minor-free graphs are graphs whose biconnected components are series parallel graphs
		\item $K_5$-minor-free graphs are almost planar graphs.
	\end{itemize}
\end{frame}
\begin{frame}[allowframebreaks]{Pagenumber}
	Embed graph $G$ on series of half-planes identified on edge where:
	\begin{itemize}
		\item $V(G)$ lies on boundary of half-planes
		\item $E(G)$ lies on a single half plane so that no two edges cross
	\end{itemize}
	Half planes are \textit{pages}, vertices lie on spines of a book. 
	So book embedding has a vertex ordering $(\leq)$ and a function $\varphi: E(G) \rightarrow \lbrace 1, 2, ..., n \rbrace$ where we assign edges to pages.
	Can use cyclic ordering to not worry about the first vertex. Many proofs are agnostic to the first starting vertex.
	\begin{block}{Pagenumber}
		The \textit{pagenumber} (or book-thickness) of $G$ is the smallest number of pages needed to embed a graph $G$.
	\end{block}
\end{frame}

\begin{frame}{Examples}
	\begin{itemize}
		\item $G$ has pagenumber 1 iff $G$ is outerplanar ( all vertices on one face )
		\item $G$ has pagenumber 2 iff $G$ is Hamiltonian and planar
	\end{itemize}
	Easy examples, much harder to classify beyond.
\end{frame}

\begin{frame}
	Yannikakis \note{Add citation}
	All planar graphs can be embedded in 4 pages.
\end{frame}
\note{Discuss easier example, look at Shor paper}

\note{Add notes on Heath + Istrail paper}

\note{Do not discuss Robertson-Seymour, will take up entire 20 minutes}
\end{document}
