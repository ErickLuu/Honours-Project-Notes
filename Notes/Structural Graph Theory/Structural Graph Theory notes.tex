\documentclass[]{article}
\usepackage[margin = 1in]{geometry}

\usepackage{amsmath}
\usepackage{amssymb}
\usepackage{amsthm}
\usepackage{url}

% Commands
\newcommand{\tree}{\mathcal{T}}
\newcommand{\tw}{\text{tw}}
\newcommand{\had}{\text{had}}
\newcommand{\pw}{\text{pw}}
\newcommand{\td}{\text{td}}
% Environments

\newtheorem{theorem}{Theorem}
\newtheorem{proposition}[theorem]{Proposition}
\newtheorem{corollary}[theorem]{Corollary}
\newtheorem{lemma}[theorem]{Lemma}
\newtheorem{definition}[theorem]{Definition}
\newtheorem{conjecture}[theorem]{Conjecture}

\theoremstyle{definition}
\newtheorem{example}[theorem]{Example}

\numberwithin{theorem}{section}
\numberwithin{equation}{section}

%opening
\title{Structural Graph theory}
\author{Eric Luu}

\begin{document}

\maketitle

\section{Treewidth}

\begin{definition}[Tree-decomposition]
	The tree-decomposition $\tree$ of a graph $G$ is defined as a tree $T$ with associated \textit{bags} $\lbrace B_x : x \in V(T) \rbrace$ such that:
	\begin{itemize}
		\item for all $v \in V(G)$, the subset of vertices $\lbrace x \in V(T): v \in B_x \rbrace$ in $V(T)$ induces a connected subtree in $V(T)$.
		\item For all edges $vw \in E(G)$, there exists a bag $B_x$ such that both $v$ and $w$ are in the bag $B_x$.
	\end{itemize}
\end{definition}
We refer to the vertices of the tree $T$ as \textit{nodes}. 
The \textit{width} of the tree decomposition $\tree$ is defined as $\max \lbrace |B_x| - 1 : x \in V(T) \rbrace$. We define the \textit{treewidth} of a graph $G$ as such:


\begin{definition}
	The treewidth of a graph $G$, denoted as $\tw(G)$, is defined to be the smallest width for all tree decompositions of the graph $G$.
\end{definition}
The reason why the $-1$ appears in the definition of the with of a tree decomposition is because the definition wanted the treewidth of a forest to be 1. However, this causes some notational confusion.
\begin{example}
	$\tw(G) = 1$ iff $G$ is a forest.
	\begin{lemma}
		If $G$ is a forest, then $\tw(G) = 1$.
	\end{lemma}
	\begin{proof}
		Suppose $G$ is a tree. Root the graph $G$ at the vertex $r$. Then let $T = G$ and $B_x:= \lbrace x, p \rbrace$ where $p$ is the parent of $x$. The bag $B_r$ will just contain $r$. Then all edges $vw$ will be between parent $v$ and child $w$, so it will be in bag $B_w$. Finally, the subgraph induced by vertex $x$ in $T$ will be $x$ and the children of $x$, which is a connected subtree.
		
		If $G$ is a forest, then we perform this operation on every connected component of $G$ and connect the roots to form a new tree. Then this tree is a tree-decomposition. This forms a tree-decomposition of width at most 1. 
	\end{proof}
	\begin{lemma}
		If $\tw(G) = 1$, then $G$ has no cycles.
	\end{lemma}
	\begin{proof}
		If $G$ has a cycle $C$, then the treewidth cannot be 1. This is because if there is a tree decomposition $\tree$ where the size of each bag is at most 2, then as the graph must have every edge, then every edge in $C$ is in separate bags. However, we have that for any vertex $v$ in $C$ to have an induced connected subgraph in $T$, then it follows that the cycle $C$ is also in $T$. Thus $T$ is not a tree, and this is not a valid tree-decomposition. 
	\end{proof}
\end{example}
\begin{lemma}[Helly Property]
	Let $T_1, ..., T_k$ be subtrees of a tree $T$ such that for every pair of trees, there is a vertex in common. Then there exists a vertex which is common to all trees.
\end{lemma}
\begin{proof}[Helly property]
	If $T_1$, $T_2$ and $T_3$ are subtrees of $T$ such that the vertex sets are pairwise nonempty, then there is a common vertex in all three subtrees. If this is not the case, denote $v_1$ as a vertex in the intersection of $T_1$ and $T_2$, $v_2$ as the vertex in $T_1 \cap T_3$, and $v_3$ as the vertex in $T_2$ and $T_3$. Then there exists a unique path $P$ in $T_1$ from $v_1$ to $v_2$. Choose two vertices $x$ and $y$ on $P$ such that they are disjoint....
\end{proof}

\begin{theorem}[Clique theorem]
	In any tree-decomposition of $G$, for every clique $C$ in $G$, there exists a node $x \in V(T)$ such that $C \subseteq B_x$. 
\end{theorem}

\begin{proof}
	Let $\tree$ be a tree-decomposition. Every vertex $v$ induces a connected subtree in $T$, call it $T_v$. Then for any two vertices $x, y$ in $C$, we have that $T_x$ and $T_y$ must intersect as the edge $xy$ is inside a bag $B_z$ corresponding to a node $z$. Then by the Helly property, there exists a node $v$ such that $C \subseteq B_v$.
\end{proof}

\begin{corollary}
	$\tw(K_n)$ is $n-1$. 
\end{corollary}

\begin{theorem}
	If $H$ is a minor of $G$, then $\tw(H) \leq \tw(G)$. 
\end{theorem}
\begin{proof}[Proof of minor]
	Suppose we have a tree-decomposition $\tree$ of $G$. If we delete an edge in $G$, then $\tree$ remains a valid tree-decomposition. If we delete a vertex $v$, then $\tree$ where we remove $v$ from every bag in $\tree$ is also a valid tree-decomposition. If we contract an edge $vw$, creating a new vertex $u$, then relabeling $v$ and $w$ in all bags to $u$ is a valid tree-decomposition as the induced subtree of $u$ is the union of the induced subtrees of $v$ and $w$, and every neighbor of $v$ or $w$ is a neighbor of $u$. But the edges in the neighborhood do not change. Thus this is a valid tree-decomposition, with width at most the width of $\tree$.
\end{proof}

\begin{example}
	The treewidth of an outerplanar graph is at most 2.
\end{example}
\begin{proof}[Proof of outerplanar treewidth.]
	Let $G$ be the outerplanar graph, and let $G'$ be the triangulation of $G$. As $G$ is a minor of $G'$, $\tw(G) \leq \tw(G')$. We look at the \textit{weak dual} of $G'$. This is a tree $T$, where every node $v_f$ in $T$ corresponds to a face $f$ in $G'$. Then let $B_{v_f}$ be the bag of the tree-decomposition, where $B_{v_f}$ is the set of vertices on the boundary of the face $f$. Then the tree $T$ with bags $B_{v_f}$ is a valid tree-decomposition of $G'$, where every bag has at most 3 vertices. Thus, $\tw(G) \leq 2$. 
\end{proof}

\subsection{Different characterisations of treewidth}
\subsubsection{$k$-trees}
We define a $k$-tree inductively. We have that the complete graph $K_k$ is a $k$-tree, and if $G$ is a $k$-tree, then we add a new vertex to $G$ that is adjacent to $k$ vertices that form a clique of size $k$ in $G$ results in a $k$-tree. 
A $k$-tree is a maximal graph with treewidth $k$. $\tw(G) \leq k$ iff $G$ is a subgraph of a $k$-tree. 

\section{Separators}
A subset $X$ of $V(G)$ is a \textit{balanced separator} of $G$ if each component of $G - X$ has at most $|V(G)|/2$ vertices. This implies that we can partition the vertices of $G$ into sets $A$ and $B$ such that there are no $AB$-edges and the size of $A$ and $B$ is at most $2/3 |V(G)|$. This is because we can order the components from smallest to largest and partition them into sets $A$ and $B$ where the sizes are at most $2/3 |V(G)|$.

\begin{theorem}
	For all graphs $G$, there exists a balanced separator of size $\tw(G) + 1$. 
\end{theorem}
\begin{proof}[Proof of balanced separator]
	We take a tree-decomposition $\tree$ of treewidth $\tw(G) - 1$. For any edge $xy$ in $T$, denote the largest subtree containing $x$ that does not contain $y$ as $T_{x,y}$, and similarly denote $T_{y, x}$ as the same thing. If the size of the union of the corresponding bags of the nodes of $T_{x,y}$ is larger than the size of the union of bags in $T_{y, x}$, orient the edge $xy$ to point from $y$ to $x$, otherwise orient it the other way. Do this for every edge. Then let $x$ be the node where all arrows are pointing inwards, and let $B_x$ be the corresponding bag. Then $B_x$ is a separator of $G$ as we have that at most $|V(G)|/2$ vertices are in any component of $T$ by definition. Thus $B_x$ is a balanced separator of $G$. 
\end{proof}

\subsection{Subset theorems}
\begin{theorem}
	For all graphs $G$, and all subsets $S$ of $V(G)$, there exists an $X$ where $|X| \leq \tw(G) + 1$ and each component of $G - X$ has $\leq |S|/2$ vertices in $S$. 
\end{theorem}
\begin{proof}
	Do the steps above but instead of weighing each vertex the same, you weigh a vertex $v$ to be 1 if it is in $S$ and 0 if it is not. 
\end{proof}

\begin{theorem}
	For all graphs $G$, and all subsets $S$ of $V(G)$, there exists  two subgraphs $G_1$ and $G_2$ such that $G = G_1 \cup G_2$ and for all $i \in \lbrace 1, 2 \rbrace$, $|S \cup V(G_i) | \leq 2/3 |S|$.
\end{theorem}
\begin{proof}
	Use the theorem above. Then we can form $G_1$ and $G_2$ to have at most $2/3 |S|$ the number of vertices in $|S|$, by sorting the subsets by the number of vertices in $S$. 
\end{proof}

\subsection{Bounds on treewidth.}
\begin{theorem}
	Let $G$ be a graph such that for all subsets $S \subseteq V(G)$ there exists another subset $x \subseteq V(G)$ such that $|X| \leq k$ and each component of $G - X$ has at most $|S|/2$ vertices in $S$. Then $\tw(G) \leq 3k$. 
\end{theorem}

\begin{lemma}
	Let $G$ be a graph such that for all subsets $S \subseteq V(G)$ of size $2k + 1$ there exists an $X \subseteq V(G)$ such that $|X| \leq k$ and each component of $G - X$ has at most $k$ vertices in $S$. Then For all $S \subseteq V(G)$ where $|S| \leq 2k+1$ there exists a tree-decomposition of $G$ with width at most $3k$ and there exists a bag containing $S$. 
\end{lemma}
\begin{proof}
	Suppose $|V(G)| \leq 3k + 1$. Then place all of the vertices in a single bag. Then this is a valid tree-decomposition with width at most $3k$ containing all $S$. 
	Now assume $|V(G)| \geq 3k + 2$ and $|S| = 2k + 1$. If $S \leq 2k + 1$, add arbitrary vertices to $S$. Then there exists a subset $X \subseteq V(G)$ such that $|X| \leq k$ and each component of $G - X$ has at most $k$ vertices in $S$. Let the components of $G - X$ be $G_1, G_2, ... G_p$. Then we can do induction on $(G_i, S_i \cup X)$ to have a tree-decomposition of $G_i$ with width at most $3k + 1$. Then for each of the tree-decompositions rooted at the node with bag containing $S_i \cup X$, we add on a parent vertex to all of those tree-decompositions of $X \cup S$ with width at most $3k + 1$.  This is a tree-decomposition of $G$ of width at most $3k + 1$ with the root vertex containing $S$ by definition. 
\end{proof}

\section{Tree-partitions}
\newcommand{\tpw}{\text{tpw}}

For a graph $G$, a \textit{tree-partition} of $G$ is a tree $T$ with associated partition of the vertices of $G$ into bags $ \lbrace B_x : x \in V(T) \rbrace$ such that if $vw$ is an edge in $G$, then $v$ and $w$ are in the same bag, or the edge $xy$ is in $E(T)$, where vertices $x$ and $y$ have corresponding bags $B_x$ and $B_y$ containing $v$ and $w$ respectively. The \textit{width} of the tree-partition is defined as the largest bag in the tree-partition. The tree-partition width of a graph $G$, denoted as $\tpw(G)$, is the smallest width of all tree-partitions. 


\begin{theorem}[Distel + Wood]
	For all graphs $G$, $\tpw(G) \leq 18 (\tw(G) + 1) \Delta(G)$. 
\end{theorem}

\begin{lemma}
	Fix $k$ and $d$. Let $G$ be a graph where $\tw(G) \leq k - 1$ and $\Delta(G) \leq d$. Then for any set $S \subseteq V(G)$ and $4k \leq |S| \leq 12kd$, there exists a tree-partition $\left( B_x : x \in V(T) \right)$ with width at most $18kd$ and bag $B_z$ where $deg_{T}(z) \leq \frac{|S|}{2k} - 1$ and $|B_z| \leq 3/2 |S| - 2k$.
\end{lemma}
\begin{proof}
	
	Case 0: $|V(G)| < 4k$: We place all the vertices in the same bag. Size of bag is $< 4k$, so the bag is definitely less than $18kd$. 
	\par
	Case 1: $|V(G) - S| \leq 18kd$. Let $T$ be the tree on two vertices ${x, z}$, where $B_x = V(G) - S$ and $B_z = S$. Then we have that $\delta(T) = 1$ and $deg_T(z) = 1$, which satisfies the requirements above. We have that $|B_z| \leq 3/2 |s| - 2k$ and $deg_T(z) \leq |S|/2k - 1$. 
	\par
	Case 2: $S$ small case. $4k \leq |S| \leq 12k$. Let $S' := \bigcup \lbrace N_G(v) - S : v \in S \rbrace$. Then $|S'| \leq d |S| \leq 12kd$. If $|S'| < 4k$, then add arbitrary vertices to $S'$ from $G - S - S'$ such that $|S'| = 4k$. Now $4k \leq |S'| \leq 12kd$. By the induction hypothesis, there exists a tree-partition of $G - S$ with width $\leq 18kd$ and $S'$ in one bag. Then we add the bag $B_z = S$ to the tree that is connected only to $S'$. We have that as $4k \leq |S|$, it implies that $|S| \geq 3/2 |S| - 2k$, so $|B_z| \leq 3/2 |S| - 2k$ and $|S| \leq 12k$. Finally, $deg_T(B_z) = 1 \leq |S|/2k - 1$.
	\par
	Case 3: $S$ large case. $12k + 1 \leq |S| \leq 12kd$. There exists induced subgraphs $G_1$, $G_2$ of $G$ where $G_1 \cup G_2 = G$ and $|G_1 \cap G_2| \leq k$, where $|S \cap V (G_i)| \leq 2/3 |S|$ for each $i$ in $\lbrace 1, 2 \rbrace$. Then let $S_i = (S \cap V(G_i)) \cup (G_1 \cap G_2)$ for each $i$ in $\lbrace 1, 2 \rbrace$. We have that $|S_2| \geq |S - V(G_1)| \geq 1/3 |S| \geq 4k$. By symmetry, $|S_1| geq 4l$. For an upper bound, $S_i \leq 2/3 |S| + k \leq 8kd + k \leq 12kd$. Therefore, $4k \leq |S_i| \leq 12kd$ for each $i$ in $\lbrace 1, 2 \rbrace$. Thus by induction, there exists a tree-partition of $G_i$ with width at most $18kd$, such that $\delta(T_i) \leq 6d$ and there is a $z_i$ such that $S_i \in B_{z_i}$, $|B_{z_i}| \leq 3/2 |S_i| - 2k$, $\deg_{T_i}(z_i) \leq |S_i|/2k - 1$. Then form the tree of $G$ by merging $z_1$ and $z_2$ together to form $z$, and let $B_z = B_{z_1} \cup B_{z_2}$. Then this is a tree-partition of $G$. By construction, $S \subseteq B_z$ and $|B_z| \leq |B_{z_1}| + |B_{z_2}| - |G_1 \cap G_2|$. Using the induction hypothesis, this is less than $18kd$, and the degree of $z$ is $|S|/2k - 1 < 6d$. Thus shown.  
\end{proof}

\section{$O(\sqrt{n})$-bounded treewidth}
\newcommand{\ltw}{\textit{ltw}}
A family of graphs $\mathcal{G}$ has $O(\sqrt{n})$ bounded treewidth if, as the number of vertices increases in $\mathcal{G}$, the treewidth is bounded above by a constant times $\sqrt{n}$. We shall show that all planar graphs have bounded treewidth, and can extend this definition to graph families of bounded genus and crossings. [Dujmovic, Morin, Wood] 

\subsection{Layered treewidth}
A \textit{layering} of a graph $G$ is a partition of the vertex set of $G$ into sorted sets $V_1, V_2, ..., V_k$ such that for all edges $vw \in E(G)$, if $v \in V_i$ and $w \in V_j$ then $|i - j| \leq 1$. The \textit{layered treewidth number} $\ltw(G)$ is defined as the smallest $k$ such that there exists a layering $V_1, V_2, ...$ of $V(G)$ and there is a tree-decomposition $(B_x: x \in V(T))$ and $|V_i \cap B_x| \leq k$ for all $i$ and all $x$. 
\begin{theorem}
	Planar graphs have $\ltw$ at most 3. 
\end{theorem}
\begin{proof}
	If $G$ is a planar triangulation, and $T$ is a bfs spanning tree, then the vertices ordered by distance from the root $r$ of $T$ is a layering of $G$. Then consider the dual graph $G^*$. Then there exists a spanning tree of $G^*$, $T^*$, such that no edge in $T^*$ crosses over an edge in $T$. Call this spanning tree the cotree of $G$. Then let $\left( B_x: x \in V(T^*) \right)$ be bags. For a face $\alpha$, the set of vertices $v_1, v_2, v_3$ on the border of $\alpha$ and the vertices on path in $T$ from $v_i$ to $r$ for all $i$ in $\lbrace 1, 2, 3 \rbrace$ is $B_\alpha$. Every edge is on the border of some face, so every edge is in $T^*$. If $x$ is a vertex, then the subtree containing $x$ goes to all the faces incident with the descendants of $x$, which is connected. Thus this is a tree-decomposition. Finally, the intersection between $V_i$ and $B_\alpha$ is at most 3 as at most 3 vertices can be on the same layer by the construction. Thus $ltw(G) \leq 3$. 
\end{proof}

\subsection{Upper bound on treewidth}
If $G$ has $n$ vertices with layered treewidth $k$, then $G$ has treewidth $2 \sqrt{kn} - 1$.
\begin{proof}
	Let $V_1, V_2, ..., V_t$ be the layering of $G$ with layered treewidth $k$. Then define $p = \lceil \sqrt{nk} \rceil$. For $j \in \lbrace 1, ..., p \rbrace$, let $W_j = V_j \cup V_{p + j} + V_{2p + j}$, such that $W_j$ separates out the layers. As $W_1, ..., W_p$ is a partition of $G$, as there is a partition $W_j$ with the size at most the average such that $W_j \leq n/p \leq \sqrt{kn}$, then we can cut out the layers of $W_j$ to just have connected components with $p-1$ consecutive layers. Each connected component of $G - W_j$ can be subdivided with the tree-decomposition into bags of size at most $k(p-1)$, thus the treewidth of each connected component is $k(p-1) - 1 = \sqrt{kn} - 1$. Putting each connected component together, we add $W_j$ to every bag in the decomposition and add $W_j$ as another bag to turn the forest into a tree. This will give us a tree-decomposition of $G$ with width at most $\sqrt{kn} - 1 + |W_j| \leq 2 \sqrt{kn} - 1$. 
\end{proof}

\begin{example}
	If $G$ is a $n \times n$ grid, then $\tw(G) = n$, but $\ltw(G) = 2$. 
\end{example}

\begin{theorem}
	If $G$ is a triangulation of a surface with Euler genus $g$, then $G$ has layered treewidth at most $2g + 3$. 
\end{theorem}

\begin{proof}
	We have that $|E(G)| = 3n + 3g - 6$ and $|F(G)| = 2n + 2g - 4$. If $T$ is a $BFS$ spanning on $G$, and $G^*$ is its dual, then we define a new graph $D$ such that $V(D) = V(G^*) = F(G)$ and $xy$ is an edge in $D$ iff $xy$ does not cross an edge of $T$. Note that $D$ is not a tree if $g > 0 $. Finally, the number of edges in $D$ is equal to $|E(G)| - |E(T)| = 3n + 3g - 6 - (n-1) = 2n + 3g - 5$. By definition, $D$ is connected, so let $T^*$ be any spanning tree of $G$, rooted at $r$. $|E(T^*)| = 2n + 2g - 5$, so $|E(D)| - |E(T)| = g$. Denote the $g$ edges as $v_1w_1, v_2w_2, ..., v_gw_g$. As $D$ has $g$ edges not in $T^*$, for every face $f = xyz$ of $G$, let $B_f := P_x \cup P_y \cup P_z \bigcup_{i = 1}^g P_{v_i} \cup P_{w_i}$, where $P_x$ is the unique path in $T$ from $x$ to $r$. Thus $\ltw(G) \leq 2g + 3$. 
\end{proof}
\begin{corollary}
	The treewidth of a graph with genus $g$ is at most $2\sqrt{(2g + 3) n } -1$. 
\end{corollary}

\subsection{Bounded genus and crossings}
A graph is $(g, k)$-planar if there exists an embedding of $G$ on a genus $g$ surface with at most $k$ crossings per edge.If $G$ is $(g, k)$-planar, we shall show that $G$ has treewidth of at most $2\sqrt{(4g + 6)(k + 1) n}$. For fixed $g$ and $k$, this admits a treewidth of $G$ which is $O(\sqrt{n})$. 
\begin{theorem}{\url{https://link.springer.com/chapter/10.1007/978-3-319-27261-0_8}}
	Every $(g, k)$-planar graph $G$ has layered treewidth at most $(4g + 6)(k + 1)$. 
\end{theorem}

\begin{proof}
	Draw $G$ on the plane with an arbitrary orientation of the edges. Then replace every crossing with a new dummy variable to form $G'$. From above $\ltw(G') \leq 2g + 3$. Let $T'$ be the tree decomposition of $G'$, and let $V_0', V_1', ...$ be the layering of $G'$. For each dummy vertex $x$ which crosses the arcs $vw$ and $ab$, where $w$ and $b$ are the tail of the arcs respectively, replace $x$ in $T$ with $w$ and $b$. Then this is tree-decomposition with layered treewidth of $2(k+1) (2g + 3)$. Thus shown. 
\end{proof}

\subsection{$K_t$-minor-free graphs}
Refer to MTH3000 report for definitions.

\begin{theorem}{\url{https://arxiv.org/abs/2104.06627}}
	If $G$ is $K_t$-minor-free, then $tw(G) \leq t^{3/2} n^{1/2}$. 
\end{theorem} 

\begin{lemma}{\url{https://www.ams.org/journals/jams/1990-03-04/S0894-0347-1990-1065053-0/}}
	Let $G$ be a graph and $A_1, ..., A_r$ be nonempty subsets of $V(G)$. Let $x \in \mathbb{R}$, $x \geq 1$. Then either:
	\begin{enumerate}
		\item There exists a tree $T$ in $G$ where $|V(T)| \leq x$ and $V(T) \cap A_i$ is nonempty for all $i$, or,
		\item There exists a set $Y$ in $V(G)$ of size at most $(r-1)n/x$ such that no component of $G-Y$ intersects all of $A_1, ..., A_r$. To rephrase, if $C$ is a component of $G - Y$, then we have there exists a set $A_i$ such that $C$ and $A_i$ are disjoint. 
	\end{enumerate}
\end{lemma}

\begin{proof}
	\newcommand{\dist}{\text{dist}}
	We form a path in $G$ of $A_1 \rightarrow A_2 \rightarrow ... \rightarrow A_r$ such that there exists a path from $A_1$ to $A_r$ passing through all of $A_i$. We then form a new graph $J$ by taking $r-1$ copies of $G$ and between copies $G_i$ and $G_{i+1}$, there is an edge between the copied vertices of $A_{i+1}$ in $G_i$ and $A_{i+1}$ in $G_{i+1}$. Denote the sets $X$ and $Y$ to be $A_1$ in $G_1$ and $A_r$ in $G_{r-1}$ respectively. 
	\par
	Then if $ \dist_J(X, Y) \leq x$, then the path $|P|$ is less than $X$, so we project $P$ back to $G$ and eliminate all loops to form a path, and therefore a tree, of length at most $x$. 
	\par
	If $ \dist_J(X, Y) > x$, then there is a bfs layering in $J$ where $L_i =  \lbrace v \in V(J) : \dist_J(X, v) = i \rbrace$. By construction, there exists a $j$ such that $|L_j| \leq (r-1)n/x$ and in the projection, $L_j$ is a separator. 
\end{proof}

\begin{theorem}[Illingsworth, Scott, Wood]
	For all $t \geq 5$, if $G$ is $K_t$-minor-free, then $G \subseteq H \boxtimes K_{\lfloor m \rfloor}$, where $\tw(H) \leq t-2$ and $m = \sqrt{(t-4)n}$.
\end{theorem}
\begin{proof}
	We shall prove something stronger. For all $t \geq 5$ and $K_t$-minor-free graphs $G$, for all $r \leq t - 1$, and all $K_r$-models $(U_1, ..., U_r)$ in $G$ with $1 \geq |U_i| \geq m$ for all $i$, there exists a $H$-partition of $G$ with width $m$ and $U_1, ..., U_r \in V(H)$, and $tw(H) \leq t-2$, where the width of a partition is the size of the largest bag.
	\paragraph{Base case}
	Let $U = \bigcup_i U_i$. Then suppose if $|V(G)| = U$, then $(U_1, ..., U_r)$ is the desired $H$-partition where $H = K_r$ has treewidth $r-1 \leq t-2$. 
	\paragraph{Inductive case}
	Let $A_i = N_G(U_i) \ U$ for each $i$. If some $A_i$ is empty, then we use the induction hypothesis on $G - U_i$. There is a partition $H'$ of treewidth at most $t-2$ which has $(U_1, ...., U_{i-1}, U_{i+1}, ..., U_{r})$ as a set, with width at most $m$. Then the neighbourhood of $U_i$ is a clique on $r-1$ vertices, so $tw(H) = \max(tw(H'), r-1) \leq t - 2$. 
	\paragraph{Disconnected $G-U$}
	Suppose $G - U$ is disconnected. Then we can partition $V(G - U)$ into two s $C$ and $D$ such that there is no edge from $C$ to $D$. Let $G_1 = G[C \cup U]$ and $G_2 = G[D \cup U]$. Then by induction, there is a partition $H_1$, $H_2$ of $G_1$ and $G_2$ respectively of width at most $t-1$ and treewidth at most $s$ that contains $(U_1, ..., U_r)$ as a bag. Then we let $H$ be the model where we identify the two $H_1$ and $H_2$ models on the bag. Then $\tw(H) = \max(\tw(H_1), \tw(H_2)) \leq s$, with width of $H$ being at most $t-1$.
	\paragraph{Connected $G-U$}
	Use the lemma above and set $A_1, ... A_r$ to be $(A_1, ..., A_r)$ and set $x = m$. Then there exists a tree in $U$ which intersects with $A_i$ for all $i$, or there exists a set $Y$ such that $C$ separates out $C$ and $D$. We can show that the first scenario cannot happen as we have that $Y$ would be too large, so there must be a $Y$ such that this holds. Then $Y \leq (r-1)n/m \leq (tn)/\sqrt(tn) = \sqrt(tn) = m$, so this would be the required set $Y$. 
\end{proof}

\section{Algorithms on bounded treewidth}
\subsection{Monadic Second-order logic}
\begin{theorem}{Courcelle's theorem}
	If a family of graphs $\mathcal{G}$ has bounded treewidth, then any graph property of any graph $G \in \mathcal{G}$ that can be expressed in Monadic Second-order logic can be decided in linear time. 
\end{theorem}

\section{Characterisations of $K_t$-minor free graphs}
What is the structure of $K_t$-minor free graphs? We shall show that we can roughly characterise all $K_t$-minor free graphs as graphs that are products of a series of operations that preserve the treewidth. 
\subsection{$K_t$-minor free minor-closed families}
We define $\had(G)$ to be the largest $t$ such that $G$ has a $K_t$ minor. 
\subsubsection{Planar graphs}
\begin{theorem}
	If $G$ is a planar graph, then $G$ is $K_5$-minor-free.
\end{theorem}
\begin{proof}
	If $G$ is planar with $n$ vertices and $m$ edges, then we have that $m \leq 3n -6$. However, we have that $K_5$ has $5$ vertices and $10$edges, but we have that $ 10 > 3 \times 5 - 6$, so $K_5$ is not planar. As the family of planar graphs is minor-closed, then if $G$ is planar, then $K_5$ is minor-free.
\end{proof}

We can use a different argument to show that $K_{3,3}$ is not embeddable on the plane, by using the fact that $K_{3,3}$ is triangle-free. 
(INSERT PROOF HERE)

\subsubsection{Genus-g graphs}
We define the genus $g$ of a surface to be 2 times the number of handles + the number of crosscaps. From topology, we have that we can add a handle to crosscaps to form 3 crosscaps. Therefore, the Euler characteristic $\chi = 2 - g$ for both orientable and non-orientable surfaces. Note that the genus is defined slightly differently from topology. We do this to allow non-orientable and orientable surfaces to coincide in definition.

We can show that if $G$ has genus $g$, then if $G$ has $n$ vertices and $m$ edges, then $n - m + f = \chi = 2-g$, then as each face has at most 3 vertices and each edge is incident to two faces, we have that $f \leq 2m/3$. Therefore, $m \leq 3(n + g - 2)$, and if $K_t$ is embeddable on a genus $g$ graph, then $\binom{t}{2} \leq 3 (t + g - 2)$. Thus $t \leq \sqrt{6g} + 4$. So if a graph has genus $g$, then it is $K_t$-minor-free, where $t > \sqrt{6g} + 4$. 

\subsubsection{Bounded treewidth graphs}
\begin{theorem}
	If $\tw(G) \leq k$, then $G$ is $K_{k+2}$-minor-free. 
\end{theorem}
\begin{proof}
	We shall prove the contrapositive: If $K_t$ is a minor of $G$, then $tw(G) \geq t-1$.
	If $K_t$ is a minor of $G$, and $\tw(G) \leq k$, then we have that $\tw(K_t) \leq \tw(G) \leq k$, but $\tw(K_t) = t-1 \leq k$, so $t \leq k + 1$. Thus shown a family of minor-closed which are $K_t$-minor free. 
\end{proof}
\subsubsection{Apex vertices}
An apex vertex $v$ is added to a graph $G$ such that it has arbitrary edges. As such, it can simply dominate all other vertices in $G$. Then if $G$ is $K_t$-minor free, $G$ with the apex vertex $v$ is $K_{t+1}$- minor free. 
\subsubsection{Clique-sums}
The \textit{$k$-clique-sum} of two graphs $G$ and $H$, denoted as $G \# H$, is the graph obtained by performing a series of operation on the cliques of $G$ and $H$. We find cliques in $G$ and $H$, $C_G$ and $C_H$ respectively, such that $C_G$ and $C_H$ have size $k$. Then we identify the vertices in $C_G$ and $C_H$ so that $G$ and $H$ are connected to each other on this clique. 

\begin{lemma}
	If $G = G_1 \# G_2$,then $\had(G) = \max(\had(G_1), \had(G_2))$ and $\tw(G) = \max(\tw(G_1), \tw(G_2))$.
\end{lemma}

\begin{example}
	If $G$ is the clique-sum of Euler genus $g$ graphs, then $G$ is $K_{\sqrt{6g} + 5}$-minor-free, but has unbounded genus.
\end{example}

\begin{theorem}[Wagner's theorem]
	If $G$ is $K_5$-minor-free, then $G$ can be obtained from $\leq 3$-clique-sums of planar graphs and the Wagner graph $W_8$.
\end{theorem}


\subsection{Torsos and adhesion}
Given a graph $G$ and a tree-decomposition $\tree$, the \textit{torso} of a bag $B_x$ of $T$ is the graph $G\langle B_x \rangle$, obtained from $G[B_x]$ where $vw$ is a vertex in $G\langle B_x \rangle$ iff $v,w \in B_x \cap B_y$, where $y$ is a neighbour of $x$ in $T$. So the set $B_x \cap B_y$ for all neighbours $y$ of $x$ in $T$ is a clique in $G\langle B_x \rangle$. 
The \textit{adhesion} of a tree is defined as $\max(|B_x \cap B_y|)$ where $xy$ is an edge in $T$.

\subsubsection{Vortices}
Let $G$ be embedded on a surface $\Sigma$, and let $F$ be a face on $G$. Let $D$ be a disc in $\Sigma$ such that $D$ only intersects $G$ only on vertices on the boundary of $F$. We denote these discs as $G$-clean. 

Then let $\Lambda = (x_1, x_2, ..., x_b)$ be a tuple of vertices on the boundary of $F$ such that they intersect $D$. Then we define a new graph $H$ such that $V(G) \cap V(H) = \Lambda$, and there is a path-decomposition of $H$ of bags $B_1, B_2, ... B_b$ such that $x_i \in B_i$ for all $i$. $H$ is denoted as a \textit{$D$-vortex} of $G$. The width of a $D$-vortex is the width of the path above, or $\max_i(|B_i| - 1)$. 

Vortices were created to solve the problem of grid-like graphs with large treewidth, torsos and adhesion, yet are all $K_t$-free for bounded $t$. 
\subsection{Robertson-Seymour theorem}
Given $g, p, a \geq 0$, $k \geq 1$, a graph $G$ is $(g, p, k, a)$- almost embeddable if there exists an $A \subseteq V(G)$ with $|A| \leq a$, and there exists subgraphs $G_0, G_1, ...,  G_{p'}$ of $G$ such that:
\begin{itemize}
	\item $G - A = G_0 \cup G_1 \cup G_2 ... G_{p'}$
	\item $p' \leq p$
	\item There is an embedding of $G_0$ onto a surface $\Sigma$ of genus $\leq g$
	\item There exists pairwise disjoint $G_0$-clean discs $D_1, D_2, ..., D_{p'}$ in $\Sigma$
	\item $G_i$ is a $D_i$-vortex of width at most $k$.
\end{itemize}

\begin{theorem}[Robertson-Seymour graph structure theorem]
	For all $t$, there exists $g, p, a \geq 0$, $k \ell \geq 1$, such that every $K_t$-minor-free graph has a tree-decomposition of adhesion $\leq \ell$ and each torso is $(g, p, k, a)$-embeddable. 
\end{theorem}
In fact, there exists a function $t(g, p, k, a)$ such that if a graph has a tree-decomposition of adhesion $\leq \ell$ and each torso is $(g, p, k, a)$-almost embeddable, then $G$ has no $K_t$ minor. One possible function is $t(g, p, k, a) = a + ck \sqrt{g + p}$. 

\section{Path-width}
We define the path-decomposition of a graph $G$ to be a sequence of bags $B_i$ such that the subsequence of bags containing a vertex $v$ induces a subpath and each edge $vw$ is in a bag $B_i$. Then we define the width of a path-decomposition as $\max_i \lbrace |B_i| \rbrace -1$, same as treewidth. The pathwidth of a graph $G$ is the minimum treewidth.
\begin{example}
	\begin{theorem}[Caterpillars]
		Graphs have pathwidth at most 11 iff every connected component is a caterpillar (graphs where removing every leaf yields a path).
	\end{theorem}
	\begin{proof}[Caterpillars]
		Suppose $G$ is a caterpillar graph and $p_1, p_2, ..., p_n$ is the central path, and the leaves of vertex $p_i$ are denoted as $v_{i, 1}, v_{i, 2} ..., v_{i, k}$. Then have the bags $(v_{1, 1}, v_1), (v_{1, 2}, v_1)... (v_{1, j}, v_1), (v_1, v_2), (v_{2, 1}, v_2), (v_{2,2}, v_2,)... $. We can see that each leaf appears once and each vertex on the central path is on a subpath of the path. Therefore, the pathwidth of $G$ is 1. If $G$ has pathwidth 1, then for each connected component, we choose a vertex $v$ in $B_1$ and a vertex $w$ in $B_n$, the final bag, and look at a path from $v$ to $w$. This path must go through every bag, thus the non-path vertices must have neighbour only of the other one in the bag and thus the graph is a caterpillar. 
	\end{proof}
	
\end{example}
\begin{example}
	If a graph $F$ is a forest, then the pathwidth of $F$ is the largest pathwidth over all connected components.
\end{example}
\begin{example}
	The pathwidth of a single vertex is 0.
\end{example}
\begin{example}
	The pathwidth of a tree $T$ is $\min_{P \subset T} 1 + \pw(T - V(P))$ where $P$ is a path. 
\end{example}

\begin{proof}[Proof of inductive path-width]
	To show $\pw(T) \leq 1 + \pw(T - V(P))$, we have that if $P$ is a path in $T$ with vertices $v_1, v_2, ...$, then consider the subtrees hanging off $v_i$ for all $i$. $T - V(P)$ will have a path-width and we can order each connected component such that they appear in the order of the trees. Then we have that adding $v_i$ to the bags of subtrees connected to $v_i$, and the bag $(v_i, v_{i+1})$ between the subtrees $v_i$ and $v_{i + 1}$ will yield a path-decomposition of width $1 + \pw(T - V(P))$. 
	\paragraph{Other direction}
	To show there exists a path $P$ such that $\pw(T) \geq 1 + \pw(T - V(P))$, we proceed by induction. Let $B_1, ... B_n$ be a path-decomposition of $T$. Let $x$ live in bag $B_1$ and $y$ live in bag $B_n$, the final bag. Then let $P$ be the unique path from $x$ to $y$. Then $P$ traverses through every bag in the path-decomposition. Then $\tw(T) \geq 1 + \tw(T - P)$ by induction. 
\end{proof}

We define a ternary tree to be a tree where every vertex has degree 1 or 4, except for the root $r$, which has degree 3. We define the complete ternary tree of edge-height $h$ to be the unique complete ternary tree where the distance from the root $r$ to any other vertex is at most $h$. Note that the treewidth of any tree is 1.
\begin{lemma}
	Let $T_h$ be the complete ternary tree of edge-height $h$. Then $\pw(T_h) = h$.
\end{lemma}

\begin{proof}
	We shall show for any path $P$, $T_h - V(P)$ has a copy of $T_{h-1}$. Let $P$ be a path, and suppose that it goes the root $r$. Then as $P$ cannot go through all three subtrees of $T_h$ hanging off $r$, there must be a subtree which $P$ does not go through. $T - V(P)$ will contain this subtree, thus $\pw(T_h) \geq 1 + \pw(T - V(P)) \geq 1 + (h-1) = h$.
	\paragraph{Lower bound}
	There exists a path-decomposition of $T_h$ such that the size of each bag is at most $h + 1$. Order the leaves of the balanced ternary trees in the standard way, left to right when drawn with no crossings. Then let $B_i$ be vertices on the path from $r$ to the leaf $\ell_i$. Then this is a tree-decomposition with at most $h + 1$ vertices, thus $\tw(T_h) \leq h$.
\end{proof}

For complete binary trees of edge-height $h$, it is easy to show there is a path-decomposition of width $\lceil h/2 \rceil$. (Consider doing the same operation, but with half the vertices.)

\subsection{Treedepth}
We define the \textit{closure} of a rooted tree $T$ to be the the graph $G$ where $V(G) = V(T)$ and $vw$ is an edge in $G$ iff $v$ is an ancestor of $w$ in $T$.
\begin{definition}
	The treedepth of a graph $G$, denoted as $\td(G)$, is defined to be the minimum vertex-height of a rooted forest $T$ such that $G \subseteq $ closure of $T$.
\end{definition}
We have that this defines a path-decomposition of $G$ by enumerating through all the vertices of $T$ in the natural order with the bag $B_i$ is the path from leaf $\ell_i$ to $r$. The size of the bag is the vertex-height, thus $\pw(G) \leq \td(G) - 1$. As every path-decomposition is a valid tree-decomposition, $\tw(G) \leq \pw(G) \leq \td(G) - 1$. However, these parameters are not \textit{tied}. We say two parameters $p(G), q(G)$ of a graph are tied if there is a function $f$ such that $ p(G) \leq f(q(G))$ and $q(G) \leq f(p(G))$ for all graphs $G$. 
Let $\mathcal{G}$ be the graph family $T_1, T_2, ...$ where $T_h$ is the complete ternary tree of height $h$. Then $\tw(G) = 1$ for all $G \in \mathcal{G}$, but $\pw(T_h) = h$, thus the pathwidth may be unbounded while the treewidth is constant. Let $\mathcal{G}$ be the set of paths $P_i$ for $i \in \mathbb{N}$. Then $\pw(P_i) =  1$, but $\td(P_i) = \lceil i/2 \rceil$. Therefore, the treedepth of graphs is not tied to the pathwidth, or to the treewidth.
\subsubsection{Bounds on treedepth}
\begin{lemma}
	If $T$ is a tree, then $\td(T) \leq \log n$, where we take $\log = \log_2$. 
\end{lemma}
\begin{proof}[Proof of above lemma]
	For all trees $T$, there exists a balanced separator with a single vertex $v$, from above. Therefore, $T-v$ has components of size at most $n/2$. Then we find the separators $w_1,... w_m$ of the other components of $T - v$ and we add an edge from $v$ to $w$, with tree-depth at most $\log(n/2) = \log(n) - 1$. Then $T$ has treedepth at most $\log(n) - 1 + 1 = \log n$. Thus shown. 
\end{proof}

\begin{corollary}
	The tree-depth of a graph $G$ with treewidth $h$ is $O(h \log n)$.
\end{corollary}
The proof of this corollary is by calculating the tree-depth of the tree-decomposition of $G$, which is $O(\log n)$ (we may fudge numbers and bound the number of vertices in the tree-decomposition by the number of edges in $G$ to get the desired result).

\section{Classification of graph families}
As before, we could classify graph families in this way:
\begin{enumerate}
	\item $\tw(G) \leq O(1)$ for all $G \in \mathcal{G}$. These graph families will include trees and paths, so the pathwidth and treedepth is unbounded, but from the lemma above, $\tw(G) \leq O(\log n)$ for all graphs in the class, where we hide the treewidth.
	\item $\tw(G) \leq O(n^{1/2})$ for all $G \in \mathcal{G}$.
	\item $\tw(G) \leq O(n^{1- \varepsilon})$ fixed $\varepsilon > 0$, for all $G \in \mathcal{G}$.
	\item $\tw(G \geq \Omega(n))$ for all $G \in \mathcal{G}$.
\end{enumerate}
We shall prove a very strong statement relating the path-width and tree-depth of a graph $G$ for families of graphs of the form $\tw(G) \leq O(n^{1- \varepsilon})$. 
\begin{theorem}
	If $\mathcal{G}$ is a graph family such that $\tw(G) \leq O(n^{1- \varepsilon})$ for all $G \in \mathcal{G}$, then $\pw(G) \leq O(n^{1-\varepsilon})$ and $\td(G) \leq O(n^{1-\varepsilon})$ for all $G \in \mathcal{G}$.
\end{theorem}
We shall prove this using the following lemma. We define a \textit{hereditary class} $\mathcal{G}$ to be a class closed under vertex deletion. This is a weaker condition than minor-closure. 
\begin{lemma}
	Fix $c > 0$ and $\varepsilon$ in $(0, 1)$. Then suppose for all $n$-vertex graphs $G \in \mathcal{G}$, $G$ has a balanced separator of order at most $c n^{1-\varepsilon}$. Then every $n$-vertex graph $G \in \mathcal{G}$ has $\pw(G) \leq \td(G) \leq c' n^{1- \varepsilon}$. 
\end{lemma}
\begin{proof}
	We use the balanced separator concept from earlier, and use induction on the number of verticesWe have that the balanced separator $S$ where $|S| \leq cn^{1-\varepsilon}$ and consider the connected components of $G - V(S)$. As $G$ is minor closed, then each connected component of $G - V(S)$ will have treedepth at most $c'(n/2)^{1 - \varepsilon}$, where $c' := \frac{c}{1 - 1/(2^{1-\varepsilon})}$. Then suppose we form a complete graph on $S$ and add edges from $S$ to all the vertices in $G - V(S)$ to form the closure of a tree of height at most $c n^{1 - \varepsilon} + c'(n/2)^{1 - \varepsilon}$. However, we have that:
	\begin{align*}
		c n^{1 - \varepsilon} + c'(n/2)^{1 - \varepsilon} &= c n^{1 - \varepsilon} + \frac{c}{1 - 1/(2^{1-\varepsilon})}(n/2)^{1 - \varepsilon}\\
		&= \frac{1}{1 - 1/2^{1-\varepsilon}} (c n^{1 - \varepsilon}(1 - 1/(2^{1-\varepsilon})) + c (n/2)^{1 - \varepsilon})\\
		&= c'n^{1 - \varepsilon}
	\end{align*}
\end{proof}
thus the treewidth is bounded by $c' n^{1-\varepsilon}$. 

\subsection{H-minor-free families}
Let $\mathcal{G}_H:=$ the class of all $H$-minor free graphs.
We have that $(\tw(\mathcal{G}_H), \pw(\mathcal{G})_H, \td(\mathcal{G})_H)$ are $O(n^{1/2})$ constants, where $n$ is the number of vertices. 
\begin{theorem}[Robertson + Seymour + Illingsworth]
	The statement: "There exists a constant $c$ such that for all $G \in \mathcal{G}_H$, $\pw(G) \leq c$" is equivalent to the statement: "$H$ is a forest".
\end{theorem}

\begin{proof}
	Suppose $H$ is not a forest. Then $H$ has a cycle. However, the set of complete ternary trees has no $H$-minor and has unbounded path-width. The other direction is proven in \url{https://www.sciencedirect.com/science/article/pii/009589569190068U}. 
\end{proof}

Therefore, a minor-closed class $\mathcal{G}$ has bounded path-width iff $\mathcal{G}$ excludes some forest. The equivalent statement for treewidth is that: 
"a minor-closed class $\mathcal{G}$ has bounded treewidth iff $\mathcal{G}$ excludes some planar graph". Proof by Robertson + Seymour, graph minors 1.
\begin{lemma}
	For all $n$-vertex planar graphs $G$, $G$ is a minor of the grid on $2n \times 2n$ vertices.
\end{lemma}
We also introduce the grid minor theorem as well.
\begin{theorem}[Grid minor theorem]
	There exists a function $f$ such that every $P_k \square P_k$-minor free graph has $\tw \leq f(k)$. 
\end{theorem}
\begin{proof}[Robertson + Seymour, GM 1]
	Suppose $\mathcal{G}$ excludes a non-planar $H$. Then the family of $n \times n$ grids is $H$-minor free with unbounded treewidth.
	Suppose $H$ is planar of $k$ verses. Then $H$ is a minor of the $P_{2k} \square P_{2k}$ grid. Then $G$ is $P_{2k} \square P_{2k}$-minor-free. Therefore, $\tw(G) \leq f(2k)$ by the grid minor theorem. 
\end{proof}

The grid minor theorem states that treewidth measures how close a graph is to being a tree and how far away a graph is from being a large grid. 
\subsection{Alternative characterisations of treewidth and path-width}
We can show the following:
For all $G$, the $\pw(G) = \min k$ such that $G$ is a spanning subgraph of an interval graph with no $K_{k + 2}$, so the intervals cross at most $K + 1$ times. 

We can show a similar result for treewidth as well:
For all $G$, $\tw(G) = \min k$ such that $G$ is a spanning subgraph of a \textit{chordal graph} with no $K_{k + 2}$ subgraph. A chordal graph is a graph with no induced cycle with more than 4 vertices. This is equivalent to saying that $G$ has $\tw \leq k$ iff $G$ is a spanning subgraph of the intersection graph of a tree $T$ with max clique size $k + 1$. 
$G$ is chordal if all minimal separators are a clique. 


\section{Proof of Path-Width theorem}
\begin{theorem}
	\label{thm:Path-Width theorem }
	For every forest $F$, if a graph $G$ is $F$-minor-free, then $\pw(G) \leq |F| - 2$.
\end{theorem}
The contrapositive is:
\begin{theorem}
	\label{thm:pw-tree contrapositive }
	If a graph $G$ has path-width at least $w$, and $F$ is a forest where $|F| \leq w + 2$, then $G$ contains $F$ as a minor. 
\end{theorem}
Note that a complete graph on $|F|-1$ vertices has pathwidth $|F| - 2$. 
We say that a separation is a pair $(A, B)$ of $V(G)$ where $A \cup B = V(G)$, there are no edges between $A - B$ and $B - A$, and the order of the separation is $|A \cap B |$. We can consider $A$ as being on the left of some separator, and $B$ on the right of some separator. We say that $(A, B) \leq (A', B')$ if:
\begin{itemize}
	\item $A \subseteq A'$
	\item $B' \subseteq B$
\end{itemize}
So $(A', B')$ is to the "right" of $(A, B)$. For each $w \geq 0$, we say that $(A, B)$ is $w$-good, if we can decompose $A$ into a path of width at most $w$ and the last bag is $A \cap B$, the separator. 

\subsection{Proof}
\begin{lemma}
	If $(A, B)$ and $(P, Q)$ are separations of $G$, $(A, B)$ is $w$-good, $(P, Q) \leq (A, B)$, and there are $|P \cap Q|$ vertex-disjoint paths of $G$ between $P$ and $B'$, then $(P, Q)$ is $w$-good. 
\end{lemma}
\begin{proof}
	Let $R_1, ..., R_t$ be disjoint paths between $P$ and $B'$, where $t = | P \cap Q|$. Then we have that each path must pass through $P \cap Q$ as this is a separator, thus there are at least $t$ elements in the separator. Additionally, each path must belong inside $A$ as $A \cap B$ is also a separator. Therefore, $t \leq |A \cap B|$. 
	Let $H = G[P] \cup \lbrace R_1, ..., R_t \rbrace$. As $H$ is in $A$, then there exists a path-decomposition of width at most $w$ where the last bag is precisely $A \cap B$, or the endpoints of $R_1, ..., R_t$. But contracting $R_1, ..., R_t$ to a single point yields a path-decomposition where the last bag is $P \cap Q$, as contracting $A \cap B$ along $R_1, ..., R_t$ yields $P \cap Q$. 
\end{proof}

We say that if $(A, B)$ and $(A', B')$ are separations of $G$, the second \textit{extends} the first if $(A, B) \leq (A', B')$ and the order of $(A', B')$ is at most the order of $(A, B)$. We say that a $w$-good separation of $G$ is maximal if no other $w$-good separation extends it. 

Let $T$ be a tree, $(A, B)$ a separation. We say $(A, B)$ is $(w, T)$ spanning if:
\begin{itemize}
	\item $|A \cap B| = |T|$
	\item There exists a model $\varphi$ of $T$ in $G[A]$ such that each block in the model contains a vertex in $A \cap B$
	\item If $|T| \leq w + 1$ then $(A, B)$ is maximal $w$-good.
\end{itemize}

\begin{lemma}
	Let $w \geq 0$ be an integer. Let $G$ be a graph with path-width at least $w$, and let $T$ be a tree, with $|T| \leq w + 2$. Then there exists a $(w, T)$ spanning separation of $G$. 
\end{lemma}

\begin{proof}
	Fix $w$. Then if $|T| = 0$, let $A = \emptyset$, $B = V(G)$. Then $(A, B)$ is a $w$-good separation which is $(w, T)$-spanning.
	
	If $|T| = 1$, then let $v = V(T)$. Then we choose an arbitrary $v \in B$ such that $A = \lbrace v \rbrace$, $B = V(G)$, and extend by maximality. 
	If $2 \leq |T| \leq w + 1$, then let $v$ be a leaf of $T$. we have that there is a  $(w, T - v)$-spanning separation $(A, B)$ of $G$ which is maximal $w$-good. Then let $u$ be the neighbour of $v$ in $T$ and let $u'$ be the intersection of the block of $u$ in $G$ and $B$. Then $u'$ has a vertex $v'$ in $B - A$, otherwise $(A, B - \lbrace u' \rbrace)$ is $w$-good and extends $(A, B)$, contradicting the maximality of $(A, B)$. 
	
	We assume that $(A \cup \lbrace v' \rbrace, B)$ is $w$-good. Therefore, there exists a maximal $w$-good separation $(A', B')$ that extends $(A \cup \lbrace v' \rbrace, B)$. Then $(A', B')$ has order exactly $|T|$ as it did not extend $(A, B)$. Now we wish to show that $(A', B')$ is $(w, T)$-spanning.
	\paragraph{Contradiction argument}
	Suppose not. Then there exists less than $|T|$ paths between $A \cup \lbrace v' \rbrace$ and $B'$. By Menger's theorem, then that implies that there exists a separator between $A \cup \lbrace v' \rbrace$ and $B'$ of order $< T$. Call this separator $(P, Q)$ with minimum order.
	
	We have that $(A \cup \lbrace v' \rbrace, B) \leq (P, Q) \leq (A', B')$. By Menger's theorem, there exists $|P \cap Q|$ vertex-disjoint paths from $P$ to $B'$. Therefore, $(P, Q)$ is $w$-good, by the above lemma. But $(P, Q)$ must extend $(A, B)$, since $|P \cap Q| \leq |A \cap B|$, and $(P, Q) \neq (A, B)$. However, this contradicts the maximality of $(A, B)$.
	Therefore, there are $|T|$ disjoint paths between $A \cup \lbrace v \rbrace$ and $B'$. Therefore, $(A', B')$ is $(w, T)$-spanning.
	
	\paragraph{Conclusion}
	Therefore, there is a $(w, T)$-spanning path separator $(A, B)$ where $|T| = w + 2$. Therefore, we have that $G[A]$ contains a model of $T$, therefore $G$ contains a model of $T$. Thus shown. 
\end{proof}
\end{document}
