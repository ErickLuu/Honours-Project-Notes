\documentclass[]{article}
\usepackage[margin = 1in]{geometry}

\usepackage{amsmath}
\usepackage{amssymb}
\usepackage{amsthm}
\usepackage{url}

% Commands
\newcommand{\tree}{\mathcal{T}}
\newcommand{\tw}{\text{tw}}
\newcommand{\had}{\text{had}}
\newcommand{\pw}{\text{pw}}
\newcommand{\td}{\text{td}}
\newcommand{\bn}{\text{bn}}
% Environments

\newtheorem{theorem}{Theorem}
\newtheorem{proposition}[theorem]{Proposition}
\newtheorem{corollary}[theorem]{Corollary}
\newtheorem{lemma}[theorem]{Lemma}
\newtheorem{definition}[theorem]{Definition}
\newtheorem{conjecture}[theorem]{Conjecture}

\theoremstyle{definition}
\newtheorem{example}[theorem]{Example}

\numberwithin{theorem}{section}
\numberwithin{equation}{section}

%opening
\title{Structural Graph theory}
\author{Eric Luu}

\begin{document}

\maketitle

\section{Treewidth}

\begin{definition}[Tree-decomposition]
	The tree-decomposition $\tree$ of a graph $G$ is defined as a tree $T$ with associated \textit{bags} $\lbrace B_x : x \in V(T) \rbrace$ such that:
	\begin{itemize}
		\item for all $v \in V(G)$, the subset of vertices $\lbrace x \in V(T): v \in B_x \rbrace$ in $V(T)$ induces a connected subtree in $V(T)$.
		\item For all edges $vw \in E(G)$, there exists a bag $B_x$ such that both $v$ and $w$ are in the bag $B_x$.
	\end{itemize}
\end{definition}
We refer to the vertices of the tree $T$ as \textit{nodes}. 
The \textit{width} of the tree decomposition $\tree$ is defined as $\max \lbrace |B_x| - 1 : x \in V(T) \rbrace$. We define the \textit{treewidth} of a graph $G$ as such:


\begin{definition}
	The treewidth of a graph $G$, denoted as $\tw(G)$, is defined to be the smallest width for all tree decompositions of the graph $G$.
\end{definition}
The reason why the $-1$ appears in the definition of the with of a tree decomposition is because the definition wanted the treewidth of a forest to be 1. However, this causes some notational confusion.
\begin{example}
	$\tw(G) = 1$ iff $G$ is a forest.
	\begin{lemma}
		If $G$ is a forest, then $\tw(G) = 1$.
	\end{lemma}
	\begin{proof}
		Suppose $G$ is a tree. Root the graph $G$ at the vertex $r$. Then let $T = G$ and $B_x:= \lbrace x, p \rbrace$ where $p$ is the parent of $x$. The bag $B_r$ will just contain $r$. Then all edges $vw$ will be between parent $v$ and child $w$, so it will be in bag $B_w$. Finally, the subgraph induced by vertex $x$ in $T$ will be $x$ and the children of $x$, which is a connected subtree.
		
		If $G$ is a forest, then we perform this operation on every connected component of $G$ and connect the roots to form a new tree. Then this tree is a tree-decomposition. This forms a tree-decomposition of width at most 1. 
	\end{proof}
	\begin{lemma}
		If $\tw(G) = 1$, then $G$ has no cycles.
	\end{lemma}
	\begin{proof}
		If $G$ has a cycle $C$, then the treewidth cannot be 1. This is because if there is a tree decomposition $\tree$ where the size of each bag is at most 2, then as the graph must have every edge, then every edge in $C$ is in separate bags. However, we have that for any vertex $v$ in $C$ to have an induced connected subgraph in $T$, then it follows that the cycle $C$ is also in $T$. Thus $T$ is not a tree, and this is not a valid tree-decomposition. 
	\end{proof}
\end{example}
\begin{lemma}[Helly Property]
	Let $T_1, ..., T_k$ be subtrees of a tree $T$ such that for every pair of trees, there is a vertex in common. Then there exists a vertex which is common to all trees.
\end{lemma}
\begin{proof}[Helly property]
	If $T_1$, $T_2$ and $T_3$ are subtrees of $T$ such that the vertex sets are pairwise nonempty, then there is a common vertex in all three subtrees. If this is not the case, denote $v_1$ as a vertex in the intersection of $T_1$ and $T_2$, $v_2$ as the vertex in $T_1 \cap T_3$, and $v_3$ as the vertex in $T_2$ and $T_3$. Then there exists a unique path $P$ in $T_1$ from $v_1$ to $v_2$. Choose two vertices $x$ and $y$ on $P$ such that they are disjoint....
\end{proof}

\begin{theorem}[Clique theorem]
	In any tree-decomposition of $G$, for every clique $C$ in $G$, there exists a node $x \in V(T)$ such that $C \subseteq B_x$. 
\end{theorem}

\begin{proof}
	Let $\tree$ be a tree-decomposition. Every vertex $v$ induces a connected subtree in $T$, call it $T_v$. Then for any two vertices $x, y$ in $C$, we have that $T_x$ and $T_y$ must intersect as the edge $xy$ is inside a bag $B_z$ corresponding to a node $z$. Then by the Helly property, there exists a node $v$ such that $C \subseteq B_v$.
\end{proof}

\begin{corollary}
	$\tw(K_n)$ is $n-1$. 
\end{corollary}

\begin{theorem}
	If $H$ is a minor of $G$, then $\tw(H) \leq \tw(G)$. 
\end{theorem}
\begin{proof}[Proof of minor]
	Suppose we have a tree-decomposition $\tree$ of $G$. If we delete an edge in $G$, then $\tree$ remains a valid tree-decomposition. If we delete a vertex $v$, then $\tree$ where we remove $v$ from every bag in $\tree$ is also a valid tree-decomposition. If we contract an edge $vw$, creating a new vertex $u$, then relabeling $v$ and $w$ in all bags to $u$ is a valid tree-decomposition as the induced subtree of $u$ is the union of the induced subtrees of $v$ and $w$, and every neighbor of $v$ or $w$ is a neighbor of $u$. But the edges in the neighborhood do not change. Thus this is a valid tree-decomposition, with width at most the width of $\tree$.
\end{proof}

\begin{example}
	The treewidth of an outerplanar graph is at most 2.
\end{example}
\begin{proof}[Proof of outerplanar treewidth.]
	Let $G$ be the outerplanar graph, and let $G'$ be the triangulation of $G$. As $G$ is a minor of $G'$, $\tw(G) \leq \tw(G')$. We look at the \textit{weak dual} of $G'$. This is a tree $T$, where every node $v_f$ in $T$ corresponds to a face $f$ in $G'$. Then let $B_{v_f}$ be the bag of the tree-decomposition, where $B_{v_f}$ is the set of vertices on the boundary of the face $f$. Then the tree $T$ with bags $B_{v_f}$ is a valid tree-decomposition of $G'$, where every bag has at most 3 vertices. Thus, $\tw(G) \leq 2$. 
\end{proof}

\subsection{Different characterisations of treewidth}
\subsubsection{$k$-trees}
We define a $k$-tree inductively. We have that the complete graph $K_k$ is a $k$-tree, and if $G$ is a $k$-tree, then we add a new vertex to $G$ that is adjacent to $k$ vertices that form a clique of size $k$ in $G$ results in a $k$-tree. 
A $k$-tree is a maximal graph with treewidth $k$. $\tw(G) \leq k$ iff $G$ is a subgraph of a $k$-tree. 

\section{Separators}
A subset $X$ of $V(G)$ is a \textit{balanced separator} of $G$ if each component of $G - X$ has at most $|V(G)|/2$ vertices. This implies that we can partition the vertices of $G$ into sets $A$ and $B$ such that there are no $AB$-edges and the size of $A$ and $B$ is at most $2/3 |V(G)|$. This is because we can order the components from smallest to largest and partition them into sets $A$ and $B$ where the sizes are at most $2/3 |V(G)|$.

\begin{theorem}
	For all graphs $G$, there exists a balanced separator of size $\tw(G) + 1$. 
\end{theorem}
\begin{proof}[Proof of balanced separator]
	We take a tree-decomposition $\tree$ of treewidth $\tw(G) - 1$. For any edge $xy$ in $T$, denote the largest subtree containing $x$ that does not contain $y$ as $T_{x,y}$, and similarly denote $T_{y, x}$ as the same thing. If the size of the union of the corresponding bags of the nodes of $T_{x,y}$ is larger than the size of the union of bags in $T_{y, x}$, orient the edge $xy$ to point from $y$ to $x$, otherwise orient it the other way. Do this for every edge. Then let $x$ be the node where all arrows are pointing inwards, and let $B_x$ be the corresponding bag. Then $B_x$ is a separator of $G$ as we have that at most $|V(G)|/2$ vertices are in any component of $T$ by definition. Thus $B_x$ is a balanced separator of $G$. 
\end{proof}

\subsection{Subset theorems}
\begin{theorem}
	For all graphs $G$, and all subsets $S$ of $V(G)$, there exists an $X$ where $|X| \leq \tw(G) + 1$ and each component of $G - X$ has $\leq |S|/2$ vertices in $S$. 
\end{theorem}
\begin{proof}
	Do the steps above but instead of weighing each vertex the same, you weigh a vertex $v$ to be 1 if it is in $S$ and 0 if it is not. 
\end{proof}

\begin{theorem}
	For all graphs $G$, and all subsets $S$ of $V(G)$, there exists  two subgraphs $G_1$ and $G_2$ such that $G = G_1 \cup G_2$ and for all $i \in \lbrace 1, 2 \rbrace$, $|S \cup V(G_i) | \leq 2/3 |S|$.
\end{theorem}
\begin{proof}
	Use the theorem above. Then we can form $G_1$ and $G_2$ to have at most $2/3 |S|$ the number of vertices in $|S|$, by sorting the subsets by the number of vertices in $S$. 
\end{proof}

\subsection{Bounds on treewidth.}
\begin{theorem}
	Let $G$ be a graph such that for all subsets $S \subseteq V(G)$ there exists another subset $x \subseteq V(G)$ such that $|X| \leq k$ and each component of $G - X$ has at most $|S|/2$ vertices in $S$. Then $\tw(G) \leq 3k$. 
\end{theorem}

\begin{lemma}
	Let $G$ be a graph such that for all subsets $S \subseteq V(G)$ of size $2k + 1$ there exists an $X \subseteq V(G)$ such that $|X| \leq k$ and each component of $G - X$ has at most $k$ vertices in $S$. Then For all $S \subseteq V(G)$ where $|S| \leq 2k+1$ there exists a tree-decomposition of $G$ with width at most $3k$ and there exists a bag containing $S$. 
\end{lemma}
\begin{proof}
	Suppose $|V(G)| \leq 3k + 1$. Then place all of the vertices in a single bag. Then this is a valid tree-decomposition with width at most $3k$ containing all $S$. 
	Now assume $|V(G)| \geq 3k + 2$ and $|S| = 2k + 1$. If $S \leq 2k + 1$, add arbitrary vertices to $S$. Then there exists a subset $X \subseteq V(G)$ such that $|X| \leq k$ and each component of $G - X$ has at most $k$ vertices in $S$. Let the components of $G - X$ be $G_1, G_2, ... G_p$. Then we can do induction on $(G_i, S_i \cup X)$ to have a tree-decomposition of $G_i$ with width at most $3k + 1$. Then for each of the tree-decompositions rooted at the node with bag containing $S_i \cup X$, we add on a parent vertex to all of those tree-decompositions of $X \cup S$ with width at most $3k + 1$.  This is a tree-decomposition of $G$ of width at most $3k + 1$ with the root vertex containing $S$ by definition. 
\end{proof}

\section{Tree-partitions}
\newcommand{\tpw}{\text{tpw}}

For a graph $G$, a \textit{tree-partition} of $G$ is a tree $T$ with associated partition of the vertices of $G$ into bags $ \lbrace B_x : x \in V(T) \rbrace$ such that if $vw$ is an edge in $G$, then $v$ and $w$ are in the same bag, or the edge $xy$ is in $E(T)$, where vertices $x$ and $y$ have corresponding bags $B_x$ and $B_y$ containing $v$ and $w$ respectively. The \textit{width} of the tree-partition is defined as the largest bag in the tree-partition. The tree-partition width of a graph $G$, denoted as $\tpw(G)$, is the smallest width of all tree-partitions. 


\begin{theorem}[Distel + Wood]
	For all graphs $G$, $\tpw(G) \leq 18 (\tw(G) + 1) \Delta(G)$. 
\end{theorem}

\begin{lemma}
	Fix $k$ and $d$. Let $G$ be a graph where $\tw(G) \leq k - 1$ and $\Delta(G) \leq d$. Then for any set $S \subseteq V(G)$ and $4k \leq |S| \leq 12kd$, there exists a tree-partition $\left( B_x : x \in V(T) \right)$ with width at most $18kd$ and bag $B_z$ where $deg_{T}(z) \leq \frac{|S|}{2k} - 1$ and $|B_z| \leq 3/2 |S| - 2k$.
\end{lemma}
\begin{proof}
	
	Case 0: $|V(G)| < 4k$: We place all the vertices in the same bag. Size of bag is $< 4k$, so the bag is definitely less than $18kd$. 
	\par
	Case 1: $|V(G) - S| \leq 18kd$. Let $T$ be the tree on two vertices ${x, z}$, where $B_x = V(G) - S$ and $B_z = S$. Then we have that $\delta(T) = 1$ and $deg_T(z) = 1$, which satisfies the requirements above. We have that $|B_z| \leq 3/2 |s| - 2k$ and $deg_T(z) \leq |S|/2k - 1$. 
	\par
	Case 2: $S$ small case. $4k \leq |S| \leq 12k$. Let $S' := \bigcup \lbrace N_G(v) - S : v \in S \rbrace$. Then $|S'| \leq d |S| \leq 12kd$. If $|S'| < 4k$, then add arbitrary vertices to $S'$ from $G - S - S'$ such that $|S'| = 4k$. Now $4k \leq |S'| \leq 12kd$. By the induction hypothesis, there exists a tree-partition of $G - S$ with width $\leq 18kd$ and $S'$ in one bag. Then we add the bag $B_z = S$ to the tree that is connected only to $S'$. We have that as $4k \leq |S|$, it implies that $|S| \geq 3/2 |S| - 2k$, so $|B_z| \leq 3/2 |S| - 2k$ and $|S| \leq 12k$. Finally, $deg_T(B_z) = 1 \leq |S|/2k - 1$.
	\par
	Case 3: $S$ large case. $12k + 1 \leq |S| \leq 12kd$. There exists induced subgraphs $G_1$, $G_2$ of $G$ where $G_1 \cup G_2 = G$ and $|G_1 \cap G_2| \leq k$, where $|S \cap V (G_i)| \leq 2/3 |S|$ for each $i$ in $\lbrace 1, 2 \rbrace$. Then let $S_i = (S \cap V(G_i)) \cup (G_1 \cap G_2)$ for each $i$ in $\lbrace 1, 2 \rbrace$. We have that $|S_2| \geq |S - V(G_1)| \geq 1/3 |S| \geq 4k$. By symmetry, $|S_1| geq 4l$. For an upper bound, $S_i \leq 2/3 |S| + k \leq 8kd + k \leq 12kd$. Therefore, $4k \leq |S_i| \leq 12kd$ for each $i$ in $\lbrace 1, 2 \rbrace$. Thus by induction, there exists a tree-partition of $G_i$ with width at most $18kd$, such that $\delta(T_i) \leq 6d$ and there is a $z_i$ such that $S_i \in B_{z_i}$, $|B_{z_i}| \leq 3/2 |S_i| - 2k$, $\deg_{T_i}(z_i) \leq |S_i|/2k - 1$. Then form the tree of $G$ by merging $z_1$ and $z_2$ together to form $z$, and let $B_z = B_{z_1} \cup B_{z_2}$. Then this is a tree-partition of $G$. By construction, $S \subseteq B_z$ and $|B_z| \leq |B_{z_1}| + |B_{z_2}| - |G_1 \cap G_2|$. Using the induction hypothesis, this is less than $18kd$, and the degree of $z$ is $|S|/2k - 1 < 6d$. Thus shown.  
\end{proof}

\section{$O(\sqrt{n})$-bounded treewidth}
\newcommand{\ltw}{\textit{ltw}}
A family of graphs $\mathcal{G}$ has $O(\sqrt{n})$ bounded treewidth if, as the number of vertices increases in $\mathcal{G}$, the treewidth is bounded above by a constant times $\sqrt{n}$. We shall show that all planar graphs have bounded treewidth, and can extend this definition to graph families of bounded genus and crossings. [Dujmovic, Morin, Wood] 

\subsection{Layered treewidth}
A \textit{layering} of a graph $G$ is a partition of the vertex set of $G$ into sorted sets $V_1, V_2, ..., V_k$ such that for all edges $vw \in E(G)$, if $v \in V_i$ and $w \in V_j$ then $|i - j| \leq 1$. The \textit{layered treewidth number} $\ltw(G)$ is defined as the smallest $k$ such that there exists a layering $V_1, V_2, ...$ of $V(G)$ and there is a tree-decomposition $(B_x: x \in V(T))$ and $|V_i \cap B_x| \leq k$ for all $i$ and all $x$. 
\begin{theorem}
	Planar graphs have $\ltw$ at most 3. 
\end{theorem}
\begin{proof}
	If $G$ is a planar triangulation, and $T$ is a bfs spanning tree, then the vertices ordered by distance from the root $r$ of $T$ is a layering of $G$. Then consider the dual graph $G^*$. Then there exists a spanning tree of $G^*$, $T^*$, such that no edge in $T^*$ crosses over an edge in $T$. Call this spanning tree the cotree of $G$. Then let $\left( B_x: x \in V(T^*) \right)$ be bags. For a face $\alpha$, the set of vertices $v_1, v_2, v_3$ on the border of $\alpha$ and the vertices on path in $T$ from $v_i$ to $r$ for all $i$ in $\lbrace 1, 2, 3 \rbrace$ is $B_\alpha$. Every edge is on the border of some face, so every edge is in $T^*$. If $x$ is a vertex, then the subtree containing $x$ goes to all the faces incident with the descendants of $x$, which is connected. Thus this is a tree-decomposition. Finally, the intersection between $V_i$ and $B_\alpha$ is at most 3 as at most 3 vertices can be on the same layer by the construction. Thus $ltw(G) \leq 3$. 
\end{proof}

\subsection{Upper bound on treewidth}
If $G$ has $n$ vertices with layered treewidth $k$, then $G$ has treewidth $2 \sqrt{kn} - 1$.
\begin{proof}
	Let $V_1, V_2, ..., V_t$ be the layering of $G$ with layered treewidth $k$. Then define $p = \lceil \sqrt{nk} \rceil$. For $j \in \lbrace 1, ..., p \rbrace$, let $W_j = V_j \cup V_{p + j} + V_{2p + j}$, such that $W_j$ separates out the layers. As $W_1, ..., W_p$ is a partition of $G$, as there is a partition $W_j$ with the size at most the average such that $W_j \leq n/p \leq \sqrt{kn}$, then we can cut out the layers of $W_j$ to just have connected components with $p-1$ consecutive layers. Each connected component of $G - W_j$ can be subdivided with the tree-decomposition into bags of size at most $k(p-1)$, thus the treewidth of each connected component is $k(p-1) - 1 = \sqrt{kn} - 1$. Putting each connected component together, we add $W_j$ to every bag in the decomposition and add $W_j$ as another bag to turn the forest into a tree. This will give us a tree-decomposition of $G$ with width at most $\sqrt{kn} - 1 + |W_j| \leq 2 \sqrt{kn} - 1$. 
\end{proof}

\begin{example}
	If $G$ is a $n \times n$ grid, then $\tw(G) = n$, but $\ltw(G) = 2$. 
\end{example}

\begin{theorem}
	If $G$ is a triangulation of a surface with Euler genus $g$, then $G$ has layered treewidth at most $2g + 3$. 
\end{theorem}

\begin{proof}
	We have that $|E(G)| = 3n + 3g - 6$ and $|F(G)| = 2n + 2g - 4$. If $T$ is a $BFS$ spanning on $G$, and $G^*$ is its dual, then we define a new graph $D$ such that $V(D) = V(G^*) = F(G)$ and $xy$ is an edge in $D$ iff $xy$ does not cross an edge of $T$. Note that $D$ is not a tree if $g > 0 $. Finally, the number of edges in $D$ is equal to $|E(G)| - |E(T)| = 3n + 3g - 6 - (n-1) = 2n + 3g - 5$. By definition, $D$ is connected, so let $T^*$ be any spanning tree of $G$, rooted at $r$. $|E(T^*)| = 2n + 2g - 5$, so $|E(D)| - |E(T)| = g$. Denote the $g$ edges as $v_1w_1, v_2w_2, ..., v_gw_g$. As $D$ has $g$ edges not in $T^*$, for every face $f = xyz$ of $G$, let $B_f := P_x \cup P_y \cup P_z \bigcup_{i = 1}^g P_{v_i} \cup P_{w_i}$, where $P_x$ is the unique path in $T$ from $x$ to $r$. Thus $\ltw(G) \leq 2g + 3$. 
\end{proof}
\begin{corollary}
	The treewidth of a graph with genus $g$ is at most $2\sqrt{(2g + 3) n } -1$. 
\end{corollary}

\subsection{Bounded genus and crossings}
A graph is $(g, k)$-planar if there exists an embedding of $G$ on a genus $g$ surface with at most $k$ crossings per edge.If $G$ is $(g, k)$-planar, we shall show that $G$ has treewidth of at most $2\sqrt{(4g + 6)(k + 1) n}$. For fixed $g$ and $k$, this admits a treewidth of $G$ which is $O(\sqrt{n})$. 
\begin{theorem}{\url{https://link.springer.com/chapter/10.1007/978-3-319-27261-0_8}}
	Every $(g, k)$-planar graph $G$ has layered treewidth at most $(4g + 6)(k + 1)$. 
\end{theorem}

\begin{proof}
	Draw $G$ on the plane with an arbitrary orientation of the edges. Then replace every crossing with a new dummy variable to form $G'$. From above $\ltw(G') \leq 2g + 3$. Let $T'$ be the tree decomposition of $G'$, and let $V_0', V_1', ...$ be the layering of $G'$. For each dummy vertex $x$ which crosses the arcs $vw$ and $ab$, where $w$ and $b$ are the tail of the arcs respectively, replace $x$ in $T$ with $w$ and $b$. Then this is tree-decomposition with layered treewidth of $2(k+1) (2g + 3)$. Thus shown. 
\end{proof}

\subsection{$K_t$-minor-free graphs}
Refer to MTH3000 report for definitions.

\begin{theorem}{\url{https://arxiv.org/abs/2104.06627}}
	If $G$ is $K_t$-minor-free, then $tw(G) \leq t^{3/2} n^{1/2}$. 
\end{theorem} 

\begin{lemma}{\url{https://www.ams.org/journals/jams/1990-03-04/S0894-0347-1990-1065053-0/}}
	Let $G$ be a graph and $A_1, ..., A_r$ be nonempty subsets of $V(G)$. Let $x \in \mathbb{R}$, $x \geq 1$. Then either:
	\begin{enumerate}
		\item There exists a tree $T$ in $G$ where $|V(T)| \leq x$ and $V(T) \cap A_i$ is nonempty for all $i$, or,
		\item There exists a set $Y$ in $V(G)$ of size at most $(r-1)n/x$ such that no component of $G-Y$ intersects all of $A_1, ..., A_r$. To rephrase, if $C$ is a component of $G - Y$, then we have there exists a set $A_i$ such that $C$ and $A_i$ are disjoint. 
	\end{enumerate}
\end{lemma}

\begin{proof}
	\newcommand{\dist}{\text{dist}}
	We form a path in $G$ of $A_1 \rightarrow A_2 \rightarrow ... \rightarrow A_r$ such that there exists a path from $A_1$ to $A_r$ passing through all of $A_i$. We then form a new graph $J$ by taking $r-1$ copies of $G$ and between copies $G_i$ and $G_{i+1}$, there is an edge between the copied vertices of $A_{i+1}$ in $G_i$ and $A_{i+1}$ in $G_{i+1}$. Denote the sets $X$ and $Y$ to be $A_1$ in $G_1$ and $A_r$ in $G_{r-1}$ respectively. 
	\par
	Then if $ \dist_J(X, Y) \leq x$, then the path $|P|$ is less than $X$, so we project $P$ back to $G$ and eliminate all loops to form a path, and therefore a tree, of length at most $x$. 
	\par
	If $ \dist_J(X, Y) > x$, then there is a bfs layering in $J$ where $L_i =  \lbrace v \in V(J) : \dist_J(X, v) = i \rbrace$. By construction, there exists a $j$ such that $|L_j| \leq (r-1)n/x$ and in the projection, $L_j$ is a separator. 
\end{proof}

\begin{theorem}[Illingsworth, Scott, Wood]
	For all $t \geq 5$, if $G$ is $K_t$-minor-free, then $G \subseteq H \boxtimes K_{\lfloor m \rfloor}$, where $\tw(H) \leq t-2$ and $m = \sqrt{(t-4)n}$.
\end{theorem}
\begin{proof}
	We shall prove something stronger. For all $t \geq 5$ and $K_t$-minor-free graphs $G$, for all $r \leq t - 1$, and all $K_r$-models $(U_1, ..., U_r)$ in $G$ with $1 \geq |U_i| \geq m$ for all $i$, there exists a $H$-partition of $G$ with width $m$ and $U_1, ..., U_r \in V(H)$, and $tw(H) \leq t-2$, where the width of a partition is the size of the largest bag.
	\paragraph{Base case}
	Let $U = \bigcup_i U_i$. Then suppose if $|V(G)| = U$, then $(U_1, ..., U_r)$ is the desired $H$-partition where $H = K_r$ has treewidth $r-1 \leq t-2$. 
	\paragraph{Inductive case}
	Let $A_i = N_G(U_i) \ U$ for each $i$. If some $A_i$ is empty, then we use the induction hypothesis on $G - U_i$. There is a partition $H'$ of treewidth at most $t-2$ which has $(U_1, ...., U_{i-1}, U_{i+1}, ..., U_{r})$ as a set, with width at most $m$. Then the neighbourhood of $U_i$ is a clique on $r-1$ vertices, so $tw(H) = \max(tw(H'), r-1) \leq t - 2$. 
	\paragraph{Disconnected $G-U$}
	Suppose $G - U$ is disconnected. Then we can partition $V(G - U)$ into two s $C$ and $D$ such that there is no edge from $C$ to $D$. Let $G_1 = G[C \cup U]$ and $G_2 = G[D \cup U]$. Then by induction, there is a partition $H_1$, $H_2$ of $G_1$ and $G_2$ respectively of width at most $t-1$ and treewidth at most $s$ that contains $(U_1, ..., U_r)$ as a bag. Then we let $H$ be the model where we identify the two $H_1$ and $H_2$ models on the bag. Then $\tw(H) = \max(\tw(H_1), \tw(H_2)) \leq s$, with width of $H$ being at most $t-1$.
	\paragraph{Connected $G-U$}
	Use the lemma above and set $A_1, ... A_r$ to be $(A_1, ..., A_r)$ and set $x = m$. Then there exists a tree in $U$ which intersects with $A_i$ for all $i$, or there exists a set $Y$ such that $C$ separates out $C$ and $D$. We can show that the first scenario cannot happen as we have that $Y$ would be too large, so there must be a $Y$ such that this holds. Then $Y \leq (r-1)n/m \leq (tn)/\sqrt(tn) = \sqrt(tn) = m$, so this would be the required set $Y$. 
\end{proof}

\section{Algorithms on bounded treewidth}
\subsection{Monadic Second-order logic}
\begin{theorem}{Courcelle's theorem}
	If a family of graphs $\mathcal{G}$ has bounded treewidth, then any graph property of any graph $G \in \mathcal{G}$ that can be expressed in Monadic Second-order logic can be decided in linear time. 
\end{theorem}

\section{Characterisations of $K_t$-minor free graphs}
What is the structure of $K_t$-minor free graphs? We shall show that we can roughly characterise all $K_t$-minor free graphs as graphs that are products of a series of operations that preserve the treewidth. 
\subsection{$K_t$-minor free minor-closed families}
We define $\had(G)$ to be the largest $t$ such that $G$ has a $K_t$ minor. 
\subsubsection{Planar graphs}
\begin{theorem}
	If $G$ is a planar graph, then $G$ is $K_5$-minor-free.
\end{theorem}
\begin{proof}
	If $G$ is planar with $n$ vertices and $m$ edges, then we have that $m \leq 3n -6$. However, we have that $K_5$ has $5$ vertices and $10$edges, but we have that $ 10 > 3 \times 5 - 6$, so $K_5$ is not planar. As the family of planar graphs is minor-closed, then if $G$ is planar, then $K_5$ is minor-free.
\end{proof}

We can use a different argument to show that $K_{3,3}$ is not embeddable on the plane, by using the fact that $K_{3,3}$ is triangle-free. 
(INSERT PROOF HERE)

\subsubsection{Genus-g graphs}
We define the genus $g$ of a surface to be 2 times the number of handles + the number of crosscaps. From topology, we have that we can add a handle to crosscaps to form 3 crosscaps. Therefore, the Euler characteristic $\chi = 2 - g$ for both orientable and non-orientable surfaces. Note that the genus is defined slightly differently from topology. We do this to allow non-orientable and orientable surfaces to coincide in definition.

We can show that if $G$ has genus $g$, then if $G$ has $n$ vertices and $m$ edges, then $n - m + f = \chi = 2-g$, then as each face has at most 3 vertices and each edge is incident to two faces, we have that $f \leq 2m/3$. Therefore, $m \leq 3(n + g - 2)$, and if $K_t$ is embeddable on a genus $g$ graph, then $\binom{t}{2} \leq 3 (t + g - 2)$. Thus $t \leq \sqrt{6g} + 4$. So if a graph has genus $g$, then it is $K_t$-minor-free, where $t > \sqrt{6g} + 4$. 

\subsubsection{Bounded treewidth graphs}
\begin{theorem}
	If $\tw(G) \leq k$, then $G$ is $K_{k+2}$-minor-free. 
\end{theorem}
\begin{proof}
	We shall prove the contrapositive: If $K_t$ is a minor of $G$, then $tw(G) \geq t-1$.
	If $K_t$ is a minor of $G$, and $\tw(G) \leq k$, then we have that $\tw(K_t) \leq \tw(G) \leq k$, but $\tw(K_t) = t-1 \leq k$, so $t \leq k + 1$. Thus shown a family of minor-closed which are $K_t$-minor free. 
\end{proof}
\subsubsection{Apex vertices}
An apex vertex $v$ is added to a graph $G$ such that it has arbitrary edges. As such, it can simply dominate all other vertices in $G$. Then if $G$ is $K_t$-minor free, $G$ with the apex vertex $v$ is $K_{t+1}$- minor free. 
\subsubsection{Clique-sums}
The \textit{$k$-clique-sum} of two graphs $G$ and $H$, denoted as $G \# H$, is the graph obtained by performing a series of operation on the cliques of $G$ and $H$. We find cliques in $G$ and $H$, $C_G$ and $C_H$ respectively, such that $C_G$ and $C_H$ have size $k$. Then we identify the vertices in $C_G$ and $C_H$ so that $G$ and $H$ are connected to each other on this clique. 

\begin{lemma}
	If $G = G_1 \# G_2$,then $\had(G) = \max(\had(G_1), \had(G_2))$ and $\tw(G) = \max(\tw(G_1), \tw(G_2))$.
\end{lemma}

\begin{example}
	If $G$ is the clique-sum of Euler genus $g$ graphs, then $G$ is $K_{\sqrt{6g} + 5}$-minor-free, but has unbounded genus.
\end{example}

\begin{theorem}[Wagner's theorem]
	If $G$ is $K_5$-minor-free, then $G$ can be obtained from $\leq 3$-clique-sums of planar graphs and the Wagner graph $W_8$.
\end{theorem}


\subsection{Torsos and adhesion}
Given a graph $G$ and a tree-decomposition $\tree$, the \textit{torso} of a bag $B_x$ of $T$ is the graph $G\langle B_x \rangle$, obtained from $G[B_x]$ where $vw$ is a vertex in $G\langle B_x \rangle$ iff $v,w \in B_x \cap B_y$, where $y$ is a neighbour of $x$ in $T$. So the set $B_x \cap B_y$ for all neighbours $y$ of $x$ in $T$ is a clique in $G\langle B_x \rangle$. 
The \textit{adhesion} of a tree is defined as $\max(|B_x \cap B_y|)$ where $xy$ is an edge in $T$.

\subsubsection{Vortices}
Let $G$ be embedded on a surface $\Sigma$, and let $F$ be a face on $G$. Let $D$ be a disc in $\Sigma$ such that $D$ only intersects $G$ only on vertices on the boundary of $F$. We denote these discs as $G$-clean. 

Then let $\Lambda = (x_1, x_2, ..., x_b)$ be a tuple of vertices on the boundary of $F$ such that they intersect $D$. Then we define a new graph $H$ such that $V(G) \cap V(H) = \Lambda$, and there is a path-decomposition of $H$ of bags $B_1, B_2, ... B_b$ such that $x_i \in B_i$ for all $i$. $H$ is denoted as a \textit{$D$-vortex} of $G$. The width of a $D$-vortex is the width of the path above, or $\max_i(|B_i| - 1)$. 

Vortices were created to solve the problem of grid-like graphs with large treewidth, torsos and adhesion, yet are all $K_t$-free for bounded $t$. 
\subsection{Robertson-Seymour theorem}
Given $g, p, a \geq 0$, $k \geq 1$, a graph $G$ is $(g, p, k, a)$- almost embeddable if there exists an $A \subseteq V(G)$ with $|A| \leq a$, and there exists subgraphs $G_0, G_1, ...,  G_{p'}$ of $G$ such that:
\begin{itemize}
	\item $G - A = G_0 \cup G_1 \cup G_2 ... G_{p'}$
	\item $p' \leq p$
	\item There is an embedding of $G_0$ onto a surface $\Sigma$ of genus $\leq g$
	\item There exists pairwise disjoint $G_0$-clean discs $D_1, D_2, ..., D_{p'}$ in $\Sigma$
	\item $G_i$ is a $D_i$-vortex of width at most $k$.
\end{itemize}

\begin{theorem}[Robertson-Seymour graph structure theorem]
	For all $t$, there exists $g, p, a \geq 0$, $k \ell \geq 1$, such that every $K_t$-minor-free graph has a tree-decomposition of adhesion $\leq \ell$ and each torso is $(g, p, k, a)$-embeddable. 
\end{theorem}
In fact, there exists a function $t(g, p, k, a)$ such that if a graph has a tree-decomposition of adhesion $\leq \ell$ and each torso is $(g, p, k, a)$-almost embeddable, then $G$ has no $K_t$ minor. One possible function is $t(g, p, k, a) = a + ck \sqrt{g + p}$. 

\section{Path-width}
We define the path-decomposition of a graph $G$ to be a sequence of bags $B_i$ such that the subsequence of bags containing a vertex $v$ induces a subpath and each edge $vw$ is in a bag $B_i$. Then we define the width of a path-decomposition as $\max_i \lbrace |B_i| \rbrace -1$, same as treewidth. The pathwidth of a graph $G$ is the minimum treewidth.
\begin{example}
	\begin{theorem}[Caterpillars]
		Graphs have pathwidth at most 11 iff every connected component is a caterpillar (graphs where removing every leaf yields a path).
	\end{theorem}
	\begin{proof}[Caterpillars]
		Suppose $G$ is a caterpillar graph and $p_1, p_2, ..., p_n$ is the central path, and the leaves of vertex $p_i$ are denoted as $v_{i, 1}, v_{i, 2} ..., v_{i, k}$. Then have the bags $(v_{1, 1}, v_1), (v_{1, 2}, v_1)... (v_{1, j}, v_1), (v_1, v_2), (v_{2, 1}, v_2), (v_{2,2}, v_2,)... $. We can see that each leaf appears once and each vertex on the central path is on a subpath of the path. Therefore, the pathwidth of $G$ is 1. If $G$ has pathwidth 1, then for each connected component, we choose a vertex $v$ in $B_1$ and a vertex $w$ in $B_n$, the final bag, and look at a path from $v$ to $w$. This path must go through every bag, thus the non-path vertices must have neighbour only of the other one in the bag and thus the graph is a caterpillar. 
	\end{proof}
	
\end{example}
\begin{example}
	If a graph $F$ is a forest, then the pathwidth of $F$ is the largest pathwidth over all connected components.
\end{example}
\begin{example}
	The pathwidth of a single vertex is 0.
\end{example}
\begin{example}
	The pathwidth of a tree $T$ is $\min_{P \subset T} 1 + \pw(T - V(P))$ where $P$ is a path. 
\end{example}

\begin{proof}[Proof of inductive path-width]
	To show $\pw(T) \leq 1 + \pw(T - V(P))$, we have that if $P$ is a path in $T$ with vertices $v_1, v_2, ...$, then consider the subtrees hanging off $v_i$ for all $i$. $T - V(P)$ will have a path-width and we can order each connected component such that they appear in the order of the trees. Then we have that adding $v_i$ to the bags of subtrees connected to $v_i$, and the bag $(v_i, v_{i+1})$ between the subtrees $v_i$ and $v_{i + 1}$ will yield a path-decomposition of width $1 + \pw(T - V(P))$. 
	\paragraph{Other direction}
	To show there exists a path $P$ such that $\pw(T) \geq 1 + \pw(T - V(P))$, we proceed by induction. Let $B_1, ... B_n$ be a path-decomposition of $T$. Let $x$ live in bag $B_1$ and $y$ live in bag $B_n$, the final bag. Then let $P$ be the unique path from $x$ to $y$. Then $P$ traverses through every bag in the path-decomposition. Then $\tw(T) \geq 1 + \tw(T - P)$ by induction. 
\end{proof}

We define a ternary tree to be a tree where every vertex has degree 1 or 4, except for the root $r$, which has degree 3. We define the complete ternary tree of edge-height $h$ to be the unique complete ternary tree where the distance from the root $r$ to any other vertex is at most $h$. Note that the treewidth of any tree is 1.
\begin{lemma}
	Let $T_h$ be the complete ternary tree of edge-height $h$. Then $\pw(T_h) = h$.
\end{lemma}

\begin{proof}
	We shall show for any path $P$, $T_h - V(P)$ has a copy of $T_{h-1}$. Let $P$ be a path, and suppose that it goes the root $r$. Then as $P$ cannot go through all three subtrees of $T_h$ hanging off $r$, there must be a subtree which $P$ does not go through. $T - V(P)$ will contain this subtree, thus $\pw(T_h) \geq 1 + \pw(T - V(P)) \geq 1 + (h-1) = h$.
	\paragraph{Lower bound}
	There exists a path-decomposition of $T_h$ such that the size of each bag is at most $h + 1$. Order the leaves of the balanced ternary trees in the standard way, left to right when drawn with no crossings. Then let $B_i$ be vertices on the path from $r$ to the leaf $\ell_i$. Then this is a tree-decomposition with at most $h + 1$ vertices, thus $\tw(T_h) \leq h$.
\end{proof}

For complete binary trees of edge-height $h$, it is easy to show there is a path-decomposition of width $\lceil h/2 \rceil$. (Consider doing the same operation, but with half the vertices.)

\subsection{Treedepth}
We define the \textit{closure} of a rooted tree $T$ to be the the graph $G$ where $V(G) = V(T)$ and $vw$ is an edge in $G$ iff $v$ is an ancestor of $w$ in $T$.
\begin{definition}
	The treedepth of a graph $G$, denoted as $\td(G)$, is defined to be the minimum vertex-height of a rooted forest $T$ such that $G \subseteq $ closure of $T$.
\end{definition}
We have that this defines a path-decomposition of $G$ by enumerating through all the vertices of $T$ in the natural order with the bag $B_i$ is the path from leaf $\ell_i$ to $r$. The size of the bag is the vertex-height, thus $\pw(G) \leq \td(G) - 1$. As every path-decomposition is a valid tree-decomposition, $\tw(G) \leq \pw(G) \leq \td(G) - 1$. However, these parameters are not \textit{tied}. We say two parameters $p(G), q(G)$ of a graph are tied if there is a function $f$ such that $ p(G) \leq f(q(G))$ and $q(G) \leq f(p(G))$ for all graphs $G$. 
Let $\mathcal{G}$ be the graph family $T_1, T_2, ...$ where $T_h$ is the complete ternary tree of height $h$. Then $\tw(G) = 1$ for all $G \in \mathcal{G}$, but $\pw(T_h) = h$, thus the pathwidth may be unbounded while the treewidth is constant. Let $\mathcal{G}$ be the set of paths $P_i$ for $i \in \mathbb{N}$. Then $\pw(P_i) =  1$, but $\td(P_i) = \lceil i/2 \rceil$. Therefore, the treedepth of graphs is not tied to the pathwidth, or to the treewidth.
\subsubsection{Bounds on treedepth}
\begin{lemma}
	If $T$ is a tree, then $\td(T) \leq \log n$, where we take $\log = \log_2$. 
\end{lemma}
\begin{proof}[Proof of above lemma]
	For all trees $T$, there exists a balanced separator with a single vertex $v$, from above. Therefore, $T-v$ has components of size at most $n/2$. Then we find the separators $w_1,... w_m$ of the other components of $T - v$ and we add an edge from $v$ to $w$, with tree-depth at most $\log(n/2) = \log(n) - 1$. Then $T$ has treedepth at most $\log(n) - 1 + 1 = \log n$. Thus shown. 
\end{proof}

\begin{corollary}
	The tree-depth of a graph $G$ with treewidth $h$ is $O(h \log n)$.
\end{corollary}
The proof of this corollary is by calculating the tree-depth of the tree-decomposition of $G$, which is $O(\log n)$ (we may fudge numbers and bound the number of vertices in the tree-decomposition by the number of edges in $G$ to get the desired result).

\section{Classification of graph families}
As before, we could classify graph families in this way:
\begin{enumerate}
	\item $\tw(G) \leq O(1)$ for all $G \in \mathcal{G}$. These graph families will include trees and paths, so the pathwidth and treedepth is unbounded, but from the lemma above, $\tw(G) \leq O(\log n)$ for all graphs in the class, where we hide the treewidth.
	\item $\tw(G) \leq O(n^{1/2})$ for all $G \in \mathcal{G}$.
	\item $\tw(G) \leq O(n^{1- \varepsilon})$ fixed $\varepsilon > 0$, for all $G \in \mathcal{G}$.
	\item $\tw(G \geq \Omega(n))$ for all $G \in \mathcal{G}$.
\end{enumerate}
We shall prove a very strong statement relating the path-width and tree-depth of a graph $G$ for families of graphs of the form $\tw(G) \leq O(n^{1- \varepsilon})$. 
\begin{theorem}
	If $\mathcal{G}$ is a graph family such that $\tw(G) \leq O(n^{1- \varepsilon})$ for all $G \in \mathcal{G}$, then $\pw(G) \leq O(n^{1-\varepsilon})$ and $\td(G) \leq O(n^{1-\varepsilon})$ for all $G \in \mathcal{G}$.
\end{theorem}
We shall prove this using the following lemma. We define a \textit{hereditary class} $\mathcal{G}$ to be a class closed under vertex deletion. This is a weaker condition than minor-closure. 
\begin{lemma}
	Fix $c > 0$ and $\varepsilon$ in $(0, 1)$. Then suppose for all $n$-vertex graphs $G \in \mathcal{G}$, $G$ has a balanced separator of order at most $c n^{1-\varepsilon}$. Then every $n$-vertex graph $G \in \mathcal{G}$ has $\pw(G) \leq \td(G) \leq c' n^{1- \varepsilon}$. 
\end{lemma}
\begin{proof}
	We use the balanced separator concept from earlier, and use induction on the number of verticesWe have that the balanced separator $S$ where $|S| \leq cn^{1-\varepsilon}$ and consider the connected components of $G - V(S)$. As $G$ is minor closed, then each connected component of $G - V(S)$ will have treedepth at most $c'(n/2)^{1 - \varepsilon}$, where $c' := \frac{c}{1 - 1/(2^{1-\varepsilon})}$. Then suppose we form a complete graph on $S$ and add edges from $S$ to all the vertices in $G - V(S)$ to form the closure of a tree of height at most $c n^{1 - \varepsilon} + c'(n/2)^{1 - \varepsilon}$. However, we have that:
	\begin{align*}
		c n^{1 - \varepsilon} + c'(n/2)^{1 - \varepsilon} &= c n^{1 - \varepsilon} + \frac{c}{1 - 1/(2^{1-\varepsilon})}(n/2)^{1 - \varepsilon}\\
		&= \frac{1}{1 - 1/2^{1-\varepsilon}} (c n^{1 - \varepsilon}(1 - 1/(2^{1-\varepsilon})) + c (n/2)^{1 - \varepsilon})\\
		&= c'n^{1 - \varepsilon}
	\end{align*}
\end{proof}
thus the treewidth is bounded by $c' n^{1-\varepsilon}$. 

\subsection{H-minor-free families}
Let $\mathcal{G}_H:=$ the class of all $H$-minor free graphs.
We have that $(\tw(\mathcal{G}_H), \pw(\mathcal{G})_H, \td(\mathcal{G})_H)$ are $O(n^{1/2})$ constants, where $n$ is the number of vertices. 
\begin{theorem}[Robertson + Seymour + Illingsworth]
	The statement: "There exists a constant $c$ such that for all $G \in \mathcal{G}_H$, $\pw(G) \leq c$" is equivalent to the statement: "$H$ is a forest".
\end{theorem}

\begin{proof}
	Suppose $H$ is not a forest. Then $H$ has a cycle. However, the set of complete ternary trees has no $H$-minor and has unbounded path-width. The other direction is proven in \url{https://www.sciencedirect.com/science/article/pii/009589569190068U}. 
\end{proof}

Therefore, a minor-closed class $\mathcal{G}$ has bounded path-width iff $\mathcal{G}$ excludes some forest. The equivalent statement for treewidth is that: 
"a minor-closed class $\mathcal{G}$ has bounded treewidth iff $\mathcal{G}$ excludes some planar graph". Proof by Robertson + Seymour, graph minors 1.
\begin{lemma}
	For all $n$-vertex planar graphs $G$, $G$ is a minor of the grid on $2n \times 2n$ vertices.
\end{lemma}
We also introduce the grid minor theorem as well.
\begin{theorem}[Grid minor theorem]
	There exists a function $f$ such that every $P_k \square P_k$-minor free graph has $\tw \leq f(k)$. 
\end{theorem}
\begin{proof}[Robertson + Seymour, GM 1]
	Suppose $\mathcal{G}$ excludes a non-planar $H$. Then the family of $n \times n$ grids is $H$-minor free with unbounded treewidth.
	Suppose $H$ is planar of $k$ verses. Then $H$ is a minor of the $P_{2k} \square P_{2k}$ grid. Then $G$ is $P_{2k} \square P_{2k}$-minor-free. Therefore, $\tw(G) \leq f(2k)$ by the grid minor theorem. 
\end{proof}

The grid minor theorem states that treewidth measures how close a graph is to being a tree and how far away a graph is from being a large grid. 
\subsection{Alternative characterisations of treewidth and path-width}
We can show the following:
For all $G$, the $\pw(G) = \min k$ such that $G$ is a spanning subgraph of an interval graph with no $K_{k + 2}$, so the intervals cross at most $K + 1$ times. 

We can show a similar result for treewidth as well:
For all $G$, $\tw(G) = \min k$ such that $G$ is a spanning subgraph of a \textit{chordal graph} with no $K_{k + 2}$ subgraph. A chordal graph is a graph with no induced cycle with more than 4 vertices. This is equivalent to saying that $G$ has $\tw \leq k$ iff $G$ is a spanning subgraph of the intersection graph of a tree $T$ with max clique size $k + 1$. 
$G$ is chordal if all minimal separators are a clique. 


\section{Proof of Path-Width theorem}
\begin{theorem}
	\label{thm:Path-Width theorem }
	For every forest $F$, if a graph $G$ is $F$-minor-free, then $\pw(G) \leq |F| - 2$.
\end{theorem}
The contrapositive is:
\begin{theorem}
	\label{thm:pw-tree contrapositive }
	If a graph $G$ has path-width at least $w$, and $F$ is a forest where $|F| \leq w + 2$, then $G$ contains $F$ as a minor. 
\end{theorem}
Note that a complete graph on $|F|-1$ vertices has pathwidth $|F| - 2$. 
We say that a separation is a pair $(A, B)$ of $V(G)$ where $A \cup B = V(G)$, there are no edges between $A - B$ and $B - A$, and the order of the separation is $|A \cap B |$. We can consider $A$ as being on the left of some separator, and $B$ on the right of some separator. We say that $(A, B) \leq (A', B')$ if:
\begin{itemize}
	\item $A \subseteq A'$
	\item $B' \subseteq B$
\end{itemize}
So $(A', B')$ is to the "right" of $(A, B)$. For each $w \geq 0$, we say that $(A, B)$ is $w$-good, if we can decompose $A$ into a path of width at most $w$ and the last bag is $A \cap B$, the separator. 

\subsection{Proof}
\begin{lemma}
	If $(A, B)$ and $(P, Q)$ are separations of $G$, $(A, B)$ is $w$-good, $(P, Q) \leq (A, B)$, and there are $|P \cap Q|$ vertex-disjoint paths of $G$ between $P$ and $B'$, then $(P, Q)$ is $w$-good. 
\end{lemma}
\begin{proof}
	Let $R_1, ..., R_t$ be disjoint paths between $P$ and $B'$, where $t = | P \cap Q|$. Then we have that each path must pass through $P \cap Q$ as this is a separator, thus there are at least $t$ elements in the separator. Additionally, each path must belong inside $A$ as $A \cap B$ is also a separator. Therefore, $t \leq |A \cap B|$. 
	Let $H = G[P] \cup \lbrace R_1, ..., R_t \rbrace$. As $H$ is in $A$, then there exists a path-decomposition of width at most $w$ where the last bag is precisely $A \cap B$, or the endpoints of $R_1, ..., R_t$. But contracting $R_1, ..., R_t$ to a single point yields a path-decomposition where the last bag is $P \cap Q$, as contracting $A \cap B$ along $R_1, ..., R_t$ yields $P \cap Q$. 
\end{proof}

We say that if $(A, B)$ and $(A', B')$ are separations of $G$, the second \textit{extends} the first if $(A, B) \leq (A', B')$ and the order of $(A', B')$ is at most the order of $(A, B)$. We say that a $w$-good separation of $G$ is maximal if no other $w$-good separation extends it. 

Let $T$ be a tree, $(A, B)$ a separation. We say $(A, B)$ is $(w, T)$ spanning if:
\begin{itemize}
	\item $|A \cap B| = |T|$
	\item There exists a model $\varphi$ of $T$ in $G[A]$ such that each block in the model contains a vertex in $A \cap B$
	\item If $|T| \leq w + 1$ then $(A, B)$ is maximal $w$-good.
\end{itemize}

\begin{lemma}
	Let $w \geq 0$ be an integer. Let $G$ be a graph with path-width at least $w$, and let $T$ be a tree, with $|T| \leq w + 2$. Then there exists a $(w, T)$ spanning separation of $G$. 
\end{lemma}

\begin{proof}
	Fix $w$. Then if $|T| = 0$, let $A = \emptyset$, $B = V(G)$. Then $(A, B)$ is a $w$-good separation which is $(w, T)$-spanning.
	
	If $|T| = 1$, then let $v = V(T)$. Then we choose an arbitrary $v \in B$ such that $A = \lbrace v \rbrace$, $B = V(G)$, and extend by maximality. 
	If $2 \leq |T| \leq w + 1$, then let $v$ be a leaf of $T$. we have that there is a  $(w, T - v)$-spanning separation $(A, B)$ of $G$ which is maximal $w$-good. Then let $u$ be the neighbour of $v$ in $T$ and let $u'$ be the intersection of the block of $u$ in $G$ and $B$. Then $u'$ has a vertex $v'$ in $B - A$, otherwise $(A, B - \lbrace u' \rbrace)$ is $w$-good and extends $(A, B)$, contradicting the maximality of $(A, B)$. 
	
	We assume that $(A \cup \lbrace v' \rbrace, B)$ is $w$-good. Therefore, there exists a maximal $w$-good separation $(A', B')$ that extends $(A \cup \lbrace v' \rbrace, B)$. Then $(A', B')$ has order exactly $|T|$ as it did not extend $(A, B)$. Now we wish to show that $(A', B')$ is $(w, T)$-spanning.
	\paragraph{Contradiction argument}
	Suppose not. Then there exists less than $|T|$ paths between $A \cup \lbrace v' \rbrace$ and $B'$. By Menger's theorem, then that implies that there exists a separator between $A \cup \lbrace v' \rbrace$ and $B'$ of order $< T$. Call this separator $(P, Q)$ with minimum order.
	
	We have that $(A \cup \lbrace v' \rbrace, B) \leq (P, Q) \leq (A', B')$. By Menger's theorem, there exists $|P \cap Q|$ vertex-disjoint paths from $P$ to $B'$. Therefore, $(P, Q)$ is $w$-good, by the above lemma. But $(P, Q)$ must extend $(A, B)$, since $|P \cap Q| \leq |A \cap B|$, and $(P, Q) \neq (A, B)$. However, this contradicts the maximality of $(A, B)$.
	Therefore, there are $|T|$ disjoint paths between $A \cup \lbrace v \rbrace$ and $B'$. Therefore, $(A', B')$ is $(w, T)$-spanning.
	
	\paragraph{Conclusion}
	Therefore, there is a $(w, T)$-spanning path separator $(A, B)$ where $|T| = w + 2$. Therefore, we have that $G[A]$ contains a model of $T$, therefore $G$ contains a model of $T$. Thus shown. 
\end{proof}

\section{Chordal partitions}
Let $G$, $H$ be graphs. We let a $H$-partition of $G$ be defined as a function $f: V(G) \rightarrow V(H)$ such that for all edges $vw \in E(G)$, we have that either:
\begin{itemize}
	\item $f(v) = f(w)$, or
	\item $f(v) f(w) \in E(H)$.
\end{itemize}
We also impose that $f^{-1}(x)$ is connected for all $x \in V(H)$. 

We define the width of a $H$-partition is $width(f) := \max( |f^{-1} (x)| : x \in V(H))$. Therefore if the width is $k$, then we have that $G \subseteq H \boxtimes K_k$, or that $G$ is a subgraph of the $k$-regular inflation of $H$. 

Recall that a chordal graph is one where every induced cycle is a triangle. Alternatively, we say that $H$ is chordal if it has a simplicial vertex $v$ such that $H - v$ is chordal, where a simplicial vertex is one where all its neighbours form a clique. Another definition is iff $\tw(H) = \omega(H) - 1 = \chi(H) - 1 = had(H) - 1$. 


\begin{theorem}
	If $G$ has no $K_t$-minor, $f$ is a chordal $H$-partition of $G$, then $H$ is a minor of $G$ and $H$ has no $K_t$-minor.
\end{theorem}

This would imply that $\tw(H) + 1 = had(H) \leq t - 1$. 

We shall prove the stronger result below. 
\begin{theorem}
	Every $K_t$-minor-free graph $G$ has a chordal $H$-partition such that each part of the partition has:
	\begin{itemize}
		\item $\Delta \leq 3t - 10$
		\item $\pw \leq 2t - 7$
		\item $bw \leq 2t - 7$
	\end{itemize}
\end{theorem}


\begin{proof}
	We say a subgraph $X \subset G$ is "processed" if $X$ has the properties listed above.
	
	Take a chordal $H$-partition of $G$ such that:
	\begin{enumerate}
		\item each unprocessed part is simplicial in $H$
		\item the number of vertices of $G$ in a processed part is maximal. 
	\end{enumerate}
	Let $A$ be an unprocessed part on a simplicial vertex of $H$. Then we have that the neighbours of $A$ in $H$ can be written as $X_1, ..., X_k$ where $k \leq t - 2$. Then we look at the vertex $r$ in $A$ with a neighbour in $X_1$ and label paths $P_1 ... P_{k - 1}$ where $P_{i + 1}$ is the shortest path from $r$ to a vertex $v_i$ where $v_i$ has a neighbour in $X_i$. Then we have that this is a depth-first subtree, therefore we partition the graph into level sets $L_1, ..., L_m$ where $L_i$ are all the vertices on $P_1, ..., P_{k - 1}$ of distance exactly $i$ from $r$. Then we let the new processed part, $A'$, be the union of all $P_1, ..., P_{k - 1}$. Then we have that $A'$ has the property that every vertex has $\Delta$ at most $3t - 10$ from the fact that edges can only go between level sets $i$, $i-1$ and $i + 1$ and there are at most $k-1$ level sets. We also have that the path-decomposition is the concatenations of levels $i$ and $i + 1$ meaning that each bag is at most $2t - 7$ vertices. Finally, the maximum distance between any two sets is $2t - 7$, thus $bw \leq 2t - 7$.
\end{proof}

\subsection{Hadwiger's conjecture and simple partial result}

\begin{theorem}
	For all graphs $G$ which are $K_t$-minor free, there exists a $H \subseteq G$ such that $|V(H)| \geq \frac{1}{2}|V(G)|$ and $\chi(H) \leq t - 1$.
\end{theorem}
We will use a simpler lemma.

\begin{lemma}
	For all connected graphs $X$, there exists an independent set $I$ and a dominating set $D$ such that $|D| = 2 |I| - 1$ for $I \subset D$. 
\end{lemma}
\begin{proof}
	Take $I$ and $D$ to be maximal such that $|D| = 2|I| - 1$ and $I$ independent, $I \subseteq D$. We have that for any single vertex $v$, $\left\lbrace v \right\rbrace = I = D$, satisfying the equation.
	
	Now suppose $D$ does not dominate $X$. We find a vertex not dominated by $D$ and find a vertex $x$ on the shortest path such that $x$ is of distance 2 from $D$. Now we take $y$, the neighbour of $D$. We add $y$ to $D$ and $x$ to $I$, which satisfies the induction hypothesis.
\end{proof}

Now to prove the theorem above.

\begin{proof}
	Let $G$ be a connected $K_t$-minor free graph and let $I_0$ and $D_0$ be the dominating set and independent set as described above. Now let $G_1 = G - D_0$ and repeat on the connected components of $G_1$ for the sets $G_1, ..., G_k$, and then repeat on the connected components and so on. After exhausting all connected components, we have that $(I, D)$ forms a tree-like structure, and in fact the sets form a closed tree. We have that the depth of this tree is at most $t - 1$ as each path from a leaf to the root is a clique. Therefore, we have that the closed tree is chordal when we contract each $D$, and in fact we can contract each $D$ to the corresponding $I$ to yield a minor $H$ such that $V(H) = \cup I \geq \frac{1}{2} |V(G)|$ and we colour (by colour we mean colouring each vertex in $I$ the same colour) each level set of $I$ the same colour, to yield that $H$ requires at most $t-1$ colours. 
\end{proof}

\section{Brambles and treewidth}
We define a bramble $B$ to be a set $B_i \subseteq V(G)$ such that for all $B_i, B_j$, we have that either $B_i \cap B_j \neq \emptyset$ or there exists an edge between $B_i$ and $B_j$ and $G[B_i]$ is connected for all $i$. We define the hitting set of a bramble $B$ to be the the set of vertices $X \subset V$ such that $X \cap B_i \neq \emptyset$ for all $i$. We define the order of a bramble $B$ to be the smallest $|X|$ where $X$ is a hitting set. Then we define $\bn(G)$ to be the maximum order over every bramble. 

For $n \times n$ grids, we take a cross to be the union of a column and a row. Then we have that the set of all crosses forms a bramble, and the hitting set is of size at least $n$, as we must intersect all rows and columns. Therefore, we have that $\bn(n\times n \text{ grid}) \geq n$. 

\begin{theorem}[Seymour and Thomas]
	\begin{equation}
		\bn(G) = \tw(G) + 1.
	\end{equation}
\end{theorem}

\begin{proof}[$\bn(G) \leq \tw(G) + 1$]
	We use the same trick that we did previously. We take a tree-decomposition of $G$ $(X_i)_i$ and we use the fact that the intersection set between two neighbouring bags is a separator of $G$. So we have that $X_v \cap X_w$ is a separator for $G$. Let $T_{v:w}$ be the subtree on the end of $X_v$ not including elements in $X_w$. Then we cannot have a bramble with one set in $T_{v:w}$ and $T_{w:v}$, as that would mean that the bags would not touch as they do not include vertices in the separator. Therefore, we have that all brambles either have a vertex in $X_v \cap X_w$, or they all live in either $T_{v: w}$ or $T_{w: v}$. Now orient the edge $vw$ to point towards the side which contains brambles in $T_{v:w}$ or $T_{w:v}$. Then we have a bag $B_x$ such that all arrows point towards $B_x$. Then $B_x$ is a hitting set. Therefore for any bramble, the hitting set is at most $\tw(G) + 1$, therefore we have that $\bn(G) \leq \tw(G) + 1$. 
\end{proof}
To prove the reverse direction, we wish to construct a bramble such that the smallest hitting set is at least $\tw(G) + 1$. As we are taking this over maximal hitting sets, then proving an upper bound. Proof by Mazoit.  
\begin{theorem}
	If $G$ has $\tw(G) = k + 1$, then $G$ has a bramble of size $k$. Thus, $\bn(G) \geq \tw(G) + 1$. 
\end{theorem}

To prove this statement, let us define some terms. Let $G$ be a graph. Fix $k$ as an integer and $(B, \tree)$ be a tree-decomposition of $G$. We say a bag $B_x$ is \textit{small} if $|B_x| \leq k$ and \textit{big} if $|B_x| \geq k + 1$. Suppose $B_x$ is a big leaf bag with neighbour $B_y$. Then define $B_x - B_y$ as a $k$-flap of $\tree$. We say a tree-decomposition is $k$-partial if every non-leaf bag is small, and $\geq 1$ bag is small. 

We shall use a technical lemma to build new tree-decompositions in a ``correct'' way.

\begin{lemma}
	Let $G$ be a graph, $k$ integer and fixed. Then suppose we have a $k$-partite tree-decomposition $\tree_X$ of $G$ and a $k$-partite $\tree_Y$ of $G$. Let $X$ be a $k$-flap of $\tree_X$ and $Y$ be a $k$-flap of $\tree_Y$. Then there is a tree-decomposition $\tree$ such that for all $k$-flaps $Z$ in $\tree$, we have that there is a $k$-flap $Z'$ in $\tree_X$ or $\tree_Y$ such that $Z' \neq X$ and $Z' \neq Y$ and $Z \in Z'$.  
\end{lemma}

We shall put off the proof of lemma until later. For now, we shall assume the lemma as given and prove the theorem. 

\begin{proof}
	Set $k = \tw(G)$.
	Let $\beta_0$ be the set of all $k$-flaps over all $k$-partial tree-decompositions of $G$. We say a set $\beta$ is \textit{upwards-closed} with respect to $k$-flaps if $C \in \beta$, $D$ is a $k$-flap, and $C \subseteq D$, then $D$ is a $k$-flap. Obviously, $\beta_0$ is minor closed. 
	
	Then let $\beta$ be a set of $k$-flaps in $G$ such that:
	\begin{enumerate}
		\item $\beta$ contains $\geq 1$ $k$-flap from every $k$-partite tree-decomposition of $G$
		\item $\beta$ is upwards-closed
		\item For all $X \in \beta$, $\beta - \langle X \rangle$ violates at least one of the conditions above. This is referred to as inclusion-wise minimal.
	\end{enumerate}
	We have that $\beta$ is nonempty as we remove elements from $\beta_0$ until the conditions above hold. 
	
	\begin{lemma}
		For all $X, Y \in \beta$, $X$ and $Y$ touch.
	\end{lemma}
	\begin{proof}
	Suppose not. Then we take a minimal $X, Y$ contained in the original such that $\beta - X$ and $\beta - Y$ does not violate (2). Then $\beta - X$ and $\beta - Y$ fails (1), so there exists $\tree_X$ and $\tree_Y$ such that the only $k$-flap in $\beta$ is $X$ and $Y$ respectively. Then we use the lemma above to create a tree-decomposition $\tree$. $\tree$ is $k$-partial, meaning that none of its internal nodes have at least $k$ elements. But this implies from treewidth that $\tree$ has a leaf node with at least $k + 1$ elements, which is big. But none of these leaf nodes are in $\beta$. They cannot be a subset of another $k$-flap, because there are no others apart from $X$ and $Y$. But they cannot be a subset of $X$ or $Y$ as $X$ and $Y$ are minimal in $\beta$. Thus shown.
	\end{proof}
	Now let us take $\beta'$ to be the set of connected subsets in $\beta$. Now $\beta'$ is a bramble. Let $S$ be a hitting set in $\beta'$, and suppose $|S| \leq k$. Then we think about the connected components of $G - S$, call them $C_1 ... C_p$. For each $i$, define $N_i = C_i \cup N(C_i)$. Then $N_1 ... N_p \cup C$ is a tree-decomposition, with $C$ as a star. But this is a $k$-partial decomposition, as $C$ is small. Furthermore, as $\tw(G) \geq k + 1$, then there is an $N_i$ which is big. But this implies that $C_i$ is a connected $k$-flap. Therefore, $C_i \in \beta'$. But $C_i$ is not hit by $S$, thus $S$ is not a hitting set.
	
	Therefore, all hitting sets of this bramble are of size $\geq k + 1$, thus $\bn(G) = \tw(G) + 1$. 
\end{proof}

Now to prove the technical lemma.

\begin{proof}
	Suppose $G$ is a graph. Let $k$ be a fixed integer. Then define $\tree_X$ and $\tree_Y$ as above, define $X$ and $Y$ as above.
	
	As $X$ and $Y$ are not touching, suppose we have a minimal separator $Z$. Let $U_X$, the bag containing $X$, neighbour $U_q$. Then $U_q$ is a separator of $X$ from the rest of $G$, therefore $U_q$ is also a separator of $X$ from $Y$. As $|U_q| \leq k$ as $U_q$ is either an internal node or a leaf of $K_t$, then $X$ has a separator of size at most $k$. 
	
	Let $S$ be a minimal separator of $A$ and $B$. By Menger's theorem, there exists $|S|$ disjoint paths from $A$ to $S$. Let $A = S \cup$ all connected components that lie on the path from $S$ to $X$ and let $B$ be $G - A + S$. 
	
	We claim that there exists a $k$-partial tree-decomposition $\tree_B$ of $G[B]$ with $S$ a leaf bag and for all $k$-flaps $Z$ of $\tree_B$, there exists a $k$-flap $Z'$ in $\tree_A$ such that $Z \subseteq Z'$ and $Z' \neq X$. 
	
	For all $s \in S$, let $w_s$ be a node of the tree of $\tree_X$ such that $s$ is in the bag $U_{w_s}$. 
	For all $v$ in the tree of $\tree_X$, let $D_v' = (D_v \cap B) \cup \lbrace s \in S : v \text{ is in the } w_s \text{x-path in the tree-decomposition } \tree_X \rbrace$. 
	
	Then $|D_v'| \leq |D_v|$ as when we add elements back to $D_v'$ we removed them already from $B$, so the size does not increase.
	Furthermore, we have that $S$ is its own bag and since it is a subset of $U_q$ then it is a leaf of $\tree_A$. Furthermore, $S$ is also small.
	
	We then do the same procedure with the set $B$. We then have $\tree_B$ as the tree, with $S$ being a small bag. Then we attach $\tree_A$ and $\tree_B$ by taking the disjoint union and adding an edge between both bags containing $S$. This is a tree-decomposition which satisfies the above properties. 
\end{proof}


We have this short lemma that may be useful in the future.
\begin{lemma}
	Suppose $G$ is a graph with components $G_0$ and $G'$. Suppose $G_0$ is embedded on a surface $\Sigma$ of genus $g$ and let $F$ be a face on $G_0$. Let $v_1, v_2, ..., v_k$ be the vertices bordering $F$, and let $C$ be the cycle bordering $F$. Let $D$ be a $d$-clean disk on $F$. Now suppose $G'$ is a vortex on $D$ with a path-decomposition $(B_0, ... B_l)$. Suppose $G_0$ has a book-embedding $(<, \phi)$. Then partition the edges $e_i = v_i v_{i + 1}$ (modulo $k$) such that the edges form a maximal $\phi$-monochromatic path on $C$. Suppose there are $m$ paths. Then $G$ has a book-embedding with $pn(G) + f(m)$ pages.
\end{lemma}
We shall prove this auxillary lemma for each $l$. 
\begin{lemma}
	Let $(B_1, ..., B_n)$ be a path-decomposition of $G$ with path-width $k$. Let $x_1, ..., x_n$ be vertices in $G$ such that $x_i \in B_i$ for all $i$. Then we have that for any one-page embedding of $x_1, ..., x_n$, $G$ has a $k + 1$-page embedding. 
\end{lemma}

\begin{proof}
	For each $v \in V(G)$, let $b(v) := \min \left\{ i : v \in B_i \right\}$. This partitions $V(G)$. Let $(\leq)$ be the one-page embedding of $x_1, ..., x_n$ and suppose we have a circular ordering. Colour the vertices such that every vertex in each bag is assigned a different colour. Now for all edges $uv$ in $E(G)$ where $b(u) \leq b(v) $, set $\phi(uv) = col(u)$. 
	
	Now for all $i$, place $b^{-1}(i)$ immediately clockwise from $x_i$. We claim that this is a book-embedding. 
	
	Suppose edges $uv$ and $xy$ are assigned the same colour, $u < v$, $u \leq x$ and $x \leq y$. Suppose $h(v) = i$. Then we can draw a line such that $B_i, B_1$ and $x_1$ is on one side and $B_{i + 1}, B_{i + 2}, ..., B_k$ from the fact that $x_1$ form a planar partition of the graph $G$. This is possible from the Jordan curve theorem and the fact that the line that goes through $x_1 ... x_k$ is planar. Then this implies that $uv$ and $xy$ do not cross. 
	If $u = x$, then we can follow the path $P$ from $x_1$ to $x_k$ to reach $u$ with no crossings. (Draw picture!)
\end{proof}

\section{A bird-eye's view}
Consider the world of graph families that are not minor-closed. This is a much stranger world than the ones that we deal with, though we do understand it rather well.

The most interesting graph families are graphs where the number of edges is sparse in some way. We will define the notion of sparsity further. 

\subsection{1-planar graphs} 
Consider the family of $1$-planar graphs $\mathcal{F}$. $G$ is $1$-planar if there exists an embedding such that each edge of $G$ has at most one crossing. 
\begin{lemma}
	$\mathcal{F}$ is not minor closed.
\end{lemma}
$\mathcal{F}$ contains graphs $K'_t$ such that $K_t$ is a minor of $K'_t$. Consider $K_t$ drawn on the plane with straight edges. For each edge, subdivide it such that the edge crosses over once. Then $K_t$ is a minor of this graph as we can unsubdivide.

Note that every $1$-planar graph is almost planar! We can construct every 1-planar graph from a 2-connected planar graph (with faces of size 3 or 4) and adding 2 crossing edges to each face of size 4.
\subsection{Shallow minors}
We say that $H$ is an $r$-shallow minor if $H$ is isomorphic to a graph obtained from $G$ by contracting disjoint subgraphs of radius $\leq r$, deleting vertices or edges. Another wy to state this is that $H$ is a model of $G$ where each preimage of the model has radius $\leq r$. 

We say $\nabla_r(G) :=$ the maximum average degree of an $r$-shallow minor of $G$. 
We define $\nabla(G)$ to be the maximum average degree of any minor of $G$. 

\begin{theorem}
	Let $\mathcal{G}$ be a class such that there exists a $c$ such that for all $G \in \mathcal{G}$, for all $r \in \mathbb{N}$, $\nabla_r(G) \leq c$. Then $\mathcal{G}$ is contained in a proper minor-closed class. 
\end{theorem}
This is because there exists a $t = c + 2$ such that for all $g \in \mathcal{G}$, $G$ is $K_t$-minor free. 
Then $\nabla_r(G) \leq c$, as $\nabla(G) \leq c$. 
Thus $\mathcal{G}$ is inside a proper minor-closed class. 

\subsection{Linear expansion}
We say a graph family $\mathcal{F}$ has linear expansion if $\nabla_r(G) \leq cr$ for all $G \in \mathcal{F}$ and for all $r$. 

Alternatively, $\nabla_r(G) = O(r)$. 
This set of families include all those contained in proper minor-closed classes, from the lemma above. 

We claim that 1-planar graphs have $\nabla_r(G) \leq cr$. 

\subsection{Proof of linear expansion of 1-planar graphs}

We will use a small lemma to prove this statement. Recall that the ltw of $G$ is the minimum $k$ such that there exists a layering $(L_1, ..., L_k)$ and a tree-decomposition $(B_x)_{x \in V(T)}$ such that for all $i, 1 \leq i \leq k$ and $j \in V(T)$, $|L_i \cap B_x | \leq k$. 

\begin{lemma}
	If $H$ is an r-shallow minor of $G$, then $\ltw(H) \leq 2 r \ltw(G)$. 
\end{lemma}

Let $H$ be a $r$-shallow minor of $G$. Let $(B_x)_x$ be a tree-decomposition of $G$. Then contracting every $r$-radius subgraph in $G$ to form $H$ means replacing all of the vertices in the $r$-radius subgraph in $(B_x)_x$ with a new vertex. Furthermore, let $a$ and $b$ be two distinct subgraphs in $G$ which are contracted to form $H$. Then if $a$ and $b$ are adjacent in $H$, then the distance between the centre of $a$ and the centre of $b$ is at most $2r + 1$. Therefore if we look at the layering $(L_1, ..., L_k)$ of $G$, then combining $(L_1, ..., L_{2r})$, $(L_{2r + 1}, ... , L_{4r})$ and so on will be a layering of $H$. Then we have that $\ltw(H) \leq 2r \ltw(G)$ as the size of each bag remains the same but the size of each layer grows by at most $2r$. 

\begin{lemma}
	For all graphs $G$, the minimum degree $\delta(G)$ is at most $3 \ltw(G)$.
\end{lemma}
Let $G$ be a graph with layering $(L_k)_k$ and book-embedding $(B_x)_x$ such that the intersections is at most $\ltw(G)$. Then let $B_s$ be a leaf-bag. If $B_s$ is a subset of its neighbouring bag, then combine $B_s$ and repeat. Then there exists a leaf bag $B_s$ such that $B_s$ contains a vertex $q$ not in the neighbouring bag. But $q$ lives in a layer $L_i$, and all of its neighbours must be in $B_s$. But this means that $q$ must only have neighbours that are in $B_s$ and in either $L_i$, $L_{i - 1}$ or $L_{i + 1}$. But this means that $q$ has at most $3 \ltw(G)$ neighbours. Thus shown.

By removing $q$ and repeating this operation, we can iterate through the entire graph. Therefore, the number of edges is at most $3n \ltw(G)$. But this implies that $2m/n = 6 \ltw(G)$, so the average degree is at most $6 \ltw(G)$. 
\end{document}
