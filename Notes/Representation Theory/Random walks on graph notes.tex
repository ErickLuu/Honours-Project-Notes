\documentclass[]{article}
\usepackage[margin=1in]{geometry}

\usepackage{amsmath}
\usepackage{amssymb}
\usepackage{amsthm}
\usepackage{url}

% Environments

\newtheorem{theorem}{Theorem}
\newtheorem{proposition}[theorem]{Proposition}
\newtheorem{corollary}[theorem]{Corollary}
\newtheorem{lemma}[theorem]{Lemma}
\newtheorem{definition}[theorem]{Definition}
\newtheorem{conjecture}[theorem]{Conjecture}

\theoremstyle{definition}
\newtheorem{example}[theorem]{Example}

\numberwithin{theorem}{section}
\numberwithin{equation}{section}

\newcommand{\supp}{\text{Supp}}

%opening
\title{Random walks on groups notes}
\author{Eric Luu}

\begin{document}

\maketitle
\section{Motivation}
Notes from Steinberg, Representation theory of finite groups, chapter 11. All group multiplication will be done left to right. 
In a famous paper " Trailing the dovetail shuffle to its lair", Bayer and Diaconis showed that you had to riffle-shuffle a deck of cards seven times to randomly shuffle the deck. The paper used random walks on $S_n$ to model shuffling the deck.

\section{Definitions}
Suppose $G$ is a finite group with random variable $X$ and $P$ is a probability distribution function where $P(g) = \text{Prob}[X = g]$ and $\sum_{g\in G} P(g) = 1$. We can omit $X$.

For a set $A \subseteq G$, we have that $P(A) = \sum_{g \in A} P(g)$. We define the support of $P$ as $\supp(P) = \lbrace g \in G : P(g) \neq 0 \rbrace$. 
\section{Creating probability distributions}
We define the uniform distribution $U$ on $G$ as $U(g) = 1/|G|$ for all $g \in G$. 
We can convolve a probability like so. Let $P$, $Q$ be probabilities on $G$. Suppose we choose $X$ according to $P$ and $Y$ according to $Q$. Then the probability that $XY = g$ is the probability that $Y = h$ for some $h \in G$ and the probability that $X = gh^{-1}$. 
So $P \ast Q(g) := \text{Prob}[XY = g] = \sum_{h\in G} P(gh^{-1})Q(h) $.

\begin{proposition}
	Let $P$ and $Q$ be probability distributions on $G$. Then $P \ast Q$ is a probability distribution on $G$, with $\supp(P \ast Q) = \supp(P)\supp(Q) = \lbrace xy : x \in \supp(P), y \in \supp Q \rbrace$. 
\end{proposition}
\begin{proof}
	Trivial to show is between 0 and 1.
	Next, we have that:
	\begin{align*}
		\sum_{g \in G} P \ast Q(g) &= \sum_{g \in G} \sum_{h \in G} P(gh^{-1}) Q(h)\\
		&=\sum_{h \in G} Q(h) \sum_{g \in G} P(gh^{-1})\\
		&= \sum_{h \in G} Q(h)\\
		&= 1
	\end{align*}
\end{proof}
Thus this is a probability distribution. Note that the support is also trivial.

\section{Notion of difference between probabilities}
We want to quantify how far a distribution is away from being uniform. We define the $L^1$ norm as $\|f\|_1 = \sum_{g \in G} |f(g)|$ for $f : G \rightarrow \mathbb{C}$. The $L^1$-norm is a metric and also $\|a \ast b\|_1 \leq \|a\|_1 \cdot \|b\|_1$. 

Proof in notes.

We define the total variation distance between probabilities $P$ and $Q$ on a group $G$ as $\| P - Q\|_{TV} = \sup_{A \subseteq G} |P(A) - Q(A)|$. 

\begin{lemma}
	Let $P$ and $Q$ be probabilities on $G$, and let $B = \lbrace g \in G : P(g) \geq Q(g) \rbrace$, $C = \lbrace g \in G : Q(g) \geq P(g) \rbrace$.
	Then $\|P - Q \|_{TV} = P(B) - Q(B) = Q(C) - P(C)$.
\end{lemma}
\begin{proof}[heading]
	We have that as $\sum_{g \in G} P(g) = \sum_{g \in G} Q(g) = 1$. 
	
	
	By definition, $\|P - Q\|_{TV} \geq |P(B) - Q(B)| = P(B) - Q(B)$.
	Let $A$ be a set such that $\|P - Q\|_{TV} = |P(A) - Q(A)|.$ Then $|P(A^c) - Q(A^c)| = |1 - P(A) - (1 - Q(A))| = |Q(A) - P(A)| = \|P - Q \|_{TV}$. Therefore, this is closed under complement. It is trivial to show that $P(B) - Q(B) = |Q(A) - P(A)|$. 
\end{proof}
Finally, we have that $\|P - Q \|_{TV}$ is a metric:
\begin{proposition}
	$\|P - Q\|_{TV} = \frac{1}{2} \|P - Q \|_1$. 
\end{proposition}

\begin{proof}
	\begin{align*}
		\|P - Q\|_{TV} &= \frac{1}{2}[P(B) - Q(B) + Q(C) - P(C)]\\
		&=\frac{1}{2}\left[ \sum_{g \in G : P(g) \geq Q(g)} (P(g) - Q(g)) + \sum_{g \in G : Q(g) \geq P(g)} (Q(g) - P(g)) \right]\\
		&= \frac{1}{2}\left[ \sum_{g \in G}|P(g) - Q(g)| \right]\\
		&= \|P - Q\|_1
	\end{align*}
\end{proof}

We say that a sequence of probabilities $\lbrace P_n \rbrace_{n \in \mathbb{N}}$ converges to $P$ if for all $\varepsilon > 0$, there exists $N > 0$ such that $\|P_n - P \|_{TV} < \varepsilon$ when $n \geq N$. 
\section{Random walks}
We denote $P^{\ast k}$ as $\underbrace{P \ast P \ast ... \ast P}_{k \text{times}}$. 
The way to think about random walks is as follows. We start at the identity and move to $X_1$ of $G$ according to $P$. Then we choose another element $X_2$ and move to $X_1 X_2$. So this is a sequence of i.i.d. variables with distribution $P$.

Let $Y_0$ be the random variable with distribution $\delta_1$, so $Y_0 = id$ with probability 1. Then $Y_k = Y_{k-1} X_k$ for $k \geq 1$. This $Y_k$ is the distribution of the position on the $k$-th step of the random walk. We have that $Y_k$ is a random variable with distribution $P^{\ast k}$. 

This forms a Markov chain. 
\subsection{Examples of random walks on graphs}
Let  $G$ be a group and $S$ be a symmetric subset. Let $\Gamma$ be the Cayley graph of $G$ with respect to $S$. The simple random walk is to choose a random element $s \in S$ with uniform distribution and move to the vertex $gs$. By construction, the vertices adjacent to $\Gamma$ on $g$ are $gs$ with $s \in S$.

Let $p, q \geq 0$, $p + q = 1$. Suppose we move around an $n$-gon. Then we go one step clockwise with probability $p$ and one step back with probability $q$. Then the probability distribution is $p \delta_1 + q \delta_{-1}$. 

\subsection{Ergodicity}
A random walk on a group $G$ driven by a probability $P$ is said to be ergodic if there exists an integer $N > 0$ such that $P^{\ast N}(g) > 0$ for all $g \in G$, so every event can be reached from $P$. 

\begin{proposition}
	Let $P$ be a probability on a finite group $G$ and suppose $P(1) > 0$ and $Supp(P)$ generates the group $G$. Then the random walk driven by $P$ is ergodic.
\end{proposition}

\begin{proof}
	Let $S = Supp(P)$. As $S$ is a generating set, then there exists an $N$ such that $S^N = G$. So we have that $P^{\ast N} g > 0$ by the definition of the support. Finally, $S^{k} \subseteq S^{k + 1}$ as $1 \in S$. Therefore, $P$ is ergodic. 
\end{proof}
\section{Proof of convergence}
In this section, we shall prove the theorem below.
\begin{theorem}
	If $P$ is an ergodic random walk on a finite group $G$, then $(P^{\ast k})$ converges to the uniform distribution $U$. 
\end{theorem}
We define $L(G)$ to the group algebra on $G$, where $L(G) = \lbrace f : f: G \rightarrow \mathbb{C} \rbrace$. This is an inner product space where $\langle f_1, f_2 \rangle = 1/|G| \sum_{g \in G} f_1(g) \overline{f_2(g)}$. 
We define a convolution operator on a probability distribuion $P$ as $M : L(G) \rightarrow L(G)$ where $M(a) = P \ast a$, where $(P \ast a) (g) =\sum_{h\in G} P(gh^{-1})a(h)$.
\begin{definition}
	The spectrum of a random walk driven by probability distribution $P$ is the set of eigenvalues of $M$. 
\end{definition} 
It is easy to show that $P \ast U = U$ and that $U$ is an eigenvalue of $M$ with eigenvalue $1$. 

\end{document}
