\documentclass[]{article}
\usepackage{amsmath}
\usepackage{amssymb}
\usepackage{bm}
%opening
\title{Representation Theory- Problem sheet 2}
\author{Eric Luu}

\begin{document}

\maketitle

\section*{Chapter 3}
\subsection*{3.1}
Suppose $f$ is a class function. Let $g, h \in G$. Then we have that $f(h (gh) h^{-1}) = f(gh)$, so $f(hg) = f(gh)$. As $g$ and $h$ are arbitrary, then $f(hg) = f(gh) $ for all $g \in G$. If $f(gh) = f(hg)$ for all $g, h \in G$, then for any $g, h \in G$, we have that $f((gh^{-1})h)= f(h (gh^{-1}))$, so $f(g) = f(ghg^{-1})$. Thus $f$ is a class function.
\subsection*{3.2}
\subsubsection*{a}
We have that $\rho_0$ is the map $1 \mapsto [1]$, $a \mapsto [1]$, and $a^2 \mapsto [1]$.
We have that $\rho_1$ is the map $1 \mapsto [1]$, $a \mapsto [e^{2 \pi i/3}]$,$a^2 \mapsto [e^{4 \pi i/3}]$.
We have that $\rho_2$ is the map $1 \mapsto [1]$, $a \mapsto [e^{4 \pi i/3}]$,$a^2 \mapsto [e^{2 \pi i/3}]$.

Therefore, we have that:
We have that $\chi_0$ is the map $1 \mapsto 1$, $a \mapsto 1$, and $a^2 \mapsto 1$.
We have that $\chi_1$ is the map $1 \mapsto 1$, $a \mapsto e^{2 \pi i/3}$,$a^2 \mapsto e^{4 \pi i/3}$.
We have that $\chi_2$ is the map $1 \mapsto 1$, $a \mapsto e^{4 \pi i/3}$,$a^2 \mapsto e^{2 \pi i/3}$.

\subsubsection*{b}
We have that $\chi$ is the map $1 \mapsto 2$, $a \mapsto -1$, $a^2 \mapsto -1$. 

We have that:

\begin{align*}
	 \langle \chi | \chi_0 \rangle &= 1/3 \left( \chi(1)\overline{\chi_0(1)} +   \chi(a)\overline{\chi_0(a)} + \chi(a^2)\overline{\chi_0(a^2)}   \right) \\
	 &= 1/3(2 - 1 - 1) \\
	 &= 0.
\end{align*}
\begin{align*}
	\langle \chi | \chi_1 \rangle &= 1/3 \left( \chi(1)\overline{\chi_1(1)} +   \chi(a)\overline{\chi_1(a)} + \chi(a^2)\overline{\chi_1(a^2)}   \right) \\
	&= 1/3(2 -e^{-2 \pi i/3} -  e^{-4 \pi i/3}) \\
	&= 1.
\end{align*}

\begin{align*}
	\langle \chi | \chi_2 \rangle &= 1/3 \left( \chi(1)\overline{\chi_2(1)} +   \chi(a)\overline{\chi_2(a)} + \chi(a^2)\overline{\chi_2(a^2)}   \right) \\
	&= 1/3(2 -e^{-4 \pi i/3} -  e^{-2 \pi i/3}) \\
	&= 1.
\end{align*}
Therefore, $\chi = \chi_1 \oplus \chi_2$.

\subsection*{3.3}
\subsubsection*{a}
Let $a \in Z(G)$. Then let $v \in V$. If $\rho(a)$ is an $A$-module homomorphism between $\rho$ and $\rho$ and $g \in G$, $v \in V$, then $\rho(a) \rho(g)(v) = \rho(g) \rho(a) (v)$. However, we have that $\rho(a) \rho(g) = \rho(ag) = \rho(ga) = \rho(g) \rho(a)$, as $a$ is a class function. Thus $\rho(a)$ is an element of $Hom_G(\rho, \rho)$.

\subsubsection*{b}
As $\rho(a)$ is an isomorphism as we have that $\rho(a) \rho(g) \rho(a^{-1}) = \rho(g)$ for all $g \in G$, then $\rho(a) = \phi(a) Id_V$ for some $\phi(a) \in C^*$, from Schur's lemma. 
\subsubsection*{c}
We have that $\phi(a) \phi(b) Id_V = \phi(a) Id_V \phi(b) Id_V = \rho(a) \rho(b) = \rho(ab) = \phi(ab) Id_V$. Thus $\phi(a) \phi(b) = \phi(ab)$. Secondly, we have that $\rho(1) = Id$, so $\phi(1) = 1$. Thus $\phi$ is a representation. 

%\subsection*{3.4}
%We have that the map $\rho_S: (\mathbb{Z}_2)^m \rightarrow GL(\mathbb{C^|S|})$ by mapping $(0, 0, ..., 0)$ to $diag(1, 1, 1, 1 ... , 1)$ and $(1, 0, 0, ..., 0) \mapsto diag(-1, 1, 1, ..., 1)$, where diag turns a vector into a diagonal matrix in the usual way. Then we have that this map is a representation as it maps the identity to the identity and $\rho_S

\subsection*{3.5}
Consider the standard representation $\rho_S$ of $Sym(X)$ on $X$, and consider the subrepresentation $\rho |_G$ on $X$. Then $tr(\rho(\sigma)) = |Fix(\sigma)|$. 
We shall show that $\rho_S = \rho_1 \oplus \rho$, where $\chi_\rho(\sigma) = |Fix(\sigma)| - 1$. We have to show that $\langle \rho_S, \rho_1 \rangle > 0$. We have that 
\begin{equation}
	\langle \rho_S, \rho_1 \rangle = \frac{1}{|G|} \sum_{\sigma \in G} \chi_S(\sigma) \overline{\chi_1(\sigma)} = \frac{1}{|G|} \sum_{\sigma \in G} \chi_S(\sigma) \overline{1} = \frac{1}{|G|} \sum_{\sigma \in G} \chi_S(\sigma).
\end{equation}
However, we have that $|Fix(\sigma)| \geq 0$ with $|Fix(1)| = |X|$, therefore we have that $ \langle \rho_S, \rho_1 \rangle > 0$, therefore as the inner product must take on non-negative integer values, $\langle \rho_S, \rho_1 \rangle \geq 1$. Thus $\rho_S = \rho_1 \oplus \rho$ for some representation $\rho$. 
Let $\chi$ be the associated character for $\rho$. 
 We have that $\chi_S = \chi_1 + \chi$, therefore $|Fix(\sigma)| = 1 + \chi(\sigma)$, so $\chi(\sigma) = |Fix(\sigma)| - 1$. Thus shown by construction.
 
Alternate proof: Consider the subspace $span\left(\lbrace 1, 1, 1 \rbrace\right)$. Then we have that this subspace is invariant under $\rho|_G$ and thus $\rho|_G = \rho_1 \oplus \rho$ where $\rho_1$ is the identity map on $span\left(\lbrace 1, 1, 1 \rbrace\right)$. Then we have that $\chi_S = \chi_1 + \chi$. 

\subsection*{3.7}
We have that $\phi: G \rightarrow H$ is a surjective homomorphism, and $\rho: H \rightarrow GL(V)$ is an irreducible representation. Let $\psi =  \rho \circ \phi$. Then let $W$ be a subspace of $V$ where $\psi_g(W) = W$ for all $g \in G$. But we have that $\psi_g(W) = \rho_{\phi(g)}(W)$, and as $\phi$ is surjective, then every element in $h$ has a corresponding $\phi(g)$. Therefore, $\psi_h(W) = W$ for all $h \in H$, thus $W$ must be only the trivial subspace or $V$ by $\rho$ being an irrep. Thus $\psi$ is an irreducible representation of $G$.
\subsection*{3.8}
\subsubsection*{a}
We have that from the first orthogonality relation,
$\sum_{g \in G} \chi(g) \overline{\chi_1(g)} = 0$. But we have that $\chi_1(g) = \overline{\chi_1(g)} = 1$, so we have that $\sum_{g \in G} \chi(g) = 0$. 

\subsubsection*{b}
We have that from the first orthogonality relation,
$\sum_{g \in G} \chi(g) \overline{sgn(g)} = 0$. But we have that $sgn(g) = \overline{sgn(g)}$, so we have that $\sum_{g \in G} \chi(g)sgn(g) = 0$. 
%\subsection*{3.19}
%\subsubsection*{a}
%Consider the eigenvectors of $\phi_g$ and $\overline{\psi_g}$. We have that if $v$ is an eigenvector of $\phi_g$ and $w$ is an eigenvector of $\overline{\psi_g}$, then $\phi_g v w^T \overline{\psi_g} = (\phi_g v) (\overline{\psi_g} w)^T = (\lambda_1 v)(\lambda_2 w)^T = \lambda_1\lambda_2 vw^T$. Thus the eigenvectors of $\chi_\rho$ are exactly the matrices of the form $vw^T$, as there are $mn$ of these eigenvectors and the dimension of $V$ is $mn$. Thus the sums of the eigenvalues of $\chi_\rho$ are the eigenvalues of all combinations of $vw^T$, which is $\chi_\phi \overline{\chi_\psi}$. 
%
%\subsubsection*{b}
%Let $A \in V^G$, so $\psi_g A \psi^*_g = A$ for all $g$. 

\end{document}
