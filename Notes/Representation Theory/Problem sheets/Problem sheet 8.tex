\documentclass[]{article}
\usepackage{amsmath}
\usepackage{amssymb}
\usepackage{bm}
\usepackage{ytableau}
\ytableausetup{centertableaux}

\newcommand{\tr}{\text{tr}}
\newcommand{\fix}{\text{Fix}}
%opening
\title{Representation Theory- Problem sheet 8}
\author{Eric Luu}

\begin{document}

\maketitle

\section*{Chapter 5}
\subsection*{5.2}
\subsubsection*{a}
We have that $G/B$ is infinite by showing the number of cosets are infinite. Consider the matrices:
\begin{equation}
	\left\{ \begin{bmatrix}
		1 & 0\\
		k & 1\\
	\end{bmatrix} : k \in \mathbb{Z}\right\}
\end{equation}

We have that 
\begin{equation}
	\begin{bmatrix}
		1 & 0\\
		k & 1\\
	\end{bmatrix}^{-1}
	=
	\begin{bmatrix}
		1 & 0\\
		-k & 1\\
	\end{bmatrix}
\end{equation}
and

\begin{equation}
	\begin{bmatrix}
		1 & 0\\
		a & 1\\
	\end{bmatrix}
	\begin{bmatrix}
		1 & 0\\
		b & 1\\
	\end{bmatrix}
	=
	\begin{bmatrix}
		1 & 0\\
		a + b & 1\\
	\end{bmatrix}
\end{equation}
therefore we have that for any two matrices $A = \begin{bmatrix}
	1 & 0\\
	a & 1\\
\end{bmatrix}
$
, $B = 
\begin{bmatrix}
	1 & 0\\
	b & 1\\
\end{bmatrix}
$
we have that $B^{-1} A = Id$ iff $b = a$. But this means that for $A$, $B$ distinct, $A$ and $B$ are in different cosets of $
\begin{bmatrix}
	a & b\\
	0 & c\\
\end{bmatrix}
$, thus there are infinitely many cosets. 

\subsubsection*{b}
We have that for any 

\begin{equation}
	A = \begin{bmatrix}
		a & b\\
		c & d
	\end{bmatrix}
\end{equation}
and suppose $c = 0$. Then $a d$ are nonzero, so $A$ is in $B$. Now consider if $c \neq 0$. Divide each entry by $c$, which is allowed as $c$ is nonzero. Then we have that:
\begin{equation}
	A' = \begin{bmatrix}
		a' & b'\\
		1 & d'
	\end{bmatrix}
\end{equation}

\begin{equation}
	A' = 
	\begin{bmatrix}
		1 & a'\\
		0 & 1
	\end{bmatrix}
	\begin{bmatrix}
		0 & 1\\
		1 & 0
	\end{bmatrix}
	\begin{bmatrix}
		1 & d'\\
		0 & b' - a'd'
	\end{bmatrix}
\end{equation}
We have that $c$ is nonzero, so these matrices are in $B$. Then multiplying this out, we get:

\begin{align*}
		\begin{bmatrix}
		1 & a'\\
		0 & 1
	\end{bmatrix}
	\begin{bmatrix}
		0 & 1\\
		1 & 0
	\end{bmatrix}
	\begin{bmatrix}
		1 & d'\\
		0 & b' - a'd'
	\end{bmatrix}
&=
	\begin{bmatrix}
	a' & 1\\
	1 & 0
\end{bmatrix}
\begin{bmatrix}
	1 & d'\\
	0 & b' - a'd'
\end{bmatrix}
&= A'
\end{align*}

Now we can multiply one of these matrices by $c$ to get $A$.
We have these two matrices are in $B$ as $b' - a' d' $ is nonzero as this is the determinant of $A'$, which is in $GL_2(\mathbb{C})$. 

\subsection{5.4}
We will use the pullback of $Z(D_4)$ to $D_4$ through the map 
\begin{equation}
	\rho : S_4 \xrightarrow{\pi} S_4/Z(D_4) \xrightarrow{\phi} GL_1(\mathbb{C})
\end{equation}
where the map $H$ is defined as above.
We have that the pullback of $S_4 /Z(D_4) \cong \mathbb{Z}_2 \times \mathbb{Z}_2$ has cosets
\begin{equation}
	\{1, r^2\}, \{r, r^3\}, \{s, r^2 s\}, {rs, r^3 s}
\end{equation}
with coset representatives the first elements

If we take the pullback of the map:

\begin{align*}
	\phi(1) &= 1\\
	\phi(r) &= 1\\
	\phi(s) &= 1\\
	\phi(rs) &= 1
\end{align*}
then we get the trivial representation.

If we take the pullback of the map:
\begin{align*}
	\phi(1) &= 1\\
	\phi(r) &= -1\\
	\phi(s) &= 1\\
	\phi(rs) &= -1
\end{align*},
the pullback of the map
\begin{align*}
	\phi(1) &= 1\\
	\phi(r) &= 1\\
	\phi(s) &= -1\\
	\phi(rs) &= -1
\end{align*}

and the pullback of
\begin{align*}
	\phi(1) &= 1\\
	\phi(r) &= -1\\
	\phi(s) &= -1\\
	\phi(rs) &= 1
\end{align*}

then we get the table below.
\begin{table}[h!]
	\centering
	\begin{tabular}{|l|l|l|l|l|l|}
		\hline
		$A_4$    & $\{ 1 \} $ & $\{ r^2 \}$ & $ \{ r, r^3 \}$ & $\{ s, r^2 s \}$ & $\{ rs, r^3 s\}$ \\ \hline
		$\chi_1$ & 1          & 1           & 1               & 1                & 1                \\ \hline
		$\chi_2$ & 1          & 1           & -1              & 1                & -1               \\ \hline
		$\chi_3$ & 1          & 1           & 1               & -1               & -1               \\ \hline
		$\chi_4$ & 1          & 1           & -1              & -1               & 1                \\ \hline
		$\chi_5$ & ?          & ?           & ?               & ?                & ?                \\ \hline
	\end{tabular}
\end{table}

To fill in $\chi_5$, we have that the sums of squares add to 8, so it is a degree 2 rep. Then we use column orthogonality to get:

\begin{table}[h!]
	\centering
	\begin{tabular}{|l|l|l|l|l|l|}
		\hline
		$A_4$    & $\{ 1 \} $ & $\{ r^2 \}$ & $ \{ r, r^3 \}$ & $\{ s, r^2 s \}$ & $\{ rs, r^3 s\}$ \\ \hline
		$\chi_1$ & 1          & 1           & 1               & 1                & 1                \\ \hline
		$\chi_2$ & 1          & 1           & -1              & 1                & -1               \\ \hline
		$\chi_3$ & 1          & 1           & 1               & -1               & -1               \\ \hline
		$\chi_4$ & 1          & 1           & -1              & -1               & 1                \\ \hline
		$\chi_5$ & 2          & -2           & 0               & 0                & 0                \\ \hline
	\end{tabular}
\end{table}

\subsection*{5.7} (standing in for 5.8)

\subsubsection*{a}
We have that $\rho^g(x) = \rho(g)^{-1} \rho(x) \rho(g)$. Treating $T := \rho(g)$ as an isomorphism $\mathbb{C}^n \rightarrow \mathbb{C}^n$, then we have that $\rho^g(x) = T^{-1} \rho(x) T$, thus $\rho^g(x)$ and $\rho(x)$ are equivalent. 

\subsection*{b}
Let us show that if $\rho$ is reducible, then $\rho^g$ is reducible. Let $U$ be a subspace in $\mathbb{C}^n$ such that $\rho(n) U = U$ for all $n \in \mathbb{N}$. Then we have that for $n \in N$, $g^{-1} n g \in \mathbb{N}$. Therefore, we have that $\rho^g(n) U = U$. Therefore, $U$ is also an invariant subspace of $\rho^g$ and $\rho^g$ is reducible. Then this implies that if $\rho^g$ is irreducible, then $\rho$ is irreducible. However, $\left(\rho^g\right)^{g^{-1}} = \rho$, thus if $\rho$ is irreducible, then $\rho^g$ is irreducible. Thus shown. 
\subsection*{c}
Let $G = S_3$, $N = A3$, $g = (12)$ and $\rho$ be the degree 1 representation that takes $\rho(123) = \exp(\frac{2 i \pi}{3})$ . Then we have that $\rho^g((123)) = \rho((132))$, but we have that degree 1 representations are equivalent iff the values are the same for each element. However, $\rho((132)) = \exp(\frac{4 i \pi}{3})$ so $\rho$ and $\rho^g$ are inequivalent. 

\subsection*{5.11}
Suppose $\rho: G \rightarrow GL(V)$ is an irrep. Then $Res^G_H(\rho)$ is a representation on $H$ of the same degree. Let $\psi$ be an irreducible subrepresentation of $Res^G_H(\rho)$, and $\psi: H \rightarrow GL(W)$. Then the dimension of $W$ is 1. Then let $V'$ be the vector subspace of $V$ generated by the images of $\psi$, so denote $\rho(g) \dot W = gW$, $g \in G$. We have that $V' = V$ as $\rho$ is an irrep. Then we have that $stW = s (tW) = sW$, meaning that the number of distinct $sW$ is $[G : H]$, which means that $V \leq [G : H]$. Thus the degree of $\rho$ is at most the index of $H$. 

As the dihedral group has an abelian subgroup $\langle r \rangle$ of index 2, every irrep has degree at most 2.

\subsection*{5.12}
To show $\rho$ is an irrep, we have to show that from Mackey's irreducibility criterion for normal groups that:
\begin{itemize}
	\item $\rho$ is irreducible
	\item $\rho$ and $\rho^s$ are inequivalent for $s \notin H$ for a given set of coset representatives.
\end{itemize}
If $\rho$ is of degree 1, then $\rho$ and $\rho^s$ are equivalent if and only if $\rho = \rho^s$. This is because if $\rho = T^{-1} \rho^s T$, then as we are working in $\mathbb{C}^*$, then we can use the fact that $\mathbb{C}$ is abelian to say that $\rho = \rho^s T T^{-1} = \rho^s$.   
Let $\rho : H \rightarrow \mathbb{C}^*$ be as described. Then $\rho$ is a degree 1 representation, thus $\rho$ is irreducible. Then let $S = {s_1, s_2, ..., s_n}$ be the set of coset representatives, but excluding $H$ as a coset.

Let us look at $\rho$ and $\rho^{s_i}$. Suppose $\rho^{s_i} = \rho$ for all $x \in H$. Then this implies that $\rho(s_i^{-1} x s_i) = \rho(x)$ for all $x$. But as we are sending each element in $H$ to a distinct complex number, then $s_i^{-1} x s_i = x$. Then $x s_i = s_i x$ for all $x \in H$, but this means that $s_i$ is in $H$. But we said that $s_i$ is not in $H$. Therefore contradiction, thus $\rho^{s_i}$ and $\rho$ are inequivalent. for all $s_i$. Therefore, $\rho$ is irreducible. 

\section*{6}
\subsection*{6.4}
We shall show $\sim$ is an equivalence relation. We have that any Young tableau is similar to itself, by the identity permutation. We have that if $\sigma$ maps Young tableau $X$ to $Y$, then $\sigma^{-1}$ maps $Y$ to $X$, so if $X \sim Y$, then $Y \sim X$. Finally, if $X \sim Y$ through $\sigma$ and $Y \sim Z$ through $\pi$, then $\pi \circ \sigma$ maps $X$ to $Z$. 


\end{document}
