\documentclass[]{article}
\usepackage{amsmath}
\usepackage{amssymb}
\usepackage{bm}

\newcommand{\tr}{\text{tr}}
\newcommand{\fix}{\text{Fix}}
%opening
\title{Representation Theory- Problem sheet 8}
\author{Eric Luu}

\begin{document}

\maketitle

\section*{Chapter 5}
\subsection*{5.7} (standing in for 5.8)

\subsubsection*{a}
We have that $\rho^g(x) = \rho(g)^{-1} \rho(x) \rho(g)$. Treating $T := \rho(g)$ as an isomorphism $\mathbb{C}^n \rightarrow \mathbb{C}^n$, then we have that $\rho^g(x) = T^{-1} \rho(x) T$, thus $\rho^g(x)$ and $\rho(x)$ are equivalent. 

\subsection*{b}
Let us show that if $\rho$ is reducible, then $\rho^g$ is reducible. Let $U$ be a subspace in $\mathbb{C}^n$ such that $\rho(n) U = U$ for all $n \in \mathbb{N}$. Then we have that for $n \in N$, $g^{-1} n g \in \mathbb{N}$. Therefore, we have that $\rho^g(n) U = U$. Therefore, $U$ is also an invariant subspace of $\rho^g$ and $\rho^g$ is reducible. Then this implies that if $\rho^g$ is irreducible, then $\rho$ is irreducible. However, $\left(\rho^g\right)^{g^{-1}} = \rho$, thus if $\rho$ is irreducible, then $\rho^g$ is irreducible. Thus shown. 
\subsection*{c}
Let $G = S_4$, $N = A4$, $g = (12)$ and $\rho$ be the standard representation. Then we have that $\rho^g((123)) = \rho((132))$, but $\rho^g((12)(34)) = \rho((12)(34))$. DO LATER

\subsection*{5.11}
Suppose $\rho: G \rightarrow GL(V)$ is an irrep. Then $Res^G_H(\rho)$ is a representation on $H$ of the same degree. Let $\psi$ be an irreducible subrepresentation of $Res^G_H(\rho)$, and $\psi: H \rightarrow GL(W)$. Then the dimension of $W$ is 1. Then let $V'$ be the vector subspace of $V$ generated by the images of $\psi$, so denote $\rho(g) \dot W = gW$, $g \in G$. We have that $V' = V$ as $\rho$ is an irrep. Then we have that $stW = s (tW) = sW$, meaning that the number of distinct $sW$ is $[G : H]$, which means that $V \leq [G : H]$. Thus the degree of $\rho$ is at most the index of $H$. 

As the dihedral group has an abelian subgroup $\langle r \rangle$ of index 2, every irrep has degree at most 2.

\subsection*{5.12}
To show $\rho$ is an irrep, we have to show that from Mackey's irreducibility criterion for normal groups that:
\begin{itemize}
	\item $\rho$ is irreducible
	\item $\rho$ and $\rho^s$ are inequivalent for $s \notin H$ for a given set of coset representatives.
\end{itemize}
If $\rho$ is of degree 1, then $\rho$ and $\rho^s$ are equivalent if and only if $\rho = \rho^s$. This is because if $\rho = T^{-1} \rho^s T$, then as we are working in $\mathbb{C}^*$, then we can use the fact that $\mathbb{C}$ is abelian to say that $\rho = \rho^s T T^{-1} = \rho^s$.   
Let $\rho : H \rightarrow \mathbb{C}^*$ be as described. Then $\rho$ is a degree 1 representation, thus $\rho$ is irreducible. Then let $S = {s_1, s_2, ..., s_n}$ be the set of coset representatives, but excluding $H$ as a coset.

Let us look at $\rho$ and $\rho^{s_i}$. Suppose $\rho^{s_i} = \rho$ for all $x \in H$. Then this implies that $\rho(s_i^{-1} x s_i) = \rho(x)$ for all $x$. But as we are sending each element in $H$ to a distinct complex number, then $s_i^{-1} x s_i = x$. Then $x s_i = s_i x$ for all $x \in H$, but this means that $s_i$ is in $H$. But we said that $s_i$ is not in $H$. Therefore contradiction, thus $\rho^{s_i}$ and $\rho$ are inequivalent. for all $s_i$. Therefore, $\rho$ is irreducible. 
\end{document}
