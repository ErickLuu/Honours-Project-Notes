\documentclass[]{article}
\usepackage{amsmath}
\usepackage{amssymb}
\usepackage{bm}

\newcommand{\tr}{\text{tr}}
\newcommand{\fix}{\text{Fix}}
%opening
\title{Representation Theory- Problem sheet 2}
\author{Eric Luu}

\begin{document}

\maketitle

\section*{Chapter 3}
\subsection*{3.4 (Standing in for 3.5)}
\subsubsection*{a}
Let $\rho_S : (\mathbb{Z}_2)^m \rightarrow \mathbb{C}^*$, $\rho_S(v) = \chi_S(v)$. We shall show $\rho_S$ is a representation. We have that $\chi_S(0, 0, ..., 0) = (-1)^{|\emptyset|} = 1$, therefore identities are preserved. Secondly, we have that $\chi_S(u)\chi_S(v) = (-1)^{|\alpha(u) \cap S|}(-1)^{|\alpha(v) \cap S|} = (-1)^{|(\alpha(u) \triangle \alpha(v)) \cap S|}$. However, we have that $\alpha(u + v) = \alpha(u) \triangle \alpha(v)$, therefore, $\chi_S(u)\chi_S(v) = \chi_S(u + v)$. Therefore, $\rho_S$ is a representation and thus $\chi_S$ is a character.
\subsection*{b}
As $(\mathbb{Z}_2)^m$ has $2^m$ elements, and $(\mathbb{Z}_2)^m$ is abelian, then every element in $(\mathbb{Z}_2)^m$ is in its own conjugacy class and thus the number of irreducible characters is $2^m$. However, we have shown that $\chi_S$ is a character, and is irreducible as it is of degree 1. The number of sets $S$ is $2^m$, therefore all irreducible characters are equal to some $\chi_S$ as the number of irreducible characters is at most the number of conjugacy classes.
\subsection*{c}
We have that $\chi_S(v) \chi_T(v) = (-1)^{|\alpha(v) \cap S|}(-1)^{|\alpha(v) \cap T|}$. However, if we have that $x \in \alpha(S \cap T)$, then we count $x$ twice or zero times in $(-1)^{|\alpha(v) \cap S|}(-1)^{|\alpha(v) \cap T|}$, therefore it has no effect as $(-1)^2 = 1$. Therefore, the only values in $S$ and $T$ that do affect the final product are those in the symmetric difference, therefore, $\chi_S\chi_T(v) = (-1)^{|\alpha(v) \cap (S \triangle T)|} = \chi_{S \triangle T}(v)$. 

\subsection*{3.9}
\subsection*{a}
We have that $ \tr(\chi_\rho(\sigma)) = \sum_{k = 1}^n \delta_{\sigma(k)}^k$, where $\delta^i_j = 0$ when $i \neq j$ and $1$ when $i = j$. Thus $\tr(\chi_\rho(\sigma))$ is the number of fixed points of $\rho$. 
\subsection*{b}
We have that the character table of $S_3$ is:
\begin{table}[h!]
	\centering
	\begin{tabular}{|l|l|l|l|}
		\hline
		$A_4$    & $(1)$ & $(12)$ & $(123)$ \\ \hline
		$\chi_1$ & 1     & 1      & 1      \\ \hline
		$\chi_2$ & 1     & -1     & 1      \\ \hline
		$\chi_3$ & 2     & 0      & -1     \\ \hline
	\end{tabular}
\end{table}
therefore, the character $\chi_\rho$ where $\rho$ is the natural representation can be written as $\chi_1 + \chi_3$ as we have that $\chi_\rho((1)) = 3$, $\chi_\rho((12)) = 1$, $\chi_\rho((123)) = 0$. Therefore,
\begin{align*}
	\langle \chi_\rho, \chi_\rho \rangle &= \langle \chi_1 + \chi_3, \chi_1 + \chi_3 \rangle\\
	&= \langle \chi_1, \chi_1 \rangle + \langle \chi_1, \chi_3 \rangle + \langle \chi_3, \chi_1 \rangle + \langle \chi_3, \chi_3 \rangle\\
	&= 1 + 0 + 0  + 1 = 2. 
\end{align*}
Thus shown.
\subsection*{c}
We have that \begin{equation}
	\langle \chi_\rho, \chi_\rho \rangle = \frac{1}{n!}\sum_{\sigma \in S_n} |\fix(\sigma)|^2.
\end{equation}
Let $X_i$ be the indicator random variable of $\sigma$ fixing $i$ in $[n]$. Then we have that:
\begin{equation}
	\frac{1}{n!}\sum_{\sigma \in S_n} |\fix(\sigma)|^2 = \mathbb{E}\left[\left(\sum_{i=1}^{n} X_i\right)^2\right]
\end{equation}
from looking at the average number of squared fixed elements. Now we have that:
\begin{align*}
	\mathbb{E}\left[\left(\sum_{i=1}^{n} X_i\right)^2\right] &= \left(\sum_{i=1}^{n}\mathbb{E}\left[X_i^2 \right] \right) + 2\left(\sum_{\substack{
		i < j\\ j = 2
	}}^{n}\mathbb{E}\left[X_i X_j \right] \right)\\
	&= \left(\sum_{i=1}^{n}\frac{1}{n} \right) + 2 \left(\sum_{\substack{
		i < j\\ j = 2
	}}^{n}\frac{1}{n(n-1)}\right)\\
	&= 1 + \frac{2 \binom{n}{2}}{n(n-1)}\\
	&= 1 + \frac{2(n)(n-1)}{2(n)(n-1)}\\
	&= 2
\end{align*}
We have that $\mathbb{E}\left[X_i X_j \right] = \frac{1}{n(n-1)}$ because the probability that $i$ is fixed is $\frac{1}{n}$ and the probability $j$ is fixed, given $i$ is fixed is $\frac{1}{(n-1)}$, thus by total expectations $\mathbb{E}\left[X_i X_j \right] = \frac{1}{n(n-1)}$.
Thus, $\langle \chi_\rho, \chi_\rho \rangle = 2$ for all $n$.
This has the consequence that the representation $\chi(\sigma) = |\fix(\sigma)| - 1$ is irreducible. 
\subsection*{3.11}
\subsubsection*{a}
We have that if all irreducible representations (up to equivalence) $\lbrace \rho_1, ..., \rho_s \rbrace$ are of degree 1, then we have that, letting $\lbrace d_1, ..., d_s \rbrace$ being the associated degrees of $\rho$, $|G| = d_1^2 + d_2^2 + ... + d_s^2 = \underbrace{1^2 + 1^2 + ... +1^2}_{s\text{ times}} = s$, therefore there are $|G|$ irreducible representations. However, the number of irreducible representations equals the number of conjugacy classes, thus every element of $g \in G$ is in its own conjugacy class. Therefore for all $g, h \in G$, $hgh^{-1} = g$. Thus $gh = hg$ for all $g, h \in G$, so $G$ is abelian. 
\subsubsection*{b}
We have that for all irreducible representations (up to equivalence) $\lbrace \rho_1, ..., \rho_s$ have associated degrees $\lbrace d_1, ..., d_s \rbrace$ where $d_i | p^2$ for all $i \in [1, s]$. By the dimension theorem, $d_i = \lbrace 1, p , p^2 \rbrace$. However, we have that
\begin{equation}
	|G| = p^2 = d_1^2 + d_2^2 + ... + d_s^2 
\end{equation}
therefore, $d_i$ cannot be $p^2$. If we let $\rho$ be the trivial representation, then $d_1 = 1$. Therefore, we cannot have $d_2, ..., d_s = p$ as that would imply $p^2 \geq p^2 + 1$, which is obviously silly. Therefore, $d_1 = d_2 = ... = d_s = 1$, and therefore, there are $p^2$ irreducible representations with different irreducible characters. Therefore, we have that there are $p^2$ conjugacy classes in $G$ and by the argument above, $G$ is abelian.

\subsection*{3.15}
We have that $\lambda: G \rightarrow GL(\mathbb{C} G)$ is the regular representation. Suppose $g \in G, g \neq 1$. Then we have that on the basis $\lbrace h \in \mathbb{C}(G) : h \in G\rbrace$, we have that $\lambda_g$ does not map $h$ to itself, therefore $\lambda_g \neq Id_{\mathbb{C}G}$. We also have that $\lambda_g = \rho_{1, g} \oplus \rho_{2, g} \oplus ... \oplus \rho_{s, g}$ by Matzche's theorem. As $\lambda_g \neq Id$, then there is a $\rho_{i, g}$ which is not the identity, as if all $\rho$ was the identity map, then $\lambda_g$ would be the identity map. Therefore, $\rho_{i, g} \neq Id_V$ and is irreducible. Thus shown. 
\subsection*{3.16}
\subsubsection*{a}
We have that:

\begin{align*}
	\lbrace \chi_3 , \chi_1 \rbrace &= \frac{1}{6}( 1 \times 1 \times 2 + 2 \times x \times 1 + 3 \times y \times 1) = 0\\
	\lbrace \chi_3 , \chi_2 \rbrace &= \frac{1}{6}( 1 \times 1 \times 2 + 2 \times x \times 1 + 3 \times y \times -1) = 0\\
	\lbrace \chi_3 , \chi_3 \rbrace &= \frac{1}{6}( 1 \times 2 \times 2 + 2 \times x \times \overline{x} + 3 \times y \times \overline{y}) = 1.
\end{align*}
Therefore, we have that:
\begin{align*}
	2x + 3y &= -2\\
	2x - 3y &= -2
\end{align*}
so $x = -1$ and $y = 0$. We use the final orthogonality relation to verify that this is a legitimate relation. Plugging in $-1$, we have that $\frac{1}{6}( 1 \times 2 \times 2 + 2 \times (-1) \times (-1) + 3 \times 0 \times 0) = 1$, as expected.
\subsubsection*{b}
We have that from the column orthogonality that:
$\lbrace g_1, g_2 \rbrace = 1 \times 1 + 1 \times 1 + 2 \times x = 0$, so $x = -1$. Secondly, we have that $\lbrace g_1, g_3, \rbrace = 1 \times 1 + 1 \times -1 + 2 \times y = 0$, so $y = 0$. Therefore, they both agree. 
\newpage
\subsection*{3.18}
\subsection*{a}
We have that there are four irreducible representations, where $\chi_1$ is the trivial group. As we have that $|A_4| = 12$, the squared sums of the remaining degrees must be $11$, thus the only tuple of degrees that work are $d_1 = 1, d_2 = 1, d_3 = 1, d_4 = 3$. 
\subsection*{b}
We have that $\chi_4 = |Fix(\sigma)| - 1$ which is a character from 3.5. To show $\chi_4$ is irreducible, we have that $\langle \chi_4, \chi_4 \rangle = \frac{1}{12} \left(1 \cdot 3 \cdot 3 + 4 \cdot 0 \cdot 0 + 4 \cdot 0 \cdot 0 + 3 \cdot -1 \cdot -1\right) = \frac{1}{12} \left(9 + 3\right) = 1$, thus $\chi_4$ is irreducible. 

\begin{table}[h!]
\centering
\begin{tabular}{|l|l|l|l|l|}
	\hline
	$A_4$    & $(1)$ & $(123)$ & $(132)$ & $(12)(34)$ \\ \hline
	$\chi_1$ & 1     & 1       & 1       & 1          \\ \hline
	$\chi_2$ & 1     & a       & b       & e          \\ \hline
	$\chi_3$ & 1     & c       & d       & f          \\ \hline
	$\chi_4$ & 3     & 0       & 0       & -1         \\ \hline
\end{tabular}
\end{table}
This is the character table.
\subsubsection*{d}
 However, we have that 
\begin{align*}
	\langle \chi_2, \chi_4 \rangle &= \frac{1}{12}\left(1 \cdot 1 \cdot 3 + 1 + 4 \cdot a \cdot 0 + 4 \cdot b \cdot 0 + 3 \cdot e \cdot -1\right) = \frac{1}{12} (3 - 3e) &= 0\\
	\langle \chi_3, \chi_4 \rangle &= \frac{1}{12}\left(1 \cdot 1 \cdot 3 + 1 + 4 \cdot a \cdot 0 + 4 \cdot b \cdot 0 + 3 \cdot f \cdot -1\right) = \frac{1}{12} (3 - 3f) &= 0\\
\end{align*}
therefore $e = f = 1$. This makes the character table like


\begin{table}[h!]
	\centering
	\begin{tabular}{|l|l|l|l|l|}
		\hline
		$A_4$    & $(1)$ & $(123)$ & $(132)$ & $(12)(34)$ \\ \hline
		$\chi_1$ & 1     & 1       & 1       & 1          \\ \hline
		$\chi_2$ & 1     & a       & b       & 1          \\ \hline
		$\chi_3$ & 1     & c       & d       & 1          \\ \hline
		$\chi_4$ & 3     & 0       & 0       & -1         \\ \hline
	\end{tabular}
\end{table}

\subsubsection*{e}
We have that $\chi_2$ is itself a representation as it has degree 1. Therefore, we have that $[\chi_2(123)]^3 = [\chi_2(132)]^3 = 1$. Similarly for $\chi_3$, we have that $[\chi_3(123)]^3 = [\chi_3(132)]^3 = 1$. We also have that by the second orthogonality relation,  $1 + a^2 + c^2 = 12/4 = 3$ but $1 + a + c = 0$, and $1 + b + d = 0$ and $1 + b^2 + d^2 = 3$. Therefore, we have that if we let $\omega = e^{2 i \pi/3}$, then the final character table 
\begin{table}[h!]
	\centering
	\begin{tabular}{|l|l|l|l|l|}
		\hline
			$A_4$    & $(1)$ & $(123)$    & $(132)$    & $(12)(34)$ \\ \hline
			$\chi_1$ & 1     & 1          & 1          & 1          \\ \hline
			$\chi_2$ & 1     & $\omega$   & $\omega^2$ & 1          \\ \hline
			$\chi_3$ & 1     & $\omega^2$ & $\omega$   & 1          \\ \hline
			$\chi_4$ & 3     & 0          & 0          & -1         \\ \hline
	\end{tabular}
\end{table}
\subsection*{f}
We have that $\chi_{\rho_4 \otimes \rho_4}(g) = \chi_{\rho_4}(g) \times \chi_{\rho_4}(g)$, from 3.6. Therefore, we have that $\chi_{\rho_4 \otimes \rho_4}((1)) = 9$, $\chi_{\rho_4 \otimes \rho_4}((123)) = 0$, $\chi_{\rho_4 \otimes \rho_4}((132)) = 0$, and $\chi_{\rho_4 \otimes \rho_4}((12)(34)) = 1$. Therefore, we have that:
\begin{align*}
	\langle \chi_{\rho_4 \otimes \rho_4}, \chi_1 \rangle &= \frac{1}{12} ( 1 \cdot 9 \cdot 1 + 4 \cdot 0 \cdot 1 + 4 \cdot 0 \cdot 1 + 3 \cdot 1 \cdot 1) = \frac{1}{12}(12) = 1\\
	\langle \chi_{\rho_4 \otimes \rho_4}, \chi_2 \rangle &= \frac{1}{12} ( 1 \cdot 9 \cdot 1 + 4 \cdot 0 \cdot \omega + 4 \cdot 0 \cdot \omega^2 + 3 \cdot 1 \cdot 1) = \frac{1}{12}(12) = 1\\
	\langle \chi_{\rho_4 \otimes \rho_4}, \chi_3 \rangle &= \frac{1}{12} ( 1 \cdot 9 \cdot 1 + 4 \cdot 0 \cdot \omega^2 + 4 \cdot 0 \cdot \omega + 3 \cdot 1 \cdot 1) = \frac{1}{12}(12) = 1\\
	\langle \chi_{\rho_4 \otimes \rho_4}, \chi_2 \rangle &= \frac{1}{12} ( 1 \cdot 9 \cdot 3 + 4 \cdot 0 \cdot 0 + 4 \cdot 0 \cdot 0 + 3 \cdot 1 \cdot -1) = \frac{1}{12}(24) = 2\\
\end{align*}
Therefore, $\chi_{\rho_4 \otimes \rho_4} = \chi_1 + \chi_2 + \chi_3 + 2 \chi_4$, so
\begin{equation}
	\rho_4 \otimes \rho_4 = \rho_1 \oplus \rho_2 \oplus \rho_3 \oplus \rho_4 \oplus \rho_4.
\end{equation}
\end{document}
