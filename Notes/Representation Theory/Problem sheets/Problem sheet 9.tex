\documentclass[]{article}
\usepackage{amsmath}
\usepackage{amssymb}
\usepackage{bm}
\usepackage{ytableau}
\usepackage{float}
\ytableausetup{centertableaux}
\allowdisplaybreaks

\newcommand{\tr}{\text{tr}}
\newcommand{\fix}{\text{Fix}}
%opening
\title{Representation Theory- Problem sheet 9}
\author{Eric Luu}

\begin{document}

\maketitle

\section*{Chapter 6}
\subsection*{Question 1}
\subsubsection*{a}
Recall that the 3 $\lambda$-tableaux of $S_3$ are:
\ydiagram{3}, \ydiagram{2, 1}, \ydiagram{1, 1, 1}.

Consider $\lambda = (3)$. For any tableau $t$, we have that $e_t = \left[\begin{ytableau}1 & 2 & 3\end{ytableau}\right]$, the unique $\ydiagram{3}$ tabloid. Therefore, $V^{(3)} = span \left\{\left[\begin{ytableau}1 & 2 & 3\end{ytableau}\right]\right\}$. For any $\sigma \in S_3$, we have that $\sigma e_t = e_{\sigma t} = \left[\begin{ytableau}1 & 2 & 3\end{ytableau}\right]$. So $\psi^{(3)} : S_3 \rightarrow GL\left(V^{(3)}\right)$ is the trivial rep. 

Consider $\lambda = (2,1)$. The $(2,1)$-tabloids are:

\begin{align*}
	t_1 &= \begin{ytableau}
		\cdot & \cdot \\
		1
	\end{ytableau}
	\\
	t_2 &= \begin{ytableau}
		\cdot & \cdot \\
		2
	\end{ytableau}
	\\
	t_3 &= \begin{ytableau}
		\cdot & \cdot \\
		3
	\end{ytableau}
\end{align*}
We have that for any $(2,1)$-tableau $t$, we send it to $t_a - t_b$ for some $a, b \in [1..3]$. The vector space that this spans is $V^{(2,1)} = span \left\{t_1 - t_2, t_2 - t_3\right\}$. Let $v_1 = t_1 - t_2$, $v_2 = t_2 - t_3$. 
We have that $\psi^{(2, 1)} : S_3 \rightarrow GL(V^{(2,1)})$ has the following property:
\begin{align*}
	\psi^{(2, 1)}_{(1)}(v_1) &= v_1\\
	\psi^{(2, 1)}_{(1)}(v_2) &= v_2\\
	\\
	\psi^{(2, 1)}_{(1\, 2)}(v_1) &= - v_1\\
	\psi^{(2, 1)}_{(1\, 2)}(v_2) &= v_1 + v_2\\
	\\
	\psi^{(2, 1)}_{(1\, 2 \, 3)}(v_1) &= - v_2\\
	\psi^{(2, 1)}_{(1\, 2 \, 3)}(v_2) &= - v_1 - v_2\\
\end{align*}
This is the standard representation. 

Now consider $\lambda = (1,1,1)$. We have that \begin{equation}
	e_t = \sum_{\pi \in C_t} sign(\pi) \pi[t] = \sum_{\pi \in S_3} sign(\pi) \pi(t).
\end{equation} Then:
\begin{equation}
	\psi^{(1,1,1)}_{\sigma} e_t = sgn(\sigma) e_t
\end{equation}
Therefore, $\psi^{(1,1,1)}_{\sigma}$ is the sign rep. 

The character table is:
\begin{table}[h!]
	\centering
	\begin{tabular}{|l|l|l|l|}
		\hline
		Character table & $(1)$ & $(12)$ & $(123)$ \\ \hline
		$\chi_1$        & 1     & 1      & 1       \\ \hline
		$\chi_2$        & 2     & 0      & -1      \\ \hline
		$\chi_3$        & 1     & -1     & 1       \\ \hline
	\end{tabular}
\end{table}

\subsubsection*{b}
We are working with $S_4$.
We have the trivial representation corresponding to $\ydiagram{4}$, the standard representation corresponding with $\ydiagram{3, 1}$, and the sign representation corresponding with $\ydiagram{1, 1, 1, 1}$. The procedure is precisely the same as in $S_3$. 

The two representations that have not been discussed yet have representations corresponding to $\ydiagram{2, 2}$ and $\ydiagram{2, 1, 1}$. 

For $\ydiagram{2, 2}$, the $(2, 2)$-tabloids are of the form:

\begin{align*}
	t_{12} &= \begin{ytableau}
	1 & 2 \\
	\cdot & \cdot \\
	\end{ytableau}\\
	t_{13} &= \begin{ytableau}
	1 & 3 \\
	\cdot & \cdot \\
	\end{ytableau}\\
	t_{14} &= \begin{ytableau}
	1 & 4 \\
	\cdot & \cdot \\
\end{ytableau}\\
	t_{23} &= \begin{ytableau}
	2 & 3 \\
	\cdot & \cdot \\
\end{ytableau}\\
	t_{24} &= \begin{ytableau}
	2 & 4 \\
	\cdot & \cdot \\
\end{ytableau}\\
	t_{34} &= \begin{ytableau}
	3 & 4 \\
	\cdot & \cdot \\
\end{ytableau}\\
\end{align*}
and every tableau has basis polytabloids $e_1 = t_{12} + t_{34} - t_{23} - t_{14}$ and $e_2 = t_{13} + t_{24} - t_{23} - t_{14}$. 
We have that $V^{(2,2)} = span\left\{{e_1, e_2}\right\}$. Now let us calculate the characters of $\psi^{(2,2)}: S_4 \rightarrow GL(V^{(2,2)})$.

We have that:
\begin{align*}
	\psi^{(2,2)}_{(1)}(e_1) &= e_1\\
	\psi^{(2,2)}_{(1)}(e_2) &= e_2\\
	\\
	\psi^{(2,2)}_{(12)}(e_1) &= t_{12} + t_{34} - t{13} - t{24} = e_1 - e_2\\
	\psi^{(2,2)}_{(12)}(e_2) &=t_{23} + t_{14} - t_{13} - t_{24}= - e_2\\
	\\
	\psi^{(2,2)}_{(123)}(e_1) &=t_{23} + t_{14} - t_{13} - t_{24}= - e_2\\
	\psi^{(2,2)}_{(123)}(e_2) &= t_{12} + t_{34} - t_{13} - t{24} = e_1 - e_2
	\\
	\psi^{(2,2)}_{(1234)}(e_1) &= t_{23} + t_{14} - t_{34} - t_{12} = - e_1\\
	\psi^{(2,2)}_{(1234)}(e_2) &= t_{24} + t_{13} - t_{34} - t_{12} = e_2 - e_1\\
	\\
	\psi^{(2,2)}_{(12)(34)}(e_1) &= t_{12} + t_{34} - t{14} - t{23} = e_1\\
	\psi^{(2,2)}_{(12)(34)}(e_2) &= t_{24} + t_{13} - t_{14} - t_{23}= e_2
\end{align*}


For $\ydiagram{2, 1, 1}$, we have the following tabloids:
\begin{align*}
	t_{12} &= \begin{ytableau}
	\cdot & \cdot\\
	1\\
	2\\
	\end{ytableau}\\
	t_{13} &= \begin{ytableau}
	\cdot & \cdot\\
	1\\
	3\\
\end{ytableau}\\
	t_{14} &= \begin{ytableau}
	\cdot & \cdot\\
	1\\
	4\\
\end{ytableau}\\
	t_{21} &= \begin{ytableau}
	\cdot & \cdot\\
	2\\
	1\\
\end{ytableau}\\
	t_{23} &= \begin{ytableau}
	\cdot & \cdot\\
	2\\
	3\\
\end{ytableau}\\
	t_{24} &= \begin{ytableau}
	\cdot & \cdot\\
	2\\
	4\\
\end{ytableau}\\
	t_{31} &= \begin{ytableau}
	\cdot & \cdot\\
	3\\
	1\\
\end{ytableau}\\
	t_{32} &= \begin{ytableau}
	\cdot & \cdot\\
	3\\
	2\\
\end{ytableau}\\
	t_{34} &= \begin{ytableau}
	\cdot & \cdot\\
	3\\
	4\\
\end{ytableau}\\
	t_{41} &= \begin{ytableau}
	\cdot & \cdot\\
	4\\
	1\\
\end{ytableau}\\
	t_{42} &= \begin{ytableau}
	\cdot & \cdot\\
	4\\
	2\\
\end{ytableau}\\
	t_{43} &= \begin{ytableau}
	\cdot & \cdot\\
	4\\
	3\\
\end{ytableau}
\end{align*}
We know that $\psi^{(2,1,1)}$ must be a degree 3 representation.
Consider $e_t$ where $t = 
\begin{ytableau}
3 & 4\\
1\\
2\\
\end{ytableau} 
$
. We have that $e_t = t_{12} + t_{31} + t_{23} - t_{21} - t_{13} - t_{32} = e_1$. 
Now consider
$t = 
\begin{ytableau}
	4 & 3\\
	1\\
	2\\
\end{ytableau} 
$. Then we have that
$e_t = t_{12} + t_{41} + t_{24} - t_{21} - t_{14} - t_{42} = e_2$. 

Finally consider
$t = 
\begin{ytableau}
	4 & 1\\
	2\\
	3\\
\end{ytableau} 
$. 
Then we have that:
$e_t = t_{23} + t_{42} + t_{34} - t_{32} - t_{24} - t_{43} = e_3$.

Therefore, we have that:
\begin{align*}
	e_1 &= t_{12} + t_{31} + t_{23} - t_{21} - t_{13} - t_{32}\\
	e_2 &= t_{12} + t_{41} + t_{24} - t_{21} - t_{14} - t_{42} \\
	e_3 &= t_{23} + t_{42} + t_{34} - t_{32} - t_{24} - t_{43}\\
\end{align*}
and $V^{(2,1,1)} = span \left\{e_1, e_2, e_3\right\}$. 
Then we have that:
\begin{align*}
	\psi^{(2,1,1)}_{(1)} e_1 &= e_1\\
	\psi^{(2,1,1)}_{(1)} e_2 &= e_2\\
	\psi^{(2,1,1)}_{(1)} e_3 &= e_3\\
	\\
	\psi^{(2,1,1)}_{(12)} e_1 &=  t_{21} + t_{32} + t_{13} - t_{12} - t_{23} - t_{31} = - e_1\\
	\psi^{(2,1,1)}_{(12)} e_2 &=  t_{21} + t_{42} + t_{14} - t_{12} - t_{24} - t_{41} = - e_2\\
	\psi^{(2,1,1)}_{(12)} e_3 &=  t_{13} + t_{41} + t_{34} - t_{31} - t_{14} - t_{43} = -e_1 + e_2 + e_3\\
	\\
	\psi^{(2,1,1)}_{(123)} e_1 &= t_{32} + t_{13} + t_{21} - t_{12} - t_{23} - t_{31} = e_1\\
	\psi^{(2,1,1)}_{(123)} e_2 &= t_{23} + t_{42} + t_{34} - t_{32} - t_{42} - t_{34} = e_3\\
	\psi^{(2,1,1)}_{(123)} e_3 &= t_{31} + t_{43} + t_{14} - t_{13} - t_{34} - t_{41} = e_1 - e_2 - e_3\\
	\\
	\psi^{(2,1,1)}_{(1234)} e_1 &= t_{23} + t_{42} + t_{34} - t_{32} - t_{24} - t_{42} = e_3\\
	\psi^{(2,1,1)}_{(1234)} e_2 &= t_{23} + t_{12} + t_{31} - t_{32} - t_{21} - t_{13} = e_1\\
	\psi^{(2,1,1)}_{(1234)} e_3 &= t_{34} + t_{13} + t_{41} - t_{43} - t_{31} - t_{31} = e_3 + e_2 - e_1\\
	\\
	\psi^{(2,1,1)}_{(12)(34)} e_1 &= t_{21} + t_{42} + t_{14} - t_{12} - t_{24} - t_{41} = - e_2\\
	\psi^{(2,1,1)}_{(12)(34)} e_2 &= t_{21} + t_{32} + t_{13} - t_{12} - t_{23} - t_{31} = - e_1\\
	\psi^{(2,1,1)}_{(12)(34)} e_3 &= t_{14} + t_{31} + t_{43} - t_{41} - t_{13} - t_{34} = -e_1 - e_2 - e_3\\
\end{align*} 
Therefore, we have that the character table is:
\begin{table}[h!]
	\centering
	\begin{tabular}{|l|l|l|l|l|l|}
		\hline
		$S_4$    & $(1)$ & $(12)$ & $(123)$ & $(12)(34)$ & $(1234)$ \\ \hline
		$\chi_1$ & 1     & 1      & 1       & 1          & 1        \\ \hline
		$\chi_2$ & 1     & -1     & 1       & 1          & -1       \\ \hline
		$\chi_3$ & 3     & 1      & 0       & -1         & -1       \\ \hline
		$\chi_4$ & 2     & 0      & -1      & 2          & 0        \\ \hline
		$\chi_5$ & 3     & -1     & 0       & -1         & 1        \\ \hline
	\end{tabular}
\end{table}

\subsubsection*{c}
For $S_5$, we have that the table looks something like this, by looking at the trivial, symmetric and standard representations. However, we can always take the pullback of representations of $S_5$. 
\begin{table}[h!]
	\centering
	\begin{tabular}{|l|l|l|l|l|l|l|l|}
		\hline
		$S_4$    & $(1)$ & $(12)$ & $(123)$ & $(12)(34)$ & $(1234)$ & $(12345)$ & $(123)(45)$ \\ \hline
		$\chi_1$ & 1     & 1      & 1       & 1          & 1        & 1         & 1           \\ \hline
		$\chi_2$ & 1     & -1     & 1       & 1          & -1       & 1         & -1          \\ \hline
		$\chi_3$ & 4     & 2      & 1       & 0          & 0        & -1        & -1          \\ \hline
		$\chi_4$ & 4     & -2     & 1       & 0          & 0        & -1        & 1           \\ \hline
		$\chi_5$ &       &        &         &            &          &           &             \\ \hline
		$\chi_6$ &       &        &         &            &          &           &             \\ \hline
		$\chi_7$ &       &        &         &            &          &           &             \\ \hline
	\end{tabular}
\end{table}
To be continued further.

\newpage
\subsection*{6.2}
\subsubsection*{a}
Obviously, all degree 1 representations remain irreducible after restriction. Since the degree $3$ characters maintain all information with the negative values removed, then the characters must be irreducible as well. This is in the Schur polynomial. Notice how $e_1 = t_{12} + t_{31} + t_{23} - t_{21} - t_{13} - t_{32}$ can be constructed from just knowing the positive values. However the degree 2 character must split. For $e_1 = t_{12} + t_{34} - t_{23} - t_{14}$, we need to know about the negative values to construct the degree 2 characters.
\subsection*{b}
We have that $\chi_1 = \chi_2$ as the signs of all values are trivially zero. Therefore, their respective reps are isomorphic. As $\chi_5 = \chi_2 \otimes \chi_3$, then $\chi_5$ must be identified with $\chi_3$ as the information from $\chi_2$ is lost as well. Respective reps are also isomorphic from characters being identical. $\chi_4$ as it splits will have complex values and is not isomorphic to anything. 

Note that representations with Young diagrams which are transposes of each other are identified to each other. If a $\lambda \vdash n$ is self-transpose, then they should split in the alternating group.
\subsection*{c}
So the irreps of $A_4$ which arise from restrictions of $S_4$ is the trivial representations, the representation of degree 3, and whatever representations $\chi_4$ splits to. 

\subsection*{6.4}
We have that $\rho^{(3,2)}$ of $S_5$ has basis $(3,2)$-tabloids that look like this:
\begin{align*}
	t_{12} &= 
	\begin{ytableau}
		\cdot & \cdot & \cdot\\
		1 & 2
	\end{ytableau}\\
		t_{13} &= 
	\begin{ytableau}
		\cdot & \cdot & \cdot\\
		1 & 3
	\end{ytableau}\\
		t_{14} &= 
	\begin{ytableau}
		\cdot & \cdot & \cdot\\
		1 & 4
	\end{ytableau}\\
		t_{15} &= 
	\begin{ytableau}
		\cdot & \cdot & \cdot\\
		1 & 5
	\end{ytableau}\\
		t_{23} &= 
	\begin{ytableau}
		\cdot & \cdot & \cdot\\
		2 & 3
	\end{ytableau}\\
			t_{24} &= 
	\begin{ytableau}
		\cdot & \cdot & \cdot\\
		2 & 4
	\end{ytableau}\\
			t_{25} &= 
	\begin{ytableau}
		\cdot & \cdot & \cdot\\
		2 & 5
	\end{ytableau}\\
			t_{34} &= 
	\begin{ytableau}
		\cdot & \cdot & \cdot\\
		3 & 4
	\end{ytableau}\\
	t_{35} &= 
	\begin{ytableau}
		\cdot & \cdot & \cdot\\
		3 & 5
	\end{ytableau}\\
	t_{45} &= 
	\begin{ytableau}
		\cdot & \cdot & \cdot\\
		4 & 5
	\end{ytableau}\\
\end{align*}
Now let us see where $\rho$ comes into play here. So $V^{(3,2)}$ has basis vectors above. We claim that the young diagrams $(5) \vDash 5$ and $(4,1) \vdash 5$ both dominate $(3,2)$, and that they are the only Young diagrams to do so. So the Specht representations $\psi^{(5)}$ and $\psi^{(4,1)}$ appears in $(3,2)$. Now let $\chi^{(3,2)}$ be the associated character of $\rho^{(3,2)}$. 

Now we have that $\chi^{(3,2)}(1) = 10$ as $V^{(3,2)} = 10$. We have that $\chi^{(3,2)}(12) = 4$ as we have that there are 4 tabloids which remain unchanged when applying $(12)$. We have that $\chi^{(3,2)}(12)(34) = 2$ as there are only two eleents in the basis that remain unchanged when they are swapped. Filling in the rest of the table, we have that:
\begin{align*}
	\chi^{(3,2)}(1) &= 10\\
	\chi^{(3,2)}(12) &= 4\\
	\chi^{(3,2)}(12)(34) &= 2\\
	\chi^{(3,2)}(123) &= 1\\
	\chi^{(3,2)}(123)(45) &= 1\\
	\chi^{(3,2)}(1234) &= 0\\
	\chi^{(3,2)}(12345) &= 0\\
\end{align*}
Now looking at the character table for $S_5$, we have:
\begin{table}[H]
\centering
\begin{tabular}{|l|l|l|l|l|l|l|l|}
	\hline
	$S_4$    & $(1)$ & $(12)$ & $(123)$ & $(12)(34)$ & $(1234)$ & $(12345)$ & $(123)(45)$ \\ \hline
	$\chi_1$ & 1     & 1      & 1       & 1          & 1        & 1         & 1           \\ \hline
	$\chi_2$ & 1     & -1     & 1       & 1          & -1       & 1         & -1          \\ \hline
	$\chi_3$ & 4     & 2      & 1       & 0          & 0        & -1        & -1          \\ \hline
	$\chi$   & 10    & 4     & 1        & 2          & 0        & 0         &  1          \\ \hline
\end{tabular}
\end{table}
However, we have that $\chi = \chi_1 + \chi_3 + \chi_5$ from $\chi$ above, and another irrep $\chi_5 = \tr(\psi^{(3,2)})$. This comes from looking at the Young diagrams that dominate $(3,2)$. Then we can construct $\chi_5$. 
We have that:
\begin{table}[H]
	\centering
	\begin{tabular}{|l|l|l|l|l|l|l|l|}
		\hline
		$S_4$    & $(1)$ & $(12)$ & $(123)$ & $(12)(34)$ & $(1234)$ & $(12345)$ & $(123)(45)$ \\ \hline
		$\chi_1$ & 1     & 1      & 1       & 1          & 1        & 1         & 1           \\ \hline
		$\chi_2$ & 1     & -1     & 1       & 1          & -1       & 1         & -1          \\ \hline
		$\chi_3$ & 4     & 2      & 1       & 0          & 0        & -1        & -1          \\ \hline
		$\chi_5$ & 5     & 1      & -1      & 1          & -1        & 0        & 1          \\ \hline
	\end{tabular}
\end{table}
\subsubsection*{6.2 c}
Finishing this question, we have that the table looks like this:
\begin{table}[H]
	\centering
	\begin{tabular}{|l|l|l|l|l|l|l|l|}
		\hline
		$S_4$                   & $(1)$ & $(12)$ & $(123)$ & $(12)(34)$ & $(1234)$ & $(12345)$ & $(123)(45)$ \\ \hline
		$\chi_1$                & 1     & 1      & 1       & 1          & 1        & 1         & 1           \\ \hline
		$\chi_2$                & 1     & -1     & 1       & 1          & -1       & 1         & -1          \\ \hline
		$\chi_3$                & 4     & 2      & 1       & 0          & 0        & -1        & -1          \\ \hline
		$\chi_4$                & 4     & -2     & 1       & 0          & 0        & -1        & 1           \\ \hline
		$\chi_5$                & 5     & 1      & -1      & 1          & -1       & 0         & 1           \\ \hline
		$\chi_5 \otimes \chi_1$ & 5     & -1     & -1      & 1          & 1        & 0         & -1          \\ \hline
		$\chi_7$                &       &        &         &            &          &           &             \\ \hline
	\end{tabular}
\end{table}
from part 6. Now we have that the sum must be 120, so the final character must be degree 6. Then we fill in using the second orthogonality relation.
\begin{table}[H]
	\centering
	\begin{tabular}{|l|l|l|l|l|l|l|l|}
		\hline
		$S_4$                   & $(1)$ & $(12)$ & $(123)$ & $(12)(34)$ & $(1234)$ & $(12345)$ & $(123)(45)$ \\ \hline
		$\chi_1$                & 1     & 1      & 1       & 1          & 1        & 1         & 1           \\ \hline
		$\chi_2$                & 1     & -1     & 1       & 1          & -1       & 1         & -1          \\ \hline
		$\chi_3$                & 4     & 2      & 1       & 0          & 0        & -1        & -1          \\ \hline
		$\chi_4$                & 4     & -2     & 1       & 0          & 0        & -1        & 1           \\ \hline
		$\chi_5$                & 5     & 1      & -1      & 1          & -1       & 0         & 1           \\ \hline
		$\chi_5 \otimes \chi_1$ & 5     & -1     & -1      & 1          & 1        & 0         & -1          \\ \hline
		$\chi_7$                & 6     & 0      & 0       & -2         & 0        & 1         & 0           \\ \hline
	\end{tabular}
\end{table}
\subsubsection*{My question}
Related to alternating groups, we wish to show that if $\lambda \vdash n$, then $\rho^\lambda(\sigma) = sgn(\sigma) \rho^{(\overline{\lambda})}(\sigma)$ where $\overline{\lambda}$ is the transpose of $\lambda$.
Consider the linear map $\phi(\sigma) = sgn(\sigma) \sigma$ where $\phi : \mathbb{C}S_n \rightarrow \mathbb{C} S_n$. Then if $\rho : S_n \rightarrow GL(V)$, we have that $\rho \circ \phi = \rho \otimes \rho_{sgn}$ as we can pass through the sign through $\rho$ with the same result.
Now let $R_t$ e the row stabiliser and $C_t$ be the column stabiliser. 
Let us see what $\phi$ does to the basis elements of $V^\lambda$, and more specifically the stabiliser. 
\begin{equation}
	Z(e_t) &= \sum_{\pi \in R_t} sgn(\pi) e_t
\end{equation}
\begin{equation}
	\phi_\sigma(e_t) = sgn(\sigma) e_{\sigma t}
\end{equation}
We can rewrite $Z(e_t)$ as a summation of things in $e_{\overline{t}}$. We have that the row stabilisers do nothing to $[t]$, so we stick them in. 

\begin{align*}
	Z(e_t) &= \sum_{\pi \in R_t} sgn(\pi) \sum_{\tau \in C_t} sgn(\tau) \tau[t]\\
	Z(e_t) &= \sum_{\pi \in R_t}  \sum_{\tau \in C_t} sgn(\pi) sgn(\tau) \tau \circ \pi [t]\\
\end{align*}
Now we use the fact that $R_t = C_{\overline{t}}$ and $C_t = R_{\overline{t}}$. We have that
\begin{align*}
	Z(e_t) &= \sum_{\pi \in C_{\overline{t}}}  \sum_{\tau \in R_{\overline{t}}} sgn(\pi) sgn(\tau) \tau \circ \pi [t]\\
	&= \sum_{\pi \in C_{\overline{t}}}  \sum_{\tau\in R_{\overline{t}}} sgn(\pi) sgn(\tau) \pi [t]\\
	&= \sum_{\tau\in R_{\overline{t}}} sgn(\tau) \sum_{\pi \in C_{\overline{t}}}  sgn(\pi)  \pi [t]\\
	&= \sum_{\tau\in R_{\overline{t}}} sgn(\tau) e_{\overline{t}}\\
	&= Z_(e_{\overline{t}})
\end{align*}

Therefore, we have that $
\end{document}
