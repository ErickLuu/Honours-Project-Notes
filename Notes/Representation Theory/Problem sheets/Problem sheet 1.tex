\documentclass[]{article}
\usepackage{amsmath}
\usepackage{amssymb}
\usepackage{bm}
%opening
\title{Representation Theory notes}
\author{Eric Luu}

\begin{document}

\maketitle

\section*{Chapter 0}

\subsection*{0.1}
\subsubsection*{a}
\paragraph{Field} A field is a set of numbers with two binary operations, addition and multiplication. It must hold that the field is an abelian group under addition with a $0$ as the identity, the field is an abelian group under multiplication with 1 as the identity, and there is a distributive property.  An example of a field is $\mathbb{R}$ and $\mathbb{Q}$. 
\paragraph{Direct sum} A direct sum of abelian groups $\bigoplus_{i \in I} A_i$ is the set of tuples where $a_i$ is nonzero for finitely many elements. Addition is defined coordinate-wise. An example of a direct sum is $\mathbb{R} \times \mathbb{R}$ or the infinite sum of complex numbers $\bigoplus \mathbb{C}$.
\paragraph{Vector space} A vector space is a set with a field $F$ where they are an abelian group addition and there is a concept of scalar multiplication which satisfies distribution properties. Two important vector spaces are $\mathbb{R}^2$ and $\mathbb{C}$ 
\paragraph{Linear map} A linear map $f : V \rightarrow W$ is a map between vector spaces $V$ and $W$ where $f(a \bm{v} + b \bm{w})  = a f(\bm{v}) + b f(\bm{w})$. A linear map between $\mathbb{R}^2$ and $\mathbb{Q}$ is $f(a, b) = a + bi$. Another map between $\mathbb{R}^2$ and $\mathbb{R}$ could be $f(a, b) = a + b$.  
\paragraph{Subspace} A subspace is subset of a vector space which is closed under scalar multiplication and addition. A subspace of $\mathbb{R}^2$ is $ \langle (x, y) | x + y = 1 \rangle$ and a subspace of $\mathbb{C}$ is $0$. 
\paragraph{Inner product} An inner product on $V$ is a binary operation $\langle \cdot, \cdot \rangle : V \times V \rightarrow F \rbrace$. It must satisfy the property that $\langle x, y \rangle  = \overline{\langle y, x \rangle}$, $\langle ax + by, z \rangle = a \langle x, z \rangle + b \langle y , z \rangle$ and $\langle x, x \rangle > 0$. An example of an inner product on $\mathbb{C}^n$ is $a^T b$ or for any square matrix, $a^T Ab$. 
\paragraph{Span} A subset $S$ of a vector space $V$ spans $V$ if all elements in $V$ can be written as a linear combination of elements in $S$. The span of $\mathbb{C}$ is $1$ and $i$. It could also be $\lbrace 2, 2 + i, 1 - i \rbrace$. 
\paragraph{Basis} A subset $B$ is a basis of a vector space $V$ if $B$ spans $V$ and there is no linear combination of elements of $B$ that sum to 0. From above, we have that $\lbrace 1, i \rbrace$ is a basis but $\lbrace 2, 2 + i, 1-i \rbrace$ is not a basis of $\mathbb{C}$.
\paragraph{Dimension} The dimension of $V$ is the number of elements in a basis of $V$. The dimension of $\mathbb{R}^n$ is $n$ and the dimension of $\mathbb{C}^n$ is $2n$. 
\paragraph{Trace} The trace of an $n \times n$ matrix $A$ is equal to $\sum_{i=1}^n [A]_{i,i}$. The trace of a diagonalisable matrix is the sum of the eigenvectors and the trace of $I_n$ is $n$.
\paragraph{Determinant} The determinant of an  $n \times n$ matrix $A$ is equal to:
\begin{equation}
	\sum_{\sigma \in S_n} \text{sign}(\sigma) \prod_{i = 1}^n [A]_{i, \sigma(i)}.
\end{equation} 
The determinant of a diagonalisable matrix is the product of the eigenvectors and the determinant of $I_n$ is 1. 

\paragraph{Characteristic polynomial} The characteristic polynomial $P$ of a square matrix $A$ is $P(\lambda) := \det(\lambda I - A )$. The characteristic polynomial of $
\begin{Bmatrix}
	1 & 1 \\
	1 & 1 \\
\end{Bmatrix}
$
is $(\lambda^2 - 2\lambda)$, and the characteristic polynomial of $I_n$ is $(\lambda - 1)^n$. 
\paragraph{Eigenvector+ eigenvalues} An eigenvector of a matrix $A$ is a vector $v$ such that $A v$ = $\lambda v$ where $\lambda$ is the eigenvalue of $A$. The eigenvectors of 
$
\begin{Bmatrix}
	1 & 1 \\
	1 & 1 \\
\end{Bmatrix}
$
are $
\begin{bmatrix}
	1\\
	1
\end{bmatrix}
$
and
$\begin{bmatrix}
	-1\\
	1
\end{bmatrix}
$
with eigenvalues 2. 
\paragraph{Jordan normal form} The Jordan normal form of a matrix $A$ is an upper triangular matrix where there are eigenvalues of $A$ on the diagonal, some ones above it in Jordan blocks (squares around degenerate eigenvalues) and zeroes elsewhere.
For example, 
$
\begin{bmatrix}
	\lambda_1 & 1         & 0         & 0         \\
	0         & \lambda_1 & 0         & 0         \\
	0         & 0         & \lambda_2 & 0         \\
	0         & 0         & 0         & \lambda_3
\end{bmatrix}
$
is in Jordan normal form, and so is $I_n$. 
\paragraph{Diagonalisable matrix} A diagonalisable matrix $A$ can be written as $P \Lambda P^{-1}$ where $P$ is an invertible matrix and $\Lambda$ is a diagonal matrix. An example of something of this form is
$
\begin{bmatrix}
	1 & 1 \\
	1 & 1 \\
\end{bmatrix}
= P\Lambda P^{-1}
$
where $P = 
\begin{bmatrix}
	-1& 1\\
	1 & 1 
\end{bmatrix}
$
and $J = \begin{bmatrix}
	0 & 0 \\
	0 & 2
\end{bmatrix}
$.
We have that if $A = I$, then $P$ and $\Lambda = I$. 
\subsubsection*{b}
\paragraph{Group}
A group $G$ is a set of elements with a binary operation $ \cdot \times \cdot : G \times G \rightarrow G$ which satisfies the associativity, identity and inverse laws. An example of a group is $K_4$ and $D_4$.
\paragraph{Symmetric group} For an set $X$, the symmetric group $S_X$ is the group of bijections from $X$ to itself. For example, the set $\lbrace 1, 2, 3\rbrace$ would have $S_3$ as the group of bijections, so $\lbrace (), (1, 2), (1, 3), (2, 3), (1,2,3), (1, 3, 2) \rbrace$. 
\paragraph{Homomorphism} A homomorphism is a function $f$ between two groups $G$ and $H$ such that $f(ab) = f(a) f(b)$ for all $a, b \in G$.  An example of a homomorphism is $S_3 \rightarrow S_4$ naturally, and $S_3 \rightarrow D_3$ through the permutation $r = (123), d = (12)$. 
\paragraph{Isomorphism} An isomorphism $\phi$ is a homomorphism between two groups $G$ and $H$ that is also a bijection. For example $S_3 \cong D_3$ and $S_2 \cong \mathbb{Z}/2 \mathbb{Z}$. 
\paragraph{Subgroup} A subgroup of $G$ is a set of elements in $G$ that are closed under inversion and group multiplication. For example, $S_3$ and $S_2$ are subgroups of $S_4$, and $\mathbb{Z}/4 \mathbb{Z}$ is a subgroup of $D_4$.
\paragraph{Normal subgroup} A normal subgroup $N$ of a group $G$ is a subgroup such that $g N g^{-1} = N$ for all $g \in G$. A normal subgroup of $S_3$ is $(), (123), (132)$ and a normal subgroup of $D_4$ is $1, r, r^2, r^3$. 
\paragraph{Coset} A left coset of a normal subgroup $N$ in $G$ are the equivalence classes $gN$ for some $g \in G$ which partition the set of elements in $G$. The cosets of $S_3$ are $\langle (), (123), (132) \rangle$ and $\langle (12), (13), (23) \rangle$.
\paragraph{Quotient group} For a normal subgroup $N$ in a group $G$, the normal subgroup $G/N$ have elements $gN$ for left cosets of $G$ and the group operation of $(gN)(hN)$ is $(gh)N$. We have that $S_3/\langle (), (123), (132) \rangle = \mathbb{Z}/2 \mathbb{Z}$ and $D_4/ \langle 1, r, r^2, r^3 \rangle = \mathbb{Z}/2 \mathbb{Z}$.
\paragraph{Cyclic group } A cyclic group is a group of the form $\mathbb{Z}/n\mathbb{Z}$ where $n\mathbb{Z}$ are the elements of $\mathbb{Z}$ which are multiples of $n$. Two examples are $\mathbb{Z}/2 \mathbb{Z} = \langle 0, 1 \rangle $ and $\mathbb{Z}/4 \mathbb{Z} = \langle 0, 1, 2, 3 \rangle$. 
\paragraph{Alternating group} The alternating group $A_X$ is the subgroup of $S_X$ where the number of transpositions are even. An example, $A_3$, is $\langle (), (123), (132) \rangle$ and $A_2$ is $()$. 
\paragraph{Group presentations} A group presentation of $G$ is the group 
\begin{equation}
	\langle r_1, r_2, ..., r_n | r_a r_b r_c, ...\rangle,
\end{equation}
where $r_1, r_2, ..., r_n$ is a set of generators and $r_a r_b r_c, ...$ is a set of relators.
The group $G$ can be written as the free group on $n$ generators quotiented out by the smallest normal group $N$ containing the relators. 
For example, we have that $\langle a | a^5 \rangle$ is isomorphic to $\mathbb{Z}_5$ and $ \langle a, b | aba^{-1}b^{-1} \rangle$ is isomorphic to $\mathbb{Z} \times \mathbb{Z}$. 
\paragraph{Direct product} The direct product of two groups $G, H$ is the group $G \times H$ with elements $(g, h)$ and multiplication is defined coordinate wise. The direct product of $\mathbb{Z}$ and $\mathbb{Z}$ is $\mathbb{Z} \times \mathbb{Z}$, and $\mathbb{Z}_2 \times \mathbb{Z}_2 \cong K_4$. 
\paragraph{Dihedral group} A dihedral group is a group with presentation $\langle r, d | r^n, d^2 , rdrd \rangle$ for some $n$. Two examples are $D_4$ and $D_3 \cong S_3$. 
\paragraph{Group action} If $G$ is a group and $X$ is a set, a group action $\rho: G \times X \rightarrow X $ is a function which satisfies $\rho_1(x) = x$ and $\rho_g \rho_h(x) = \rho_{gh} (x)$. A group action could be $S_3$ on a tuple $(1, 2, 3)$ or $D_4$ on the set of vertices of the square.

\subsection*{0.3}
\subsubsection*{a}
\begin{itemize}
	\item Reflexive: We have that $a = 1 a 1^{-1}$ so $a$ is conjugate to a.
	\item Symmetric: We have that if $a = g b g^{-1}$, then $b = g^{-1} a g$. 
	\item Transitive: If $a = g b g^{-1}$, $b = h c h^{-1}$, then $a = gh c h^{-1} g^{-1} = (gh) c (gh)^{-1}$. 
\end{itemize}
Thus conjugacy is an equivalence relation. 
\subsubsection*{b}
We have that $D_4$ has elements $1, r, d, r^2, r^3, rd, r^2d, r^3d$ with the relationship $drd = r^{-1}$. Therefore, the conjugacy classes are: $\lbrace 1 \rbrace, \lbrace r^2 \rbrace, \lbrace r, r^3 \rbrace, \lbrace d, r^2 d \rbrace, \lbrace rd, r^3 d \rbrace$. 
\subsubsection*{c}
\paragraph{$\Leftarrow$}
Suppose $a$ and $b$ are permutations of the set $X$. Let $a_i = (a_{i, 1}, a_{i, 2}, ...a_{i, m})$ be a cycle of length $n$. We have that $a$ is comprised of disjoint cycles $(a_1, a_2, ..., a_k)$ with lengths $(\ell_1, \ell_2, ..., \ell_k)$ in non-increasing order, and $b$ has disjoint cycles $(b_1, b_2, ..., b_k)$ with the same lengths in non-increasing order. We have all elements in $X$ appear once in a cycle of $a$, and same for $b$. There is a bijection $\sigma_i : a_i \rightarrow b_i$ where we send element $a_{i, j}$ to $b_{i, j}$ in the cycle. We have that each element appears once in these cycles, thus we can combine these sigmas to form the function $\sigma: X \rightarrow X$. We have that $\sigma(a_{i, j} = b_{i,j})$, if $a_{i,j}$ appears in $a$. Then we have that $b = \sigma a \sigma^{-1}$, as for any element $x \in X$ we take $x$ to $\sigma^{-1}$ to move $x$ from $b_{i,j}$ to $a_{i,j}$, cycle $a_{i,j}$ to $a_{i, j+1}$ and then send $a_{i, j+1}$ back to $b_{i, j+1}$. Therefore, if $a$ and $b$ have the same cycle type, there is a $\sigma$ such that $b = \sigma a \sigma^{-1}$. 
\paragraph{$\Rightarrow$}
Suppose $a = \sigma b \sigma^{-1}$ for some $\sigma$, then it follows that $\sigma$ simply relabels the elements of $a$ to a new label in $b$, permutes it according to $b$, then undoes the labelling to go from $b$ to $a$. Thus if the elements $(a_1, a_2, a_3, ... a_k)$ form a cycle in $a$, but we have that $a \circ \sigma(b_i) = \sigma \circ b(b_i)$, but $\sigma(b_i) = a_i$ and $a (a_i) = a_{i+1}$, thus we have that the elements $(b_1, b_2, ..., b_k) $ where $\sigma^{-1} a_i = b_i$ also form a disjoint cycle. Therefore, disjoint cycles in $a$ are mapped to disjoint cycles in $b$, and therefore the lengths of disjoint cycles $(\ell_1, \ell_2, ..., \ell_k)$ must be identical for $a$ and $b$. 

\subsubsection*{d}
We have the conjugacy classes of $S_4$ on the set $\lbrace 1, 2, 3, 4 \rbrace$ are: 
\begin{itemize}
	\item $\lbrace () \rbrace$
	\item $\lbrace (1, 2), (1, 3), (1, 4), (2, 3), (2, 4), (3, 4) \rbrace$
	\item $\lbrace (1, 2, 3) , (1,3, 2), (1,2, 4), (1, 4, 2), (1, 3, 4), (1, 4, 3), (2, 3, 4), (2, 4, 3) \rbrace$
	\item $\lbrace (1, 2, 3) , (1,3, 2), (1,2, 4), (1, 4, 2), (1, 3, 4), (1, 4, 3), (2, 3, 4), (2, 4, 3) \rbrace$
	\item $\lbrace (1,2)(3, 4), (1, 3)(2, 4), (1,4)(2,3) \rbrace$
	\item $\lbrace(1, 2, 3, 4), (1, 2, 4, 3), (1, 3, 2, 4), (1, 3, 4, 2), (1, 4, 2, 3), (1, 4, 3, 2) \rbrace$
\end{itemize}

\subsubsection*{e}
We have that the number of conjugacy classes in the symmetric group $S_5$ is equivalent to finding the number of unique nondecreasing sequences of numbers between 1 and 5 that add to 5 by the characterization of cycles from part $c$.
We have the following list:
\begin{enumerate}
	\item 1 + 1 + 1 + 1 + 1
	\item 2 + 1 + 1 + 1
	\item 3 + 1 + 1
	\item 4 + 1
	\item 5
	\item 2 + 2 + 1
	\item 3 + 2
\end{enumerate}
Thus there are seven conjugacy classes in $S_5$.

\subsection*{0.4}
\begin{equation*}
tr(AB) = \sum_{i = 1}^n [AB]_{i,i} = \sum_{i = 1}^n \sum_{j = 1}^n [A]_{i,j} [B]_{j, i } = \sum_{j = 1}^n\sum_{i = 1}^n  [B]_{j, i }[A]_{i,j}   = \sum_{j = 1}^n [BA]_{j,j} = tr(BA)
\end{equation*}
Thus shown. 
\section*{1}
\subsection*{1.1}
\subsubsection{a}
We act left to right, so $rs$ means we do $r$ and then $s$ next.
We have that $s$ sends $
\begin{bmatrix}
	1\\
	0
\end{bmatrix}
$
to 
$
\begin{bmatrix}
	1\\
	0
\end{bmatrix}
$
and it sends 
$
\begin{bmatrix}
	0\\
	1
\end{bmatrix}
$
to
$
\begin{bmatrix}
	0\\
	-1
\end{bmatrix}
$
So $\rho_s =
\begin{bmatrix}
	1 & 0 \\
	0 & -1\\
\end{bmatrix}
$.
\begin{itemize}
	\item $\rho_s =
	\begin{bmatrix}
		1 & 0 \\
		0 & -1\\
	\end{bmatrix}
	$.
	\item $\rho_{r^2} = 
	\begin{bmatrix}
		-1 & 0 \\
		0 & -1\\
	\end{bmatrix}
	$
	\item $\rho_{r^3} = 
	\begin{bmatrix}
		0 & 1 \\
		-1 & 0\\
	\end{bmatrix}
	$
	\item $\rho_{rs} = 
	\begin{bmatrix}
		0 & 1 \\
		1 & 0\\
	\end{bmatrix}
	$
	\item $\rho_{r^2s} = 
	\begin{bmatrix}
		-1 & 0 \\
		0 & 1\\
	\end{bmatrix}
	$
	\item $\rho_{r^3s} = 
	\begin{bmatrix}
		0 & -1 \\
		-1 & 0\\
	\end{bmatrix}
	$
\end{itemize}
\subsubsection{b}
We have that $s$ sends $A$ to $D$ and $D$ to $A$, and $B$ to $C$ and $C$ to $B$.
So $\phi_s = 
\begin{bmatrix}
	0 & 0 & 0 & 1\\
	0 & 0 & 1 &0 \\
	0 & 1 & 0 & 0 \\
	1 & 0 & 0 & 0\\
\end{bmatrix}$
\begin{itemize}
	\item $\phi_s = 
		\begin{bmatrix}
			0 & 0 & 0 & 1\\
			0 & 0 & 1 &0 \\
			0 & 1 & 0 & 0 \\
			1 & 0 & 0 & 0\\
		\end{bmatrix}$
	\item $\phi_{r^2} = 
		\begin{bmatrix}
			0 & 0 & 1 & 0\\
			0 & 0 & 0 & 1 \\
			1 & 0 & 0 & 0 \\
			0 & 1 & 0 & 0\\
		\end{bmatrix}$
	\item $\phi_{r^3} = 
		\begin{bmatrix}
			0 & 0 & 1 & 0 \\
			0 & 0 & 0 & 1 \\
			1 & 0 & 0 & 0 \\
			0 & 1 & 0 & 0
		\end{bmatrix}$
	\item $\phi(rd) = 
	\begin{bmatrix}
		0 & 0 & 1 & 0 \\
		0 & 1 & 0 & 0 \\
		1 & 0 & 0 & 0 \\
		0 & 0 & 0 & 1
	\end{bmatrix}$
	\item 
$\phi(r^2 d) = 
\begin{bmatrix}
	0 &  1 & 0 & 0\\
	1 & 0 & 0 & 0 \\
	0 & 0 & 0 & 1 \\
	0 & 0 & 1 & 0\\
\end{bmatrix}$
	\item 
$\phi(r^3 d) = 
\begin{bmatrix}
	1 &  0 & 0 & 0\\
	0 & 0 & 0 & 1 \\
	0 & 0 & 1 & 0 \\
	0 & 1 & 0 & 0\\
\end{bmatrix}$
\end{itemize}
\subsubsection*{c}

\subsubsection*{d}

\subsection*{1.2}
Recall that two representations $\rho$ and $\phi$ are equivalent if there exists an isomorphism of vector spaces $T: \mathbb{C} \rightarrow \mathbb{C}$ such that $T \rho_g = \phi_g T$ for all $g \in G$. 
If $\rho, \phi$ are equivalent functions, then $\rho_g = [1] \phi_g [1]$ for all $g$ in $G$, where $[1] : \mathbb{C} \rightarrow \mathbb{C}$ is the identity function. Thus $\rho_g$ and $\phi_g$ are equivalent representations.
If $\rho$ and $\phi$ are equivalent representations, then $\rho_g = T \phi_g T^{-1}$ for some isomorphism $T$ for all $g \in G$. However, we can represent $T = [a]$, where $ a \in \mathbb{C}^*$. Then $\rho_g = a \phi_g a^{-1}$ as $\phi_g \in \mathbb{C}^*$. However, multiplication commutes in $\mathbb{C}$, so $\rho_g = a a^{-1} \phi_g = \phi_g$ for all $g \in G$. Thus $\rho$ and $\phi$ are equal functions. 

\subsection{1.3}
\subsubsection*{a}
We must show that $\rho(1) = 1$. We have that $\rho(1) \rho(g) =\rho(1 g) =  \rho(g)$ for all $g \in G$ by the commutativity of $\rho$. Thus $\rho(1)$ is the identity in $\mathbb{C}^*$ which is $1$. Therefore, $\rho(g^n) = \underbrace{\rho(g) \rho(g) \cdots \rho(g)}_{n \text{ times}} = \rho(1) = 1$, therefore $\rho(g)^n = 1$. Thus $\rho(g)$ is an $n$-th root of unity. 
\subsubsection*{b}
We have that $\rho^{(k)}(0) = 1$ and $\rho^{(k)}(a) = \exp( 2\pi i ka/n)$ where $k \in \lbrace 0, 1, 2, \cdots n-1 \rbrace$.
To show that this is a representation, we have that 
\begin{equation}
	\rho^{(k)}(a) \rho^{(k)}(b)  = \exp(k 2\pi i a/n) \exp(k 2\pi i b/n) = \exp(k 2\pi i (a + b)/n) = \rho^{(k)}(a + b)
\end{equation}
Finally, $\rho^{(k)}(1^n)= (\rho^{(k)}_{1})^n = \exp(k 2 \pi) = 1$, so the generating element $1$ is an $n$-th root of unity, meaning that this is a representation of $\mathbb{Z}_n \rightarrow \mathbb{C}^*$.
As each $\rho^{(k)}$ sends $1$ to different values on the unit circle, by $1.2$, each $\rho^{(k)}$ is not equal and thus not equivalent. 
\subsubsection*{c}
Suppose $\rho$ is a representation of $\mathbb{Z}_n \rightarrow \mathbb{C}^*$. Then $1$ is an element of order $n$, thus $\rho(1)^n = 1$, so $\rho(1)$ is an $n$-th root of unity. However, all roots of unity in $\mathbb{C}$ are written as$\exp(k i 2\pi/n)$ for some integer $k$ in $\lbrace 0, 1, ..., n-1 \rbrace$, however, there is already a representation in the upper list that sends $1$ to $ \exp(k i 2\pi/n)$, namely $\rho^{(k)}$. As we have that where $1$ goes determines where every other element goes, then $\rho$ is equal to a representation in the question above. 

\end{document}
