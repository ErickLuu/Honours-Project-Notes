\documentclass{article}
\usepackage[margin=1in]{geometry}
\usepackage{amsmath}
\usepackage{amssymb}
\usepackage{amsthm}
\usepackage{url}
\usepackage{todonotes}

% Environments

\newtheorem{theorem}{Theorem}
\newtheorem{proposition}[theorem]{Proposition}
\newtheorem{corollary}[theorem]{Corollary}
\newtheorem{lemma}[theorem]{Lemma}
\newtheorem{definition}[theorem]{Definition}
\newtheorem{conjecture}[theorem]{Conjecture}
\newtheorem{remark}[theorem]{Remark}


\theoremstyle{definition}
\newtheorem{example}[theorem]{Example}

\numberwithin{theorem}{section}
\numberwithin{equation}{section}

%opening
\title{Assignment 1}
\author{Eric Luu}

\begin{document}

\maketitle

\section{1}
\subsection{a}
We let $F(x, y, z) : \mathbb{R}^3 \rightarrow \mathbb{R}^2$ be the smooth map
\begin{equation}
    F(x, y, z) = (G(x, y, z), H(x, y, z)).
\end{equation}

Then consider the Jacobian with respect to coordinates $x$ and $z$. 

\begin{equation}
    D_2 F(x, y, z) = 
    \begin{bmatrix}
        \dfrac{\partial G}{\partial x} & \dfrac{\partial G}{\partial z}\\
        \\
        \dfrac{\partial H}{\partial x} & \dfrac{\partial H}{\partial z}
    \end{bmatrix}
\end{equation}

If the Jacobian is invertible, then by the implicit function theorem, then there exist functions $g(y)$ and $h(y)$ such that $g(-1) = 2$ and $h(-1) = 1$, and:
$F(g(y), y, h(y)) = (0, 0)$ on an open interval $A$ which contains $-1$. 

We have that 

\begin{align*}
    \dfrac{\partial G}{\partial x} &= D_x(f)\\
    \dfrac{\partial G}{\partial z} &= 2z\\
    \dfrac{\partial H}{\partial x} &= z\\
    \dfrac{\partial H}{\partial z} &= x + 3z^2
\end{align*}
so if we want the Jacobian to be invertible, we must have that:

\begin{equation}
    (D_x f)(x + 3 z^2) - 2 z^2
\end{equation}
is nonzero everywhere. 

\subsection{b}


\section{2}
\subsection{a}

Let $\phi_i^{-1} : \mathbb{R}^n \rightarrow U_i$ be the $(x^0, x^1, \ldots, \hat{x^i}, \ldots, x^n)$ to $[x^0, x^1, x^2, \ldots, 1, \ldots, x^n]$ in $\mathbb{RP}^n$ where we have that $x^i = 1$. Then we have that $\phi \circ \phi^{-1} (x^0, \ldots, x^n) = (x^0, \ldots, x^n)$ and $\phi^{-1} \circ \phi(x^0, \ldots, x^n) = [x^0, \ldots, x^n]$. Therefore, $\phi_i$ is a well-defined bijection as there is a well-defined inverse. 

\subsection{b}

Consider $\phi_i \circ \phi_j^{-1}$ on the set ${(x^0, x^1, \ldots, x^n) | x^i \neq 0}$. We have that:

\begin{equation}
    \phi_i \circ \phi_j^{-1}(x^0, x^1, \ldots, x^n) = (\frac{x^0}{x^i}, \frac{x^1}{x^i}, \ldots, \hat{\frac{x^i}{x^i}}, \ldots, \frac{1}{x^i}, \ldots, \frac{x^n}{x^i})
\end{equation}
where $\frac{1}{x_i}$ is the $j$-th coordinate. As this is smooth in all coordinates on the open subset $x^i \in \mathbb{R} - \{0\}$, then we have that this is a smooth transition map. 

\subsection{c}
We have that $\left\{U_i\right\}$ is a cover of $\mathbb{RP}^n$ and $\left\{\phi_i\right\}$ are also smoothly compatible. This is a cover as the point $(0, 0, \ldots, 0)$ is not included in $\mathbb{RP}^n$, so all points must have a nonzero entry. Therefore, $\mathbb{RP}^n$ is a smooth $n$-manifold. 

\subsection{d}
To show that the projection is smooth, we want to show that $\phi_i \circ \pi : \mathbb{R}^{n + 1}_+ \rightarrow \mathbb{R}^{n}$ is a smooth map. We have that this map is $(x^1, x^2, \ldots, x^n) \mapsto (\frac{x^1}{x_i}, \ldots, \widehat{\frac{x^i}{x^i}}, \ldots, \frac{x^n}{x^i})$ which is well-defined for all $y = \lambda x$ where $\lambda \neq 0$ and is also smooth. 

\section{3}

Consider 
\end{document}
