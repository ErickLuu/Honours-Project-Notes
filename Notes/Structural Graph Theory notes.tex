\documentclass[]{article}

\usepackage{amsmath}
\usepackage{amssymb}
\usepackage{amsthm}

% Commands
\newcommand{\tree}{\mathcal{T}}

% Environments

\newtheorem{theorem}{Theorem}
\newtheorem{proposition}[theorem]{Proposition}
\newtheorem{corollary}[theorem]{Corollary}
\newtheorem{lemma}[theorem]{Lemma}
\newtheorem{definition}[theorem]{Definition}
\newtheorem{conjecture}[theorem]{Conjecture}

\theoremstyle{definition}
\newtheorem{example}[theorem]{Example}

\numberwithin{theorem}{section}
\numberwithin{equation}{section}

%opening
\title{Structural Graph theory}
\author{Eric Luu}

\begin{document}

\maketitle

\section{Treewidth and balanced separators}

\begin{definition}[Tree-decomposition]
	The tree-decomposition $\tree$ of a graph $G$ is defined as a tree $T$ with associated \textit{bags} $\lbrace B_x : x \in V(T) \rbrace$ such that:
	\begin{itemize}
		\item for all $v \in V(G)$, the subset of vertices $\lbrace x \in V(T): v \in B_x \rbrace$ in $V(T)$ induces a connected subtree in $V(T)$.
		\item For all edges $vw \in E(G)$, there exists a bag $B_x$ such that both $v$ and $w$ are in the bag $B_x$.
	\end{itemize}
\end{definition}
The \textit{width} of the tree decomposition $\tree$ is defined as $\max \lbrace |B_x| - 1 : x \in V(T) \rbrace$. We define the \textit{treewidth} of a graph $G$ as such:


\begin{definition}
	The treewidth is defined to be the smallest width for all tree decompositions of the graph $G$.
\end{definition}
\end{document}
