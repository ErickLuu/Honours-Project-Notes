\documentclass{article}
\usepackage[margin=2in]{geometry}
\usepackage{amsmath}
\usepackage{amssymb}
\usepackage{amsthm}
\usepackage{url}

% Environments

\newtheorem{theorem}{Theorem}
\newtheorem{proposition}[theorem]{Proposition}
\newtheorem{corollary}[theorem]{Corollary}
\newtheorem{lemma}[theorem]{Lemma}
\newtheorem{definition}[theorem]{Definition}
\newtheorem{conjecture}[theorem]{Conjecture}

\theoremstyle{definition}
\newtheorem{example}[theorem]{Example}

\numberwithin{theorem}{section}
\numberwithin{equation}{section}

\newcommand{\alg}{\mathcal{A}}
\newcommand{\sig}{\mathcal{S}}
\newcommand{\mono}{\mathcal{M}}
\newcommand{\sigmes}{\Lambda_{\mu^*}}
%opening
\title{Measures}
\author{Eric Luu}

\begin{document}

\maketitle
\section{What is a measure?}

\subsection{Algebras and $\sigma$-algebras}
Let $X$ be a nonempty set, and let $2^X$ be its powerset. Then a collection $\alg \subset 2^X$ is an algebra if it satisfies the properties:

\begin{enumerate}
	\item $\emptyset$ is in $\alg$
	\item $\alg$ is closed under complementation
	\item $\alg$ is closed under finite union.
\end{enumerate}
Note that this implies that $X \in \alg$ and $\alg$ is closed under finite intersection, by De Morgan's law. Note that algebras are not closed under inclusion, meaning that if $B \subset A$ then $B$ may not be in $\alg$.

\subsubsection{$\sigma$-algebra}
Let $x$ be a nonempty set. Then $\sig \subset 2^X$ is a $\sigma$-algebra if:
\begin{enumerate}
	\item $\sig$ is an algebra
	\item $\sig$ is closed under countable unions. 
\end{enumerate}
Every $\sigma$-algebra is also closed under countable intersection. 

\begin{example}
	\begin{enumerate}
		\item $2^X$ is a $\sigma$-algebra. This is the smallest $\sigma$-algebra on a set.
		\item $\lbrace \emptyset, X \rbrace$ is a $\sigma$-algebra. Note that this is the largest $\sigma$-algebra on a set.
		\item If $A \subset X$, then $\lbrace \emptyset, X, A, A^C \rbrace$ is a $\sigma$-algebra.
		\item $X$ is infinite, collection of finite elements is not an algebra-not closed under complementation. 
		\item If $X$ is infinite, then $\mathcal{C}:= \lbrace A \subset X : A \text{ or } A^c \text{ is finite} \rbrace$ is an algebra but not a $\sigma$-algebra. Consider countable union of singleton elements.
		\item If $X$ is uncountable, then the collection of countable elements is not an algebra. This is not closed under complementation.
		\item If $X$ is uncaountable, the collection $\mathcal{C} := \lbrace A \subset X : A \text{ or } A^C \text{ is countable} \rbrace$ is a $\sigma$-algebra. It is closed under complementation. (Show)
		\item The collection $C_1$ of $\mathbb{R}$ that can be written as a finite union of the type $(a, b]$, $(a, \infty)$, $(-\infty, b]$, $(-\infty, \infty)$ is an algebra but not a $C^*$-algebra. Note that $(1/2, 1] \cup (1/4, 1/3] \cup ...$ is not in $C_1$.
	\end{enumerate}
\end{example}

Remarks: Intersections of $\sigma$-algebras is a $\sigma$-algebra. The union of two algebras is not necessarily an algebra. For all $\mathcal{F} \subset 2^X$, there exists a smallest $\sigma$-algebra $\sigma(\mathcal{F})$ which contains $\mathcal{F}$. This comes from taking the intersections of all $\sigma$-algebras containing $\mathcal{F}$. 

\begin{example}[Borel sets]
	Let $(X, d)$ be a separable metric space, such that there exists a countable subset such that all open sets contains at least one element in the subset. Then the Borel-$\sigma$-algebra is the smallest $\sigma$-algebra on the open sets in $X$.
\end{example}

A monotone class on $X$ is a collection $\mathcal{M}$ of subsets of $X$ that is closed under monotone unions and monotone intersections.
\begin{enumerate}
	\item if $A_1 \subset A_2 \subset ..., $ then $\bigcup_{n \in \mathbb{N}} A_n \in \mathcal{M}$.
	\item if $A_1 \supset A_2 \supset ..., $ then $\bigcap_{n \in \mathbb{N}} A_n \in \mathcal{M}$.
\end{enumerate}
\begin{lemma}[Monotone class lemma]
	Let $\alg$ be an algebra. Then $\mathcal{M}(\alg) = \sigma(\alg)$.
\end{lemma}

\begin{proof}
	Every $\sigma$-algebra is a monotone class, so the smallest monotone class containing $\alg$ is inside the smallest $\sigma$-algebra containing $\alg$. Thus it suffices to show that $\mono$ is a $\sigma$-algebra.
	
	Let $\mono_0$ be the subset of $\mono(\alg)$ of all $F \in \mono(\alg)$ where $F^c \in \mono(\alg)$. Then $\alg \in \mono_0$, and $\mono_0$ is a monotone class from De Morgan's law. $(\bigcup_{n \in \mathbb{N}} A_n)^c = \bigcap_{n \in \mathbb{N}} A_n^c$ thus the complement is contained in $\mono_0$. Therefore, $\mono_0 = \mono(\alg)$ from definition.
	\par
	... Thus $\mono(A) = \sigma(A)$.
\end{proof}

\subsection{Measures on algebras}
Let $\alg$ be an algebra on $X$. Then $\mu : \alg \rightarrow [0, \infty]$ is a measure if:

\begin{enumerate}
	\item $\mu(\emptyset) = 0$
	\item If $\lbrace A_n \rbrace_{n \in \mathbb{N}} \subset \alg$ is a pw disjoint family with $\cup_{n \in \mathbb{N}} A_n \in \alg$, then
	\begin{equation}
		\mu \left( \bigcup_{n \in \mathbb{N}} A_n\right) = \sum_{n \in \mathbb{N}} \mu(A_n).
	\end{equation}
	This is known as $\sigma$-additivity.
\end{enumerate}

\subsection{Extension of measures.}
Since algebras and $\sigma$-algebras are not closed under subsets, then we do not get that if $M \subset N$ and $\mu(N) = 0$, then $\mu(M) = 0$ as it may be the case that $M$ is not in $\alg$. Thus, we want to complete the measure in some formal sense.

\begin{definition}
	If $A \in 2^X$, then let
	\begin{equation}
		\mu^*(A) := \inf \left\lbrace \sum_{j \in \mathbb{N}} \mu(E_j) : A \subset \bigcup_{j \in \mathbb{N}} E_j, E_j \in \alg \right\rbrace
	\end{equation}
\end{definition} 
Note that $\mu^*(A) \leq \mu(X)$, as $X$ is in $\alg$. We set $\mu^*(A) = \infty$ if no cover exists with finite summation.

\begin{lemma}
	We have that:
	\begin{enumerate}
		\item $\mu^*(\emptyset) = 0$
		\item $\mu^*(A) \geq 0$ for all $A \in 2^X$
		\item $\mu^*(A) \leq \mu^*(B)$ when $A \subset B$. If $\lbrace E_j \rbrace_{j \in \mathbb{N}}$ is a cover of $B$, then it is a cover of $A$.
		\item $\mu^*(A) = \mu(A)$ when $A \in \alg$. 
		\item If $\lbrace A_n \rbrace_{n \in \mathbb{N}}$ is a sequence of subsets of $X$, then
		\begin{equation}
			\mu^*\left(\bigcup_{n \in \mathbb{N}} A_n \right) \leq \sum_{n \in \mathbb{N}} \mu^*(A_n).
		\end{equation}
	\end{enumerate}
\end{lemma}
The last property states that $\mu^*$ is $\sigma$-subadditive.

\begin{proof}[Item 4]
	We have that $\mu^*(A) \leq \mu(A)= \mu(A)$ as $A$ is a cover of itself. However, consider any cover of $A$ in $\alg$. Note that $A = \bigcup_{j \in \mathbb{N}}(A_j \cap A)$, meaning that $\mu(A) \leq \sum_{j \in \mathbb{N}} \mu(A \cap A_j) \leq \sum_{j \in \mathbb{N}} \mu(A_j)$
	Thus $\mu(A) \leq \sum_{j \in \mathbb{N}} \mu(A_j)$ for any cover $\lbrace A_j \rbrace_{j \in \mathbb{N}}$. 
\end{proof}

\begin{proof}[Subadditivity]
	Let $\varepsilon > 0$. Then for each $n \in \mathbb{N}$, let $\lbrace A_{n_j} \rbrace_{j \in \mathbb{N}}$ be a sequence where $A_n \subset \bigcap_{j \in \mathbb{N}}  A_{n_j}$ and $\sum_{j \in \mathbb{N}}\mu(A_{n_j}) \leq \mu(A_n) + 2^{-n} \varepsilon$. We have that the size of the covering sequence gets exponentially smaller for each element. 
	
	Then we have that 
	\begin{equation}
		\mu^* \left(\bigcup_{n \in \mathbb{N}}A_n \right) \leq \sum_{n \in \mathbb{N}} \sum_{j \in \mathbb{N}} \mu(A_{n_j}) \leq \sum_{n \in \mathbb{N}}\mu^*(A_n) + \varepsilon
	\end{equation}
\end{proof}

\subsubsection{Caratheodory's theorem}
We have that $E \subset X$ is called $\mu^*$-measurable whenever:
$\mu^*(A) = \mu^*(A \cap E) + \mu^*(A \cap E^C)$ for all $A \subseteq X$. The collection of all $\mu^*$-measurable subsets of $X$ is called $\sigmes$. It contains $\alg$.
Note that $ \alg$ is a subset of $\sigmes$. Let $A$ be in $\alg$ and let $B \subset X$. Then we have that $\mu^*(B) = \sum_{n \in \mathbb{N}}\mu(A_n)$ for some collection of $A_n \in \alg$. However, we can split apart $A_n = (A_n \cap A) \cup (A_n \cap A^c)$ such that
\begin{equation}
	\mu^*(B) = \sum_{n \in \mathbb{N}}\mu(A_n\cap A \cup A_n \cap A^c) = \sum_{n \in \mathbb{N}} \mu(A_n \cap A) + \sum_{n \in \mathbb{N}} \mu(A_n \cap A^c) \geq \mu^*(A \cap B) + \mu^*(A^c \cap B)
\end{equation}
 Finally, $\mu^*(A \cap B) + \mu^*(A^c \cap B) \geq \mu^*((A \cap B) \cup (A^c \cap B)) = \mu^*(A \cap (E \cup E^c)) = \mu^*(A)$. 

What sets are $\mu^*$-measurable?
\begin{itemize}
	\item Sets in $\alg$
	\item Null sets
\end{itemize}
\end{document}
