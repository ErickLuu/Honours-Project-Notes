\documentclass{article}
\usepackage[margin=0.5in]{geometry}
\usepackage{amsmath}
\usepackage{amssymb}
\usepackage{amsthm}
\usepackage{url}

% Environments

\newtheorem{theorem}{Theorem}
\newtheorem{proposition}[theorem]{Proposition}
\newtheorem{corollary}[theorem]{Corollary}
\newtheorem{lemma}[theorem]{Lemma}
\newtheorem{definition}[theorem]{Definition}
\newtheorem{conjecture}[theorem]{Conjecture}

\theoremstyle{definition}
\newtheorem{example}[theorem]{Example}

\numberwithin{theorem}{section}
\numberwithin{equation}{section}

\newcommand{\alg}{\mathcal{A}}
\newcommand{\sig}{\mathcal{S}}
\newcommand{\mono}{\mathcal{M}}
\newcommand{\sigmes}{\Lambda_{\mu^*}}
\newcommand{\salg}{$\sigma$-algebra}
%opening
\title{Assignment 1}
\author{Eric Luu}

\begin{document}

\maketitle
\section{Part 1}
\subsection{Problem 1}
\paragraph{Extension of condition to all open sets}
From topology, any open set $U$ can be written as $\bigcup_{n \in \mathbb{N}} J_n$ for some disjoint family of open intervals $\lbrace J_n \rbrace_{n \in \mathbb{N}}$ in $\mathbb{R}$. As $\lambda(A \cap J_n) \leq 1/2 \lambda(J_n)$, we have from $A$ being Lebesgue measurable and the $\sigma$-additivity of $\lambda$ that $ \lambda(A \cap U) = \lambda(A \cap \bigcup_{n \in \mathbb{N}} J_n)= \lambda(\bigcup_{n \in \mathbb{N}} (A \cap J_n)) =\sum_{n \in \mathbb{N}} \lambda(A \cap J_n) \leq 1/2 \sum \lambda(J_n) = 1/2 \lambda(\bigcup_{n \in \mathbb{N}} J_n) = 1/2 \lambda(U)$. Thus shown that we can extend the condition to any open set in $\mathbb{R}$. 
\paragraph{Proof}
First, $\lambda(A)$ is a defined value as $A$ is Lebesgue-measurable. We use the regularity of $A$ to prove $A$ is a null-set. As $A$ is Lebesgue-measurable, then for all $\varepsilon > 0$ there exists an open set $U$ such that $A \subseteq U$ and $\lambda(U\setminus A) \leq \epsilon$. However, we have as $A$ is Lebesgue-measurable and as the Lebesgue measure is an outer-measure, then $\lambda(U) = \lambda(U \cap A) + \lambda(U \cap A^c) = \lambda(U \cap A) + \lambda(U \setminus A)$, as $A \subseteq U$. However we have that $\lambda(U \cap A) \leq 1/2 \lambda(U)$, thus $\lambda(U \setminus A)) \geq 1/2 \lambda(U)$. But we have that $\varepsilon> \lambda(U \setminus A)) \geq 1/2 \lambda(U)$. As we can find a $U$ for every $\varepsilon> 0$ such that $A \subseteq U$ and $\lambda(U) < 2 \varepsilon$, then by the monotonicity of $\lambda$, $\lambda(A) \leq \lambda(U) < 2 \varepsilon$ for all $\varepsilon > 0$. Thus $\lambda(A) = 0$, therefore $A$ is a null-set.  
\subsection{Problem 2}
Consider the function $f: \mathbb{R} \rightarrow \mathbb{R}$ where $f(a) = \lambda((-\infty, a) \cap E)$. We will show that $f$ is continuous everywhere.

\subsection{Problem 3}
We have that $f(n) = \underbrace{f(1) + f(1) + f(1) ... + f(1)}_{n \text{ times}} = n f(1) $, therefore the function is linear when restricted to $\mathbb{N}$. We additionally have that $f(0) = f(0) + f(0)$, thus $f(0) = 0$. We also have that $f(1 - 1) = f(1) + f(-1) = f(0) = 0$, therefore $f(-1) = - f(1)$. Thus $f(x) = x f(1)$ for all $x \in \mathbb{Z}$. Now consider two nonzero integers $p$, $q$. We have that $q f(p/q) = \underbrace{f(p/q) + f(p/q) + ... + f(p/q)}_{q \text{ times}} = f(p) = p f(1)$, therefore $f(p/q) = p/q f(1)$, so for all rational numbers $a \in \mathbb{Q}$, $f(a) = a f(1)$. 
\paragraph{Continuity everywhere}
Now this function is linear when restricted to $\mathbb{Q}$, and we have that at $x = 0$, the function is continuous. We want to have that the function is continuous everywhere. Let $b \in \mathbb{R}$. Then we have for all $x \in \mathbb{R}$, $f(x) = f(x - b) + f(b)$. We have that $f(x - b)$ is continuous at $x - b$, and as $f(b)$ is constant, $f(x) = f(x - b) + f(b)$ is also continuous at $b$. Thus $f$ is continuous everywhere.
\paragraph{$f(x) = f(1) x$}
We have that the rational numbers are everywhere dense and $f$ is continuous. Therefore, we have that for any real number $x$, there exists a sequence of rational numbers $\lbrace x_n \rbrace_{n \in \mathbb{N}}$ such that $f(x_n) = x_n f(1)$ that appproach $x$ in the limit. Therefore as $f$ is continuous everywhere, then it must hold that $f(x) = x f(1)$ for all $x$. Therefore, $f(x) = f(1) x$ for all $x \in \mathbb{R}$. 
\end{document}
