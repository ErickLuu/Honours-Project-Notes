\documentclass{article}
\usepackage[margin=0.5in]{geometry}
\usepackage{amsmath}
\usepackage{amssymb}
\usepackage{amsthm}
\usepackage{url}

% Environments

\newtheorem{theorem}{Theorem}
\newtheorem{proposition}[theorem]{Proposition}
\newtheorem{corollary}[theorem]{Corollary}
\newtheorem{lemma}[theorem]{Lemma}
\newtheorem{definition}[theorem]{Definition}
\newtheorem{conjecture}[theorem]{Conjecture}

\theoremstyle{definition}
\newtheorem{example}[theorem]{Example}

\numberwithin{theorem}{section}
\numberwithin{equation}{section}

\newcommand{\alg}{\mathcal{A}}
\newcommand{\sig}{\mathcal{S}}
\newcommand{\mono}{\mathcal{M}}
\newcommand{\sigmes}{\Lambda_{\mu^*}}
\newcommand{\salg}{$\sigma$-algebra}
%opening
\title{Assignment 1}
\author{Eric Luu}

\begin{document}

\maketitle
\section{Part 1}
\subsection{Problem 1}
\paragraph{Extension of condition to all open sets}
From topology, any open set $U$ can be written as $\bigcup_{n \in \mathbb{N}} J_n$ for some disjoint family of open intervals $\lbrace J_n \rbrace_{n \in \mathbb{N}}$ in $\mathbb{R}$. As $\lambda(A \cap J_n) \leq 1/2 \lambda(J_n)$, we have from $A$ being Lebesgue measurable and the $\sigma$-additivity of $\lambda$ that $ \lambda(A \cap U) = \lambda(A \cap \bigcup_{n \in \mathbb{N}} J_n)= \lambda(\bigcup_{n \in \mathbb{N}} (A \cap J_n)) =\sum_{n \in \mathbb{N}} \lambda(A \cap J_n) \leq 1/2 \sum \lambda(J_n) = 1/2 \lambda(\bigcup_{n \in \mathbb{N}} J_n) = 1/2 \lambda(U)$. Thus shown that we can extend the condition to any open set in $\mathbb{R}$. 
\paragraph{Proof}
First, $\lambda(A)$ is a defined value as $A$ is Lebesgue-measurable. We use the regularity of $A$ to prove $A$ is a null-set. As $A$ is Lebesgue-measurable, then for all $\varepsilon > 0$ there exists an open set $U$ such that $A \subseteq U$ and $\lambda(U\setminus A) \leq \epsilon$. However, we have as $A$ is Lebesgue-measurable and as the Lebesgue measure is an outer-measure, then $\lambda(U) = \lambda(U \cap A) + \lambda(U \cap A^c) = \lambda(U \cap A) + \lambda(U \setminus A)$, as $A \subseteq U$. However we have that $\lambda(U \cap A) \leq 1/2 \lambda(U)$, thus $\lambda(U \setminus A)) \geq 1/2 \lambda(U)$. But we have that $\varepsilon> \lambda(U \setminus A)) \geq 1/2 \lambda(U)$. As we can find a $U$ for every $\varepsilon> 0$ such that $A \subseteq U$ and $\lambda(U) < 2 \varepsilon$, then by the monotonicity of $\lambda$, $\lambda(A) \leq \lambda(U) < 2 \varepsilon$ for all $\varepsilon > 0$. Thus $\lambda(A) = 0$, therefore $A$ is a null-set.  
\subsection{Problem 2}
Consider the function $f: \mathbb{R} \rightarrow \mathbb{R}$ where $f(a) = \lambda((-\infty, a) \cap E)$. We will show that $f$ is continuous everywhere. Let $x$ be a point in $\mathbb{R}$. Then as we have that $(-\infty, x)$ is open, thus Lebesgue measurable, it holds that $(-\infty, x) \cap E$ is also Lebesgue measurable and by the monotonicity of the outer-measure, $\lambda((-\infty, a) \cap E) \leq \lambda(E) = 1$, therefore $f$ is defined at all points in $\mathbb{R}$. 
\paragraph{Continuity}
Then to show that $f$ is continuous everywhere, fix $x_0$ and let $L = f(x_0)$. Fix $\varepsilon > 0$. We will show that every point $x \in (x_0 - \varepsilon, x_0 + \varepsilon)$ will have that $|f(x) - L | < \varepsilon$. 
\paragraph{$x \leq x_0$}
We have that if $x \leq x_0$, then we have that the set $(x, x_0)$ is an interval (or null-set) and thus is Lebesgue-measurable. Therefore by the additivity of $(x, x_0)$, $\lambda((-\infty, x_0) \cap E) = \lambda((-\infty, x_0) \cap E \cap (x, x_0)) + \lambda((-\infty, x_0) \cap E \cap (x, x_0)^c) = \lambda(E \cap (x, x_0)) + \lambda(E \cap (-\infty, x))$. However, we have that $\lambda(E \cap (x, x_0)) < \lambda((x, x_0)) < \varepsilon$ by the definition of the Lebesgue measure of an interval. Therefore, $\lambda((-\infty, x_0) \cap E) - \lambda(E \cap (-\infty, x)) = L - f(x) = \lambda(E \cap (x, x_0)) < \varepsilon$ for all $x \leq x_0$ in the interval defined.
\paragraph{$x \geq x_0$}
If $x \geq x_0$, then we have that the set $(x_0, x)$ is an interval (or null-set) and thus is Lebesgue-measurable. Therefore by the additivity of $(x_0, x)$, $\lambda((-\infty, x) \cap E) =  \lambda((-\infty, x) \cap E \cap (x_0, x)) + \lambda((-\infty, x) \cap E \cap (x_0, x)^c) = \lambda(E \cap (x_0, x)) + \lambda((-\infty, x_0) \cap E)$. But we have that $\lambda(E \cap (x_0, x)) \leq \lambda(x_0, x) < \varepsilon$, therefore $f(x) - L = \lambda(E \cap (x_0, x)) < \varepsilon$. Thus we have that $|f(x) - L| \leq \varepsilon$ for all $x \in (x_0 - \varepsilon, x_0 + \varepsilon)$. Thus $f$ is continuous at $x_0$. As $x_0$ is arbitrary, $f$ is continuous everywhere.
\paragraph{Final steps}
We have that as $x \rightarrow -\infty$, $f(x) = \lambda((-\infty, x) \cap E)$ becomes $\lambda(\emptyset \cap E) = 0$ in the limit, and as $x \rightarrow \infty$, $f(x) = \lambda((-\infty, x) \cap E) = \lambda(\mathbb{R} \cap E) = \lambda(E) = 1$ in the limit. As the function is continuous, there exists an $x$ such that $f(x) = 1/2$, from the intermediate value theorem. Therefore, $\lambda((-\infty, x) \cap E) = 1/2$, so $F = (-\infty, x) \cap E$ is the set required.

\subsection{Problem 3}
\subsubsection{Part a}
We have that $f(n) = \underbrace{f(1) + f(1) + f(1) ... + f(1)}_{n \text{ times}} = n f(1) $, therefore the function is linear when restricted to $\mathbb{N}$. We additionally have that $f(0) = f(0) + f(0)$, thus $f(0) = 0$. We also have that $f(1 - 1) = f(1) + f(-1) = f(0) = 0$, therefore $f(-1) = - f(1)$. Thus $f(x) = x f(1)$ for all $x \in \mathbb{Z}$. Now consider two nonzero integers $p$, $q$. We have that $q f(p/q) = \underbrace{f(p/q) + f(p/q) + ... + f(p/q)}_{q \text{ times}} = f(p) = p f(1)$, therefore $f(p/q) = p/q f(1)$, so for all rational numbers $a \in \mathbb{Q}$, $f(a) = a f(1)$. 
\paragraph{Continuity everywhere}
Now this function is linear when restricted to $\mathbb{Q}$, and we have that at $x = 0$, the function is continuous. We want to have that the function is continuous everywhere. Let $b \in \mathbb{R}$. Then we have for all $x \in \mathbb{R}$, $f(x) = f(x - b) + f(b)$. We have that $f(x - b)$ is continuous at $x - b$, and as $f(b)$ is constant, $f(x) = f(x - b) + f(b)$ is also continuous at $b$. Thus $f$ is continuous everywhere.
\paragraph{$f(x) = f(1) x$}
We have that the rational numbers are everywhere dense and $f$ is continuous. Therefore, we have that for any real number $x$, there exists a sequence of rational numbers $\lbrace x_n \rbrace_{n \in \mathbb{N}}$ such that $f(x_n) = x_n f(1)$ that approach $x$ in the limit. Therefore as $f$ is continuous at $x$, then it must hold that $f(x) = x f(1)$ for all $x$. Therefore, $f(x) = f(1) x$ for all $x \in \mathbb{R}$. 

\subsubsection{Part b}
We have that for all $a \in \mathbb{Q}$, $f(a) = a f(1)$, from part $a$. It remains to show that $f$ is continuous around $0$. We have that $f^{-1}([0, \varepsilon]) = \lbrace x \in \mathbb{R}: f(x) \leq \varepsilon \rbrace$ is Lebesgue-measurable as $f$ is measurable.
\end{document}
