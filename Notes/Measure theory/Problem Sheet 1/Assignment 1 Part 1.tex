\documentclass{article}
\usepackage[margin=0.5in]{geometry}
\usepackage{amsmath}
\usepackage{amssymb}
\usepackage{amsthm}
\usepackage{url}

% Environments

\newtheorem{theorem}{Theorem}
\newtheorem{proposition}[theorem]{Proposition}
\newtheorem{corollary}[theorem]{Corollary}
\newtheorem{lemma}[theorem]{Lemma}
\newtheorem{definition}[theorem]{Definition}
\newtheorem{conjecture}[theorem]{Conjecture}

\theoremstyle{definition}
\newtheorem{example}[theorem]{Example}

\numberwithin{theorem}{section}
\numberwithin{equation}{section}

\newcommand{\alg}{\mathcal{A}}
\newcommand{\sig}{\mathcal{S}}
\newcommand{\mono}{\mathcal{M}}
\newcommand{\sigmes}{\Lambda_{\mu^*}}
\newcommand{\salg}{$\sigma$-algebra}
%opening
\title{Assignment 1}
\author{Eric Luu}

\begin{document}

\maketitle
\section{Part 1}
\subsection{Problem 1}
\newcommand{\Erel}{\mathcal{S}_E}
\subsubsection{a}
\paragraph{\salg}
We shall show that $\Erel$ is a \salg. 
\par
We have that $\emptyset = E \cap \emptyset$, thus $\emptyset \in \Erel$. 
\par
If $S \in \Erel$, then $S = E \cap A$ for some $A \in \sig$. However, we have that $X - A \in \sig$ by the definition of a \salg. But that means that $E \cap (X - A) \in \Erel$, but $E \cap (X - A) = (E \cap X) - (E \cap A) = E - S \in \Erel$. Thus $S^c = E - S \in \Erel$. 
\par
If $S = \bigcup_{n \in \mathbb{N}} S_n$, then $S = \bigcup_{n \in \mathbb{N}} E \cap A_n$ for some $A_n$. But we have that $ \bigcup_{n \in \mathbb{N}} E \cap A_n = E \cap (\bigcup_{n \in \mathbb{N}}A_n)$. But as $\sig$ is a \salg, then $\bigcup_{n \in \mathbb{N}}A_n \in \sig$. Therefore, $S = E \cap \bigcup_{n \in \mathbb{N}}A_n$ is in $\Erel$. 
\paragraph{Measure}
To show $\nu$ is a measure, we have that $\nu(\emptyset) = \mu^*|_{\Erel}(\emptyset) = \mu^*(\emptyset) = 0$, as $\emptyset \subset \Erel$. 
\paragraph{Countable disjoint sets}
Suppose $\left\lbrace S_n \right\rbrace_{n \in \mathbb{N}}$ is a family of pairwise disjoint sets in $\Erel$. Then we have that $S_n = E \cap A_n$ for some $A_n \in \mathbb{N}$. Now define a new set $B_i = A_i \cap (\bigcup_{j \in \mathbb{N}, j \neq i} A_j)^c$, which is in $\sig$ for all $i$ as $\bigcup_{j \in \mathbb{N}, j \neq i} A_j$ is a countable set. Therefore, $\bigcup_{j \in \mathbb{N}, j \neq i} A_j$ is in $\sig$ thus its complement and the intersection with $A_i$ is in $\sig$. However, we have that each $S_i$ is pairwise disjoint in $E$, thus it is pairwise disjoint in $\sig$. Therefore, $S_i = B_i \cap E$ as each point in $S_i$ appears only in $B_i$ and nowhere else, so it is not removed. Finally, we have that $B_i$ is disjoint from $B_j$ for all $i \neq j$ as we remove any points in $A_i$ common to $A_j$ in the construction. 
\par
Therefore, we have that $B_i$ is pairwise disjoint and in $\sig$, and contains $S_i$. Therefore, we have that:
\begin{align*}
	\nu(\bigcup_{n \in \mathbb{N}} S_n) &= \nu(\bigcup_{n \in \mathbb{N}}(E \cap B_n))\\
	&= \nu(E \cap \bigcup_{n \in \mathbb{N}}( B_n))\\
	&= \mu^*|_{\Erel}(E \cap \bigcup_{n \in \mathbb{N}}( B_n))\\
	&= \mu^*|_{\Erel}(E \cap \bigcup_{n \in \mathbb{N}}( B_n)) + \mu^*|_{\Erel}(E^c \cap \bigcup_{n \in \mathbb{N}}( B_n))
\end{align*}
We use the fact that $\mu^*|_{\Erel}(E^c \cap \bigcup_{n \in \mathbb{N}}( B_n)) =  \mu^*(\emptyset) = 0$. We now use that $E$ is $\mu^*$-measurable and nonempty, so $\mu^*|_{\Erel}(A) = \mu^*|_{\Erel}(E \cap A) + \mu^*|_{\Erel}(E^c \cap A)$. So we get that this is equal to 
\begin{align*}
	&= \mu^*|_{\Erel}(\bigcup_{n \in \mathbb{N}}( B_n))\\
	&= \sum_{n \in \mathbb{N}} \mu^*|_{\Erel}(B_n)
\end{align*}
where we use the fact that $B_n$ is pairwise disjoint in $\sig$, so $\mu^*$ and thus its restriction is additive. Finally, we have that $\sum_{n \in \mathbb{N}} \mu^*|_{\Erel}(B_n) = \sum_{n \in \mathbb{N}}\nu (E \cap B_n) = \sum_{n \in \mathbb{N}}\nu(S_n)$, which is the desired result. 

\subsubsection{b}
\paragraph{Proof that $\mu^*|_{2^E}(S) \leq \nu^*(S)$:}
Let $S \subset E$. Then for all $\varepsilon > 0$, there exists a cover of $S$ , $\lbrace S_n \rbrace_{n \in \mathbb{N}}$ in $S^E$ such that $\sum_{n \in \mathbb{N}} \nu(S_n) \leq \nu^*(S) + \varepsilon$. However, we have that $\sum_{n \in \mathbb{N}} \nu(S_n) = \sum_{n \in \mathbb{N}} \mu^*|_{\Erel}(B_n)$ for some $B_n$. However, $ \mu^*|_{\Erel}(B_n) = \mu^*|_{2^E}(B_n)$ for all $B_n$ as $B_n \cap E$ has a well-defined value in $\mu^*|_{\Erel}(B_n)$, thus the extension to $2^E$ must also be well-defined. However, note that $\mu^*|_{2^E}(S) \leq \sum_{n \in \mathbb{N}} \mu^*|_{2^E}(B_n)$. Thus, $\mu^*|_{2^E}(S) \leq \nu^*(S) + \varepsilon$ for all $\varepsilon > 0$, thus $ \mu^*|_{2^E}(S) \leq \nu^*(S) $.
\paragraph{Proof that $\mu^*|_{2^E}(S) \geq \nu^*(S)$:}
For all $\varepsilon > 0$, let $\lbrace B_n \rbrace_{n \in \mathbb{N}}$ be a cover of $S$ in $\sig$ such that $\sum_{n \in \mathbb{N}}\mu^*|_{2^E}(B_n) \leq \mu^*|_{2^E}(S) + \varepsilon$. Then $\mu^*(B_n) = \mu^*(E \cap B_n) + \mu^*(E^c \cap B_n)$, thus $\mu^*|_{2^E}(B_n) = \mu^*(E \cap B_n)$. However, as $S \subset E$, it holds that $ \lbrace E \cap B_n \rbrace_{n \in \mathbb{N}}$ is a cover of $S$, thus $\nu^*(S) \leq \sum_{n \in \mathbb{N}}\nu^*(B_n) = \sum_{n \in \mathbb{N}}\mu^*|_{2^E}(B_n) \leq \mu^*|_{2^E}(S) + \varepsilon$. Thus $\nu^*(S) \leq \mu^*|_{2^E}(S) + \varepsilon$. 
\subsubsection{c}
\paragraph{$\Rightarrow$}
We shall show that if $F \subseteq E$ is $\mu^*$-measurable, then $F$ is $\nu^*$-measurable. Let $G \subseteq E$. Then $\mu^*(G) = \mu^*(G \cap F) + \mu^*(G \cap F^c)$, as $F$ is $\mu^*$-measurable. But as we have that $G \subseteq E$, then $\mu^*(G) = \mu^*|_{2^E}(G)$ as the restriction is well-defined on subsets of $E$. However, we have that $\mu^* (G \cap F^c) = \mu^*(G \cap F^c \cap E) + \mu^*(G \cap F^c \cap E^c)$, therefore restricting $\mu^*$ to subsets of $E$ will have that $\mu^*|_{2^E}(G \cap F^c) = \mu^*|_{2^E}(G \cap F^c \cap E) +  \mu^*|_{2^E}(G \cap F^c \cap E^c) = \mu^*|_{2^E}(G \cap F^c \cap E)$.
\par
Therefore, we have that $\mu^*|_{2^E}(G) = \mu^*|_{2^E}(G \cap F) + \mu^*|_{2^E}(G \cap F^C \cap E)$, therefore $\nu^*(G) = \nu^*(G \cap F) + \nu^*(G \cap (E- F))$. As this holds for arbitrary $G$, then $F$ is $\nu^*$-measurable.
\paragraph{$\Leftarrow$} 
Suppose $F \subseteq E$ and $F$ is $\nu^*$-measurable. Then let $G \subset X$, thus as $E$ is $\mu^*$-measurable, then $\mu^*(G) = \mu^*(G \cap E) + \mu^*(G \cap E^c)$. But $\mu^*(G \cap E) = \mu^*|_{2^E}(G \cap E) = \nu^*(G \cap E)$, therefore, $\mu^*(G \cap E) = \nu^*(F \cap G \cap E) + \nu^*(F^c \cap G \cap E)$. However, both of these sets live in $E$ thus $\mu^*$ agrees on these sets. Therefore, $\mu^*(G \cap E) = \mu^*(F \cap G \cap E) + \mu^*(F^c \cap G \cap E)$. However, we can write $F \cap G \cap E$ as $F \cap G$ and we can also write $G \cap E^c$ as $F^c \cap G \cap E^c$, as $F \subset E$. Therefore, 
\begin{align*}
	\mu^*(G) &= \mu^*(F \cap G) + \mu^*((F^c \cap G) \cap E) + \mu^*((F^c \cap G) \cap E)\\
	\mu^*(G) &= \mu^*(F \cap G) + \mu^*(F^c \cap G)
\end{align*}
where we use the fact that $\mu^*(F^c \cap G) = \mu^*((F^c \cap G) \cap E) + \mu^*((F^c \cap G) \cap E)$ as $E$ is $\mu^*$-measurable. As $G$ is arbitrary, then $F$ is $\mu^*$-measurable. 
\subsection{Problem 2}
\subsubsection{a}
Suppose $A \subset B$, where $B$ is countably infinite and therefore in $\sig$. Then $A$ is either countable infinite, or finite. Thus $A \in \sig$. Furthermore, we have that $\mu(B) = \infty$, and $\mu(A)$ is either $0$ or $\infty$, thus $\mu(A) \leq \mu(B)$. If $B$ is finite, then $A$ must be finite as well, and it is in $\sig$. Thus $\mu(A) = \mu(B)$.
\subsubsection{b}
Suppose no family of $\left\lbrace A_j \right\rbrace_{j = 1}^m$ is countably infinite. Then that implies $\bigcup_{j = 1}^m A_j$ is finite, thus $\mu(\bigcup_{j = 1}^m A_j) = \sum_{j = 1}^m \mu(A_j) = 0$. If at least one of the $A_j$ is countably infinite, then $\mu(\bigcup_{j = 1}^m A_j)$ is also countably infinite. Thus $\mu(\bigcup_{j = 1}^m A_j) = \sum_{j = 1}^m \mu(A_j) = \infty$.
\subsubsection{c}
No, as measures are defined on algebras, but $\sig$ is not an algebra. We have that $\emptyset$ is in $\sig$, but the complement of $\emptyset$, $\mathbb{R}$, is not countable. Thus $\sig$ does not contain $\mathbb{R}$ and therefore it is not closed under complement.
\end{document}
