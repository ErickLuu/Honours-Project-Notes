\documentclass{article}
\usepackage[margin=1in]{geometry}
\usepackage{amsmath}
\usepackage{amssymb}
\usepackage{amsthm}
\usepackage{url}

% Environments

\newtheorem{theorem}{Theorem}
\newtheorem{proposition}[theorem]{Proposition}
\newtheorem{corollary}[theorem]{Corollary}
\newtheorem{lemma}[theorem]{Lemma}
\newtheorem{definition}[theorem]{Definition}
\newtheorem{conjecture}[theorem]{Conjecture}

\theoremstyle{definition}
\newtheorem{example}[theorem]{Example}

\numberwithin{theorem}{section}
\numberwithin{equation}{section}

\newcommand{\alg}{\mathcal{A}}
\newcommand{\sig}{\mathcal{S}}
\newcommand{\mono}{\mathcal{M}}
\newcommand{\sigmes}{\Lambda_{\mu^*}}
\newcommand{\salg}{$\sigma$-algebra}
\newcommand{\intd}{\, d}
%opening
\title{Assignment 1}
\author{Eric Luu}

\begin{document}

\maketitle
\section*{Part 4}
\subsection*{Question 1}
\subsubsection*{a}
Suppose $a$ and $b$ are positive. Let $\lambda > \sqrt[p-1]{1- \varepsilon}$. Then we have that:
\begin{align*}
	| a + b |^p &= | \frac{\lambda- 1}{\lambda} A a + \frac{1}{\lambda}(\lambda b) |^p\\
	&\leq \frac{\lambda- 1}{\lambda} |A a |^p + \lambda^{p-1} |b|^p\\
	&\leq \frac{\lambda- 1}{\lambda} A^p |a|^p + \lambda^{p-1} |b|^p
\end{align*}
where we use the convexity of $t \mapsto |t|^p$ and we let
$A = \frac{\lambda}{\lambda - 1} > 1$. 

Then we subtract $|b|^p$ from both sides to get that:

\begin{align*}
	| a + b |^p - |b|^p &\leq \frac{\lambda- 1}{\lambda} A^p |a|^p + (\lambda^{p-1} - 1) |b|^p\\
	&\leq \frac{\lambda- 1}{\lambda} A^p |a|^p + |1 - \lambda^{p-1}| |b|^p
\end{align*}
but we have that $\varepsilon >1 - \lambda^{p-1}$ as $\lambda > \sqrt[p-1]{1- \varepsilon}$. Therefore, we have that 
\begin{equation*}
	| a + b |^p - |b|^p \leq \frac{\lambda- 1}{\lambda} A^p |a|^p + 
	\varepsilon|b|^p
\end{equation*}
Therefore, we set $C_\varepsilon = \frac{\lambda- 1}{\lambda} A^p $ and 
\begin{equation*}
	| a + b |^p - |b|^p \leq C_\varepsilon |a|^p + 
	\varepsilon|b|^p
\end{equation*}.

\subsubsection*{b}
We have that:
\begin{align*}
	| |f_n|^p - |f_n - f|^p - |f|^p | \leq  | |f_n|^p - |f_n - f|^p | + |f|^p 
\end{align*}
from the triangle inequality.
However, if we let $f(x) = a$ for all $x$, $f_n(x) - f(x) = b$, then we have that from the inequality above, for all $\varepsilon > 0$, that:
\begin{align*}
	 | |f_n|^p - |f_n - f|^p| \leq \varepsilon |f_n - f|^p + C_\varepsilon |f|
\end{align*}

Therefore we have that:
\begin{align*}
	| |f_n|^p - |f_n - f|^p - |f|^p | &\leq  \varepsilon |f_n - f|^p + C_\varepsilon |f|^p + |f|^p \\
	&=  \varepsilon |f_n - f|^p +( C_\varepsilon + 1) |f|^p
\end{align*}

\subsubsection*{c}
We have that \begin{equation}
	\liminf_{n \rightarrow \infty} \int (\varepsilon |f_n - f|^p + (1 + C_\varepsilon)|f|^p - G_n^\varepsilon) \, d\mu \leq \int \liminf_{n \rightarrow \infty} (\varepsilon |f_n - f|^p + (1 + C_\varepsilon)|f|^p -  G_n^\varepsilon) \, d\mu
\end{equation} by Fatou's lemma. But we have that:
\begin{align*}
	 \int \liminf_{n \rightarrow \infty} (\varepsilon |f_n - f|^p + (1 + C_\varepsilon)|f|^p -  G_n^\varepsilon) \, d\mu 
	 &=
	 \int (1 + C_\varepsilon)|f|^p \, d\mu + \liminf (\varepsilon)|f_n - f|^p - G_n^\varepsilon \, d\mu\\
	 &= (1 + C_\varepsilon) C^p + \liminf \int (\varepsilon)|f_n - f|^p - G_n^\varepsilon \, d\mu
\end{align*}

and we also have that
\begin{equation}
	\int (1 + C_\varepsilon)|f|^p \leq	\liminf_{n \rightarrow \infty} \int (\varepsilon |f_n - f|^p + (1 + C_\varepsilon)|f|^p - G_n^\varepsilon) \, d\mu 
\end{equation}
meaning that
$0 \leq \liminf \int (\varepsilon)|f_n - f|^p - G_n^\varepsilon \, d\mu$.
Therefore, 
\begin{align*}
	\limsup \int G_n^\varepsilon \, d\mu &\leq \liminf \int\varepsilon|f_n - f|^p \, d\mu\\
	 &\leq \varepsilon 2^{p + 1} C^p
\end{align*}
But as we let $\varepsilon$ go to 0 and $n$ go to infinity, this means that $0 \limsup_{n \rightarrow \infty} \int G_n^\varepsilon \, d\mu \leq A \varepsilon$ so the limit of $\lim_{n \rightarrow \infty} \int G_n^\varepsilon \, d\mu = 0$.

\subsubsection*{d}
Let $H^\varepsilon_n = (|f_n|^p - |f_n - f|^p - |f|^p)_-$. We have that $H^\varepsilon_n \leq |f_n - f|^p +( C_\varepsilon + 1) |f|^p$ for the same reason as $G^\varepsilon_n$ above, so we have that $\int \lim_{n \rightarrow \infty} H^\varepsilon_n \, dx = 0$ for the same reason. Then we have that $\lim_{n \rightarrow \infty}\int | |f_n|^p - |f_n - f|^p - |f|^p | \, dx = \lim_{n \rightarrow \infty} \int G^\varepsilon_n + H^\varepsilon_n \, dx = 0$. Thus shown.
\subsubsection*{e}
Yes. We have that $| |a + b| - |b|| \leq |a|$ and we use the same logic as above, but $C = 1$. Therefore, we have that $||f_n| - |f_n - f| - |f|| \leq 2 |f|$, then we use the dominated convergence theorem to yield the same result.
\subsubsection*{f}
Fatou's lemma does not give us anything because $\liminf_{n \rightarrow \infty} f_n = f$. 
\subsection*{Question 2}

Let $f_n : [0, \infty) \rightarrow \mathbb{R}$ where
\begin{equation}
	f(x) = \left( 1 + \frac{x}{n}\right)^n e^{-2x}
\end{equation}
We have that $f_n(x) \leq f_{n+1}(x)$. We use the AM-GM inequality to yield that:

\begin{equation}
	\sqrt[n + 1]{(1) \left( 1 + \frac{x}{n}\right)^n} < \frac{1 + n \left( 1 + \frac{x}{n}\right)}{n + 1}
\end{equation}
Therefore,
\begin{equation}
	\left( 1 + \frac{x}{n}\right)^{\frac{n}{n + 1}} < \frac{1 + n + x}{n + 1} = 1 + \frac{x}{n + 1}
\end{equation}
Therefore,
\begin{equation}
	\left( 1 + \frac{x}{n}\right)^n < \left( 1 + \frac{x}{n+1}\right)^{n + 1}
\end{equation}
therefore, we have that these functions are monotonically increasing. In the limit,
$\lim_{n \rightarrow \infty} f_n(x) = e^{x} e^{-2x} = e^{-x}$. 
Therefore, by monotone convergence theorem,
\begin{equation}
	\lim_{n \rightarrow \infty} \int_{0}^\infty f_n(x) \, dx = \int_{0}^\infty \lim_{n \rightarrow \infty} f_n(x) dx = \int_{0}^\infty e^{-x} dx = 1. 
\end{equation}
\subsection*{Question 3}
\subsubsection*{a}
We use lemma 3.13 to yield that:
\begin{equation}
	\| f \chi_X \|_p \leq \| \chi_X \|_p \|f \|_\infty = (\mu(X))^{1/p} \|f \|_{\infty}
\end{equation}

If we take $p \rightarrow \infty$, we get that $(\mu(X))^{1/p} \rightarrow 1$, so 

\begin{equation}
	\lim_{p \rightarrow \infty} \| f \chi_X \|_p \leq \|f \|_{\infty}.
\end{equation}

To show equality, fix $\varepsilon > 0$, and let $C = \lbrace x : |f(x)| \geq \| f \|_{\infty} - \varepsilon \rbrace$. $C$ is measurable as it is a level set. Then we have that $ \|f \|_p = \left(\int_X |f|^p \, dx \right)^{1/p} \geq \left(\int_C |\| f \|_{\infty} - \varepsilon|^p \, dx \right)^{1/p}$. We also have that $\mu(C) > 0$, as if it is a null set, then $\|f \|_{\infty} \leq \sup_{x \in C^c} |f(x)| < \|f\|_{\infty} - \varepsilon$, which is a contradiction of $\|f \|_{\infty}$ being finite. 

But this is constant, thus this is equal to $\mu(C)^{1/p}( \| f \|_{\infty} - \varepsilon)$, and as $\mu(X)$ is finite, then $\mu(C)$ is also finite and thus in the limit $\mu(C)^{1/p}$ approaches 1. Therefore, we have that $\lim_{p \rightarrow \infty} \| f \chi_X \|_p \geq \|f \|_{\infty} - \varepsilon$ for all $\varepsilon > 0$, therefore $\lim_{p \rightarrow \infty} \| f \chi_X \|_p \geq \|f \|_{\infty}$. Therefore we have equality.
\subsubsection*{b}
No. Consider the Lebesgue measure space $(\mathbb{R}, Vol^*, \lambda)$ and $f : \mathbb{R} \rightarrow \mathbb{R}$, $f(x) = 1$. Then $f(x)$ is in no $L^p$ space but it is in $L^\infty$. Thus this does not hold if $(X, \mu)$ is not finite, or even $\sigma$-finite. 
\subsection*{Question 4}
\subsubsection*{Part a}
We have that the function $g: \mathbb{R} \rightarrow \mathbb{R}$, $g(x) = \sqrt{1 + x^2}$ is convex as $g''(x) = (1 + x^2)^{-3/2}$, which is greater than 0 for all $x$. 

Therefore by Jensen's inequality we have that:

\begin{equation}
	\sqrt{1 + \left(\frac{1}{\mu(X)} \int_X f \, d\mu\right)^2} \leq \frac{1}{\mu(X)} \int_X \left(1 + f^2\right)^{1/2} \, dx
\end{equation}
However, we have that $\mu(X) = 1$, so 
\begin{equation}
	\sqrt{1 + \left( \int_X f \, d\mu\right)^2} \leq \int_X \left(1 + f^2\right)^{1/2} \, dx
\end{equation}
Therefore, substituting $A := \int_X f \, dx$, we have that:
\begin{equation}
	\sqrt{1 + A^2} \leq \int_X \left(1 + f^2\right)^{1/2} \, dx
\end{equation}
To have the right inequality, consider all $x$ non-negative. We have that $(1 + x^2) \leq (1 + x^2 + 2x)$ for all non-negative $x$ and as the $\sqrt{ \cdot }$ function is monotone increasing, it holds that
\begin{equation}
	\sqrt{1 + x^2} \leq \sqrt{1 + x^2 + 2x} = \sqrt{(1 + x)^2} = 1 + x
\end{equation}
Therefore, $\sqrt{1 + f(x)^2} \leq 1 + f(x)$ pointwise, so as integration respects inequalities, then:

\begin{equation}
\int_X \left(1 + f^2\right)^{1/2} \, dx \leq \int_X 1 + f(x) \, dx
\end{equation}
and as $\int_X 1 \, dx = \mu(X) = 1$, then 
\begin{equation}
	\int_X \left(1 + f^2\right)^{1/2} \, dx \leq 1 + A
\end{equation}
\subsubsection*{Part b}
Consider $F(x) = 2/3 x^{3/2}$. Then $f = F' = x^{1/2}$. Then $A := \int_{[0,1]} f(x) \, dx =  F(1) - F(0) = 2/3$. 

We have that
\begin{equation}
	\int_{[0,1]} \sqrt{1 + x} \, dx = \frac{2}{3} (2)^{3/2} - \frac{2}{3} = 1.22
\end{equation}
We also have that $\sqrt{1 + A^2} = \sqrt{1 + \frac{2}{3}^2} = 1.20$ and $1 + A = 1 + 2/3 = 1.33$. Therefore, 

\begin{equation}
	\sqrt{1 + A^2} \leq \int_{[0,1]} \sqrt{1 + x} \, dx \leq 1 + A
\end{equation}
as needed.
\end{document}