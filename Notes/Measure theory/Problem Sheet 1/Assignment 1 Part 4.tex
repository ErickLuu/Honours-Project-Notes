\documentclass{article}
\usepackage[margin=1in]{geometry}
\usepackage{amsmath}
\usepackage{amssymb}
\usepackage{amsthm}
\usepackage{url}

% Environments

\newtheorem{theorem}{Theorem}
\newtheorem{proposition}[theorem]{Proposition}
\newtheorem{corollary}[theorem]{Corollary}
\newtheorem{lemma}[theorem]{Lemma}
\newtheorem{definition}[theorem]{Definition}
\newtheorem{conjecture}[theorem]{Conjecture}

\theoremstyle{definition}
\newtheorem{example}[theorem]{Example}

\numberwithin{theorem}{section}
\numberwithin{equation}{section}

\newcommand{\alg}{\mathcal{A}}
\newcommand{\sig}{\mathcal{S}}
\newcommand{\mono}{\mathcal{M}}
\newcommand{\sigmes}{\Lambda_{\mu^*}}
\newcommand{\salg}{$\sigma$-algebra}
\newcommand{\intd}{\, d}
%opening
\title{Assignment 1}
\author{Eric Luu}

\begin{document}

\maketitle
\section*{Part 4}
\subsection*{Question 1}

\subsubsection*{a}

\subsubsection*{b}

\subsubsection*{c}

\subsubsection*{d}

\subsubsection*{e}

\subsubsection*{f}

\subsection*{Question 2}

Let $f_n : [0, \infty) \rightarrow \mathbb{R}$ where
\begin{equation}
	f(x) = \left( 1 + \frac{x}{n}\right)^n e^{-2x}
\end{equation}
We have that $f_n(x) \leq f_{n+1}(x)$. We use the AM-GM inequality to yield that:

\begin{equation}
	\sqrt[n + 1]{(1) \left( 1 + \frac{x}{n}\right)^n} < \frac{1 + n \left( 1 + \frac{x}{n}\right)}{n + 1}
\end{equation}
Therefore,
\begin{equation}
	\left( 1 + \frac{x}{n}\right)^{\frac{n}{n + 1}} < \frac{1 + n + x}{n + 1} = 1 + \frac{x}{n + 1}
\end{equation}
Therefore,

\begin{equation}
	\left( 1 + \frac{x}{n}\right)^n < \left( 1 + \frac{x}{n+1}\right)^{n + 1}
\end{equation}
therefore, we have that these functions are monotonically increasing. In the limit,
$\lim_{n \rightarrow \infty} f_n(x) = e^{x} e^{-2x} = e^{-x}$. 
Therefore, by monotone convergence theorem,
\begin{equation}
	\lim_{n \rightarrow \infty} \int_{0}^\infty f_n(x) \, dx = \int_{0}^\infty \lim_{n \rightarrow \infty} f_n(x) dx = \int_{0}^\infty e^{-x} dx = 1. 
\end{equation}
\subsection*{Question 3}
\subsubsection*{a}
We use lemma 3.13 to yield that:
\begin{equation}
	\| f \chi_X \|_p \leq \| \chi_X \|_p \|f \|_\infty = (\mu(X))^{1/p} \|f \|_{\infty}
\end{equation}

If we take $p \rightarrow \infty$, we get that $(\mu(X))^{1/p} \rightarrow 1$, so 

\begin{equation}
	\lim_{p \rightarrow \infty} \| f \chi_X \|_p \leq \|f \|_{\infty}.
\end{equation}

To show equality FIX LATER

\subsubsection*{b}
No. Consider the Lebesgue measure space $(\mathbb{R}, Vol^*, \lambda)$ and $f : \mathbb{R} \rightarrow \mathbb{R}$, $f(x) = 1$. Then $f(x)$ is in no $L^p$ space but it is in $L^\infty$. Thus this does not hold if $(X, \mu)$ is not finite, or even $\sigma$-finite. 
\subsection*{Question 4}
\subsubsection*{Part a}
We have that the function $g: \mathbb{R} \rightarrow \mathbb{R}$, $g(x) = \sqrt{1 + x^2}$ is convex as $x^2$ is convex, $\sqrt{1 + x}$ is concave and nondecreasing, so by function composition $g(x)$ is convex. 

Therefore by Jensen's inequality we have that:

\begin{equation}
	\sqrt{1 + \left(\frac{1}{\mu(X)} \int_X f \, d\mu\right)^2} \leq \frac{1}{\mu(X)} \int_X \left(1 + f^2\right)^{1/2} \, dx
\end{equation}
However, we have that $\mu(X) = 1$, so 
\begin{equation}
	\sqrt{1 + \left( \int_X f \, d\mu\right)^2} \leq \int_X \left(1 + f^2\right)^{1/2} \, dx
\end{equation}
Therefore, substituting $A := \int_X f \, dx$, we have that:
\begin{equation}
	\sqrt{1 + A^2} \leq \int_X \left(1 + f^2\right)^{1/2} \, dx
\end{equation}
To have the right inequality, consider all $x$ non-negative. We have that $(1 + x^2) \leq (1 + x^2 + 2x)$ for all non-negative $x$ and as the $\sqrt{ \cdot }$ function is monotone increasing, it holds that
\begin{equation}
	\sqrt{1 + x^2} \leq \sqrt{1 + x^2 + 2x} = \sqrt{(1 + x)^2} = 1 + x
\end{equation}
Therefore, $\sqrt{1 + f(x)^2} \leq 1 + f(x)$ pointwise, so as integration respects inequalities, then:

\begin{equation}
\int_X \left(1 + f^2\right)^{1/2} \, dx \leq \int_X 1 + f(x) \, dx
\end{equation}
and as $\int_X 1 \, dx = \mu(X) = 1$, then 
\begin{equation}
	\int_X \left(1 + f^2\right)^{1/2} \, dx \leq 1 + A
\end{equation}
\subsubsection*{Part b}
Consider $F(x) = 2/3 x^{3/2}$. Then $f = F' = x^{1/2}$. Then $A := \int_{[0,1]} f(x) \, dx =  F(1) - F(0) = 2/3$. 

We have that
\begin{equation}
	\int_{[0,1]} \sqrt{1 + x} \, dx = \frac{2}{3} (2)^{3/2} - \frac{2}{3} = 1.22
\end{equation}
We also have that $\sqrt{1 + A^2} = \sqrt{1 + \frac{2}{3}^2} = 1.20$ and $1 + A = 1 + 2/3 = 1.33$. Therefore, 

\begin{equation}
	\sqrt{1 + A^2} \leq \int_{[0,1]} \sqrt{1 + x} \, dx \leq 1 + A
\end{equation}
as needed.
\end{document}