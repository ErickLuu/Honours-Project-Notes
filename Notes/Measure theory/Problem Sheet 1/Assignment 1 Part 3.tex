\documentclass{article}
\usepackage[margin=1in]{geometry}
\usepackage{amsmath}
\usepackage{amssymb}
\usepackage{amsthm}
\usepackage{url}

% Environments

\newtheorem{theorem}{Theorem}
\newtheorem{proposition}[theorem]{Proposition}
\newtheorem{corollary}[theorem]{Corollary}
\newtheorem{lemma}[theorem]{Lemma}
\newtheorem{definition}[theorem]{Definition}
\newtheorem{conjecture}[theorem]{Conjecture}

\theoremstyle{definition}
\newtheorem{example}[theorem]{Example}

\numberwithin{theorem}{section}
\numberwithin{equation}{section}

\newcommand{\alg}{\mathcal{A}}
\newcommand{\sig}{\mathcal{S}}
\newcommand{\mono}{\mathcal{M}}
\newcommand{\sigmes}{\Lambda_{\mu^*}}
\newcommand{\salg}{$\sigma$-algebra}
\newcommand{\intd}{\, d}
%opening
\title{Assignment 1}
\author{Eric Luu}

\begin{document}

\maketitle
\section*{Part 3}
\subsection*{Question 1}
We have that if $g = \chi_{[0, 1]}$ and $f = \chi_{[0, 1]}$, then $\int_\mathbb{R} g \intd \mu = \int_\mathbb{R} f \intd \mu = 1$, thus they are Lebesgue-integrable.  

But $f \circ g(x) = f(0)$ or $f(1)$. But $f(0) = 1$, $f(1) = 1$, thus $f \circ g(x) = 1$. But we have that $\int_\mathbb{R} 1 \intd \mu = \int_\mathbb{R} \chi_{\mathbb{R}} \intd \mu = \mu(\chi_{\mathbb{R}})$, which is infinite, thus not Lebesgue-integrable. 
\subsection*{Question 2}
Let $X_n = \lbrace x : f(x) \geq n \rbrace$, and $S_n = \lbrace x : n \leq f(x) < n + 1 \rbrace$. Then we have that $\sum_{n \geq 0} X_n = \sum_{n \geq 0} (n + 1) \mu(S_n)$, and we can use the additivity of $\mu$ to have that $\mu(X_n) = \sum_{k \geq n} S_k$, we count $\mu(S_n)$ $n + 1$ times. We have that
\begin{equation}
	\int f \intd \mu = \sum_{n \geq 0} \int_{S_n} f \intd \mu,
\end{equation} by $S_n$ being a partition of $X$. 
However, we have that 
\begin{equation}
	\sum_{n \geq 0} \int_{S_n} f \intd \mu = \sum_{n \geq 0} \int \chi_{S_n} f \intd \mu =  \int \sum_{n \geq 0}  \chi_{S_n} f \intd \mu
\end{equation} 
by corollary 2.22.  But we have that
\begin{equation}
	\int \sum_{n \geq 0}  \chi_{S_n} f \intd \mu \leq \int \sum_{n \geq 0} (n + 1)\chi_{S_n} \intd \mu = \sum_{n \geq 0} (n + 1)\mu(S_n) = \sum_{n \geq 0} \mu(X_n)
\end{equation}. Therefore, if $\sum_{n \geq 0} X_n$ converges and is finite, then $\int f \intd \mu < \infty$. Thus $f$ is integrable. 
Finally, we have that:
\begin{equation}
	\int \sum_{n \geq 0}  \chi_{S_n} f \intd \mu \geq \int \sum_{n \geq 0} (n)\chi_{S_n} \intd \mu = \sum_{n \geq 0} (n + 1)\mu(S_n) - \mu(S_n) = \sum_{n \geq 0}\mu(X_n) - \mu(X).
\end{equation}
Then we have that $\int f \intd \mu + \mu(X) \geq \sum_{n \geq 0} X_n$, therefore if $f$ is integrable, then $\int f \intd \mu < \infty$, therefore we have that $\sum_{n \geq 0} \mu(X_n)$ is bounded above. Since $\mu(X_n)$ is positive and the summation is bounded above then the summation converges. 

\subsection{Problem 3}
\subsubsection{Part 1}
We have that $|f_n - f| \leq |f_n| + |f|$, so let $g_n := |f_n| + |f| - |f_n - f|$. We have by Fatou's lemma,
\begin{equation}
	\int_X \liminf_{n \rightarrow \infty} g_n \intd \mu \leq \liminf_{n \rightarrow \infty} \int_X g_n \intd \mu.
\end{equation}
However, we have that $f_n$ converges to $f$, therefore $|f_n|$ converges to $|f|$ pointwise everywhere and  $\liminf_{n \rightarrow \infty} f_n$ converges to $f$ pointwise everywhere. Therefore, $\liminf_{n \rightarrow \infty} g_n = 2 |f|$ pointwise everywhere. Additionally, 
\begin{equation}
	\liminf_{n \rightarrow \infty} \int_X g_n \intd \mu = \liminf_{n \rightarrow \infty} \int_X |f_n| + |f| - |f_n - f| \intd \mu = \liminf_{n \rightarrow \infty} \int_X |f_n| \intd \mu + \int_X |f| \intd \mu - 
	\limsup_{n \rightarrow \infty} \int_X |f_n - f| \intd \mu.
\end{equation}
as we have that $\liminf(-x_n) = - \limsup (x_n)$. 
However, we have that $\lim_{n \rightarrow \infty} \int_X |f_n| \intd \mu = \int_X |f| \intd \mu$, so we have that from above that:
\begin{equation}
	\int_X 2 |f| \intd \mu \leq 2 \int_X |f| \intd \mu -  \limsup_{n \rightarrow \infty} \int_X |f_n - f| \intd \mu
\end{equation}
and by using the scalar property of the integral and rearranging,
 $\limsup_{n \rightarrow \infty} \int_X |f_n - f| \intd \mu \leq 0$. We also have that $\limsup_{n \rightarrow \infty} \int_X |f_n - f| \intd \mu$ is bounded above by $0$ and is non-negative, therefore, $\lim_{n \rightarrow \infty} \int_X |f_n - f| \intd \mu = 0$. 
\subsubsection{Part 2}
We have almost the same thing, but we have that
$f_n$ converges to $f$ pointwise almost everywhere, therefore $|f_n|$ converges to $|f|$ pointwise almost everywhere and  $\liminf_{n \rightarrow \infty} f_n$ converges to $f$ pointwise almost everywhere . Therefore, $\liminf_{n \rightarrow \infty} g_n = 2 |f|$ pointwise almost everywhere.

However, this is not an issue, as $\liminf_{n \rightarrow \infty} g_n$ is increasing and non-negative, so by lemma 2.21, we have that $2 \int_X |f| \intd \mu = \liminf \int_X g_n \intd \mu$. Therefore, we can use the same technique above to have that
\begin{equation}
	\lim_{n \rightarrow \infty} \int_X |f_n - f| \intd \mu = 0
\end{equation}
\end{document}