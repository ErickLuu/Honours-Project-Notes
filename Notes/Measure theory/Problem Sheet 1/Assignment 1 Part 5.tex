\documentclass{article}
\usepackage[margin=1in]{geometry}
\usepackage{amsmath}
\usepackage{amssymb}
\usepackage{amsthm}
\usepackage{url}

% Environments

\newtheorem{theorem}{Theorem}
\newtheorem{proposition}[theorem]{Proposition}
\newtheorem{corollary}[theorem]{Corollary}
\newtheorem{lemma}[theorem]{Lemma}
\newtheorem{definition}[theorem]{Definition}
\newtheorem{conjecture}[theorem]{Conjecture}

\theoremstyle{definition}
\newtheorem{example}[theorem]{Example}

\numberwithin{theorem}{section}
\numberwithin{equation}{section}

\newcommand{\alg}{\mathcal{A}}
\newcommand{\sig}{\mathcal{S}}
\newcommand{\mono}{\mathcal{M}}
\newcommand{\sigmes}{\Lambda_{\mu^*}}
\newcommand{\salg}{$\sigma$-algebra}
\newcommand{\intd}{\, d}
%opening
\title{Assignment 1}
\author{Eric Luu}

\begin{document}

\maketitle
\section{Part 5}
\subsection{Question 1}


\begin{equation}
	g(x) := \tilde{g}(x) - \frac{1}{T} \int_{0}^{T} \tilde{g}(t) \, dt
\end{equation}

Then we have that for all $x$,
\begin{align*}
	\int_{x}^{x + T} g(t) \, dt &= \int_{x}^{x + T} \left(\tilde{g}(u) - \frac{1}{T} \int_{0}^{T} \tilde{g}(t) \, dt\right) \, du\\
	&= \int_{x}^{x + T} \tilde{g}(u)\, du - \int_{0}^{T} \tilde{g}(t) \, dt
\end{align*}
However, as $\tilde{g}(x)$ is $T$-periodic, it must hold that integration over any interval of length $T$ will yield the same result. Therefore, we have that
\begin{equation}
	\int_{x}^{x + T} \tilde{g}(u)\, du - \int_{0}^{T} \tilde{g}(t) \, dt = 0
\end{equation}

Now for any pair $p \leq q$ for any real number where $q > p + T$, we have that:
\begin{equation}
	\left| \int_{p}^{q} g(x) \, dx \right| = \left| \int_{p}^{q-T} g(x) \, dx +  \int_{q - T}^{q} g(x) \, dx\right| = \left| \int_{p}^{q-T} g(x) \, dx  \right|
\end{equation}
from above.

Therefore we can iterate this operation to find $q'$, where $\left| \int_{p}^{q} g(x) \, dx \right| = \left| \int_{p}^{q'} g(x) \, dx \right|$ and $p \leq q \leq p + T$. Then we have that from $T$-periodicity,
\begin{equation}
	\left| \int_{p}^{q'} g(x) \, dx \right| = \left| \int_{a}^{b} g(x) \, dx + \int_{c}^{d} g(x) \, dx \right|
\end{equation}
for some $ 0 \leq a \leq b \leq c \leq d \leq T$, by shuffling around the domains. We have that:
\begin{align*}
	\left| \int_{a}^{b} g(x) \, dx + \int_{c}^{d} g(x) \, dx \right| &\leq \left| \int_{a}^{b} g(x) \, dx \right| + \left| \int_{c}^{d} g(x) \, dx \right|\\
	&\leq  \int_{a}^{b} |g(x)| \, dx +\int_{c}^{d} |g(x)| \, dx\\
	&\leq \int_{0}^T |g(x)| \, dx
\end{align*}
But we have that 
\begin{equation}
	|g(x)| \leq \max_{x \in [0, T]} |g(x)|
\end{equation}
pointwise in the interval $[0, T]$, therefore we have that

\begin{equation}
	\left| \int_{p}^{q} g(x) \, dx \right| \leq \int_{0}^T |g(x)| \, dx \leq \int_{0}^T \max_{x \in [0, T]} |g(x)| \, dx = T \max_{x \in [0, T]} |g(x)|.
\end{equation}
Thus shown the lemma above. 

\subsubsection{Proving the integral}
Let $A$ be a measurable set where $\lambda(A) < \infty$ and $\chi_A$ be its associated characteristic function. Then we have that

\begin{equation}
	\int_{\mathbb{R}} \tilde{g}(nx) \chi_A \, dx = \int_{\mathbb{R}} \left(g(nx) + M\right) \chi_A \, dx
\end{equation}
We can split this integral into two sides. 

\begin{equation}
	\int_{\mathbb{R}} \left(g(nx) + M\right) \chi_A \, dx = \int_{\mathbb{R}}  g(nx) \chi_A \, dx + \int_{\mathbb{R}} M \chi_A \, dx = \int_{\mathbb{R}}  g(nx) \chi_A \, dx + M \lambda(A)
\end{equation}
What remains is to show is that
\begin{equation}
	\lim_{n \rightarrow \infty} \int_{\mathbb{R}}  g(nx) \chi_A \, dx = 0. 
\end{equation}

Suppose that $A$ is an interval. Then for all intervals $[p, q]$, we have that:
\begin{equation}
	\lim_{n \rightarrow \infty} \left| \int_{p}^{q} g(nx) \, dx \right| = T \max_{nx \in [0, T]} |g(nx)|.
\end{equation}
We have that $g(nx)$ is $T/n$-periodic, therefore we have that:

\begin{equation}
	\lim_{n \rightarrow \infty} \left| \int_{p}^{q} g(nx) \, dx \right| =\lim_{n \rightarrow \infty}  T/n \max_{x \in [0, T/n]} |g(x)|.
\end{equation}
However, $|g(x)|$ is bounded, suppose $|g(x)| \leq C$. Therefore we have that 
\begin{equation}
	\lim_{n \rightarrow \infty} \frac{T}{n} \max_{x \in [0, T/n]} |g(x)| \leq \lim_{n \rightarrow \infty} \frac{TC}{n} = 0.
\end{equation}

Now let $A$ be measurable with finite Lebesgue measure. Then we can approximate $A$ with a family of compact sets $K_t = \cup_{i \in \mathbb{N}} [p_i, q_i]$ and $K_t \subseteq A$ such that $\lambda(A \setminus K_t) < 1/t, t \in \mathbb{N}$, from 1.30. We can also impose that $K_1 \subseteq K_2 \subseteq K_3 ... \subseteq A$.
Then we have that from MCT, we have that

\begin{equation}
	\lim_{t \rightarrow \infty} \lim_{n \rightarrow \infty} \int_{\mathbb{R}} g(nx) \chi_{K_t} \, dx = \lim_{n \rightarrow \infty}  \int_{\mathbb{R}} g(nx) \lim_{t \rightarrow \infty}\chi_{K_t} \, dx = \lim_{n \rightarrow \infty} \int g(nx) \chi_{A} \, dx
\end{equation}
However, $\lim_{n \rightarrow \infty} \int_{\mathbb{R}} g(nx) \chi_{K_t} \, dx = 0$ for all $t$, so we have that:

\begin{equation}
	\lim_{n \rightarrow \infty} \int g(nx) \chi_{A} \, dx = 0.
\end{equation}

\subsubsection{Extending to integrable functions}
As we have that $\int_{\mathbb{R}} \tilde{g}(nx) \chi_A \, dx = M \int_{\mathbb{R}} \chi(A) \, dx$, we have that for any $k \in \mathbb{R}$, $\int_{\mathbb{R}}\tilde{g}(nx) k \chi_A \, dx = k M \int_{\mathbb{R}} \chi_A  \, dx $ for any measurable $A$ by the linearity of the integral, and that for any measurable $A, B$, we have that
\begin{align*}
	\lim_{n \rightarrow \infty} \int_{\mathbb{R}} \tilde{g}(nx) (\chi_A + \chi_B) \, dx &= \lim_{n \rightarrow \infty} \int_{\mathbb{R}} \tilde{g}(nx) \chi_A \, dx + \lim_{n \rightarrow \infty} \int_{\mathbb{R}} \tilde{g}(nx) \chi_B \, dx\\
	&= M \int_{\mathbb{R}} \chi_A \, dx + M \int_{\mathbb{R}} \chi_B \, dx\\
	&= M \int_{\mathbb{R}} \chi_A + \chi_B \, dx 
\end{align*}
 by the linearity of the integral as well. So the property holds for any simple integrable function $f$ by linearity.

Let $f$ be a integrable function and let $\left(  f_k \right)_k$ be a sequence of pointwise increasing simple functions converging to $f$. Then we have that
\begin{equation}
	h_k(x) := \lim_{n \rightarrow \infty} \tilde{g}(nx)f_k(x)
\end{equation}
is also a monotonically increasing function which converges to $h(x) := \lim_{n \rightarrow \infty} \tilde{g}(nx) f(x)$. Therefore by monotone convergence theorem, we have that:
\begin{align*}
	M \int_{\mathbb{R}} f(x) \, dx
	&= M \lim_{k \rightarrow \infty} \int_{\mathbb{R}} f_k(x) \, dx\\
	&= \lim_{k \rightarrow \infty} \lim_{n \rightarrow \infty} \int_{\mathbb{R}}  \tilde{g}(nx)f_k(x)\, dx\\
	&= \lim_{k \rightarrow \infty} \int_{\mathbb{R}}  h_k(x) \, dx\\
	&= \int_{\mathbb{R}} h(x) \, dx\\
	&= \int_{\mathbb{R}} \tilde{g}(nx)f_k(x) \, dx
\end{align*}
Thus shown.

\section{2}
We build up from characteristic functions again to show $f(x, y)$ can be approximated with characteristic functions. Let $A \in \mathcal{X}$ and $B \in \mathcal{Y}$ be respectively two $\mu$-integrable and $\nu$-integrable sets. Then we have that integrating $f(x, y) = \chi_A(x) \times \chi_B(y)$ yields:
\begin{equation}
	\int_{X \times Y} f(x, y) \, d(\mu \times \nu) = \mu(A) \times \nu(B) = \int_X \mu(A) \, d\mu \int_Y \nu(B) \, d\nu
\end{equation}
from the definition of the product measure on measurable sets. Then we extend this identity linearly to simple integrable functions.

Now let $g$ be a $\mu$-integrable function and $h$ be a $\nu$-integrable function. Let $(g_n)_n$ be a family of simple pointwise-increasing $\mu$-integrable functions converging to $g$ and let  $(h_n)_n$ be a family of simple pointwise-increasing $\nu$-integrable functions converging to $h$. Then let $f_n(x, y) := g_n(x) h_n(y)$. We have that $f_n(x, y)$ is also simple, monotone increasing and $\mu \times \nu$-integrable. We also have, from the MCT:

\begin{align*}
	\int_{X \times Y} f \, d(\mu \times \nu) &= \lim_{n \rightarrow \infty} \int_{X \times Y} f_n \, d(\mu \times \nu)\\
	&= \lim_{n \rightarrow \infty} \int_{X \times Y} g_n(x) h_n(y) \, d(\mu \times \nu)\\
	&= \lim_{n \rightarrow \infty} \int_X g_n \, d\mu \int_Y h_n \, d\nu\\
	&= \left(\lim_{n \rightarrow \infty} \int_X g_n \, d\mu\right) \left(\lim_{n \rightarrow \infty}\int_Y h_n \, d\nu\right)\\
	&= \int_X g \, d\mu \int_Y h \, d\nu
\end{align*}

To show $f$ is integrable, we have that $f_+ = g_+ h_+ + g_- h_-$. Then we have that 
\begin{equation}
	\int_{X \times Y} f_+ \, d(\mu \times d\nu) = \int_X g_+ \, d\mu \int_Y h_+ \, d\nu +  \int_X g_- \, d\mu \int_Y h_- \, d\nu.
\end{equation}
As $g$ and $h$ are resp. $\mu$ and $\nu$-interable, it holds that all of these integrals are finite and thus their product is also finite. Thus $\int_{X \times Y} f_+ \, d(\mu \times d\nu) < \infty$. Then we have that $f_- = g_+ h_- + g_- h_+$. Then we have that
\begin{equation}
	\int_{X \times Y} f_- \, d(\mu \times d\nu) = \int_X g_+ \, d\mu \int_Y h_- \, d\nu +  \int_X g_- \, d\mu \int_Y h_+ \, d\nu < \infty
\end{equation}

Thus $f$ is $(\mu + \nu)$-integrable. 

\section{3}
Let $\omega = \mu + \nu$. We have that $\mu \leq \omega$ for all $\omega$. Then by Radon-Nicodym, there exists a $(\mu + \nu)$-measurable function $f$ with $0 \leq f \leq 1$ such that
\begin{equation}
	\mu(E) = \int_E f \, d\omega \qquad \forall E \in \mathcal{S}.
\end{equation} 
We will now use a fact from the proof of Radon-Nicodym that $\int_E f \, d\omega = \int_E f \, d\mu +  \int_E f \, d\nu$. Then we use the fact that $\mu(E) = \int_E 1 d\mu$ to get:
\begin{equation}
	\int_E 1 d\mu = \int_E f \, d\mu +  \int_E f \, d\nu \qquad \forall E \in \mathcal{S}.
\end{equation}
Then we move the $\int_E f \, d\mu$ term to the left hand side to have that:
\begin{equation}
	\int_E \left(1 - f\right) d\mu = \int_E f \, d\nu \qquad \forall E \in \mathcal{S}.
\end{equation}
where $f$ is $(\mu + \nu)$-measurable. 
\section{4}

\subsection{a}
We have that:
\begin{equation}
	\mu([0, \infty)) = \sum_{n = 1}^{\infty} \frac{1}{n^3} \int_{[n, n + 1)} x \, d\lambda \leq \sum_{n = 1}^{\infty} \frac{1}{n^3} (n + 1) = \sum_{n = 1}^{\infty} \frac{1}{n^2} + \sum_{n = 1}^{\infty} \frac{1}{n^3}
\end{equation}
However, we have that $\sum_{n = 1}^{\infty} \frac{1}{n^2}$ and $\sum_{n = 1}^{\infty} \frac{1}{n^3}$ are both finite, therefore $\mu([0, \infty)) < \infty$. 
We additionally have that
\begin{equation}
	\nu([1, \infty)) = \int_{[1, \infty)} x^{-2} \, d\lambda
\end{equation}
and using Riemannian integration and the fact that Riemannian and Lebesgue integration agree on continuous functions, we have that $\int_{[1, \infty)} x^{-2} \, d\lambda = \left[ -\frac{1}{x} \right]^\infty_{1} = 1$. Thus $\nu([1, \infty)) < \infty$ and $\mu$ and $\nu$ are finite measures. 
\subsection{b}
\paragraph{$\mu <<\lambda$:}

We have that $\mu <<\lambda$. Let $A$ be a $\lambda$-null set. Then
\begin{equation}
	\mu(A) = \sum_{n = 1}^{\infty} \frac{1}{n^3} \int_{A\cap [n, n + 1)} x \, dx 
\end{equation}
but we have that for all $n$, $A\cap [n, n + 1)$ is also a null-set as it is a subset of $A$. Therefore, we can write
\begin{equation}
	\mu(A) = \sum_{n = 1}^{\infty} \frac{1}{n^3} \int_{\mathbb{R}} x \chi_{A\cap [n, n + 1)} \, dx 
\end{equation}
and we use the fact that $\int_{\mathbb{R}} x \chi_{A\cap [n, n + 1)} \, dx  = 0$ for all $n$ to yield $\mu(A) = 0$. Thus shown.

\paragraph{$\nu <<\lambda$:}

Let $A$ be a $\lambda$-null set. Then:
\begin{equation}
	\nu(A) = \int_{A \cap [1, \infty)} x^{-2} \, dx
\end{equation}
but as $A \cap [1, \infty)$ is also a null-set as it is a subset of $A$, it holds that $\nu(A) = 0$ as well. 

\paragraph{$\lambda << \mu$:}
It does not hold that $\lambda << \mu$. We have that:
\begin{equation}
	\mu([0, 1)) = 
	\sum_{n = 1}^{\infty} \frac{1}{n^3} \int_{[0, 1)\cap [n, n + 1)} x\, dx 
\end{equation}
but $[0, 1)\cap [n, n + 1) = \emptyset$ for all $n \geq 1$. Thus $\mu([0, 1)) = 0$, but $\lambda([0, 1)) = 1$. 

\paragraph{$\lambda << \nu$:}
For a similar reason above, we have that

\begin{equation}
	\nu([0, 1)) = \int_{[0, 1) \cap [1 \infty)} x^{-2} \, dx\int_{\emptyset} x^{-2} \, dx = 0
\end{equation}
Thus shown.

\paragraph{$\mu << \nu$:}
We have that:

\begin{align*}
	\mu(A) &= \sum_{n = 1}^{\infty} \frac{1}{n^3} \int_{A\cap [n, n + 1)} x \, dx \\
	&=\sum_{n = 1}^{\infty} \int_{\mathbb{R}} \frac{x}{n^3} \chi_{A\cap [n, n + 1)} \, dx \\
\end{align*}
We have that $g_n(x) = \frac{x}{n^3} \chi_{A\cap [n, n + 1)}$ is a measurable non-negative function, therefore by 2.22, we can pass through the summation to yield:
\begin{equation}
		\mu(A) = \int_{\mathbb{R}}\sum_{n = 1}^{\infty} \frac{x}{n^3} \chi_{A\cap [n, n + 1)} \, dx \\
\end{equation}

Now we have that the support of $\mu$ is $[1, \infty)$, therefore we can restrict $\mathbb{R}$ to only be $[1, \infty)$ to yield:
\begin{equation}
	\mu(A) = \int_{[1, \infty)}\sum_{n = 1}^{\infty} \frac{x}{n^3} \chi_{A\cap [n, n + 1)} \, dx \\
\end{equation}

To show $\mu << \nu$, let $A$ be a null-set in $\nu$, so
\begin{equation}
	\int_{A \cap [1, \infty)} x^{-2}\, dx = 0
\end{equation}
As $x$ is greater than 0 at every point, this implies that $A$ is a null-set in $\lambda$ restricted to $[1, \infty)$.

Then we have that this implies that $ A \cap [n, n + 1)$ is a $\lambda$-null-set for all $n \in \mathbb{N}$. Therefore, we have that
\begin{align*}
	\mu(A) &= \int_{[1, \infty)}\sum_{n = 1}^{\infty} \frac{x}{n^3} \chi_{A\cap [n, n + 1)} \, dx \\
	&= 0
\end{align*}
as integrating over characteristic functions of null-sets yields 0.


\paragraph{$\nu << \mu$:}

Suppose $A$ is a null-set of $\mu$. Then we have that:

\begin{equation}
	\mu(A) = \int_{[1, \infty)}\sum_{n = 1}^{\infty} \frac{x}{n^3} \chi_{A\cap [n, n + 1)} \, dx  = 0
\end{equation}
Then as $\frac{x}{n^3}$ is positive for all $x \geq 1$, then we have that $A \cap [n, n + 1)$ is also a $\lambda$-null set for all $n \in \mathbb{N}$. But this means that the union of all these null-sets is also a null-set, meaning that $A \cap [1, \infty)$ is also a $\lambda$-null set. Therefore, 

\begin{equation}
	\nu(A) = \int_{A \cap [1, \infty)} x^{-2} \, dx = 0
\end{equation}
\end{document}