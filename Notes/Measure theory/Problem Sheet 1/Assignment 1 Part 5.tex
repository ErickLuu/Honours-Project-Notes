\documentclass{article}
\usepackage[margin=1in]{geometry}
\usepackage{amsmath}
\usepackage{amssymb}
\usepackage{amsthm}
\usepackage{url}

% Environments

\newtheorem{theorem}{Theorem}
\newtheorem{proposition}[theorem]{Proposition}
\newtheorem{corollary}[theorem]{Corollary}
\newtheorem{lemma}[theorem]{Lemma}
\newtheorem{definition}[theorem]{Definition}
\newtheorem{conjecture}[theorem]{Conjecture}

\theoremstyle{definition}
\newtheorem{example}[theorem]{Example}

\numberwithin{theorem}{section}
\numberwithin{equation}{section}

\newcommand{\alg}{\mathcal{A}}
\newcommand{\sig}{\mathcal{S}}
\newcommand{\mono}{\mathcal{M}}
\newcommand{\sigmes}{\Lambda_{\mu^*}}
\newcommand{\salg}{$\sigma$-algebra}
\newcommand{\intd}{\, d}
%opening
\title{Assignment 1}
\author{Eric Luu}

\begin{document}

\maketitle
\section{Part 5}
\subsection{Question 1}


\begin{equation}
	g(x) := \tilde{g}(x) - \frac{1}{T} \int_{0}^{T} \tilde{g}(t) \, dt
\end{equation}

Then we have that for all $x$,
\begin{align*}
	\int_{x}^{x + T} g(t) \, dt &= \int_{x}^{x + T} \left(\tilde{g}(u) - \frac{1}{T} \int_{0}^{T} \tilde{g}(t) \, dt\right) \, du\\
	&= \int_{x}^{x + T} \tilde{g}(u)\, du - \int_{0}^{T} \tilde{g}(t) \, dt
\end{align*}
However, as $\tilde{g}(x)$ is $T$-periodic, it must hold that integration over any interval of length $T$ will yield the same result. Therefore, we have that
\begin{equation}
	\int_{x}^{x + T} \tilde{g}(u)\, du - \int_{0}^{T} \tilde{g}(t) \, dt = 0
\end{equation}

Now for any pair $p \leq q$ for any real number where $q > p + T$, we have that:
\begin{equation}
	\left| \int_{p}^{q} g(x) \, dx \right| = \left| \int_{p}^{q-T} g(x) \, dx +  \int_{q - T}^{q} g(x) \, dx\right| = \left| \int_{p}^{q-T} g(x) \, dx  \right|
\end{equation}
from above.

Therefore we can iterate this operation to find $q'$, where $\left| \int_{p}^{q} g(x) \, dx \right| = \left| \int_{p}^{q'} g(x) \, dx \right|$ and $p \leq q \leq p + T$. Then we have that from $T$-periodicity,
\begin{equation}
	\left| \int_{p}^{q'} g(x) \, dx \right| = \left| \int_{a}^{b} g(x) \, dx + \int_{c}^{d} g(x) \, dx \right|
\end{equation}
for some $ 0 \leq a \leq b \leq c \leq d \leq T$, by shuffling around the domains. We have that:
\begin{align*}
	\left| \int_{a}^{b} g(x) \, dx + \int_{c}^{d} g(x) \, dx \right| &\leq \left| \int_{a}^{b} g(x) \, dx \right| + \left| \int_{c}^{d} g(x) \, dx \right|\\
	&\leq  \int_{a}^{b} |g(x)| \, dx +\int_{c}^{d} |g(x)| \, dx\\
	&\leq \int_{0}^T |g(x)| \, dx
\end{align*}
But we have that $\int_{0}^T |g(x)| \, dx \leq T \max_{x \in [0, T]} |g(x)|$.

\subsubsection{Proving the integral}
Let $A$ be a measurable set where $\lambda(A) < \infty$ and $\chi_A$ be its associated characteristic function. Then we have that

\begin{equation}
	\int_{\mathbb{R}} \tilde{g}(nx) \chi_A \, dx = \int_{\mathbb{R}} \left(g(nx) + M\right) \chi_A \, dx
\end{equation}
We can split this integral into two sides. 

\begin{equation}
	\int_{\mathbb{R}} \left(g(nx) + M\right) \chi_A \, dx = \int_{\mathbb{R}}  g(nx) \chi_A \, dx + \int_{\mathbb{R}} M \chi_A \, dx = \int_{\mathbb{R}}  g(nx) \chi_A \, dx + M \lambda(A)
\end{equation}

\end{document}