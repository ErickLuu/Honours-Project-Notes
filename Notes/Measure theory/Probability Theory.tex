\documentclass{article}
\usepackage[margin=2in]{geometry}
\usepackage{amsmath}
\usepackage{amssymb}
\usepackage{amsthm}
\usepackage{url}

% Environments

\newtheorem{theorem}{Theorem}
\newtheorem{proposition}[theorem]{Proposition}
\newtheorem{corollary}[theorem]{Corollary}
\newtheorem{lemma}[theorem]{Lemma}
\newtheorem{definition}[theorem]{Definition}
\newtheorem{conjecture}[theorem]{Conjecture}

\theoremstyle{definition}
\newtheorem{example}[theorem]{Example}

\numberwithin{theorem}{section}
\numberwithin{equation}{section}

\newcommand{\alg}{\mathcal{A}}
\newcommand{\sig}{\mathcal{S}}
\newcommand{\mono}{\mathcal{M}}
\newcommand{\sigmes}{\Lambda_{\mu^*}}
%opening
\title{Measures}
\author{Eric Luu}

\begin{document}

\maketitle
\section{What is an algebra?}
An algebra is a vector space over a ring with a distributive product. 
Let $\triangle$ be the symmetric difference of two sets and let $\cap$ be the intersection of two sets. 

Now, let $K = \lbrace 0, 1 \rbrace$. Then $(K, +_2, \times)$ is a field. 

For $a$ in $K$ $A \in \mathcal{F}$,\begin{equation}
	 a . A = 
	 \begin{cases}
	 	\emptyset &\text{ if } a = 0\\
	 	A &\text{ if } a = 1
	 \end{cases}
\end{equation} 

We have that $\mathcal{F}$ is a vector space with vector addition $\triangle$ and scalar multiplication $.$ over the set $K$. 

Finally, the distributive product $\cap$ makes $\mathcal{F}$ an algebra. 

This is why algebras are called algebras - $\mathcal{F}$ is an algebra in abstract sense if it is also an algebra in the measure sense.

Recall an algebra contains the empty set $\emptyset$, closed under finite disjoint unions and closed under complementation. These definitions are equivalent to the algebra defined above.

\subsection{Intersections of algebras}
Notation change! Algebras are called $\mathcal{F}$ and the universal set is called $\Omega$. 
It is obvious that intersections of algebras are also algebras. However, if $\left(\mathcal{F}_\lambda \right)_{\lambda \in \Lambda}$ (family of $\sigma$-algebras) are all $\sigma$-algebras, then
\begin{equation}
	\cap_{\lambda \in \Lambda} \mathcal{F}_\lambda
\end{equation} 
is also a $\sigma$-algebra. 

Let $\mathcal{C}$ be family of subsets of $\omega$ and let $\left(\mathcal{F}_\lambda \right)_{\lambda \in \Lambda}$ be the family of $\sigma$-algebras that contain $\mathcal{C}$. Then:
\begin{itemize}
	\item $\left(\mathcal{F}_\lambda \right)_{\lambda \in \Lambda}$ is nonempty (contains discrete algebra).
	\item $\cap_{\lambda \in \Lambda} \mathcal{F}_\lambda$ is the smallest $\sigma$-algebra that contains $\mathcal{C}$ (by definition).
\end{itemize}
$\cap_{\lambda \in \Lambda} \mathcal{F}_\lambda$ is denoted as $\sigma(\mathcal{C})$. 

If $\mathcal{C}$ are the open sets in $\mathbb{R}$, then this is the Borel algebra, denoted $\mathcal{B}(\mathbb{R})$.

\subsection{$\lambda$, $\pi$ and $d$-systems}
\begin{definition}[$\pi$-system]
	A $\pi$-system is a family of sets closed under intersection.
\end{definition}

\begin{definition}[$\lambda$-system]
	A $\lambda$-system is a family of sets $\mathcal{F}$ where:
\begin{itemize}
	\item $\emptyset \in \mathcal{F}$.
	\item $\mathcal{F}$ closed under complement.
	\item $\mathcal{F}$ closed under countable pairwise disjoint union.
\end{itemize}
\end{definition}

Algebras are $\pi$-systems. $\sigma$-algebras are algebras which are also $\lambda$-systems.
\begin{theorem}
	If $\mathcal{C}$ is a $\lambda$ system and a $\pi$-system, then $\mathcal{C}$ is a $\sigma$-algebra.
\end{theorem}
Let $\mathcal{F}$ be an algebra. An event in $\mathcal{F}$ is a measurable set in $\mathcal{F}$. 

\begin{definition}[$d$-system]
	Let $\mathcal{C}$ be a family of subsets in $\omega$. We say $\mathcal{C}$ is a $d$-system if:

\begin{itemize}
	\item $\Omega \in \mathcal{C}$
	\item $A, B \in \mathcal{C}$, $A \subseteq B$, then $B \setminus A \in \mathcal{C}$
	\item If $(A_n)_n \subset \mathcal{C}$, $A_n \nearrow A$, then $A \in \mathcal{C}$ (upwards monotone closed). 
\end{itemize}
\end{definition}

\begin{lemma}
	$\mathcal{C}$ is a $d$-system iff $\mathcal{C}$ is a $\sigma$-algebra.
\end{lemma}

\begin{proof}
	Let $\mathcal{C}$ be a $d$-system. We have that $\Omega \setminus \Omega = \emptyset \in \mathcal{C}$. We have that for all $A \in \mathcal{C}$, $\Omega \setminus A = A^c \in \mathcal{C}$. Finally, suppose we have a set $(A_n)_n \in \mathcal{C}$, $A_n$ p/w disjoint. Then let $B_k = \cup_{i = 1}^{k} A_i$. Then we can show that $B_k \in \mathcal{C}$ for all $k$, so $B_k \nearrow \cup_{n \geq 1} A_n$ so $\mathcal{C}$ is closed under countable unions. 
	
	Easy to show other way. 
\end{proof}
\end{document}
