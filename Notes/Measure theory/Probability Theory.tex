\documentclass{article}
\usepackage[margin=1in]{geometry}
\usepackage{amsmath}
\usepackage{amssymb}
\usepackage{amsthm}
\usepackage{url}

% Environments

\newtheorem{theorem}{Theorem}
\newtheorem{proposition}[theorem]{Proposition}
\newtheorem{corollary}[theorem]{Corollary}
\newtheorem{lemma}[theorem]{Lemma}
\newtheorem{definition}[theorem]{Definition}
\newtheorem{conjecture}[theorem]{Conjecture}

\theoremstyle{definition}
\newtheorem{example}[theorem]{Example}

\numberwithin{theorem}{section}
\numberwithin{equation}{section}

\newcommand{\alg}{\mathcal{A}}
\newcommand{\sig}{\mathcal{S}}
\newcommand{\mono}{\mathcal{M}}
\newcommand{\sigmes}{\Lambda_{\mu^*}}
%opening
\title{Measures}
\author{Eric Luu}

\begin{document}

\maketitle
\section{What is an algebra?}
An algebra is a vector space over a ring with a distributive product. 
Let $\triangle$ be the symmetric difference of two sets and let $\cap$ be the intersection of two sets. 

Now, let $K = \lbrace 0, 1 \rbrace$. Then $(K, +_2, \times)$ is a field. 

For $a$ in $K$ $A \in \mathcal{F}$,\begin{equation}
	 a . A = 
	 \begin{cases}
	 	\emptyset &\text{ if } a = 0\\
	 	A &\text{ if } a = 1
	 \end{cases}
\end{equation} 

We have that $\mathcal{F}$ is a vector space with vector addition $\triangle$ and scalar multiplication $.$ over the set $K$. 

Finally, the distributive product $\cap$ makes $\mathcal{F}$ an algebra. 

This is why algebras are called algebras - $\mathcal{F}$ is an algebra in abstract sense if it is also an algebra in the measure sense.

Recall an algebra contains the empty set $\emptyset$, closed under finite disjoint unions and closed under complementation. These definitions are equivalent to the algebra defined above.

\subsection{Intersections of algebras}
Notation change! Algebras are called $\mathcal{F}$ and the universal set is called $\Omega$. 
It is obvious that intersections of algebras are also algebras. However, if $\left(\mathcal{F}_\lambda \right)_{\lambda \in \Lambda}$ are all $\sigma$-algebras, then
\begin{equation}
	\cap_{\lambda \in \Lambda} \mathcal{F}_\lambda
\end{equation} 
is also a $\sigma$-algebra. 

Let $\mathcal{C}$ be family of subsets of $\omega$ and let $\left(\mathcal{F}_\lambda \right)_{\lambda \in \Lambda}$ be the family of $\sigma$-algebras that contain $\mathcal{C}$. Then:
\begin{itemize}
	\item $\left(\mathcal{F}_\lambda \right)_{\lambda \in \Lambda}$ is nonempty (contains discrete algebra).
	\item $\cap_{\lambda \in \Lambda} \mathcal{F}_\lambda$ is the smallest $\sigma$-algebra that contains $\mathcal{C}$ (by definition).
\end{itemize}
$\cap_{\lambda \in \Lambda} \mathcal{F}_\lambda$ is denoted as $\sigma(\mathcal{C})$. 

If $\mathcal{C}$ are the open sets in $\mathbb{R}$, then this is the Borel algebra, denoted $\mathcal{B}(\mathbb{R})$.

\subsection{$\lambda$, $\pi$ and $d$-systems}
\begin{definition}[$\pi$-system]
	A $\pi$-system is a family of sets closed under intersection.
\end{definition}

\begin{definition}[$\lambda$-system]
	A $\lambda$-system is a family of sets $\mathcal{F}$ where:
\begin{itemize}
	\item $\emptyset \in \mathcal{F}$.
	\item $\mathcal{F}$ closed under complement.
	\item $\mathcal{F}$ closed under countable pairwise disjoint union.
\end{itemize}
\end{definition}

Algebras are $\pi$-systems. $\sigma$-algebras are algebras which are also $\lambda$-systems.
\begin{theorem}
	If $\mathcal{C}$ is a $\lambda$ system and a $\pi$-system, then $\mathcal{C}$ is a $\sigma$-algebra.
\end{theorem}
Let $\mathcal{F}$ be an algebra. An event in $\mathcal{F}$ is a measurable set in $\mathcal{F}$. 

\begin{definition}[$d$-system]
	Let $\mathcal{C}$ be a family of subsets in $\omega$. We say $\mathcal{C}$ is a $d$-system if:

\begin{itemize}
	\item $\Omega \in \mathcal{C}$
	\item $A, B \in \mathcal{C}$, $A \subseteq B$, then $B \setminus A \in \mathcal{C}$
	\item If $(A_n)_n \subset \mathcal{C}$, $A_n \nearrow A$, then $A \in \mathcal{C}$ (upwards monotone closed). 
\end{itemize}
\end{definition}

\begin{lemma}
	$\mathcal{C}$ is a $d$-system iff $\mathcal{C}$ is a $\lambda$-system
\end{lemma}

\begin{proof}
	Let $\mathcal{C}$ be a $d$-system. We have that $\Omega \setminus \Omega = \emptyset \in \mathcal{C}$. We have that for all $A \in \mathcal{C}$, $\Omega \setminus A = A^c \in \mathcal{C}$. Finally, suppose we have a set $(A_n)_n \in \mathcal{C}$, $A_n$ p/w disjoint. Then let $B_k = \cup_{i = 1}^{k} A_i$. Then we can show that $B_k \in \mathcal{C}$ for all $k$, so $B_k \nearrow \cup_{n \geq 1} A_n$ so $\mathcal{C}$ is closed under countable unions. 
	
	If $\mathcal{C}$ is a $\lambda$-system then $\Omega \in \mathcal{C}$, $(A \cup B^c)^c = B \setminus A \in \mathcal{C}$, upwards closed comes from splitting $A_n$ into parts. 
\end{proof}

\begin{corollary}
	$\mathcal{G}$ is a $\sigma$-algebra iff $\mathcal{G}$ is a $\pi$-system and a $d$-system.  
\end{corollary}

We denote $d(\mathcal{C})$ as the smallest $d$-system that contains $\mathcal{C}$. If $\mathcal{C}$ is a $\pi$-system, then $d(\mathcal{C}) = \sigma(\mathcal{C})$.

\section{Probability Measures}
We say $\mathbb{P}: \mathcal{F} \rightarrow [0, 1]$ is a probability measure if $\mathbb{P}$ is a finite measure with $\mathbb{P}(\Omega) = 1$. 

An event is $\mathbb{P}$-null if $\mathbb{P}(A) = 0$. $A$ is a null-set. 

If $\mathbb{P}(A \triangle B) = 0$, then $A = B$ a.e. or almost surely.

\subsection{Basic properties}
We take $(\Omega, \mathcal{F}, \mathbb{P})$ to be a probability space.

\begin{itemize}
	\item If $A \subseteq B$ and $A, B \in \mathcal{F}$, then $\mathbb{P}(B \setminus A) = \mathbb{P}(B) - \mathbb{P}(A)$.
	\item $A_n \nearrow A$ implies $\mathbb{P}(A_n) \nearrow \mathbb{P}(A)$.
	\item $\mathbb{P}\left(\cup_n A_n\right) \leq \sum_n \mathbb{P}(A_n)$. 
	\item $\mathbb{P}\left(\cup_n A_n\right) \geq \sup_n \mathbb{P}(A_n)$. 
	\item For all $n$, $\mathbb{P}(A_n) = 0$ implies $\mathbb{P}(\cup_n A_n) = 0$
	\item For all $n$, $\mathbb{P}(A_n) = 1$ implies $\mathbb{P}(\cap_n A_n) = 1$
	\item Inclusion-Exclusion formula:
	
	\begin{itemize}
		\item 	$\mathbb{P}(B \cup A) = \mathbb{P}(B) + \mathbb{P}(A) - \mathbb{P}(B \cap A)$
		\item\begin{equation}
			 \mathbb{P}\left(\cup_{i = 1}^k A_i\right) = \sum_{\substack{K \subset \lbrace 1 .. , \rbrace \\ K \neq \emptyset } } (-1)^{|K| + 1} \mathbb{P}\left(\cap_{\ell \in K} A_\ell\right)
		\end{equation}
		
	\end{itemize}
\end{itemize}

Important theorem!
\begin{theorem}
	Two probability measures $\mathbb{P}$ and $\mathbb{Q}$ that coincide on a $\pi$-system are equal on the $\sigma$-algebra it generates.
\end{theorem}

\begin{corollary}
	Extending $\mathbb{P}$ from a $\pi$ system $\mathcal{C}$ to the $\sigma$-algebra $\sigma(\mathcal{C})$ it generates is unique.
\end{corollary}

We can do another extension using Caratheodory to extend $\mathbb{P}$.

\begin{definition}
	We say that for a family $(A_n)_n$, 
	\begin{equation}
		\limsup_n A_n = \cap_n \cup_{m \geq n} A_m
	\end{equation}
	is $A_n$ infinitely often (i.o.)
	and 
	\begin{equation}
		\liminf_n A_n = \cup_n \cap_{m \geq n} A_m 
	\end{equation}
	is $A_n$ eventually. 
\end{definition}
We have that $\liminf_n A_n \subset \limsup_n A_n$. 

\begin{lemma}[Fatou's lemma]
	\begin{equation}
		\mathbb{P}(\limsup_n A_n) \geq \limsup_n \mathbb{P}(A_n)
	\end{equation} and \begin{equation}
	\mathbb{P}(\liminf_n A_n) \leq \liminf_n \mathbb{P}(A_n).
	\end{equation}
\end{lemma}

\begin{theorem}[Borel-Cantelli]
	Let $(A_n)_n$ be such that $\sum_n \mathbb{P}(A_n) \leq +\infty$. Then $\mathbb{P}(A_n i.o.) = 0$. 
\end{theorem}


\section{Random variables}
A random variable $X$ is a measurable function on a probability space  $(\Omega, \mathcal{F}, \mathbb{P})$. 

For any given variable $X$, we say $\sigma(X)$ is the intersection of all spaces such that $X$ is measurable on this space. 

\subsection{Distribution of random variable}

The function $\mathbb{P}_X: \mathcal{B}(\mathbb{R}) \rightarrow [0,1]$ which takes sets $\mathbb{P}_X(A) = \mathbb{P}(X \in A)$ is a probability measure on $(\mathbb{R}, \mathcal{B}(\mathbb{R}))$. This is the law/distribution on $X$. 

\subsection{Distribution function}
We say $F_X : \mathbb{R} \rightarrow [0, 1]$ is the distribution function, so
\begin{equation}
	F_X(x) = \mathbb{P}_X((-\infty, x]).
\end{equation}

Properties of distribution function:
\begin{itemize}
	\item $F_X$ is an increasing function
	\item $\lim_{n \rightarrow -\infty} F_x(x) = 0$,  $\lim_{n \rightarrow \infty} F_x(x) = 1$
	\item $F_x$ is right-continuous, meaning that right limits converge. 
\end{itemize}

\begin{definition}
	Skorokhod representation: Any function which satisfies the 3 properties above is the distribution of a random variable on the Lebesgue measure on $[0, 1]$.
\end{definition}

\begin{theorem}[Monotone Class Theorem]
	Let $\mathcal{C}$ be a $\pi$-system and $\mathcal{H}$ be a class of bounded real valued functions on $\Omega$ satisfying:
	\begin{itemize}
		\item $\mathcal{H}$ is a vector space over $\mathbb{R}$
		\item $1 \in \mathcal{H}$
		\item If $(f_n)_n$ is a series of non-negative increasing functions in $\mathcal{H}$ such that $f_n$ converges to $f$ where $f$ is bounded, then $f \in \mathcal{H}$
		\item For all $A \in \mathcal{C}$, $1_A$ is in $\mathcal{H}$
	\end{itemize}
	Then $\mathcal{H}$ contains every bounded $\sigma(\mathcal{C})$-measurable function. 
\end{theorem}
Proof from Monotone class theorem (of measure theory).

\subsection{Independence}
\begin{definition}
	Two events are independent if $\mathbb{P}(A \cap B) = \mathbb{P}(A) \mathbb{P}(B)$.
	
	Events $A_1, ..., A_n$ are independent if for all $A \subset [n]$, we have that:
	\begin{equation}
		\mathbb{P}(\cap_{i \in A} A_i) = \cap_{i \in A} \mathbb{P}(A_i).
	\end{equation}
	Finally, we have that $A_\lambda$ is independent if for all $K \subset \lambda$, $2 \leq |K| \leq \infty$, we have that
	\begin{equation}
		\mathbb{P}(\cap_{k \in K} A_k) = \cap_{k \in K} \mathbb{P}(A_k).
	\end{equation}
\end{definition}
Sub $\sigma$-algebras $\mathcal{G}_\lambda$ are independent if for all $\{\lambda_n\}_n \subset \Gamma$, for all $A_1 \in \mathcal{G}_{\lambda_1}, ..., A_n \in \mathcal{G}_{\lambda_n}$, $A_1, ..., A_n$ are independent.

\begin{example}
	Let $ (\xi_n)_n $ be i.i.d with law $\mathbb{P}(\xi_n = 1) = \mathbb{P}(\xi_n = -1) = -1/2$. Let $\eta_n = \prod_{k = 1}^{n} \xi_k$. We have that the $\eta_n$'s are independent with the same laws. 
\end{example} 

\begin{lemma}
	Second Borel-Cantelli Lemma:
	Let $(A_n)_n$ be a sequence of independent events. If $\sum_n \mathbb{P}(A_n) = \infty$, then $\mathbb{P}(A_n i.o.) = 1$. 
\end{lemma}

 
\subsection{Kolmogorov 0-1 Law}
\end{document}
