\documentclass{article}
\usepackage[margin=2in]{geometry}
\usepackage{amsmath}
\usepackage{amssymb}
\usepackage{amsthm}
\usepackage{url}

% Environments

\newtheorem{theorem}{Theorem}
\newtheorem{proposition}[theorem]{Proposition}
\newtheorem{corollary}[theorem]{Corollary}
\newtheorem{lemma}[theorem]{Lemma}
\newtheorem{definition}[theorem]{Definition}
\newtheorem{conjecture}[theorem]{Conjecture}

\theoremstyle{definition}
\newtheorem{example}[theorem]{Example}

\numberwithin{theorem}{section}
\numberwithin{equation}{section}

\newcommand{\alg}{\mathcal{A}}
\newcommand{\sig}{\mathcal{S}}
\newcommand{\mono}{\mathcal{M}}
\newcommand{\sigmes}{\Lambda_{\mu^*}}
%opening
\title{Measures}
\author{Eric Luu}

\begin{document}

\maketitle
\section{What is an algebra?}
An algebra is a vector space over a ring with a distributive product. 
Let $\triangle$ be the symmetric difference of two sets and let $\cap$ be the intersection of two sets. 

Now, let $K = \lbrace 0, 1 \rbrace$. Then $(K, +_2, \times)$ is a field. 

For $a$ in $K$ $A \in \mathcal{F}$,\begin{equation}
	 a . A = 
	 \begin{cases}
	 	\emptyset &\text{ if } a = 0\\
	 	A &\text{ if } a = 1
	 \end{cases}
\end{equation} 

We have that $\mathcal{F}$ is a vector space with vector addition $\triangle$ and scalar multiplication $.$ over the set $K$. 

Finally, the distributive product $\cap$ makes $\mathcal{F}$ an algebra. 
\end{document}
