\documentclass{article}
\usepackage[margin=1in]{geometry}
\usepackage{amsmath}
\usepackage{amssymb}
\usepackage{amsthm}
\usepackage{url}

% Environments

\newtheorem{theorem}{Theorem}
\newtheorem{proposition}[theorem]{Proposition}
\newtheorem{corollary}[theorem]{Corollary}
\newtheorem{lemma}[theorem]{Lemma}
\newtheorem{definition}[theorem]{Definition}
\newtheorem{conjecture}[theorem]{Conjecture}

\theoremstyle{definition}
\newtheorem{example}[theorem]{Example}

\numberwithin{theorem}{section}
\numberwithin{equation}{section}

\newcommand{\alg}{\mathcal{A}}
\newcommand{\sig}{\mathcal{S}}
\newcommand{\mono}{\mathcal{M}}
\newcommand{\sigmes}{\Lambda_{\mu^*}}
\newcommand{\salg}{$\sigma$-algebra}
\newcommand{\intd}{\, d}
%opening
\title{Assignment 1}
\author{Eric Luu}

\begin{document}

\maketitle
\section{Question 1}
\subsection{a}
Let $A \in \sigma(\mathcal{C}_1 \cup \mathcal{C}_2)$. Then $A$ is in the intersection of all $\sigma$-algebras that contain $C_1$ and $C_2$. However, we have that $\sigma(\mathcal{C}_1) \vee \sigma(\mathcal{C}_2) = \sigma(\sigma(\mathcal{C}_1) \cup \sigma(\mathcal{C}_2))$. But this set contains both $\mathcal{C}_1$ and $\mathcal{C}_2$ and is a $\sigma$-algebra, therefore $A \in \sigma(\mathcal{C}_1) \vee \sigma(\mathcal{C}_2)$. So $ \sigma(\mathcal{C}_1 \cup \mathcal{C}_2) \subseteq \sigma(\mathcal{C}_1) \vee \sigma(\mathcal{C}_2)$.

Now suppose $A \in \sigma(\mathcal{C}_1) \vee \sigma(\mathcal{C}_2)$. Then $A$ is in the intersection of all $\sigma$-algebras that contain $\sigma(\mathcal{C}_1)$ and $\sigma(\mathcal{C}_2)$. But we have that $\sigma(\mathcal{C}_1 \cup \mathcal{C}_2)$ is a $\sigma$-algebra that contains $\mathcal{C}_1$, so it necessarily also contains $\sigma(\mathcal{C}_1)$. By symmetry, $\sigma(\mathcal{C}_1 \cup \mathcal{C}_2)$ contains $\sigma(\mathcal{C}_2)$ as well. Therefore, $A \in \sigma(\mathcal{C}_1 \cup \mathcal{C}_2)$. Thus $\sigma(\mathcal{C}_1) \vee \sigma(\mathcal{C}_2) \subseteq \sigma(\mathcal{C}_1 \cup \mathcal{C}_2)$. 

\subsection{b}
We shall show $\mathcal{H}$ is both a $\pi$-system and a $\lambda$-system, and thus a $\sigma$-algebra. Suppose $A_1, A_2 \in \mathcal{H}$, with associated $B_1, B_2 \in \mathcal{G}$ such that $\mathbb{P}(A_i \triangle B_i) = 0$ for all $i$. 
Then let $A = A_1 \cap A_2$. We claim that $A \in \mathcal{H}$ as $P(A \triangle (B_1 \cap B_2)) = 0$, where $B_1 \cap B_2 \in \mathcal{G}$ as $\mathcal{G}$ is a $\sigma$-algebra.
\paragraph{$\mathcal{H}$ is a $\pi$-system:}
We have that $A_i \triangle B_i$ is a null-set. Write $A \triangle (B_1 \cap B_2)$ as $\left(A \setminus (B_1 \cap B_2)\right) \cup \left((B_1 \cap B_2) \triangle A\right)$. Then we have that $A \setminus B_1 \cap B_2$ is a subset of $(A_1 \setminus B_1) \cup (A_2 \setminus B_2)$. Let $x \in A \setminus B_1 \cap B_2$. Then $x \in A_1 \cap A_2$. Suppose $x$ is in neither $B_1$ nor $B_2$. Then $x \in A_1 \setminus B_1$. If $x$ is in $B_1$ but not in $B_2$, then $x$ is in $A_2 \setminus B_2$. By symmetry, this means that $\left((B_1 \cap B_2) \triangle A\right)$ is a subset of $(B_1 \setminus A_1) \cup (B_2 \setminus A_2)$. But we have that $(B_1 \setminus A_1) \cup (B_2 \setminus A_2) \cup (A_1 \setminus B_1) \cup (A_2 \setminus B_2)$ is the union of $A_1 \triangle B_1$ and $A_2 \triangle B_2$. But these are both $\mathbb{P}$-null sets, meaning that the union is a $\mathbb{P}$-null set. As $\mathbb{F}$ is a $\sigma$-algebra, then this means that all subsets of $\mathbb{P}$-null sets are null sets, meaning that $A \triangle (B_1 \cap B_2)$ is also a $\mathbb{P}$-null set. Thus $\mathcal{H}$ is a $\pi$-system.

\paragraph{$\mathcal{H}$ is a $\lambda$-system}
We have that $\emptyset$ is in $\mathcal{H}$. As we have that $\emptyset \in \mathcal{G}$ as $\mathcal{G}$ is a $\sigma$-algebra, then $\mathbb{P}(\emptyset \triangle \emptyset) = \mathbb{P}(\emptyset) = 0$. If $A \in \mathcal{H}$, then $\Omega \setminus A$ is also in $\mathcal{H}$. Suppose $B \in \mathcal{G}$ such that $\mathbb{P}(A \triangle B) = 0$. Then $B^c \in \mathcal{G}$ as $\mathcal{G}$ is a $\sigma$-algebra. But we have that $A \triangle B = A^c \triangle B^c$. Therefore, $\mathbb{P}(A^c \triangle B^c) = \mathbb{P}(A \triangle B) = 0$. Thus $A^c \in \mathcal{H}$. Finally, let $(A_n)_n$ be a sequence of pairwise disjoint sets in $\mathcal{H}$. Suppose further there is a sequence of sets $(B_n)_n$ in $\mathcal{G}$ such that $\mathbb{P}(A_n \cap B_n) = 0$. Then let $A = \bigcup_{n = 1}^\infty A_n$, and let $B = \bigcup_{n = 1}^\infty B_n$. We claim that $\mathbb{P}(A \triangle B) = 0$. We have that $(A \triangle B)$ is contained in $\bigcup_{n = 1}^\infty A_i \triangle B_i$ as all of the $A_i$ are pairwise disjoint. Now we have that $\mathbb{P}(A \triangle B) \leq \mathbb{P}(\bigcup_{n = 1}^\infty A_i \triangle B_i) \leq \sum_{n = 1}^{\infty}\mathbb{P}(A_i \triangle B_i) = 0$. As $B \in \mathcal{G}$, then $A \in \mathcal{H}$. Thus $\mathcal{H}$ is a $\lambda$-system.
\paragraph{Conclusion}
As $\mathcal{H}$ is a $\pi$ and $\lambda$-system, $\mathcal{H}$ is also a $\sigma$-algebra. 

\subsection{c}
We have that $\sigma(\mathcal{G}) = \mathcal{G}$. 
Let $X \in \sigma(\mathcal{G} \cup \mathcal{N})$. Then this means that $X$ is in all $\sigma$-algebras containing $\mathcal{G}$ and $\sigma(\mathcal{N})$. Then this implies that $X \in \mathcal{F}$ as this is a subset of two sigma-algebras of $\mathcal{F}$. ...

Suppose $X$ is in $\mathcal{F}$ such that there exists a $B \in \mathcal{G}$ such that $\mathbb{P}(X \triangle B) = 0$. Then this implies that 


\section{Question 2}
Consider $\mathcal{H} = \left\{ B \in \mathcal{F}: \forall \varepsilon > 0, \text{ there exists a } A \in \mathcal{A} \text{ such that } \mathbb{P}(A \triangle B) < \varepsilon \right\}$. We wish to show $\mathcal{H}$ is a $\sigma$-algebra, and that $A$ is in $\mathcal{H}$. If these two conditions hold, then $\sigma(A) = \mathcal{F} \subseteq \mathcal{H}$, so the above conditions hold. 

Let $A \in \mathcal{H}$. Then $\mathbb{P}(A \triangle A) = \mathbb{P}(\emptyset) = 0$ so $\mathcal{A} \subseteq \mathcal{H}$.
We have that $\Omega$ is in $\mathcal{H}$ as we have that as $\mathcal{A}$ is an algebra, then $\Omega$ is in $\mathcal{A}$. Now suppose $B \in \mathcal{H}$. Then we have that there is a sequence $(A_n)_n \subset \mathcal{A}$ such that for all $n \rightarrow \infty$, $\mathbb{P}(B \triangle A_n) < \frac{1}{n}$. 
Now consider $B^c$ and the sequence $(A_n^c)_n$. We have that $P(B^c \triangle A^c) = P(B \triangle A^c) < \frac{1}{n}$. Then $B^c$ is in $\mathcal{H}$. 
Finally consider a sequence of $(B_n)_n \subset \in \mathcal{H}$ which is pairwise disjoint, such that $P(B_i) \geq P(B_{i + 1})$ for all $i$. Fix $\varepsilon > 0$. Then since $B_i$ is pairwise disjoint then $P(\cup_n B_n) = \sum_n P(B_n) \leq 1$. Then we have that $P(B_n)$ is increasing and the partial sums $\sum_n^N B_n$ converge to a constant value $C \leq 1$. Therefore, we have that for all $\varepsilon$ there exists an $N$ such that $P(\cup_{n \geq N}B_n) < \varepsilon/N$. Then let $(A_i)_{1 \geq i \geq N - 1}$ be a subset of $\mathcal{A}$ such that for all $B_i$ where $1 \geq i \geq N$, we have that $P(A_i \triangle B_i) \leq \varepsilon/N$. Then we have that $P((\cup_n B_n) \triangle \cup_i A_i ) \leq \sum_{1 \leq i < N} P(B_i \triangle A_i) + \sum_{n \geq N} P(B_n) = (N-1)\frac{\varepsilon}{N} + \varepsilon/N = \varepsilon$. 

Therefore, 
\section{Question 3}
Let $A_i$ be the event that $\xi_i= 1$. From the question, we have that $\sum_{i = 1}^{\infty}\mathbb{P}(A_i) < \infty$. But from Borel-Cantellini, this implies that $\mathbb{P}(A_i i.o) = 0$. Therefore, we have that $\xi_i = 1$ i.o. almost never, so $\xi_i = 1$ finitely often. Thus, $\mathbb{P}\left(\sum_{i = 1}^{\infty} \xi_i = - \infty \right) = 1$, so for all $a > 0$, $\mathbb{P}\left(\sum_{i = 1}^{\infty} \xi_i = - a \right) = 1$. 

\end{document}