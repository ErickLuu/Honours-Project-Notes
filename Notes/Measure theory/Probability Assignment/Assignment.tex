\documentclass{article}
\usepackage[margin=1in]{geometry}
\usepackage{amsmath}
\usepackage{amssymb}
\usepackage{amsthm}
\usepackage{url}

% Environments

\newtheorem{theorem}{Theorem}
\newtheorem{proposition}[theorem]{Proposition}
\newtheorem{corollary}[theorem]{Corollary}
\newtheorem{lemma}[theorem]{Lemma}
\newtheorem{definition}[theorem]{Definition}
\newtheorem{conjecture}[theorem]{Conjecture}

\theoremstyle{definition}
\newtheorem{example}[theorem]{Example}

\numberwithin{theorem}{section}
\numberwithin{equation}{section}

\newcommand{\alg}{\mathcal{A}}
\newcommand{\sig}{\mathcal{S}}
\newcommand{\mono}{\mathcal{M}}
\newcommand{\sigmes}{\Lambda_{\mu^*}}
\newcommand{\salg}{$\sigma$-algebra}
\newcommand{\intd}{\, d}
%opening
\title{Assignment 1}
\author{Eric Luu}

\begin{document}

\maketitle
\section{Question 1}
Let $A \in \sigma(\mathcal{C}_1 \cup \mathcal{C}_2)$. Then $A$ is in the intersection of all $\sigma$-algebras that contain $C_1$ and $C_2$. However, we have that $\sigma(\mathcal{C}_1) \vee \sigma(\mathcal{C}_2) = \sigma(\sigma(\mathcal{C}_1) \cup \sigma(\mathcal{C}_2))$. But this set contains both $\mathcal{C}_1$ and $\mathcal{C}_2$ and is a $\sigma$-algebra, therefore $A \in \sigma(\mathcal{C}_1) \vee \sigma(\mathcal{C}_2)$. 
\end{document}